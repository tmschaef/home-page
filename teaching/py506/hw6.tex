\documentstyle[12pt]{article}
\begin{document}   
\pagestyle{plain}

\centerline{\bf\Large Homework 6, due 10-13}
\vspace*{0.5cm}

In class we introduced product wave functions
\[
 \uparrow\uparrow\; = \chi_\uparrow(1)
\chi_\uparrow(2)=
\left(\begin{array}{c}1 \\ 0\end{array}\right)
\left(\begin{array}{c}1 \\ 0\end{array}\right).
\]
Spin operators act on product spin wave 
functions as follows
\[
\sigma_{x,1}\sigma_{y,2} 
 \chi_\uparrow(1)\chi_\uparrow(2)=
\left[ \sigma_{x}\chi_\uparrow (1)\right]
\left[ \sigma_{y}\chi_\uparrow (2)\right].
\]
Expectation values are defined as
\[
( \chi_\uparrow(1)\chi_\uparrow(2) )^\dagger
\sigma_{x,1}\sigma_{y,2} 
( \chi_\uparrow(1)\chi_\uparrow(2) ) =
\left[ \chi_\uparrow^\dagger (1)\sigma_{x}\chi_\uparrow (1)\right]
\left[ \chi_\uparrow^\dagger (2)\sigma_{y}\chi_\uparrow (2)\right].
\]
In class we argued that $\chi_{A,S}$
\[
\chi_{A,S}=\frac{1}{\sqrt{2}}
\left( \uparrow\downarrow \mp \downarrow\uparrow
 \right)
\]
have spin zero and one, respectively. Check this 
statement explicitely by computing 
\[
 \vec{S}^2\chi_{A,S}
\]
where 
\[
\vec{S} = \vec{S}_1+\vec{S}_2 = \frac{\hbar}{2}\left(\vec{\sigma}_1
 +\vec{\sigma}_2 \right).
\]
Use you result to compute the expectation value
of $\vec{S}_1\cdot\vec{S}_2$ in the spin zero and
one states,
\[
 \chi_A^\dagger (\vec{S}_1\cdot\vec{S}_2) \chi_A = ?
\]
\[
 \chi_S^\dagger (\vec{S}_1\cdot\vec{S}_2) \chi_S = ?
\]



\end{document}   
 

 


  

