\documentstyle[12pt]{article}

\textwidth=17cm
\textheight=24cm
\begin{document}
\pagestyle{empty}

The Yang Mills instantons is the solution with minimal
euclidean action connecting vacua with different Chern-Simons number.
To find these solutions, it is convenient to exploit the following identity
\begin{equation} 
S = \frac{1}{4} \int d^4x G^a_{\mu\nu} G^a_{\mu\nu} \;=\;
 \int d^4x\, \left[\pm\frac{1}{4} G^a_{\mu\nu} \tilde G^a_{\mu\nu}
 + \frac{1}{8} \left( G^a_{\mu\nu}\mp \tilde G^a_{\mu\nu}\right)^2
  \right],
\end{equation}
where $\tilde G_{\mu\nu}=1/2\epsilon_{\mu\nu\rho\sigma}G_{\rho\sigma}$
is the dual field strength tensor (the field tensor in which the roles
of electric and magnetic fields are interchanged). Since the last term 
is always positive, it is clear that the action is minimal if the field 
is (anti) self-dual
\begin{equation} 
\label{self_dual}
G^a_{\mu \nu}=\pm\tilde G^a_{\mu \nu},
\end{equation}
One can also show directly that the self-duality condition implies the
equations of motion, $D_\mu G_{\mu\nu}=0$. We also note that the 
action of a solution with topological charge $Q$ is $S=8\pi^2|Q|/g^2$.
The simplest winding number configuration is 
\begin{equation}
A_\mu=iU\partial_\mu U^\dagger\hspace{1cm}
U=i\hat x_\mu\tau_\mu^+, 
\end{equation}
where $\tau_\mu^\pm=(\vec\tau,\mp i)$. This is a simple generaliztion
of the standard hedgehog configuration. Then $A_\mu^a=(2/g)\eta_{a\mu\nu}
x_\nu/x^2$, where we have introduced the 't Hooft symbol $\eta_{a\mu\nu}$. 
It is given by 
\begin{equation} 
\label{eta_def} 
\eta_{a\mu\nu}=\left\{ \begin{array}{rcl}
 \epsilon_{a\mu\nu} &\hspace{0.5cm}& \mu,\,\nu=1,\,2,\,3, \\
 \delta_{a\mu}      &              & \nu=4,  \\
-\delta_{a\nu}      &              & \mu=4.
\end{array}\right.
\end{equation}
We also define $\overline\eta_{a\mu\nu}$ by changing the sign of
the last two equations. We can now look for a solution of 
the self-duality equation using the ansatz $A_\mu^a=(2/g)
\eta_{a\mu\nu} x_\nu f(x^2)/x^2$. This gives 
\begin{equation} 
 f(1-f) -x^2f'= 0
\end{equation}
from which we conclude $f=x^2/(x^2+\rho^2)$ and
\begin{equation} 
A^a_\mu(x)= \left(\frac{2}{g}\right)\frac{\eta_{a\mu\nu}x_\nu}
  {x^2+\rho^2},
\end{equation}
where $\rho$ is an arbitrary parameter characterizing the size of
the instanton. A solution with topological charge $Q=-1$ can be
obtained by replacing $\eta_{a\mu\nu}\to\overline\eta_{a\mu\nu}$.
The corresponding field strength is
\begin{equation} 
(G^a_{\mu\nu})^2 = \frac{192\rho^4}{(x^2+\rho^2)^4} .
\end{equation}
\end{document}
