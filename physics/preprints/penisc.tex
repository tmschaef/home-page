% paper as revised on Feb. 7 1991
% proceedings draft from Aug. 8 1991
\documentstyle[12pt,thesis]{article}
%
% **** to be used with thesis.sty  ******
% **** modifies \cite command      ******
% 
% **** use \citenum to obtain number     ****
% **** of reference without any ornament ****
%       
\newcommand{\pr}[3]{Phys.~Rev.~{\bf #1} (19#2) #3}
\newcommand{\prd}[3]{Phys.~Rev.~{\bf D#1} (19#2) #3}
\newcommand{\prp}[3]{Phys.~Rep.~{\bf #1} (19#2) #3}
\newcommand{\ij}[3]{Int.~J. Mod.~Phys.~{\bf A #1}(19#2)#3}
\newcommand{\aj}[3]{Aust.~J. Phys.~{\bf #1} (19#2) #3}
\newcommand{\plb}[3]{Phys.~Lett.~{\bf{B#1}} (19#2) #3}
\newcommand{\prl}[3]{Phys.~Rev.~Lett.~{\bf #1} (19#2) #3}
\newcommand{\jg}[3]{J. Phys.~{\bf G #1} (19#2) #3}
\newcommand{\npb}[3]{Nucl.~Phys.~{\bf B#1} (19#2) #3}
\newcommand{\npa}[3]{Nucl.~Phys.~{\bf A#1} (19#2) #3}
\newcommand{\an}[3]{Ann.~Phys.~{\bf #1} (19#2) #3}
\newcommand{\ptp}[3]{Prog.~Theor.~Phys.~{\bf #1} (19#2) #3}
\newcommand{\zpc}[3]{Z. Phys.~{\bf C#1} (19#2) #3}
\newcommand{\zpa}[3]{Z. Phys.~{\bf A#1} (19#2) #3}
\newcommand{\fp}[3]{Fortschr.~Phys.~{\bf #1} (19#2) #3}
\newcommand{\nc}[3]{Nuovo Cimento{\bf #1} (19#2) #3}  
%
\newcommand{\pbar}{\bar{\psi}}
\newcommand{\chs}{SU(3)_L\times SU(3)_R}
\newcommand{\spm}{\hspace{0.3cm}}
\newcommand{\beq}{\begin{eqnarray}}
\newcommand{\eeq}{\end{eqnarray}}
\newcommand{\be}{\begin{equation}}
\newcommand{\ee}{\end{equation}}
\newcommand{\kl}{\scriptstyle}
\newcommand{\mini}{\scriptscriptstyle}
\newcommand{\qfl}{Q_{5{\mini L}}}
\newcommand{\qfr}{Q_{5{\mini R}}}
%
\newcommand{\su}{\cdot 10^{-3} m_\pi^{-1}}
\newcommand{\Rop}{\gamma p \to \pi^0 p}
\newcommand{\Ron}{\gamma n \to \pi^0 n}
\newcommand{\Rmp}{\gamma n \to \pi^- p}
\newcommand{\Rpn}{\gamma p \to \pi^+ n}
%
\newcommand{\Eop}{E_{0+}(\pi^0 p)}
\newcommand{\Eon}{E_{0+}(\pi^0 n)}
\newcommand{\Epn}{E_{0+}(\pi^+ n)}
\newcommand{\Emp}{E_{0+}(\pi^- p)}
%
\newcommand{\DEop}{\Delta E_{0+}(\pi^0 p)}
\newcommand{\DEon}{\Delta E_{0+}(\pi^0 n)}
\newcommand{\DEcn}{\Delta E_{0+}(\pi^\pm N)}
\newcommand{\cl}{\centerline}
%
%  arabic extensions for equation numbers
%  to be used with article style.
%  use as : \stepcounter{equation}
%           \alpheqn
%           ----  eqnarray ----------	
%           \reseteqn
%
\newcounter{saveeqn}
\newcommand{\alpheqn}{\setcounter{saveeqn}{\value{equation}}%
\setcounter{equation}{0}%
\renewcommand{\theequation}{\mbox{\arabic{saveeqn}-\alph{equation}}}}
\newcommand{\reseteqn}{\setcounter{equation}{\value{saveeqn}}%
\renewcommand{\theequation}{\arabic{equation}}}
%
% ************* new caption *************                             
%              (article style)
%
\makeatletter
\long\def\@makecaption#1#2{
 \vskip 10pt 
 \setbox\@tempboxa\hbox{\sc#1: \small#2}
 \ifdim \wd\@tempboxa >\hsize \sc#1: \small#2\par \else \hbox
to\hsize{\hfil\box\@tempboxa\hfil} 
 \fi}
\makeatother 
%
% *** this file to be used with modified art12.sty ***
% *** file  containing definition of wsection and  ***
% ***             wsubsection                      ***
%
\pagestyle{empty} 
\textheight22.5cm
\textwidth15.0cm
\oddsidemargin1cm
\evensidemargin1cm
\hoffset-25pt
\begin{document}
\renewcommand{\thefootnote}{\fnsymbol{footnote}} 
\centerline{\LARGE Neutral Pion Photoproduction at Threshold} 
\vskip8pt
\centerline{\LARGE and Explicit Chiral Symmetry Breaking}
\vskip1.5cm
\centerline{T. Sch\"afer\footnote{Work supported in part by
 DFG grant We 655/9-3 and EC project ${\rm SCC}^{*}$/0233-C.}}
\centerline{ Institute of Theoretical Physics, 
         University of Regensburg}
\centerline{  D-8400 Regensburg, W. Germany}
\vskip1.5cm  
\centerline{ABSTRACT}
\begin{center}
\begin{minipage}[t]{12.7cm}
{\footnotesize
We discuss the recent experiments on neutral
pion photoproduction on the proton close to threshold. Contrary to
earlier claims we find that the violation of the Low Energy Theorem
(LET) is small directly at threshold. We go on to discuss the 
importance of explicit chiral symmetry breaking corrections to the
LET result. Using chiral models of the nucleon to estimate
the size of these corrections, we find that the modification of the
LET prediction is rather small $\Delta E_{0+}(\pi^0p)<0.5\su$.}
\end{minipage}
\end{center}
\vskip1cm
\setcounter{footnote}{0}
\renewcommand{\thefootnote}{\arabic{footnote}}
\wsection{Introduction}
Using tagged photons beams, experimental groups at Saclay \cite{Maz86} and
Mainz \cite{Bec89,Bec90} have recently performed precise measurements of 
neutral pion photoproduction on the nucleon near threshold. From these 
results one can determine the threshold amplitude $E_{0+}$ defined by
\begin{equation}
|E_{0+}|^2 =
\left. \frac{k}{q}\, 
\frac{d \sigma}{d \Omega}
 \right|_{th}  ,
\end{equation}
where $k=|\vec{k}|$ and $q=|\vec{q}|$ denote the photon and pion momenta, 
respectively. The electric dipole amplitude $E_{0+}$ is a quantity of 
fundamental interest in the study of strong interactions since  Low Energy
Theorems (LET) based on charged current conservation and approximate chiral
symmetry (PCAC) give predictions for processes involving soft pions in terms
of just a few fundamental parameters.

Prior to 1986 existing experiments seemed to be in excellent agreement with 
Low Energy Theorems. It came as a big surprise when the first
data analyses gave values \cite{Maz86,Bec89} $\Eop=-0.35\pm 0.1 \su$ and
$\Eop=-0.5\pm 0.3\su$ indicating a strong violation of the 
Low Energy Theorem 
$
\Eop = -2.3 \su 
$.

In section 2 of this paper we will give a short review of the
multipole analysis. We show that there is an ambiguity in the
determination of $E_{0+}$  and that special care
is required in order to select the correct  solution. We argue
that both the angular distribution and the total cross section
data favor a value of $\Eop$ which is compatible with the
LET prediction. 

The original claims concerning a strong discrepancy between theory
and experiment have triggered a considerable amount of work on
possible corrections to the Low Energy Theorem.
Even if careful analyses and additional experiments do
in fact remove the discrepancy, the ideas developed are still important
in improving our understanding of the Low Energy Theorem and in
making predictions for neutral pion production on the neutron.
Of particular interest in this case is the amount of isospin
symmetry breaking, which is assumed to be zero in the standard
LET derivation. Furthermore, the data show a very strong energy
dependence of the electric dipole amplitude in the vicinity of
the charged pion threshold. This is an indication for the importance
of rescattering corrections. Present calculations \cite{LYL91},
while being able to explain this effect qualitaively have not yet 
achieved a quantitative description of the data based on a model 
consistent with chiral symmetry.
     
In this contribution we want to focus on the possible
role  of explicit chiral symmetry breaking by  the non vanishing 
current quark masses. Using a current algebra approach introduced
by Furlan and collaborators \cite{FPV74} in the early seventies it 
was suggested \cite{NS89} that a correction term analogous to the 
sigma commutator in low energy pion-nucleon scattering would give
a very large correction to the Low Energy Theorem. We review
the methods on which this claim is based and give estimates
of the correction using  chiral models of the nucleon.
  
\wsection{Pion Photoproduction Formalism}
\wsubsection{Threshold Amplitudes}
Before going into details we want to give a short review of the 
relevant formalism and the standard low energy
theorem. The photoproduction reaction $\gamma(k)+N(p_1)\to\pi^{\alpha}(q)+
N(p_2)$ is described by the T-matrix
\begin{equation}
\label{Tmat}
\epsilon^{\mu}T_{\mu}^{\alpha}= i\epsilon^{\mu}
<N(p_2)\pi^{\alpha}(q)|V^{em}_{\mu}|N(p_1)> ,
\end{equation}
where $V_{\mu}^{em}$ is the electromagnetic current and $\epsilon^{\mu}$
denotes the photon polarization vector. Introducing the isospin 
decomposition
\begin{equation}
 T_{\mu}^{\alpha}=\delta^{\alpha 3}T_{\mu}^{(+)}
+ i\epsilon^{\alpha 3\beta}\tau^\beta T^{(-)}_{\mu} +
 \tau^{\alpha}T^{(0)}_{\mu} ,
\end{equation} 
the amplitudes for the reaction $\Rop$ is given by
$
T_{\mu}(\pi^0 p) =  T_{\mu}^{(+)}+T_{\mu}^{(0)}  
$.
The threshold electric dipole amplitude is related to $T_\mu$ by
\be
 \left. E_{0+} \right|_{thr} \; \chi^\dagger_f 
      (\vec{\sigma}\cdot\vec{\epsilon})\chi_i =
   -\frac{e}{4\pi(1+\mu)}\left. \epsilon^\mu T_\mu  \right|_{thr} 
\ee     
where $\mu=m_\pi /M$ denotes the ratio of pion and nucleon masses.


\wsubsection{Multipole Analysis}
In this section we want to discuss the multipole
analysis carried out by the Mainz group and present arguments 
in favor of a threshold amplitude compatible with the Low 
Energy Theorem. Close to threshold the differential cross section
is dominated by s and p-waves
\be
\label{xang}
\frac{d\sigma}{d\Omega} = \frac{q}{k} \left(
 A + B\cos \Theta + C \cos^2\Theta \right) \; ,
\ee
the coefficients being determined by the electric and magnetic 
dipole amplitudes
\stepcounter{equation}
\alpheqn
\beq
\label{angcoef}
 A & = & |E_{0+}|^2 + \frac{1}{2} ( |P_1|^2 + |P_3|^2 )  \\
 B & = & 2 {\rm Re} (E_{0+}P_1^* ) \\
 C & = & |P_1|^2 - \frac{1}{2} ( |P_2|^2 + |P_3|^2 ) \; . 
\eeq
\reseteqn
Here we have introduced the p-wave mutipoles
\stepcounter{equation}
\alpheqn  
\beq
\label{pmult}
 P_{1,2} &=& 3 E_{1+} \pm M_{1+} \mp M_{1-} \\
 P_3     &=& 2 M_{1+} + M_{1-} \;\;\;\;\;\;\;\;\; .
\eeq
\reseteqn
Below the charged pion threshold, all amplitudes are 
essentially real. Neglecting the imaginary parts and using 
$E_{0+}<0$ and $P_1>0$ one ends up with two pairs of solutions 
\stepcounter{equation}
\alpheqn
\beq
 E_{0+} &=& \frac{1}{2} \left( -\sqrt{A-B+C} \pm \sqrt{A+B+C} \right) \\
 P_1    &=& \frac{1}{2} \left( +\sqrt{A-B+C} \pm \sqrt{A+B+C} \right)  \; . 
\eeq
\reseteqn
Here it is understood that the same sign is chosen in both equations.
Depending on this choice, one either has $|E_{0+}|>|P_1|$ 
or $|E_{0+}|<|P_1|$. Using the threshold behavior of the
amplitudes, one expects that $|E_{0+}|>|P_1|$ provides the 
correct result directly at threshold while the other solution has to be
used above some critical photon energy $E_\gamma$.

At the critical energy, one has $|E_{0+}|=|P_1|$, so that $A+B+C=0$.
In Table 1 we have listed the values of $A+B+C$ from ref.~\citenum{Bec90}.
The data clearly suggest that switching occurs
at $E_\gamma=149.1$ MeV which corresponds to the solution labeled
$E_{0+}^{I}$. Nevertheless, the authors of ref.~\citenum{Bec90} do
not entirely rule out the non-switching scenario $E_{0+}^{II}$.
\begin{table}
\caption{Electric dipole amplitudes extracted from the Mainz
data. Angular distribution coefficients in units $nb\,sr^{-1}$,
dipole amplitudes given in  $10^{-3} m_\pi^{-1}$.}
\begin{center}
\begin{tabular}{c|r|c|c|c} 
 $E_{\gamma}$  & $A+B+C$  &
         $ E_{0+}^I $ &  $E_{0+}^{II}$   & $E_{0+}^f$  \\ \hline
 146.8    & $36\pm 26$   & $-0.25\pm 0.20$  & $-1.59\pm 0.38$ 
                         & $-1.64\pm 0.10$     \\ 
 149.1    & $19\pm 27$   & $-0.72\pm 0.29$  & $-1.69\pm 0.45$ 
                         & $-1.26\pm 0.15$     \\  	      
 151.4    & $40\pm 31$   & $-0.56\pm 0.21$  & $-0.56\pm 0.21$ 
                         & $-0.79\pm 0.26$     \\  
 153.7    & $90\pm 34$   & $-0.25\pm 0.17$  & $-0.25\pm 0.17$ 
                         & $-0.54\pm 0.33$     \\ 
 156.1    & $189\pm 40$  & $-0.35\pm 0.14$  & $-0.35\pm 0.14$ 
                         & $-1.02\pm 0.39$     \\ 
\end{tabular}
\end{center}
\end{table}

A simpler way of extracting the electric dipole amplitude at threshold 
is based directly on the total cross section data \cite{Ber91,BH91}. 
Parametrizing the total cross section in terms of $E_{0+}$ and a p-wave 
contribution proportional to $qk$ gives 
\be
\label{xfit}  
\sigma_{tot}  \simeq  4\pi \left( \frac{q}{k} \right) 
 \left( ({\rm Re} E_{0+})^2 + 2f_0^2 (qk)^2 \right) \; .
\ee
The coefficient $f_0$ is dominated by the resonant $M_{1+}$ amplitude
which contributes $f_0\simeq 8.0 \cdot 10^{-3}m_\pi^{-3}$. Using
this number, one can determine $E_{0+}$ without any reference to the 
measured angular distributions. The results are listed in the last column 
of table 1. Close to threshold they show a clear preference for scenario II.
Extrapolating to the threshold energy $E_\gamma =144.7$ MeV, we obtain an
electric dipole amplitude $E_{0+}\simeq -2.1\su$ consistent with LET. 
At the highest energies, where  $l=1$ amplitudes other than $M_{1+}$ become
important, the agreement with the full multipole analysis is lost.  
 
\wsection{Low Energy Theorems}
\wsubsection{Current Algebra Ward Identities}
Low Energy Theorems for the pion photoproduction amplitude can be 
obtained by constructing Ward Identities for the two point function
\cite{MS79,Tab90}
\begin{equation}
\label{Pimunu}
\Pi^\alpha_{\nu\mu}(q) = \int d^4 x\, e^{iq\cdot x}<N(p_2)| T\left(
\overline{A}^\alpha_\nu (x) V_\mu^{em} (0) \right) |N(p_1)> \; .
\end{equation} 
where $\overline{A}_\nu^\alpha (x) = A_\nu^\alpha (x) + f_\pi \partial_\nu 
\pi^\alpha (x)$ denotes the axial current with the  pion contribution
subtracted. It has the important property 
\begin{equation}
\partial^\nu \overline{A}_\nu^\alpha (x) = f_\pi (m_\pi^2 + \Box )\pi^\alpha 
(x)=- f_\pi j_\pi^\alpha (x)
\end{equation}
with $j_\pi^\alpha (x)$  the pion source function. Removing the pion 
contribution has the advantage of giving a non-singular amplitude for 
physical pions with $q^2=m_\pi^2$. Contracting $\Pi^{a}_{\nu\mu}$ with the 
pion  momentum and using the definition of the photoproduction $T$-matrix
we get
\begin{equation}
\label{LET2}
T_\mu^\alpha (q) = \frac{1}{f_\pi}\left\{
iq^\nu \Pi_{\nu\mu}^\alpha (q) \, + \, C_\mu^\alpha (q)  \, + \,
i\frac{\omega_\pi}{m_\pi^2} \Sigma^\alpha_\mu (q) \right\} \; .
\end{equation}
The second term on the right involves an equal time commutator 
\be
 C_\mu^\alpha (q)  =i\int d^4x\, e^{iq\cdot x} \delta (x^0)
  <N(p_2)|[A_\mu^{\alpha}(x),V_\mu^{em}(0)] |N(p_1)>
\ee
of the axial current with the electromagnetic current. Using standard
current algebra this commutator gives an axial current leading to
the familiar Kroll-Ruderman term which dominates the production of
charged pions. The first term contains the two point function 
introduced above. In the soft pion limit $q\to 0$ it is dominated
by nucleon Born terms. Replacing  form factors by coupling constants
and using the Goldberger Treiman relation $g_A/f_\pi=2f/m_\pi$  
this contribution is easily recognized as the result of a tree
level calculation using a pseudovector $\pi N$ lagrangian.
The last term contains an equal time commutator involving the
divergence of the axial current
\be
 \Sigma_\mu^\alpha (q) = \int d^4 x\, e^{iq\cdot x} \delta (x^0)
 <N(p_2)| [\partial^\nu A_\nu^\alpha (x), V_\mu^{em} (0)] |N(p_1)>,
\ee
which is a direct measure of explicit chiral symmetry breaking.
Only the time component of $\Sigma_\mu^\alpha$ is determined
by current algebra. Using
\be
\int d^4x\, \delta(x^0) [\partial^\mu A_\mu^\alpha (x),V_0^{em}(0)]
  = -i\epsilon^{\alpha3\gamma}\partial^\mu A_\mu^\gamma (0)
\ee
this term can be identified as the pion pole contribution to the 
longitudinal part of the photoproduction amplitude at threshold.
The space components, however, are model dependent. Neglecting 
their contribution, we recover the standard LET prediction
\cite{Bae70}   
\be
\label{LET}
\Eop = \frac{e}{4\pi} \frac{f}{m_\pi}
     \left\{ -\mu + \frac{\mu^2}{2}(3+\kappa_p ) +
  {\cal O}(\mu^3) \right\}    \cong -2.32  \su \; .
\ee
In this result, the relative order of possible corrections  
has been estimated by assuming a power series expansion for the 
residual amplitude and exploiting the consequences of crossing 
invariance \cite{Bae70,Nau91}. An important ingredient in this argument
is the assumption that the coefficients of this power series 
can be determined in the chiral limit. This assumption is most 
probably incorrect. An explicit calculation of loop contributions 
in the context of chiral perturbation theory \cite{BKG91}  
indicates the existence of terms that diverge in the chiral
limit and therefore spoil the naive $m_\pi$ counting. 



\wsubsection{The method of Furlan et al.}
An interesting derivation of Low Energy Theorem based on 
currrent algebra commutators and completeness sum rules was
introduced by Furlan, Paver and Verzegnassi \cite{FPV74}.
It is based on the charge operator
\be
  \qfl^\alpha = Q_5^\alpha +\frac{i}{m_\pi} \dot{Q}_5^\alpha
\ee
where $Q_5^\alpha$ denotes the axial charge and $\dot{Q}_5^\alpha$
its time derivative. $\qfl^\alpha$ has the matrix elements
\be
  <0|\qfl^\alpha|\pi^\beta>=2i\delta^{\alpha\beta}f_\pi m_\pi
  \hspace{1cm} 
  <\pi^\beta|\qfl^\alpha|0> =0 \; .
\ee
Note that this approach is intimately connected with the method 
presented in the last section. In particular, $\qfl^\alpha$ has
the same matrix elements between pion states as the charge 
associated with the current $\overline{A}_\mu^\alpha$.
Furlan et al.~consider  completeness sums for the commutator
$D_\mu^\alpha=[\qfl^\alpha,V_\mu^{em}]$. Using the properties listed 
above, one can isolate the photoproduction $T$-matrix        
\beq
\label{fpv}
f_{\pi} T_{\mu}^{\alpha} &=&
 i  <N(p_2)|[Q_5,V_\mu^{em}]|N(p_1)> 
  + \frac{i}{m_\pi}<N(p_2)|[\qfl^{\alpha},V_{\mu}^{em}]
    |N(p_1)> \\
   & &\mbox{}-\sum_{N(p')}<N(p_2)| Q_{5L}^{\alpha}|N(p')>
   <N(p')|V_{\mu}^{em}|N(p_1)>
  +\, c.t.   \;+\; f_\pi \delta T_\mu^\alpha  \nonumber \; .
\eeq
Again one can identify the various contributions to the 
photoproduction amplitude. The sum in eqn. (\ref{fpv}) runs
over intermediate nucleon states only and 
together with the crossed contribution $c.t.$ represents
the nucleon Born terms. Other contributions to the completeness
sum are hidden in the residual amplitude $\delta T_\mu^\alpha$.
The two commutators in the first line are the Kroll-Ruderman
term and the contributions from explicit chiral symmetry breaking.

\wsubsection{Explicit chiral symmetry breaking}
As mentioned above, the space components of $\Sigma_\mu^\alpha$
are not determined by current algebra. However, representing
the currents in terms of quark fields, the commuator can be evaluated
using the canonical commutation relations among those fields 
\beq
\label{com}
\int d^3 x\, [\partial^{\nu}A_{\nu}^{\alpha}(\vec{x},0),V_{i}^{em}(0)]
& =&
i\,\overline{m} \epsilon_{ijk} \left\{  \delta^{\alpha 3}
     J_{jk}^0 +
 \frac{1}{3} J_{jk}^{\alpha}\right\}   \nonumber \\
& &\mbox{} + i\frac{\delta m}{2}\epsilon_{ijk}\delta^{\alpha 3} \left\{
 \frac{1}{3} J_{jk}^0
  + J_{jk}^{3}  \right\}  \; .
\eeq
Here $\overline{m}=1/2(m_u+m_d)$ and $\delta m=m_u-m_d$ denote the 
average and the difference of the light quark masses. Furthermore,
we have introduced the tensor currents
\be
\label{Ten}
 J_{\mu\nu}^{\alpha} = \bar{\psi} \sigma_{\mu\nu}
 \frac{\tau^{\alpha}}{2}\psi \; . 
\ee
Although eqn. (\ref{com}) is based on canonical commutators,
it was noted in ref. \citenum{ST77} that this result might suffer from
anomalies in the interacting theory. Using the  result from the
free theory the correction to the electric dipole amplitude can
be expressed in terms of tensor form factors 
\be
\label{DEon}
\Delta E_{0+}(\pi^0 N) = \frac{e}{4\pi f_\pi}\frac{\overline{m}}{m_\pi (1+\mu)}
  \left\{ \left( 1+\frac{\delta m}{6\overline{m}} \right) g_T^0
     \pm \left(\frac{1}{3}+\frac{\delta m}{2\overline{m}}\right) g_T^3
     \right\} \; ,
\ee
defined by
\be
\label{gt}
<N_f| \bar{\psi}\sigma_{\mu\nu}\frac{\tau^{\alpha}}{2}\psi|N_i>
= g_T^{\alpha}(q^2)\bar{u}_f\sigma_{\mu\nu}
 \frac{\tau^{\alpha}}{2} u_i + \ldots
\ee
The plus and minus signs in (\ref{DEon}) refer to protons 
and neutrons, respectively. Using standard values for the current
quark masses we see that since $\delta m/2\overline{m} \cong -1/3$,
$\Delta E_{0+}(\pi^0 N)$ is almost completely determined by $g_T^0$.

Using a non-relativistic constituent quark model of the nucleon,
the tensor couplings $g_T^\alpha$ are identical to the axial 
coupling constants $g_A^\alpha$. Within such a model, we have $g_A^0=1$
and $g_A^3=5/3$ which leads to a rather large correction 
$\DEop=1.8\su$. The non-relativistic quark model, however, violates
chiral symmetry even in the absence of quark masses. More refined
estimates of $\DEop$ will be given in section 4.   

\wsubsection{Gauge Invariance}

The consequences of current conservation can be studied by contracting
the two point function (\ref{Pimunu}) with the photon momentum $k^\mu$. 
Proceeding this way, we get
\begin{equation}
\label{gi}
ik^\mu \Pi_{\nu\mu}^\alpha (q) = - C^\alpha_\nu (q) + \frac{i}{m_\pi^2}
\left\{ q_\nu \Sigma_0^\alpha (q) - \delta_{\nu 0} k^\rho \Sigma_\rho^\alpha
 (q) \right\} .
\end{equation}
Using the fact that the pole contributions together with the Kroll-Ruderman
term  give a gauge invariant approximation to the full amplitude we may 
formally eliminate the pole terms from this equation. In deriving the
LET result we have neglected the residual contributions to 
$\Pi_{\nu\mu}^\alpha$ and $C_\mu^\alpha$. Therefore, we are left with
a condition for the space components of $\Sigma_\mu^\alpha$ 
\be
\vec{k}\cdot\vec{\Sigma}^\alpha (q) =0 \, .
\ee
Using the results from the last section, we see that this condition
is not satisfied. Additional contact terms in the longitudinal part
of the amplitude have to be added by hand in order to ensure gauge
invariance.
 
\wsection{Chiral Models of the Nucleon}
\wsubsection{Chiral Bag Model}
Now let us a proceed to an estimate of the symmetry breaking
correction using chiral models of the nucleon. We first consider
the chiral bag model \cite{BR88} defined by the lagrangian 
\be
\label{lcb}
{\cal L}= \left( \bar{\psi}\frac{i}{2}\gamma^{\mu}\stackrel{\leftrightarrow}
{\partial}_{\mu}\psi -B\right) \Theta (R-r) - \frac{1}{2} \bar{\psi}
e^{i\gamma_5 \vec{\tau}\cdot\vec{\phi}}\psi\delta (r-R)
+{\cal L}_{mes}\Theta (r-R) \; .
\ee
Here $\psi$ denotes an isodoublet quark spinor and $\vec{\phi}$ is the
isotriplet  pion field which couples to the quarks at the bag surface 
via a chirally invariant interaction. The quarks are confined to the 
interior  of the bag with radius $R$ while the meson
fields are restricted to the exterior region.

As usual the classical equations of motion are solved using  the Hedgehog
Ansatz. States with good spin and isospin are constructed by means of the 
semiclassical cranking procedure.

In the bag interior quarks are free so that we recover the results
(\ref{com},\ref{DEon}). In the exterior region the divergence of the axial 
current
is given by $\partial^{\mu}A_{\mu}^{\alpha}=f_{\pi}m_{\pi}^{2}\pi^{\alpha}$,
where $\pi^{\alpha}=\hat{r}^{\alpha}\sin F$ denotes the canonical pion field.
Since the space components of the electromagnetic current do not contain the
conjugate field $\dot{\pi}^{\alpha}$, the commutator 
$[\partial^{\mu}A_{\mu}^{\alpha},V_{i}^{em}]$ vanishes outside the bag.

The formfactors $g_T^{\alpha}$ can be determined in a 
straightforward manner. For the dominant singlet part we get
\be
\label{g0c}
g_T^{0}=\frac{T_{\sigma\tau}}{2\Lambda_{tot}}\; ,
\ee
where
\be
T_{\sigma\tau}=\sum_{ph}
\frac{<H|\vec{\tau}|ph>\cdot<ph|\gamma_0\vec{\sigma}|H>}{E_p-E_h}
\ee
involves a sum over particle hole excitations of the hedgehog
ground state and  $\Lambda_{tot}=\Lambda_{q}+\Lambda_{mes}$ denotes 
the total (quark core plus meson cloud) moment of inertia.

The results for $\DEop$ are given in figure 1. The exhibit a very strong
dependence on the bag radius $R$, suggesting that important gluonic
effects are missing from our description. Using the results for
$R=0.5$ fm which is the typical value used in the chiral bag model,
we get $\DEop=0.22\su$.  

\begin{figure}
\caption{Explicit chiral symmetry breaking corrections $\Delta
E_{0+}(\pi^0 p)$ (full line) and $\Delta E_{0+} (\pi^0 n)$ (dashed 
line) calculated in a chiral bag model with vector mesons as a function
of the bag radius $R$. The model parameters are $f_{\pi}=93$ MeV,
$m_{\pi}=139$ MeV and $g_{\rho\pi\pi}=5.85$.}
\vspace{7cm}
\end{figure}


\wsubsection{Non topological Soliton Model}
In order to study the model dependence of $\DEop$ we discuss
the chiral symmetry breaking correction in the non-topological chiral soliton 
model \cite{BB85}. The model Lagrangian is the one of the linear
sigma model :
\begin{eqnarray}
\label{Lsig}
{\cal L}&=&\bar{\psi}\left[ i\gamma^{\mu}\partial_{\mu}
-g(\sigma+i\gamma_5\vec{\tau}\cdot\vec{\pi})\right]\psi 
+\frac{1}{2}(\partial_{\mu}\sigma)^2
+\frac{1}{2}(\partial_{\mu}\vec{\pi})^2 \nonumber \\
& &\mbox{}-\frac{\lambda^2}{4}\left(\sigma^2+\vec{\pi}^2-\nu^2\right)^2
 + {\cal L}_{SB} \hspace{0.5cm},
\end{eqnarray}  
where $\sigma$ denotes a scalar-isoscalar field and $\vec{\pi}$ is a
pseudoscalar isotriplet pion field belonging to a linear representation of
$SU(2)_L\times SU(2)_R$. The parameters $\lambda$ and $g$ are the meson self
coupling and quark-meson coupling constants, respectively. Introducing 
explicit chiral symmetry breaking via non vanishing current quark massses,
we again get a correction $\DEop$ proportional to the nucleon tensor
coupling constants. 

As in the chiral bag model, we have used the hedgehog ansatz in order
to solve the equations of motion. The main difference as compared to
the chiral bag model is the fact that quarks are not confined. 
Instead the fields extend over the whole space. Nucleon states
are constructed using Peierls-Yoccoz projection. For the singlet
coupling, we get \cite{SW90}
\be
g_T^{0} = \frac{3}{2}\,\frac{1}{{\cal N}}
\int_0^{\pi}du\,\int_0^1 dt\,t^{5}\cos^{4}{\kl (}u{\kl )}\sin^2
{\kl (}u{\kl )}
\,e^{-\frac{4}{3}\overline{{\rm N}}_{\pi}(1-t^2 \cos^2 u)}\,\cdot R
\ee
where $R=0.82$ denotes a relativistic reduction and ${\cal N}$ is a
normalisation integral. The suppression factor
from the mesonic overlapp integral is controlled by the average number 
of pions $\overline{\rm N}_\pi$ in the hedgehog state. Using standard
values for the model parameters, the correction to the electric dipole
amplitude at threshold is $\DEop = 0.89 \su$. This number is rather
large  and indicates that the model is reminiscent of a constiuent
model. 

\wsection{Conclusions}
In summary we have studied the importance of explicit chiral symmetry
breaking corrections to Low Energy Theorems for neutral pion
photoproduction at threshold. The correction turns out to be strongly
model dependent with simple constituent quark models favoring a
rather large value for $\DEop$. Using chiral models of the nucleon,
the correction is significantly reduced. 

\wsection*{Acknowledgements}
I would like to acknowledge fruitful discussions with 
W.~Weise, H.~W.~L.~Nauss, T.-S.~H.~Lee, B.~Schoch and L.~Tiator.
    
\begin{thebibliography}{99}
\bibitem{Maz86} E. Mazzucato {\em et al.},
              Phys. Rev. Lett. {\bf 57} (1986) 3144
\bibitem{Bec89} R. Beck, Ph.~D.~Thesis, Mainz (1989), unpublished	      
\bibitem{Bec90} R. Beck {\em et al.}, Phys. Rev. Lett. {\bf 65} (1990) 1841
\bibitem{Ber91} J. C. Bergstrom, preprint, Sasketchawan Accelerator 
              Laboratory, (1991)
\bibitem{BH91} A. M. Bernstein, B. R. Holstein, to be published in 
              Comments on Nucl. and Part. Phys.  	       
\bibitem{Bae70} P. de Baenst, Nucl. Phys. {\bf B24} (1970) 633
\bibitem{NS89} L. M. Nath, S. K. Singh, Phys. Rev. {\bf C39} (1989) 1207 \\
              L. Tiator, D. Drechsel, Nucl. Phys. {\bf A508} (1990) 541
\bibitem{SW90}  T. Sch\"afer, W. Weise, Phys. Lett. {\bf B250} (1990) 6;
              preprint, University of Regensburg TPR 90-40, Nucl. Phys.
	      {\bf A}, in print 	      
\bibitem{FPV74} G. Furlan, N. Paver, G. Verzegnassi, Nuo. Cim. {\bf 20A}
              (1974) 295;\\
	      V. de Alfaro, S. Fubini, G. Furlan, C. Rosetti, Currents in
	      Hadron Physics, North Holland Publishing Company (1973)
\bibitem{MS79} J. T. MacMullen, M. D. Scadron, Phys. Rev. {\bf D20} (1979)
              1069; {\em ibid.} 1081  	      	        
\bibitem{Tab90} A. N. Tabachenko, Yad. Fiz. {\bf 51} (1990) 1026 	      
\bibitem{ST77} S. S. Shei, H. S. Tsao, Phys. Rev. {\bf D15} (1977) 3049 	      
\bibitem{BR88}  G. E. Brown, M. Rho, Comments Nucl. Part. Phys. {\bf 18} (1988)
              1 and refs. therein       
\bibitem{BB85} M. C. Birse, M. K. Banerjee, Phys. Rev. {\bf D31} (1985) 118
\bibitem{BKG91} V. Bernard, N. Kaiser, U. G. Meissner, preprint,
              Bern university BUTP-91/28
\bibitem{Nau91} H. W. L. Nauss, Phys. Rev. {\bf C43} (1991) R365, Phys. Rev.
              {\bf C44} (1991) 531 	      
\bibitem{LYL91} C. Lee, S. N. Yang, T.-S. H. Lee, J. Phys. {\bf G17} (1991)
              L131	      
\end{thebibliography}

\end{document}
