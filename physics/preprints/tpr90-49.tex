% revised Oct. 28 1990
\documentstyle[12pt]{article}
\pagestyle{empty} 
\textheight20.5cm
\textwidth16.5cm
\oddsidemargin1cm
\evensidemargin1cm
\hoffset-25pt
\begin{document}

\def\boxit#1{\vbox{\hrule\hbox{\vrule\kern3pt
         \vbox{\kern3pt#1\kern3pt}\kern3pt\vrule}\hrule}}
\setbox1=\vbox{\hsize 33pc  \centerline{ \bf Institut f\"ur Theoretische 
Physik der Universit\"at Regensburg} } 
$$\boxit{\boxit{\box1}}$$
\vskip0.8truecm

\hrule width 16.5 cm  \vskip1pt \hrule width 16.5 cm height1pt
\vskip3pt
October 1990\hfill{TPR-90-49}
\vskip3pt\hrule  width 16.5 cm height1pt  \vskip1pt  \hrule width 16.5 cm 
\vskip 1.5cm 

\renewcommand{\thefootnote}{\fnsymbol{footnote}} 
\centerline{\LARGE Threshold Pion Photoproduction} 
\vskip8pt
\centerline{\LARGE and Chiral Models of the Nucleon\footnote{Work 
supported in part by DFG grant We 655/9-3 and by BMFT grant 06 OR 762}}
\vskip2cm
\centerline{T. Sch\"afer and W. Weise\footnote{Visiting Scientist (Aug. -
Sept. 1990), Physics
Division, Argonne National Laboratory, Argonne, Ill., USA}}
\centerline{ Institute of Theoretical Physics, 
         University of Regensburg}
\centerline{  D-8400 Regensburg, W. Germany}
\vskip0.5cm
\vskip3cm  
{\underline{Abstract:}} 
We present calculations of the chiral and isospin symmetry breaking corrections
to the threshold pion photoproduction amplitude $E_{0+}$. The calculations are 
based on two chiral models of the nucleon: the chiral bag model and the
non topological soliton model. We point out a strong correlation between the 
symmetry breaking correction $\Delta E_{0+}$ and the flavor singlet axial
coupling constant $g_A^0$. We find that $\Delta E_{0+}$ from explicit chiral
symmetry breaking is substantial, but not sufficient to explain the 
discrepancy between the measured $\gamma p \to \pi^0 p$ threshold amplitude
and its predicted value based on the chiral Low Energy Theorem. 
\setcounter{page}{0}
\setcounter{footnote}{0}
\renewcommand{\thefootnote}{\arabic{footnote}}
\newpage
\pagestyle{plain}
\section{Introduction}
Thanks to the efforts of experimental groups at Saclay \cite{Maz} and
Mainz \cite{Bec} precise measurements of pion photoproduction on the nucleon
near threshold have recently become available. These results  
allow for a  
determination of the threshold amplitude $E_{0+}$ defined by
\begin{equation}
\left. \frac{| \vec{k}\,|}{| \vec{q}\, |}\, \frac{d \sigma}{d \Omega}
 \right|_{th}= | E_{0+} |^{2} ,
\end{equation}
where $\vec{k}$ and $\vec{q}$ denote the photon and pion momenta, respectively.
The electric dipole amplitude $E_{0+}$ is a quantity of fundamental interest
in the study of strong interactions since the time honoured Low Energy
Theorems (LET) based on charged current conservation and approximate chiral
symmetry (PCAC) give predictions for processes involving soft pions in terms
of just a few fundamental parameters.

Prior to 1986 existing experiments
seemed to be in excellent agreement with Low Energy Theorems. With the 
new results taken into account this agreement only persists in the case of
charged pion production. For neutral pion production on the proton, the
determination of the amplitude directly at threshold turns out to be ambigous
\cite{Bec} with the result depending on assumptions made about the magnitude of
the magnetic dipole amplitude $M_{1}$. This ambiguity disappears at slightly  
higher energies where the recent Mainz analysis gives a value
\newcommand{\E}{E_{0+}(\pi^0 p)}
\newcommand{\Ep}{E_{0+}}
\newcommand{\DE}{\Delta E_{0+}}
\newcommand{\su}{\cdot 10^{-3}m_{\pi}^{-1}}
\begin{equation}
\label{Exp}
\left. \E \right|_{k^0 =151.4\,{\rm MeV}} = (-0.56\pm 0.13) \su .
\end{equation}
This is clearly incompatible with the LET prediction \cite{Bae}
\begin{equation}
\label{pre}
\left.\E\right|_{k^0 =144.7\,{\rm MeV}} \cong -2.3 \su \; .
\end{equation}
At least in the absence of rescattering effects  chiral
Lagrangians do not allow for such a strong variation of the amplitude
with energy close to threshold. 
Several attempts have been made to resolve this discrepancy
\cite{Nat,Noz,Yan,Kam}.
Here we want to focus on the importance of explicit chiral symmetry
breaking.
 
Based on a method developed by Furlan and collaborators \cite{Fur}
in the early seventies it was recently suggested \cite{Nat} that a correction
term analogous to the sigma commutator in low energy pion-nucleon scattering
could neatly explain the difference between theory and experiment.
This result relies on the use of a simple non-relativistic
constituent quark model to estimate the magnitude of the photoproduction 
sigma term. Such simple models however break chiral symmetry even in the 
absence of quark masses. Therefore, their applicability in the context of 
dicussing corrections to low energy theorems based on PCAC appears to be
doubtful. 

The purpose of this paper is  to present calculations
of the symmetry breaking corrections based on chiral
models of the nucleon which have been applied successfully to the description
of nucleon structure. Specifically, we will consider the chiral bag model
\cite{Hos} as well as the non-topological chiral soliton model \cite{Bir}. In
both cases special attention will be paid to the connection between the
photoproduction sigma term and other  nucleon observables, in particular the
axial couplings of the nucleon. 

We shall find that the symmetry breaking correction $\Delta E_{0+}$ is
approximately proportional to the flavor singlet axial coupling constant
$g_A^0$. This is our main result. Much interest has recently been focused 
on $g_A^0$ as it measures the fraction of the nucleon spin carried
by the  quark constituents. The analysis of deep inelastic muon-proton
scattering with polarized beam and target \cite{EMC} gives $g_A^0 < 0.35$.
This implies a constraint for $\Delta E_{0+}$ which we shall dicuss
in detail.


\section{Pion Photoproduction Formalism}
\subsection{Threshold Amplitudes}
Before studying the effects of explicit chiral symmetry breaking let us give 
a short review of the relevant formalism and the standard low energy
theorem. The photoproduction reaction $\gamma(k)+N(p_1)\to\pi^{\alpha}(q)+
N(p_2)$ is described by the T-matrix
\begin{equation}
\label{Tmat}
\epsilon^{\mu}T_{\mu}^{\alpha}= i\epsilon^{\mu}
<N(p_2)\pi^{\alpha}(q)|V^{em}_{\mu}|N(p_1)> ,
\end{equation}
where $V_{\mu}^{em}$ is the electromagnetic current and $\epsilon^{\mu}=(0,\vec
{\epsilon})$ denotes the photon polarisation vector. Introducing the isospin 
decomposition
\begin{equation}
 T_{\mu}^{\alpha}=\delta^{\alpha 3}T_{\mu}^{(+)}
+\frac{1}{2} [\tau^{\alpha},\tau^3 ] T^{(-)}_{\mu} +
 \tau^{\alpha}T^{(0)}_{\mu} ,
\end{equation} 
the amplitudes for the various reaction channels are given by
\newcounter{saveeqn}
\newcommand{\alpheqn}{\setcounter{saveeqn}{\value{equation}}%
\setcounter{equation}{0}%
\renewcommand{\theequation}{\mbox{\arabic{saveeqn}-\alph{equation}}}}
\newcommand{\reseteqn}{\setcounter{equation}{\value{saveeqn}}%
\renewcommand{\theequation}{\arabic{equation}}}
\stepcounter{equation}
\alpheqn
\begin{eqnarray}
T_{\mu}(\pi^+ n) &=& \sqrt{2} (T_{\mu}^{(0)}+T_{\mu}^{(-)} ) \\
T_{\mu}(\pi^- p) &=& \sqrt{2} (T_{\mu}^{(0)}-T_{\mu}^{(-)} ) \\
T_{\mu}(\pi^0 p) &=& \;\;      T_{\mu}^{(+)}+T_{\mu}^{(0)}   \\
T_{\mu}(\pi^0 n) &=& \;\;      T_{\mu}^{(+)}-T_{\mu}^{(0)} .
\end{eqnarray}
\reseteqn
The threshold electric dipole amplitude $E_{0+}$ is related to $T$
by
\begin{equation}
 E_{0+} \; \chi_f^\dagger (\vec{\sigma}\cdot\vec{\epsilon}\,)\chi_i
  =- \frac{e}{4\pi(1+\mu)}\left. \, \epsilon^\mu T_\mu \right|_{th}
\end{equation}
where $\mu=m_\pi/M$, the ratio of (charged) pion and proton masses.

Using gauge invariance and PCAC several authors have derived low energy
theorems (LET) for pion photoproduction \cite{AG,Bae,VZ}. We quote the electric
dipole amplitudes at threshold in the form given by de Baenst \cite{Bae}: 
\stepcounter{equation}
\alpheqn
\begin{eqnarray}
\label{LET}
\Ep (\pi^+ n) &=& \frac{e}{4\pi} \frac{\sqrt{2}f}{m_\pi}
    \left\{ 1 - \frac{3}{2}\mu + {\cal O}(\mu^2) \right\}
    \cong 26.6 \su \\
\Ep (\pi^- p) &=& \frac{e}{4\pi} \frac{\sqrt{2}f}{m_\pi}
     \left\{ -1 + \frac{1}{2}\mu + {\cal O}(\mu^2) \right\}
    \cong -31.7 \su \\
\Ep (\pi^0 p) &=& \frac{e}{4\pi} \frac{f}{m_\pi}
     \left\{ -\mu + \frac{\mu^2}{2}(3+\kappa_p ) +
  {\cal O}(\mu^3) \right\}    \cong 2.32
  \su \\
\Ep (\pi^0 n) &=& \frac{e}{4\pi} \frac{f}{m_\pi}
     \left\{  \frac{\mu^2}{2}\kappa_n  +
  {\cal O}(\mu^3) \right\}  \cong -0.51 \su .
\end{eqnarray}
\reseteqn
We note that the low energy theorems have allowed us to express the 
photoproduction amplitude in terms of just a few on shell
properties of the nucleon, the pseudovector pion nucleon 
coupling constant $f^2/4\pi=0.08$ and the anomalous magnetic
moments $\kappa_p=1.793$ and $\kappa_n=-1.913$. Alternatively,
had we replaced $f/m_\pi$ by $g_A/2 f_\pi$ (Goldberger-Treiman relation)
and used the empirical values $g_A=1.26$ and $f_\pi=93.2$ MeV, the numbers 
in equation (\arabic{saveeqn}) would have to multiplied by a factor 0.94.  

The LET prediction for charged pion photoproduction (basically the 
Kroll-Ruderman theorem) agrees quite well with the empirical values
\cite{BL} $E_{0+}(\pi^+ n)=(27.9\pm 0.5)\su$ and $E_{0+}(\pi^- p)
=(-31.6\pm 0.6)\su$, whereas this is not the case for the $\pi^0 p$
channel as allready noted.

Equations (\arabic{saveeqn}) represent the first terms of an expansion in
powers of the soft pion parameter $\mu=m_{\pi}/M$. Notably, the 
leading terms for charged pion production are ${\cal O}(1)$ whereas
the leading term for $E_{0+}(\pi^0 p)$ is ${\cal O}(\mu)$: 
when discussing $\pi^0$ production, we are dealing with a small amplitude,
suppressed by one order of magnitude as compared to the Kroll-Ruderman term
which drives charged pion production. Corrections of $E_{0+}$ due 
to explicit chiral symmetry breaking are formally of order ${\cal O}(\mu)$.
Hence they can compete with the leading order result for neutral pion
production whereas they are expected to be only of minor importance 
in the case of $E_{0+}(\pi^+ n)$ and $E_{0+}(\pi^- p)$.      


\subsection{Chiral and Isospin Symmetry Breaking Terms}
Let us now come to the discussion of symmetry breaking contributions to
$E_{0+}$. A technique which is especially  well suited for this purpose is the
method of Furlan et al.\ \cite{Fur}. Using their scheme  the amplitude for
producing a pion at rest in the nucleon Breit frame can be represented as
\newpage
\begin{eqnarray}
\label{Amp}
f_{\pi} T_{\mu}^{\alpha}(\vec{q}=0) &=&
 i \epsilon^{\alpha 3 \gamma} <N(\vec{p}\,)|A_{\mu}^{\gamma}|N(-\vec{p}\,)> 
  \nonumber \\
   & &\mbox{}-\sum_{N'}^{\mbox{}}<N(\vec{p}\,)| Q_{5L}^{\alpha}|N'>
   <N'|V_{\mu}^{em}|N(-\vec{p}\,)>
  \hspace{0.5cm} + \hspace{0.5cm}  c.t.   \\ \vspace{0.4cm}
   & &\mbox{}  + \frac{i}{m_\pi}<N(\vec{p}\,)|[\dot{Q}_5^{\alpha},V_{\mu}^{em}]
    |N(-\vec{p}\,)> \hspace{0.5cm}
    + \hspace{0.5cm} {\cal O}(m_{\pi}^2)^{\mbox{}} \nonumber . 
\end{eqnarray}
where {\em c.t.} denotes the matrix element with the operators in the
reverse order.
Here $A^{\alpha}_{\mu}$ is the axial current, $Q_5^{\alpha}=\int d^3 x\,
A_0^{\alpha}(x)$ the associated charge and 
\begin{equation}
 \dot{Q_5^{\alpha}}=\int d^3 x\, \partial^{\nu}A_{\nu}^{\alpha}(x)
\end{equation}
its time derivative, which is a direct measure of explicit chiral 
symmetry breaking. Furthermore $Q^{\alpha}_{5L}=
Q^{\alpha}_5 + \frac{i}{m_\pi}\dot{Q_5^{\alpha}}$ is a specific 
combination of axial charges introduced in \cite{Fur} in order to
isolate the photoproduction T-matrix in the completeness sum.
The first two terms in (\ref{Amp}) generate the familiar Kroll-Ruderman
amplitude and, to order ${\cal O}(m_\pi)$, the result from evaluating the 
Born terms. Contributions to the sum over intermediate states
that come from higher resonances and the $\pi N$ continuum  are hidden
in the ${\cal O}(m_{\pi}^2)$ term. These parts of the T-matrix are 
the object of standard photoproduction theory, and we shall not deal 
with them here. Our interest is in the commutator
\begin{equation}
\label{Sig}
 \Sigma_{\mu}^\alpha = \frac{i}{f_\pi m_\pi} <N(\vec{p}\,)|
  [\dot{Q}_5^{\alpha},V_{\mu}^{em}(0)]|N(-\vec{p}\,)> \; ,
\end{equation}
sometimes referred to as the photoproduction sigma term in analogy
with the sigma term in pion-nucleon scattering.

Let us now specify the currents $V_{\mu}^{em}$ and $A_{\mu}^{\alpha}$
in terms of quark fields $\psi=(u,d,s)$. For completeness, we start
from flavor $SU(3)$ although it will become clear that the strange quarks
decouple from the sigma term (\ref{Sig}). We have
\begin{equation}
\label{vector}
 V_{\mu}^{em}= \frac{1}{2} \bar{\psi}\gamma_{\mu}(\lambda^3+\frac{1}
 {\sqrt{3}}\lambda^8)\psi
\end{equation}
and
\begin{equation}
\label{axial}
A_{\mu}^{\alpha}=\frac{1}{2}\bar{\psi}\gamma_{\mu}\gamma_5\lambda^{\alpha}
\psi
\end{equation}
where $\lambda^{\alpha}$ are the standard Gell-Mann flavor matrices.
We assume that the explicit breaking of chiral $SU(3)\times SU(3)$
comes entirely from the quark mass term ${\cal L}_{SB} = -\bar{\psi}
M\psi$ in the QCD  Lagrangian, where $M={\rm diag}(m_u,m_s,m_d)$ is the
mass matrix for the three light flavors. Then
\begin{equation}
\label{Div}
\partial^{\nu} A_{\nu}^{\alpha} = i\bar{\psi}\gamma_5 \left\{M,
 \frac{\lambda^{\alpha}}{2}\right\}\psi
\end{equation}
and we get the following explicit expression for the  commutator
in (\ref{Sig}) :
\begin{eqnarray}
\label{Com}
\int d^3 x\, [\partial^{\nu}A_{\nu}^{\alpha}(x),V_{i}^{em}(0)]& =&
i\,\overline{m} \epsilon_{ijk} \left\{  \delta^{\alpha 3}
\frac{1}{\sqrt{3}}\left( \sqrt{2} J_{jk}^{0}+J_{jk}^{8} \right) +
 \frac{1}{3} J_{jk}^{\alpha}\right\}   \nonumber \\
& &\mbox{} + i\frac{\delta m}{2}\epsilon_{ijk}\delta^{\alpha 3} \left\{
 \frac{1}{3\sqrt{3}}\left(\sqrt{2} J_{jk}^{0}+ J_{jk}^{8} \right)
  + J_{jk}^{3}  \right\}  
\end{eqnarray}
which is valid for $\alpha=1,2,3$.
Here we have introduced the tensor currents
\begin{equation}
\label{Ten}
 J_{\mu\nu}^{\alpha}(x) = \bar{\psi}(x) \sigma_{\mu\nu}
 \frac{\lambda^{\alpha}}{2}\psi (x) 
\end{equation}
(to be taken at $x=0$ in eq. (\ref{Com}) ) and in our normalisation
$\lambda^0 =\sqrt{\frac{2}{3}}{\bf  1}$. Furthermore
\stepcounter{equation}
\alpheqn
\begin{eqnarray}
\overline{m} &=& \frac{1}{2} (m_u + m_d) ,  \\
\delta m     &=& m_u - m_d .
\end{eqnarray}
\reseteqn
In the actual calculations we use $\overline{m}=(7\pm 2)$ MeV and
$\delta m/2\overline{m} =(-0.29\pm 0.03)$ as quoted in \cite{GaL}.

Evidently, the strange quark sector is not involved in (\ref{Com})
since $(\sqrt{2}\lambda^0 +\lambda^8)/\sqrt{3}$ is just the unit
matrix in the $SU(2)$ (isospin) subspace. In order to evaluate (\ref{Sig})
we introduce tensor form factors $g_{T}^{\alpha}(q^2)$ of the nucleon by
\begin{equation}
\label{For}
<N_f| \bar{\psi}\sigma_{\mu\nu}\tau^{\alpha}\psi|N_i>
= g_T^{\alpha}(q^2)\bar{u}_f\sigma_{\mu\nu}
 \tau^{\alpha} u_i + \ldots
\end{equation}
where $\tau^0={\bf 1}$ and $\tau^\alpha \,(\alpha=1,2,3)$ are the Pauli
matrices. Furthermore, we select the nucleon Breit frame in which
\begin{equation}
\label{Breit}
\bar{u}(-\vec{p}\,)\sigma_{jk}u(\vec{p}\,)=\epsilon_{jkm}\chi^{\dagger}_f 
(\sigma_{T\, m} +\frac{E}{M} \sigma_{L\, m})\chi_i \; ,
\end{equation} 
\begin{equation}
\vec{\sigma}_{L}=\hat{p}(\vec{\sigma}\cdot\hat{p})  \hspace{0.5cm},
\hspace{0.5cm}
\vec{\sigma}_{T}=\vec{\sigma}-\vec{\sigma}_{L} .
\end{equation}
Approximating $g_T^\alpha (q^2 )$ by its value $g_T^\alpha $ at $q^2=0$,
the symmetry breaking correction to $E_{0+}(\pi^0 N)$ is finally obtained as
\begin{equation}
\label{result}
\Delta E_{0+}(\pi^0 N) = \frac{e}{4\pi f_\pi}\frac{\overline{m}}{m_\pi (1+\mu)}
  \left\{ \left( 1+\frac{\delta m}{6\overline{m}} \right) g_T^0
     \pm \left(\frac{1}{3}+\frac{\delta m}{2\overline{m}}\right) g_T^3
     \right\} \; ,
\end{equation}
where  the plus and minus signs refer to protons and neutrons, respectively.
Using the actual numbers we see that since $\delta m/2\overline{m} \cong -1/3$,
$\Delta E_{0+}(\pi^0 N)$ is almost completely determined by $g_T^0$. We point
out that $\overline{m}/m_\pi$ is actually proportional to $m_\pi$: The
Gell-Mann-Oaks-Renner relation implies $\overline{m}=m_\pi^2f_\pi^2
/|<\bar{u}u+\bar{d}d>|$, with the quark condensates $<\bar{u}u>\cong
<\bar{d}d>\cong -(240{\rm MeV})^3$.
For charged pion production, we have
\begin{equation}
\label{charged}
\Delta E_{0+} (\pi^- p)= \Delta E_{0+} (\pi^+ n) = 
\frac{\sqrt{2}e}{4\pi f_\pi} \frac{\overline{m}}{m_\pi (1+\mu) }
 \frac{1}{3} g_T^3\; .
\end{equation}
Note  that the isospin breaking term does not enter in the correction for
the charged amplitudes. In particular, the charge asymmetry $E_{0+}(\pi^+ n)
-E_{0+}(\pi^- p)$ is not affected by isospin breaking.


The main task in determining $\Delta E_{0+}$ is now to calculate the 
tensor form factors $g_T^{\alpha}$ of the nucleon. Note that these
relate to genuine tensor currents, not to be confused with the induced
tensor currents that appear in the anomalous Pauli terms of the nucleon
vector current matrix elements.       

\subsection{Gauge Invariance}
The gauge invariance condition for the photoproduction amplitude (\ref{Tmat})
reads $k^{\mu}T_\mu^\alpha =0$. Taking into account that the standard 
low energy theorem \cite{Bae} gives a manifestly gauge invariant amplitude,
one might expect that the photoproduction sigma term should be gauge
invariant by itself, i.e. $k^\mu \Sigma^\alpha_\mu = 0$. In this section 
we want to show that this naive gauge invariance constraint is not
satisfied and we will try to give an improved version. 

Let us first note that the time component $\Sigma_0^\alpha$ 
is uniquely determined by current
algebra and electromagnetic current conservation. Using the commutator
\begin{equation}
 [Q_5^\alpha ,V_\mu^{em}(x)] =-i\epsilon^{\alpha 3 \gamma} A_\mu^\gamma (x)
\end{equation}
and $\partial^\mu V_\mu^{em}(x)=0$ one obtains\footnote{Note that one can
also reproduce this result directly, using the QCD currents (\ref{vector})
and (\ref{axial}).} 
\begin{equation}
\Sigma_0^\alpha =\frac{1}{f_\pi m_\pi}\epsilon^{\alpha 3 \gamma}
 <N(\vec{p}\,)|\partial^\mu A_\mu^\gamma (0) |N(-\vec{p}\,)> \; . 
\end{equation} 
Using PCAC and the $\pi NN$ form factor defined by
\begin{equation}
 <N_f| j_\pi^\alpha|N_i> = g_{\pi{\scriptscriptstyle NN}} (q^2)
 \, \bar{u}_f i\gamma_5 \tau^\alpha u_i
\end{equation}  
we get 
\begin{equation}
\label{Sig0}
 \Sigma_0^\alpha = \frac{g_{\pi{\scriptscriptstyle NN}}
 m_\pi}{t-m_\pi^2}\epsilon^{\alpha 3 \gamma}
 \bar{u}(\vec{p}\,)i\gamma_5\tau^\alpha u(-\vec{p}\,) \; .
\end{equation}
At threshold, $t=-m_\pi^2/(1+\mu)$ so that $\Sigma_0^\alpha$ gives a 
sizeable isospin-odd term
whereas the space components $\Sigma_i^\alpha$ only contribute to
the symmetric components of the $T$-matrix $T^{(+,0)}$. Therefore, 
the  condition $k^\mu \Sigma^\alpha_\mu=0$ is certainly not valid. However, 
we also note that (\ref{Sig0}) is just the pion
pole contribution to $T_0^\alpha$ as evaluated in a  chiral lagrangian model
with pseudoscalar or pseudovector $\pi N$ interaction. In such a treatment
of the problem the pion pole term (\ref{Sig0}) forms a gauge invariant part of 
the amplitude together with the nucleon Born terms.  This illustrates the need 
to eliminate the pole contributions before imposing any constraints on
the chiral symmetry breaking part of the amplitude.

In order to proceed in a systematic way let us consider Ward identities
constructed from the time ordered product \cite{Rad,MMS}
\begin{equation}
\label{Pimunu}
\Pi^\alpha_{\nu\mu}(q) = \int d^4 x\, e^{iq\cdot x}<N(p_2)| T\left(
\bar{A}^\alpha_\nu (x) V_\mu^{em} (0) \right) |N(p_1)>
\end{equation}
where $\bar{A}_\nu^\alpha (x) = A_\nu^\alpha (x) + f_\pi \partial_\nu 
\pi^\alpha (x)$ has the  property 
\begin{equation}
\partial^\nu \bar{A}_\nu^\alpha (x) = f_\pi (m_\pi^2 + \Box )\pi^\alpha (x)
=- f_\pi j_\pi^\alpha (x)
\end{equation}
with $j_\pi^\alpha (x)$  the pion source function. The advantage of this
formalism  as compared to the simpler method of Furlan introduced in the last
section is that it generates the complete LET amplitude without the 
necessity to use the gauge invariance condition already at this point.
Using the method of Furlan and restricting the set of intermediate states 
to the nucleon, a  model independent determination of the transverse part
of $T_i^{(+,0)}$ is not  possible \cite{FPV}. In order to reproduce the
low energy theorem for $\E$ one has to use $T_0^{(+,0)}$ as well as the 
gauge invariance condition\footnote{Strictly speaking,  gauge invariance
does not constrain the transverse part of the amplitude so that
an additional assumption is necessary to reproduce the ${\cal O}(\mu)$
term in $E_{0+}(\pi^0 p)$ (see\cite{FPV}).} 
 $k^\mu T^\alpha_\mu =0$. 

Contracting (\ref{Pimunu}) with the pion momentum we again get 
the photoproduction amplitude
\begin{equation}
\label{LET2}
T_\mu^\alpha (q) = \frac{1}{f_\pi}\left\{
iq^\nu \Pi_{\nu\mu}^\alpha (q) \, + \, C_\mu^\alpha  \, + \,
\frac{i q^0}{m_\pi^2} \Sigma^\alpha_\mu (q) \right\} \; ,
\end{equation}
now in terms of the quantity $q^\nu \Pi_{\nu\mu}^\alpha (q)$ which 
can be approximated by evaluating the pole contrbutions, a current
algebra contribution
\begin{equation}
 C_\mu^\alpha  =i \epsilon^{\alpha 3 \gamma} <N(p_2)| A_\mu^\gamma (0)|N(p_1)>
\end{equation}
and the sigma term
\begin{equation}
 \Sigma_\mu^\alpha (q) = \int d^4 x\, e^{iq\cdot x} \delta (x^0)
 <N(p_2)| [\partial^\nu A_\nu^\alpha (x), V_\mu^{em} (0)] |N(p_1)>.
\end{equation}
The consequences of current conservation can be studied by contracting
(\ref{Pimunu}) with the photon momentum $k^\mu$. Proceeding this way, we get
\begin{equation}
\label{gi}
ik^\mu \Pi_{\nu\mu}^\alpha (q) = - C^\alpha_\nu + \frac{i}{m_\pi^2}
\left\{ q_\nu \Sigma_0^\alpha (q) - \delta_{\nu 0} k^\rho \Sigma_\rho^\alpha
 (q) \right\} .
\end{equation}
Assuming that the pole contributions together with the Kroll Rudermann
term  give a gauge invariant approximation to the amplitude we may formally
eliminate the pole terms from this equation. Denoting the quantities 
obtained that way by $\tilde{\Pi}^\alpha_{\nu\mu}$,$\tilde{C}^\alpha_{\nu\mu}$
and $\tilde{\Sigma}_\mu^\alpha$ we get
\begin{equation}
ik^\mu \tilde{\Pi}^\alpha_{\nu\mu}(q)=-\tilde{C}^\alpha_\nu +
\frac{i}{m_\pi^2} \delta_{\nu 0} \vec{k}\cdot\tilde{\vec{\Sigma}}^\alpha (q)
\end{equation}
where we have again used pion pole dominance for the time component
$\Sigma_0^\alpha$. In derivations of the low energy theorems it is
usually assumed that non pole terms are small at threshold. For this reason we
can get a consistent description by neglecting the non pole parts in the 
amplitude and imposing the constraint
\begin{equation}
\vec{k}\cdot\vec{\Sigma}^\alpha (q) =0 \, .
\end{equation}
Using the Breit frame kinematics we observe that the sigma term as calculated
in the last section does not satisfy this condition since it contains
longitudinal as well as transverse components (\ref{Breit}). While the 
transverse parts are explicitly invariant under a gauge transformation
$\vec{\epsilon} \to \vec{\epsilon} +\chi\vec{k}$, the longitudinal terms
are not.  In order to
construct a gauge invariant amplitude we have to add a correction $\delta
\vec{\Sigma}^\alpha$ such that $\vec{\Sigma}^\alpha_{\scriptscriptstyle GI}=
\vec{\Sigma}^\alpha + \delta\vec{\Sigma}^\alpha$ satisfies $\vec{k}\cdot
\vec{\Sigma}_{\scriptscriptstyle GI}^\alpha =0$. This can be achieved by 
subtracting the longitudinal part of the amplitude, i.e.
\stepcounter{equation}
\alpheqn
\begin{eqnarray}
\delta\vec{\Sigma}^{(+)} &=&-\frac{\overline{m}}{f_\pi m_\pi}
\left\{ g_T^0 +\frac{\delta m}{2\overline{m}} \left( \frac{1}{3}g_T^0
\pm g_T^3 \right)
 \right\}\; \chi^\dagger_f \left(\frac{E}{M} \vec{\sigma}_L \right)
  \chi_i  \;\; , \\
\delta\vec{\Sigma}^{(0)} &=& -\frac{\overline{m}}{3 f_\pi m_\pi}
  g_T^3 \; \chi_f^{\dagger} \left(\frac{E}{M} \vec{\sigma}_L \right) \chi_i
  \;\; .
\end{eqnarray}
\reseteqn 
It is important to note that this subtraction does not change the 
corrections (\ref{result},\ref{charged}) to the electric dipole amplitude at
threshold.  It does
however eliminate any contribution from the photoproduction sigma term 
to the longitudinal multipole $L_{0+}$ with
\begin{equation}
-k^2 \left| \frac{L_{0+}}{k^0} \right|^2 =\left. \frac{|\vec{k}\,|}{|\vec{q}\,|}
\frac{d\sigma_L}{d\Omega} \right|_{th}
\end{equation}
which can be measured in pion electroproduction experiments. 
    
   
\subsection{An example: Non-relativistic Quark Model}
Before going into the calculation of $g_T^{0}$ and $g_T^{3}$ in chiral nucleon
models, it is useful to go through a critical examination of results
for $\Delta E_{0+}(\pi^0 p)$ obtained in naive non-relativistic quark models.
In such models the lower components of quark Dirac spinors vanish.  
Therefore, the tensor matrix elements coincide with those of the axial current:
both $\epsilon_{ijk}\sigma_{jk}/2$ and $\gamma_i \gamma_5$ reduce to
$\sigma_i$ in the non relativistic limit. Hence $g_T^{0}$ equals $g_A^0$ and
$g_T^{3}$ equals $g_A^3 \equiv g_A$, with the axial coupling constants defined
by
\begin{equation}
<N_f |\bar{\psi}\gamma_{\mu}\gamma_5 \frac{\tau^{\alpha}}{2}\psi|N_i >
= g_A^{\alpha} \bar{u}_f \gamma_{\mu}\gamma_5 \frac{\tau^{\alpha}}{2}
 u_i .
\end{equation}
Returning to equation (\ref{result}) we find 
\begin{equation}
\label{thumb}
\Delta\E \cong \frac{e}{4\pi f_\pi}\frac{\overline{m}}{m_\pi (1+\mu) }
 (0.90 \cdot g_A^0 + 0.04 \cdot g_A^3 ).
\end{equation}
In the non-relativistic quark model we have $g_A^0=1$ and $g_A^3=5/3$, so
that $\Delta\E\cong e\overline{m}/(4\pi f_\pi m_\pi ) \cong 1.8 \su$. It is
amusing to note that this large number would almost cancel the 
LET result $\E \cong -2.3\su$ and  bring the total $E_{0+}$  down 
in magnitude to the measured values. This is basically the 
observation made in \cite{Nat}.

However, there are several weak points in this result. First the 
non-relativistic quark model violates chiral symmetry and is therefore
not consistent with the current algebra rules that lead to the 
derivation of $\Delta E_{0+}$.   Secondly, the empirical $g_A^0$
taken  from the measurements of the spin dependent structure 
function of the proton is probably only a fraction of unity, so that $\E$ 
should be significantly smaller than the value suggested by the naive
quark model. Note that replacing $g_A=5/3$ by the empirical value
$g_A\cong 5/4$ has very little influence on the final result.
 
\section{Chiral Models of the Nucleon}
\subsection{Chiral Bag Model}
Now let us proceed to a determination of the photoproduction sigma
term (\ref{Sig}) in the chiral bag model. The lagrangian of the model is given
by
\begin {equation}
\label{LCB}
{\cal L}= \left( \bar{\psi}\frac{i}{2}\gamma^{\mu}\stackrel{\leftrightarrow}
{\partial}_{\mu}\psi -B\right) \Theta (R-r) - \frac{1}{2} \bar{\psi}
e^{i\gamma_5 \vec{\tau}\cdot\vec{\phi}}\psi\delta (r-R)
+{\cal L}_{mes}\Theta (r-R)
\end{equation}
Here $\psi$ denotes an isodoublet quark spinor and $\vec{\phi}$ is the
isotriplet  pion field coupled to the quarks at the bag surface 
via the standard $SU(2)_L \times SU(2)_R$ invariant
coupling. The quarks are confined to the interior of the bag with
radius $R$ while the meson
fields are restricted to the exterior region. ${\cal L}_{mes}$
incorporates vector and axial vector mesons
$\rho,a_1,\omega$ in addition to the pion
fields as descibed in detail in reference \cite{Hos}. Here we neglect
the presence of gluons inside the bag except for their cumulative 
effect on the energy density which is included in the bag constant
$B$.

As usual the classical equations of motion are solved using  the Hedgehog
Ansatz.  The resulting meson profiles and quark wavefunctions can be 
found elsewhere \cite{Hos}. Here we will only need the value of the 
chiral angle $F(r)$ defined by $\phi^{\alpha}=\hat{r}^{\alpha}F(r)$  
at the bag surface and the results for the quark and meson moments of inertia.

States with good spin and isospin are constructed by means of the 
semiclassical cranking procedure. To first order in the cranking
frequency $\vec{\Omega}$, this leads to the quark wavefunction
\begin{equation}
\label{wav}
|\psi>=|H>-\sum_{ph}\frac{|ph><ph|\vec{\tau}|H>}{E_p -E_h}
\cdot \frac{\vec{\Omega}}{2}
\end{equation}
where $|H>$ denotes the hedeghog quark state and $|ph>$ are its 
particle hole excitations.$E_p$ and $E_h$ are the energies of 
particle and hole states, respectively. 

In the bag interior the sigma like commutator is given in terms of 
the quark currents by
(\ref{Com}) so that all we have to do is calculate the formfactors
(\ref{For}) using the explicit form of the wavefunction  (\ref{wav})
in the chiral bag model. In the exterior region the divergence of 
the axial current is given by
\begin{equation}
\label{PCAC}
\partial^{\mu}A_{\mu}^{\alpha}=f_{\pi}m_{\pi}^{2}\pi^{\alpha} \;\; , 
\end{equation}
where $\pi^{\alpha}=\hat{r}^{\alpha}\sin F$ denotes the canonical pion field.
Since the space components of the electromagnetic current do not contain the
 conjugate field
$\dot{\pi}^{\alpha}$, the commutator $[\partial^{\mu}A_{\mu}^{\alpha},
V_{i}^{em}]$ vanishes outside the bag. 

Inside the bag the formfactors $g_T^{\alpha}$ can be determined in a 
straightforward manner. For the singlet part we get
\begin{equation}
\label{G0C}
g_T^{0}=\frac{T_{\sigma\tau}}{2\Lambda_{tot}}
\end{equation}
with
\begin{equation}
T_{\sigma\tau}=\frac{N_c}{3}\sum_{ph}
\frac{<H|\vec{\tau}|ph>\cdot<ph|\gamma_0\vec{\sigma}|H>}{E_p-E_h}.
\end{equation}
Here $\Lambda_{tot}=\Lambda_{q}+\Lambda_{mes}$ denotes the total 
(quark core plus meson cloud) moment
of inertia (see  \cite{Hos}) and $N_c$ is the number of colors.
This result is very similar to the one for the quark part of the flavor
singlet axial coupling $g_A^0$. In fact, the two coincide if the bag radius
corresponds to the 'magic` chiral angle $F(R)=\pi/2$. At this point the
valence  quark level meets the Dirac sea and the lower components of the quark
Dirac spinor vanish. One finds $g_T^0=g_A^0$ in accordance with the naive
quark model, but now with $g_A^0({\rm quark}) <1$.
The valence quark contribution to the triplet form factor is given by
\begin{equation}
\label{G3C}
g_T^{3}=\frac{1}{3}\cdot\frac{2+2y^2-3y/\Omega}{1+y^2-2y/\Omega}
\end{equation}
where $\Omega$ denotes the energy eigenvalue (in units of the bag radius,
$\Omega=ER$) and
\begin{equation}
\label{y}
y=\frac{\cos F(R)}{1+\sin F(R)}
\end{equation}
is a function of the chiral angle at the bag surface. As it is the case
for the axial coupling the result (\ref{G3C}) remains finite even for 
vanishing bag radius. Therefore it is  necessary to include 
contributions coming from states in the Dirac sea. Using the standard
point splitting prescription  the result can be written as
\begin{equation}
\label{mod}
g_T^{3}(F,\tau)=-\frac{1}{2}\sum_n {\rm sgn}(\Omega_n)g_T^{3}
(\Omega_n)e^{-\tau|\Omega_n|}\;+\;\Theta(\Omega_0)g_T^{3}(\Omega_0)
\end{equation}
where $g_T^{3}(\Omega_n)$ denotes the contribution from the level with
eigenvalue $\Omega_n$ and $\Omega_0$ is taken to be the valence level. The
sum (\ref{mod}) diverges in the limit $\tau\rightarrow 0$ so that some
regularisation method is needed. In contrast to the situation for the axial
coupling this regularisation can not be unique since there are no 
conservation laws that can be used to fix finite counterterms. 
Nevertheless, given the close connection between $g_T^3$ and $g_A$
we  propose to proceed as in the case of the axial coupling
\cite{Vep}
\begin{equation}
\label{reg}
g_T^{3}(F)=\lim_{\tau\to 0} \left\{g_T^{3}(F,\tau)-\frac{1}{2}
\sin(2F)\left.\frac{dg_T^{3}(F,\tau)}{dF}\right|_{F=0}\right\} \; .
\end{equation}
In the numerical calculations we have for simplicity restricted the summation
in (\ref{mod}) to states with the quantum numbers $G^{\pi}=0^{\pm}$, where 
$G=J+I$ denotes grandspin while $\pi=(-)^{L}$ is parity. This restriction
is exact for $F(R)=0,\pi/2$ and $\pi$. As one can check in the case of the 
axial coupling for which good parametrisations of the exact result are 
available \cite{Gol}, this approximation introduces only small errors for
other values of the  chiral angle. In any case, with isospin symmetry 
breaking included, we recall from eq. (\ref{result}) that $\Delta E_{0+}
(\pi^0 N)$ is dominated by $g_T^0$ and therefore insensitive to small
errors in $g_T^3$. 

The final results for $\Delta E_{0+}(\pi^0 N)$ in the chiral bag model are 
shown in figure 1. They depend strongly on the bag radius. In particular,
$\Delta E_{0+}(\pi^0 N)$ vanishes for $R=0$ which corresponds to the result
for a vector meson stabilized skyrmion. The result for $R\to\infty$ however,
remains somewhat smaller than the one expected from the MIT bag model,
$\Delta\E=1.34\su$. This discrepancy arises solely from the isotriplet
form factor and is due to a failure of the cranking method for very large 
bag radii.

In figure 1 we also show the chiral symmetry breaking correction to 
neutral pion photoproduction on the neutron. There is no experimental
information available in this channel, but we note that the correction can
be quite large as compared to the LET prediction $E_{0+}(n\pi^0)\cong
-0.5\su$.

In order to discuss the applicability of the nonrelativistic approximation
(\ref{thumb}),
\begin{equation}
\Delta E_{0+}(\pi^0 p) \cong \frac{e}{4\pi f_\pi}
 \frac{\overline{m}}{m_\pi (1+\mu)} 
  ( a_0\, g_A^0 + a_3 \, g_A^3 ) \; ,
\end{equation}
\begin{equation}
  a_0= 1+ \frac{\delta m}{6\overline{m}} \; , \hspace{1cm}
   a_3 =\frac{1}{3} +\frac{\delta m}{2\overline{m}} \; ,
\end{equation} 
 we compare the axial and tensor couplings of the nucleon
in figure 2. In the case of the axial couplings only the quark contributions
are shown. Note however, that the dominant correction is determined by
the singlet coupling and that the meson contribution to $g_A^0$ is  
small \cite{Wei}. From the figure, we conclude that $g_A^{\alpha}({\rm quark})$
is in fact very close to $g_T^{\alpha}$ for all bag radii. Therefore
the naive formula (\ref{thumb}) gives a reliable approximation to the
correct result, but with $g_A^0$  replaced by the much smaller value calculated
within the  chiral bag model. 

The quantity $g_A^0$ 
has recently been the subject of a lively discussion in  
connection with the so called 'proton spin crisis`. Measurements of 
polarized muon-proton scattering by the EMC-collaboration \cite{EMC}
seem to indicate that only a rather small fraction of the proton
spin is carried by the intrinsic quark spins. Assuming good flavor $SU(3)$
symmetry, the EMC analysis \cite{EMC} gives $g_A^0 =0.12 \pm 0.23$.
In the context of the chiral bag model this implies $\Delta\E <0.5\su$.
(In the naive parton model , $g_A^0=\Delta u +\Delta d+\Delta s$  is the sum
of  the  spin fractions carried by up, down and strange quarks. Our
chiral bag calculation would correspond to $\Delta u+\Delta d$ for which
the EMC analysis suggests a value $0.3\pm0.1$. this would imply 
$\Delta \E<0.6 \su$.) 


\subsection{Non topological Soliton Model}
In order to study the model dependence of $\Delta\E$ let us now discuss
the photoproduction sigma term in the non-topological chiral soliton 
model \cite{Bir}. The model lagrangian the one of the linear
sigma model :
\begin{eqnarray}
\label{Lsig}
{\cal L}&=&\bar{\psi}\left[ i\gamma^{\mu}\partial_{\mu}
-g(\sigma+i\gamma_5\vec{\tau}\cdot\vec{\pi})\right]\psi 
+\frac{1}{2}(\partial_{\mu}\sigma)^2
+\frac{1}{2}(\partial_{\mu}\vec{\pi})^2 \nonumber \\
& &\mbox{}-\frac{\lambda^2}{4}\left(\sigma^2+\vec{\pi}^2-\nu^2\right)^2
 + {\cal L}_{SB} \hspace{0.5cm},
\end{eqnarray}  
where $\sigma$ denotes a scalar-isoscalar field and $\vec{\pi}$ is a
pseudoscalar isotriplet pion field belonging to a linear representation of
$SU(2)_L\times SU(2)_R$. The parameters $\lambda$ and $g$ are the meson self
coupling and quark-meson coupling constants, respectively. The pion decay
constant $f_\pi=93$ MeV and the pion mass $m_{\pi}=139.6$ MeV are 
taken at their experimental values and $\nu^2=f_{\pi}^2-m_{\pi}^2 /
\lambda^2$ to select the proper ground state with spontaneously broken 
chiral symmetry. 

In the linear sigma model the chiral symmetry breaking term is usually
introduced as ${\cal L}_{SB}= f_\pi m_\pi^2 \sigma$, so that PCAC is
realized in the form (\ref{PCAC}). Then the commutator $(\ref{Sig})$
vanishes for the same reason as explained for the exterior (meson)
sector of the chiral bag. At a more basic level, ${\cal L}_{SB}=
-\bar{\psi}M\psi$ which leads to PCAC in the form (\ref{Div}). We
shall use this form here and employ the bosonized version of 
the symmetry breaking term only in order to solve the equations
of motion.  

As in the chiral bag model we have used the hedgehog ansatz 
for this purpose. In the context of the linear sigma model
it reads
\stepcounter{equation}
\alpheqn 
\begin{eqnarray}
\label{hed}
\sigma (\vec{r}) &=& \sigma (r)  \;\; , \\
\pi^a (\vec{r}) &=& \hat{r}^a h(r) \;\; ,\\
\psi(\vec{r}) &=& \frac{1}{\sqrt{4\pi}}\left(
\begin{array}{c} G(r)\\ i\vec{\sigma}\cdot\hat{r} F(r) \end{array}
\right)\;|\chi_H> \; ,
\end{eqnarray}
\reseteqn
where $|\chi_H>=\frac{1}{\sqrt{2}}(u\!\downarrow-d\!\uparrow)$ is a 
spinor with quantum numbers $G^{\pi}=0^{+}$. The main difference as compared
to the chiral bag model is that  the fields extend over
all space; they are not separated by boundaries. For this reason
one does not have to rely on semiclassical projection methods but may
use exact Peierls-Yoccoz projection in order to obtain states of good
spin and isospin. For a proton state with spin up, we have
\begin{equation}
\label{pro}
|p\uparrow\,>\, \sim \int [d\Omega]\, {\cal D}_{1/2,-1/2}^{1/2\, *}
(\Omega){\cal R}(\Omega)|h>
\end{equation}
where $\Omega$ is a shorthand notation for the Euler angles, ${\cal R}
(\Omega)$ is a rotation matrix and ${\cal D}^{1/2}(\Omega)$ is the spin 1/2
Wigner D-function. 

In order to obtain the meson part of the hedgehog wavefunction
$|h>$ we interpret the classical meson fields as coherent quantum 
states \cite{Bi2}. The shape of the pion field then enters in the 
expressions for the various couplings only via the expectation value of the 
pion number operator in the coherent meson state \cite{Sch} :
\begin{equation}
\label{Npi}
\overline{{\rm N}}_{\pi} =\, <h_{mes}|\hat{N}_{\pi}|h_{mes}>.
\end{equation}
Once the proton wave function (\ref{pro}) is given it is straightforward to
calculate matrix elements of tensor currents. After some elementary but
tedious angular momentum algebra we get (with $N_c=3$)
\newcommand{\kl}{\scriptstyle}
\begin{equation}
\label{G1L}
g_T^{0} = \frac{N_c}{2}\,\frac{1}{{\cal N}}
\int_0^{\pi}du\,\int_0^1 dt\,t^{N_c +2}\cos^{N_c +1}{\kl (}u{\kl )}\sin^2
{\kl (}u{\kl )}
\,e^{-\frac{4}{3}\overline{{\rm N}}_{\pi}(1-t^2 \cos^2 u)}\,\cdot I
\end{equation}
for the singlet part and 
\begin{equation}
\label{G3L}
g_T^{3} =  \frac{N_c}{3} \left\{ 1\,+\, \frac{1}{{\cal N}}
\int_0^{\pi} du\, \int_0^1 dt\, t^{N_c}(1-t^2 )\cos^{N_c -1}{\kl (}u{\kl )}
\,e^{-\frac{4}{3}\overline{{\rm N}}_{\pi}(1-t^2 \cos^2 u)} \right\}\cdot I
\end{equation}
for the isotriplet form factor. Here, the normalisation is given by
\begin{equation}
\label{Nor}
{\cal N} = \int_0^{\pi} du\, \int_0^1 dt\, t^{N_c +2}\cos^{N_c +1}{\kl(}u{\kl)}
\,e^{-\frac{4}{3}\overline{{\rm N}}_{\pi}(1-t^2 \cos^2 u)}  
\end{equation}
and $I$ denotes a radial integral over the upper and lower components of
the quark wave functions
\begin{equation}
\label{red}
I=\int_0^{\infty} r^2 dr\, ( G^2 +\frac{1}{3} F^2 )\;.
\end{equation}
In the limit of vanishing pion field strength, $\overline{{\rm N}}_{\pi}=0$, 
all the integrals can be performed analytically and we get $g_T^{0}= I$
and $g_T^{3}=5\cdot I/3$ as one would expect for a relativistic
constituent quark model. The final results for more realistic parameter
sets are shown in figure 3 as a function of the quark meson coupling g.
If $\lambda$ is not too small we find that $\Delta\E$ is rather
 insensitive to changes
in the meson self coupling. Here we have taken $\lambda=10$. The
optimal values for the masses of nucleon and delta are then obtained
for $g= 5.38$\ .  
As one can see, this model predicts a correction  $\Delta \E
\cong 0.9\su$ which is more reminiscent of a constituent quark model. One
should note however, that the axial couplings of the nucleon come out
too large as compared to their experimental values: We find 
$g_A^0=0.45$ and $g_A^3=1.64$. This might indicate 
that, if Dirac sea effects are not included, the model tends to 
overemphasize the role of the valence quarks \cite{Wak}.

Of course one might also suspect that the difference between the results
for $\Delta E_{0+}$ in the two models we have discussed is due to the different 
projection methods used. Unfortunately, the conceptually more appealing
Peierls-Yoccoz technique is not easily applicable to the chiral bag model
because of the sharp bag boundary. We have therefore studied the 
semiclassical cranking technique in the context of the non topological 
soliton model. The cranking equation reads \cite{Coh}
\begin{equation}
\label{cra}
(H-\epsilon)\delta\psi=-\frac{\vec{\Omega}\cdot\vec{\tau}}{2}\psi_H .
\end{equation}
Here $H=\vec{\alpha}\cdot\vec{p}+g\beta(\sigma+i\gamma_5\vec{\tau}\cdot
\hat{r}h)$
denotes the Dirac hamiltonian within the  hedgehog ansatz and $\psi_H$ is
an eigenstate with eigenvalue $\epsilon$. Furthermore, $\delta\psi$ is
the first order shift in the eigenstate due to the collective rotation with
angular velocity $\vec{\Omega}$. This equation is of course equivalent to 
the first order perturbation theory result (\ref{wav}), but since we do not
have analytic  expressions for the eigenstates of $H$  we prefer to solve
the differential equation (\ref{cra}) directly. To this end we parametrize
$\delta \psi$ as in \cite{Coh} :
\begin{equation}
\delta\psi(\vec{r}) = \left( 
\begin{array}{c}
 A(r)\vec{\Omega}\cdot\vec{\sigma} \;+\; B(r) (\frac{\vec{\Omega}\cdot
\vec{\sigma}}{3}-(\vec{\Omega}\cdot\hat{r})(\vec{\sigma}\cdot\hat{r})) \\
iC(r)\vec{\Omega}\cdot\hat{r} \;-\; D(r) (\vec{\Omega}\times\hat{r})
\cdot\vec{\sigma}
\end{array}   \right) |\chi_H>\;\;,
\end{equation}
and solve the resulting coupled equations for the scalar functions
$A,B,C$ and $D$. In order to calculate matrix elements of tensor currents 
one has to insert the collective fields $\psi=\psi_H+\delta\psi$ into
the quark bilinears and then quantise in the standard fashion. 
Proceeding this way we finally get
\begin{equation}
 g_T^{0}= \frac{N_c}{2}\frac{1}{\Lambda_{tot}} \int dr\, r^2
 ( AG-\frac{1}{3}(C+2D)F)
\end{equation}
and
\begin{equation}
g_T^{3}=\frac{N_c}{3} \int dr\, r^2 (G^2+\frac{1}{3}F^2)
\end{equation}
where $\Lambda_{tot}$ denotes the total moment of inertia of the 
soliton.

In figure 3 we also show the results from the cranking calculation.
Using this technique the dependence of $\Delta\E$ on $g$ is much stronger
as compared to the previous method. In particular we find that for
the prefered values of $g$ the correction $\Delta\E$ comes out to
be significantly smaller.

\section{Conclusions}
In summary we have studied the model dependence of explicit chiral
symmetry breaking corrections to low energy theorems for neutral pion
photoproduction at threshold. We have found values ranging from 
$\Delta \E=0.22\su$ for a chiral bag with $R=0.5$ fm to $\Delta\E
=1.09\su$ for a non topological soliton with g=4.
These numbers have the correct sign to
reduce the absolute magnitude of the $\E$ amplitude but they are
significantly smaller than  earlier estimates \cite{Nat}.

This can be seen clearly from figure 4 where we compare our 
estimate $E_{0+}^{LET}(\pi^0 p)+\Delta E_{0+}(\pi^0 p)$ with
the recent results from the Mainz group \cite{Bec}. For 
illustration we have taken the results calculated in the chiral bag
model. Even for very large bag radii the corrected low energy 
theorem remains below the data for photon energies larger than 
151.4 MeV.

Our main conclusion is as follows: all models examined in this paper
suggest an approximate proportionality between $\Delta E_{0+}(\pi^0 p)$
and the singlet axial vector coulping constant $g_A^0$, namely
\begin{equation}
\label{con}
\Delta E_{0+}(\pi^0p) \cong \frac{e}{4 \pi f_\pi} \frac{\overline{m}}{m_\pi}
 g_A^0 \; .
\end{equation}
While the calculated values of $g_A^0$ depend on model assumptions
(e.g. the specific quantisation procedure employed), the  relationship
given above is found independent of such details\footnote{In the 
non-topological we find a linear relation $\Delta E_{0+}(\pi^0 p)
= c_0 g_A^0 + \delta $. When the cranking method is applied
to this model the (parameter independent) correction $\delta $ can 
be as large as $0.3 \su$.}. An additional term 
proportional to $g_A \equiv g_A^3$ has only a marginal effect once 
isospin breaking is properly included (without isospin breaking, $g_A^0$
in (\ref{con}) would have to be replaced by $g_A^0+g_A/3$).  

Given the empirical upper limit $g_A^0 < 0.4$, eq. (\ref{con}) implies an upper 
bound
\begin{equation}
\Delta E_{0+}(\pi^0p)< 0.5 \su 
\end{equation}
which is satisfied in the chiral bag model
for bag radii smaller than 0.7 fm. Taking into
account the better overall agreement of the chiral bag predictions
with measured nucleon observables, we consider this to be a more realistic 
estimate than the one obtained from the cranked non-topological soliton
model for which $\Delta E_{0+}(\pi^0 p) < 0.8 \su$. 

The chiral symmetry breaking correction $\Delta E_{0+}(\pi^\pm N)$ for
charged pion photoproduction is of the same order of magnitude 
as $\Delta E_{0+}(\pi^0 p)$ but negligibly small as compared 
to the leading Kroll-Ruderman amplitude (8-a,b). On the other hand, the 
predicted correction $\Delta E_{0+}(\pi^0 n)$ is sizeable in comparison
with the small value (8-d) expected from low energy theorems.

Our results indicate that the difference between $E_{0+}(\pi^0 p)
\cong -2.3\su$ as predicted by low energy theorems and the empirical
values of $E_{0+}(\pi^0p)$ deduced from \cite{Maz,Bec} cannot
be explained in terms of explicit chiral symmetry breaking effects alone. 
There is room for rescattering corrections of the type studied in 
\cite{Noz,Yan}. Recent improved calculations \cite{Ya2} based on a consistent
chiral effective Lagrangian approach for both $\pi N$ scattering 
and photoproduction do in fact suggest a rather large positive
rescattering correction  to $E_{0+}(\pi^0p)$.
\section*{Acknowledgements}
We would like to thank B. Blankleider, D. Drechsel, T.-S. H. Lee,
S. Nozawa and L. Tiator for many helpful
discussions. Part of this paper was prepared while we
enjoyed the hospitality of the Institute for Nuclear Theory in
Seattle.  
\newpage   
\begin{thebibliography}{99}
\bibitem{Maz} E. Mazzucato {\em et al.}, Phys. Rev. Lett. {\bf 57} (1986) 3144
\bibitem{Bec} R. Beck {\em et al.}, Phys. Rev. Lett. {\bf 65} (1990) 1841
\bibitem{BL}  P. Bosted, J. M. Laget, Nucl. Phys. {\bf A296} (1978) 413	      
\bibitem{AG}  S. L. Adler, F. J. Gilman, Phys. Rev. {\bf 152} (1966) 1460
\bibitem{Bae} P. de Baenst, Nucl. Phys. {\bf B24} (1970) 633
\bibitem{VZ}  A. I. Vainshtein, V. I. Zakharov, Nucl. Phys. {\bf B36} 
              (1972) 589
\bibitem{NKF} H. W. L. Nauss, J. H. Koch, J. L. Friar, Phys. Rev. {\bf C41}
              (1990) 2852
\bibitem{Nat} L. M. Nath, S. K. Singh, Phys. Rev. {\bf C39} (1989) 1207 \\
              L. Tiator, D. Drechsel, Nucl. Phys. {\bf A508} (1990) 541c
\bibitem{Noz} S. Nozawa, T.-S. H. Lee, B. Blankleider, Phys. Rev. {\bf C41}
              (1990) 213, {\em ibid.} 1306
\bibitem{Yan} S. N. Yang,
	      Phys. Rev. {\bf C40} (1989) 1810
\bibitem{Lag} J. M. Laget, Phys. Rep. {\bf 69} (1981) 1	      
\bibitem{Kam} A. N. Kamal, Phys. Rev. Lett. {\bf 63} (1989) 2346;
              Preprint, University of Alberta, Alberta Thy-8-90
\bibitem{Ara} M. Araki, Phys. Lett. {\bf B219} (1989) 135	      
\bibitem{Fur} G. Furlan, N. Paver, G. Verzegnassi, Nuo. Cim. {\bf 20A}
              (1974) 295;\\
	      V. de Alfaro, S. Fubini, G. Furlan, C. Rosetti, Currents in
	      Hadron Physics, North Holland Publishing Company (1973)  
\bibitem{FPV} G. Furlan, N. Paver, G. Verzegnassi, Nuo. Cim. {\bf 62A} 
              (1969) 519	      
\bibitem{Rad} G. M. Radutskii, V. A. Sverdutskii, A. N. Tabachenko, Yad. Fiz.
              {\bf 24} (1976) 400; \\ 
	      A. N. Tabachenko, Preprint, Nuclear Physics Institute, 
	      Tomsk (1990) 	      
\bibitem{MMS} J. T. MacMullen, M. D. Scadron, Phys. Rev. {\bf D20} (1979)
              1069; {\em ibid.} 1081  	      
\bibitem{GaL} J. Gasser, H. Leutwyler, Phys. Rep. {\bf 87} (1982) 77	      
\bibitem{Hos} A. Hosaka, H. Toki, W. Weise, Nucl. Phys. {\bf A506} (1990) 501
\bibitem{Bir} M. C. Birse, M. K. Banerjee, Phys. Rev. {\bf D31} (1985) 118
\bibitem{Wei} A. Hosaka, W. Weise, Phys. Lett. {\bf B232} (1989) 442 
\bibitem{Sch} A. Hosaka, T. Sch\"afer, U. Kalmbach, Z. Phys. (1990), in print
\bibitem{Vep} A. D. Jackson, D. E. Kahana, L. Vepstas, H. Verschelde, E.
              W\"ust, Nucl. Phys. {\bf A462} (1987) 661
\bibitem{Gol} L. Vepstas, A. D. Jackson, A. S.	Goldhaber, Phys. Lett.
              {\bf B140} (1984) 280	                          
\bibitem{EMC} EMC Collaboration, J. Ashman {\em et al.}, Nucl. Phys. {\bf B328}
              (1989) 1
\bibitem{Bi2} M. C. Birse, Phys. Rev. {\bf D33} (1986) 1934
\bibitem{Wak} M. Wakamatsu, Phys. Lett. {\bf B234} (1990) 223
\bibitem{Coh} T. D. Cohen, W. Broniowski, Phys. Rev. {\bf D34 } (1986) 3472
\bibitem{Ya2} T.-S. H. Lee, B. C. Pearce, preprint, 
              Argonne National Laboratory (1990) 
	      
\end{thebibliography}
\newpage
\section*{Table Captions}
\begin{itemize}
\item[Tab.\ 1]{Explicit chiral symmetry breaking corrections to
the various photoproduction channels as calculated in the chiral
bag model. The results (in units of $10^{-3}m_\pi^{-1}$) 
are given for various bag radii while the
other model parameters $f_{\pi}=93$ MeV and $g_{\rho\pi\pi}=5.85$
are held fixed}
\item[Tab.\ 2]{$\Delta\Ep$  (in units of $10^{-3}m_\pi^{-1}$) 
as calculated in the non topological soliton
model for various values fo $g$. Both the results from Peierls Yoccoz
projection and from the semiclassical cranking procedure are given.}
\end{itemize}
\section*{Figure Captions}
\begin{itemize}
\item[Fig.\ 1]{Explicit chiral symmetry breaking corrections $\Delta
E_{0+}(\pi^0 p)$ (full line) and $\Delta E_{0+} (\pi^0 n)$ (dashed 
line) calculated in a chiral bag model with vector mesons as a function
of the bag radius $R$. The model parameters are $f_{\pi}=93$ MeV,
$m_{\pi}=139$ MeV and $g_{\rho\pi\pi}=5.85$.}
\item[Fig.\ 2]{Comparison of the quark contributions to the tensor 
and axial vector coupling constants in the chiral bag model as a 
function of the bag radius $R$.}
\item[Fig.\ 3]{ Symmetry breaking corrections $\Delta E_{0+}(\pi^0 p)$
(full lines) and $\Delta E_{0+}(\pi^0n)$ (dashed lines) calculated in a
non-topological chiral soliton model. The two upper curves show the
values obtained from Peierls-Yoccoz projection while the lower ones 
correspond to the cranking calculation.
 The results are given as a function
of the quark-meson coupling $g$. The other model parameters are given
by $\lambda=10$, $f_{\pi}=93$ MeV and $m_{\pi}=139$ MeV.}
\item[Fig.\ 4]{Experimental results obtained by the
Mainz \cite{Bec} group for the real part of the electric dipole
amplitude $\E$ as a function of the incident photon energy.
For photon energies smaller than 151.4 MeV two solutions exist.
 The solid curves show our results 
$\E =\E_{LET}+\Delta\E$ for two different bag radii
$R=0.5$ fm and $R=1.0$ fm. The line labeled R=0 corresponds
 to  the result obtained
from a chiral lagrangian model which reproduces the Low Energy Theorem
at threshold.  }
\end{itemize}   
\newpage
\centerline{\Large Table 1}
\vspace{2cm}
\begin{center}
\begin{tabular}{c||c|c|c}
 R [fm]   & $\DE (p\pi^0)$  & $\DE (n\pi^0)$&$\DE (N\pi^\pm)$\\ \hline\hline
 0.30     & 0.07            &    0.09       &   0.16         \\ \hline
 0.52     &   0.22          &    0.20       &   0.37         \\  \hline
 0.75     &   0.51          &    0.28       &   0.56         \\  \hline
 1.00     &   0.89          &    0.27       &   0.61          \\  \hline
 1.30     &   1.14          &    0.25       &   0.62          \\
\end{tabular}
\end{center}
\par
\vspace{5cm}
\centerline{\Large Table 2}
\vspace{2cm}
\begin{center}
\begin{tabular}{c||c|c|c||c|c|c}
          & \multicolumn{3}{|c||}{exact projection} 
	  & \multicolumn{3}{c}{semiclassical projection}  \\  
   g      &$\DE (p\pi^0)$   &$\DE (n\pi^0)$  &$\DE (N\pi^\pm)$
          &$\DE (p\pi^0)$   &$\DE (n\pi^0)$  &$\DE (N\pi^\pm)$\\ \hline\hline
   4      &   1.09 & 0.84 & 0.99 & 1.03 & 0.86 & 0.66 \\ \hline
   5      &   0.95 & 0.72 & 0.92 & 0.68 & 0.52 & 0.63 \\ \hline
   6      &   0.89 & 0.66 & 0.88 & 0.53 & 0.37 & 0.61 \\ \hline
   7      &   0.85 & 0.63 & 0.85 & 0.43 & 0.28 & 0.59 \\ \hline
   7.8    &   0.82 & 0.61 & 0.83 & 0.39 & 0.24 & 0.58 \\
\end{tabular}   
\end{center}
\end{document} 

