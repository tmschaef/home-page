% revised Jan. 22 1991
\documentstyle[12pt]{article}
\pagestyle{empty} 
\textheight20.5cm
\textwidth16.5cm
\oddsidemargin1cm
\evensidemargin1cm
\hoffset-25pt
\begin{document}

\def\boxit#1{\vbox{\hrule\hbox{\vrule\kern3pt
         \vbox{\kern3pt#1\kern3pt}\kern3pt\vrule}\hrule}}
\setbox1=\vbox{\hsize 33pc  \centerline{ \bf Institut f\"ur Theoretische 
Physik der Universit\"at Regensburg} } 
$$\boxit{\boxit{\box1}}$$
\vskip0.8truecm

\hrule width 16.5 cm  \vskip1pt \hrule width 16.5 cm height1pt
\vskip3pt
January 1991\hfill{TPR-91-3}
\vskip3pt\hrule  width 16.5 cm height1pt  \vskip1pt  \hrule width 16.5 cm 
\vskip 1.5cm 

\renewcommand{\thefootnote}{\fnsymbol{footnote}} 
\centerline{\LARGE A Note on the Photoproduction} 
\vskip8pt
\centerline{\LARGE Sigma Term\footnote{prepared for $\pi N$ newsletter,
ed. G. H\"ohler, W. Kluge and B. M. K. Nefkens}}
\vskip2cm
\centerline{T. Sch\"afer and W. Weise}
\centerline{ Institute of Theoretical Physics, 
         University of Regensburg}
\centerline{  D-8400 Regensburg, W. Germany}
\vskip0.5cm
\vskip3cm  
{\underline{Abstract:}} The discrepancy between Low Energy Theorems and the
recently analyzed data on neutral pion photoproduction at threshold has been 
attributed
to the role of explicit chiral symmetry breaking. In this note we briefly
comment on the current algebra method on which this claim is based
and study the model dependence of the photoproduction sigma term. We point
out a possible correlation between the sigma term and the
 spin content of the nucleon as measured by the 
flavor singlet axial coupling constant. We conclude that chiral symmetry
breaking correction is likely to be smaller than originally estimated. 
 
\setcounter{page}{0}
\setcounter{footnote}{0}
\renewcommand{\thefootnote}{\arabic{footnote}}
\newpage
\pagestyle{plain}
\newcommand{\su}{\cdot 10^{-3}m_\pi^{-1}}
\newcommand{\Ep}{\Delta E_{0+}}
Low Energy Theorems (LET) based on gauge invariance and partial conservation
of the axial current have been widely applied to the description of reactions
involving soft photons and pions. These theorems determine the  relevant 
threshold amplitudes in terms of just a few fundamental parameters like the
electric charge, the axial  coupling  constant  $g_A$ of the nucleon  and the 
pion decay constant $f_\pi$.  While these results are exact in the case of 
purely electromagnetic reactions, an  additional   extrapolation   away  from 
the unphysical  threshold  $q^2=0$ to the physical one at $q^2=m_\pi^2$ 
is necessary in the case of pion production.

As  long  as  one  does   not  take  into  account   rescattering  
corrections, the electric dipole amplitude for photoproduction of 
a pion on the nucleon can be written as a power series 
in the soft  pion  parameter  $\mu=m_\pi/M_{\scriptscriptstyle N}$.  
 The  results  were 
derived in the early seventies \cite{Bae} and are known up to relative order 
${\cal O}(\mu^3)$. In the case of neutral pion photoproduction on 
the proton,  the leading  Kroll  Ruderman  term vanishes  and the 
resulting amplituide is predicted to be
\begin{equation}
   E_{0+}   (\pi^0   p)  =  \frac{e}{4\pi} 
\frac{f}{m_\pi}  \left\{  -\mu  + \frac{\mu^2}{2}(3+\kappa_p  ) + 
{\cal O}(\mu^3) \right\} \cong -2.32 \su . 
\end{equation} 
Here,  $f^2/(4\pi)  = 0.08$  is  the  pseudovector  pion  nucleon 
coupling  constant  and  $\kappa_p=1.793$  the anomalous  magnetic 
moment of the proton.

Due to the difficulties  involved  in a precise determination  of 
the amplitudes close to threshold, accurate data on neutral pion
photoproduction  on the proton have only recently become available 
\cite{Maz,Bec}. Experimental information is still missing for the neutron.   
Also it was recognized  that the electric  dipole  amplitude  is not 
uniquely fixed by the measured angular distributions. Between the 
production threshold at a photon energy $\omega=144.7$ MeV and the 
opening  of the charged  pion channel  at $\omega=151.2$ MeV the 
recent Mainz analysis  \cite{Bec}  has two solutions  (see figure 
1). One of them is very small ($\Ep(\pi^0p)\approx -0.5 \su$) for 
all  measured  energies  while  the  other  one  shows  a  strong 
variation close to threshold.  In any case, none of the solutions 
is consistent  with the LET prediction  (labeled  $R=0$ in figure 
1).  This observation came as a big surprise, in particular since 
Low Energy  Theorems  seem  to work pretty  well  in the case  of 
charged pion production.

\begin{figure}[t]
\vspace{7cm}
\begin{center}
\begin{minipage}{14cm}
{\small \underline{Fig.1}: Experimental results obtained by the Mainz
group for the 
real part of $E_{0+}(\pi^0 p)$. For photon energies smaller than 
151.4 MeV two solutions exist. The solid curves show our results
$E_{0+}=E_{0+}^{LET}+\Delta E_{0+}$ for two different bag radii
$R= 0.5$ fm and $R=1$ fm. The line labeled $R=0$ corresponds to the
LET result.}
\end{minipage}
\end{center}
\end{figure}
 
Following this discovery several authors have considered additional
corrections not included in the standard LET amplitude. 
Here we want to focus on the importance of explicit 
chiral symmetry breaking, a subject already discussed in the May 
1990 issue of the $\pi N$ Newsletter \cite{TD}.  Based on a method 
developed  by Furlan  and collaborators  \cite{FPV},  Tiator  and 
Drechsel suggested that an additional correction analogous to the 
sigma  term in $\pi N$-scattering  might explain  the discrepancy. 
This result  relies  on the use of a constituent  quark  model to 
estimate the magnitude of the photoproduction sigma term.  Such a 
model, however, does not respect chiral symmetry. In this note, we 
want  to  point  out  that  chiral  models  of the  nucleon  give 
estimates for the chiral symmetry breaking correction $\Ep$ which 
are significantly  smaller  than  those  provided  by constituent 
models.

As shown  by Furlan  et al.  \cite{FPV}  the contribution  of the 
sigma term to the photoproduction  T-matrix  can be written as an 
equal time commutator
\begin{equation} \label{Sig} \epsilon^\mu\Sigma_{\mu}^\alpha = 
\frac{i}{f_\pi m_\pi} <N(\vec{p}\,)| [\dot{Q}_5^{\alpha},
\epsilon^\mu  V_{\mu}^{em}(0)]|N(-\vec{p}\,)> \; . 
\end{equation} 
Here, $Q_5^\alpha$  denotes the axial charge  and $V_\mu$  the
electromagnetic  current. The time derivative of the axial charge
 $\dot{Q}^5_\alpha$  is a direct measure of explicit chiral
 symmetry breaking.
 
This result, however, has been critized \cite{Dre,Kam} since the 
current algebra techniques used by Furlan and collaborators are
based on a very specific choice of operators whose matrix elements 
are considered in a restricted set of basis states. This method
is not sufficiently powerful to reproduce the full ${\cal O}
(\mu^3)$ amplitude quoted above. On the other hand, standard 
derivations which lead to this amplitude do not seem to give the
sigma term found by Furlan.

We feel that this discrepancy might be related to the use of an 
unphysical threshold  $q^2=0$ in most of these calculations.
Employing PCAC and the reduction theorem, derivations of the LET
amplitude usually start from  the expression
\begin{equation}
\int d^4x\, e^{iqx} <N(p_2)|T(\partial^\nu A_\nu^\alpha (x)
V_\mu^{em} (0))|N(p_1)> = -\frac{f_\pi m_\pi^2}{q^2-m_\pi^2}
T_\mu^\alpha (q)\; .
\end{equation}
Here, $T_\mu^\alpha$ is the photoproduction T-matrix and $A_\mu^\alpha$
denotes the axial current. Since the right hand side of this equation
is singular
at $q^2=m_\pi^2$ one has to take the limit $q^2 \to 0$. This problem
can be avoided by subtracting the pion pole term from the axial current
\cite{BPP}, $\overline{A}_\mu^\alpha=A_\mu^\alpha-f_\pi \partial_\mu
 \pi^\alpha$. Working with $\overline{A}_\mu^\alpha$ instead of 
 $A_\mu^\alpha$ no singularity occurs in (3). Furthermore, pulling
 the derivative out of the time ordering symbol gives an additional equal
 time commutator which is precisely the sigma term considered
 by Furlan and collaborators.
 

Representing  the currents appearing in the sigma commutator (2)
in terms of quark fields one finds the 
following correction to the electric dipole amplitude in the 
$\pi^0 p$ channel \cite{SW}
\begin{equation}
\label{result}
\Delta E_{0+}(\pi^0 p) = \frac{e}{4\pi f_\pi}\frac{\overline{m}}{m_\pi (1+\mu)}
  \left\{ \left( 1+\frac{\delta m}{6\overline{m}} \right) g_T^0
     + \left(\frac{1}{3}+\frac{\delta m}{2\overline{m}}\right) g_T^3
     \right\} \; 
\end{equation}
where  $\overline{m}=\frac{1}{2}(m_u  + m_d)  \approx  5$ MeV  and 
$\delta  m=m_u-m_d   \approx  -3$ MeV  are  the  average  and  the 
difference  of the light quark masses.  Furthermore,  $g_T^0$ and 
$g_T^3$ denote the flavor singlet and triplet tensor couplings of 
the nucleon defined by
\begin{equation}
<N_f|\bar{\psi}\sigma_{\mu\nu}\tau^\alpha\psi|N_i>=g_T^\alpha(q^2) 
\bar{u}_f\sigma_{\mu\nu}\tau^\alpha u_i\;\; .
\end{equation}
For non-relativistic  quarks,  these  couplings  reduce  to axial 
couplings :  $g_T^0=g_A^0=1$ and $g_T^3=g_A^3=5/3$.   Using these 
numbers, one gets $\Ep (\pi^0 p)=1.8\su$ which would almost cancel
the  LET  result  $E_{0+}(\pi^0  p)=-2.32\su$. However, as already
mentioned in the beginning, naive quark models break chiral symmetry
even in the absence of quark masses. Therefore, using them in order to 
estimate corrections to amplitudes predicted by chiral symmetry appears
to be doubtful. 
  

Furthermore,  since 
$\delta  m/(2\overline{m})\approx  -1/3$,  the  result  is almost 
entirely determined by the singlet coupling. The singlet coupling 
in the axial  channel  has been the focus of intense  discussions 
recently  since  it measures  the 'spin content`  of the nucleon. 
Experimental results obtained by the EMC collaboration \cite{EMC} 
seem to indicate that $g_A^0<0.35$  (in the SU(2) sector) is only 
a fraction of unity.  

In order to study these questions we have evaluated the chiral 
symmetry breaking correction in several chiral models of the 
nucleon \cite{SW}. For illustration, let us consider the chiral
bag model introduced by the Stony Brook group \cite{BR}.
In this model, quarks are confined to a bag with radius $R$. At the
bag surface they couple via a chirally invariant interaction to the
meson fields which exist outside the bag volume and determine the
long range properties of the nucleon. Note that this model is very 
different from the cloudy bag model which starts from a similar 
lagrangian but assumes the pion field to be weak. As far as matrix 
elements of the nucleon are concerned, the cloudy bag model does not differ
very much from the MIT bag model.  

In order to calculate tensor couplings of the nucleon, we have used the
standard cranking method. For the singlet coupling, we get  
\begin{equation}
\label{G0C}
g_T^{0}=\frac{T_{\sigma\tau}}{2\Lambda_{tot}}
\end{equation}
where $\Lambda_{tot}$ denotes the moment of inertia of the soliton and
$T_{\sigma\tau}$ is determined by the particle-hole excitations of the
groundstate. The quantity $T_{\sigma\tau}$ is similar in structure to the quark
part of the moment of inertia. Qualitatively, $g_T^0$ therefore follows the 
fractioning of the total angular momentum between quarks and mesons. 
Furthermore, as long as the bag radius is smaller than $R \approx 1$ fm,
$g_T^0$ is almost identical to $g_A^0$. For $R=0.5$ fm which gives
optimal phenomenology, we find a correction to the electric dipole 
amplitude $\Delta E(\pi^0 p) = 0.22 \su$ which is substantially smaller
as compared to the result from a naive constituent model. 

Within the chiral bag model, we find an approximate relationship between
$\Delta E_{0+}(\pi^0 p)$ and the flavor singlet axial coupling constant
\begin{equation}
 \Delta E_{0+}(\pi^0 p) \cong \frac{e}{4\pi f_\pi} \frac{\overline{m}}{m_\pi}
  g_A^0 \; .
\end{equation}
This equation is still reminiscent of a constituent quark model and it
might have to be modified if other degrees of freedom (e.g. gluonic) are
included. Taking it literally, we can use the upper bound quoted by the
EMC collaboration to get a constraint on $\Delta E_{0+}$ :
\begin{equation}
  \Delta E_{0+}(\pi^0 p) < 0.5 \su \; .
\end{equation}    
This bound is satisfied in the chiral bag model for bag radii $R<0.7$ fm.     

Thus we conclude that the chiral symmetry breaking correction
is not sufficiently large to explain the discrepancy between theory and 
experiment.  The most important additional contributions then seem
to be rescattering corrections of the type studied in \cite{NLB,LP}.  
Recent calculations \cite{LP} based on an effective Lagrangian which
is consistent with low energy theorems for $\pi N$-scattering indicate 
a rather large positive rescattering correction to $E_{0+}(\pi^0 p)$.
The authors of ref. \cite{LP} have also tried to include a chiral
symmetry breaking correction $\Delta E_{0+}(\pi^0 p) \cong 0.5 \su$
of the size estimated here. They find that this additional amplitude 
gives a slight improvement in the $\gamma p \to \pi^0 p$ angular 
distribution at backward angles, but   the  accuracy of the 
experimental data  does not permit quantitative conclusions concerning
this point.
  
\begin{thebibliography}{99}
\bibitem{Maz} E. Mazzucato {\em et al.}, Phys. Rev. Lett. {\underline{57}}
              (1986) 3144
\bibitem{Bec} R. Beck {\em et al.}, Phys. Rev. Lett. {\underline{65}} (1990)
              1841
\bibitem{Bae} P. de Baenst, Nucl. Phys. {\underline{B24}} (1970) 633
\bibitem{TD}  L. Tiator, D. Drechsel, $\pi$N Newsletter 2 (May 1990), 99;
              Nucl. Phys. {\underline{A508}} (1990) 541c
\bibitem{SW}  T. Sch\"afer, W. Weise, Phys. Lett. {\underline{B250}} (1990) 6;
              preprint, University of Regensburg TPR-90-49
\bibitem{FPV} G. Furlan, N. Paver, G. Verzegnassi, Nuo. Cim. {\underline{20A}}
              (1974) 295
\bibitem{Dre} D. Drechsel, Lecture Notes for International Summerschool
              "Structure of Hadrons and Hadronic Matter", Dronten Netherlands
	      (1990), preprint, University of Mainz MKPH-T-90-15
\bibitem{Kam} A. N. Kamal, preprint, University of Alberta, Alberta Thy-8-90 
\bibitem{BPP} L. S. Brown, W. J. Pardee, R. D. Peccei, Phys. Rev. 
              {\underline{D4}} (1971) 2801      
\bibitem{EMC} EMC Collaboration, J. Ashman {\em et al.}, Nucl. Phys.
              {\underline{B328}}
\bibitem{BR}  G. E. Brown, M. Rho, Comments Nucl. Part. Phys. 
              \underline{18} (1988) 1
\bibitem{NLB} S. Nozawa, T.-S. H. Lee, B. Blankleider, Phys. Rev. 
              \underline{C41} 213 	      	      
\bibitem{LP}  T.-S. H. Lee, B. C. Pearce, preprint, 
              Argonne National Laboratory (1990) 
	      
\end{thebibliography}
\end{document}n