\documentstyle[12pt]{article}
\textheight20.5cm
\textwidth16.5cm
\oddsidemargin1cm
\evensidemargin1cm
\hoffset-20pt
\begin{document}

\def\boxit#1{\vbox{\hrule\hbox{\vrule\kern3pt
         \vbox{\kern3pt#1\kern3pt}\kern3pt\vrule}\hrule}}
\setbox1=\vbox{\hsize 33pc  \centerline{ \bf Institut f\"ur Theoretische 
Physik der Universit\"at Regensburg} } 
$$\boxit{\boxit{\box1}}$$
\vskip0.8truecm

\hrule width 16.5 cm  \vskip1pt \hrule width 16.5 cm height1pt
\vskip3pt
October 1989\hfill{TPR-89-36}
\vskip3pt\hrule  width 16.5 cm height1pt  \vskip1pt  \hrule width 16.5 cm 
\vskip 1.5cm 

\renewcommand{\thefootnote}{\fnsymbol{footnote}} 
\centerline{\LARGE Spin Content of the Nucleon in a} 
\vskip2pt
\centerline{\LARGE non-topological Chiral Soliton Model\footnote{Work 
supported in part by DFG grant We 665/9-3}}
\vskip2cm
\centerline{A. Hosaka\footnote{Present address:  Department of Physics,
University of Pennsylvania, Philadelphia, PA19104, USA}
  and T. Sch\"afer}
\centerline{ Institute of Theoretical Physics, 
         University of Regensburg}
\centerline{  D-8400 Regensburg, W. Germany}
\vskip0.5cm
\centerline{and}
\vskip0.5cm
\centerline{U. Kalmbach}
\centerline{Institute   of  Theoretical  Physics,  University  of 
Gie\ss en}
\centerline{ D-6300 Gie\ss en, W. Germany}   
\vskip3cm  
{\underline{Abstract:}}\  The spin content of the proton is investigated
by studying the flavor singlet axial structure of the nucleon in a
non-topological chiral soliton model. In 
order  to  construct  a  nucleon  state  we  used  the  generator 
coordinate  projection method as well as a coherent state for the 
meson wave function. Using a standard set of parameters we found the   
value $g_A^0 \simeq 0.44$ for the flavor singlet axial vector coupling
constant.   This result is not far from that of a typical       
valence quark model.           

\newpage
\renewcommand{\thefootnote}{\arabic{footnote}}
\setcounter{footnote}{0}
Spin observables provide us with a good chance to test different         
models of   the nucleon.  Particular interest has been paid to the
flavor singlet axial vector coupling constant $g_A^0$ of the nucleon.  It
is defined by
\begin{equation}
\int d^3x
\langle N \vert A _\mu \vert N \rangle =
\int d^3x
\langle N \vert \bar \psi {1 \over 2} \gamma _\mu \gamma _5 \psi
                                      \vert N \rangle
\equiv {1 \over 2} \langle N \vert \sigma _\mu \vert N \rangle \,
                                       g_A^0,
\end{equation}
where the flavor singlet axial vector current $A_\mu$ is given in terms
of the quark  field  $\psi$,  and $\sigma  _\mu$  is the standard spin
operator for the nucleon.   The matrix elements  are
taken  in  the  nucleon   state  $\vert  N  \rangle$.    Ignoring
effects    caused by anomalous gluon contributions the  parameter
$g_A^0$ can be  interpreted as the fraction of the total
nucleon spin carried by the quarks.


Recent measurements of polarized muon-proton scattering by the European
Muon Collaboration (EMC) imply that \cite{EMC}
\begin{equation}
g_A^0 = 0.00 \pm 0.24, \, \, \, (\mu ^2 = 11 \hbox{GeV})
\end{equation}
where $\mu$ is the scale at which the operator in (1) is renormalized.
Apparently
the result (2) contradicts naive valence quark model predictions,
where $g_A^0$ comes out to be essentially one:  the nucleon structure appears
to be more complex than expected in a naive quark model.
On the other hand, the Skyrme model predicts $g_A^0=0$ 
\cite{BEK},  which
seems   to be consistent with (2).  The importance of the anomalous
$U_A(1)$ breaking due to the coupling of gluons to the
axial vector current $A_\mu$ has also been pointed out \cite{AR}. 


The purpose of this paper is to study $g_A^0$ in a non-topological
chiral soliton model \cite{BB}, which has been successfully
applied to the description of nucleon structure.
An advantage
of this model is that quantum effects  in the soliton  sector can,
at least to some extent, be taken into account in the framework of the
variational  method,  in which the generator  coordinate  projection
method \cite{BG,PY}  is applied to construct  a physical  nucleon
state  instead  of  the  semiclassical  cranking  method  usually
applied to the Skyrme model \cite{ANW}.  We calculate $g_A^0$ for
several different model parameters and discuss the properties  of
$g_A^0$.   We compare  our results  with those of the chiral  bag
model  \cite{PH}  and  point  out  the  importance  of vacuum
polarization effects on the quark field.  The valence
quark model and the Skyrme model limits are also investigated.

The model is based  on the lagrangian  of the linear  sigma model
with quark degrees of freedom :
\begin{eqnarray}  {\cal L} & = & \bar  \psi [i
\partial  _\mu \gamma _\mu + g ( \sigma  + i \vec \tau \cdot \vec
\pi \gamma  _5 ) ] \psi \nonumber  \\
 & & + {1 \over 2} (\partial
_\mu  \sigma)^2  + {1 \over  2} (\partial  _\mu  \vec  \pi )^2  -
{\lambda^2  \over  4} 
 ( \sigma  ^2 + \vec  \pi  ^2  - \nu  ^2 )^2
-f_\pi m_\pi^2\sigma,
\end{eqnarray}
where  $f_\pi  , g$ and $\lambda$,  are  the pion
decay  constant,  quark  meson coupling  constant  and four point
meson coupling constant, respectively. Using the value 
$ \nu^2 = f_\pi^2 -m_\pi^2 / \lambda^2$ 
the classical minimum of the quartic
potential is fixed at $(\sigma,\vec{\pi})=(-f_\pi,\vec{0})$. The
last  term in (3) is introduced to explicitly break chiral
symmetry and gives the pion its observed mass, $m_\pi=139$ MeV.
We restrict our discussion to the $SU(2)$ sector. The effects
of strange quarks and $SU(3)$ symmetry breaking  are expected  to
be small \cite{SC}. 

Static properties  of the nucleon derived from (3) have  been studied     
extensively by a number of authors \cite{BB,BG}.
Here we are interested in the flavor
singlet axial properties (1).
The flavor singlet axial vector current $A_\mu$ for the system (3) is
\begin{equation}
A_\mu = {1 \over 2} \bar \psi \gamma _\mu \gamma _5 \psi .
\end{equation}
We note that  $A_\mu$ is not conserved, $\partial _\mu A^\mu \neq 0$.
Axial vector current conservation can be recovered by
introducing the flavor singlet
pseudoscalar meson $\eta ^\prime$ which couples
to the quark field $\psi$ in a chirally invariant way.  This is done in a
  way analogous to the introduction of the pion field in the          
isovector sector.  The total axial vector current then 
consists of a quark
term  (4) and an $\eta ^\prime$ piece.  Since the $\eta ^\prime$
has a non-zero mass $m_{\eta^\prime} \neq 0$,
the total axial vector current is not conserved.
This non-conservation is now related to the $U_A(1)$
anomaly through the QCD anomaly equation \cite{ChL}
\begin{equation}
\partial _\mu A^\mu = {{\alpha_s N_f} \over {2 \pi}}
               \hbox{tr} F_{\mu \nu} \tilde F^{\mu \nu}
\end{equation}
and the current algebra relation
\begin{equation}
\partial _\mu A^\mu = f_{\eta ^\prime } m^2_{\eta ^\prime } \eta ,
\end{equation}
where $\alpha_s$, $N_f$ and $f_{\eta^\prime}$ are the quark-gluon coupling
constant, number of flavors and the decay constant for the
$\eta ^\prime$.  Furthermore, $F_{\mu \nu}$ is the gluon field tensor
and $\tilde F_{\mu \nu}$ its dual tensor.

In order to incorporate these aspects, the $\eta ^\prime$ field must be
properly introduced in the lagrangian (3).
However, as shown in the chiral bag calculation \cite{PH},
the $\eta ^\prime$ contribution to $g_A^0$ is small as compared to the quark
piece if the large physical  $\eta ^\prime$ mass
 $m_{\eta^\prime} = 958$ MeV is used.
Furthermore, the $\eta ^\prime$-quark coupling is not  strong enough to
modify the quark wave function significantly.  We expect a   similar
situation in the chiral soliton model and assume that the major
contribution  to $g_A^0$ comes from the explicit  quark piece (4)
whereas the $\eta ^\prime$-quark coupling is negligible.

The system (3) allows a classical soliton solution within the hedgehog
ansatz:
\begin{eqnarray}
\psi _H & = & \left( \begin{array}{c}
                 u(r) , \\
                 i\vec \sigma \cdot \hat r v(r)
                 \end{array} \right)
\chi , \; \; \; \chi = {1 \over {\sqrt 2}}
            (\vert u\downarrow \rangle - \vert d\uparrow \rangle ) ,
                                        \nonumber  \\
\phi _0  & = & \sigma (r) ,  \\
\phi _i  & = & \hat r_i h(r) , \nonumber
\end{eqnarray}
with spherically symmetric functions $u(r), v(r), \sigma (r)$ and $h(r)$.
The valence quark wave function is normalized as $4\pi \int_0^\infty
r^2dr (u^2 + v^2)  =  1$, and the spin-isospin spinor $\chi$ has  good
quantum  numbers $K^P = 0^+$, where $K = L + S + I$ with $L$, $S$
and $I$ being  an orbital  angular  momentum,  spin  and isospin,
respectively,  and $P$ is the parity.   $u \, (d)$  and $\uparrow
(\downarrow  )$ denote the isospin  up (down)  and spin up (down)
states, respectively.

A possible quantum interpretation of the classical meson configuration
in (7) is that they are expectation values of the corresponding field
operators in the hedgehog state $\vert h \rangle$:
\begin{eqnarray}
\langle h \vert \phi _0 (x) \vert h \rangle & = & \sigma (r) ,
                                            \nonumber \\
\langle h \vert \phi _i (x) \vert h \rangle & = & \hat r_i h(r) .
\end{eqnarray}
A state which has the property (8) can be constructed using a coherent
state. Introducing the canonical momentum operators $\pi _\alpha $
conjugate to $\phi _\alpha$,
the coherent state for the meson sector $\vert h_{mes}\rangle $ is
written as \cite{BG}
\begin{equation}
\vert h_{mes} \rangle = \exp \{ i \int d^3x (\pi_0 (x) \sigma (r)
                      + \pi _i (x) \hat r_i h(r) ) \}
                      \vert 0 \rangle .
\end{equation}
It is convenient to expand the operators $\pi _\alpha$ in momentum space:
\begin{equation}
      \pi _\alpha (x) =  i \int {{d^3k} \over {(2\pi)^{3/2}}}
\sqrt{{\omega _k} \over 2} ( - e^{i\vec k \cdot \vec x} a_{k,\alpha}
                    + e^{-i\vec k \cdot \vec x} a_{k,\alpha}^\dagger ) ,
\end{equation}
where $\omega _k = \sqrt{k^2 + m_\pi^2}$.  The vacuum state $\vert 0 \rangle$
in (3) is then defined  as the state which is annihilated  by all
the  opertators  $a_{k,\alpha}$ :   $\; a_{k,\alpha}  \vert  0
\rangle = 0$.

The total hedgehog state
$\vert h \rangle$ is a direct product of the meson and quark wave
functions, $\vert h_{mes}\rangle$ and $\vert h_q \rangle$,
\begin{eqnarray}
\vert h \rangle & = & \vert h_q \rangle \vert h_{mes} \rangle , \\
                &   & \vert h _q \rangle = (b_{0^+}^\dagger)^{N_C}
                                   \vert 0 \rangle , \nonumber
\end{eqnarray}
where the operator $b_{0^+}^\dagger$
creates the quark state $\psi _H$ in (7) and $N_C$ denotes the
number of colors.  The c-number functions $u(r)$, $v(r)$,
$\sigma(r)$  and $h(r)$  are regarded  as variational  parameters
to be determined by  energy  minimization:
$\; \delta  \langle  h \vert  H \vert h \rangle = 0$.

Because of its $K$-symmetry, the hedgehog state is written as
a superpositon
of states with various spins $J$ and isospins $I$. A proper nucleon state
with definite spin and isospin is obtained by applying the projection
method.  For example, a proton spin-down state can be written as
\begin{equation}
\vert p \downarrow \rangle \sim \int d[ \Omega ] \,
        D^{1/2 *}_{1/2,1/2}(\Omega) \, R(\Omega) \vert h \rangle ,
\end{equation}
where $R(\Omega)$ is the $SU(2)$ isospin rotational operator with the
angle $\Omega$ and $D^J_{m,t}(\Omega)$ is a representation of
$R(\Omega)$.
Once we have the proton wave function (12), it is straightforward to
calculate the matrix element (4).  Using some elementary angular momentum
algebra, we  find 
\begin{eqnarray}
{ {\langle p \downarrow \vert A_3 \vert p \downarrow \rangle }
  \over
  {\langle p \downarrow \vert  p \downarrow \rangle } }
& = & - {N_C \over 2}
{ {\int d [ \Omega ] \, c_1^{N_C+1} \, c_2^{N_C-1} \, s^2
\langle h_{mes} \vert R(\Omega)  \vert h_{mes} \rangle  }
 \over
{\int d [ \Omega ] \, c_1^{N_C+1}  \, c_2^{N_C+1} \langle h_{mes}
\vert R(\Omega)  \vert h_{mes}  \rangle } }
 \, \, R, \\ & & c_1 =
\cos({\beta  \over 2}), \; \; \; c_2 = \cos({{\alpha  + \gamma  }
\over 2}), \nonumber  \\ & & s = \sin({{\alpha  + \gamma  } \over
2}),
 \nonumber
\end{eqnarray}
where  the  angle  $\Omega$  is
represented  by the Euler angles $\alpha$,  $\beta$ and $\gamma$.
Furthermore,  $R$ denotes the radial integral including the quark
wave functions
\begin{equation}
R = 4\pi  \int _0
^\infty r^2 dr \, (u^2 - {1 \over 3} v^2)
\end{equation}
whereas the meson overlap function is given by
\begin{equation}
\langle h_{mes} \vert
R(\Omega)  \vert h_{mes} \rangle = \exp\{ -a (1 - c_1^2 c_2^2) \}
\end{equation}
with
\begin{eqnarray}
a & = & - {{16}  \over  3}
f_\pi ^2 \int _0^\infty k^2dk\, \omega _k \, \tilde h^2(k) , \\ &
& \tilde  h (k)  = \int  _0^\infty  r^2drj_1(kr)h(r).   \nonumber
\end{eqnarray}

Dividing (11) by the factor $\langle p \downarrow \vert
{\sigma _3 \over 2} \vert p \downarrow \rangle = - 1/2$, one obtains
$g_A^0$.  We have calculated $g_A^0$ for several different
model parameters.   In
Fig. 1 we show $g_A^0$ as a function of the quark-meson coupling constant
$g$ with the hedgehog mass $M_H$ fixed at 1088 MeV.  To keep
$M_H$ = {\em const}, the pion decay constant $f_\pi$
is varied as shown on top
of the figure.  The four point meson coupling constant $\lambda$ is
fixed
at $\lambda = 10$.  We found that $g_A^0$ depends rather weakly on
$g$.  At the physical value of
$f_\pi = 93$ MeV where $g \simeq 5.6$ ,  $g_A^0$ comes out to be 0.44
which is somewhat larger than the upper limit of the EMC result (2).

Now we study the properties of $g_A^0$ in more detail.  To do this, it is
convenient to plot $g_A^0$ as a function of the baryon charge radius      
carried by the quarks\footnote{In the non topological chiral soliton model
the baryon charge is carried entirely by the quraks, whereas in the chiral
bag it is shared by the quarks inside the bag and the mesons fields outside}
\begin{equation}
\langle r^2 \rangle ^{1/2}  =
( \int _0^\infty d^3x \, r^2 \langle \psi ^\dagger \psi \rangle )^{1/2}.
\end{equation}
We use this parameter rather than the total baryon charge radius
since in the chiral bag model the isosinglet axial vector coupling
is essentially due the quark part of the  axial vector current \cite{PH}.
Comparing the curves calculated in the two models as given in figure 2 it
is interesting to see an apparent correlation between them 
for $\langle r^2 \rangle^{1/2}$ larger than 0.6 fm.  For small
$\langle r^2 \rangle ^{1/2}$,  however,
they tend to deviate from each other.  We argue that this is because the
vacuum polarization effect \cite{KS}
 on the quark wave function is totally
ignored in a standard treatment of the chiral soliton model.
In fact, the point $\langle r^2 \rangle ^{1/2} \simeq 0.3$ fm corresponds
to a bag radius $R \simeq 0.5$ fm in the chiral bag model, where about
half of the baryon number in the quark bag
is carried by the vacuum. Similarly it is expected that
the vacuum contribution to $g_A^0$ can
no longer be neglected and $g_A^0$ would be reduced \cite{W}.

At the physically realistic point where $f_\pi$ = 93 MeV, the radius of the 
quark distribution in the chiral soliton model
is $\langle r^2 \rangle ^{1/2} \simeq 0.7$ fm and the
corresponding chiral bag size is $R \simeq 0.8$ fm. In this case
physical quantities in the quark bag are dominated by the valence
quarks. Therefore, for any realistic parameter set, the non-toplogical
chiral soliton
model behaves like a valence quark model rather than like the Skyrme model.

Finally, we would like to briefly discuss how the valence
quark (MIT bag) model and
the Skyrme model limits are reached  in the 
non-topological chiral soliton model. The
MIT bag model limit is obtained by simply setting $a=0$ in (13).
Physically this implies the disappearance  of the pion  cloud  around
the  nucleon, since
$a$ is related to the expectation value of the pion number by \cite{BG}
\begin{equation}
  a=-\frac{4}{3} \langle h_{mes} \vert N_\pi \vert h_{mes}
         \rangle
\end{equation}
In this case, the integral  (13) can be
performed  analytically.   We find  $g_A^0  = R$, independent  of
$N_C$, as expected. The Skyrme model limit, on the other hand, is
slightly  subtle.   This  is indicated   by the fact that a naive
limit      $N_C \to \infty$ does not lead to the Skyrmion  result
$g_A^0 = 0$, contrary to a widespread  expectation  \cite{BEK,M}.
In  this  naive  limit       
the coefficient $f_\pi$ which enters in $a$ (16) is also scaled\footnote{If
$f_\pi$ is kept fixed while $N_C$ goes to infinity, (13) yields the 
quark model result $g^0_A = R$}
according to $f_\pi \sim \sqrt{N_C}$ whereas the soliton profile
is considered to be independent of $N_C$. The Skyrmion result
 $g_A^0 = 0$ is obtained
either by letting $a \to \infty$  faster than $N_C \to \infty$ or
by letting $a \to \infty$ with $N_C$ fixed.  In any case the pion
number  must become  infinitely large  for the chiral  
soliton  model  to
approach  the  Skyrme  model.   The infinite  pion  number  is
in fact an implicit  assumption  when the semiclassical  cranking
method is applied in the Skyrme model.

In summary we have studied the flavor singlet axial vector coupling
constant $g_A^0$ in a non-topological chiral soliton model.  The model
predicts $g_A^0 \simeq 0.44$ for its standard parameter set, a result 
reminiscent of a valence quark model.  We have also investigated the
valence quark model and the Skyrme model limits in
the chiral soliton model, where the pion number plays a crucial role in
achieving  these limits.

\section*{Acknowledgement}
We thank W. Weise for useful discussions and comments. A. H. is grateful to
R. Amado and M. Oka for discussions and a careful reading of the manuscript.
 

\begin{thebibliography}{99}
\bibitem{EMC}{EMC collaboration, J.\ Ashman {\em et al}, Phys.\ Lett.\
              {\bf B206} (1988) 364.}
\bibitem{BEK} S. Brodsky, J.\ Ellis and M.\ Karliner,
              Phys.\ Lett. \ {\bf B206} (1988) 309.
\bibitem{AR} G.\ Altarelli and G.\ G.\ Ross,
              Phys. Lett. {\bf B212} (1988) 391; \newline
             R.\ D.\ Carlitz, J.\ C.\ Collins and A.\ H. Mueller,
             Phys.\ Lett. {\bf B214} (1988) 229. \\
             B.-Y. Park, V. Vento, Preprint Univ. of Valencia, FTUV/89-23
\bibitem{BB}  M.\ C.\ Birse and M.\ K.\ Banerjee,  Phys.\  Lett.\ 
{\bf B136} (1984) 284; Phys.\ Rev.\ {\bf D31} (1985)  118; \\ S.\ 
Kahana,  G.\ Ripka and V.\ Soni, Nucl.\  Phys.  {\bf A415} (1984) 
351.  
\bibitem{BG}  M.\ C.\ Birse, Phys.\  Rev.  {\bf D33} (1986) 
1934; \newline  K.\ Goeke, M.\ Harvey, F.\ Gr\"ummer  and J.\ N.\ 
Urbano, Phys.\ Rev.  {\bf D37} (1988) 754.  
\bibitem{PY}  R.\ E.\ Peierls  and J.\ Yoccoz,  Proc.\  Roy.\  Soc.\  
London  {\bf A70} (1957) 381.  
\bibitem{ANW}  G.\ S.\ Adkins, C.\ R.\ Nappi and E.\ Witten, Nucl.\ 
Phys. {\bf B228} (1983) 552;\\ I.\ Zahed and G.\ E.\ 
Brown, Phys.\ Rep.\ {\bf 142} (1986) 1. 
\bibitem{PH} B.-Y.\ Park, V.\ Vento, M.\ Rho and G.\ E.\ Brown,  
Preprint  SUNY Stony Brook (1989); 
\newline  A.\ Hosaka and W.  Weise, Phys. Lett. B (1989), in print
\bibitem{SC}  J.\  Stern  and  G.\  Clement, 
Preprint  Saclay,  SPhT/89/107  (1989).  
\bibitem{ChL}  See,  for example,  T.-P.\ Cheng and L.-F.\ Li, "Gauge theory of elementary 
particle physics", Oxford (1982).  
\bibitem{KS} U.\ Kalmbach, T.\ 
Sch\"afer, T.\ S.\ Biro and U.\ Mosel, Preprint Univ. of Giessen, 
UGI 89-10 (1989). 

\bibitem{W}   M.\  Wakamatsu,   Preprint   Osaka  Univ.   (1989).
\bibitem{M} A.\ V.\ Manohar, Nucl. Phys. {\bf B248} (1984) 19.

\end{thebibliography}

\newpage

\section*{Figure Captions}

\begin{itemize}
\item[Fig.\ 1] Flavor singlet axial vector coupling constant $g_A^0$ as
a function of the quark-meson coupling constant $g$.  The hedgehog mass
is fixed at $M_H = 1088$ MeV.  The pion decay constant is shown on the
upper scale of the figure.
\item[Fig.\ 2] Flavor singlet axial vector coupling constant $g_A^0$ as a
function of the quark piece of the baryon charge distribution 
$\langle r^2 \rangle ^{1/2}$.  The
chiral bag values are given by the dashed line.
\end{itemize}

\end{document}

