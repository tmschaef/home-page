%\documentstyle[12pt,german,twoside]{report}
\documentstyle[12pt,twoside]{report}


%

\def\thefootnote{\arabic{footnote}}

%

\newcommand{\pr}[3]{Phys.~Rev.~{\bf #1} (19#2) #3}

\newcommand{\prc}[3]{Phys.~Rev.~{\bf C#1} (19#2) #3}

\newcommand{\prd}[3]{Phys.~Rev.~{\bf D#1} (19#2) #3}

\newcommand{\prp}[3]{Phys.~Rep.~{\bf #1} (19#2) #3}

\newcommand{\ij}[3]{Int.~J. Mod.~Phys.~{\bf A #1}(19#2)#3}

\newcommand{\aj}[3]{Aust.~J. Phys.~{\bf #1} (19#2) #3}

\newcommand{\plb}[3]{Phys.~Lett.~{\bf{B#1}} (19#2) #3}

\newcommand{\prl}[3]{Phys.~Rev.~Lett.~{\bf #1} (19#2) #3}

\newcommand{\jg}[3]{J. Phys.~{\bf G #1} (19#2) #3}

\newcommand{\npb}[3]{Nucl.~Phys.~{\bf B#1} (19#2) #3}

\newcommand{\npa}[3]{Nucl.~Phys.~{\bf A#1} (19#2) #3}

\newcommand{\an}[3]{Ann.~Phys.~{\bf #1} (19#2) #3}

\newcommand{\ptp}[3]{Prog.~Theor.~Phys.~{\bf #1} (19#2) #3}

\newcommand{\zpc}[3]{Z. Phys.~{\bf C#1} (19#2) #3}

\newcommand{\zpa}[3]{Z. Phys.~{\bf A#1} (19#2) #3}

\newcommand{\fp}[3]{Fortschr.~Phys.~{\bf #1} (19#2) #3}

\newcommand{\nc}[3]{Nuovo Cimento{\bf #1} (19#2) #3}  

%

\newcommand{\pbar}{\bar{\psi}}

\newcommand{\chs}{SU(3)_L\times SU(3)_R}

\newcommand{\spm}{\hspace{0.3cm}}

\newcommand{\beq}{\begin{eqnarray}}

\newcommand{\eeq}{\end{eqnarray}}

\newcommand{\be}{\begin{equation}}

\newcommand{\ee}{\end{equation}}

\newcommand{\kl}{\scriptstyle}

\newcommand{\mini}{\scriptscriptstyle}

\newcommand{\aflm}{A_{\mu \,{\mini L}}^{a}}

\newcommand{\afln}{A_{\nu \,{\mini L}}^{a}} 

\newcommand{\qfl}{Q_{5{\mini L}}^{a}}

\newcommand{\qfr}{Q_{5{\mini R}}^{a}}

%

\newcommand{\gev}{\rm GeV}

\newcommand{\mev}{\rm MeV} 

%

\newcommand{\su}{\cdot 10^{-3} m_\pi^{-1}}

\newcommand{\Rop}{\gamma p \to \pi^0 p}

\newcommand{\Ron}{\gamma n \to \pi^0 n}

\newcommand{\Rmp}{\gamma n \to \pi^- p}

\newcommand{\Rpn}{\gamma p \to \pi^+ n}

%

\newcommand{\Eop}{E_{0+}(\pi^0 p)}

\newcommand{\Eon}{E_{0+}(\pi^0 n)}

\newcommand{\Epn}{E_{0+}(\pi^+ n)}

\newcommand{\Emp}{E_{0+}(\pi^- p)}

%

\newcommand{\DEop}{\Delta E_{0+}(\pi^0 p)}

\newcommand{\DEon}{\Delta E_{0+}(\pi^0 n)}

\newcommand{\DEcn}{\Delta E_{0+}(\pi^\pm N)}

\newcommand{\cl}{\centerline}

\newcommand{\cld}{\pagestyle{empty}\cleardoublepage\pagestyle{headings}}

%

%  arabic extensions for equation numbers

%  use as : \stepcounter{equation}

%           \alpheqn

%           ----  eqnarray ----------   

%           \reseteqn

%

\newcounter{saveeqn}

\newcommand{\alpheqn}{\setcounter{saveeqn}{\value{equation}}%

\setcounter{equation}{0}%

\renewcommand{\theequation}{\mbox{\arabic{chapter}.%

\arabic{saveeqn}-\alph{equation}}}}

\newcommand{\reseteqn}{\setcounter{equation}{\value{saveeqn}}%

\renewcommand{\theequation}{\arabic{chapter}.\arabic{equation}}}

%

% ************* new caption *************                             

%                            

%\def\figurename{Abb.~}                 %Kuerzel vor Abb. nr              

%\def\tablename{Tab.~}

\makeatletter                                                           

\long\def\@makecaption#1#2{                                            

 \vskip 10pt                                                            

 \setbox\@tempboxa\hbox{\sc#1 : \small#2 }%                               

                                      %\bf, bzw. \sm fuer Nr. bzw. Text

 \leftskip=0mm\rightskip=0mm          %Linker, rechter Rand der Caption

 \ifdim \wd\@tempboxa >\hsize                                           

%

% ************* separate label ***********

%  

% \setbox3=\hbox{\sc#1: }              %Vorstehendes Label               

% \hangafter=1\hangindent=\wd3         %Nach 1 Zeile um soviel einruecken

% \unhbox\@tempboxa

\sc#1 : \small#2\par \else \hbox                                      

 to\hsize{\hfil\box\@tempboxa\hfil}                                    

 \fi}                                                                   

\makeatother                                                            

%                                                                       

%

\textwidth16.0cm

\textheight22.0cm

\voffset-1.5cm

\hoffset-1.5cm

%

%                Aufi gehts 

%

\begin{document}

\baselineskip18pt

\parindent0pt

\evensidemargin1.00cm

\oddsidemargin1.85cm

\begin{titlepage}

\centerline{\huge \bf Stromalgebra und}

\vspace{0.8cm}

\centerline{\huge \bf Photoproduktion von Mesonen}

\vspace{0.8cm}

\centerline{\huge \bf an der Schwelle}  

\vspace{2.5cm}

\centerline{\bf Dissertation}

\centerline{\bf zur Erlangung des akademischen Grades}

\centerline{\bf eines Doktors der Naturwissenschaften}

\centerline{\bf (Dr. rer. nat.)}

\vspace{2.0cm}

\centerline{\bf vorgelegt an}

\centerline{\bf der Naturwissenschaftlichen Fakult\"at II - Physik}

\centerline{\bf der Universit\"at Regensburg}

\vspace{2.5cm}

\centerline{\bf von}

\centerline{\bf Thomas Sch\"afer }

\centerline{\bf aus}

\centerline{\bf Hanau} 

\vspace{1cm}

\centerline{\bf Regensburg}

\centerline{\bf 1992}

\vspace{0.5cm}

\end{titlepage}

\pagestyle{empty}

\vspace*{13cm}
Die Arbeit wurde angeleitet von Prof.~Dr.~W.~Weise.
\begin{tabbing}
Promotionsgesuch eingereicht   \= : \= 7.~Januar 1992\\
Tag der m\"undlichen Pr\"ufung \> : \> 18.~Februar 1992
\end{tabbing} 
\vspace*{2cm}
\begin{tabbing}
 \hspace*{3cm} \= \hspace{5cm} \= muster  \kill
    \> Pr\"ufungsausschu\ss :   \>   \\
    \> {\sc Prof.~Dr.~H.~Hoffman}              \> Vorsitzender  \\
    \> {\sc Prof.~Dr.~W.~Weise}         \> 1. Gutachter  \\
    \> {\sc Prof.~Dr.~J.~R\"o\ss ler}   \> 2. Gutachter  \\
    \> {\sc Prof.~Dr.~U.~Krey}              \> weiterer Pr\"ufer 
\end{tabbing}


\newpage

\pagenumbering{roman}

\tableofcontents

\newpage

\pagenumbering{arabic}

\setcounter{footnote}{0}

\pagestyle{empty}

%\vspace*{14cm}
\hspace*{4cm}
\begin{minipage}[b]{12cm}
{\em And now reader, bestir thyself - for though we will always
lend thee proper assistance in difficult places, as we do not,
like some others, expect thee to use the arts of divination to
discover our meaning, yet we shall not indulge thy laziness
where nothing but thy own attention is required; for thou art
highly mistaken if thou dost imagine that we intended when we
begin this great to leave thy sagacity nothing to do. \\
\hspace*{6cm} Henry Fielding}
\end{minipage}


\pagestyle{plain}

\addcontentsline{toc}{chapter}{Einleitung}
% revised Jan. 1, 1992  
\vspace*{4cm}
{\Huge \bf Einleitung} 
\\[3.5cm]
Das Pion nimmt als leichtestes aller stark wechselwirkenden Teilchen
eine zentrale Stellung in der Kern- und Elementarteilchenphysik ein.
Ein besonders geeigneter Proze\ss\ zum Studium der Piondynamik ist die
Photoproduktion, das hei\ss t die Erzeugung von Pionen mittels
energetischer Photonen.

Den angemessenen theoretischen Rahmen f\"ur die Untersuchung der
Pionphotoproduktion liefert die Quantenchromodynamik, die
inzwischen anerkannte Eichtheorie der starken Wechselwirkung.
Auf Grund des nichtperturbativen Charakters der Theorie bei kleinen
Energien ist allerdings die Beschreibung der Pionphotoproduktion 
auf der Basis der QCD  bis heute nicht m\"oglich. Einen 
sehr erfolgreichen Ansatz liefert dagegen die Verwendung von
Stromalgebratechniken zur Bestimmung der Photoproduktionsamplitude 
an der Schwelle. Diese Methode verwendet die chirale Invarianz 
der QCD-Lagrangedichte, um die Schwellenamplitude mit Hilfe 
elementarer hadronischer Parameter, wie der Pionzerfallskonstante
und der axialen Kopplung des Nukleons, auszudr\"ucken.     

Neue Experimente zur Pionphotoproduktion an der Schwelle wurden an
den Elektronenbeschleunigern ALS in Saclay und MAMI A in Mainz 
durchgef\"uhrt. Diese Messungen haben zu Diskrepanzen mit den 
klassischen Stromalgebravorhersagen gef\"uhrt. Wir wollen daher
im ersten Teil der Arbeit eine kritische Analyse der experimentellen
Resultate und des theoretischen Niederenergietheorems vornehmen.
Konzentrieren werden  wir uns in diesem Zusammenhang auf m\"ogliche 
Korrekturen zum Niederenergietheorem, die sich aus der expliziten 
Brechung der chiralen Symmetrie durch die Stromquarkmassen in der 
QCD-Lagrangedichte ergeben. \newpage

Wichtige R\"uckschl\"usse auf die Rolle der expliziten Symmetriebrechung
erm\"oglicht auch das Studium der Photoproduktion von Eta-Mesonen, mit
dem wir uns in Kapitel 3 dieser Arbeit befassen wollen.  Das
Eta-Meson ist wie das Pion ein pseudoskalares Meson, besitzt aber eine
deutlich gr\"o\ss ere Masse. Aus diesem Grund haben Korrekturen
zum einfachen Niederenergietheorem eine deutlich gr\"o\ss ere Bedeutung,
als dies in der Photoproduktion von Pionen der Fall ist.

Im zweiten Teil der Arbeit wollen wir eine der Stromlagebra entlehnte
Technik, die Analyse spektraler Summenregeln, auf das Spektrum der 
Vektor- und Axialvektormesonen anwenden. Diese Spektren lassen sich 
experimentell in der $e^+e^-$-Annihilation bzw.~in hadronischen 
Zerf\"allen des $\tau$-Leptons bestimmen. 

Bei niedrigen Massen zeigen die Spektren die komplexe Struktur der 
Quark-Antiquark Bindungszust\"ande in der Quantenchromodynamik.
Dagegen beschreiben die Spektralfunktionen im Bereich gro\ss er invarianter
Massen die Propagation eines nur schwach wechselwirkenden 
Quark-Antiquark Paares im QCD-Vakuum. Auch der Grundzustand der Theorie
ist au\ss erordentlich kompliziert, insbesondere erwartet man die
Kondensation von Gluonen und Quark-Antiquark Paaren. Diese Kondensate 
lassen sich bislang nur in einfachen Modellen des QCD-Vakuums 
bestimmen. Ihr Einflu\ss\ auf das Spektrum bei sehr hohen Massen 
kann aber in st\"orungstheoretischen Rechnungen systematisch 
ber\"ucksichtigt werden.

QCD-Summenregeln verkn\"upfen das experimentell bestimmte Spektrum bei
kleinen Massen mit der asymptotischen Form des Spektrums in 
st\"orungstheoretischer QCD. Sie er-m\"og\-li\-chen damit einen wichtigen
Test f\"ur die oben angesprochenen Modelle der Vakuumstruktur.

Im Vektormesonkanal gibt es sowohl f\"ur Systeme schwerer Quarks als
auch f\"ur leichte Quarks eingehende Untersuchungen \"uber die 
Konsistenz der Daten mit QCD-Sum\-men\-re\-geln. Wir wollen in dieser 
Arbeit eine solche Analyse auch f\"ur den Axialvektorkanal vornehmen.
Theoretisch unterscheiden sich diese beiden Kan\"ale bei hohen 
invarianten Massen durch die Form der auftretenden Quarkkondensate.  
Eine Untersuchung des Spektrums der Axialvektormesonen erm\"oglicht daher
Aussagen \"uber die Faktorisierbarkeit der Quarkkondensate.


\pagestyle{empty}

\addcontentsline{toc}{chapter}{Teil I 

Pionphotoproduktion an der Schwelle} 

\vspace*{5cm}

\centerline{\bf\huge Teil I}

\vspace*{4cm}

\centerline{\bf\huge Pionphotoproduktion an der}

\vspace*{1cm}

\centerline{\bf\huge Schwelle}

\cld

\chapter{Pionphotoproduktion}
\section{Einleitung}
%revised Jan. 2, 1992
Das Pion spielt eine herausragende Rolle als Bindeglied zwischen 
der klassischen Kernphysik als Theorie der Wechselwirkung von 
Nukleonen im Kernverband und der heute anerkannten fundamentalen
Theorie der starken Wechselwirkung, der Quantenchromodynamik.  
Diese Bedeutung ergibt sich aus der Sonderstellung der Pionen
$(\pi^\pm,\pi^0)$, welche mit einer Masse von 139 MeV (bzw.~135 MeV)
die bei weitem leichtesten aller stark wechselwirkenden
Teilchen sind. 

Aus diesem Grund dominiert das Pion mit einer Comptonwellenl\"ange
von etwa 1.4 fm den langreichweitigen Teil der Nukleon-Nukleon
Wechselwirkung. Ebenso entstehen Pionen beim Zerfall der meisten
Anregungszust\"ande des Nukleons und liefern daher wichtige Informationen 
\"uber die Struktur der Hadronen bei niederen Energien.

Auf der anderen Seite versteht man die geringe Masse des Pions
als Manifestation der im Grundzustand spontan 
gebrochenen chiralen Symmetrie der Quantenchromodynamik.
Das Pion wird in diesem Zusammenhang als Goldstoneboson der
gebrochenen Symmetrie interpretiert und dominiert daher
den Niederenergiesektor der QCD. Dieser Bereich der Theorie
ist theoretisch nur schwer zug\"anglich, da die QCD-Kopplungskonstante
bei kleinen Energien sehr gro\ss\ wird. St\"orungstheoretische
Berechnungen mit Hilfe der fundamentalen Konstituenden der Theorie, den
Quarks und Gluonen, sind daher wenig sinnvoll. Pionische 
Reaktionen  sind ein wichtiges Feld, um 
unser Verst\"andnis dieses Teils der Theorie zu verbessern.

Im folgenden betrachten wir speziell die Photoproduktion 
von Pionen am Nukleon. Der Vorzug elektromagnetischer  
Prozesse liegt in der Tatsache, da\ss\  diese Wechselwirkung
besonders gut verstanden ist. Insbesondere erm\"oglicht die
kleine Kopplung $\frac{e^2}{4\pi}=\frac{1}{137}$ die Behandlung
der elektromagnetischen Wechselwirkung in niedrigster Ordnung. 
Die schwache Kopplung erweist sich auch als Vorteil
beim Studium von Photoproduktionsreaktionen an Kernen.
Photonen unterliegen nur einer geringen Absorption und erm\"oglichen 
die Produktion von Pionen \"uber das ganze Kernvolumen.

Das Studium der elementaren Produktion von Pionen am Nukleon ist vor 
allem aus zwei Gr\"unden bedeutsam. Zum einen ist das 
Verst\"andnis dieses Prozesses eine grundlegende Voraussetzung,
um mit Hilfe photoinduzierter Reaktionen am Kern das 
Verhalten von Pionen und nukleonischen Anregungen in 
Kernmaterie zu studieren. In dieser Arbeit wollen wir uns 
aber auf einen anderen Aspekt der Pionphotoproduktion
konzentrieren. In der N\"ahe der Schwelle erlaubt die Anwendung
von Niederenergietheoremen \cite{AD68}, die Photoproduktionsamplitude
in modellunabh\"angiger Weise als Funktion elementarer
hadronischer Parameter auszudr\"ucken. Diese Theoreme beruhen 
ausschlie\ss lich auf der Anwendung fundamentaler Symmetrien der
QCD und sind daher ein wichtiges Bindeglied zwischen der zugrunde 
liegenden Theorie und ph\"anomenologischen Modellen zur Beschreibung
des Pion-Nukleon Systems.
             

\section{Definition der invarianten Amplituden}
Die Photoproduktionsreaktion $\gamma(k)+N(p_1)\to\pi(q)+N(p_2)$
wird  durch die \"Ubergangsmatrix
\be
\label{tmat}
T^{a}=-e \epsilon^\mu \langle N(p_2)\pi^{a}(q)|V_\mu(0)|N(p_1)\rangle 
\ee
beschrieben. Dabei bezeichnen $k,q$ die Impulse des Photons sowie
des erzeugten Pions und $p_1,p_2$ die Impulse der ein- und 
auslaufenden Nukleonen. Dar\"uber hinaus ist $e$ die elektrische Ladung
und $\epsilon_\mu$ der Polarisationsvektor des einlaufenden Photons.
Der Index $a$ beschreibt den Isospin des produzierten Pions. Die 
\"Ubergangsmatrix l\"a\ss t sich wie folgt als Summe lorentzinvarianter
Amplituden schreiben
\be
\label{invamp}
T^{a} = \bar{u}(p_2)\,\sum_{\lambda=1}^{n_\lambda} A_{\lambda}(s,t)
   {\cal M}_\lambda \, u(p_1) \, .
\ee    
Die invarianten Amplituden $A_\lambda(s,t)$ sind Funktionen der 
Mandelstamvariablen $s$ und $t$, die wir in Anhang A definieren.
Sie liefern die Koeffizienten der 
Diracoperatoren ${\cal M}_\lambda$, deren Matrixelemente zwischen
freien Nukleonspinoren genommen sind. In der Literatur findet man
verschiedene m\"ogliche Entscheidungen bez\"uglich der Wahl dieser
Operatoren. Wir folgen hier de Baenst \cite{Bae70} und definieren
\newpage
\beq
\label{baeamp}
T^{a} &=& \frac{ie}{2M}\,\epsilon^\mu\, \bar{u}(p_2)\gamma_5 \left\{
    \frac{q_\mu}{2M} A_1^{a} + \frac{P_\mu}{2M} A_2^{a}
    + \gamma_\mu A_3^{a} \right .  \\
    & & \hspace{2.5cm} + \left. \left( \frac{q_\mu}{2M} A_4^{a}
     + \frac{P_\mu}{2M} A_5^{a} + \gamma_\mu A_6^{a} \right)
     \frac{\gamma\cdot k}{2M} \right\} u(p_1)\, , \nonumber 
\eeq  
wobei $P_\mu = \frac{1}{2} (p_1+p_2)_\mu$ den mittleren Impuls 
der Nukleonen bezeichnet.
Die Amplituden $A_\lambda^a$ sind  dimensionslos und frei von 
kinematischen Singularit\"aten. Dar\"uber hinaus haben sie den 
Vorzug, einen besonders kompakten Ausdruck f\"ur die 
Schwellenamplitude zu liefern.
 
Auf Grund der zus\"atzlichen Bedingungen, die sich aus der Forderung
nach Eichinvarianz der \"Ubergangsmatrix ergeben, sind  nicht
alle sechs Amplituden unabh\"angig voneinander. Verwendet man die
Erhaltung des elektromagnetischen Stroms $\partial_\mu V^\mu (x)=0$,
so ergibt sich
\be
\label{curcon}
  k^\mu T_\mu^{a} = -ek^\mu \langle N(p_2)\pi^{a}(q)|V_\mu(0)|N(p_1)\rangle =0 \; .
\ee   
Diese Bedingung liefert  zwei Gleichungen f\"ur 
die sechs Amplituden  $A_\lambda$ 
\be
\label{gaugecond}
\begin{array}{rcl} 
   2\nu_1 A_1^{a} + \nu A_2^{a}  &=& 0             \; , \\[0.2cm]
   4 A_3^{a} + 2 \nu_1 A_4^{a} + \nu A_5^{a} &=& 0 \; .
\end{array}   
\ee
Dabei haben wir die  dimensionslosen kinematischen 
Variablen 
\be
\label{dimvar}
  \nu = \frac{P\cdot k}{M^2} \hspace{1.5cm}
  \nu_1 = \frac{k\cdot q}{2 M^2}
\ee
eingef\"uhrt.  
Man kann die beiden Eichinvarianzbedingungen (\ref{gaugecond}) verwenden, 
um die Zahl der invarianten Amplituden auf vier zu reduzieren. Wir
geben eine solche Wahl von Operatoren in Appendix B an. Auf Grund
der genannten Vorz\"uge wollen wir im Hauptteil der Arbeit
aber bei der Definition (\ref{baeamp}) verbleiben.

Betrachtet man  die Elektroproduktion von Pionen in der Reaktion
$e(k_1)+N(p_1)\to e(k_2)+N(p_2)+\pi(q)$, so ist das ausgetauschte
Photon virtuell und besitzt eine nicht verschwindende invariante Masse
$k^2$. Man ben\"otigt daher  zwei
weitere Amplituden, um den hadronischen Teil der \"Ubergangsmatrix
zu spezifizieren. Eine denkbare Wahl f\"ur die entsprechenden
Diracoperatoren lautet
\beq
\label{elamp}
{\cal M}_7 &=& \frac{ie}{2M}\gamma_5 \frac{\epsilon\cdot k}{2M} \; ,\\
{\cal M}_8 &=& \frac{ie}{2M}\gamma_5 \frac{\epsilon\cdot k}{2M}
               \frac{\gamma\cdot k}{2M}  \; .  
\eeq
Neben der Zerlegung im Dirac-Raum ist es sinnvoll, die \"Ubergangsmatrix
auch nach verschiedenen Isospinkomponenten zu entwickeln.
Im Isospinraum transformiert sich der elektromagnetische Strom wie
die Summe eines Isoskalars und der dritten Komponente eines 
Isovektors
\be
\label{emcur}
 V_\mu = V_\mu^S\,+\, V_\mu^{3} \; ,
\ee
w\"ahrend die Pionquellfunktion ein reiner Isovektor ist. Aus der
Definition der \"Ubergangsmatrix ergibt sich daher folgende Isospinstruktur
f\"ur die invarianten Amplituden $A_\lambda$
\be
\label{isodec}
A_\lambda^{a} = A_\lambda^{(+)} \delta^{a3} + A_\lambda^{(-)}
\frac{1}{2} [\tau^{a},\tau^{3}] + A_\lambda^{(0)} \tau^{a} \; ,
\ee
wobei die Paulimatrizen $\tau^{a}$ auf den Isospinanteil der
Nukeonspinoren wirken. Amplituden, die zu bestimmten Isospins des
$\pi N$-System geh\"oren, lassen sich durch
\beq
  A_\lambda^{(\frac{1}{2})} &=& A_\lambda^{(+)}+2A_\lambda^{(-)} ,\\
  A_\lambda^{(\frac{3}{2})} &=& A_\lambda^{(+)}-A_\lambda^{(-)} 
\eeq
definieren. Die Amplitude $A_\lambda^{(0)}$ f\"uhrt auschlie\ss lich
zu $\pi N$-Zust\"anden mit dem Isospin $1/2$. Die Amplituden f\"ur die
vier physikalischen Kan\"ale sind durch
\beq
A_\lambda(\gamma p \to \pi^{+}n) &=&
           \sqrt{2} (A_\lambda^{(0)}+A_\lambda^{(-)} ), \\
A_\lambda(\gamma n \to \pi^{-}p) &=&
           \sqrt{2} (A_\lambda^{(0)}-A_\lambda^{(-)} ), \\
A_\lambda(\gamma p \to \pi^{0\,}p) &=&
            A_\lambda^{(+)}+A_\lambda^{(0)},  \\
A_\lambda(\gamma n \to \pi^{0\,}n) &=&
            A_\lambda^{(+)}-A_\lambda^{(0)}   
\eeq
gegeben. Ber\"ucksichtigt man isospinbrechende Effekte, wie sie durch die 
elektromagnetische Massendifferenz der Pionen und die kleine 
Differenz der up und down Stromquarkmassen in der QCD-Lagrangedichte
gegeben sind, so ist die Zerlegung (\ref{isodec}) nicht mehr
m\"oglich. In diesem Fall sind die Amplituden f\"ur die vier
Isospinkan\"ale unabh\"angig und eine experimentelle Bestimmung
der \"Ubergangsmatrix erfordert die Messung aller Reaktionskan\"ale.

Aus der Invarianz der \"Ubergangsmatrix unter Ladungskonjugation folgt, 
da\ss\ die invarianten Amplituden ein wohldefiniertes Verhalten unter
Austausch der Mandelstamvariablen $s$ und $u$ besitzen
\be
\label{cross}
 A_\lambda^{(+0,-)}(s,t,u) = \eta_\lambda^{(+0,-)} A_\lambda^{(+0,-)}
 (u,t,s) \; .
\ee  
Die Phasen $\eta_\lambda^{(+0,-)}$ sind durch die Gleichung
\be
\label{defphase}
 {\cal M}_\lambda (P,k,q) =-\eta_\lambda^{(+0)} C^{-1}
 {\cal M}_\lambda (-P,k,q) C
\ee
sowie $\eta_\lambda^{(-)}=\eta_\lambda^{(+0)}$ bestimmt.  Dabei
bezeichnet $C$ den Operator der Ladungskonjugation. Mit Hilfe
von (\ref{defphase}) findet man
\be
\label{cphase}
  \eta_\lambda^{(+0)} = \{ -1,+1,-1,-1,+1,+1 \}\, .
\ee   
Das Verhalten der dimensionslosen Gr\"o\ss en $\nu,\nu_1$ unter
dem Austausch der  Variablen im s- und u-Kanal lautet
$(\nu,\nu_1)\to (-\nu,\nu_1)$.  

\section[Multipolanalyse]{Multipolanalyse des differentiellen
 Wirkungs\-quer\-schnitts}
Um eine Multipolanalyse des differentiellen Wirkungsquerschnitts
durchf\"uhren zu k\"onnen, ist es zun\"achst sinnvoll, die 
\"Ubergangsamplitude nach Paulimatrizen und zweikomponentigen
Spinoren zu entwickeln. Eine solche Zerlegung ist nat\"urlich von
der Wahl des Bezugsystems abh\"angig. Wir definieren im 
Pion-Nukleon Schwerpunktsystem
\be
\label{famp}
\bar{u}(p_2)\, \sum_{\lambda} A_\lambda(s,t) {\cal M}_\lambda u(p_1)
 = \frac{4\pi\sqrt{s}}{M} \, \chi^{\dagger}_2 {\cal F} \chi_1\; ,
\ee
wobei $\chi_1$ und $\chi_2$ die zweikomponentigen Paulispinoren der ein- 
bzw.~auslaufenden Nu\-kle\-onen sind. Der kinematische Faktor in 
(\ref{famp}) hat 
den Zweck, die Definition der Amplitude ${\cal F}$ an die \"ublichen 
Konventionen f\"ur nichtrelativistische \"Ubergangsamplituden anzupassen.
Damit ergibt sich f\"ur den differentiellen Wirkungsquerschnitt
\be
\label{xdiff}
\frac{d \sigma}{d \Omega} = \frac{q}{k} \,
\left|\chi^\dagger_2 {\cal F} \chi_1 \right|^{2} \; .
\ee
Dabei bezeichnen $q=|\vec{q}\,|$ und $k=|\vec{k}|$ die Betr\"age der
Dreierimpulse des Pions bzw.~Photons. Wir werden diese Bezeichnung immer 
dann verwenden, wenn keine Verwechslungsgefahr mit den entsprechenden 
Vierervektoren besteht.
Der differentielle Wirkungsquerschnitt f\"ur unpolarisierte Teilchen
ergibt sich wie \"ublich,  indem man \"uber
die Spins im Endkanal summiert und im Eingangskanal mittelt. 

In Coulombeichung $\vec{\epsilon}\cdot\vec{k}=0$ l\"a\ss t sich der
Operator ${\cal F}$ nach vier linear unabh\"angigen Paulimatrizen
entwickeln 
\beq
\label{fdec}
{\cal F} &=& i F_1 (\vec{\sigma}\cdot\vec{\epsilon}\,) + F_2
             (\vec{\sigma}\cdot\hat{q}) (\vec{\sigma}\cdot
	     (\hat{k}\times\vec{\epsilon}\,))  \\
	 & &\mbox{} + i F_3 (\vec{\sigma}\cdot\hat{k})(\hat{q}\cdot
	     \vec{\epsilon}\,) + i F_4 (\vec{\sigma}\cdot\hat{q})
	     (\hat{q}\cdot\vec{\epsilon}\,) \nonumber \; .
\eeq
Die Amplituden $F_i$ sind komplexwertige Funktionen der
Mandelstamvariablen $s$ und $t$. Ihre Abh\"angigkeit vom
Streuwinkel $x=\hat{q}\cdot\hat{k}$ 
l\"a\ss t sich durch eine Entwicklung nach Legendrepolynomen 
extrahieren:
\beq
\label{f1mult}
F_1 &=& \sum_{l=0}^{\infty} \left[ l M_{l+} + E_{l+} \right] P^{'}_{l+1}(x)
	             +  \left[ (l+1) M_{l-} + E_{l-} \right] P^{'}_{l-1}(x), 
		     \\  
F_2 &=& \sum_{l=1}^{\infty} \left[ (l+1) M_{l+} + l M_{l-} \right] P^{'}_{l}(x)
		    , \\	  
F_3 &=& \sum_{l=1}^{\infty} \left[ E_{l+} - M_{l+} \right] P^{''}_{l+1}(x)
	             +  \left[ E_{l-} + M_{l-} \right] P^{''}_{l-1}(x), 
		     \\	  
\label{f4mult}		     
F_4 &=& \sum_{l=2}^{\infty} \left[ M_{l+} - E_{l+} - M_{l-} - E_{l-} \right] 
			     P^{''}_{l}(x) .
\eeq			     
Die energieabh\"angigen Multipolamplituden 
$E_{l\pm}$ und $M_{l\pm}$ geh\"oren zu Eigenzust\"anden des $\pi$N-Systems
mit dem Gesamtdrehimpuls $j=l\pm 1/2$. Elektrische bzw.~magnetische 
\"Ubergangsamplituden sind durch ihre Parit\"at $\pi=\pm (-1)^{l}$ 
charakterisiert. 

Die Multipolentwicklung dient vor allem zur experimentellen 
Bestimmung von Mul\-ti\-pol\-am\-plituden aus den gemessenen 
Winkelverteilungen. Um theoretisch bestimmte Spi\-nor\-am\-pli\-tuden
$F_i$ auf gegebene Multipolarit\"aten zu projizieren, ben\"otigt
man die Inversen der Beziehungen (\ref{f1mult}-\ref{f4mult}). 
Als Beispiel zitieren wir das Resultat f\"ur die elektrische 
Dipolamplitude \cite{BDW67}
\be
\label{eop}
E_{0+}(s) = \frac{1}{2}\int_{-1}^{1} dx\, \left(
  F_1 - x F_2 + \frac{1}{3} (1-P_{2}(x)) F_4 \right) .
\ee
Auch der totale Wirkungsquerschnitt l\"a\ss t sich als Funktion der
Multipolamplituden angeben. Mit Hilfe von (\ref{xdiff}) und 
(\ref{f1mult}-\ref{f4mult}) findet man 
\begin{figure}
\caption{Totaler Wirkungsquerschnitt f\"ur die Reaktionen 
$\gamma p\to\pi^{+}n$ und $\gamma p\to \pi^{0}p$ im Bereich 
von Laborenergien bis $\omega_{lab}=800$ MeV.}
\vspace{9cm}
\end{figure}   
\beq
\label{xtot}
 \sigma_{\rm tot} &=& 2\pi \frac{q}{k} \sum_{l=1}^{\infty}
 \left\{ l(l+1)^2 \left[ |M_{l+}|^2 +|E_{(l+1)-}|^2 \right] \right. \\
 & & \hspace{3cm} + l^2(l+1) \left.\left[ |M_{l-}|^2+|E_{(l-1)+}|^2 \right]
 \right\} \nonumber \; .
\eeq
Die Entwicklung nach Multipolen erweist sich als besonders
hilfreich, um die Beitr\"age von Nukleonresonanzen zur Photoproduktion
zu diskutieren. Solche Resonanzen besitzen wohl definierten Spin und Parit\"at
und tragen daher im $s$-Kanal selektiv zu bestimmten Multipolamplituden 
bei.

Zur Illustration zeigen wir in Abbildung 1.1 den totalen Wirkungsquerschnitt 
f\"ur die Produktion geladener und neutraler Pionen 
am Proton im Bereich von Photonenergien im
Laborsystem bis maximal $\omega_{lab}=800$ MeV. Als charakteristische
Eigenschaft der Reaktion in diesem Bereich erkennt man
die Anregung der $\Delta(1232)$-Resonanz. Auf Grund ihrer Quantenzahlen
$I(J^\pi)=\frac{3}{2}(\frac{3}{2}^{+})$ tr\"agt diese Resonanz vor 
allem zur $M_{1+}$-Amplitude bei und bewirkt eine ausgepr\"agte 
p-Wellenstruktur des differentiellen Wirkungsquerschnitts in der 
Resonanzregion. Die relative Gr\"o\ss e des Resonanzquerschnitts im
neutralen  und geladenen Kanal sollte wegen des Isospins der Anregung
genau $2:1$ betragen. Die Abweichung der experimentellen Daten von 
dieser Vorhersage ist eine Konsequenz der unterschiedlichen St\"arke 
des s-Wellen-Untergrunds in den beiden Kan\"alen.
 
\section{Photoproduktion an der Schwelle}
Wir haben bereits in der Einleitung angemerkt, da\ss\ die Photoproduktion
von Pionen an der Schwelle von gro\ss er Bedeutung im Zusammenhang mit 
chiralen Niederenergietheoremen ist. Wir wollen daher in diesem Aschnitt die 
spezielle Situation an der Schwelle etwas n\"aher untersuchen.

An der Produktionsschwelle ist  die Gesamtenergie
im Schwerpunktsystem durch die Sum\-me der Ruhemassen der Teilchen
im Endzustand gegeben. Die Einschu\ss energie des Photons im 
Laborsystem betr\"agt in diesem Fall
\be
\label{wthr}
\omega^{th}_{lab} = \frac{1}{2M_1} \left( (M_2 + m_{\pi^{a}})^2
   - M_1^2 \right) \; .
\ee
Dabei bezeichnet $M_1$ die Masse des Targetnukleons, $M_2$ die Masse
des auslaufenden Nukleons und $m_{\pi^{a}}$ die Masse des produzierten
Pions. Auf Grund der unterschiedlichen Massen 
\beq
\Delta m_\pi &=& m_\pi^\pm -m_\pi^0=4.6\,\mev ,  \\
\Delta M_N   &=& M_n-M_p = 1.3\, \mev
\eeq
der verschiedenen Ladungszust\"ande der Pionen und Nukleonen sind die
Schwellenenergien der Kan\"ale nicht identisch (siehe Tabelle 1.1). Von 
Bedeutung f\"ur die Behandlung von Endzustandskorrekturen ist vor allem die
Tatsache, da\ss\ bei gegebenem Isospin des Targets die Produktionsschwelle
f\"ur geladene Pionen einige MeV \"uber derjenigen f\"ur neutrale Pionen
liegt.
    
\begin{table}
\caption{Schwellenenergie des einlaufenden Photons im Labor und 
 Schwerpunktsystem}
 \begin{center}
 \begin{tabular}{|c||c|c|} \hline
         &  $\omega_{th}^{lab} $[MeV]  & $\omega_{th}^{\em cms}$ [MeV] \\
	 \hline \hline
 $\gamma p\to \pi^+ n$   &  151.43     &   131.67  \\ 
 $\gamma n\to \pi^- p$   &  148.45     &   129.40  \\
 $\gamma p\to \pi^0 p$   &  144.68     &   126.49  \\
 $\gamma n\to \pi^0 n$   &  144.67     &   126.50  \\ \hline
 \end{tabular}
 \end{center}
 \end{table}

Das Verhalten der Multipolamplituden in der N\"ahe der Schwelle l\"a\ss t
sich aus der Darstellung der Amplituden mit Hilfe von Dispersionsrelationen 
entnehmen. Verwendet man die Analytizit\"at und Austauschsymmetrie 
der Spinoramplituden, so ergibt sich   
\be
\label{thramp}
\begin{array}{cclc}
E_{l+}   & \sim & q^l k^l     & l\ge 0  \, ,\\
E_{l-}   & \sim & q^l k^{l-2} & l\ge 2  \, ,\\
M_{l\pm} & \sim & q^l k^l     & l\ge 1  \, . 
\end{array}
\ee
An der Schwelle wird der differentielle Wirkungsquerschnitt von der
elektrischen Di\-pol\-am\-plitude $E_{0+}$ dominiert und zeigt ein typisches
s-Wellenverhalten 
\be
\label{thrxdiff}
\left. \frac{d\sigma}{d\Omega}\right|_{thr} = 
\frac{q}{k} \left|E_{0+}\right|^2 \, .
\ee
Direkt an der Schwelle verschwindet $q$, w\"ahrend $k\simeq m_\pi$. 
Die elektrische Dipolamplitude bestimmt daher die Steigung des 
differentiellen Wirkungsquerschnitts in der N\"ahe der Schwelle als 
Funktion von $q/k \sim (\omega^{lab}-\omega_{th}^{lab})^{1/2}$. In der Praxis
ist dieses Verhalten allerdings nur schwer zu beobachten, da der
Wirkungsquerschnitt bereits bei sehr kleinen Pionenergien von
p-Wellen-Multipolen mit einer charakteristischen Energieabh\"angigkeit
$(\omega^{lab}-\omega^{lab}_{th})^{3/2}$ dominiert wird.

Die elektrische Dipolamplitude $E_{0+}$ an der Schwelle l\"a\ss t sich 
mit Hilfe der Gleichungen (\ref{famp}) und (\ref{eop}) aus den
invarianten Amplituden $A_\lambda$ bestimmen.  Zu diesem Zweck 
wertet man Matrixelemente
der Operatoren $M_\lambda$ zwischen freien Nukleonspinoren an der
Schwelle im Schwerpunktsystem aus. Identifiziert man den  
Koeffizienten der $(\vec{\epsilon}\cdot\vec{\sigma})$-Struktur, so
findet man 
\be
\label{thrinvamp}
 \left. E_{0+} \right|_{thr} = \frac{e}{16\pi M} \frac{2+\mu}{(1+\mu)^{3/2}}
  \left. ( A_3 + \frac{\mu}{2}  A_6 ) \right|_{thr}\, ,
\ee
wobei $\mu=m_\pi/M$ das Verh\"altnis der Pion- und Nukleonmassen bezeichnet.
 
Eine einfache Absch\"atzung der relativen St\"arke der Dipolamplitude
in den einzelnen Ladungskan\"alen 
kann man aus folgender \"Uberlegung gewinnen. An der Schwelle ist die
Wellenl\"ange des Photons $k^{-1}\simeq 1.4$ fm so gro\ss , da\ss\ es nicht 
in der Lage ist, die detaillierte Struktur des Nukleons aufzul\"osen. Die 
Dipolamplitude ist daher allein vom klassischen Dipolmoment des
Pion-Nukleon Systems im Endzustand abh\"angig \cite{EW88}. Aus dieser
Betrachtung ergibt sich
\be
\label{elpred}
\left.
\begin{array}{c}
 \Epn \\[0.2cm] \Emp \\[0.2cm] \Eop \\[0.2cm] \Eon 
\end{array}
\right\}
= \frac{ef}{4\pi m_\pi} \left( 1+\mu \right)^{-\frac{3}{2}}
\left\{
\begin{array}{c}
\sqrt{2} \\[0.2cm]
-\sqrt{2} ( 1 + \mu ) \\[0.2cm]
-\mu \\[0.2cm]
0
\end{array}
\right.  ,
\ee
wobei wir h\"ohere Ordnungen in $\mu=m_\pi/M$ vernachl\"assigt haben. Die
St\"arke der elektrischen Dipolamplitude ist durch den Faktor $f/m_\pi$
bestimmt, wobei $f$ die pseudovektorielle Pion-Nukleon-Kopplungskonstante
bezeichnet, $f^2/(4\pi) = 0.08$.  Das Resultat zeigt, da\ss\ die 
Produktion neutraler Pionen am Proton um etwa eine Gr\"o\ss enordnung 
gegen\"uber dem geladenen Kanal unterdr\"uckt ist. Insbesondere 
verschwindet die $\pi^0$-Produktionsamplitude, wenn die Pionmasse gegen 
Null geht. Die
Produktion neutraler Pionen ist daher besonders sensitiv auf die 
explizite Brechung der chiralen Symmetrie, die sich in einer von Null
verschiedenen Pionmasse widerspiegelt.  Diese Feststellung gilt 
in besonderer Weise f\"ur den $\pi^0 n$-Kanal, in dem die 
klassische N\"aherung eine verschwindende elektrische Dipolamplitude 
liefert. 


\section{Zur experimentellen Situation}
In diesem Abschnitt wollen wir eine detaillierte Diskussion der 
experimentellen Ergebnisse zur Pionphotoproduktion an der Schwelle
vornehmen. Bis vor einigen Jahren existierten zuverl\"assige Daten in
diesem Bereich nur in den beiden geladenen Kan\"alen. 
W\"ahrend diese Ergebnisse im Fall der Reaktion $\Rpn$ direkt an einem 
Wasserstofftarget gewonnen werden k\"onnen, stammen die Daten f\"ur die
Reaktion $\Rmp$ entweder aus der quasifreien Produktion am Deuteron oder dem 
inversen Proze\ss\ $\pi^- p \to \gamma n$. Die Ergebnisse finden sich
in Tabelle 1.2 und zeigen eine gute \"Ubereinstimmung mit den Vorhersagen
der chiralen Niederenergietheoreme. 

\begin{table}
\caption{Theoretische Vorhersagen und experimentelle Ergebnisse
f\"ur die Schwellenamplitude $ E_{0+}^{th}$ in Einheiten von
$10^{-3}m_\pi^{-1}$.}
\begin{center}
\begin{tabular}{|c||c|rl|} \hline
Kanal                  & $E_{0+}$(LET) & $E_{0+}$(Exp)  & Referenz \\ \hline
                                                                      \hline 
$\gamma p \to \pi^+ n$ &$\spm 26.6$    &  $28.3\pm 0.5$ & [Ada76] \\ \hline
$\gamma n \to \pi^- p$ &    $-31.7$    & $-31.9\pm 0.5$ & [Ada76] \\ \hline
$\gamma p \to \pi^0 p$ &    $-2.3$     &  $-2.7\pm 0.4$ & [NSV74] \\
                       &               &  $-0.5\pm 0.3$ & [Maz86] \\
		       &               & $-0.35\pm 0.1$ & [Bec89] \\ \hline  
$\gamma n \to \pi^0 n$ &    $-0.5$     & $-\hspace{1cm}$  &        \\ \hline
\end{tabular}
\end{center}
\end{table}

Gleiches gilt f\"ur den bis zum Jahre 1986 allgemein akzeptierten 
Wert f\"ur die $\pi^0 p$ Amplitude an der Schwelle, $\Eop=
2.7\pm 0.4 \su$ \cite{NSV74}. Dieses Ergebnis stammt allerdings aus
einer ph\"anomenologischen Multipolanalyse, die sich im wesentlichen
auf Daten aus der Resonanzregion st\"utzt. Dabei werden die Realteile
der niedrigsten Multipole ($l$=0,1) bei einer gegebenen  Energie mit
Hilfe eines freien Fits an die gemessenen Winkelverteilungen bestimmt. 
Die zugeh\"origen Imagin\"arteile fixiert man  aus
den bekannten $\pi N$-Streuphasen. H\"ohere Multipolamplituden werden
aus Dispersionsanalysen oder theoretischen Modellen \"ubernommen. 
Gew\"ohnlich ergeben sich bei
dieser Prozedur verschiedene S\"atze von Multipolamplituden, die eine
\"ahnlich gute \"Ubereinstimmung mit den Daten zeigen. Aus diesem Satz
von Amplituden w\"ahlt man mit Hilfe zus\"atzlicher Kriterien, wie
zum Beispiel der Forderung nach einer m\"oglichst stetigen 
Energieabh\"angigkeit, die beste L\"osung aus. Es sind vor allem diese
Kriterien, welche die Extrapolation zur Schwelle beeinflussen.

Inzwischen konnten jedoch unter Verwendung moderner 
Dau\-er\-strich-Elek\-tro\-nen\-be\-schleu\-niger in Saclay \cite{Maz86} und
Mainz \cite{Bec90} bedeutende Fortschritte bei der Bestimmung der
Photoproduktionsamplitude direkt an der Schwelle gemacht werden. 
Der wesentliche Vorzug dieser Maschinen besteht in der Tatsache, 
da\ss\ dank der hohen kontinuierlichen Intensit\"at
Mehrfach-Koinzidenzexperimente mit markierten (``tagged'')
Photonen m\"oglich sind. Diese Technik gestattet die Produktion von
Photonen mit sehr genau definierter Energie ($\sim 200$ keV) und 
erm\"oglicht pr\"azise Experimente direkt an der Schwelle. 
Es war daher eine betr\"achtliche \"Uberraschung, als die ersten 
Datenanalysen mit Resultaten $\Eop = (-0.5\pm 0.3) \su$ \cite{Maz86}
bzw. $(-0.35 \pm 0.1) \su$ \cite{Bec89} signifikante Abweichungen
vom Niederenergietheorem zeigten. 

Bei der Analyse dieser Resultate wollen wir uns im folgenden auf 
das Mainzer Experiment \cite{Bec90,Str90} beschr\"anken. Dieses Experiment 
zeichnet sich im Vergleich zu den Messungen aus Saclay durch eine deutlich 
bessere Statistik aus. Dar\"uber hinaus besteht das Problem,
da\ss\ an den Saclay-Daten, ebenso wie an den ersten, unver\"offentlichten 
Daten aus Mainz \cite{Bec89}, bereits eine theoretische Korrektur f\"ur 
Endzustandswechselwirkungen vorgenommen worden ist.
\begin{figure}
\caption{Differentielle Wirkungsquerschnitte f\"ur die Reaktion
$\gamma p\to \pi^0 p$. Gezeigt sind die Resultate aus Mainz
[Bec90] f\"ur verschiedene Laborenergien des einlaufenden
Photons.}
\vspace{20cm}
\end{figure}

In der N\"ahe der Schwelle ist es ausreichend, in der Multipolentwicklung
(\ref{f1mult}-\ref{f4mult}) nur Drehimpulse bis $l=1$ zu ber\"ucksichtigen. 
Auf Grund der Dominanz der resonanten $M_{1+}$-Amplitude ist diese N\"aherung
auch jenseits der unmittelbaren Schwellenregion in vielen F\"allen
ausreichend. Mit Hilfe von (\ref{xdiff}) ergibt sich dann f\"ur den 
differentiellen Wirkungsquerschnitt 
\be
\label{xang}
\frac{d\sigma}{d\Omega} = \frac{q}{k} \left(
 A + B\cos \Theta + C \cos^2\Theta \right) 
\ee
mit den Koeffizienten
\beq
 A & = & |E_{0+}|^2 + \frac{1}{2} ( |P_1|^2 + |P_3|^2 ), \nonumber \\ 
 \label{angcoef}
 B & = & 2 {\rm Re} (E_{0+}P_1^* ), \\
 C & = & |P_1|^2 - \frac{1}{2} ( |P_2|^2 + |P_3|^2 ) ,\nonumber
\eeq
wobei wir die p-Wellen-Multipole
\beq
\label{pmult}
 P_{1,2} &=& 3 E_{1+} \pm M_{1+} \mp M_{1-} \; , \\
 P_3     &=& 2 M_{1+} + M_{1-}
\eeq
eingef\"uhrt haben. Die in Mainz gemessenen differentiellen 
Wirkungsquerschnitte in der Schwellenregion zeigen wir in 
Abbildung 1.2. Ebenfalls dargestellt ist die Parametrisierung
(\ref{xang}). Die zugeh\"origen Koeffizienten $A,B,C$ finden
sich in der Referenz \cite{Bec90}. Um die Multipolamplituden 
aus diesen Daten zu extrahieren, schreiben wir (\ref{angcoef}) 
als ein System von Gleichungen f\"ur $E_{0+}$ und $P_1$ :
\begin{table}
\label{e0tab}
\caption{Bestimmung der elektrischen Dipolamplitude $\Eop$ aus 
den Mainzer Daten. Winkelverteilungskoeffizienten in Einheiten von
$nb\,sr^{-1}$, Dipolamplituden in $10^{-3} m_\pi^{-1}$.}
\begin{center}
\begin{tabular}{|c||r|c|c|c|} \hline
 $E_{\gamma}$ 
          & $A+B+C$      & $ E_{0+}^I $     &  $E_{0+}^{II}$   
	                 & $E_{0+}^f$         \\ \hline \hline
 146.8    & $36\pm 26$   & $-0.25\pm 0.20$  & $-1.59\pm 0.38$ 
                         & $-1.64\pm 0.10$     \\ 
 149.1    & $19\pm 27$   & $-0.72\pm 0.29$  & $-1.69\pm 0.45$ 
                         & $-1.26\pm 0.15$     \\  	      
 151.4    & $40\pm 31$   & $-0.56\pm 0.21$  & $-0.56\pm 0.21$ 
                         & $-0.79\pm 0.26$     \\  
 153.7    & $90\pm 34$   & $-0.25\pm 0.17$  & $-0.25\pm 0.17$ 
                         & $-0.54\pm 0.33$     \\ 
 156.1    & $189\pm 40$  & $-0.35\pm 0.14$  & $-0.35\pm 0.14$ 
                         & $-1.02\pm 0.39$     \\  \hline
\end{tabular}
\end{center}  
\end{table}
\be
\begin{array}{rcl}
A+C &=& |E_{0+}|^2 + |P_1|^2  \\[0.2cm]
 B\hspace{0.5cm}  &=& 2 {\rm Re} (E_{0+}P_1^* )  \; .
\end{array} 
\ee
Man erkennt bereits an dieser Stelle, da\ss\ die Gleichungen
v\"ollig symmetrisch in $E_{0+}$ und $P_1$ sind. Es sind daher
zus\"atzliche Kriterien erforderlich, um die korrekte
L\"osung auszuw\"ahlen. Unterhalb der $\pi^+ n$-Schwelle sind keine 
weiteren Kan\"ale offen und alle Multipole bis auf einen kleinen 
Beitrag aus der $\pi^0 p$-Streuphase reell. Vernachl\"assigt man 
die Imagin\"arteile, so ergeben sich vier L\"osungen f\"ur die Amplituden
$E_{0+}$ und $P_1$. Stellt man dar\"uber hinaus die physikalischen
Bedingungen $E_{0+}<0$ und $P_1>0$, so reduziert sich diese Zahl auf nur 
noch zwei L\"osungen\footnote{Bei der Auswahl der L\"osungen
haben wir von der Tatsache Gebrauch gemacht, da\ss\ f\"ur die 
gemessenen Winkelverteilungen $A>0$ sowie $B,C<0$ gilt.}
\beq
 E_{0+} &=& \frac{1}{2} \left( -\sqrt{A-B+C} \pm \sqrt{A+B+C} \right) \; ,\\
 P_1    &=& \frac{1}{2} \left( +\sqrt{A-B+C} \pm \sqrt{A+B+C} \right) \; . 
\eeq
Dabei ist in beiden Gleichungen jeweils dasselbe Vorzeichen zu w\"ahlen. 
Je nach dieser Wahl erh\"alt man $|E_{0+}|<|P_1|$ (L\"osung I) oder
$|E_{0+}|>|P_1|$ (L\"osung II). Nach Gleichung (\ref{thramp}) ist
das Schwellenverhalten der Amplituden durch $E_{0+} \sim {\rm const}$
und $P_1 \sim qk$ gegeben. Direkt an der Schwelle verschwindet $P_1$,
so da\ss\ die physikalisch korrekten Amplituden durch L\"osung II
gegeben sind. In der Resonanzregion dagegen ist die $M_{1+}$-Amplitude deutlich
gr\"o\ss er als die elektrische Dipolamplitude, so da\ss\ in diesem 
Bereich L\"osung I zu w\"ahlen ist. Das Problem besteht daher in der
Bestimmung der kritischen Energie, bei der von der einen L\"osung zur
anderen zu wechseln ist. Wir wollen im folgenden zu diesem Zweck 
sowohl die Winkelverteilungen als auch die totalen Wirkungsquerschnitte 
zu Rate ziehen.
 
\begin{table}
\label{p1tab}
\caption{Bestimmung der p-Wellen-Amplitude $P_1$ aus den Mainzer Daten. Impulse 
$kq$ in Einheiten $fm^{-2}$, Multipolamplituden in $10^{-3} m_\pi^{-1}$.}
\begin{center}
\begin{tabular}{|c||c|c|c|c|} \hline
$E_{\gamma}$ & $(kq)$ & $P_1^I$        & $P_1^{II}$      & $P_1^f$    \\ \hline
                                                                         \hline
 146.8  & 0.069     & $1.59\pm 0.38$   & $0.25\pm 0.20$  & $1.10\pm 0.07$ \\
 149.1  & 0.101     & $1.69\pm 0.45$   & $0.72\pm 0.29$  & $1.62\pm 0.10$ \\  
 151.4  & 0.127     & $1.97\pm 0.39$   & $1.97\pm 0.39$  & $2.03\pm 0.13$ \\
 153.7  & 0.149     & $2.42\pm 0.21$   & $2.42\pm 0.21$  & $2.39\pm 0.15$ \\
 156.1  & 0.170     & $3.41\pm 0.27$   & $3.41\pm 0.27$  & $2.73\pm 0.17$ \\
 \hline
\end{tabular}
\end{center}
\end{table}
 

Bei der kritischen Energie ist $|E_{0+}|=|P_1|$ und daher auch
$A+B+C=0$. Wir haben in Tabelle 1.3 die Summe der gemessenen 
Winkelverteilungskoeffizienten angegeben. Dabei erkennt man ein deutliches
Minimum bei der Photonenergie $E_\gamma=149.1$ MeV, wo die Summe der
Koeffizienten mit Null vertr\"aglich ist. Wir haben daher diesen Wert
mit der kritischen Energie identifiziert und dabei (etwas willk\"urlich)
den Punkt $E_\gamma=149.1$ MeV noch in den Bereich von L\"osung II genommen.
Das entsprechende Resultat findet sich in der vierten Spalte von Tabelle
1.3 und ist mit $E_{0+}^{II}$ bezeichnet. Die Experimentatoren
schlie\ss en jedoch ein Szenario, in dem die kritische Energie kleiner 
als $146.8$ MeV ist, nicht v\"ollig aus \cite{Bec90}.
In diesem Fall ergeben sich die mit $E_{0+}^{I}$ bezeichneten
L\"osungen. 

Die beiden L\"osungen f\"ur $E_{0+}$ sind in Abbildung 1.3 dargestellt.
W\"ahrend $E_{0+}^{II}$ im Rahmen der Fehler mit dem Niederenergietheorem
direkt an der Schwelle vertr\"aglich erscheint, steht das Szenario $E_{0+}^I$
in deutlichem Widerspruch zu der LET-Vorhersage. Wir wollen nun 
demonstrieren, da\ss\ eine Betrachtung der totalen Wirkungsquerschnitte 
eine klare Pr\"aferenz f\"ur die mit dem Niederenergietheorem
vertr\"agliche L\"osung $E_{0+}^{II}$ ergibt \cite{Ber91,Sch91}.

\begin{figure}
\caption{Ergebnis der Mainzer Multipolanalyse f\"ur die
s- und p-Wellen-Multipole $E_{0+}$ und $P_1$. Gezeigt sind
die Resultate f\"ur die beiden Szenarien I (kein Vorzeichenwechsel) und II
(Vorzeichenwechsel bei $E_\gamma=149.1$ MeV) als Funktion von $(qk)$.}
\vspace{9cm}
\end{figure}
\begin{figure}
\caption{Totaler Wirkungsquerschnitt f\"ur die Reaktion $\Rop$ nach
[Str90]. Aus den Daten wurde der kinematische Faktor $4\pi\frac{q}{k}$
herausskaliert. Die durchgezogene Linie ist der Beitrag der 
p-Wellen-Amplitude $2f_0^2(qk)^2$ f\"ur $f_0=8\cdot 10^{-3}m_\pi^{-3}$.}
\vspace{9cm}
\end{figure}     

Ber\"ucksichtigt man wie oben nur die Beitr\"age der Multipole mit
$l=0$ und $l=1$, so ist der totale Wirkungsquerschnitt durch
\beq
 \sigma_{\rm tot} &=& 4\pi \left( \frac{q}{k} \right) 
 \left( |E_{0+}|^2 + 2|M_{1+}|^2 + |M_{1-}|^2 + 6|E_{1+}|^2 \right)
  \nonumber \\
\label{xfit}  
 & \simeq & 4\pi \left( \frac{q}{k} \right) 
 \left( ({\rm Re} E_{0+})^2 + 2f_0^2 (qk)^2 \right) 
\eeq
gegeben, wobei wir in der zweiten Zeile den Imagin\"arteil von $E_{0+}$
vernachl\"assigt und das Schwellenverhalten der Dipolamplituden 
verwendet haben. Diese Annahme ist zumindest f\"ur die resonante 
$M_{1+}$-Amplitude im Bereich der gemessenen Daten sicher gerechtfertigt.
F\"ur die anderen beiden p-Wellen-Amplituden erwartet man jedoch
Korrekturen an dieser einfachen Parametrisierung. 

Die einfachste M\"oglichkeit zur Bestimmung von $f_0$ besteht darin, 
diesen Parameter mit der $M_{1+}$-Amplitude zu identifizieren. 
Ph\"anomenologische Analysen ergeben den Wert $M_{1+}\simeq 8qk 
\cdot 10^{-3} m_\pi^{-3}$ \cite{NSV74}. Alternativ l\"a\ss t sich
$f_0$ auch direkt aus den gemessenen Daten extrahieren, indem 
man den totalen Wirkungsquerschnitt in der Form
\be
 \frac{1}{4\pi} \left( \frac{k}{q} \right) \sigma_{tot}\simeq
 c_0^2 + 2f_0^2 (qk)^2 
\ee
parametrisiert. Die Ergebnisse zeigen wir in Abbildung 1.4.
Unter Verwendung der Daten im Bereich $E_\gamma=144.7$ MeV bis 
$E_\gamma=156.3$ MeV finden  wir $f_0=7.1\cdot 10^{-3}m_\pi^{-3}$.
F\"ur $c_0=0$ ergibt sich $f_0=8.0\cdot 10^{-3}m_\pi^{-3}$, in 
\"Ubereinstimmung mit dem ph\"anomenologischen Wert der $M_{1+}$-Amplitude. 
Man erkennt, da\ss\ die totalen Wirkungsquerschnitte bei h\"oheren
Energien sehr gut durch den Beitrag der p-Wellen-Amplitude allein
beschrieben werden. Dagegen ergibt sich direkt an der Schwelle eine 
Diskrepanz, die  auf eine nichtverschwindende $E_{0+}$-Amplitude 
hinweist.
 
Benutzt man den Wert $f_0=8\cdot 10^{-3}m_\pi^{-3}$, um mit Hilfe von 
(\ref{xfit}) den Realteil von $E_{0+}$ zu extrahieren, so ergeben sich die 
in der letzten Spalte der Tabelle 1.3 angegebenen Werte. Die zitierten
Fehler ber\"ucksichtigen lediglich den experimentellen Fehler in der
Bestimmung des totalen Wirkungsquerschnitts und beinhalten nicht die 
Unsicherheit im Wert von $f_0$. Die resultierenden Dipolamplituden $E_{0+}^f$
zeigen bei kleinen $E_\gamma$ eine deutliche Pr\"aferenz f\"ur die 
mit dem Niederenergietheorem vertr\"agliche L\"osung $E_{0+}^{II}$. 
Bei der h\"ochsten Energie ist die \"Ubereinstimmung nicht mehr 
so gut. Wir werten diese Tatsache als einen deutlichen Hinweis auf 
Abweichungen von der naiven $qk$-Abh\"angigkeit der p-Wellen-Amplitude. 

Auf Grund der besseren Statistik sind die totalen Wirkungsquerschnitte
in kleineren Energieintervallen ($\Delta E_\gamma \simeq 0.3$ MeV)
bestimmt worden als die Winkelverteilungen. In Tabelle 1.3 haben wir 
jeweils \"uber mehrere Intervalle gemittelt, um die mit Hilfe der 
Formel (\ref{xfit}) bestimmten elektrischen Dipolamplituden  
mit der Multipolanalyse vergleichen zu k\"onnen. Der niedrigste Datenpunkt
liegt in diesem Fall bei $E_\gamma = 146.8$ MeV, also etwa 2 MeV 
oberhalb der Schwelle. N\"aher an die Schwelle heran kommt man, indem
man auf die Mittelwertbildung verzichtet \cite{Ber91}.
Verwendet man den  Datensatz in  Abbildung 1.4, um $E_{0+}^f$ bis an die 
Schwelle zu  extrapolieren, so findet man den Wert $E_{0+}^f =-(2.1\pm 0.2) 
\su$. Dieses Resultat ist im Bereich der experimentellen Unsicherheiten
konsistent mit der Vorhersage des Niederenergietheorems.   

In Tabelle 1.4 vergleichen wir die verschiedenen Resutate 
f\"ur die Amplitude $P_1$. Keine der beiden L\"osungen der
Multipolanalyse ist mit einer einfachen $qk$-Abh\"angigkeit
ver\-tr\"ag\-lich. Die von uns bevorzugte L\"osung $P_1^{II}$ zeigt
eine signifikante p-Wellen-Unterdr\"uckung in der N\"ahe der
Schwelle.

Zusammenfassend stellen wir fest, da\ss\ die Bestimmung der 
$E_{0+}$-Am\-pli\-tu\-de an der Schwel\-le von Mehrdeutigkeiten betroffen 
ist. Sorgf\"altige Analysen zeigen allerdings keine signifikante Verletzung
des Niederenergietheorems direkt an der Schwelle. 
Dagegen findet man aber eine \"uberraschend starke
Energieabh\"angigkeit der $E_{0+}$-Amplitude und eine Abweichung
der p-Wellen-Amplitude vom erwarteten $qk$-Verhalten.
\cld

\chapter{Niederenergietheoreme zur Pionphotoproduktion}
%revised Jan. 2, 1992
\section{Einleitung}
Nachdem wir uns im letzten Kapitel mit der experimentellen 
Bestimmung der elektrischen Dipolamplitude an der Schwelle
befa\ss t haben, wollen wir uns nun auf die theoretische 
Bestimmung von $E_{0+}$ mit Hilfe von Niederenergietheoremen
konzentrieren. Die spezielle Bedeutung der Photoproduktion
neutraler Pionen ergibt sich dabei aus der Tatsache, da\ss\
die entsprechende Schwellenamplitude in einer hypothetischen Welt
mit masselosen Pionen  verschwinden w\"urde.
Dieser Kanal ist daher besonders sensitiv auf die Rolle der
expliziten chiralen Symmetriebrechung, welche sich in  dem
nur approximativen Charakter des Pions als Goldstoneboson 
widerspiegelt. 

Die physikalische Grundidee der Niederenergietheoreme
({\em engl.} Low Energy Theorem, LET) l\"a\ss t sich besonders
\"ubersichtlich an rein elektromagnetischen Reaktionen
diskutieren. Die Anwendung  von Niederenergietheoremen wird 
in diesem Fall durch zwei physikalische Kriterien kontrolliert. 
Die beiden Forderungen lauten, da\ss\ die Wellenl\"ange des 
Photons gro\ss\ ist im Vergleich zur Ausdehnung des Streuzentrums
und die Energie des Photons klein ist gegen die typische 
Anregungsenergie. Sind diese Voraussetzungen erf\"ullt, so ist das 
Photon nicht in der Lage, die innere Struktur des Targets aufzul\"osen. 
Der differentielle Wirkungsquerschnitt ist daher ausschlie\ss lich durch 
globale elektromagnetische Eigenschaften des Streuzentrums bestimmt.
Betrachtet man die Comptonstreuung niederenergetischer
Photonen an einem hadronischen Target, so folgt aus dieser
\"Uberlegung, da\ss\ der differentielle Wirkungsquerschnitt 
nur von der Gesamtladung $Ze$  und der Targetmasse $M$ 
abh\"angt. Insbesondere ergibt 
sich im Grenzfall $k_\mu=(\omega,\vec{k})\to 0$ die klassische
Thomson-Streuung
\be
\label{thomson}
 \lim_{\omega \to 0} \frac{d\sigma}{d\Omega} =
  \frac{Z^2e^2}{4\pi M^2} (\vec{\epsilon}_1 \cdot\vec{\epsilon}_2)^2 ,
\ee     
wobei $\vec{\epsilon}_1$ und $\vec{\epsilon}_2$ die 
Polarisationsvektoren der ein- und auslaufenden Photonen
bezeichnen. Low, Gell-Mann und Goldberger \cite{Low54,Low58,GMG54}
konnten dar\"uber hinaus zeigen, da\ss\ auf Grund von 
Eich- und Lorentzinvarianz die Amplitude f\"ur Comptonstreuung 
in Vorw\"artsrichtung an einem Spin-1/2 Target mit der Ladung $e$ sogar 
bis auf Terme linear in der Laborenergie $\omega$ bestimmt ist
\be
 \lim_{\omega \to 0}  T(\omega) =- \frac{e^2}{M}  
  (\vec{\epsilon}_1 \cdot\vec{\epsilon}_2) -i \frac{e^2}{2 M^2}
  \kappa^2 \omega  (\vec{\epsilon}_1 \times\vec{\epsilon}_2)
  \cdot \vec{\sigma} \; .
\ee
Der erste Term beschreibt  die Thomsonamplitude, w\"ahrend der
zweite Term eine Korrektur liefert, die proportional zum
anomalen magnetischen Moment $\kappa$ des Streuzentrums ist.

Die Anwendung von Niederenergietheoremen auf pionische 
Reaktionen wird durch die endliche Masse des Pions 
erschwert. Die Pionmasse setzt eine untere Grenze f\"ur 
die Energie  $\omega_\pi=(\vec{q}^{\, 2} +m_\pi^2)^{1/2}$
des Pions. W\"ahrend daher die Wellenl\"ange beliebig gro\ss\
gemacht werden kann, existiert eine prinzipielle Schranke 
f\"ur die Energie. Die Anwendbarkeit von Niederenergietheoremen
setzt daher voraus, da\ss\ die Masse des Pions klein gegen die 
charakteristische Energieskala der Reaktion ist. In hadronischen 
Prozessen ist eine solche Skala durch die Masse des Nukeons gegeben. 
Das Verh\"altnis $m_\pi/M\simeq 1/7$ ist daher ein nat\"urlicher
Parameter, der die Abweichung der  Amplitude vom unphysikalischen
Grenzfall $q_\mu \to 0$ kontrolliert.

Formal basieren Niederenergietheoreme f\"ur ''weiche`` Pionen auf
der G\"ultigkeit von Strom\-al\-ge\-bra, chiraler Symmetrie und
PCAC. Historisch wurden diese Konzepte in den sechziger Jahren 
als Hypothesen \"uber das Transformationsverhalten der in
hadronische Reaktionen eingehenden Str\"ome entwickelt
\cite{AD68,AFF73}. Sie erwiesen sich als au\ss erordentlich
fruchtbar, um die  experimentellen 
Informationen \"uber hadronische Reaktionen zu verstehen. 
Die klassischen Anwendungen liegen im Bereich
der $\pi\pi$- und $\pi N$-Streuung, sowie der Photo- und schwachen
Produktion pseudoskalarer Mesonen. Wichtige Vorhersagen ergeben
sich dar\"uber hinaus f\"ur leptonische und semileptonische Zerf\"alle
stark wechselwirkender Teilchen.    
  
Nach der Gr\"underphase trat die Anwendung von Stromalgebra und
chiraler Symmetrie zun\"achst in den Hintergrund gegen\"uber
der Entwicklung der Quantenchromodynamik als fundamentaler
Eichtheorie der starken Wechselwirkung. In diesem Zusammenhang 
zeigte sich allerdings, da\ss\ die G\"ultigkeit der erw\"ahnten Konzepte 
eine direkte Konsequenz der QCD ist. Dar\"uber hinaus liefert die chirale 
Symmetrie  einen wichtigen Zugang zur starken Wechselwirkung
in einem Bereich, in dem die direkte Anwendung der QCD
bislang noch gro\ss en Schwierigkeiten gegen\"ubersteht.

\section[Quantenchromodynamik, Stromalgebra \ldots]{Quantenchromodynamik,
Stromalgebra und chirale Symmetrie}
Die Quantenchromodynamik ist eine nichtabelsche Eichtheorie, 
beschrieben durch die Lagrangedichte
\be
\label{lqcd}
{\cal L}_{QCD} = -\frac{1}{4} G_{\mu\nu}^{\;\;a} G^{\mu\nu\, a}
 + \sum_{j=1}^{n_f} \bar{\psi}^{\alpha}_{j}( i\gamma^{\mu}
 {\cal D}_\mu^{\alpha\beta} - \delta^{\alpha\beta} m_j )
 \psi_j^\beta \; ,
\ee 
wobei die  Summation \"uber $n_f$ verschiedene Quarkarten (flavors)
ausgef\"uhrt wird. W\"ahrend sich die Quarkspinoren $\psi^{\alpha}$ nach
der fundamentalen Darstellung der Eichgruppe $SU(3)$ transformieren,
sind die Gluonen $A_\mu^{a}$ Vektorfelder und tragen einen Index 
in der adjungierten Darstellung der $SU(3)$. Die zugeh\"origen
Elemente der Lie-Algebra sind
\be
 A_\mu(x) = A_\mu^{a}(x)\frac{\lambda^{a}}{2}\; ,
\ee
wobei $\lambda^{a}$ die Generatoren der Algebra bezeichnet.
Sie erf\"ullen die fundamentalen Vertauschungsrelationen
\be
 [\lambda^{a},\lambda^b] = 2if^{abc}\lambda^c
\ee
und k\"onnen durch
\be
 Tr \,\lambda^{a}\lambda^b = 2\delta^{ab}
\ee
normiert werden. Dabei bezeichnet $f^{abc}$ die Strukturkonstanten
von $SU(3)$. Der Yang-Mills-Feldst\"arketensor ist durch
\be
\label{fmunu}
 G_{\mu\nu}^{\;\; a} = \partial_\mu A_\nu^{a} -\partial_\nu A_\mu^{a} 
 + g f^{abc} A_\mu^b A_\nu^c
\ee
gegeben, w\"ahrend die kovariante Ableitung durch
\be
\label{kovd}
 {\cal D}_\mu^{\alpha\beta} = \delta^{\alpha\beta}\partial_\mu
  + i\frac{g}{2} (\lambda^{a})^{\alpha\beta} A_\mu^{a}
\ee
definiert ist. Entscheidend f\"ur die Eichinvarianz der Theorie
ist die Tatsache, da\ss\ die Quark-Gluon-Wechselwirkung durch dieselbe
Kopplungskonstante $g$ wie die gluonische Selbstwechselwirkung 
bestimmt ist.
   
Neben der lokalen $SU(3)$ Eichsymmetrie besitzt die $QCD$-Lagrangedichte 
noch eine Reihe kontinuierlicher globaler Symmetrien.
So ist ${\cal L}_{QCD}$ invariant unter der globalen 
$U(1)_V$-Transformation
\be
\label{uone}
\psi_j(x) \to \exp (-i\theta) \psi_j (x) \, .
\ee
Der dazugeh\"orige erhaltene Vektorstrom ist der Baryonenstrom
\be
 j_\mu(x) = \sum_{j=1}^{n_f} \bar{\psi}_j \gamma_\mu \psi_j
\ee
mit der baryonischen Ladung 
\be
B=\int d^3x\, j_0(\vec{x},t)\, .
\ee
F\"ur masselose Quarks ist ${\cal L}_{QCD}$ ebenfalls invariant
unter axialen $U(1)_A$-Transformationen
\be
\label{uaone}
\psi_j(x) \to \exp (-i\theta\gamma_5) \psi_j (x) \; .
\ee
Der entsprechende Strom besitzt allerdings eine anomale Divergenz
\be
\label{axanom}
\partial^\mu j_{\mu\, 5}(x) = \frac{g^2}{4\pi}\frac{n_f}{8}
 \epsilon^{\mu\nu\rho\sigma} G_{\mu\nu}^{\;\; a}G_{\rho\sigma}^{\;\; a}
 \; ,
\ee
so da\ss\  die axiale Ladung $Q_5=\int d^3x\, j_{0\,5}(\vec{x},t)$
nur in  Abwesenheit instantonartiger L\"osungen erhalten ist.

Vernachl\"assigt man die Quarkmassen, so ist ${\cal L}_{QCD}$ dar\"uber 
hinaus invariant unter Skalentransformationen. Diese Symmetrie wird in 
der quantisierten Theorie durch die Notwendigkeit der Renormierung
gebrochen. Dabei tritt ein dimensionsbehafteter Parameter, der
QCD-Skalenparameter $\Lambda_{\mini QCD}$, auf. Sein Wert ist unter anderem 
aus der Skalenbrechung in der tief-inelastischen Lepton-Nukleon-Streuung zu
$\Lambda_{\mini QCD}^{\mini\overline{MS}} =230\pm 80$ MeV bestimmt worden
\cite{PDG90}. Der Index ${\kl \overline{MS}}$ bezeichnet eine spezielle
Renormierungsvorschrift, die sogenannte modifizierte minimale 
Subtraktion.

Ebenfalls auf Grund der Renormierung sind  auch die Werte der 
Quarkmassen von der experimentellen Skala abh\"angig.
Im Bereich typischer hadronischer Prozesse lassen sich 
die Stromquarkmassen mit Hilfe von QCD-Summenregeln 
extrahieren. Bei $\mu^2=1\,{\rm GeV}^2$ findet man
\cite{GL82,DR87}
\beq
   m_u &=& 5.1 \pm 1.5 \;{\rm MeV}, \nonumber  \\
   m_d &=& 8.9 \pm 2.6 \;{\rm MeV}, \\
   m_s &=& 175 \pm 55 \;{\rm MeV}.  \nonumber
\eeq   
Alle anderen bekannten Flavors haben Massen \"uber einem GeV. Die
up- und down-Quarks sind au\ss erordentlich leicht verglichen mit dem
QCD-Skalenparameter, w\"ahrend das seltsame Quark eine Zwischenstellung
einnimmt.   

Wir wollen daher im folgenden die drei leichten Quarks zun\"achst als
masselos betrachten. In diesem Fall besitzt ${\cal L}_{QCD}$
eine chirale $SU(3)_L \times SU(3)_R$ Symmetrie im Raum der $u$-,
$d$- und $s$-Flavors
\beq
\label{suv}
\psi_i(x) &\to& \exp (-i\theta^{a}\lambda^{a})_{ij} \psi_j (x)\, ,\\
\label{sua}
\psi_i(x) &\to& \exp (-i\phi^{a}\lambda^{a}\gamma_5)_{ij} \psi_j (x)\, ,
\eeq
wobei die $SU(3)$ Matrizen $\lambda^{a}$ auf den Flavorindex der
Quarks wirken. Die zugeh\"origen N\"otherstr\"ome sind die
Vektor- und Axialvektorstr\"ome
\beq
   V_\mu^{a}(x) &=& \bar{\psi}_i \gamma_\mu \frac{\lambda^{a}_{ij}}{2}
     \psi_j \, , \\
   A_\mu^{a}(x) &=& \bar{\psi}_i \gamma_\mu \gamma_5
   \frac{\lambda^{a}_{ij}}{2}   \psi_j  \, , 
\eeq
deren Zeitkomponenten auf die erhaltenen Ladungen 
\beq
 Q^{a}(t)   &=& \int d^3x \, V_0^{a}(\vec{x},t) \, , \\
 Q^{a}_5(t) &=& \int d^3x \, A_0^{a}(\vec{x},t) 
\eeq
f\"uhren. Die Struktur der zugeh\"origen Lie-Algebra erkennt man 
am einfachsten, indem man zu den 
Linearkombinationen $Q^{a}_{L,R}=\frac{1}{2}(Q^{a}\mp Q^{a}_5)$ \"ubergeht.
Sie erf\"ullen die Vertauschungsrelationen
\beq
\label{chalg}
\,[Q_{L}^{a}(t),Q_{L}^{b}(t)] &=& i f^{abc} Q_{L}^{c}(t)\, ,  \\
\,[Q_{R}^{a}(t),Q_{R}^{b}(t)] &=& i f^{abc} Q_{R}^{c}(t)\, ,  \\
\,[Q_{L}^{a},Q_{R}^{b}]     &=& 0\, ,
\eeq
charakteristisch f\"ur die Lie-Algebra $SU(3)_L \times SU(3)_R$.  
Das Transformationsverhalten der Str\"ome unter der chiralen
$SU(3)_L\times SU(3)_R$ ergibt sich aus den kanonischen
Vertauschungsregeln f\"ur die Quarkfelder
\beq
\label{curalg}
\,[ Q^{a}(t),V_\mu^b (\vec{x},t)] &=&
               \!\!\spm i f^{abc}V_\mu^{c}(\vec{x},t)\, ,\\
\,[ Q^{a}(t),A_\mu^b (\vec{x},t)] &=&
               \!\!\spm i f^{abc}A_\mu^{c}(\vec{x},t)\, ,\\  
\,[ Q^{a}_5(t),V_\mu^b (\vec{x},t)] &=&
               \!\! -i f^{abc}A_\mu^{c}(\vec{x},t)\, ,\\ 
\,[ Q^{a}_5(t),A_\mu^b (\vec{x},t)] &=&
               \!\!\spm i f^{abc}V_\mu^{c}(\vec{x},t) \; .
\eeq 
Diese Relationen legen  die $SU(3)_L\times SU(3)_R$ Darstellung
fest, nach der sich die Str\"ome transformieren. Sie wurden von 
Gell-Mann auf Grund rein ph\"anomenologischer \"Uberlegungen 
postuliert \cite{AD68}. 

Soll die QCD im Grenzfall verschwindender Massen 
der leichten Quarks eine sinnvolle 
N\"aherung an die vollst\"andige Theorie darstellen, dann kann der 
QCD-Grundzustand nicht $SU(3)_L\times SU(3)_R$ symmetrisch sein. 
W\"are n\"amlich
\be
\label{symvac}
 Q_L^{a}|0\rangle  = Q_R^{a}|0\rangle  = 0 \; ,
\ee
dann sollten auch die Zweipunktfunktionen der Vektor- und Axialvektorstr\"ome
identisch sein
\be
\label{symspec}
\langle 0|T(A_\mu^{a}(x)A_\nu^{b}(0))|0\rangle  =\langle 0|T(V_\mu^{a}(x)V_\nu^{b}(0))|0\rangle  \; .
\ee  
Zum Beweis zerlegt man die beiden Str\"ome in ihre links- und rechtsh\"andigen
Komponenten 
\beq
 V_\mu^{a} &=& J_{\mu\, R}^{a} + J_{\mu\, L}^{a}\, , \\
 A_\mu^{a} &=& J_{\mu\, R}^{a} - J_{\mu\, L}^{a}
\eeq
und folgert aus (\ref{symvac}) und dem Transformationsverhalten der Str\"ome,
da\ss\  die nichtdiagonale Kombination $\langle 0|T(J_{\mu\, L}^{a}J_{\nu\,R}^{a})|0\rangle $
verschwindet.

Die spektralen Dichten zu den beiden Zweipunktfunktionen (\ref{symspec}) sind
experimentell zu\-g\"ang\-lich und zeigen eine sehr verschiedenartige Gestalt.
W\"ahrend der Vektorkanal vor allem durch die $\rho$-Resonanz bei 
$m_\rho=770$ MeV dominiert wird, ist die Axialvektorspektralfunktion in 
diesem Bereich klein und zeigt erst im Bereich der $a_1$-Resonanz bei
$m_{a_1}=1260$ MeV eine ausgepr\"agte Struktur. Wir folgern daraus, da\ss\ 
der QCD Grundzustand nicht $\chs$ symmetrisch ist. Dieses Ph\"anomen, 
da\ss\  der Grundzustand der Theorie
nicht die volle Symmetrie der Lagrangedichte besitzt, bezeichnet man
als spontane Symmetriebrechung.

Nach dem Goldstone-Theorem impliziert die spontane Brechung einer 
kontinuierlichen Symmetrie
das Auftreten masseloser Bosonen. Ist $Q^{a}$ ein  $\chs$ Generator,
der das Vakuum nicht invariant l\"a\ss t, dann gibt es einen physikalischen
Zustand $Q^{a}|0\rangle $, der mit dem Vakuum energetisch entartet ist.  
Handelt es sich bei $Q^{a}$ um eine Vektorladung, dann beschreibt
$Q^{a}|0\rangle $ ein masseloses skalares Teilchen. Ist $Q^{a}$ dagegen eine
axiale Ladung, so fordert das Goldstone-Theorem das Auftreten masseloser
pseudoskalarer Bosonen.

In der Natur sind die acht leichtesten Hadronen $(\pi,K,\eta)$ pseudoskalar.
Dagegen sind die leichtesten skalaren Teilchen schwerer als das Nukleon.
Die chirale Symmetrie ist daher in der Form
\beq
   Q_V^{a}|0\rangle  &=& 0 \, , \\
   Q_A^{a}|0\rangle  &\neq& 0
\eeq
realisiert. Die Vektorsymmetrie bleibt erhalten, so da\ss\  sich Hadronen
nach irreduziblen Darstellungen von $SU(3)_V$ klassifizieren lassen. Im
$SU(2)$-Sektor der Theorie folgt daraus die Isospinsymmetrie der 
starken Wechselwirkung.

Endliche Quarkmassen brechen die chirale Symmetrie explizit. Die 
Divergenz der Vektor- und Axialvektorstr\"ome lautet 
\beq
\label{divv}
\partial^\mu V_\mu^{a} &=& \frac{i}{2} 
               \bar{\psi}\left[M,\lambda^{a}\right]\psi\, , \\ 
\label{diva}
\partial^\mu A_\mu^{a} &=& \frac{i}{2} 
               \bar{\psi}\left\{M,\lambda^{a}\right\}\psi \, ,
\eeq
wobei $M={\rm diag}(m_u,m_d,m_s)$ die Massenmatrix f\"ur die drei 
leichten Flavors bezeichnet. Die rechte Seite von (\ref{divv},\ref{diva})
l\"a\ss t sich besonders \"ubersichtlich darstellen, indem man $M$ nach 
Gell-Mann-Matrizen entwickelt, $M=\epsilon_0\lambda^0+\epsilon_3\lambda^3+
\epsilon_8\lambda^8$. Dabei ist
\beq
\epsilon_0 &=& \frac{1}{\sqrt{2}} (m_u+m_d+m_s)\, , \\
\epsilon_3 &=& \frac{1}{2} (m_u-m_d)\, ,  \\
\epsilon_8 &=& \frac{1}{2\sqrt{3}} (m_u+m_d-2m_s)\, .
\eeq
Im Fall entarteter Quarkmassen $m_u=m_d=m_s$ ist $\epsilon_3=\epsilon_8=0$,
und die $SU(3)_V$ Flavor-Symmetrie bleibt ungebrochen.
 
Im Hadronenspektrum manifestieren sich die nichtverschwindenden 
Quarkmassen in einer endlichen Masse f\"ur die Goldstonebosonen 
$(\pi,K,\eta)$. Der Zusammenhang dieser Massen mit denen der Quarks
ist durch die Gell-Mann, Oakes, Renner (GOR)-Relation gegeben
\cite{GOR68}. Um die bei der Herleitung von Niederenergietheoremen
verwendeten Methoden vorzustellen, wollen wir im folgenden einen
kurzen Beweis der GOR-Relation vorstellen. Zu diesem Zweck definieren
wir die Pionzerfallskonstante durch das Matrixelement
\be
\label{fpi}
 \langle 0|A_\mu^{a}(x)|\pi^{b}(q)\rangle  = 
 i\delta^{ab} f_\pi q_\mu e^{-iq\cdot x} .
\ee
Diese \"Ubergangsmatrix kontrolliert den schwachen Zerfall der
geladenen Pionen $\pi^\pm \to \mu^\pm \nu_\mu$. F\"ur das
Matrixelement der Divergenz des Axialstroms findet man
\be
 \langle 0|\partial^\mu A_\mu^{a}(x)|\pi^{b}(q)\rangle  = 
         \delta^{ab} f_\pi m_\pi^2 e^{-iq\cdot x}\; ,
\ee	     
so da\ss\ sich ein interpolierendes Feld f\"ur das Pion durch
die Beziehung
\be
\label{PCAC}
\partial^\mu A_\mu^{a}(x) = f_\pi m_\pi^2 \phi^{a} (x)
\ee
definieren l\"a\ss t \cite{Col67}. Diese Gleichung wird als PCAC (partially
conserved axial current) Relation bezeichnet. Sie ist 
\"aquivalent zur Divergenzbeziehung (\ref{diva}) und dr\"uckt wie
diese die Tatsache aus, da\ss\  der Axialvektorstrom im chiralen 
Limes $m_\pi \to 0$ erhalten ist.

Wir wollen im folgenden Wardidentit\"aten f\"ur die Zweipunktfunktionen
\beq
\label{axtwop}
\Pi_{5\, \mu\nu}^{ab}(q) &=& i\int d^4x\, e^{iq\cdot x}
          \langle 0|T(A_\mu^{a}(x)A_\nu^{b}(0))|0\rangle  \; ,   \\
\label{divtwop}	  
\psi_{5}^{ab} (q) &=& i\int d^4x\, e^{iq\cdot x}
          \langle 0|T(\partial^\mu A_\mu^{a}(x)\partial^\nu A_\nu^{b}(0))|0\rangle 
\eeq
konstruieren. Zweimaliges Differenzieren des zeitgeordneten Produkts 
(\ref{axtwop}) liefert die Beziehung
\beq
\label{wi}
q^\mu q^\nu  \Pi^{ab}_{5\,\mu\nu} (q) &=& \psi_5^{ab}(q)
   -q^\nu \int d^4x\, e^{iq\cdot x} 
   \delta (x^0) \langle 0|[A_0^{a}(x),A_\nu^{b}(0)]|0\rangle  \\
   & & \mbox{} -i\int d^4x\,  e^{iq\cdot x} 
   \delta (x^0) \langle 0|[A_0^{a}(x),\partial^\mu A_\mu^{b}(0)]|0\rangle \, , \nonumber
\eeq
die sich im Limes $q_\mu \to 0$ auf die Gleichung
\be
\label{psi0}
  \psi_5^{ab}(0) = -i \int d^4x \, \delta(x^0) 
        \langle 0|[A_0^{a}(x),\partial^\mu A_\mu^{b}(0)]|0\rangle     	   	       
\ee	
reduziert. Die linke Seite dieser Gleichung folgt aus der PCAC Relation, 
w\"ahrend man die rechte Seite mit Hilfe von (\ref{diva}) und
den kanonischen Vertauschungsregeln f\"ur die Quarkfelder 
bestimmen kann. Das Resultat ergibt schlie\ss lich die GOR-Relation
\be
\label{GOR}
 -f_\pi^2 m_\pi^2 = m_u\langle 0|\bar{u}u|0\rangle + 
 m_d \langle 0|\bar{d}d|0\rangle \; .
\ee
Sie verbindet die hadronischen Parameter $m_\pi$ und $f_\pi$ mit
den Quarkmassen und dem Ordnungsparameter f\"ur die spontane 
Brechung der chiralen Symmetrie, dem Quarkkondensat $\langle 0|\bar uu
+\bar dd|0\rangle $. Wir haben in der Herleitung von dem Grenz\"ubergang 
$q_\mu \to 0$ Gebrauch gemacht. Die GOR-Relation ergibt daher 
eine Aussage \"uber die Pionzerfallskonstante im chiralen Limes.
F\"ur den physikalischen Wert von $f_\pi$ liefert die
GOR-Relation nur den f\"uhrenden Term in einer Entwicklung
in Potenzen der Quarkmasse.     

%Die Wardidentit\"at (\ref{wi}) gilt unabh\"angig von den 
%physikalischen Zust\"anden, zwischen denen die Matrixelemente
%genommen sind. Ersetzt man die Vakuumzust\"ande durch ein und
%auslaufende Nukleonen, so l\"a\ss t sich die Divergenzamplitude 
%$\psi_5^{ab}$  mit der Pion-Nukleon-Streumatrix
%$T_{\pi N}^{ab}$ identifizieren. Auf diese
%Weise k\"onnen wir die Rolle der expliziten Symmetriebrechung
%in physikalischen Streuprozessen studieren. Durch zweimaliges
%Differenzieren der Zweipunktfunktion $\Pi^{ab}_{5\,\mu\nu}$ ergibt sich 
%folgende Wardidentit\"at f\"ur $T_{\pi N}^{ab}$ \cite{BPP71}:
%\beq
%\label{tpin}
% T_{\pi N}^{ab}(p_1,q_1;p_2,q_2) &=& T_{PV}^{ab}(p_1,q_1;p_2,q_2)
%   +\frac{q_1^2+q_2^2-m_\pi^2}{f_\pi^2 m_\pi^2} \Sigma^{ab}(p_1,p_2) \\
%   & &\mbox{} +\frac{1}{2f_\pi^2} (q_1+q_2)^\mu V_\mu^{ab}(p_1,p_2)
%   +q_1^\mu q_2^\nu R^{ab}_{\mu\nu}(p_1,q_1;p_2,q_2)\, . \nonumber
%\eeq
%Dabei haben wir $q_1^\mu q_2^\nu\Pi^{ab}_{5\,\mu\nu}$ in die 
%Beitr\"age der Pseudovektor-Bornterme $T_{PV}^{ab}$ und die 
%Hintergrundamplitude $q_1^\mu q_2^\nu R_{\mu\nu}^{ab}$ zerlegt.
%Sie enth\"alt weder Nukleon- noch Pionpole
%und verschwindet daher im Niederenergielimes $q_1,q_2 \to 0$.
%Des weiteren bezeichnet $V_\mu^{ab}(p_1,p_2)$ den Stromalgebraterm
%\be
% V_\mu^{ab}(p_1,p_2) = i\epsilon^{abc}\langle N(p_2)|V_\mu^c(0)|N(p_1)\rangle \, .
%\ee
%Unser spezielles Augenmerk liegt auf dem Pion-Nukleon Sigmaterm
%$\Sigma^{ab}(p_1,p_2)$, der ein Ma\ss\ ist f\"ur die St\"arke der 
%expliziten Symmetriebrechung. Wie bei der Herleitung der 
%GOR-Relation ergibt sich
%\be
%\label{pinsig}
%\Sigma^{ab}(p_1,p_2) = \frac{\delta^{ab}}{2}(m_u+m_d)
%    \langle N(p_2)|\bar{u}u+\bar{d}d|N(p_1)\rangle \; .
%\ee
%Der Sigmaterm liefert an der Schwelle den f\"uhrenden Beitrag zum
%isospinsymmetrischen Teil der Streuamplitude. Dar\"uber hinaus kann
%man den  Wert von
%$\Sigma^{ab}(p_1,p_2)$ am unphysikalischen Punkt $p_1=p_2$
%als Beitrag der Strommassen der leichten Quarks zur Nukleonmasse
%interpretieren. Diese Gr\"o\ss e l\"a\ss t sich prinzipiell aus dem
%beobachteten Baryonspektrum ermitteln. In chiraler St\"orungstheorie
%findet man \cite{GL80}
%\be
% \sigma =\frac{1}{2}(m_u+m_d)\langle p|\bar{u}u+\bar{d}d|p\rangle 
%    = \frac{35\pm 5}{1+y}\; {\rm MeV}\, ,              
%\ee
%wobei $y=\langle p|\bar{s}s|p\rangle /\langle p|\bar{u}u|p\rangle $ das Verh\"altnis des
%Kondensats der seltsamen Quarks zu dem  der leichten Quarks im
%Nukleon bezeichnet. Um den Wert von $\sigma$ aus
%Pion-Nukleon Streuphasen zu extrahieren, ist ein aufwendiges
%Extrapolationsverfahren notwendig. Im Gegensatz zu fr\"uheren
%Auswertungen liefern neuere Dispersionsanalysen einen 
%Wert $\sigma=45 \pm 5$ MeV \cite{GLS91}, der vertr\"aglich
%ist mit der Annahme eines verschwindenden Kondensats seltsamer
%Quarks im Nukleon.

\section{Ableitung des Niederenergietheorems}
Die Herleitung von Niederenergietheoremen zur Pionphotoproduktion
verl\"auft analog zu der im letzten Abschnitt geschilderten Ableitung 
der GOR-Relation. Ausgangspunkt ist die Darstellung der Streumatrix mit Hilfe
der LSZ-Reduktionsformel
\beq
\label{LSZ}
 S^{a} &=& -(2\pi)^4 \,\delta^4 (p_1+k-p_2-q)\, Z_\gamma^{-1/2}
   Z_\pi^{-1/2} \\
   & & \mbox{}\cdot \int d^4x\, e^{iq\cdot x} (\Box +m_\pi^2)
   \langle N(p_2)|T\left(\epsilon^\mu V_\mu^{em}(0) \phi^{a}(x)\right)|N(p_1)\rangle 
   \nonumber\; .
\eeq
Dabei bezeichnet $\phi^{a}$ das kanonische Pionfeld, $Z_\pi=(2\pi)^3
2\omega_\pi$ dessen kovariante Normierung und $Z_\gamma=(2\pi)^3
2\omega$ die Normierung des elektromagnetischen Feldes. Die
\"Ubergangsmatrix nach der Definition aus dem ersten Kapitel ist
durch
\be
\label{deft}
 S^{a} = i(2\pi)^4\,\delta^4 (p_1+k-p_2-q) Z_\gamma^{-1/2}
  Z_\pi^{-1/2} \epsilon^\mu T_\mu^{a}
\ee
gegeben. Wir betrachten  die Zweipunktfunktion
\be
\label{Pimunu}
\overline{\Pi}^a_{\mu\nu}(q) = \int d^4 x\, e^{iq\cdot x}\langle N(p_2)| 
T\left( V_\mu^{em} (0) B_\nu^{a}(x) \right) |N(p_1)\rangle  \; .
\ee
des elektromagnetischen Stroms $V_\mu^{em}$ und des 'transversalen` Axialstroms
\be
B_\mu^{a}(x) =A_\mu^{a}(x)+\frac{1}{m_\pi^2}\partial_\mu D^{a}(x)\; ,
\ee
wobei $D^{a}(x)=\partial^\mu A_\mu^{a}(x)$ die Divergenz des Axialstroms
bezeichnet. Mit Hilfe der PCAC-Relation und der Definition der
Pionquellfunktion findet man
\be
\label{defb}
\partial^\mu B_\mu^a (x) = f_\pi (\Box +m_\pi^2)\phi^{a}(x) =
     -f_\pi j_\pi^{a}(x)\, .
\ee
Einmaliges Differenzieren des zeitgeordneten Produktes in der 
Zweipunktfunktion $\overline{\Pi}_{\mu\nu}^{a}$ liefert schlie\ss lich 
die gesuchte Wardidentit\"at f\"ur $T_\mu^{a}$ \cite{RST76}
\be
\label{avward}
T_\mu^a (q) = \frac{1}{f_\pi}\left\{
q^\nu \overline{\Pi}_{\mu\nu}^a (q) \, - \,i C_\mu^a (q)  \, - \,
\frac{\omega_\pi}{m_\pi^2} \Sigma^a_\mu (q) \right\} \; .
\ee
Dabei haben wir die LSZ-Formel (\ref{LSZ}) verwendet, um die 
Photoproduktionsamplitude $T_\mu^{a}$ zu identifizieren. Die Wirkung 
der Ableitung auf den Zeitordnungsoperator liefert die Kommutatoren
\beq
\label{cmua}
 C_\mu^{a}(q) &=& \int d^4x\, e^{iq\cdot x}\delta (x^0)
   \langle N(p_2)|[A^{a}_0(x),V_\mu^{em}(x)]|N(p_1)\rangle \; , \\
\label{sig}     
 \Sigma_\mu^{a}(q) &=& \int d^4x\, e^{iq\cdot x}\delta (x^0)
   \langle N(p_2)|[D^{a}(x),V_\mu^{em}(x)]|N(p_1)\rangle \; ,
\eeq
wobei wir das Resultat in den Stromalgebraterm $C_\mu^{a}$
und den Kommutator der Divergenz des Axialstroms zerlegt haben.  
In Analogie zum Sigmaterm in der Pion-Nukleon-Streuung bezeichnet
man diesen Beitrag auch als $(\gamma,\pi)$-Sigmaterm. Wie der
$\pi N$-Sigmaterm liefert er eine zus\"atzliche Korrektur, die direkt
proportional zu den Stromquarkmassen in der QCD-Lagrangedichte ist.

Es ist instruktiv, die Wardidentit\"at zu studieren, 
die sich aus der Zweipunktfunktion $\Pi^{a}_{\mu\nu}$ des
\"ublichen Axialstroms ergibt. Analog zu (\ref{avward})
erh\"alt man
\be
\label{avward2}
-\frac{m_\pi^2}{q^2-m_\pi^2} T_\mu^a (q) = \frac{1}{f_\pi}\Big\{
q^\nu \Pi_{\mu\nu}^a (q) \, - \, iC_\mu^a \Big\} \; .
\ee
Auf Grund des Pionpropagators vor der Amplitude $T_\mu^{a}$
liefert diese Beziehung die Photoproduktionsamplitude zun\"achst
nur am unphysikalischen ''weichen`` Punkt $q_\mu=0$.  Um die physikalische 
Schwelle $q^2=m_\pi^2$ zu erreichen, ist es notwendig, den Pionpol
explizit abzuseparieren. Diesem Zweck dient der transversale Axialstrom
$B_\mu^{a}$. Tats\"achlich l\"a\ss t sich $B_\mu^{a}$ mit Hilfe der
PCAC-Relation als der nichtpionische Anteil des Axialstroms interpretieren
\be
 B_\mu^{a}(x) = A_\mu^{a}(x) +f_\pi\partial_\mu \phi^{a}(x)
    = A_\mu^{a}(x) - A_{\mu}^{a\, (\pi)} (x) \, .
\ee
Wir wollen nun die verschiedenen Beitr\"age zur Photoproduktionsamplitude
$T_\mu^{a}$ im einzelnen studieren. Beginnen werden wir dabei mit
der Zweipunktfunktion $q^\nu\overline{\Pi}_{\mu\nu}^{a}$. Da dieser
Term proportional zum Impuls $q$ ist, tragen im Grenzfall weicher Pionen
nur die Polterme in $\overline{\Pi}_{\mu\nu}^{a}$ zur Amplitude bei.
Die Summe der Nukleonpolterme im direkten und im Austauschkanal
lautet
\beq
\label{nborn}
f_\pi T_\mu^{a\,(N)} &=& \bar{u}(p_2) \Big( q^\nu \Gamma_\nu^{B^{a}}
   (p_1-k,-q,p_2) S_F(p_1+k) \Gamma_\mu^\gamma (p_1,k,p_1+k)
      \\[0.2cm]
   & & \hspace{0.5cm} \mbox{}+ \Gamma_\mu^\gamma (p_1-q,k,p_2)
   S_F(p_1-q) q^\nu \Gamma_\nu^{B^{a}}(p_1,-q,p_1-q) \Big) u(p_1)
   \; .\nonumber
\eeq      
Dabei bezeichnet $S_F(p)$ den Nukleonpropagator und $\Gamma_\mu^\gamma$
bzw. $\Gamma_\nu^{B^{a}}$ die Vertexfunktionen f\"ur die Kopplung
des Nukleons an das elektromagnetische Feld und den Axialstrom $B_\nu^{a}$.
In den Poltermen sind alle Teilchen mit Ausnahme des
intermedi\"aren Nukleons auf der Massenschale. Die allgemeine Gestalt der
Vertexfunktion lautet in diesem Fall \cite{NK87}
\beq
\label{emvert}
i\Gamma_\mu^\gamma (p_1,k,p_1+k)u(p_1) &=& \left( e\gamma_\mu F_1 
   +  \frac{M+(p_1+k)\cdot\gamma}{2M} \frac{ie\sigma_{\mu\nu} k^\nu}{2M}F_2^+ 
   \right. \\
 & & \hspace{1.5cm}\left. \mbox{}
   +  \frac{M-(p_1+k)\cdot\gamma}{2M} \frac{ie\sigma_{\mu\nu} k^\nu}{2M}F_2^-
   \right) u(p_1),  \nonumber 
\eeq
\newpage
\beq   
\label{bavert}
i\Gamma_\nu^{B^{a}} (p_1,-q,p_1-q)u(p_1) &=& \left( \gamma_\nu \overline{G}_A 
   +  \frac{M+(p_1-q)\cdot\gamma}{2M} \frac{q_\nu}{2M} \overline{G}_P^{\, +}
   \right. \\
  & & \hspace{1.5cm}\left. \mbox{}  
   +  \frac{M-(p_1-q)\cdot\gamma}{2M} \frac{q_\nu}{2M} \overline{G}_P^{\, -} 
   \right)\gamma_5\frac{\tau^{a}}{2} u(p_1) \; .\nonumber
\eeq     
Analoge Ausdr\"ucke ergeben sich f\"ur den Vertex des auslaufenden 
Nukleons. Die Formfaktoren $F_i$ sind Funktionen
des Impuls\"ubertrages $k^2$ und des off-shell-Parameters $\delta^2=(p_1+k)^2
-M^2$. Ihre Isospinstruktur lautet
\be
 F_i=F_i^{s}+F_i^{v}\tau^3 \; .
\ee
Entsprechend h\"angen die Formfaktoren $\overline G_A$ und 
$\overline G_P^{\,\pm}$ am $NNB^{a}$-Vertex von den 
Variablen $q^2$ und ${\delta '}^2=(p_1-q)^2-M^2$ ab. Wir haben diese 
Formfaktoren mit einem Querstrich gekennzeichnet, um sie von den 
entsprechenden Funktionen am  Vertex des Axialvektorstroms zu 
unterscheiden.

Die Abh\"angigkeit der Photoproduktionsamplitude vom off-shell-Verhalten 
der Formfaktoren wurde in einer Arbeit von Naus, Koch
und Friar \cite{NKF90} untersucht. Die Autoren zeigen, da\ss\ 
off-shell-Korrekturen erst in derselben Ordnung in $m_\pi$ 
wie andere mo\-dell\-ab\-h\"angige Korrekturen auftreten. Wir verwenden
daher im folgenden die on-shell Vertices
\beq
i\Gamma_\mu^\gamma &=& e\gamma_\mu F_1 + 
          \frac{ie\sigma_{\mu\nu}k^\nu}{2M} F_2\; , \\
i\Gamma_\nu^{B^{a}}&=& \left( \gamma_\nu \overline{G}_A
         + \frac{q_\mu}{2M}\overline{G}_P \right) \gamma_5 
	 \frac{\tau^{a}}{2}\; ,
\eeq
wobei die auftretenden Formfaktoren nur mehr Funktionen der
Impuls\"ubertr\"age $k^2$ bzw.~$q^2$ sind. F\"ur reelle
Photonen ist
\be
\begin{array}{rclcrcl}
  F_1^{s}(0)&=& 1/2,        &\hspace{1cm}& F_1^{v}(0)&=& 1/2,     \\[0.2cm]
  F_2^{s}(0)&=&\kappa^s,    &            & F_2^{v}(0)&=&\kappa^v,  
\end{array}
\ee
mit den anomalen magnetischen Momenten $\kappa^{s,v}=\frac{1}{2}
(\kappa_p\pm \kappa_n)$. Die Vertexfunktionen f\"ur die beiden 
Str\"ome $A_\nu^{a}$ und $B_\nu^{a}$ unterscheiden sich nur 
um die Matrixelemente des pionischen Beitrags 
$A_\nu^{a(\pi)}=-f_\pi \partial_\nu \phi^{a}$. 
Dieser Term erzeugt den Pionpol im induzierten 
pseudoskalaren Formfaktor
\be
  G_P^{\pi -Pol} (q^2)=\frac{4Mf_\pi}{m_\pi^2-q^2} G_{\pi NN}(q^2)\; ,
\ee
wobei $G_{\pi NN}$ den Pion-Nukleon Formfaktor
\be
  \langle N(p_2)|j_\pi^{a}(0)|N(p_1)\rangle  = G_{\pi NN}(t) 
  \bar{u}(p_2)i\gamma_5 \tau^{a}u(p_1)
\ee
bezeichnet. Der Zusammenhang der Formfaktoren an den Vertices ist 
daher durch $\overline{G}_A=G_A$ und $\overline{G}_P=G_P-G_P^{\pi -Pol}$ 
gegeben. Empirische Untersuchungen zeigen, da\ss\ $G_P$ in sehr guter 
N\"aherung durch den Polterm alleine beschrieben wird. Wir setzen daher 
$\overline{G}_P=0$ und erhalten
\beq
\label{nborn2}
f_\pi T_\mu^{a\,(N)} &=& \bar{u}(p_2) \left\{ g_A q\cdot\gamma \gamma_5 
 \,\frac{\tau^{a}}{2} \frac{i}{(p_1+k)\cdot\gamma -M} \,\Gamma_\mu^\gamma
 \right. \\
 & & \hspace{1cm}\left. \mbox{} + \Gamma_\mu^\gamma 
     \,\frac{i}{(p_1-q)\cdot\gamma -M}\,
  g_A q\cdot\gamma \gamma_5 \frac{\tau^{a}}{2} \right\} u(p_1), \nonumber 
\eeq
wobei wir dar\"uber hinaus den axialen Formfaktor $G_A(q^2)$ durch den
Wert bei $q^2=0$, $g_A=G_A(q^2=0)$, ersetzt haben. Die axiale
Kopplung l\"a\ss t sich mit Hilfe der Goldberger-Treiman-Relation
\be
\label{GT}
\frac{g_A}{2f_\pi} = \frac{f}{m_\pi}
\ee
durch die pseudovektorielle Pion-Nukleon-Kopplungskonstante $f$ ausdr\"ucken.
Das Resultat (\ref{nborn2}) entspricht daher der Born-Approximation f\"ur
eine effektive Pion-Nukleon Lagrangedichte mit dem Kopplungsterm
\be
\label{pv}
{\cal L} = \frac{f}{m_\pi} \bar{\psi}\gamma_5\gamma_\mu \tau^{a}\psi
   \partial^\mu \phi^{a}\; .
\ee    
Die geschilderte Herleitung f\"uhrt also in nat\"urlicher Weise
auf eine pseudovektorielle Kopplung des Pions an das Nukleon. Dies 
steht im Gegensatz zu vielen klassischen Arbeiten, in denen 
gew\"ohnlich mit einer pseudoskalaren Kopplung gerechnet wird.
Um Konsistenz mit der Wardidentit\"at (\ref{avward}) zu erzielen,
m\"ussen in diesem Fall zus\"atzliche Korrekturterme zur Bornamplitude
addiert werden.

Der Beitrag des Kommutators $C_\mu^{a}$ l\"a\ss t sich mit Hilfe 
der Stromalgebraregeln berechnen
\be
\label{curcom}
 C_\mu^{a} = -i\epsilon^{a3c} \langle N(p_2)|A_\mu^{c}(0)|N(p_1)\rangle \; .
\ee
Vernachl\"assigt man den Hintergrundbeitrag im induzierten 
pseudoskalaren Formfaktor, so ergibt sich
\be
\label{kr}
\epsilon^\mu C_\mu^{a} = -i\epsilon^{a3c} \bar{u}(p_2)
  \left\{ G_A(t)\epsilon\cdot\gamma + 2f_\pi G_{\pi NN}(t)   
   \frac{\epsilon\cdot (k-q)}{m_\pi^2-t} \right\}
   \gamma_5 \frac{\tau^{c}}{2}u(p_1) 
\ee   	 	  
als Funktion der Mandelstamvariable $t=(q-k)^2$.
Das Resultat ist antisymmetrisch in den Isospinindices und
tr\"agt daher nur zur Produktion geladener Pionen bei. 
Der erste Term liefert den f\"uhrenden Beitrag zur 
Pho\-to\-pro\-duk\-ti\-ons\-amplitude im Grenzfall weicher Pionen
\be
\label{krtheo}
\lim_{q,k\to 0} T^{a}(q) =\frac{eg_A}{f_\pi}\epsilon^{a3c}
   \bar{u}(p_2) \epsilon\cdot\gamma\gamma_5
   \frac{\tau^{c}}{2}u(p_1)\; ,
\ee
den sogenannten Kroll-Ruderman-Term \cite{KR54}. 
Der zweite Kommutator enth\"alt die Divergenz des Axialstroms
und l\"a\ss t sich daher mit Ausnahme der Zeitkomponente 
$\Sigma_0^{a}$ nicht modellunabh\"angig bestimmen. Mit Hilfe
des Stromalgebraresultats
\be
 \,[Q_5^{a},V_\mu^{em}(0)]=-i\epsilon^{a3c}A_\mu^{c}(0)
\ee
und der Erhaltung des elektromagnetischen Stroms, 
$\partial^\mu V_\mu^{em}=0$, findet man
\be
\label{sig0}
  \int d^4x \,\delta (x^0)\, [\partial^\mu A_\mu^{a}(x),V_0^{em}
  (0)] = -i\epsilon^{a3c} \partial^\mu A_\mu^{c} (0) \; .
\ee     
Das Matrixelement der Divergenz des Axialstroms zwischen
Nukleonzust\"anden $|N(p)\rangle $ ist durch die axialen Formfaktoren des
Nukleons bestimmt:
\be
\langle N(p_2)| D^{a}(x) |N(p_1)\rangle  = \bar{u}(p_2) \left[ M G_A (t)
  + \frac{t}{4M} G_P(t) \right] \gamma_5 \tau^{a} u(p_1) \; .
\ee
Ber\"ucksichtigt man wie oben nur den Pionpolbeitrag,
so l\"a\ss t sich die Summe der beiden Kommutatoren 
an der Schwelle in die Form
\be
\label{picont}
 \epsilon^\mu T_\mu^{a\,(\pi)}  = eg_{\pi NN} \epsilon^{a3c} 
   \,\frac{\epsilon\cdot (k-2q)}{m_\pi^2 -t} \,
   \bar{u}(p_2)\gamma_5 \tau^{c} u(p_1)
\ee
bringen. 
Die Raumkomponenten des Sigmaterms $\Sigma_\mu^{a}$ enthalten die 
Information \"uber die explizite Brechung der chiralen Symmetrie 
durch die Quarkmassen in der QCD-Lagrangedichte. Ihre Bestimmung
ist jedoch modellabh\"angig und wird uns in den n\"achsten 
Abschnitten noch besch\"aftigen. Vernachl\"assigt man diese
Korrektur, so ergeben die oben diskutierten Beitr\"age 
(\ref{nborn2},\ref{kr},\ref{picont}) folgende Bestimmung der invarianten 
Amplituden       
\beq
\label{let1}
A^{(+0,-)}_1 &=&  \frac{2f}{\mu} \spm
      \left\{ -\frac{1}{\nu+\nu_1} \mp \frac{1}{\nu-\nu_1} 
      + \frac{1\mp 1}{\nu_1} \right\}\, , \\
A^{(+0,-)}_2 &=&  \frac{2f}{\mu} \spm
      \left\{ -\frac{2}{\nu+\nu_1} \pm \frac{2}{\nu-\nu_1} \right\}\, , \\    
A^{(+0,-)}_3 &=& \; \frac{2f}{\mu} \kappa \; (-1 \pm 1) \, ,  \\
A^{(+0,-)}_4 &=& \frac{2f}{\mu}\;\kappa
      \left\{ \frac{2}{\nu+\nu_1} \pm \frac{2}{\nu-\nu_1} \right\}\, , \\ 
A^{(+0,-)}_5 &=&  \frac{2f}{\mu}\;\kappa
      \left\{ \frac{4}{\nu+\nu_1} \mp \frac{4}{\nu-\nu_1} \right\}\, , \\
\label{let6}       
A^{(+0,-)}_6 &=&  \frac{2f}{\mu}(1+2\kappa)
      \left\{ \frac{1}{\nu+\nu_1} \mp \frac{1}{\nu-\nu_1} \right\} 
      + \frac{2f}{\mu} \kappa\, (1\pm 1) \; .
\eeq
Dabei haben wir der \"Ubersichtlichkeit halber den Isospinindex
der anomalen magnetischen Momente unterdr\"uckt. Es gilt
$\kappa^{(\pm)}=\kappa^v$ und $\kappa^{(0)}=\kappa^s$. 
Mit Hilfe der im ersten Kapitel abgeleiteten Formel f\"ur die
Schwellenamplitude
\be
 \left. E_{0+}\right|_{thr} = \frac{e}{16\pi M}
 \frac{2+\mu}{(1+\mu)^{3/2}} \, \left. \left(
   A_3 + \frac{\mu}{2} A_6 \right) \right|_{thr}
\ee                
und der in Anhang A diskutierten Kinematik
ergibt sich schlie\ss lich folgendes Resultat f\"ur die elektrische
Dipolamplitude in den vier physikalischen Kan\"alen 
\beq
\label{LET1}
\Epn &=& \frac{e}{4\pi} \frac{\sqrt{2}f}{m_\pi}
    \left\{ 1 - \frac{3}{2}\mu + {\cal O}(\mu^2) \right\}
    \cong 26.6 \su , \\[0.1cm]
\label{LET2}    
\Emp &=& \frac{e}{4\pi} \frac{\sqrt{2}f}{m_\pi}
     \left\{ -1 + \frac{1}{2}\mu + {\cal O}(\mu^2) \right\}
    \cong -31.7 \su , \\[0.1cm]
\label{LET3}    
\Eop &=& \frac{e}{4\pi} \frac{f}{m_\pi}
     \left\{ -\mu + \frac{\mu^2}{2}(3+\kappa_p ) +
  {\cal O}(\mu^3) \right\}    \cong -2.32
  \su , \\[0.1cm]
\label{LET4}  
\Eon &=& \frac{e}{4\pi} \frac{f}{m_\pi}
     \left\{  \frac{\mu^2}{2}\kappa_n  +
  {\cal O}(\mu^3) \right\}  \cong -0.51 \su ,
\eeq
wobei wir die Werte $\kappa_p=1.79$ und $\kappa_n=-1.91$ sowie
$f^2/(4\pi)=0.08$ verwendet 
haben. Dieses Resultat liefert den Inhalt des Niederenergietheorems
\cite{Bae70,VZ72}. Die relative Ordnung der nicht bestimmten
Korrekturen wurde mit Hilfe verschiedener Annahmen \"uber
das Verhalten der Hintergrundamplitude festgelegt. Wir werden
diese Annahmen und ihre Rechtfertigung im n\"achsten Abschnitt
diskutieren.

\begin{figure}
\label{feyn}
\caption{Feynman-Diagramme f\"ur die Bornterme in der 
Pionphotoproduktionsamplitude.}
\vspace{9cm}
\end{figure}

Niederenergietheoreme zur Pionphotoproduktion lassen sich auch
direkt aus der Bestimmung der Bornterme in effektiven 
chiralen Meson-Nukleon Theorien ableiten \cite{Pec69}. Die
entsprechende Lagrangedichte unter Einbeziehung der 
elektromagnetischen Wechselwirkung lautet
\beq
\label{leff}
{\cal L} & =& \bar{\psi}(i\gamma\cdot{\cal D}-M)\psi 
  +\frac{1}{2}({\cal D}_\mu\phi^{a})^2 - \frac{1}{2}m_\pi^2
  (\phi^{a})^2  \\
 & & \mbox{} + \frac{f}{m_\pi} \bar{\psi}\gamma_5 \gamma_\mu
 \tau^{a} {\cal D}^\mu \phi^{a}\psi 
  + \frac{e}{4M}\bar{\psi} (\kappa^s +\kappa^v \tau^3)
  \sigma_{\mu\nu}\psi F^{\mu\nu}\; , \nonumber  
\eeq
wobei ${\cal D}_\mu=\partial_\mu+iQ{\cal A}_\mu$ die kovariante
Ableitung, ${\cal A}_\mu$ das elektromagnetische Potential und
$Q$ den Ladungsoperator bezeichnet. Die Eichung der pseudovektoriellen
Pion-Nukleon-Kopplung erzeugt eine $\gamma\pi NN$-Kontaktwechselwirkung
\be
{\cal L}_{\gamma\pi NN} = \frac{ef}{m_\pi}\epsilon^{3ab}
  \bar{\psi}\gamma_5 \gamma_\mu \tau^{a}\psi {\cal A}^\mu \phi^b\; ,
\ee  
die im Rahmen der effektiven Theorie die Kroll-Ruderman-Amplitude
liefert. Die Wirkung der kovarianten Ableitung auf das Pionfeld
bestimmt die Kopplung des elektromagnetischen Feldes an die
geladenen Pionen. Die $\gamma\pi\pi$-Wechselwirkung 
\be  
{\cal L}_{\gamma\pi\pi} = e\epsilon^{3ab}\phi^{a}\partial_\mu
 \phi^{b} {\cal A}^\mu
\ee
liefert schlie\ss lich den Pionpol in der Photoproduktionsamplitude
f\"ur geladene Pionen. Wir haben die entsprechenden Diagramme in 
Abbildung 2.1 zusammengestellt. Zus\"atzlich sind dort die 
Beitr\"age von Resonanzen im s- und t-Kanal gezeigt, die 
wir in Abschnitt 2.7 diskutieren werden. 
   
    
\section{Absch\"atzung der vernachl\"assigten Beitr\"age}
Um die Modellabh\"angigkeit des im letzten Abschnitt
vorgestellten Niederenergietheorems zu studieren, ist es
hilfreich, die \"Ubergangsmatrix in der Form 
\be
 T_\mu^{a} = T_\mu^{a({\rm LET})} + \delta T_\mu^{a}
\ee
zu zerlegen. Dabei bezeichnet $T_\mu^{a(\rm LET)}$ die 
T-Matrix, die zu den invarianten Amplituden  (\ref{let1}-\ref{let6})
geh\"ort und $\delta T_\mu^{a}$ die vernachl\"assigte 
Hintergrundamplitude. Nach dem Kroll-Ruderman-Theorem gilt
\be
  \lim_{q,k\to 0} \delta T_\mu^{a} =0 \, ,
\ee
so da\ss\ $\delta T_\mu^{a}$ am ''weichen`` Punkt $q_\mu=0$ verschwindet.
Um zu untersuchen, in welcher Ordnung in der Pionmasse 
die Amplitude $\delta T_\mu^{a}$ 
Korrekturen zur elektrischen Dipolamplitude an der physikalischen Schwelle
liefert, definieren wir
\be
  \delta T_\mu^{a} = \bar{u}(p_2) \sum_{\lambda} 
   \delta A_\lambda^{a}(\nu,\nu_1) {\cal M}_\lambda u(p_1)
\ee
und nehmen an, da\ss\ sich die die invarianten Amplituden 
$\delta A_\lambda (\nu,\nu_1)$ in eine Taylorreihe um den Punkt
$\nu=\nu_1=0$ entwickeln lassen
\be
 \delta A_\lambda^{a} (\nu,\nu_1) = a^{a}_{\lambda\, 00}
    + a^{a}_{\lambda\, 10} \nu + a^{a}_{\lambda\, 01}\nu_1
    + \ldots \; .
\ee
Diese Voraussetzung ist gerechtfertigt, da lediglich die Pion- und 
Nukleonpolterme Singularit\"aten bei $\nu=0$ oder $\nu_1=0$ 
enthalten. Diese Terme haben wir aber explizit in $T_\mu^{a(
{\rm LET})}$ ber\"ucksichtigt. Dar\"uber hinaus wollen wir 
in den folgenden Betrachtungen voraussetzen, da\ss\ alle
Koeffizienten $a^{a}_{\lambda\, ij}$ im Limes $m_\pi\to 0$
regul\"ar sind. Das bedeutet, da\ss\ sich diese Koeffizienten
beim Abz\"ahlen von Potenzen in $\mu$ als Gr\"o\ss en der
Ordnung ${\cal O}(1)$ betrachten lassen. 

Diese Annahme ist vermutlich unzutreffend, denn Pion-Schleifendiagramme 
k\"onnen Bei\-tr\"a\-ge liefern, die nichtanalytisch in $m_\pi$ sind 
\cite{LP71,PP71}. Die Gegenwart solcher Terme ist von Bernard et 
al.~\cite{BKG91} durch 
eine explizite Rechnung im Rahmen der chiralen St\"orungstheorie best\"atigt
worden\footnote{Dagegen bestreitet Naus \cite{Nau91} auf Grund von
\"Uberlegungen allgemeiner Natur die Existenz 
nichtanalytischer Terme in der Pionphotoproduktionsamplitude.}.
Trotzdem ist es von Interesse, die Gr\"o\ss enordnung der analytischen
Beitr\"age in $\delta T_\mu^{a}$ zu studieren. 
 
Die Amplitude $T_\mu^{a(\rm LET)}$ enth\"alt im isospinsymmetrischen 
Fall neben den Nukleon- und Pion-Polen auch einen Kontaktterm. 
Die Gegenwart dieses Terms unterscheidet die Born\-am\-pli\-tuden in
pseudovektorieller bzw.~pseudoskalarer Kopplung und ist deshalb
eine Konsequenz der PCAC-Relation. Dieses Resultat l\"a\ss t sich
als eine Bedingung f\"ur die invariante Amplitude $A_6^{(+0)}$
formulieren \cite{AG66}
\be
\label{FFR}
 \lim_{\nu\to 0} \lim_{\nu_1\to 0} A_6^{(+0)} (\nu,\nu_1)
   =  \frac{4f}{\mu} \kappa^{v,s} \; .
\ee
Die Reihenfolge der beiden Grenz\"uberg\"ange in (\ref{FFR})
ist nicht beliebig. Sie ist so gew\"ahlt, da\ss\ der Polterm
keinen Beitrag zum Grenzwert liefert. 
Da der Kontaktterm (\ref{FFR}) bereits in $T_\mu^{a(\rm LET)}$
enthalten ist, verschwindet $a^{(+0)}_{6\,00}$ im Grenzfall $q_\mu
\to 0$. De Baenst \cite{Bae70} verwendet daher in seiner
Diskussion der nicht bestimmten Amplitude $\delta T_\mu^{a}$ die 
zus\"atzliche Annahme $a^{(+0)}_{6\,00}=0$. 
          
Nur $\delta A_3$ und $\delta A_6$ tragen zur elektrischen 
Dipolamplitude an der Schwelle bei. Mit Hilfe der 
Eichinvarianzbedingung (\ref{gaugecond}) und der Forderung nach korrektem
Verhalten der Amplituden unter der Austauschtransformation
$(\nu,\nu_1)\to(-\nu,\nu_1)$ l\"a\ss t sich die m\"ogliche
Form der Taylorentwicklungen f\"ur $\delta A_{3,6}$ erheblich
einschr\"anken. F\"ur die isospinsymmetrischen Komponenten
findet man
\beq
 \delta A_{3}^{(+0)} &=& a_{3\, 11}^{(+0)} \nu\nu_1 + \ldots \; ,\\
 \delta A_{6}^{(+0)} &=& a_{6\, 01}^{(+0)} \nu_1
               + a_{6\, 20}^{(+0)} \nu^2
	       + a_{6\, 02}^{(+0)} \nu_1^2 + \ldots \; .
\eeq
An der Schwelle ist $\nu={\cal O}(\mu)$ und $\nu_1={\cal O}(\mu^2)$,
so da\ss\ die Austauschsymmetrie im wesentlichen das 
Transformationsverhalten der Amplitude unter $m_\pi\to -m_\pi$
spezifiziert. Auf Grund der Beziehung 
\be
\delta E_{0+} \sim \delta A_3 + \frac{\mu}{2} \delta A_6
\ee
folgt, da\ss\ die Hintergrundamplitude $\delta E_{0+}^{(+0)}$
an der Schwelle von der Ordnung $\mu^3$ ist. Verwendet man
an Stelle der Annahme $a_{6\,00}^{(+0)}=0$ die Absch\"atzung 
$a_{6\,00}^{(+0)}={\cal O}(\mu)$, so ergibt sich das schw\"achere 
Resultat $\delta E_{0+}^{(+0)}={\cal O}(\mu^2)$. Eine analoge
Argumentation l\"a\ss t sich auch f\"ur die isospinungeraden Komponenten 
durchf\"uhren. In diesem Fall findet man $\delta E_{0+}^{(-)}= 
{\cal O}(\mu^2)$.

\section{Die Methode von Furlan, Paver und Verzegnassi}
Die Ableitung des Niederenergietheorems im  Abschnitt 2.2
basierte im wesentlichen auf der Reduktionsformel und auf 
Wardidentit\"aten f\"ur die Zweipunktfunktion $\overline{\Pi}_{\mu\nu}^{a}$.
In diesem Abschnitt wollen wir auf eine andere Methode eingehen, die
direkt mit Ladungskommutatoren und Vollst\"andigkeitsrelationen
arbeitet. Im Rahmen dieses Verfahrens wurde erstmals darauf hingewiesen, 
da\ss\ die explizite Brechung der chiralen Symmetrie Korrekturen an 
das Standard-Niederenergietheorem liefern kann \cite{FPV74,NS89}.

Allerdings wird in der \"ublichen Diskussion der Methode nur der
Nukleonbeitrag in der Vollst\"andigkeitssumme explizit ber\"ucksichtigt.
In diesem Fall fehlt der f\"uhrende Beitrag zur Photoproduktion 
neutraler Pionen, und man mu\ss\ nachtr\"aglich Eichinvarianz 
erzwingen, um die korrekte Schwellenamplitude zu reproduzieren.
 
In diesem Abschnitt wollen wir demonstrieren, da\ss\ unter Einbeziehung
der Beitr\"age von Antinukleonen im Zwischenzustand auch das Verfahren
von Furlan et al.~die komplette Schwellenamplitude liefert. Zu diesem
Zweck betrachten wir die zu dem Strom $B_\mu^{a}$ geh\"orende Ladung
\be
 \overline{Q}^{a}_5(t) = Q^{a}_5(t) +  \frac{1}{m_\pi^2}\,
 \frac{d}{dt} \, \int d^3x\, D^{a} (\vec{x},t)\, .
\ee
Zwischen physikalischen Zust\"anden reduziert sich dieser Operator
auf die Ladung $\qfl$:
\beq
\label{q5l}
 \qfl (t) &=& Q_5^{a}(t) +\frac{i}{m_\pi}\dot{Q}^{a}_5(t)\, , \\
\label{q5r} 
 \qfr (t) &=& \left( \qfl (t)\right)^\dagger 
                =  Q_5^{a}(t) -\frac{i}{m_\pi}\dot{Q}^{a}_5(t)  \, .
\eeq
Den zugeh\"origen hermitesch konjugierten Operator haben wir mit
$\qfr$ bezeichnet. Die  Pionmatrixelemente dieser Operatoren lauten:  
\beq
  \langle\, 0\,|\,\qfl |\pi^{b}(q)\rangle  &=& \spm 2if_\pi m_\pi \,\delta^{ab}
                           (2\pi)^3  \delta^3 (\vec{q}\,)\, , \\[0.2cm]  
  \langle \pi^{b}(q)|\,\qfr |\,0\,\rangle  &=& -2if_\pi m_\pi \,\delta^{ab}
                            (2\pi)^3 \delta^3 (\vec{q}\,)\, , \\[0.2cm]
  \langle\pi^b(q)|\,\qfl |\,0\,\rangle &=& \langle\,0\,|\,\qfr |
  \pi^{b}(q)\rangle  = 0 \; .
\eeq
Die axialen Ladungen $Q_{5\,{\mini L,R}}^{a}$ sind nicht hermitesch
und haben die Eigenschaft, zwischen Pionen im Eingangs- und Ausgangskanal 
zu unterscheiden. Diese Tatsache erweist sich als besonders n\"utzlich 
bei der  Konstruktion von Summenregeln, da sie es erm\"oglicht,
bestimmte Prozesse in der Vollst\"andigkeitssumme zu selektieren. 
Im folgenden betrachten wir Summenregeln f\"ur das Matrixelement 
\be
 M_\mu^{a} = \langle N(p_2)| [\qfl ,V_\mu^{em}(0)]|N(p_1)\rangle \, ,
\ee
in dem sich mit Hilfe der oben angegebenen Matrixelemente die 
Photoproduktionsamplitude identifizieren l\"a\ss t. Das Resultat
besitzt eine sehr \"ubersichtliche Struktur als Summe von Poltermen 
und einem Dispersionsintegral, das die Hintergrundamplitude 
repr\"asentiert. 

Da der Operator $\qfl$ nur Pionen in Ruhe produziert, 
verlangt die Berechnung der Schwellenamplitude die Kenntnis
von $M_\mu^{a}$ im Schwerpunktsystem. Um die folgende Rechnung
etwas zu vereinfachen, werden wir $M_\mu^{a}$ statt dessen an
der Breitschwelle berechnen, das hei\ss t f\"ur Pionen, die
im Breitsystem des Nukleons ruhen\footnote{An der physikalischen Schwelle 
$\sqrt s=M+m_\pi$ ruht das Pion im Schwerpunktsystem. Der Impuls des Pions 
im Breitsystem des Nukleons ist dann tats\"achlich sehr klein, $|\vec{q}\,|
=\frac{1}{4}m_\pi\mu^2+{\cal O}(m_\pi^4)$.}:
\be
\begin{array}{rclcrcl}
  \vec{p}_2 &=&\spm \vec{p}, &\hspace{1cm} & \vec{q} &=& 0, \\[0.2cm]
  \vec{p}_1 &=&-\vec{p}    , &\hspace{1cm} & \vec{k} &=& 2\vec{p}.
\end{array}
\ee
Die eine Seite der Summenregel f\"ur $M_\mu^{a}$ ergibt sich, indem 
man das Matrixelement
\beq
\label{comqfl}
\lefteqn{\langle N(\vec p\,)|[\qfl,V_\mu^{em}(0)] |N(-\vec p\,)\rangle \;\;= } \\
&\hspace{1.0cm} & \langle  N(\vec p\,)|[Q_5^{a},V_\mu^{em}(0)]|N(-\vec p)\rangle 
 +\frac{i}{m_\pi}\langle N(\vec p\,)| [\dot{Q}_5^{a},V_\mu^{em}(0)]
 |N(-\vec p\,)\rangle  \nonumber 
\eeq
direkt auswertet. Dabei findet man den  Kroll-Ruderman-Term sowie
die bereits diskutierten Beitr\"age der expliziten Symmetriebrechung. Die 
andere Seite der Summenregel ergibt sich aus der Vollst\"andigkeitsrelation 
f\"ur den Kommutator (\ref{comqfl}). Die Clusterzerlegung \cite{AFF73} 
ist eine systematische Methode, um die verschiedenen Beitr\"age zur 
Vollst\"andigkeitssumme
\beq
\sum_n \langle N(\vec{p}\,)|\qfl |n\rangle \langle n|V_\mu^{em}|N(-\vec{p}\,)\rangle 
\eeq
zu identifizieren. Sie tr\"agt der Tatsache Rechnung, da\ss\
in einer relativistischen Theorie Beitr\"age mit unterschiedlichen 
Teilchenzahlen auftreten k\"onnen. Konkret zerlegt man $M_\mu^{a}$ 
in der Form    
\begin{figure}
\label{diag}
\caption{Beitr\"age zur Vollst\"andigkeitssumme f\"ur das
Operatorprodukt $V_\mu^{em}\qfl$.}
\vspace{9cm}
\end{figure}
\be
\label{cluster}
M_\mu^{a\;\;}  = M_\mu^{a\,I}+M_\mu^{a\,II} \; ,
\ee
\newpage
\beq
M_\mu^{a\,I\,} &=& \sum_\alpha \langle N(\vec{p})|\qfl |\alpha\rangle _c
                        \langle \alpha|V_\mu^{em}|N(-\vec{p}\,)\rangle _c \\   
 & &  \hspace{0.5cm} -  \sum_\beta \langle 0|\qfl |N(-\vec{p}\,)\beta\rangle 
 \langle N(\vec{p}\,)\beta|V_\mu^{em}|0\rangle \; +\; {\em c.~t.}\; ,\nonumber \\
M_\mu^{a\,II} &=& \sum_{\gamma_1} \langle N(\vec{p}\,)|\qfl |N(-\vec{p}\,)\gamma_1\rangle _c
                             \langle \gamma_1|V_\mu^{em}|0\rangle  \\   
 & &       \hspace{0.5cm} +  \sum_{\gamma_2} \langle 0|\qfl |\gamma_2\rangle 
       \langle \gamma_2 N(\vec{p}\,)|V_\mu^{em}|N(-\vec{p}\,)\rangle \; +\; 
       {\em c.~t.}\; , \nonumber
\eeq
wobei der Index $c$  den zusammenh\"angenden Teil des Matrixelements
und $c.t.$  die Voll\-st\"an\-dig\-keitssumme mit den Operatoren in der
anderen Reihenfolge bezeichnet. Wir haben die verschiedenen Terme 
schematisch in Abbildung 2.2 dargestellt. Der erste Teil der 
Clusterzerlegung beinhaltet solche Zust\"ande, die Baryonenzahl tragen. 
Die f\"uhrenden Terme in diesem Beitrag stammen von Nukleonen
$|\alpha\rangle \,=|N(\vec{p}\,)\rangle $ und Antinukleonen $|\beta\rangle 
\,=|\bar{N}(\vec{p}\,)\rangle $. Der zweite Teil von (\ref{cluster}) 
beschreibt die Produktion
eines Zustands $\gamma_{1,2}$ aus dem Vakuum, gefolgt von der Reaktion
$\gamma_1 +N(p_1) \to \qfl + N(p_2)$ bzw.~ $V_\mu^{em}+N(p_1)
\to \gamma_2 + N(p_2)$. Insbesondere findet man f\"ur Pionzust\"ande
$|\gamma_2\rangle \,=|\pi^{a}(\vec{q}\,)\rangle $ die Photoproduktionsamplitude
\be
\sum_{\pi^{b}(\vec{q})} \langle 0|\qfl |\pi^{b}(\vec{q}\,)\rangle \langle \pi^{b}(\vec{q}\,)
  N(\vec{p}\,)|V_\mu^{em}|N(-\vec{p}\,)\rangle \; = f_\pi S_\mu^{a}(\vec{q}=0)\, .
\ee
Auf Grund der speziellen Eigenschaften des Operators  $\qfl$
enth\"alt die Vollst\"andigkeitssumme keine Beitr\"age von der
inversen Reaktion $\pi^{a}(q)+N(p_1)\to V_\mu^{em}+N(p_2)$. Isoliert
man die Photoproduktionsamplitude $S_\mu^{a}$ und separiert 
die Nukleonbeitr"age in $M_{\mu}^{a\, I}$, so ergibt sich 
schlie\ss lich folgender Ausdruck f\"ur $S_\mu^{a}$
\beq
\label{fpv}
f_{\pi} S_{\mu}^{a}(\vec{q}=0) &=&
 i \epsilon^{a3c} \langle N(\vec{p}\,)|A_{\mu}^{c}|N(-\vec{p}\,)\rangle  
                   \\[0.3cm]
   & &\mbox{}-\sum_{N(\vec{p}_n)} \langle N(\vec{p}\,)|\qfl |N(\vec{p}_n)\rangle 
   \langle N(\vec{p}_n)|V_{\mu}^{em}|N(-\vec{p}\,)\rangle  
           \;+ \; {\em c.~t.} \nonumber \\
   & &\mbox{}  + \frac{i}{m_\pi}\langle N(\vec{p}\,)|[\dot{Q}_5^{a},V_{\mu}^{em}]
    |N(-\vec{p}\,)\rangle  
    \; + \;  f_\pi \delta S_\mu^{a}  \nonumber ,
\eeq
wobei $\delta S_\mu^{a}$ die vernachl\"assigten Beitr\"age in der
Vollst\"andigkeitssumme bezeichnet. Dabei handelt es sich vor allem um
$\pi N$-Kontinuumszust\"ande und Antinukleonen in $M_\mu^{a\, I}$, 
sowie Vektormesonen in $M_\mu^{a\, II}$.
Mit Hilfe der Eigenschaften des Operators $\qfl$ l\"a\ss t sich folgende
Darstellung der Hintergrundamplitude ableiten \cite{AFF73}
\be
\label{ressum}
\delta S_\mu^{a} = -im_\pi \sum_{n\neq\pi,N} (2\pi)^3 \delta^3 
  (\vec{p}-\vec{p}_n) 
  \frac{ \langle N(\vec{p}\,)|j_\pi^{a}|\,n\,\rangle 
  \langle \,n\,|V_\mu^{em}|N(-\vec{p}\,)\rangle  }{ 
       (E_p-E_n)(E_p+m_\pi-E_n+i\epsilon) }
  \;-\; c.t. \; . 
\ee
Die Summation l\"auft \"uber beliebige intermedi\"are Zust\"ande 
mit Ausnahme von Nukleonen und Pionen. $E_p=(\vec{p}^{\,2}+M^2)^{1/2}$
bezeichnet die Energie des auslaufenden Nukleons, $E_n$ die Energie
des Zwischenzustands. Der Energienenner in (\ref{ressum}) verschwindet
nur f\"ur Nukleonzust\"ande, so da\ss\ alle anderen Beitr\"age im Limes 
$m_\pi \to 0$ unterdr\"uckt sind. 

Wir betrachten nun im einzelnen die verschiedenen Beitr\"age zur
Photoproduktionsamplitude (\ref{fpv}). Den Kroll-Ruderman-Term 
sowie den Sigmakommutator haben wir bereits in Abschnitt 2.2 
diskutiert. Um den Nukleonterm zu berechnen, ben\"otigen wir die
Matrixelemente 
\beq
  \langle N(\vec{p}\,)|A_0^{a}(0)|\,N(\vec{p}\,)\,\rangle\,  &=&
     \frac{g_A}{M} \,\chi^\dagger_f (\vec{\sigma}\cdot\vec{p}\,)
     \frac{\tau^{a}}{2} \chi_i\, ,  \\  
 \langle N(\vec{p}\,)|V_0^{em}(0)|N(-\vec{p}\,)\rangle  &=&
     e \,\chi^\dagger_f (G_E^s (t) + \tau^3 G_E^v (t) ) \chi_i \, ,\\[0.1cm]
 \langle N(\vec{p}\,)|\vec{V}^{em}(0)|N(-\vec{p}\,)\rangle  &=&
     \frac{e}{M} \,\chi^\dagger_f (G_M^s(t) + \tau^3 G_M^v(t))  
     i(\vec{\sigma}\times\vec{p}) \chi_i \, .
\eeq     
Im Breitsystem treten die elektrischen und magnetischen Formfaktoren
des Nukleons,
\beq
  G_E(t) &=& F_1(t)+\frac{t}{4M^2}F_2(t) \, , \\[0.1cm]
  G_M(t) &=& F_1(t)+F_2(t)\, ,
\eeq
auf. Da wir bereits ein spezielles Bezugssystem gew\"ahlt haben,
ist es sinnvoll, die $T$-Matrix in einer nicht kovarianten Form
anzugeben. Der Nukleonbeitrag zur Photoproduktionsamplitude an  
der Breitschwelle $t=m_\pi^2$ ergibt sich schlie\ss lich zu
\beq
   T_0^{a}  &=& \spm i\frac{g_A}{f_\pi}\,\frac{1}{E_p}
        \chi^\dagger_f (G_E^s(t) \tau^{a} + G_E^v \delta^{a3})
	(\vec{\sigma}\cdot\vec{p}\,)\chi_i \, , \\
\vec{T}^{a} &=& -i \frac{g_A}{f_\pi} \, \frac{\vec{p}^{\, 2}}{mE_p}
       G_M^v(t) \,\chi^\dagger_f \frac{1}{4} [\tau^{a},\tau^3] 
       \,\vec{\sigma}_{\mini T}\, \chi_i\, ,
\eeq
wobei wir die transversalen und longitudinalen Spins 
\beq
   \vec{\sigma}_{\mini T} &=& \vec{\sigma} - \hat{p}(\vec{\sigma}
              \cdot\hat{p}), \\
   \vec{\sigma}_{\mini L} &=&  \hat{p}(\vec{\sigma}\cdot\hat{p})	      
\eeq
eingef\"uhrt haben. Nur der transversale Anteil liefert einen Beitrag
zur Photoproduktion  mit reellen Photonen. Im Falle des Nukleonterms
ist dieser Beitrag von der Ordnung $m_\pi^2$ und proportional zum
magnetischen Moment des Nukleons.   	             
      
Als pseudoskalares Teilchen koppelt das Pion stark an die unteren
Komponenten der Nukleonspinoren. Der f\"uhrende Beitrag zur Produktion
neutraler Pionen kommt daher von Zwischenzust\"anden, die propagierende 
Antinukleonen enthalten ('Z-Graphen`). Mit Hilfe der Darstellung
(\ref{ressum}) findet man
\be
 \vec{T}^{a} = i\frac{g_A}{f_\pi} \frac{m_\pi}{2E_p} \,
     \chi^\dagger_f (G_E^s(t)\tau^{a} + G_E^v(t) \delta^{a3})
     \vec{\sigma} \chi_i\; ,
\ee
wobei wir h\"ohere Ordnungen in $m_\pi/M$ vernachl\"assigt haben.
Damit ist die elektrische Dipolamplitude bis zur Ordnung $m_\pi^2$
bestimmt. Vernachl\"assigt man den Beitrag aus der expliziten
chiralen Symmetriebrechung, so ergibt sich
\beq
\label{LETa1}
\Epn &=& \frac{e}{4\pi} \frac{g_A}{\sqrt{2}f_\pi}
    \left\{ 1 - \frac{3}{2}\mu + {\cal O}(\mu^2) \right\}
    \cong 24.1  \su , \\
\Emp &=& \frac{e}{4\pi} \frac{g_A}{\sqrt{2}f_\pi}
     \left\{ -1 + \frac{1}{2}\mu + {\cal O}(\mu^2) \right\}
    \cong -29.6  \su  ,\\
\Eop &=& \frac{e}{4\pi} \,\frac{g_A}{2f_\pi}\;
     \bigg\{ -\mu + {\cal O}(\mu^2) \bigg\}  \cong -3.3 \su . 
\eeq
Die elektrische Dipolamplitude f\"ur die Produktion neutraler
Pionen am Neutron verschwindet in dieser Ordnung. Die von
(\ref{LET1},\ref{LET2}) abweichenden Werte in den geladenen Kan\"alen 
sind eine Konsequenz der Tatsache, da\ss\ die 
Goldberger-Treiman-Relation $\frac{g_A}{2f_\pi}=\frac{f}{m_\pi}$
experimentell um ca.~6\% verletzt ist. Diese Abweichung ist 
formal von der Ordnung $m_\pi^2$ und entspricht daher der 
oben vorgenommenen Absch\"atzung. 

\section{Explizite chirale Symmetriebrechung}
Die r\"aumlichen Komponenten des Beitrags aus der expliziten
chiralen Symmetriebrechung 
\be
\label{csbcom}
 \Sigma_\mu^{a}(\vec{q}=0) = 
  \int d^4 x \,\delta (x^0)\,\langle N(p_2)| [\partial^\nu A_\nu^{a}(x),
  V_\mu^{em}(0)] |N(p_1)\rangle
\ee
sind nicht durch Stromalgebra festgelegt. Aus diesem Grund haben 
wir ihren Beitrag zur Photoproduktionsamplitude bislang 
vernachl\"assigt. Repr\"asentiert man jedoch die Str\"ome
durch Quarkfelder, so ist auch dieser Kommutator durch die
kanonischen Vertauschungsregeln der Felder bestimmt. 
Der Vollst\"andigkeit halber arbeiten wir in Flavor-$SU(3)$,
so da\ss\
\beq
   \partial^\nu A_\nu^{a} &=& \frac{i}{2} \bar{\psi} \gamma_5
      \left\{ M,\lambda^{a} \right\} \psi\; ,  \\
    V_\mu^{em}            &=& \frac{1}{2} \bar{\psi} \gamma_\mu
      ( \lambda^3 + \frac{1}{\sqrt{3}} \lambda^8 ) \psi
\eeq
mit $M={\rm diag}(m_u,m_d,m_s)$. Es wird sich allerdings zeigen, da\ss\
die Masse der seltsamen Quarks nicht in das Resultat eingeht.
Der Kommutator der beiden Bilinearformen l\"a\ss t sich mit Hilfe 
der Relation              
\beq
\label{bilcom}
 \lefteqn{\delta (x^0-y^0) [\psi^\dagger (y)\frac{\lambda^{a}}{2}
      \Gamma \psi (y),\psi^\dagger (x)\frac{\lambda^{b}}{2}
      \Gamma' \psi (x)] = }  \\
    & & \hspace{1cm}   \frac{1}{2} \delta^4 (x-y) 
      \psi^\dagger (x) \left( if^{abc} \{\Gamma,\Gamma'\} 
      + d^{abc} [\Gamma,\Gamma' ] \right) \frac{\lambda^c}{2}
      \psi (x) \nonumber
\eeq
auswerten. Dabei bezeichnen $\Gamma$ und $\Gamma'$ die Diracoperatoren,
$f^{abc}$ und $d^{abc}$ die antisymmetrischen bzw.~symmetrischen 
$SU(3)$-Strukturkonstanten. Mit Hilfe von (\ref{bilcom}) ergibt sich f\"ur
$a=1,2,3$
\beq
\label{sig0q}
\int d^4x\, \delta (x^0) [\partial^{\nu}A_{\nu}^{a}(x),
   V_{0}^{em}(0)] &=& \,\,\overline{m} \,\epsilon^{3ab} \bar{\psi}
   \gamma_5\lambda^b \psi ,  \\  
\label{sigcom}
\int d^4 x\, \delta (x^0) [\partial^{\nu}A_{\nu}^{a}(x),
        V_{i}^{em}(0)] &=&
i\,\overline{m} \,\epsilon_{ijk} \left\{  \delta^{a3}
\frac{1}{\sqrt{3}}\left( \sqrt{2} J_{jk}^{0}+J_{jk}^{8} \right) +
 \frac{1}{3} J_{jk}^{a}\right\}    \\
& &\mbox{} + i\frac{\delta m}{2}\epsilon_{ijk}\delta^{a3} \left\{
 \frac{1}{3\sqrt{3}}\left(\sqrt{2} J_{jk}^{0}+ J_{jk}^{8} \right)
  + J_{jk}^{3}  \right\} , \nonumber
\eeq
wobei wir die Tensorstr\"ome
\be
 J_{\mu\nu}^{a} = \bar{\psi}\sigma_{\mu\nu}\frac{\lambda^a}{2}\psi
\ee
eingef\"uhrt haben. In der von uns verwendeten Normierung ist
$\lambda^0=\sqrt{2/3}\,1\!\! 1$. Man beachte, da\ss\ $(1/\sqrt{3})
(\sqrt{2}\lambda^0 +\lambda^8)$ gerade die Einheitsmatrix im 
$SU(2)$-Unterraum ist. Die Str\"ome (\ref{sig0q},\ref{sigcom})
enthalten daher keine Beitr\"age der seltsamen Quarks. Die St\"arke
der chiralen Symmetriebrechung sowie der Isospinbrechung wird
durch die Parameter
\beq
  \overline{m} &=& \frac{m_u+m_d}{2}\; ,  \\
  \delta m     &=& m_u -m_d
\eeq
kontrolliert. Die Zeitkomponente des Sigmakommutators ist bis auf
Korrekturen aus der Isospinbrechung durch die Divergenz des
Axialstroms gegeben.  Das Ergebnis (\ref{sig0q}) reproduziert 
daher das in Abschnitt 2.2 diskutierte Stromalgebraresultat (\ref{sig0}). 

Die r\"aumlichen Komponenten des Kommutators (\ref{csbcom}) 
liefern dagegen einen zus\"atzlichen Beitrag zur Photoproduktionsamplitude.
Dieser Beitrag verschwindet am ''weichen`` Punkt $q_\mu=0$ und 
bestimmt daher die Extrapolation der Amplitude zur
physikalischen Schwelle.

Die Berechnung dieses Terms beruht allein auf kanonischen 
Vertauschungsrelationen, die durch die Wechselwirkung der Quarks nicht 
modifiziert werden. In der quantisierten Theorie besteht allerdings
die Notwendigkeit, das Produkt der Str\"ome zu regularisieren. 
Shei und Tsao \cite{ST77} haben darauf hingewiesen, da\ss\ diese
Tatsache zu anomalen Beitr\"agen in den Vertauschungsrelationen 
f\"uhren kann. Das oben angegebene Resultat (\ref{sigcom}) beruht
daher strenggenommen auf einem freien Quarkmodell.

Ein weiteres Modell, in dem sich der Kommutator (\ref{csbcom}) 
bestimmen l\"a\ss t, ist das lineare $\sigma$-Modell. Dieses
Modell liefert die Str\"ome
\beq
  \partial^\nu A_\nu^{a} &=& f_\pi m_\pi^2 \pi^{a} \, ,\\
  V_\mu^{em} &=& \frac{1}{2}\bar\Psi (1+\tau_3)\gamma_\mu\Psi
    + \epsilon^{3ab}\pi^{a}\partial_\mu\pi^{b}   \, ,
\eeq      
wobei $\Psi$ einen $SU(2)$-Nukleonspinor und $\pi^{a}$ das
Triplet der Pionfelder bezeichnet. Im linearen $\sigma$-Modell
postuliert man kanonische Vertauschungsrelationen f\"ur das
Pionfeld $\pi^{a}$. In diesem Fall ergibt sich $\vec{\Sigma}^{a}
=0$, da der elektromagnetische Strom $\vec{V}^{em}$ nicht das 
zu $\pi^{a}$ konjugierte Feld $\dot\pi^{a}$ enth\"alt. 

Die QCD liefert also eine kompliziertere Form der Symmetriebrechung
als die Meson-Nukleon-Lagrangedichte des linearen $\sigma$-Modells.
Um Matrixelemente des Kommutators (\ref{sigcom}) zu studieren, 
f\"uhren wir Formfaktoren f\"ur die Tensorstr\"ome $J_{\mu\nu}^{a}$
ein \cite{FPV74,MS76}
\beq
  \langle N(p_2)|\bar{\psi}\sigma_{\mu\nu}\tau^{a}\psi|N(p_1)\rangle  &=& 
        \bar{u}(p_2) \left[
     G_T^{a}(t) \sigma_{\mu\nu} + iG_2^{a}(t)
     \frac{\gamma_\mu \Delta_\nu - \Delta_\mu \gamma_\nu}{2M} 
     \right. \\
 & & \mbox{}+ \left. iG_3^{a}(t) 
     \frac{\Delta_\mu P_\nu - P_\mu \Delta_\nu}{M^2}
     + iG_4^{a} \frac{\gamma_\mu P_\nu - P_\mu \gamma_\nu}{M^2}    
     \right] \tau^{a} u(p_1) \nonumber \, ,
\eeq
wobei $\Delta_\mu=(p_2-p_1)_\mu$ den Impuls\"ubertrag und $\tau^0 ={\bf 1}$ 
sowie $\tau^{a}\; (a=1,2,3)$ die Paulimatrizen bezeichnet.
Das Matrixelement vereinfacht sich erheblich, wenn man die
r\"aumlichen Komponenten der Str\"ome im Breitsystem des Nukleons
betrachtet
\beq
   \langle N(\vec{p}\,)|\bar{\psi}\sigma_{jk}\tau^a\psi|N(-\vec{p}\,)\rangle     
  & = & \epsilon_{jkm}\chi^\dagger_f \left[
     \left( G_T^{a}(t) +\frac{t}{4M^2} G_2^{a}(t)\right)\sigma_{{\mini T}m} 
    \right. \\
 & & \hspace{2.7cm} \mbox{} + \left. G_T^{a}(t)\frac{E_p}{M} 
 \sigma_{{\mini L}m} \right] \tau^{a} \chi_i \; . \nonumber
\eeq
Bis auf Korrekturen der Gr\"o\ss enordnung $m_\pi^2$ kann man die
Formfaktoren durch ihren Wert bei $t=0$ ersetzen.
Mit der Definition $g_T^{a}=G_T^{a}(0)$ ergibt sich schlie\ss lich  
folgende Korrektur zur Schwellenamplitude f\"ur neutrale Pionen 
\be
\label{delneu}
\Delta E_{0+}(\pi^0 N) = \frac{e}{4\pi f_\pi}\frac{\overline{m}}{m_\pi (1+\mu)}
  \left\{ \left( 1+\frac{\delta m}{6\overline{m}} \right) g_T^0
     \pm \left(\frac{1}{3}+\frac{\delta m}{2\overline{m}}\right) g_T^3
     \right\} \; ,
\ee
wobei sich die unterschiedlichen  Vorzeichen auf die Produktion am Proton 
bzw.~Neutron beziehen. Verwendet man die oben zitierten  Werte der
Quarkmassen, so ist $\delta m/(2\overline{m}) \simeq -1/3$, und
$\Delta E_{0+}(\pi^0N)$ ist fast vollst\"andig durch die Tensorkopplung
im Singletkanal bestimmt. Man beachte, da\ss\ der Korrekturterm formal
von der Ordnung $m_\pi$ ist, denn nach der GOR-Relation gilt $\overline{m}
= m_\pi^2f_\pi^2/|\langle \bar{u}u+\bar{d}d\rangle |$. Dagegen ergibt sich 
in der chiralen St\"orungstheorie kein zus\"atzlicher Beitrag in dieser 
Ordnung in $m_\pi$ \cite{BKG91}. Diese Tatsache ist konsistent mit der oben 
gemachten Feststellung, da\ss\ $\vec{\Sigma}^{a}$ in einer 
Meson-Nukleon-Theorie verschwindet.

Die entsprechende Korrektur f\"ur die Produktion geladener Pionen
lautet
\be
\label{delchar}
 \Delta E_{0+}(\pi^-p)=\Delta E_{0+}(\pi^+n) =
  \frac{\sqrt{2}e}{4\pi f_\pi}\frac{\overline{m}}{m_\pi (1+\mu)}
  \,\frac{g_T^3}{3}\; .
\ee  
In diesem Fall tr\"agt der isospinbrechende Term proportional
zu $\delta m$ nicht bei. Der Korrekturterm modifiziert nicht
die Ladungsasymmetrie $|\Epn|-|\Emp|$, liefert aber einen
kleinen Beitrag zum Panofskyverh\"altnis $\Epn/\Emp$.

Die wesentliche Aufgabe bei der Berechnung von $\Delta E_{0+}$
ist nun die Bestimmung der Tensorkopplungen $g_T^{a}$ des Nukleons. 
Diese sind experimentell nicht direkt zug\"anglich, so da\ss\ man 
in diesem  Zusammenhang auf Modelle angewiesen bleibt.
Die einfachste M\"oglichkeit ist die Verwendung eines
nichtrelativistischen Konstituentenmodells zur 
Beschreibung der Struktur des Nukleons. In diesem Fall
reduzieren sich die Tensorstr\"ome $\frac{1}{2}\epsilon_{ijk}
\bar{\psi}\sigma_{jk}\psi$ auf Axialstr\"ome $\bar{\psi}
\gamma_i\gamma_5 \psi$. Deren Nukleonmatrixelemente sind
durch die axialen Kopplungen
\be
 \langle N(p_2)|\bar\psi\gamma_\mu\gamma_5\tau^{a}\psi|N(p_1)\rangle =g_A^{a}\bar u(p_2)
  \gamma_\mu\gamma_5\tau^{a}u(p_1) + \ldots
\ee
bestimmt. Die Korrektur zur elektrischen Dipolamplitude lautet dann
\be
 \DEop = \frac{e}{4\pi f_\pi}\frac{\overline{m}}{m_\pi (1+\mu)}
    (0.90 \cdot g_A^0 + 0.04 \cdot g_A^3) \; .
\ee
In einem nichtrelativistischen Quarkmodell findet man $g_A^0=1$ und 
$g_A^3=5/3$, so da\ss\ $\DEop = 1.6 \su$. Diese Korrektur reduziert
die elektrische Dipolamplitude an der Schwelle auf den Wert 
$E_{0+}(\pi^0p)=-0.7\su$, in \"Ubereinstimmung mit den ersten 
Analysen des MAMI A Experiments \cite{NS89,TD90}.  

Allerdings bricht das verwendete nichtrelativistische Quarkmodell 
die chirale Symmetrie bereits im Ansatz, so da\ss\ die oben 
vorgenommene Absch\"atzung von $\Delta E_{0+}$ mit Vorsicht zu
betrachten ist. Wir werden eine sorgf\"altigere Bestimmung dieser
Korrektur in Kapitel 4 in Angriff nehmen. 

\section{Eichinvarianz}
Die Forderung nach Eichinvarianz der \"Ubergangsmatrix $T_\mu^{a}$
liefert wichtige Einschr\"ankungen f\"ur die Form der invarianten Amplituden.
Im Falle von Pionen auf der Massenschale ergeben sich diese 
Bedingungen aus der Erhaltung des elektromagnetischen Stroms
im \"Ubergangsmatrixelement
\be
\label{ongi}
k^\mu T_\mu^{a} = ie\langle \pi^{a}(q)N(p_2)|\partial^\mu V_\mu^{em}(0)|N(p_1)\rangle 
=0 \; .
\ee
Die aus dieser Gleichung folgenden Beziehungen (\ref{gaugecond}) haben 
wir bereits im ersten Kapitel angegeben. Man pr\"uft leicht nach, da\ss\ 
die in Abschnitt 2.2 abgeleiteten Amplituden diese Bedingungen erf\"ullen.
Diese Feststellung gilt jedoch nur f\"ur die Summe von Stromalgebra-, 
Nukleon- und Pionpolbeitr\"agen. Keiner dieser Terme ist f\"ur sich genommen 
eichinvariant. 

%Die nicht eichinvarianten Terme in den einzelnen Beitr\"agen heben
%sich allerdings nur dann gegenseitig weg, wenn keine ph\"anomenologischen 
%Formfaktoren an den Vertices verwendet werden. Um die Rolle
%der Formfaktoren n\"aher zu untersuchen, wollen wir unsere 
%Betrachtungen auf die Elektroproduktion von Pionen erweitern.
%In diesem Fall ist das ausgetauschte Photon virtuell und besitzt 
%eine nicht verschwindende invariante Masse $k^2$.  Die Kopplung
%des Photons wird durch die elektrischen Formfaktoren des 
%Nukleons sowie des Pions 
%\beq
% \Gamma_\mu^\gamma &=& F_1(k^2) \gamma_\mu + \frac{i\sigma_{\mu\nu}
%               k^\nu}{2M} F_2(k^2) \\
% \Gamma_\mu^{\gamma\pi} &=& F_\pi (k^2)(2q-k)_\mu
%\eeq
%beschrieben. Dar\"uber hinaus liefert der Stromalgebraterm
%\be
% C_\mu^{a} = -i\epsilon^{a3c} F_A(t) g_A \bar{u}(p_2)\gamma_\mu
%    \gamma_5 \frac{\tau^c}{2} u(p_1)
%\ee    
%einen Beitrag, welcher den normierten axialen Formfaktor
%$F_A(t)=G_A(t)/G_A(0)$ enth\"alt. Wie im Falle reeller 
%Photonen lautet die Eichinvarianzbedingung $k^\mu T_\mu^{a}=0$.
%Wir zerlegen  die Amplitude in der Form
%\be
% T_\mu^{a} = T_\mu^{a(Born)} + \Delta T_\mu^{a} + T_\mu^{a(Res)}
%\ee
%wobei $T_\mu^{a(Born)}$ die Polterme sowie des Stromalgebrabeitrag
%enth\"alt. Der Korrekturterm $\Delta T_\mu^{a}$ ist durch die Bedingung
%\be
% k^\mu ( T_\mu^{a(Born)}+\Delta T_\mu^{a}) =0
%\ee
%definiert, w\"ahrend $T_\mu^{a(Res)}$ eine Untergrundamplitude bezeichnet,
%die bis auf die Eichinvarianzforderung $k^\mu T_\mu^{a(Res)}=0$  unbestimmt 
%bleibt.
%
%Ber\"ucksichtigt man die Formfaktoren an den Vertices, so ist die
%Divergenz des isopsinantisymmetrischen Teils der Bornmaplitude
%\beq
%\label{ngi}
% k^\mu T_\mu^{(-)(Born)} &=& \frac{ief}{m_\pi} \bar{u}(p_2)\Big(
%          2M (2F_1^v(k^2) - F_\pi (k^2) ) \\
%   & & \hspace{3cm} \mbox{} - \gamma\cdot k 
%	  (2F_1^v(k^2) - F_A(t)) \Big) \gamma_5 u(p_1) \nonumber
%\eeq 
%Alle anderen Isospinkomponenten erf\"ullen die Eichinvarianzbedingung.
%F\"ur die $(-)$-Komponente ist dies nur am Photonpunkt $k^2=0$ der
%Fall. Um eine eichinvariante Amplitude zu erhalten, mu\ss\ man einen
%Korrekturterm \cite{VZ72,SK91}
%\beq
%\label{gcor}
%\Delta T_\mu^{(-)} &=& -\frac{ief}{m_\pi} \bar{u}(p_2)\left(
%          \frac{2Mk_\mu}{k^2} (2F_1^v(k^2) - F_\pi (k^2) ) \right.\\
% & & \hspace{3cm} \mbox{}	  
%	  - \left. \frac{k_\mu\gamma\cdot k}{k^2} (2F_1^v(k^2) - F_A(t))
%	   \right) \gamma_5 u(p_1) \nonumber
%\eeq 
%addieren. Dieser Term ist nicht eindeutig bestimmt. Jeder beliebige
%Ausdruck, der sich von (\ref{gcor}) nur um einen divergenzfreien
%Beitrag unterscheidet, ist ebenfalls ein m\"oglicher Korrekututerm.
%Die Summe $T_\mu^{a(Born)}+\Delta T_\mu^{a}$ liefert schlie\ss lich
%eine eichinvariante Elektroproduktionsamplitude. 
%
Es ist instruktiv, die Konsequenzen von Eichinvarianz auch f\"ur 
Pionen abseits der Massenschale zu untersuchen. Dieses Problem
ist vor allem  bei der Bestimmung der Amplitude am weichen Punkt
von Bedeutung. Da sich das Pion nicht in einem asymptotischen
Zustand befindet, kann man zu diesem Zweck allerdings nicht von
Gleichung (\ref{ongi}) Gebrauch machen.  Statt dessen betrachten 
wir die zu (\ref{avward}) analoge Vektorwardidentit\"at
\be
\label{vwi}
ik^\mu \overline{\Pi}_{\nu\mu}^\alpha (q) = - C^\alpha_\nu + 
\frac{i}{m_\pi^2} \Big\{ q_\nu \Sigma_0^\alpha (q) - 
 g_{\nu 0} k^\rho \Sigma_\rho^\alpha (q) \Big\} .
\ee
Auch diese Relation beruht auf der Erhaltung des elektromagnetischen 
Stroms. Sie enth\"alt aber keine zus\"atzlichen Annahmen \"uber den
Impuls des Pions. In Verbindung mit der Axialvektorwardidentit\"at
(\ref{avward}) ergibt sich folgender Ausdruck f\"ur die Divergenz
von $T_\mu^{a}$
\be
\label{offgi}
 k^\mu T_\mu^{a} = -i\epsilon^{a3c} \frac{q^2-m_\pi^2}{f_\pi m_\pi^2}
   \langle N(p_2)|D^c(0)|N(p_1)\rangle  \; .
\ee
F\"ur Pionen auf der Massenschale findet man die bekannte Beziehung
$k^\mu T_\mu^{a} =0$. Abseits der Massenschale liefern geladene 
virtuelle Pionen einen zus\"atzlichen Quellterm f\"ur den elektromagnetischen
Strom und bewirken eine nichtverschwindende Divergenz von $T_\mu^{a}$.

Bei der Herleitung der Relation (\ref{offgi}) ben\"otigt man keine 
Annahmen \"uber die modell\-ab\-h\"angigen Komponenten der symmetriebrechenden
Amplitude $\Sigma_\mu^{a}$. Betrachtet man die einzelnen Beitr\"age 
zur linken Seite von (\ref{offgi})
\be
\label{divamp}
 k^\mu T_\mu^{a} = \frac{1}{f_\pi} \left\{ k^\mu q^\nu 
   \overline{\Pi}_{\mu\nu}^{a}
   -ik^\mu C_\mu^{a} -\frac{\omega_\pi}{m_\pi^2} k^\mu \Sigma_\mu 
   \right\}\, ,
\ee     
so tragen diese Terme jedoch bei. Die Polterme erf\"ullen in Verbindung
mit dem Stromalgebrabeitrag auch die verallgemeinerte Eichinvarianzbedingung
(\ref{offgi}). Man kann diese Terme daher aus der Gleichung (\ref{divamp})
eliminieren. In der Herleitung des Niederenergietheorems vernachl\"assigt
man Hintergrundbeitr\"age zu den Amplituden. Die Gleichung (\ref{divamp})
reduziert sich daher auf eine Beziehung f\"ur die symmetriebrechende
Amplitude: $k^\mu \Sigma_\mu^{(+0)}=0$. Die im Abschnitt 2.5 bestimmten
Beitr\"age erf\"ullen diese Gleichung nicht. Wir definieren daher den 
eichinvarianten Teil von $\Sigma_\mu^{a}$
\be
  \Sigma_\mu^{a(gi)} = \Sigma_\mu^{a} +\Delta\Sigma_\mu^{a}
\ee
durch die Forderung $k^\mu\Sigma_\mu^{(+0)(gi)}=0$. Die isospinungeraden
Komponenten sind durch die Bedingung
\be
  k^\mu T_\mu^{(-)}= -i\frac{q^2-m_\pi^2}{f_\pi m_\pi^2}
    \langle N(p_2)|D(0)|N(p_1)\rangle 
\ee
bestimmt. Eine m\"ogliche L\"osung dieser Gleichungen lautet
\beq
 \frac{\omega_\pi}{m_\pi^2}\Sigma_\mu^{(-)(gi)} &=& -
                 \frac{f_\pi}{m_\pi^2-t}\, g_{\pi NN}
                   \, \bar{u}(p_2)i\gamma_5  q_\mu u(p_1)\, , \\
 \frac{\omega_\pi}{m_\pi^2}\Sigma_\mu^{(0)(gi)} &=& \spm
             \frac{4\overline{m}M}{m_\pi^2} \,\frac{g_T^3}{3}
	     \, \bar{u}(p_2)i\gamma_5 \frac{\gamma_\mu \gamma\cdot k}{2M}u(p_1)\, , \\
 \frac{\omega_\pi}{m_\pi^2}\Sigma_\mu^{(+)(gi)} &=& \spm
             \frac{4\overline{m}M}{m_\pi^2} \,
	     \left\{ g_T^0 \left(  1+\frac{\delta m}{6\overline{m}} \right)
	     \pm g_T^3 \frac{\delta m}{2\overline{m}} \right\}
	 \, \bar{u}(p_2)i\gamma_5 \frac{\gamma_\mu \gamma\cdot k}{2M}u(p_1)\, .
\eeq
Diese Amplituden sind durch die Forderung $k^\mu\Sigma_\mu^{(+0)(gi)}=0$
nicht eindeutig bestimmt. Wir haben sie deshalb durch die zus\"atzliche
Bedingung festgelegt, da\ss\ $\Delta\Sigma_\mu$ die Schwellenamplitude 
(\ref{delneu}) nicht ver\"andert und auch keine Beitr\"age zu der 
longitudinalen Multipolamplitude $L_{0+}$ auftreten. Diese Amplitude 
l\"a\ss t sich in der Elektroproduktion von Pionen bestimmen \cite{SK91}.
Obwohl wir kein eichinvariantes, mikroskopisches Modell f\"ur die 
Photoproduktion von Pionen am Nukleon besitzen, haben wir damit eine 
manifest eichinvariante Amplitude konstruiert, die die Beitr\"age der 
expliziten Symmetriebrechung auf dem Quarkniveau ber\"ucksichtigt.	 
   

\section{Resonanzbeitr\"age}
Das Niederenergietheorem zur Pionphotoproduktion beruht auf der
Annahme, da\ss\ sich die Zweipunktfunktion $q^\nu\overline\Pi^a_{\mu\nu}$
in der N\"ahe des ''weichen`` Punktes $q_\mu=0$ durch die Nukleonpolterme
approximieren l\"a\ss t. Dabei vernachl\"assigt man  die Beitr\"age
von Schleifendiagrammen sowie den Austausch von Resonanzen im s- oder
t-Kanal. 

In der Photoproduktion von Pionen bei mittleren Energien 
$\omega^{lab}= 0.3-1.5$ GeV ist die Bedeutung von s-Kanal-Resonanzen
in den Multipolamplituden deutlich zu erkennen. 
An der Schwelle sind diese Beitr\"age jedoch durch das
Verh\"altnis $m_\pi/\Delta E_R$ der Pionmasse zur Anregungsenergie
der Resonanz unterdr\"uckt. Der niedrigste Anregungszustand des
Nukleons ist die Deltaresonanz bei $\Delta E_R =294$ MeV. Dieser
Zustand koppelt au\ss erordentlich stark an das Pion-Nukleon-System
und dominiert aus diesem Grund die resonante $M_{1+}$-Amplitude
bis in die Schwellenregion. Der niedrigste resonante Beitrag zur
$E_{0+}$-Amplitude stammt vom $N(1535)$ bei einer deutlich h\"oheren
Anregungsenergie $\Delta E_R= 597$ MeV. Im Gegensatz zur Deltaresonanz
zerf\"allt dieser Zustand zu etwa 50\% in $\eta N$ und liefert 
insgesamt nur einen geringen Beitrag zur $E_{0+}$-Amplitude
an der Schwelle.

Um diese Aussagen quantitativ zu belegen, wollen wir die 
Resonanzbeitr\"age mit Hilfe effektiver chiraler Lagrangedichten studieren 
\cite{Pec69,OO75,NB80}. Diese Methode ignoriert die intrinsische 
Struktur der Resonanz, hat aber den wesentlichen Vorteil, mit einem
Minimum an freien Parametern auszukommen. Diese Parameter beschreiben
neben der Masse der Resonanz die Kopplungen $\gamma N\to N^{*}$ 
sowie $N^{*}\to N\pi$ und lassen sich aus den experimentell bestimmten
Helizit\"atsamplituden und Zerfallsbreiten extrahieren.

Zu diesem Zweck betrachten wir die resonante Photoproduktion
$\gamma N(\Lambda_i=\frac{1}{2},\frac{3}{2}) \to N^{*} \to \pi N$
mit definierter Helizit\"at $\Lambda_i$ im Eingangskanal. Die
zugeh\"origen Helizit\"atsamplituden $A_{1/2}$ und $A_{3/2}$ sind durch
\beq
\label{helamp}
 A_{l\pm} &=& \mp \alpha C_{N\pi} A_{1/2}\; ,  \\
 B_{l\pm} &=& \pm \frac{4\alpha}{\sqrt{(2J-1)(2J+3)}} C_{N\pi} A_{3/2}
\eeq
definiert \cite{PDG90}. Die Helizit\"atskomponenten $(A_{l\pm},B_{l\pm})$ 
sind Linearkombinationen der Multipolamplituden $(E_{l\pm},M_{l\pm})$.
Die entsprechenden Zusammenh\"ange finden sich im Anhang B. Der
Parameter $\alpha$ lautet
\be
 \alpha = \left[ \frac{1}{\pi} \frac{k}{q} \frac{M\Gamma_\pi}{(2J+1)
    M_R \Gamma^2} \right]^{1/2} \; .
\ee
Dabei bezeichnet $M_R$ die Masse der Resonanz, $J$ ihren Spin
und $\Gamma$ sowie $\Gamma_\pi$ die totalen bzw.~partiellen Zerfallsbreiten.
$C_{N\pi}$ ist der Clebsch-Gordan-Koeffizient f\"ur den Zerfall der 
Resonanz in den relevanten $N\pi$-Ladungszustand.  Die Definition
(\ref{helamp}) hat den Vorzug, da\ss\ alle Gr\"o\ss en, die mit der
Propagation und dem Zerfall der Resonanz zusammenh\"angen, aus
der  eigentlichen Resonanzamplitude eliminiert werden. Die 
Helizit\"atsamplituden $A_{1/2,3/2}$ liefern daher ein zuverl\"assiges
Ma\ss\ f\"ur die St\"arke des \"Ubergangsmatrixelements $\gamma N\to N^{*}$. 
In Tabelle 2.1 haben wir die entsprechenden Werte f\"ur die
wichtigsten Resonanzen mit Massen unterhalb 1.65 GeV zusammengefa\ss t. 
    
\begin{table}
\caption{Helizit\"atsamplituden (in $10^{-3}\,{\rm GeV}^{1/2}$) sowie
totale und partielle Breiten (in MeV) f\"ur die wichtigsten Nukleonresonanzen
mit Massen unterhalb 1.65 GeV. Alle Angaben nach [PDG90].}
\begin{center}
\begin{tabular}{|l||c|r|r|r|r|} \hline
  Resonanz             & Hel.  &  $A_{1/2,3/2}^p$ & $A_{1/2,3/2}^n$ 
		& $\Gamma_{tot}$ & $\Gamma_\pi$ \\ \hline\hline
 $N(1440)\,P_{11}$ & 1/2   &  $-69\pm 7\;\,$  & $37\pm 19$
                &  200         & 120   \\ 
 $N(1520)\,D_{13}$ & 1/2   &  $-22\pm 10$     & $-65\pm 13$
                &  125         &  70    \\
                       & 3/2   &  $167\pm 10$     & $144\pm 14$
		&              &        \\
 $N(1535)\,S_{11}$ & 1/2   &  $73\pm 14$      & $-76\pm 32$
                &  150         &   65    \\
 $N(1650)\,S_{11}$ & 1/2   &  $48\pm 16$      & $-17\pm 37$ 
                & 150	       &   90    \\
 $\Delta (1232)\,\rm P_{33}$ & 1/2 & $-141\pm 5\;\,$&
                &  115         &  115   \\
		        & 3/2  &  $-258\pm 11$    &          
		&              &        \\ \hline
\end{tabular}
\end{center}
\end{table}

Die dominante Resonanz in der $E_{0+}$-Amplitude ist die 
$N(1535)S_{11}$-Anregung. Dieser Zustand besitzt wie das Nukleon Spin und 
Isospin 1/2, aber negative Parit\"at. Anregung und Zerfall der Resonanz 
werden durch die Kopplungen
\beq
\label{s11coup}
 {\cal L}_{\pi NN^{*}} &=& \frac{f_R}{m_\pi} \bar{\psi}_{N^{*}}
   \gamma_\mu \tau^{a}\psi \partial^\mu \phi^{a} + h.c. \; ,\\
 {\cal L}_{\gamma NN^{*}} &=& \frac{e}{4M} \bar{\psi}_{N^{*}} 
   \gamma_5 \sigma_{\mu\nu} (\kappa^s_R +\kappa^v_R \tau^3) \psi
    F^{\mu\nu} + h.c.
\eeq
beschrieben. Die beiden Parameter $f_R$ und $\kappa_R$ werden mit  
Hilfe der Beziehungen
\beq
\label{rescoup}
       f_R         &=& \frac{2m_\pi}{M_R-M} 
       \sqrt{\frac{\pi M_R \Gamma_\pi}{ p_1(E_1+M)}} 
       \simeq 0.27 , \\
 e\kappa^{p}_R   &=& \frac{(2M)^{3/2}}{\sqrt{(M_R+M)(M_R-M)}} A^{p}_{1/2}
       \simeq 0.51 e
\eeq
bestimmt. Dabei bezeichnen $E_1$ und $p_1$ die Energie sowie den Impuls
des Nukleons im Ruhesystem des angeregten Zustands bei der Resonanzenergie
$\sqrt{s}=M_R$. Unter Verwendung der Vertices (\ref{s11coup}) lassen sich
nun die Borndiagramme zur resonanten Photoproduktion berechnen. Die 
zugeh\"origen invarianten Amplituden finden sich im Anhang B. Der Beitrag
der s-Kanal-Anregung der N(1535)-Resonanz zur elektrischen Dipolamplitude
an der Schwelle lautet
\beq
  E_{0+}^{N^{*}}(p\pi^0) &=& \frac{e\kappa_R}{16\pi M}\frac{f_R}{m_\pi}
    \mu^2\frac{2+\mu}{(1+\mu)^{3/2}} 
    \frac{M(M_R+M+m_\pi)}{M_R^2-(M+m_\pi)^2} \\[0.2cm]
    &\simeq& 0.06 \su \, .  \nonumber
\eeq    
Dieses Ergebnis ist formal von der Ordnung $\mu^2$ und widerspricht daher
der in Abschnitt 2.3 vorgenommenen Absch\"atzung der vernachl\"assigten
Amplitude. Das liegt darin begr\"undet, da\ss\ der s-Kanal-Beitrag f\"ur
sich genommen nicht Austauschinvariant ist. 

Die leichteste Anregung mit denselben Quantenzahlen wie das Nukleon ist
die Roperesonanz $N(1440)$. Dieser Zustand liefert einen resonanten
Beitrag zur $M_{1-}$-Amplitude, ist in der elektrischen Dipolamplitude 
aber nur als Untergrund pr\"asent. Die effektive Lagrangedichte, welche
die Kopplung des $N(1440)$ an das Nukleon beschreibt, lautet
\beq        
\label{nstarcoup}
 {\cal L}_{\pi NN^{*}} &=& \frac{f_R}{m_\pi} \bar{\psi}_{N^{*}}
   \gamma_\mu \gamma_5\tau^{a}\psi \partial^\mu \phi^{a} + h.c. \; ,\\
 {\cal L}_{\gamma NN^{*}} &=& \frac{e}{4M} \bar{\psi}_{N^{*}} 
    \sigma_{\mu\nu} (\kappa^s_R +\kappa^v_R \tau^3) \psi
    F^{\mu\nu} + h.c. \; .
\eeq
Bestimmt man die Kopplungskonstanten aus der Zerfallsbreite und der
Helizit\"atsamplitude bei der Resonanzenergie $\sqrt{s}=m_R$, so
ergibt sich $f_R=0.48$ und $\kappa_R^p=0.58$. Mit diesen Werten 
findet man folgenden Beitrag der s-Kanal-Anregung
%\be
% E_{0+}^{N^{*}}(p\pi^0) &=& \frac{e\kappa_R}{16\pi M}\frac{f_R}{m_\pi}
%    \frac{2\mu+\mu^2}{(1+\mu)^{3/2}} 
%     \frac{m_\pi(M_R-M-m_\pi)}{(M+m_\pi)^2-M_R^2}  
%\ee    
\be    
   E_{0+}^{N^{*}}(p\pi^0)  \simeq -0.025 \su \, . 
\ee 
Wie erwartet ist die entsprechende Amplitude au\ss erordentlich klein.
Eine gewisse Schwierigkeit stellt die Behandlung der Deltaresonanz
$\Delta (1232)$ dar \cite{DMW91,NS89,NB80}. Dieser Zustand ist
eine $P_{33}$-Anregung und sollte daher nicht zur s-Wellen-Produktion 
beitragen. In einer relativistischen Beschreibung
der Deltaresonanz als elementares Spin-3/2 Rarita-Schwinger-Feld
enth\"alt der Deltapropagator allerdings abseits der Massenschale auch
Spin-1/2 Komponenten. Die Kopplung dieser Beitr\"age an 
die Zerfallskan\"ale $\gamma N$ und $\pi N$ ist im wesentlichen
unbestimmt.  Je nach Wahl der entsprechenden Parameter findet man
\cite{NS89}
\be
\label{delta}
   E_{0+}^\Delta(\pi^0p) = (-0.10 \ldots 0.34) \su  \; .
\ee
Das angegebene Intervall entspricht der Streuung, die sich aus
verschiedenen Fits der Parameter an die nicht resonanten
Amplituden ergibt.  Das Resultat zeigt deutlich die Grenzen der 
Verwendung effektiver chiraler Lagrangedichten bei der Beschreibung
angeregter Zust\"ande auf. Trotzdem sind auch die Korrekturen
auf Grund der Deltaresonanz letztlich relativ gering. 

Wir haben unsere Untersuchung bislang auf die Rolle von Resonanzen
im s-Kanal beschr\"ankt. Aus dem Studium von Dispersionsrelationen 
ist jedoch bekannt, da\ss\ die Einbeziehung von Vektormesonen als
t-Kanal-Resonanzen die Beschreibung der differentiellen Wirkungsquerschnitte 
besonders bei kleinen Energien verbessert \cite{BDW67}. Es scheint daher 
angemessen, die Rolle von Vektormesonen auch direkt an der Schwelle zu 
untersuchen. Dabei beschr\"anken wir uns auf die $\rho$- und $\omega$-Mesonen. 
Das $\phi$-Meson sowie die schweren Vektormesonen liefern nur geringe
Beitr\"age. Die relevanten Terme in der effektiven Lagrangedichte lauten:
\beq
\label{lvm}
 {\cal L}_{\rho NN} &=& f_{\rho NN} \bar{\psi}
         \left( \gamma_\mu +\frac{\kappa_\rho}{2M}\sigma_{\mu\nu}
	 \partial^\nu \right) \vec{\tau}\cdot\vec{\rho}^{\,\mu} \psi , \\ 
 {\cal L}_{\omega NN} &=& f_{\omega NN} \bar{\psi}
         \left( \gamma_\mu +\frac{\kappa_\omega}{2M}\sigma_{\mu\nu}
	 \partial^\nu \right) \omega^\mu \psi  ,\\
 {\cal L}_{\rho\pi\gamma} &=& \frac{eg_{\rho\pi\gamma}}{2m_\pi}
         \epsilon_{\alpha\beta\gamma\delta} F^{\alpha\beta}
	 \vec{\phi}\cdot\partial^\gamma\vec{\rho}^{\,\delta} ,\\
 {\cal L}_{\omega\pi\gamma} &=& \frac{eg_{\omega\pi\gamma}}{2m_\pi}
         \epsilon_{\alpha\beta\gamma\delta} F^{\alpha\beta}
	 \phi_3\cdot\partial^\gamma\omega^\delta \; .
\eeq
Die Kopplung des Photons l\"a\ss t sich aus den gemessenen Zerfallsbreiten
$\Gamma(\rho,\omega\to\pi\gamma)$ bestimmen. Die 
Vektormeson-Nukleon-Kopplungskonstante mu\ss\ dagegen indirekt, aus 
detaillierten Analysen
des Nukleon-Nukleon-Potentials gewonnen werden \cite{Dum82}. Die
resultierenden Werte finden sich in Tabelle 2.2.  
 
\begin{table}
\caption{Parameter f\"ur die wichtigsten t-Kanal-Beitr\"age 
zur Pionphotoproduktion.}
\begin{center}
\begin{tabular}{|rcl|rcl|rcl|}\hline
   & $\pi$ &             &  & $\rho$ &             &    &  $\omega$ &  \\ 
                                                                \hline\hline
$m_{\pi^\pm}$&=&139 MeV  & $m_\rho$&=&$770$ MeV    &  $m_\omega$&=&$783$ MeV\\
$f_{\pi NN}$&=&$1.00$    &  $f_{\rho NN}$&=&$2.66$ & $f_{\omega NN}$&=&$7.98$\\
$g_{\pi\pi\gamma}$&=&$1$ &  $g_{\rho\pi\gamma}$&=&$0.125$ 
                                        &   $g_{\omega\pi\gamma}$&=&$0.374$  \\
    & &       &  $\kappa_\rho$&=&$6.6$  &   $\kappa_\omega$&=&$0.$   \\ \hline
\end{tabular}
\end{center}
\end{table}                    

Die invarianten Amplituden, die sich aus der Wechselwirkung (\ref{lvm})
ergeben, haben wir in Anhang B gesammelt. Der Beitrag zur
elektrischen Dipolamplitude an der Schwelle ist
\be
  E_{0+}^V(\pi^0p) = \frac{e}{16\pi} \sum_V\frac{g_V}{m_\pi} 
    f_V (1+\kappa_V)\mu^3
     \frac{2+\mu}{(1+\mu)^{3/2}}\frac{M^2}{m_\pi^2+m_V^2 (1+\mu)}\, ,
\ee     	 
wobei $V=\rho,\omega$ zu setzen ist. Dieses Resultat ist explizit
von der Ordnung $\mu^3$ und entspricht daher der Absch\"atzung
aus dem Abschnitt 2.5. Mit den Werten aus Tabelle 2.2 findet man 
$g_\rho f_\rho (1+\kappa_\rho)=2.53$ und  $g_\omega f_\omega 
(1+\kappa_\omega)=2.98$, so da\ss\ sich schlie\ss lich die 
Korrektur $E_{0+}^V(\pi^0p)=0.21\su$ ergibt. Auch dieser Beitrag 
liefert also keine wesentliche Modifikation der elektrischen 
Dipolamplitude an der Schwelle.  
 
\section{Zusammenfassung}
Wir haben in diesem Kapitel Stromalgebratechniken zur Bestimmung der 
Pionphotoproduktionsamplitude an der Schwelle eingesetzt. Das klassische
Niederenergietheorem f\"ur die elektrische Dipolamplitude $E_{0+}$
ergibt sich, indem man die \"Ubergangsmatrix durch die Nukleon- und
Pion-Bornterme approximiert. Wir haben gezeigt, da\ss\ die \"ubliche 
Absch\"atzung der Korrekturen zu diesem Resultat auf zus\"atzlichen 
Annahmen \"uber das Verhalten der Photoproduktionsamplitude beruhen, 
die nicht notwendig gerechtfertigt sind. 

Im Abschnitt 2.5 sind wir auf die Rolle der expliziten Symmetriebrechung
durch die Quarkmassen in der QCD-Lagrangedichte eingegangen. Wir haben 
gezeigt, da\ss\ dieser Effekt in einem Modell freier Quarks zu einem 
Korrekturterm f\"uhrt, der proportional zu den Massen der leichten 
Quarks und der Tensorkopplungskonstante des Nukleons ist. Einfache 
Absch\"atzungen f\"uhren zu dem Schlu\ss , da\ss\ dieser Term einen 
erheblichen Beitrag zur elektrischen Dipolamplitude f\"ur neutrale Pionen 
liefern kann.

Abschlie\ss end haben wir demonstriert, da\ss\ Resonanzen im s- oder 
t-Kanal keine wesentlichen Korrekturen zur elektrischen Dipolamplitude
hervorrufen. Eine wichtige Frage, auf die wir in dieser Arbeit nicht 
eingehen k\"onen, betrifft die Bedeutung der Endzustandswechselwirkung 
im Pion-Nukleon-System. Insbesondere bei der Produktion neutraler Pionen,
wo der elementare Proze\ss\ deutlich unterdr\"uckt ist, kann die Reaktion
$\gamma p\to \pi^+n\to\pi^0p$ bedeutende Korrekturen liefern. Die 
theoretische Bestimmung dieser Beitr\"age wird allerdings gegenw\"artig 
noch sehr kontrovers diskutiert \cite{DT91,Kam89,NLB90,LYL91,Ber91,BKG91}.
\cld

%\chapter{Niederenergietheoreme zur Pionphotoproduktion}
Nachdem wir uns im letzten Kapitel mit der experimentellen 
Bestimmung der elektrischen Dipolamplitude an der Schwelle
befa\ss t haben, wollen wir uns nun auf die theoretische 
Bestimmung von $E_{0+}$ mit Hilfe von Niederenergietheoremen
konzentrieren. Die spezielle Bedeutung der Photoproduktion
neutraler Pionen ergibt sich dabei aus der Tatsache, da\ss\
die entsprechende Schwellenamplitude in einer hypothetischen Welt
mit masselosen Pionen  verschwinden w\"urde.
Dieser Kanal ist daher besonders sensitiv auf die Rolle der
expliziten chiralen Symmetriebrechung, welche sich in  dem
nur approximativen Charakter des Pions als Goldstoneboson 
widerspiegelt. 

Die physikalische Grundidee der Niederenergietheoreme
({\em engl.} Low Energy Theorem, LET) l\"a\ss t sich besonders
\"ubersichtlich an rein elektromagnetischen Reaktionen
diskutieren. Die Anwendung  von Niederenergietheoremen wird 
in diesem Fall durch zwei physikalische Kriterien kontrolliert. 
Die beiden Forderungen lauten, da\ss\ die Wellenl\"ange des 
Photons gro\ss\ ist im Vergleich zur Ausdehnung des Streuzentrums
w\"ahrend die Energie des Photons klein ist gegen die typische 
Anregungsenergie. Sind diese Voraussetzungen erf\"ullt, so ist das 
Photon nicht in der Lage, die innere Struktur des Targets aufzul\"osen. 
Der differentielle Wirkungsquerschnitt ist daher ausschlie\ss lich durch 
globale elektromagnetische Eigenschaften des Streuzentrums bestimmt.
Betrachtet man die Comptonstreuung niederenergetischer
Photonen an einem hadronischen Target, so folgt aus dieser
\"Uberlegung, da\ss\ der differentielle Wirkungsquerschnitt 
nur von der Gesamtladung $Ze$ abh\"angt und der Targetmasse $M$ 
abh\"angt. Insbesondere ergibt 
sich im Grenzfall $k_\mu=(\omega,\vec{k})\to 0$ die klassische
Thomson-Streuung
\be
\label{thomson}
 \lim_{\omega \to 0} \frac{d\sigma}{d\Omega} =
  \frac{Z^2e^2}{4\pi M^2} (\vec{\epsilon}_1 \cdot\vec{\epsilon}_2)
\ee     
wobei $\vec{\epsilon}_1$ und $\vec{\epsilon}_2$ die 
Polarisationsvektoren der ein- und auslaufenden Photonen
bezeichnen. Low, Gell-Mann und Goldberger \cite{Low54,Low58,GMG54}
konnten dar\"uber hinaus zeigen, da\ss\ auf Grund von 
Eich- und Lorentzinvarianz die Amplitude f\"ur Comptonstreuung 
in Forw\"artsrichtung an einem Spin-1/2 Target mit der Ladung sogar bis
auf Terme linear in der Laborenergie $\omega$ bestimmt ist
\be
 \lim_{\omega \to 0}  T(\omega) =- \frac{e^2}{M}  
  (\vec{\epsilon}_1 \cdot\vec{\epsilon}_2) -i \frac{e^2}{8 M^2}
  \kappa^2 \omega  (\vec{\epsilon}_1 \times\vec{\epsilon}_2)
  \cdot \vec{\sigma} \; .
\ee
Der erste Term beschreibt  die Thomsonamplitude, w\"ahrend der
zweite Term eine Korrektur liefert, die proportional zum
anomalen magnetischen Moment $\kappa$ des Streuzentrums ist.

Die Anwendung von Niederenergietheoremen auf pionische 
Reaktionen wird durch die endliche Masse des Pions 
erschwert. Die Pionmasse setzt eine untere Grenze f\"ur 
die Energie  $\omega_\pi=(\vec{q}^{\, 2} +m_\pi^2)^{1/2}$
des Pions. W\"ahrend daher die Wellenl\"ange beliebig gro\ss\
gemacht werden kann, gibt es eine prinzipielle Schranke 
f\"ur die Energie. Die Anwendbarkeit von Niederenergietheoremen
setzt daher voraus,
da\ss\ die Masse des Pions klein gegen die charakteristische
Energieskala der Reaktion ist. In hadronischen Prozessen ist
eine solche Skala durch die Masse des Nukeons gegeben. 
Das Verh\"altnis $m_\pi/M\simeq 1/7$ ist daher ein nat\"urlicher
Parameter, der die Abweichung der  Amplitude vom unphysikalischen
Grenzfall $q_\mu \to 0$ kontrolliert.

Formal basieren Niederenergietheoreme f\"ur ''weiche`` Pionen auf
der G\"ultigkeit von Stromalgebra, chiraler Symmetrie und
PCAC. Historisch wurden diese Konzepte in den sechziger Jahren 
als Hypothesen \"uber das Transformationsverhalten der in
hadronische Reaktionen eingehenden Str\"ome entwickelt
\cite{AD68,AFF73}. Sie erwiesen sich als au\ss erordentlich
fruchtbar, um die  experimentellen 
Informationen \"uber hadronische Reaktionen zu verstehen. 
Die klassischen Anwendungen liegen im Bereich
der $\pi\pi$- und $\pi N$-Streuung, sowie der Photo- und schwachen
Produktion pseudoskalarer Mesonen. Wichtige Vorhersagen ergeben
sich dar\"uber hinaus f\"ur leptonische und semileptonische Zerf\"alle
stark wechsewirkender Teilchen.    
  
Nach der Gr\"underphase trat die Anwendung von Stromalgebra und
chiraler Symmetrie zun\"achst in den Hintergrund gegen\"uber
der Entwicklung von Quantenchromodynamik als fundamentaler
Eichtheorie der starken Wechselwirkung. In diesem Zusammenhang 
zeigte sich allerdings, da\ss\ die G\"ultigkeit dieser Konzepte eine direkte
Konsequenz von QCD ist. Dar\"uber hinaus liefert die chirale 
Symmetrie  einen wichtigen Zugang zur starken Wechselwirkung
in einem Bereich, in dem die direkte Anwendung von QCD
bislang noch gro\ss en Schwierigkeiten gegen\"uber steht.

\section[Quantenchromodynamik, Stromalgebra \ldots]{Quantenchromodynamik,
 Stromalgebra und chirale Symmetrie}
Die Quantenchromodynamik ist eine nichtabelsche Eichtheorie, 
beschrieben durch die Lagrangedichte
\be
\label{lqcd}
{\cal L}_{QCD} = -\frac{1}{4} F_{\mu\nu}^{\;\;a} F^{\mu\nu\, a}
 + \sum_{j=1}^{n_f} \bar{\psi}^{\alpha}_{j}( i\gamma^{\mu}
 {\cal D}_\mu^{\alpha\beta} - \delta^{\alpha\beta} m_j )
 \psi_j^\beta \; ,
\ee 
wobei die  Summation \"uber $n_f$ verschiedene Quarkarten (flavors)
ausgef\"uhrt wird. W\"ahrend sich die Quarkspinoren $\psi^{\alpha}$ nach
der fundamentalen Darstellung der Eichgruppe $SU(3)$ transformieren,
sind die Gluonen $A_\mu^{a}$ Vektorfelder und tragen einen Index 
in der adjungierten Darstellung der $SU(3)$. Die zugeh\"origen
Elemente der Lie-Algebra sind
\be
 A_\mu(x) = A_\mu^{a}(x)\frac{\lambda^{a}}{2}\; ,
\ee
wobei $\lambda^{a}$ die Generatoren der Algebra bezeichnet.
Sie erf\"ullen die fundamentalen Vertauschungsrelationen
\be
 [\lambda^{a},\lambda^b] = 2if^{abc}\lambda^c
\ee
und k\"onnen durch
\be
 Tr \,\lambda^{a}\lambda^b = 2\delta^{ab}
\ee
normiert werden. Dabei bezeichnet $f^{abc}$ die Strukturkonstanten
von $SU(3)$. Der Yang-Mills-Feldst\"arketensor ist durch
\be
\label{fmunu}
 F_{\mu\nu}^{\;\; a} = \partial_\mu A_\nu^{a} -\partial_\nu A_\mu^{a} 
 + g f^{abc} A_\mu^b A_\nu^c
\ee
gegeben, w\"ahrend die kovariante Ableitung durch
\be
\label{kovd}
 {\cal D}_\mu^{\alpha\beta} = \delta^{\alpha\beta}\partial_\mu
  + i\frac{g}{2} (\lambda^{a})^{\alpha\beta} A_\mu^{a}
\ee
definiert ist. Entscheidend f\"ur die nichtabelsche Eichsymmetrie
ist die Tatsache, da\ss\ die Quark-Gluon-Wechselwirkung durch dieselbe
Kopplungskonstante $g$ wie die gluonische Selbstwechselwirkung 
bestimmt ist.
   
Neben der lokalen $SU(3)$ Eichsymmetrie besitzt die $QCD$
Lagrangedichte noch eine Reihe kontinuierlicher globaler Symmetrien.
So ist ${\cal L}_{QCD}$ invariant unter der globalen 
$U(1)_V$ Transformation
\be
\label{uone}
\psi_j(x) \to \exp (-i\theta) \psi_j (x) \, .
\ee
Der dazugeh\"orige erhaltene Vektorstrom ist der Baryonenstrom
\be
 j_\mu(x) = \sum_{j=1}^{n_f} \bar{\psi}_j \gamma_\mu \psi_j
\ee
mit der baryonischen Ladung 
\be
B=\int d^3x\, j_0(\vec{x},t)\, .
\ee
F\"ur masselose Quarks ist ${\cal L}_{QCD}$ ebenfalls invariant
unter der axialen $U(1)_A$ Transformation
\be
\label{uaone}
\psi_j(x) \to \exp (-i\theta\gamma_5) \psi_j (x) \; .
\ee
Der entsprechende Strom besitzt allerdings eine anomale Divergenz
\be
\label{axanom}
\partial^\mu j_{\mu\, 5}(x) = \frac{g^2}{4\pi}\frac{n_f}{8}
 \epsilon^{\mu\nu\rho\sigma} F_{\mu\nu}^{\;\; a}F_{\rho\sigma}^{\;\; a}
 \; ,
\ee
so da\ss\  die axiale Ladung $Q_5=\int d^3x\, j_{0\,5}(\vec{x},t)$
nur in  Abwesenheit instantonartiger L\"osungen erhalten ist.

Vernachl\"assigt man die Quarkmassen, so ist ${\cal L}_{QCD}$ auch
invariant unter Skalentransformationen. Diese Symmetrie wird in der
quantisierten Theorie durch die Notwendigkeit der Renormierung
gebrochen. Dabei tritt ein dimensionsbehafteter Parameter, der
QCD Skalenparameter $\Lambda_{\mini QCD}$, auf. Sein Wert ist unter anderem 
aus der Skalenbrechung in der tief-inelastischen Lepton-Nukleon Streung zu
$\Lambda_{\mini QCD}^{\mini\overline{MS}} =230\pm 80$ MeV bestimmt worden
\cite{PDG90}. Der Index ${\kl \overline{MS}}$ bezeichnet eine spezielle
Renormierungsvorschrift, die sogenannte modifizierte minimale 
Subtraktion.

Ebenfalls auf Grund der Renormierung sind  auch die Werte der 
Quarkmassen von der experimentellen Skala abh\"angig.
Im Bereich typischer hadronischer Prozesse lassen sich 
die Stromquarkmassen mit Hilfe von QCD-Summenregeln 
extrahieren \cite{GL82}. Bei $\mu^2=1\,{\rm GeV}^2$ findet man
\cite{Leu89}
\beq
   m_u &=& 5.1 \pm 1.5 \;{\rm MeV}, \nonumber  \\
   m_d &=& 8.9 \pm 2.6 \;{\rm MeV}, \\
   m_s &=& 175 \pm 55 \;{\rm MeV}.  \nonumber
\eeq   
Alle anderen bekannten Flavors haben Massen \"uber einem GeV. Die
up und down Quarks sind ausserordentlich leicht verglichen mit dem
QCD Skalenparameter, w\"ahrend das seltsame Quark eine Zwischenstellung
einnimmt.   

Wir wollen daher im folgenden die drei leichten Quarks zun\"achst als
masselos betrachten. In diesem Fall besitzt ${\cal L}_{QCD}$
eine chirale $SU(3)_L \times SU(3)_R$ Symmetrie im Raum der $u$-,
$d$- und $s$-Flavors
\beq
\label{suv}
\psi_i(x) &\to& \exp (-i\theta^{a}\lambda^{a})_{ij} \psi_j (x)\, ,\\
\label{sua}
\psi_i(x) &\to& \exp (-i\phi^{a}\lambda^{a}\gamma_5)_{ij} \psi_j (x)\, ,
\eeq
wobei die $SU(3)$ Matrizen $\lambda^{a}$ auf den Flavorindex der
Quarks wirken. Die zugeh\"origen N\"otherstr\"ome sind die
Vektor- und Axialvektorstr\"ome
\beq
   V_\mu^{a}(x) &=& \bar{\psi}_i \gamma_\mu \frac{\lambda^{a}_{ij}}{2}
     \psi_j \, , \\
   A_\mu^{a}(x) &=& \bar{\psi}_i \gamma_\mu \gamma_5
   \frac{\lambda^{a}_{ij}}{2}   \psi_j  \, , 
\eeq
deren Zeitkomponenten auf die erhaltenen Ladungen 
\beq
 Q^{a}(t)   &=& \int d^3x \, V_0^{a}(\vec{x},t) \, , \\
 Q^{a}_5(t) &=& \int d^3x \, A_0^{a}(\vec{x},t) 
\eeq
f\"uhren. Die Struktur der zugeh\"origen Lie-Algebra erkennt man 
am einfachsten, indem man zu den 
Linearkombinationen $Q^{a}_{L,R}=Q^{a}\pm Q^{a}_5$ \"ubergeht.
Sie erf\"ullen die Vertauschungsrelationen
\beq
\label{chalg}
\,[Q_{L}^{a}(t),Q_{L}^{b}(t)] &=& i f^{abc} Q_{L}^{c}(t)\, ,  \\
\,[Q_{R}^{a}(t),Q_{R}^{b}(t)] &=& i f^{abc} Q_{R}^{c}(t)\, ,  \\
\,[Q_{L}^{a},Q_{R}^{b}]     &=& 0\, ,
\eeq
chrakteristisch f\"ur die Lie-Algebra $SU(3)_L \times SU(3)_R$.  
Das Transformationsverhalten der Str\"ome unter der chiralen
$SU(3)_L\times SU(3)_R$ ergibt sich aus den kanonischen
Vertauschungsregeln f\"ur die Quarkfelder
\beq
\label{curalg}
\,[ Q^{a}(t),V_\mu^b (\vec{x},t)] &=&\!\spm i f^{abc}V_\mu^{c}(\vec{x},t)\, ,\\
\,[ Q^{a}(t),A_\mu^b (\vec{x},t)] &=&\!\spm i f^{abc}A_\mu^{c}(\vec{x},t)\, ,\\  
\,[ Q^{a}_5(t),V_\mu^b (\vec{x},t)] &=&\! -i f^{abc}A_\mu^{c}(\vec{x},t)\, ,\\ 
\,[ Q^{a}_5(t),A_\mu^b (\vec{x},t)] &=&\!\spm i f^{abc}V_\mu^{c}(\vec{x},t) \; .
\eeq 
Diese Relationen legen  die $SU(3)_L\times SU(3)_R$ Darstellung
fest, nach der sich die Str\"ome transformieren. Sie wurden von 
Gell-Mann auf Grund rein ph\"anomenologischer \"Uberlegungen 
postuliert \cite{AD68}. 

Soll die Quantenchromodynamik im Grenzfall verschwindender Massen 
der leichten Quarks eine sinnvolle 
N\"aherung an die vollst\"andige Theorie darstellen, dann kann der 
QCD-Grundzustand nicht $SU(3)_L\times SU(3)_R$ symmetrisch sein. 
W\"are n\"amlich
\be
\label{symvac}
 Q_L^{a}|0> = Q_R^{a}|0> = 0 \; ,
\ee
dann sollten auch die Zweipunktfunktionen der Vektor und Axialvektorstr\"ome
identisch sein
\be
\label{symspec}
<0|T(A_\mu^{a}(x)A_\nu^{b}(0))|0> =<0|T(V_\mu^{a}(x)V_\nu^{b}(0))|0> \; .
\ee  
Zum Beweis zerlegt man die beiden Str\"ome in ihre links- und rechtsh\"andigen
Komponenten 
\beq
 V_\mu^{a} &=& J_{\mu\, R}^{a} + J_{\mu\, L}^{a} \\
 A_\mu^{a} &=& J_{\mu\, R}^{a} - J_{\mu\, L}^{a}
\eeq
und folgert aus (\ref{symvac}) und dem Transformationsverhalten der Str\"ome,
da\ss\  die nichtdiagonale Kombination $<0|T(J_{\mu\, L}^{a}J_{\nu\,R}^{a})|0>$
verschwindet.

Die spektrale Dichten zu den beiden Zweipunktfunktionen (\ref{symspec}) sind
experimentell zu\-g\"ang\-lich und zeigen ein sehr verschiedenartige Gestalt.
W\"ahrend der Vektorkanal vor allem durch die $\rho$-Resonanz bei 
$m_\rho=770$ MeV dominiert wird, ist die Axialvektorspektralfunktion in 
diesem Bereich klein und zeigt erst im Bereich der $a_1$-Resonanz bei
$m_{a_1}=1260$ MeV eine ausgepr\"agte Struktur. Wir folgern daraus, da\ss\ 
der QCD Grundzustand nicht $\chs$ symmetrisch ist. Dieses Ph\"anomen, 
da\ss\  n\"amlich der Grundzustand der Theorie
nicht die volle Symmetrie der Lagrangedichte besitzt, bezeichnet man
als spontane Symmetriebrechung.

Nach dem Goldstone-Theorem impliziert die spontane Brechung einer Symmetrie
das Auftreten masseloser Bosonen. Ist $Q^{a}$ ein  $\chs$ Generator,
der das Vakuum nicht invariant l\"a\ss t, dann gibt es einen physikalischen
Zustand $Q^{a}|0>$, der mit dem Vakuum energetisch entartet ist.  
Handelt es sich bei $Q^{a}$ um eine Vektorladung, dann beschreibt
$Q^{a}|0>$ ein masseloses skalares Teilchen. Ist $Q^{a}$ dagegen eine
axiale Ladung, so fordert das Goldstonetheorem das Auftreten masseloser
pseudoskalarer Bosonen.

In der Natur sind die acht leichtesten Hadronen $(\pi,K,\eta)$ pseudoskalar.
Dagegen sind die leichtesten skalaren Teilchen schwerer als das Nukleon.
Die chirale Symmetrie ist daher in der Form
\beq
   Q_V^{a}|0> &=& 0 \, , \\
   Q_A^{a}|0> &\neq& 0
\eeq
realisiert. Die Vektorsymmetrie bleibt erhalten, so da\ss\  sich Hadronen
nach irreduziblen Darstellungen von $SU(3)_V$ klassifizieren lassen. Im
$SU(2)$-Sektor der Theorie folgt daraus die Isospinsymmetrie der 
starken Wechselwirkung.

Endliche Quarkmassen brechen die chirale Symmetrie explizit. Die 
Divergenz der Vektor und Axialvektorstr\"ome lautet 
\beq
\label{divv}
\partial^\mu V_\mu^{a} &=& \frac{i}{2} 
               \bar{\psi}\left[M,\lambda^{a}\right]\psi\, , \\ 
\label{diva}
\partial^\mu A_\mu^{a} &=& \frac{i}{2} 
               \bar{\psi}\left\{M,\lambda^{a}\right\}\psi \, ,
\eeq
wobei $M={\rm diag}(m_u,m_d,m_s)$ die Massenmatrix f\"ur die drei 
leichten Flavors bezeichnet. Die rechte Seite von (\ref{divv},\ref{diva})
l\"a\ss t sich besonders \"ubersichtlich darstellen, indem man $M$ nach 
Gell-Mann Matrizen entwickelt, $M=\epsilon_0\lambda^0+\epsilon_3\lambda^3+
\epsilon_8\lambda^8$. Dabei ist
\beq
\epsilon_0 &=& \frac{1}{\sqrt{2}} (m_u+m_d+m_s)\, , \\
\epsilon_3 &=& \frac{1}{2} (m_u-m_d)\, ,  \\
\epsilon_8 &=& \frac{1}{2\sqrt{3}} (m_u+m_d-2m_s)\, .
\eeq
Im Fall entarteter Quarkmassen $m_u=m_d=m_s$ ist $\epsilon_3=\epsilon_8=0$
und die $SU(3)_V$ Flavor-Symmetrie bleibt ungebrochen.
 
Im Hadronenspektrum manifestieren sich die nichtverschwindenden 
Quarkmassen in einer endlichen Masse f\"ur die Goldstonebosonen 
$(\pi,K,\eta)$. Der Zusammenhang dieser Massen mit denen der Quarks
ist durch die Gell-Mann, Oakes, Renner (GOR) Relation gegeben
\cite{GOR68}. Um die bei der Herleitung von Niederenergietheoremen
verwendeten Methoden vorzustellen, wollen wir im folgenden einen
kurzen Beweis der GOR-Relation vorstellen. Zu diesem Zweck definieren
wie die Pionzerfallskonstante durch das Matrixelement
\be
\label{fpi}
 <0|A_\mu^{a}(x)|\pi^{b}(q)> = \delta^{ab} f_\pi q_\mu e^{iq\cdot x}
\ee
Diese \"Ubergangsmatrix kontrolliert den schwachen Zerfall der
geladenen Pionen $\pi^\pm \to \mu^\pm \nu_\mu$. F\"ur das
Matrixelement der Divergenz des Axialstroms findet man
\be
 <0|\partial^\mu A_\mu^{a}(x)|\pi^{b}(q)> = 
         \delta^{ab} f_\pi m_\pi^2 e^{iq\cdot x}\; ,
\ee	     
so da\ss\ sich ein interpolierendes Feld f\"ur das Pion durch
die Beziehung
\be
\label{PCAC}
\partial^\mu A_\mu^{a}(x) = f_\pi m_\pi^2 \phi^{a} (x)
\ee
definieren l\"a\ss t \cite{Col67}. Diese Gleichung wird als PCAC (partially
conserved axial current) Relation bezeichnet. Sie ist 
\"aquivalent zur Divergenzbeziehung (\ref{diva}) und dr\"uckt wie
diese die Tatsache aus, da\ss\  der Axialvektorstrom im chiralen 
Limes $m_\pi \to 0$ erhalten ist.

Wir wollen im folgenden Wardidentit\"aten f\"ur die Zweipunktfunktionen
\beq
\label{axtwop}
\Pi_{5\, \mu\nu}^{ab}(q) &=& i\int d^4x\, e^{iq\cdot x}
          <0|T(A_\mu^{a}(x)A_\nu^{b}(0))|0>    \\
\label{divtwop}	  
\psi_{5}^{ab} (q) &=& i\int d^4x\, e^{iq\cdot x}
          <0|T(\partial^\mu A_\mu^{a}(x)\partial^\nu A_\nu^{b}(0))|0>
\eeq
konstruieren. Zweimaliges Differenzieren des zeitgeordneten Produkts 
(\ref{axtwop}) liefert die Beziehung
\beq
\label{wi}
q^\mu q^\nu  \Pi^{ab}_{5\,\mu\nu} (q) &=& \psi_5^{ab}(q)
   -q^\nu \int d^4x\, e^{iq\cdot x} 
   \delta (x^0) <0|[A_0^{a}(x),A_\nu^{b}(0)]|0> \\
   & & \mbox{} -i\int d^4x\,  e^{iq\cdot x} 
   \delta (x^0) <0|[A_0^{a}(x),\partial^\mu A_\mu^{b}(0)]|0>\, , \nonumber
\eeq
die sich im Limes $q_\mu \to 0$ auf die Gleichung
\be
\label{psi0}
  \psi_5^{ab}(0) = -i \int d^4x \, \delta(x^0) 
        <0|[A_0^{a}(x),\partial^\mu A_\mu^{b}(0)]|0>    	   	       
\ee	
reduziert. Die linke Seite dieser Gleichung folgt aus der PCAC Relation, 
w\"ahrend man die rechte Seite mit Hilfe von (\ref{diva}) und
den kanonischen Vertauschungsregeln f\"ur die Quarkfelder 
bestimmmen kann. Das Resultat ergibt schlie\ss lich die GOR-Relation
\be
 f_\pi^2 m_\pi^2 = - \frac{1}{2}( m_u+m_d)<0|\bar{u}u+\bar{d}d|0>\; .
\ee
Sie verbindet die hadronischen Parameter $m_\pi$ und $f_\pi$ mit
den Quarkmassen und dem Ordnungsparameter f\"ur die spontane 
Brechung der chiralen Symmetrie, dem Quarkkondensat $<\! 0|\bar uu
+\bar dd|0\!>$. Wir haben in der Herleitung von dem Grenz\"ubergang 
$q_\mu \to 0$ Gebrauch gemacht. Die GOR-Relation ergibt daher 
eine Aussage \"uber die Pionzerfallskonstante im chiralen Limes.
F\"ur den physikalischen Wert von $f_\pi$ ergibt die
GOR-Relation daher nur den f\"uhrenden Term in einer Entwicklung
in Potenzen der Quarkmasse.     

Die Wardidentit\"at (\ref{wi}) gilt unabh\"angig von den 
physikalischen Zust\"anden, zwischen denen die Matrixelemente
genommen sind. Ersetzt man die Vakuumzust\"ande durch ein und
auslaufende Nukleonen, so l\"a\ss t sich die Divergenzamplitude 
$\psi_5^{ab}$  mit der Pion-Nukleon-Streumatrix
$T_{\pi N}^{ab}$ identifizieren. Auf diese
Weise k\"onnen wir die Rolle der expliziten Symmetriebrechung
in physikalischen Streuprozessen studieren. Durch zweimaliges
Differenzieren der Zweipunktfunktion $\Pi^{ab}_{5\,\mu\nu}$ ergibt sich 
folgende Wardidentit\"at f\"ur $T_{\pi N}^{ab}$ \cite{BPP71}:
\beq
\label{tpin}
 T_{\pi N}^{ab}(p_1,q_1;p_2,q_2) &=& T_{PV}^{ab}(p_1,q_1;p_2,q_2)
   +\frac{q_1^2+q_2^2-m_\pi^2}{f_\pi^2 m_\pi^2} \Sigma^{ab}(p_1,p_2) \\
   & &\mbox{} +\frac{1}{2f_\pi^2} (q_1+q_2)^\mu V_\mu^{ab}(p_1,p_2)
   +q_1^\mu q_2^\nu R^{ab}_{\mu\nu}(p_1,q_1;p_2,q_2)\, . \nonumber
\eeq
Dabei haben wir $q_1^\mu q_2^\nu\Pi^{ab}_{5\,\mu\nu}$ in die 
Beitr\"age der Pseudovektor-Bornterme $T_{PV}^{ab}$ und die 
Hintergrundamplitude $q_1^\mu q_2^\nu R_{\mu\nu}^{ab}$ zerlegt.
Sie enth\"alt weder Nukleon- noch Pionpole
und verschwindet daher im Niederenergielimes $q_1,q_2 \to 0$.
Des weiteren bezeichnet $V_\mu^{ab}(p_1,p_2)$ den Stromalgebraterm
\be
 V_\mu^{ab}(p_1,p_2) = i\epsilon^{abc}<N(p_2)|V_\mu^c(0)|N(p_1)>\, .
\ee
Unser spezielles Augenmerk liegt auf dem Pion-Nukleon Sigmaterm
$\Sigma^{ab}(p_1,p_2)$, der ein Ma\ss\ ist f\"ur die St\"arke der 
expliziten Symmetriebrechung. Wie bei der Herleitung der 
GOR-Relation ergibt sich
\be
\label{pinsig}
\Sigma^{ab}(p_1,p_2) = \frac{\delta^{ab}}{2}(m_u+m_d)
    <N(p_2)|\bar{u}u+\bar{d}d|N(p_1)>\; .
\ee
Der Sigmaterm liefert an der Schwelle den f\"uhrenden Beitrag zum
isospinsymmetrischen Teil der Streuamplitude. Dar\"uber hinaus kann
man den  Wert von
$\Sigma^{ab}(p_1,p_2)$ am unphysikalischen Punkt $p_1=p_2$
als Beitrag der Strommassen der leichten Quarks zur Nukleonmasse
interpretieren. Diese Gr\"o\ss e l\"a\ss t sich prinzipiell aus dem
beobachteten Baryonspektrum ermitteln. In chiraler St\"orungstheorie
findet man \cite{GL80}
\be
 \sigma =\frac{1}{2}(m_u+m_d)<p|\bar{u}u+\bar{d}d|p>
    = \frac{35\pm 5}{1+y}\; {\rm MeV}\, ,              
\ee
wobei $y=<p|\bar{s}s|p>/<p|\bar{u}u|p>$ das Verh\"altnis des
Kondensats der seltsamen Quarks zu dem  der leichten Quarks im
Nukleon bezeichnet. Um den Wert von $\sigma$ aus
Pion-Nukleon Streuphasen zu extrahieren, ist ein aufwendiges
Extrapolationsverfahren notwendig. Im Gegensatz zu fr\"uheren
Auswertungen liefern neuere Dispersionsanalysen einen 
Wert $\sigma=45 \pm 5$ MeV \cite{GLS91}, der vertr\"aglich
ist mit der Annahme eines verschwindenden Kondensats seltsamer
Quarks im Nukleon.

\section{Ableitung des Niederenergietheorems}
Die Herleitung von Niederenergietheoremen zur Pionphotoproduktion
verl\"auft analog zu der im letzten Abschnitt geschilderten Ableitung 
der GOR-Relation. Ausgangspunkt ist die Darstellung der Streumatrix mit Hilfe
der LSZ-Reduktionsformel
\beq
\label{LSZ}
 S^{a} &=& -(2\pi)^4 \,\delta^4 (p_1+k-p_2+q)\, Z_\gamma^{-1/2}
   Z_\pi^{-1/2} \\
   & & \mbox{}\cdot \int d^4x\, e^{iq\cdot x} (\Box +m_\pi^2)
   <N(p_2)|T\left(\epsilon^\mu V_\mu^{em}(0) \phi^{a}(x)\right)|N(p_1)>
   \nonumber\; .
\eeq
Dabei bezeichnet $\phi^{a}$ das kanonische Pionfeld, $Z_\pi=(2\pi)^3
2\omega_\pi$ dessen kovariante Normierung und $Z_\gamma=(2\pi)^3
2\omega$ die Normierung des elektromagnetischen Feldes. Die
\"Ubergangsmatrix nach der Definition aus dem ersten Kapitel ist
durch
\be
\label{deft}
 S^{a} = i(2\pi)^4\,\delta^4 (p_1+k-p_2+q) Z_\gamma^{-1/2}
  Z_\pi^{-1/2} \epsilon^\mu T_\mu^{a}
\ee
gegeben. Wir betrachten  die Zweipunktfunktion
\be
\label{Pimunu}
\overline{\Pi}^a_{\mu\nu}(q) = \int d^4 x\, e^{iq\cdot x}<N(p_2)| 
T\left( V_\mu^{em} (0) B_\nu^{a}(x) \right) |N(p_1)> \; .
\ee
des elektromagnetischen Stroms $V_\mu^{em}$ und des 'transversalen` Axialstroms
\be
B_\mu^{a}(x) =A_\mu^{a}(x)+\frac{1}{m_\pi^2}\partial_\mu D^{a}(x)
\ee
wobei $D^{a}(x)=\partial^\mu A_\mu^{a}(x)$ die Divergenz des Axialstroms
bezeichnet. Mit Hilfe der PCAC-Relation und der Definition der
Pionquellfunktion findet man
\be
\label{defb}
\partial^\mu B_\mu^a (x) = (\Box +m_\pi^2)\phi^{a}(x) =-j_\pi^{a}(x)\, .
\ee
Einmaliges Differenzieren des zeitgeordneten Produktes in der 
Zweipunktfunktion $\overline{\Pi}_{\mu\nu}$ liefert schlie\ss lich 
die gesuchte Wardidentit\"at f\"ur $T_\mu^{a}$
\be
\label{avward}
T_\mu^a (q) = \frac{1}{f_\pi}\left\{
iq^\nu \overline{\Pi}_{\mu\nu}^a (q) \, - \, C_\mu^a (q)  \, - \,
\frac{i \omega_\pi}{m_\pi^2} \Sigma^a_\mu (q) \right\} \; .
\ee
Dabei haben wir die LSZ-Formel (\ref{LSZ}) verwendet, um die 
Photoproduktionsamplitude $T_\mu^{a}$ zu identifizieren. Die Wirkung 
der Ableitung auf den Zeitordnungsoperator liefert die Kommutatoren
\beq
\label{cmua}
 C_\mu^{a}(q) &=& \int d^4x\, e^{iq\cdot x}\delta (x^0)
   <N(p_2)|[A^{a}_0(x),V_\mu^{em}(x)]|N(p_1)> \\
\label{sig}     
 \Sigma_\mu^{a}(q) &=& \int d^4x\, e^{iq\cdot x}\delta (x^0)
   <N(p_2)|[D^{a}(x),V_\mu^{em}(x)]|N(p_1)>\;
\eeq
wobei wir das Resultat in den Stromalgebraterm $C_\mu^{a}$
und den Kommutator der Divergenz des Axialstroms zerlegt haben.  
In Analogie zum Sigmaterm in der Pion-Nukleon-Streuung bezeichnet
man diesen Beitrag auch als $(\gamma,\pi)$-Sigmaterm. Wie der
$\pi N$-Sigmaterm liefert er eine zus\"atzliche Korrektur, die direkt
proportional zu den Stromquarkmassen in der QCD-Lagrangedichte ist.

Es ist instruktiv, die Wardidentit\"at zu studieren, 
die sich aus der Zweipunktfunktion $\Pi^{a}_{\mu\nu}$ des
\"ublichen Axialstroms ergibt. Analog zu (\ref{avward})
erh\"alt man
\be
\label{avward2}
\frac{m_\pi^2}{q^2-m_\pi^2} T_\mu^a (q) = \frac{1}{f_\pi}\Big\{
iq^\nu \Pi_{\mu\nu}^a (q) \, - \, C_\mu^a \Big\} \; .
\ee
Auf Grund des Pionpropagators vor der Amplitude $T_\mu^{a}$
liefert diese Beziehung die Photoproduktionsamplitude zun\"achst
nur am unphysikalischen ''weichen`` Punkt $q_\mu=0$.  Um die physikalische 
Schwelle $q^2=m_\pi^2$ zu erreichen, ist es notwendig, den Pionpol
explizit abzuseparieren. Diesem Zweck dient der transversale Axialstrom
$B_\mu^{a}$. Tats\"achlich l\"a\ss t sich $B_\mu^{a}$ mit Hilfe der
PCAC Relation als der nichtpionische Anteil des Axialstroms interpretieren
\be
 B_\mu^{a}(x) = A_\mu^{a}(x) +f_\pi\partial_\mu \phi^{a}(x)
    = A_\mu^{a}(x) - A_{\mu}^{a\, (\pi)} (x) \, .
\ee
Wir wollen nun die verschiedenen Beitr\"age zur Photoproduktionsamplitude
$T_\mu^{a}$ im Einzelnen studieren. Beginnen werden wir dabei mit
der Zweipunktfunktion $q^\nu\overline{\Pi}_{\mu\nu}^{a}$. Da dieser
Term proportional zum Impuls $q$ ist, tragen im Grenzfall weicher Pionen
nur die Polterme in $\overline{\Pi}_{\mu\nu}^{a}$ zur Amplitude bei.
Die Summe der Nukleonpolterme im direkten und im Austauschkanal
lautet
\beq
\label{nborn}
f_\pi T_\mu^{a\,(N)} &=& \bar{u}(p_2) \Big( iq^\nu \Gamma_\nu^{B^{a}}
   (p_1-k,-q,p_2) S_F(p_1+k) \Gamma_\mu^\gamma (p_1,k,p_1+k)
      \\[0.2cm]
   & & \hspace{0.5cm} \mbox{}+ \Gamma_\mu^\gamma (p_1-q,k,p_2)
   S_F(p_1-q) iq^\nu \Gamma_\mu^{B^{a}}(p_1,-q,p_1-q) \Big) u(p_1)
   \; .\nonumber
\eeq      
Dabei bezeichnet $S_F(p)$ den Nukleonpropagator und $\Gamma_\mu^\gamma$
bzw. $\Gamma_\nu^{B^{a}}$ die Vertexfunktionen f\"ur die Kopplung
des Nukleons an das elektromagnetische Feld und den Axialstrom $B_\nu^{a}$.
In den Poltermen sind alle Teilchen mit Ausnahme des
intermedi\"aren Nukleons auf der Massenschale. Die allgemeine Gestalt der
Vertexfunktion lautet in diesem Fall
\beq
\label{emvert}
\Gamma_\mu^\gamma (p_1,k,p_1+k)u(p_1) &=& \left( \gamma_\mu F_1 
   +  \frac{M+(p_1+k)\cdot\gamma}{2M} \frac{i\sigma_\mu\nu k^\nu}{2M}F_2^+ 
   \right. \\
 & & \hspace{1.5cm}\left. \mbox{}
   +  \frac{M-(p_1+k)\cdot\gamma}{2M} \frac{i\sigma_\mu\nu k^\nu}{2M}F_2^-
   \right) u(p_1)  \nonumber \\
\label{bavert}
\Gamma_\nu^{B^{a}} (p_1,-q,p_1-q)u(p_1) &=& \left( \gamma_\nu \overline{G}_A 
   +  \frac{M+(p_1-q)\cdot\gamma}{2M} \frac{q_\nu}{2M} \overline{G}_P^{\, +}
   \right. \\
  & & \hspace{1.5cm}\left. \mbox{}  
   +  \frac{M-(p_1-q)\cdot\gamma}{2M} \frac{q_\nu}{2M} \overline{G}_P^{\, -} 
   \right)\gamma_5\frac{\tau^{a}}{2} u(p_1) \; .\nonumber
\eeq     
Analoge Ausdr\"ucke ergeben sich, wenn der Impuls $p_2$ die 
Massenschalenbedingung erf\"ullt. Die Formfaktoren $F_i$ sind Funktionen
des Impuls\"ubertages $k^2$ und des off-shell Parameters $\delta^2=(p_1+k)^2
-M^2$. Ihre Isospinstruktur lautet
\be
 F_i=F_i^{s}+F_i^{v}\tau^3 \; .
\ee
Entsprechend h\"angen die Formfaktoren am $NNB^{a}$-Vertex von den 
Variablen $q^2$ und ${\delta '}^2=(p_1-q)^2-M^2$ ab. Wir haben diese 
Formfaktoren mit einem Querstrich gekennzeichnet, um sie von den 
entsprechenden Funktionen am  Vertex des Axialvektorstroms zu 
unterscheiden.

Die Abh\"angigkeit der Photoproduktionsamplitude vom off-shell
Verhalten der Formfaktoren wurde in einer Arbeit von Naus, Koch
und Friar \cite{NKF90} untersucht. Die Autoren zeigen, da\ss\ 
off-shell Korrekturen erst in derselben Ordnung in $m_\pi$ 
wie andere modellabh\"angige Korrekturen auftreten. Wir verwenden
daher von nun an die on-shell Vertices
\beq
\Gamma_\mu^\gamma &=& \gamma_\mu F_1 + 
          \frac{i\sigma_{\mu\nu}k^\nu}{2M} F_2 \\
\Gamma_\nu^{B^{a}}&=& \left( \gamma_\mu \overline{G}_A
         + \frac{q_\mu}{2M}\overline{G}_P \right) \gamma_5 
	 \frac{\tau^{a}}{2}
\eeq
wobei die auftretenden Formfaktoren nur mehr Funktionen der
Impuls\"ubertr\"age $k^2$ bzw.~$q^2$ sind. F\"ur reelle
Photonen ist
\be
\begin{array}{rclcrcl}
  F_1^{s}&=& 1/2        &\hspace{1cm}& F_1^{v}&=& 1/2     \\[0.2cm]
  F_2^{s}&=&\kappa^s    &            & F_2^{v}&=&\kappa^v  
\end{array}
\ee
mit den anomalen magnetischen Momenten $\kappa^{s,v}=\frac{1}{2}
(\kappa_p\pm \kappa_n)$. Die Vertexfunktionen f\"ur die beiden 
Str\"ome $A_\nu^{a}$ und $B_\nu^{a}$ unterscheiden sich nur 
um die Matrixelemente des pionischen Beitrags 
$A_\nu^{a(\pi)}=-f_\pi \partial_\nu \phi^{a}$. 
Dieser Term erzeugt den Pionpol im induzierten 
pseudoskalaren Formfaktor
\be
  G_P^{\pi -Pol} (q^2)=\frac{4Mf_\pi}{m_\pi^2-q^2} G_{\pi NN}(q^2)
\ee
wobei $G_{\pi NN}$ den Pion-Nukleon Formfaktor
\be
  <N(p_2)|j_\pi^{a}(0)|N(p_1)> = G_{\pi NN}(t) \bar{u}(p_2)i\gamma_5
         \tau^{a}u(p_1)
\ee
bezeichnet. Der Zusammenhang der Formfaktoren an den Vertices ist 
daher durch $\overline{G}_A=G_A$ und $\overline{G}_P=G_P-G_P^{\pi -Pol}$ 
gegeben. Empirische Untersuchungen zeigen, da\ss\ $G_P$ in sehr guter 
N\"aherung durch den Polterm beschrieben wird. Wir setzen daher 
$\overline{G}_P=0$ und erhalten
\beq
\label{nborn2}
f_\pi T_\mu^{a\,(N)} &=& \bar{u}(p_2) \left\{ g_A iq\cdot\gamma \gamma_5 
 \,\frac{\tau^{a}}{2} \frac{i}{(p_1+k)\cdot\gamma -M} \,\Gamma_\mu^\gamma
 \right. \\
 & & \hspace{1cm}\left. \mbox{} + \Gamma_\mu^\gamma 
     \,\frac{i}{(p_1-q)\cdot\gamma -M}\,
  g_A iq\cdot\gamma \gamma_5 \frac{\tau^{a}}{2} \right\} u(p_1), \nonumber 
\eeq
wobei wir daueber hinaus den axialen Formfaktor $G_A(q^2)$ durch den
Wert beim Impuls\"ubertrag $g_A=G_A(q^2=0)$ ersetzt haben. Die axiale
Kopplung l\"a\ss t sich mit Hilfe der Goldberger-Treiman Relation
\be
\label{GT}
\frac{g_A}{2f_\pi} = \frac{f}{m_\pi}
\ee
durch die pseudovektorielle Pion-Nukleon Kopplungskonstante $f$ ausdr\"ucken.
Das Resultat (\ref{nborn2}) entspricht daher der Born-Approximation f\"ur
eine effektive Pion-Nukleon Lagrangedichte mit dem Kopplungsterm
\be
\label{pv}
{\cal L} = \frac{f}{m_\pi} \bar{\psi}\gamma_5\gamma_\mu \tau^{a}\psi
   \partial^\mu \phi^{a}\; .
\ee    
Die geschilderte Herleitung f\"uhrt also in nat\"urlicher Weise
auf eine pseudovektorielle Kopplung des Pions an das Nukleon. Dies 
steht im Gegensatz zu vielen klassischen Arbeiten, in denen 
gew\"ohnlich mit einer pseudoskalaren Kopplung gerechnet wird.
Um Konsistenz mit der Wardidentit\"at (\ref{avward}) zu erzielen,
m\"ussen in diesem Fall zus\"atzliche Korrekturterme zur Bornamplitude
addiert werden.

Der Beitrag des Kommutators $C_\mu^{a}$ l\"a\ss t sich mit Hilfe 
der Stromalgebraregeln berechnen
\be
\label{curcom}
 C_\mu^{a} = -\epsilon^{a3c} <N(p_2)|A_\mu^{c}(0)|N(p_1)>\; .
\ee
Vernachl\"assigt man den Hintergrundbeitrag im induzierten 
pseudoskalaren Formfaktor, so ergibt sich
\be
\label{kr}
\epsilon^\mu C_\mu^{a} = -\epsilon^{a3c} \bar{u}(p_2)
  \left\{ G_A(t)\epsilon\cdot\gamma + G_{\pi NN}(t) f_\pi  
   \frac{\epsilon\cdot (k-q)}{m_\pi^2-t} \right\}
   \gamma_5 \tau^{c}u(p_1) 
\ee   	 	  
als Funktion der Mandelstamvariable $t=(q-k)^2$.
Das Resultat ist antisymmetrisch in den Isospinindices und
tr\"agt daher nur zur Produktion geladener Pionen bei. 
Der erste Term liefert den f\"uhrenden Beitrag zur 
Pho\-to\-pro\-duk\-ti\-ons\-amplitude im Grenzfall weicher Pionen
\be
\label{krtheo}
\lim_{q,k\to 0} T^{a}(q) =\frac{g_A}{f_\pi}\epsilon^{a3c}
   \bar{u}(p_2) \epsilon\cdot\gamma\gamma_5\tau^{a}u(p_1)\; ,
\ee
den sogenannten Kroll-Ruderman-Term \cite{KR54}. 

Der zweite Kommutator enth\"alt die Divergenz des Axialstroms
und l\"a\ss t sich daher mit Ausnahme der Zeitkomponente 
$\Sigma_0^{a}$ nicht modellunabh\"angig bestimmen. Mit Hilfe
des Stromalgebraresultats
\be
 \,[Q_5^{a},V_\mu^{em}(0)]=-i\epsilon^{a3c}A_\mu^{c}(0)
\ee
und der Erhaltung des elektromagnetischen Stroms findet man
\be
\label{sig0}
  \int d^4x \,\delta (x^0)\, [\partial^\mu A_\mu^{a}(x),V_0^{em}
  (0)] = -i\epsilon^{a3c} \partial^\mu A_\mu^{c} (0) \; .
\ee     
Das Matrixelement der Divergenz des Axialstroms zwischen
Nukleonzust\"anden $|N(p)>$ ist die axialen Formfaktoren des
Nukleons bestimmt:
\be
<N(p_2)| D^{a}(x) |N(p_1)> = \bar{u}(p_2) \left[ M G_A (t)
  + \frac{t}{4M} G_P(t) \right] \gamma_5 \tau^{a} u(p_1) \; .
\ee
Ber\"ucksichtigt man wie oben nur den f\"uhrenden Pionbeitrag,
so l\"a\ss t sich die Summe der beiden Kommutatoren 
an der Schwelle in die Form
\be
 \epsilon^\mu T_\mu^{a\,(\pi)}  = \epsilon^{a3c} g_{\pi NN}
   \,\frac{\epsilon\cdot (k-2q)}{m_\pi^2 -t} \,
   \bar{u}(p_2)\gamma_5 \tau^{a} u(p_1)
\ee
bringen. 
Die Raumkomponenten des Sigmaterms $\Sigma_\mu^{a}$ enthalten die 
Information \"uber die explizite Brechung der chiralen Symmetrie 
durch die Quarkmassen in der QCD-Lagrangedichte. Ihre Bestimmung
ist jedoch modellabh\"angig und wird uns in den n\"achsten 
Abschnitten noch besch\"aftigen. Vernachl\"assigt man diese
Korrektur, so ergeben die oben diskutierten Beitr\"age (\ref{nborn2},
\ref{kr},\ref{sig0}) folgende Bestimmung der invarianten Amplituden       
\beq
\label{letamp}
A^{(+0,-)}_1 &=&  \frac{2f}{\mu} \spm
      \left\{ -\frac{1}{\nu+\nu_1} \mp \frac{1}{\nu-\nu_1} 
      + \frac{1\mp 1}{\nu_1} \right\} \\
A^{(+0,-)}_2 &=&  \frac{2f}{\mu} \spm
      \left\{ -\frac{2}{\nu+\nu_1} \pm \frac{2}{\nu-\nu_1} \right\} \\      
A^{(+0,-)}_3 &=& \; \frac{2f}{\mu} \kappa \; (-1 \pm 1)   \\
A^{(+0,-)}_4 &=& \frac{2f}{\mu}\;\kappa
      \left\{ \frac{2}{\nu+\nu_1} \pm \frac{2}{\nu-\nu_1} \right\} \\ 
A^{(+0,-)}_5 &=&  \frac{2f}{\mu}\;\kappa
      \left\{ \frac{4}{\nu+\nu_1} \mp \frac{4}{\nu-\nu_1} \right\} \\       
A^{(+0,-)}_6 &=&  \frac{2f}{\mu}(1+2\kappa)
      \left\{ \frac{1}{\nu+\nu_1} \mp \frac{1}{\nu-\nu_1} \right\} 
      + \frac{2f}{\mu} \kappa\, (1\pm 1) \; .
\eeq
Dabei haben wir der \"Ubersichtlichkeit halber den Isospinindex
der anomalen magnetischen Momente unterdr\"uckt. Es gilt
$\kappa^{(\pm)}=\kappa^v$ und $\kappa^{(0)}=\kappa^s$. 
Mit Hilfe der im ersten Kapitel abgeleiteten Formel f\"ur die
Schwellenamplitude,
\be
 \left. E_{0+}\right|_{thr} = \frac{e}{16\pi M}
 \frac{2+\mu}{(1+\mu)^{3/2}} \, \left. \left(
   A_3 + \frac{\mu}{2} A_6 \right) \right|_{thr}
\ee                
und der in Anhang A diskutierten Kinematik
ergibt sich schlie\ss lich folgende Bestimmung der elektrischen
Dipolamplitude f\"ur die vier physikalischen Kan\"ale 
\beq
\label{LET}
\Epn &=& \frac{e}{4\pi} \frac{\sqrt{2}f}{m_\pi}
    \left\{ 1 - \frac{3}{2}\mu + {\cal O}(\mu^2) \right\}
    \cong 26.6 \su \\[0.1cm]
\Emp &=& \frac{e}{4\pi} \frac{\sqrt{2}f}{m_\pi}
     \left\{ -1 + \frac{1}{2}\mu + {\cal O}(\mu^2) \right\}
    \cong -31.7 \su \\[0.1cm]
\Eop &=& \frac{e}{4\pi} \frac{f}{m_\pi}
     \left\{ -\mu + \frac{\mu^2}{2}(3+\kappa_p ) +
  {\cal O}(\mu^3) \right\}    \cong -2.32
  \su \\[0.1cm]
\Eon &=& \frac{e}{4\pi} \frac{f}{m_\pi}
     \left\{  \frac{\mu^2}{2}\kappa_n  +
  {\cal O}(\mu^3) \right\}  \cong -0.51 \su 
\eeq
wobei wir die Werte $\kappa_p=1.79$ und $\kappa_n=-1.91$ verwendet 
haben. Dieses Resultat liefert den Inhalt des Niederenergietheorems
\cite{Bae70,VZ72}. Die relative Ordnung der nicht bestimmten
Korrekturen wurde mit Hilfe verschiedener Annahmen \"uber
das Verhalten der Untergrundamplitude festgelegt. Wir werden
diese Annahmen und ihre Rechtfertigung im n\"achsten Abschnitt
diskutieren.

\begin{figure}
\label{feyn}
\caption{Diagrammatische Darstellung der f\"uhrenden Beitr\"age zur
Pionphotoproduktionsamplitude}
\vspace{8.5cm}
\end{figure}

Niederenergietheoreme zur Pionphotoproduktion lassen sich auch
direkt aus der Bestimmung der Bornterme in effektiven 
chiralen Meson-Nukleon Theorien ableiten \cite{Pec69}. Die
entsprechende Lagrangedichte unter Einbeziehung der 
elektromagnetischen Wechselwirkung lautet
\beq
\label{leff}
{\cal L} & =& \bar{\psi}(i\gamma\cdot{\cal D}-M)\psi 
  +\frac{1}{2}({\cal D}_\mu\phi^{a})^2 - \frac{1}{2}m_\pi^2
  (\phi^{a})^2  \\
 & & \mbox{} + \frac{f}{m_\pi} \bar{\psi}\gamma_5 \gamma_\mu
 \tau^{a} {\cal D}^\mu \phi^{a}\psi 
  + \frac{e}{4m}\bar{\psi} (\kappa^s +\kappa^v \tau^3)
  \sigma_{\mu\nu}\psi F^{\mu\nu} \nonumber 
\eeq
wobei ${\cal D}_\mu=\partial_\mu+iQ{\cal A}_\mu$ die kovariante
Ableitung, ${\cal A}_\mu$ das elektromagnetische Potential und
$Q$ den Ladungsoperator bezeichnet. Die Eichung der pseudovektoriellen
Pion-Nukleon-Kopplung erzeugt eine $\gamma\pi NN$-Kontaktwechselwirkung,
\be
{\cal L}_{\gamma\pi NN} = \frac{ef}{m_\pi}\epsilon^{3ab}
  \bar{\psi}\gamma_5 \gamma_\mu \tau^{a}\psi {\cal A}^\mu \phi^b
\ee  
die im Rahmen der effektiven Theorie die Kroll-Ruderman-Amplitude
liefert. Die Wirkung der kovarianten Ableitung auf das Pionfeld
bestimmt die Kopplung des elektromagnetischen Feldes an die
geladenen Pionen. Die $\gamma\pi\pi$-Wechselwirkung 
\be  
{\cal L}_{\gamma\pi\pi} = e\epsilon^{3ab}\phi^{a}\partial_\mu
 \phi^{b} {\cal A}^\mu
\ee
bewirkt schlie\ss lich die bereits angesprochene Struktur des
Pionpolterms. 
   
    
\section{Abs

%\section{Ableitung des Niederenergietheorems}
Die Herleitung von Niederenergietheoremen zur Pionphotoproduktion
verl\"auft analog zu der im letzten Abschnitt geschilderten Ableitung 
der GOR-Relation. Ausgangspunkt ist die Darstellung der Streumatrix mit Hilfe
der LSZ-Reduktionsformel
\beq
\label{LSZ}
 S^{a} &=& -(2\pi)^4 \,\delta^4 (p_1+k-p_2+q)\, Z_\gamma^{-1/2}
   Z_\pi^{-1/2} \\
   & & \mbox{}\cdot \int d^4x\, e^{iq\cdot x} (\Box +m_\pi^2)
   <N(p_2)|T\left(\epsilon^\mu V_\mu^{em}(0) \phi^{a}(x)\right)|N(p_1)>
   \nonumber\; .
\eeq
Dabei bezeichnet $\phi^{a}$ das kanonische Pionfeld, $Z_\pi=(2\pi)^3
2\omega_\pi$ dessen kovariante Normierung und $Z_\gamma=(2\pi)^3
2\omega$ die Normierung des elektromagnetischen Feldes. Die
\"Ubergangsmatrix nach der Definition aus dem ersten Kapitel ist
durch
\be
\label{deft}
 S^{a} = i(2\pi)^4\,\delta^4 (p_1+k-p_2+q) Z_\gamma^{-1/2}
  Z_\pi^{-1/2} \epsilon^\mu T_\mu^{a}
\ee
gegeben. Wir betrachten  die Zweipunktfunktion
\be
\label{Pimunu}
\overline{\Pi}^a_{\mu\nu}(q) = \int d^4 x\, e^{iq\cdot x}<N(p_2)| 
T\left( V_\mu^{em} (0) B_\nu^{a}(x) \right) |N(p_1)> \; .
\ee
des elektromagnetischen Stroms $V_\mu^{em}$ und des 'transversalen` Axialstroms
\be
B_\mu^{a}(x) =A_\mu^{a}(x)+\frac{1}{m_\pi^2}\partial_\mu D^{a}(x)
\ee
wobei $D^{a}(x)=\partial^\mu A_\mu^{a}(x)$ die Divergenz des Axialstroms
bezeichnet. Mit Hilfe der PCAC-Relation und der Definition der
Pionquellfunktion findet man
\be
\label{defb}
\partial^\mu B_\mu^a (x) = (\Box +m_\pi^2)\phi^{a}(x) =-j_\pi^{a}(x)\, .
\ee
Einmaliges Differenzieren des zeitgeordneten Produktes in der 
Zweipunktfunktion $\overline{\Pi}_{\mu\nu}$ liefert schlie\ss lich 
die gesuchte Wardidentit\"at f\"ur $T_\mu^{a}$
\be
\label{avward}
T_\mu^a (q) = \frac{1}{f_\pi}\left\{
iq^\nu \overline{\Pi}_{\mu\nu}^a (q) \, - \, C_\mu^a (q)  \, - \,
\frac{i \omega_\pi}{m_\pi^2} \Sigma^a_\mu (q) \right\} \; .
\ee
Dabei haben wir die LSZ-Formel (\ref{LSZ}) verwendet, um die 
Photoproduktionsamplitude $T_\mu^{a}$ zu identifizieren. Die Wirkung 
der Ableitung auf den Zeitordnungsoperator liefert die Kommutatoren
\beq
\label{cmua}
 C_\mu^{a}(q) &=& \int d^4x\, e^{iq\cdot x}\delta (x^0)
   <N(p_2)|[A^{a}_0(x),V_\mu^{em}(x)]|N(p_1)> \\
\label{sig}     
 \Sigma_\mu^{a}(q) &=& \int d^4x\, e^{iq\cdot x}\delta (x^0)
   <N(p_2)|[D^{a}(x),V_\mu^{em}(x)]|N(p_1)>\;
\eeq
wobei wir das Resultat in den Stromalgebraterm $C_\mu^{a}$
und den Kommutator der Divergenz des Axialstroms zerlegt haben.  
In Analogie zum Sigmaterm in der Pion-Nukleon-Streuung bezeichnet
man diesen Beitrag auch als $(\gamma,\pi)$-Sigmaterm. Wie der
$\pi N$-Sigmaterm liefert er eine zus\"atzliche Korrektur, die direkt
proportional zu den Stromquarkmassen in der QCD-Lagrangedichte ist.

Es ist instruktiv, die Wardidentit\"at zu studieren, 
die sich aus der Zweipunktfunktion $\Pi^{a}_{\mu\nu}$ des
\"ublichen Axialstroms ergibt. Analog zu (\ref{avward})
erh\"alt man
\be
\label{avward2}
\frac{m_\pi^2}{q^2-m_\pi^2} T_\mu^a (q) = \frac{1}{f_\pi}\Big\{
iq^\nu \Pi_{\mu\nu}^a (q) \, - \, C_\mu^a \Big\} \; .
\ee
Auf Grund des Pionpropagators vor der Amplitude $T_\mu^{a}$
liefert diese Beziehung die Photoproduktionsamplitude zun\"achst
nur am unphysikalischen ''weichen`` Punkt $q_\mu=0$.  Um die physikalische 
Schwelle $q^2=m_\pi^2$ zu erreichen, ist es notwendig, den Pionpol
explizit abzuseparieren. Diesem Zweck dient der transversale Axialstrom
$B_\mu^{a}$. Tats\"achlich l\"a\ss t sich $B_\mu^{a}$ mit Hilfe der
PCAC Relation als der nichtpionische Anteil des Axialstroms interpretieren
\be
 B_\mu^{a}(x) = A_\mu^{a}(x) +f_\pi\partial_\mu \phi^{a}(x)
    = A_\mu^{a}(x) - A_{\mu}^{a\, (\pi)} (x) \, .
\ee
Wir wollen nun die verschiedenen Beitr\"age zur Photoproduktionsamplitude
$T_\mu^{a}$ im Einzelnen studieren. Beginnen werden wir dabei mit
der Zweipunktfunktion $q^\nu\overline{\Pi}_{\mu\nu}^{a}$. Da dieser
Term proportional zum Impuls $q$ ist, tragen im Grenzfall weicher Pionen
nur die Polterme in $\overline{\Pi}_{\mu\nu}^{a}$ zur Amplitude bei.
Die Summe der Nukleonpolterme im direkten und im Austauschkanal
lautet
\beq
\label{nborn}
f_\pi T_\mu^{a\,(N)} &=& \bar{u}(p_2) \Big( iq^\nu \Gamma_\nu^{B^{a}}
   (p_1-k,-q,p_2) S_F(p_1+k) \Gamma_\mu^\gamma (p_1,k,p_1+k)
      \\[0.2cm]
   & & \hspace{0.5cm} \mbox{}+ \Gamma_\mu^\gamma (p_1-q,k,p_2)
   S_F(p_1-q) iq^\nu \Gamma_\mu^{B^{a}}(p_1,-q,p_1-q) \Big) u(p_1)
   \; .\nonumber
\eeq      
Dabei bezeichnet $S_F(p)$ den Nukleonpropagator und $\Gamma_\mu^\gamma$
bzw. $\Gamma_\nu^{B^{a}}$ die Vertexfunktionen f\"ur die Kopplung
des Nukleons an das elektromagnetische Feld und den Axialstrom $B_\nu^{a}$.
In den Poltermen sind alle Teilchen mit Ausnahme des
intermedi\"aren Nukleons auf der Massenschale. Die allgemeine Gestalt der
Vertexfunktion lautet in diesem Fall
\beq
\label{emvert}
\Gamma_\mu^\gamma (p_1,k,p_1+k)u(p_1) &=& \left( \gamma_\mu F_1 
   +  \frac{M+(p_1+k)\cdot\gamma}{2M} \frac{i\sigma_\mu\nu k^\nu}{2M}F_2^+ 
   \right. \\
 & & \hspace{1.5cm}\left. \mbox{}
   +  \frac{M-(p_1+k)\cdot\gamma}{2M} \frac{i\sigma_\mu\nu k^\nu}{2M}F_2^-
   \right) u(p_1)  \nonumber \\
\label{bavert}
\Gamma_\nu^{B^{a}} (p_1,-q,p_1-q)u(p_1) &=& \left( \gamma_\nu \overline{G}_A 
   +  \frac{M+(p_1-q)\cdot\gamma}{2M} \frac{q_\nu}{2M} \overline{G}_P^{\, +}
   \right. \\
  & & \hspace{1.5cm}\left. \mbox{}  
   +  \frac{M-(p_1-q)\cdot\gamma}{2M} \frac{q_\nu}{2M} \overline{G}_P^{\, -} 
   \right)\gamma_5\frac{\tau^{a}}{2} u(p_1) \; .\nonumber
\eeq     
Analoge Ausdr\"ucke ergeben sich, wenn der Impuls $p_2$ die 
Massenschalenbedingung erf\"ullt. Die Formfaktoren $F_i$ sind Funktionen
des Impuls\"ubertages $k^2$ und des off-shell Parameters $\delta^2=(p_1+k)^2
-M^2$. Ihre Isospinstruktur lautet
\be
 F_i=F_i^{s}+F_i^{v}\tau^3 \; .
\ee
Entsprechend h\"angen die Formfaktoren am $NNB^{a}$-Vertex von den 
Variablen $q^2$ und ${\delta '}^2=(p_1-q)^2-M^2$ ab. Wir haben diese 
Formfaktoren mit einem Querstrich gekennzeichnet, um sie von den 
entsprechenden Funktionen am  Vertex des Axialvektorstroms zu 
unterscheiden.

Die Abh\"angigkeit der Photoproduktionsamplitude vom off-shell
Verhalten der Formfaktoren wurde in einer Arbeit von Naus, Koch
und Friar \cite{NKF90} untersucht. Die Autoren zeigen, da\ss\ 
off-shell Korrekturen erst in derselben Ordnung in $m_\pi$ 
wie andere modellabh\"angige Korrekturen auftreten. Wir verwenden
daher von nun an die on-shell Vertices
\beq
\Gamma_\mu^\gamma &=& \gamma_\mu F_1 + 
          \frac{i\sigma_{\mu\nu}k^\nu}{2M} F_2 \\
\Gamma_\nu^{B^{a}}&=& \left( \gamma_\mu \overline{G}_A
         + \frac{q_\mu}{2M}\overline{G}_P \right) \gamma_5 
	 \frac{\tau^{a}}{2}
\eeq
wobei die auftretenden Formfaktoren nur mehr Funktionen der
Impuls\"ubertr\"age $k^2$ bzw.~$q^2$ sind. F\"ur reelle
Photonen ist
\be
\begin{array}{rclcrcl}
  F_1^{s}&=& 1/2        &\hspace{1cm}& F_1^{v}&=& 1/2     \\[0.2cm]
  F_2^{s}&=&\kappa^s    &            & F_2^{v}&=&\kappa^v  
\end{array}
\ee
mit den anomalen magnetischen Momenten $\kappa^{s,v}=\frac{1}{2}
(\kappa_p\pm \kappa_n)$. Die Vertexfunktionen f\"ur die beiden 
Str\"ome $A_\nu^{a}$ und $B_\nu^{a}$ unterscheiden sich nur 
um die Matrixelemente des pionischen Beitrags 
$A_\nu^{a(\pi)}=-f_\pi \partial_\nu \phi^{a}$. 
Dieser Term erzeugt den Pionpol im induzierten 
pseudoskalaren Formfaktor
\be
  G_P^{\pi -Pol} (q^2)=\frac{4Mf_\pi}{m_\pi^2-q^2} G_{\pi NN}(q^2)
\ee
wobei $G_{\pi NN}$ den Pion-Nukleon Formfaktor
\be
  <N(p_2)|j_\pi^{a}(0)|N(p_1)> = G_{\pi NN}(t) \bar{u}(p_2)i\gamma_5
         \tau^{a}u(p_1)
\ee
bezeichnet. Der Zusammenhang der Formfaktoren an den Vertices ist 
daher durch $\overline{G}_A=G_A$ und $\overline{G}_P=G_P-G_P^{\pi -Pol}$ 
gegeben. Empirische Untersuchungen zeigen, da\ss\ $G_P$ in sehr guter 
N\"aherung durch den Polterm beschrieben wird. Wir setzen daher 
$\overline{G}_P=0$ und erhalten
\beq
\label{nborn2}
f_\pi T_\mu^{a\,(N)} &=& \bar{u}(p_2) \left\{ g_A iq\cdot\gamma \gamma_5 
 \,\frac{\tau^{a}}{2} \frac{i}{(p_1+k)\cdot\gamma -M} \,\Gamma_\mu^\gamma
 \right. \\
 & & \hspace{1cm}\left. \mbox{} + \Gamma_\mu^\gamma 
     \,\frac{i}{(p_1-q)\cdot\gamma -M}\,
  g_A iq\cdot\gamma \gamma_5 \frac{\tau^{a}}{2} \right\} u(p_1), \nonumber 
\eeq
wobei wir daueber hinaus den axialen Formfaktor $G_A(q^2)$ durch den
Wert beim Impuls\"ubertrag $g_A=G_A(q^2=0)$ ersetzt haben. Die axiale
Kopplung l\"a\ss t sich mit Hilfe der Goldberger-Treiman Relation
\be
\label{GT}
\frac{g_A}{2f_\pi} = \frac{f}{m_\pi}
\ee
durch die pseudovektorielle Pion-Nukleon Kopplungskonstante $f$ ausdr\"ucken.
Das Resultat (\ref{nborn2}) entspricht daher der Born-Approximation f\"ur
eine effektive Pion-Nukleon Lagrangedichte mit dem Kopplungsterm
\be
\label{pv}
{\cal L} = \frac{f}{m_\pi} \bar{\psi}\gamma_5\gamma_\mu \tau^{a}\psi
   \partial^\mu \phi^{a}\; .
\ee    
Die geschilderte Herleitung f\"uhrt also in nat\"urlicher Weise
auf eine pseudovektorielle Kopplung des Pions an das Nukleon. Dies 
steht im Gegensatz zu vielen klassischen Arbeiten, in denen 
gew\"ohnlich mit einer pseudoskalaren Kopplung gerechnet wird.
Um Konsistenz mit der Wardidentit\"at (\ref{avward}) zu erzielen,
m\"ussen in diesem Fall zus\"atzliche Korrekturterme zur Bornamplitude
addiert werden.

Der Beitrag des Kommutators $C_\mu^{a}$ l\"a\ss t sich mit Hilfe 
der Stromalgebraregeln berechnen
\be
\label{curcom}
 C_\mu^{a} = -\epsilon^{a3c} <N(p_2)|A_\mu^{c}(0)|N(p_1)>\; .
\ee
Vernachl\"assigt man den Hintergrundbeitrag im induzierten 
pseudoskalaren Formfaktor, so ergibt sich
\be
\label{kr}
\epsilon^\mu C_\mu^{a} = -\epsilon^{a3c} \bar{u}(p_2)
  \left\{ G_A(t)\epsilon\cdot\gamma + G_{\pi NN}(t) f_\pi  
   \frac{\epsilon\cdot (k-q)}{m_\pi^2-t} \right\}
   \gamma_5 \tau^{c}u(p_1) 
\ee   	 	  
als Funktion der Mandelstamvariable $t=(q-k)^2$.
Das Resultat ist antisymmetrisch in den Isospinindices und
tr\"agt daher nur zur Produktion geladener Pionen bei. 
Der erste Term liefert den f\"uhrenden Beitrag zur 
Pho\-to\-pro\-duk\-ti\-ons\-amplitude im Grenzfall weicher Pionen
\be
\label{krtheo}
\lim_{q,k\to 0} T^{a}(q) =\frac{g_A}{f_\pi}\epsilon^{a3c}
   \bar{u}(p_2) \epsilon\cdot\gamma\gamma_5\tau^{a}u(p_1)\; ,
\ee
den sogenannten Kroll-Ruderman-Term \cite{KR54}. 

Der zweite Kommutator enth\"alt die Divergenz des Axialstroms
und l\"a\ss t sich daher mit Ausnahme der Zeitkomponente 
$\Sigma_0^{a}$ nicht modellunabh\"angig bestimmen. Mit Hilfe
des Stromalgebraresultats
\be
 \,[Q_5^{a},V_\mu^{em}(0)]=-i\epsilon^{a3c}A_\mu^{c}(0)
\ee
und der Erhaltung des elektromagnetischen Stroms findet man
\be
\label{sig0}
  \int d^4x \,\delta (x^0)\, [\partial^\mu A_\mu^{a}(x),V_0^{em}
  (0)] = -i\epsilon^{a3c} \partial^\mu A_\mu^{c} (0) \; .
\ee     
Das Matrixelement der Divergenz des Axialstroms zwischen
Nukleonzust\"anden $|N(p)>$ ist die axialen Formfaktoren des
Nukleons bestimmt:
\be
<N(p_2)| D^{a}(x) |N(p_1)> = \bar{u}(p_2) \left[ M G_A (t)
  + \frac{t}{4M} G_P(t) \right] \gamma_5 \tau^{a} u(p_1) \; .
\ee
Ber\"ucksichtigt man wie oben nur den f\"uhrenden Pionbeitrag,
so l\"a\ss t sich die Summe der beiden Kommutatoren 
an der Schwelle in die Form
\be
 \epsilon^\mu T_\mu^{a\,(\pi)}  = \epsilon^{a3c} g_{\pi NN}
   \,\frac{\epsilon\cdot (k-2q)}{m_\pi^2 -t} \,
   \bar{u}(p_2)\gamma_5 \tau^{a} u(p_1)
\ee
bringen. 
Die Raumkomponenten des Sigmaterms $\Sigma_\mu^{a}$ enthalten die 
Information \"uber die explizite Brechung der chiralen Symmetrie 
durch die Quarkmassen in der QCD-Lagrangedichte. Ihre Bestimmung
ist jedoch modellabh\"angig und wird uns in den n\"achsten 
Abschnitten noch besch\"aftigen. Vernachl\"assigt man diese
Korrektur, so ergeben die oben diskutierten Beitr\"age (\ref{nborn2},
\ref{kr},\ref{sig0}) folgende Bestimmung der invarianten Amplituden       
\beq
\label{letamp}
A^{(+0,-)}_1 &=&  \frac{2f}{\mu} \spm
      \left\{ -\frac{1}{\nu+\nu_1} \mp \frac{1}{\nu-\nu_1} 
      + \frac{1\mp 1}{\nu_1} \right\} \\
A^{(+0,-)}_2 &=&  \frac{2f}{\mu} \spm
      \left\{ -\frac{2}{\nu+\nu_1} \pm \frac{2}{\nu-\nu_1} \right\} \\      
A^{(+0,-)}_3 &=& \; \frac{2f}{\mu} \kappa \; (-1 \pm 1)   \\
A^{(+0,-)}_4 &=& \frac{2f}{\mu}\;\kappa
      \left\{ \frac{2}{\nu+\nu_1} \pm \frac{2}{\nu-\nu_1} \right\} \\ 
A^{(+0,-)}_5 &=&  \frac{2f}{\mu}\;\kappa
      \left\{ \frac{4}{\nu+\nu_1} \mp \frac{4}{\nu-\nu_1} \right\} \\       
A^{(+0,-)}_6 &=&  \frac{2f}{\mu}(1+2\kappa)
      \left\{ \frac{1}{\nu+\nu_1} \mp \frac{1}{\nu-\nu_1} \right\} 
      + \frac{2f}{\mu} \kappa\, (1\pm 1) \; .
\eeq
Dabei haben wir der \"Ubersichtlichkeit halber den Isospinindex
der anomalen magnetischen Momente unterdr\"uckt. Es gilt
$\kappa^{(\pm)}=\kappa^v$ und $\kappa^{(0)}=\kappa^s$. 
Mit Hilfe der im ersten Kapitel abgeleiteten Formel f\"ur die
Schwellenamplitude,
\be
 \left. E_{0+}\right|_{thr} = \frac{e}{16\pi M}
 \frac{2+\mu}{(1+\mu)^{3/2}} \, \left. \left(
   A_3 + \frac{\mu}{2} A_6 \right) \right|_{thr}
\ee                
und der in Anhang A diskutierten Kinematik
ergibt sich schlie\ss lich folgende Bestimmung der elektrischen
Dipolamplitude f\"ur die vier physikalischen Kan\"ale 
\beq
\label{LET}
\Epn &=& \frac{e}{4\pi} \frac{\sqrt{2}f}{m_\pi}
    \left\{ 1 - \frac{3}{2}\mu + {\cal O}(\mu^2) \right\}
    \cong 26.6 \su \\[0.1cm]
\Emp &=& \frac{e}{4\pi} \frac{\sqrt{2}f}{m_\pi}
     \left\{ -1 + \frac{1}{2}\mu + {\cal O}(\mu^2) \right\}
    \cong -31.7 \su \\[0.1cm]
\Eop &=& \frac{e}{4\pi} \frac{f}{m_\pi}
     \left\{ -\mu + \frac{\mu^2}{2}(3+\kappa_p ) +
  {\cal O}(\mu^3) \right\}    \cong -2.32
  \su \\[0.1cm]
\Eon &=& \frac{e}{4\pi} \frac{f}{m_\pi}
     \left\{  \frac{\mu^2}{2}\kappa_n  +
  {\cal O}(\mu^3) \right\}  \cong -0.51 \su 
\eeq
wobei wir die Werte $\kappa_p=1.79$ und $\kappa_n=-1.91$ verwendet 
haben. Dieses Resultat liefert den Inhalt des Niederenergietheorems
\cite{Bae70,VZ72}. Die relative Ordnung der nicht bestimmten
Korrekturen wurde mit Hilfe verschiedener Annahmen \"uber
das Verhalten der Untergrundamplitude festgelegt. Wir werden
diese Annahmen und ihre Rechtfertigung im n\"achsten Abschnitt
diskutieren.

\begin{figure}
\label{feyn}
\caption{Diagrammatische Darstellung der f\"uhrenden Beitr\"age zur
Pionphotoproduktionsamplitude}
\vspace{8.5cm}
\end{figure}

Niederenergietheoreme zur Pionphotoproduktion lassen sich auch
direkt aus der Bestimmung der Bornterme in effektiven 
chiralen Meson-Nukleon Theorien ableiten \cite{Pec69}. Die
entsprechende Lagrangedichte unter Einbeziehung der 
elektromagnetischen Wechselwirkung lautet
\beq
\label{leff}
{\cal L} & =& \bar{\psi}(i\gamma\cdot{\cal D}-M)\psi 
  +\frac{1}{2}({\cal D}_\mu\phi^{a})^2 - \frac{1}{2}m_\pi^2
  (\phi^{a})^2  \\
 & & \mbox{} + \frac{f}{m_\pi} \bar{\psi}\gamma_5 \gamma_\mu
 \tau^{a} {\cal D}^\mu \phi^{a}\psi 
  + \frac{e}{4m}\bar{\psi} (\kappa^s +\kappa^v \tau^3)
  \sigma_{\mu\nu}\psi F^{\mu\nu} \nonumber 
\eeq
wobei ${\cal D}_\mu=\partial_\mu+iQ{\cal A}_\mu$ die kovariante
Ableitung, ${\cal A}_\mu$ das elektromagnetische Potential und
$Q$ den Ladungsoperator bezeichnet. Die Eichung der pseudovektoriellen
Pion-Nukleon-Kopplung erzeugt eine $\gamma\pi NN$-Kontaktwechselwirkung,
\be
{\cal L}_{\gamma\pi NN} = \frac{ef}{m_\pi}\epsilon^{3ab}
  \bar{\psi}\gamma_5 \gamma_\mu \tau^{a}\psi {\cal A}^\mu \phi^b
\ee  
die im Rahmen der effektiven Theorie die Kroll-Ruderman-Amplitude
liefert. Die Wirkung der kovarianten Ableitung auf das Pionfeld
bestimmt die Kopplung des elektromagnetischen Feldes an die
geladenen Pionen. Die $\gamma\pi\pi$-Wechselwirkung 
\be  
{\cal L}_{\gamma\pi\pi} = e\epsilon^{3ab}\phi^{a}\partial_\mu
 \phi^{b} {\cal A}^\mu
\ee
liefert schlie\ss lich den Pionpol in der Photoproduktionsamplitude
f\"ur geladene Pionen. 
   
    
\section{Absch\"atzung der vernachl\"assigten Beitr\"age}
Um die Modellabh\"angigkeit des im letzten Abschnitt
vorgestellten Niederenergietheorems zu studieren, ist es
hilfreich, die \"Ubergangsmatrix in der Form 
\be
 T_\mu^{a} = T_\mu^{a({\rm LET})} + \delta T_\mu^{a}
\ee
zu zerlegen. Dabei bezeichnet $T_\mu^{a(\rm LET)}$ die 
T-Matrix, die zu den invarianten Amplituden  (\ref{letamp})
geh\"ort, und $\delta T_\mu^{a}$ die vernachl\"assigte 
Hintergrundamplitude. Nach dem Kroll-Ruderman Theorem gilt
\be
  \lim_{q,k\to 0} \delta T_\mu^{a} =0 \, ,
\ee
so da\ss\ $\delta T_\mu^{a}$ am ''weichen`` Punkt $q_\mu=0$ verschwindet.
Um zu untersuchen, in welcher Ordnung in der Pionmasse $\delta T_\mu^{a}$ 
Korrekturen zur elektrischen Dipolamplitude an der physikalischen Schwelle
liefert, definieren wir
\be
  \delta T_\mu^{a} = \bar{u}(p_2) \sum_{\lambda} 
   \delta A_\lambda^{a}(\nu,\nu_1) {\cal M}_\lambda u(p_1)
\ee
und nehmen an, da\ss\ sich die die invarianten Amplituden 
$\delta A_\lambda (\nu,\nu_1)$ in eine Taylorreihe um den Punkt
$\nu=\nu_1=0$ entwickeln lassen:
\be
 \delta A_\lambda^{a} (\nu,\nu_1) = a^{a}_{\lambda\, 00}
    + a^{a}_{\lambda\, 10} \nu + a^{a}_{\lambda\, 01}\nu_1
    + \ldots
\ee
Diese Annahme ist gerechtfertigt, da lediglich die Pion- und 
Nukleonpolterme Singularit\"aten bei $\nu=0$ oder $\nu_1=0$ 
enthalten. Diese Terme haben wir aber explizit in $T_\mu^{a(
{\rm LET})}$ ber\"ucksichtigt. Dar\"uber hinaus wollen wir 
in den folgenden Betrachtungen voraussetzen, da\ss\ alle
Koeffizienten $a^{a}_{\lambda\, ij}$ im Limes $m_\pi\to 0$
regul\"ar sind. Das bedeutet, da\ss\ sich diese Koeffizienten
beim Abz\"ahlen von Potenzen in $\mu$ als Gr\"o\ss en der
Ordnung ${\cal O}(1)$ betrachten lassen. 

Diese Annahme ist vermutlich unzutreffend, denn Pion-Schleifendiagramme 
k\"onnen Beitr\"age liefern, die nichtanalytisch in $m_\pi$ sind 
\cite{LP71,PP71}. Die Gegenwart solcher Terme ist von Bernard et al.~durch 
eine explizite Rechnung im Rahmen der chiralen St\"orungstheorie best\"atigt
worden\footnote{Dagegen bestreitet Naus \cite{Nau91} auf Grund von
\"Uberlegungen allgemeiner Natur die Existenz 
nichtanalytischer Terme in der Pionphotoproduktionaamplitude}.
Trotzdem ist es von Interesse, die Gr\"o\ss enordnung der analytischen
Beitr\"age in $\delta T_\mu^{a}$ zu studieren. 
 
Die Amplitude $T_\mu^{a(\rm LET)}$ enth\"alt im isospinsymmetrischen 
Fall neben den Nukleon- und Pion-Polen auch einen Kontaktterm. 
Die Gegenwart dieses Terms unterscheidet die Bornamplituden in
pseudovektorieller bzw.~pseudoskalarer Kopplung und ist deshalb
eine Konsequenz der PCAC-Relation. Dieses Resultat l\"a\ss t sich
als eine Bedingung f\"ur die invariante Amplitude $A_6^{(+0)}$
formulieren \cite{AG66}
\be
\label{FFR}
 \lim_{\nu\to 0} \lim_{\nu_1\to 0} A_6^{(+0)} (\nu,\nu_1)
   =  \frac{4f}{\mu} \kappa^{v,s} \; .
\ee
Die Reihenfolge der beiden Grenz\"uberg\"ange in (\ref{FFR})
ist nicht beliebig. Sie ist so gew\"ahlt, da\ss\ der Polterm
keinen Beitrag zum Grenzwert liefert. 
Da der Kontaktterm (\ref{FFR}) bereits in $T_\mu^{a(\rm LET)}$
enthalten ist, verschwindet $a^{(+0)}_{6\,00}$ im Grenzfall $q_\mu
\to 0$. De Baenst \cite{Bae70} verwendet daher in seiner
Diskussion der Amplitude $\delta T_\mu^{a}$ die zus\"atzliche
Annahme $a^{(+0)}_{6\,00}=0$. 
          
Nur $\delta A_3$ und $\delta A_6$ tragen zur elektrischen 
Dipolamplitude an der Schwelle bei. Mit Hilfe der 
Eichinvarianzbedingung (\ref{gaugecond}) und der Forderung nach korrektem
Verhalten der Amplituden unter der Austauschtransformation
$(\nu,\nu_1)\to(-\nu,\nu_1)$ l\"a\ss t sich die m\"ogliche
Form der Taylorentwicklungen f\"ur $\delta A_{3,6}$ erheblich
einschr\"anken. F\"ur die isospinsymmetrischen Komponenten
findet man
\beq
 \delta A_{3}^{(+0)} &=& a_{3\, 11}^{(+0)} \nu\nu_1 + \ldots \\
 \delta A_{6}^{(+0)} &=& a_{6\, 01}^{(+0)} \nu_1
               + a_{6\, 20}^{(+0)} \nu^2
	       + a_{6\, 02}^{(+0)} \nu_1^2 + \ldots \; .
\eeq
An der Schwelle ist $\nu={\cal O}(\mu)$ und $\nu_1={\cal O}(\mu^2)$,
so da\ss\ die Austauschsymmetrie im wesentlichen das 
Transformationsverhalten der Amplitude unter $m_\pi\to -m_\pi$
spezifiziert. Auf Grund der Beziehung 
\be
\delta E_{0+} \sim \delta A_3 + \frac{\mu}{2} \delta A_6
\ee
folgt, da\ss\ die Hintergrundamplitude $\delta E_{0+}^{(+0)}$
an der Schwelle von der Ordnung $\mu^3$ ist. Verwendet man
an Stelle der Annahme $a_{6\,00}^{(+0)}=0$ die Absch\"atzung 
$a_{6\,00}^{(+0)}={\cal O}(\mu)$, so ergibt sich das schw\"achere 
Resultat $\delta E_{0+}^{(+0)}={\cal O}(\mu^2)$. Eine analoge
Argumentation l\"a\ss t sich auch f\"ur die isospinungeraden Komponenten 
durchf\"uhren. In diesem Fall findet man $\delta E_{0+}^{(-)}= 
{\cal O}(\mu^2)$.

\section{Die Methode von Furlan, Paver und Verzegnassi}
Die Ableitung des Niederenergietheorems im  Abschnitt 2.2
basierte im wesentlichen auf der Reduktionsformel und 
Wardidentit\"aten f\"ur die Zweipunktfunktion $\overline{\Pi}_{\mu\nu}^{a}$.
In diesem Abschnitt wollen wir auf eine andere Methode eingehen, die
direkt mit Stromalgebrakommutatoren und Vollst\"andigkeitsrelationen
arbeitet. Im Rahmen dieses Verfahrens wurde erstmals darauf hingewiesen, 
da\ss\ die explizite Brechung der chiralen Symmetrie Korrekturen an 
das Standard-Niederenergietheorem liefern kann \cite{FPV74,NS89}.

Allerdings wird in der \"ublichen Diskussion der Methode nur der
Nukleonbeitrag in der Vollst\"andigkeitssumme explizit ber\"ucksichtigt.
In diesem Fall fehlt der f\"uhrende Beitrag zur Photoproduktion 
neutraler Pionen und man mu\ss\ nachtr\"aglich Eichinvarianz 
erzwingen, um die korrekte Schwellenamplitude zu reproduzieren.
 
In diesem Abschnitt wollen wir demonstrieren, da\ss\ unter Einbeziehung
der Beitr\"age von Antinukleonen im Zwischenzustand auch das Verfahren
von Furlan et al.~die komplette Schwellenamplitude liefert. Zu diesem
Zweck betrachten wir die zu dem Strom $B_\mu^{a}$ geh\"orende Ladung
\be
 \overline{Q}^{a}_5(t) = Q^{a}_5(t) +  \frac{1}{m_\pi^2}\,
 \frac{d}{dt} \, \int d^3x\, D^{a} (\vec{x},t)\, .
\ee
Zwischen physikalischen Zust\"anden reduziert sich dieser Operator
auf die Ladung $\qfl$,
\beq
\label{q5l}
 \qfl (t) &=& Q_5^{a}(t) +\frac{i}{m_\pi}\dot{Q}^{a}_5(t)\, , \\
\label{q5r} 
 \qfr (t) &=& \left( \qfl (t)\right)^\dagger 
                =  Q_5^{a}(t) -\frac{i}{m_\pi}\dot{Q}^{a}_5(t)  \, .
\eeq
Den zugeh\"origen hermitesch konjugierten Operator haben wir mit
$\qfr$ bezeichnet. Die  Pionmatrixelemente dieser Operatoren lauten:  
\beq
  <0|\qfl |\pi^{b}(q)> &=& \spm 2if_\pi m_\pi \,\delta^{ab}
                           (2\pi)^3  \delta^3 (\vec{q}\,)\, , \\[0.2cm]  
  <\pi^{b}(q)|\qfr |0> &=& -2if_\pi m_\pi \,\delta^{ab}
                            (2\pi)^3 \delta^3 (\vec{q}\,)\, , \\[0.2cm]
 <\pi^{b}(q)|\qfl |0>&=& <0|\qfr |\pi^{b}(q)> = 0 \; .
\eeq
Die axialen Ladungen $Q_{5\,{\mini L,R}}^{a}$ sind nicht hermitesch
und haben die Eigenschaft, zwischen Pionen im Eingangs- und Ausgangskanal 
zu unterscheiden. Diese Tatsache erweist sich als besonders n\"utzlich 
bei der  Konstruktion von Summenregeln, da sie es erm\"oglicht,
bestimmte Prozesse in der Vollst\"andigkeitssumme zu selektieren. 
Im folgenden betrachten wir Summenregeln f\"ur das Matrixelement 
\be
 M_\mu^{a} = <N(p_2)| [\qfl ,V_\mu^{em}(0)]|N(p_1)>\, ,
\ee
in dem sich mit Hilfe der oben angegebenen Matrixelemente die 
Photoproduktionsamplitude identifizieren l\"a\ss t. Das Resultat
besitzt eine sehr \"ubersichtliche Struktur als Summe von Poltermen 
und einem Dispersionsintegral, das die Hintergrundamplitude 
repr\"asentiert. 

Da der Operator $\qfl$ nur Pionen in Ruhe produziert, 
verlangt die Berechnung der Schwellenamplitude die Kenntnis
von $M_\mu^{a}$ im Schwerpunktsystem. Um die folgende Rechnung
etwas zu vereinfachen, werden wir $M_\mu^{a}$ statt dessen an
der Breitschwelle berechnen, das hei\ss t f\"ur Pionen, die
im Breitsystem des Nukleons ruhen\footnote{An der Schwelle ruht
das Pion im Schwerpunktsystem. Der Impuls des Pions im Breitsystem 
des Nukleons dann tats\"achlich sehr klein, $|\vec{q}|=m_\pi^2
+{\cal O}(m_\pi^3)$.}:
\be
\begin{array}{rclcrcl}
  \vec{p}_2 &=&\spm \vec{p}, &\hspace{1cm} & \vec{q} &=& 0, \\[0.2cm]
  \vec{p}_1 &=&-\vec{p}    , &\hspace{1cm} & \vec{k} &=& 2\vec{p}.
\end{array}
\ee
Die eine Seite der Summenregel f\"ur $M_\mu^{a}$ ergibt sich, indem 
man das Matrixelement
\beq
\label{comqfl}
\lefteqn{<\!N(\vec p\,)|[\qfl,V_\mu^{em}(0)] |N(-\vec p\,)\!>= } \\
&\hspace{0.5cm} & <\! N(\vec p\,)|[Q_5^{a},V_\mu^{em}(0)]|N(-\vec p)\!>
 +\frac{i}{m_\pi}<\!N(\vec p\,)| [\dot{Q}_5^{a},V_\mu^{em}(0)]
 |N(-\vec p\,)\!> \nonumber 
\eeq
direkt ausgwertet. Dabei findet man den  Kroll-Ruderman Term sowie
den bereits diskutierten Beitrag der expliziten Symmetriebrechung. Die 
andere Seite der Summenregel ergibt sich aus der Vollst\"andigkeitsrelation 
f\"ur den Kommutator (\ref{comqfl}). Die Clusterzerlegung \cite{AFF73} 
ist eine systematische Methode, um die verschiedenen Beitr\"age zur 
Vollst\"andigkeitssumme
\beq
\sum_n <N(\vec{p}\,)|\qfl |n><n|V_\mu^{em}|N(-\vec{p}\,)>
\eeq
zu identifizieren. Sie tr\"agt der Tatsache Rechnung, da\ss\
in einer relativistischen Theorie Beitr\"age mit unterschiedlichen 
Teilchenzahlen auftreten k\"onnen. Konkret zerlegt man $M_\mu^{a}$ 
in der Form    
\begin{figure}
\label{diag}
\caption{Beitr\"age zur Vollst\"andigkeitssumme f\"ur das
Operatorprodukt $V_\mu^{em}\qfl$.}
\vspace{7.5cm}
\end{figure}
\beq
\label{cluster}
M_\mu^{a\;\;}  &=& M_\mu^{a\,I}+M_\mu^{a\,II}  \\
M_\mu^{a\,I\,} &=& \sum_\alpha <N(\vec{p})|\qfl |\alpha>_c
                             <\alpha|V_\mu^{em}|N(-\vec{p}\,)>_c \\   
 & &        \hspace{0.5cm} -  \sum_\beta <0|\qfl |N(-\vec{p}\,)\beta>
              <N(\vec{p}\,)\beta|V_\mu^{em}|0>\; +\; {\em c.~t.} \nonumber \\
M_\mu^{a\,II} &=& \sum_{\gamma_1} <N(\vec{p}\,)|\qfl |N(-\vec{p}\,)\gamma_1>_c
                             <\gamma_1|V_\mu^{em}|0> \\   
 & &       \hspace{0.5cm} +  \sum_{\gamma_2} <0|\qfl |\gamma_2>
       <\gamma_2 N(\vec{p}\,)|V_\mu^{em}|N(-\vec{p}\,)>\; +\; 
       {\em c.~t.} \nonumber
\eeq
wobei der Index $c$  den zusammenh\"angenden Teil des Matrixelements
und $c.t.$  die Voll\-st\"an\-dig\-keitssumme mit den Operatoren in der
anderen Reihenfolge bezeichnet. Der erste Teil der Clusterzerlegung
beinhaltet solche Zust\"ande, die Baryonenzahl tragen. Die
f\"uhrenden Terme in diesem Beitrag stammen von Nukleonen
$|\alpha>=|N(\vec{p}\,)>$ und Antinukleonen $|\beta>=|\bar{N}
(\vec{p}\,)>$. Der zweite Teil von (\ref{cluster}) beschreibt die Produktion
eines Zustands $\gamma_{1,2}$ aus dem Vakuum, gefolgt von der Reaktion
$\gamma_1 +N(p_1) \to \qfl + N(p_2)$ bzw.~ $V_\mu^{em}+N(p_1)
\to \gamma_2 + N(p_2)$. Insbesondere findet man f\"ur Pionzust\"ande
$|\gamma_2>=|\pi^{a}(\vec{q}\,)>$ die Photoproduktionsamplitude
\be
\sum_{\pi^{b}(\vec{q})} <0|\qfl |\pi^{b}(\vec{q}\,)><\pi^{b}(\vec{q}\,)
  N(\vec{p}\,)|V_\mu^{em}|N(-\vec{p}\,)> = f_\pi T_\mu^{a}(\vec{q}=0)\, .
\ee
Auf Grund der speziellen Eigenschaften des Operators  $\qfl$
enth\"alt die Vollst\"andigkeitssumme keine Beitr\"age von der
inversen Reaktion $\pi^{a}(q)+N(p_1)\to V_\mu^{em}+N(p_2)$. Isoliert
man die Photoproduktionsamplitude $T_\mu^{a}$ und separiert 
die Nukleonbeitr"age in $M_{\mu}^{a\, I}$, so ergibt sich 
schlie\ss lich folgender Ausdruck f\"ur $T_\mu^{a}$
\beq
\label{fpv}
f_{\pi} T_{\mu}^{\alpha}(\vec{q}=0) &=&
 i \epsilon^{\alpha 3 \gamma} <N(\vec{p}\,)|A_{\mu}^{\gamma}|N(-\vec{p}\,)> 
                   \\[0.3cm]
   & &\mbox{}-\sum_{N(\vec{p}_n)} <N(\vec{p}\,)|\qfl |N(\vec{p}_n)>
   <N(\vec{p}_n)|V_{\mu}^{em}|N(-\vec{p}\,)> 
           \;+ \; {\em c.~t.} \nonumber \\
   & &\mbox{}  + \frac{i}{m_\pi}<N(\vec{p}\,)|[\dot{Q}_5^{\alpha},V_{\mu}^{em}]
    |N(-\vec{p}\,)> 
    \; + \;  f_\pi \delta T_\mu^{a}  \nonumber ,
\eeq
wobei $\delta T_\mu^{a}$ die vernachl\"assigten Beitr\"age in der
Vollst\"andigkeitssumme bezeichnet. Dabei handelt es sich vor allem um
$\pi N$-Kontinuumszust\"ande und Antinukleonen in $M_\mu^{a\, I}$ 
sowie Vektormesonen in $M_\mu^{a\, II}$.
Die beiden Kommutatoren in (\ref{fpv}) liefern wie in (\ref{avward})
den Kroll-Ruderman Term und die Korrekturen auf Grund der expliziten
Symmetriebrechung.
Mit Hilfe der Eigenschaften des Operators $\qfl$ l\"a\ss t sich folgende
Darstellung der Hintergrundamplitude ableiten \cite{AFF73}
\be
\label{ressum}
\delta T_\mu^{a} = -im_\pi \sum_{n\neq\pi,N} (2\pi)^3 \delta^3 
  (\vec{p}-\vec{p}_n) 
  \frac{ <N(\vec{p}\,)|j_\pi^{a}|n><n|V_\mu^{em}|N(-\vec{p}\,)> }{ 
       (E_p-E_n)(E_p+m_\pi-E_n+i\epsilon) }
  \;-\; c.t.
\ee
Die Summation l\"auft \"uber beliebige intermedi\"are Zust\"ande 
mit Ausnahme von Nukleonen und Pionen. $E_p=(\vec{p}^{\,2}+M^2)^{1/2}$
bezeichnet die Energie des auslaufenden Nukleons, $E_n$ die Energie
des Zwischenzustands. Der Energienenner in (\ref{ressum}) verschwindet
nur f\"ur Nukleonzust\"ande, so da\ss\ alle anderen Beitr\"age im Limes 
$m_\pi \to 0$ unterdr\"uckt sind. 

Wir betrachten nun im Einzelnen die verschiedenen Beitr\"age zur
Photoproduktionsamplitude (\ref{fpv}). Den Kroll-Ruderman Term 
und den Sigmakommutator haben wir bereits im Abschnitt 2.2 
diskutiert. Um den Nukleonterm zu berechnen, ben\"otigen wir die
Matrixelemente 
\beq
  <N(\vec{p}\,)|A_0^{a}(0)|N(\vec{p}\,)> &=&
     \frac{g_A}{M} \,\chi^\dagger_f (\vec{\sigma}\cdot\vec{p}\,)
     \frac{\tau^{a}}{2} \chi_i\, ,  \\  
 <N(\vec{p}\,)|V_0^{em}(0)|N(-\vec{p}\,)> &=&
     e \,\chi^\dagger_f (G_E^s (t) + \tau^3 G_E^v (t) ) \chi_i \, ,\\[0.1cm]
 <N(\vec{p}\,)|\vec{V}^{em}(0)|N(-\vec{p}\,)> &=&
     \frac{e}{M} \,\chi^\dagger_f (G_M^s(t) + \tau^3 G_M^v(t))  
     i(\vec{\sigma}\times\vec{p}) \chi_i \, .
\eeq     
Im Breitsystem treten die elektrischen und magnetischen Formfaktoren
des Nukleons,
\beq
  G_E(t) &=& F_1(t)+\frac{t}{4M^2}F_2(t) \, , \\[0.1cm]
  G_M(t) &=& F_1(t)+F_2(t)\, ,
\eeq
auf. Da wir bereits ein spezielles Bezugssystem gew\"ahlt haben,
ist es sinnvoll, die $T$-Matrix in einer nicht kovarianten Form
anzugeben. Der Nukleonbeitrag zur Photoproduktionsamplitude an  
der Breitschwelle $t=m_\pi^2$ ergibt sich schlie\ss lich zu
\beq
   T_0^{a}  &=& -\frac{g_A}{f_\pi}\,\frac{1}{E_p}
        \chi^\dagger_f (G_E^s(t) \tau^{a} + G_E^v \delta^{a3})
	(\vec{\sigma}\cdot\vec{p}\,)\chi_i \, , \\
\vec{T}^{a} &=& \spm \frac{g_A}{f_\pi} \, \frac{\vec{p}^{\, 2}}{mE_p}
       G_M^v(t) \,\chi^\dagger_f \frac{1}{4} [\tau^{a},\tau^3] 
       \,\vec{\sigma}_{\mini T}\, \chi_i\, ,
\eeq
wobei wir die transversalen und longitudinalen Spins 
\beq
   \vec{\sigma}_{\mini T} &=& \vec{\sigma} - \hat{p}(\vec{\sigma}
              \cdot\hat{p}) \\
   \vec{\sigma}_{\mini L} &=&  \hat{p}(\vec{\sigma}\cdot\hat{p})	      
\eeq
eingef\"uhrt haben. Nur der transversale Anteil liefert einen Beitrag
zur Photoproduktion  mit reellen Photonen. Im Falle des Nukleonterms
ist dieser Beitrag von der Ordnung $m_\pi^2$ und proportional zum
magnetischen Moment des Nukleons.   	             
      
Als pseudoskalares Teilchen koppelt das Pion stark an die unteren
Komponenten der Nukleonspinoren. Der f\"uhrende Beitrag zur Produktion
neutraler Pionen kommt daher von Zwischenzust\"anden, die propagierende 
Antinukleonen enthalten ('Z-Graphen`). Mit Hilfe der Darstellung
(\ref{ressum}) findet man
\be
 \vec{T}^{a} = -\frac{g_A}{f_\pi} \frac{m_\pi}{2E_p} \,
     \chi^\dagger_f (G_E^s(t)\tau^{a} + G_E^v(t) \delta^{a3})
     \vec{\sigma} \chi_i
\ee
wobei wir h\"ohere Ordnungen in $m_\pi/M$ vernachl\"assigt haben.
Damit ist die elektrische Dipolamplitude bis zur Ordnung $m_\pi^2$
bestimmt. Vernachl\"assigt man den Beitrag aus der expliziten
chiralen Symmetriebrechung, so ergibt sich
\beq
\label{LET2}
\Epn &=& \frac{e}{4\pi} \frac{g_A}{\sqrt{2}f_\pi}
    \left\{ 1 - \frac{3}{2}\mu + {\cal O}(\mu^2) \right\}
    \cong 24.1  \su \\
\Emp &=& \frac{e}{4\pi} \frac{g_A}{\sqrt{2}f_\pi}
     \left\{ -1 + \frac{1}{2}\mu + {\cal O}(\mu^2) \right\}
    \cong -29.6  \su \\
\Eop &=& \frac{e}{4\pi} \,\frac{g_A}{2f_\pi}\;
     \bigg\{ -\mu + {\cal O}(\mu^2) \bigg\}  \cong -3.3 \su
\eeq
Die elektrische Dipolamplitude f\"ur die Produktion neutraler
Pionen an Neutron verschwindet in dieser Ordnung. Die von
(\ref{LET}) abweichenden Werte in den geladenen Kan\"alen 
sind eine Konsequenz der Tatsache, da\ss\ die 
Goldberger-Treiman-Relation $\frac{g_A}{2f_\pi}=\frac{f}{m_\pi}$
experimentell um ca.~6\% verletzt ist. Diese Abweichung ist 
formal von der Ordnung $m_\pi^2$ und entspricht daher der 
oben vorgenommenen Absch\"atzung. 


%\section{Explizite chirale Symmetriebrechung}
Die r\"aumlichen Komponenten des Beitrags aus der expliziten
chiralen Symmetriebrechung 
\be
\label{csbcom}
 \Sigma_\mu^{a}(\vec{q}=0) = 
  \int d^4 x \,\delta (x^0) [\partial^\nu A_\nu^{a}(x),
  V_\mu^{em}(0)]
\ee
sind nicht durch Stromalgebra festgelegt. Aus diesem Grund haben 
wir Ihren Beitrag zur Photoproduktionsamplitude bislang 
vernachl\"assigt. Repr\"asentiert man jedoch die Str\"ome
durch Quarkfelder, so ist auch dieser Kommutator durch die
kanonischen Vertauschungsregeln der Felder bestimmt. 
Der Vollst\"andigkeit halber arbeiten wir in Flavor-$SU(3)$,
so da\ss\
\beq
   \partial^\nu A_\nu^{a} &=& \frac{i}{2} \bar{\psi} \gamma_5
      \left\{ M,\lambda^{a} \right\} \psi  \\
    V_\mu^{em}            &=& \frac{1}{2} \bar{\psi} \gamma_\mu
      ( \lambda^3 + 1/\sqrt{3} \lambda^8 ) \psi
\eeq
mit $M={\rm diag}(m_u,m_d,m_s)$. Es wird sich allerdings zeigen, da\ss\
die Masse der seltsamen Quarks nicht in das Resultat eingeht.
Der Kommutator der beiden Bilinearformen l\"a\ss t sich mit Hilfe 
der Relation              
\beq
\label{bilcom}
 \lefteqn{\delta (x^0-y^0) [\psi^\dagger (y)\frac{\lambda^{a}}{2}
      \Gamma \psi (y),\psi^\dagger (x)\frac{\lambda^{b}}{2}
      \Gamma' \psi (x)] = }  \\
    & & \hspace{1cm}   \frac{1}{2} \delta^4 (x-y) 
      \psi^\dagger (x) \left( if^{abc} \{\Gamma,\Gamma'\} 
      + id^{abc} [\Gamma,\Gamma' ] \right) \frac{\lambda^c}{2}
      \psi (x) \nonumber
\eeq
auswerten. Dabei bezeichnen $\Gamma$ und $\Gamma'$ die Diracoperatoren,
$f^{abc}$ und $d^{abc}$ die antisymmetrschen bzw.~symmetrischen $SU(3)$
Strukturkonstanten. Mit Hilfe von (\ref{bilcom}) ergibt sich f\"ur
$a=1,2,3$
\be
\label{sig0q}
\int d^4x\, \delta (x^0) [\partial^{\nu}A_{\nu}^{a}(x),
   V_{0}^{em}(0)] = i\,\overline{m} \,\epsilon^{3ab} \bar{\psi}
   \gamma_5\lambda^b \psi\hspace{4cm}
\ee
und
\beq   
\label{sigcom}
\int d^4 x\, \delta (x^0) [\partial^{\nu}A_{\nu}^{a}(x),
        V_{i}^{em}(0)] & =&
i\,\overline{m} \,\epsilon_{ijk} \left\{  \delta^{a3}
\frac{1}{\sqrt{3}}\left( \sqrt{2} J_{jk}^{0}+J_{jk}^{8} \right) +
 \frac{1}{3} J_{jk}^{a}\right\}    \\
& &\mbox{} + i\frac{\delta m}{2}\epsilon_{ijk}\delta^{a3} \left\{
 \frac{1}{3\sqrt{3}}\left(\sqrt{2} J_{jk}^{0}+ J_{jk}^{8} \right)
  + J_{jk}^{3}  \right\}  \nonumber
\eeq
wobei wir die Tensorstr\"ome
\be
 J_{\mu\nu}^{c} = \bar{\psi}\sigma_{\mu\nu}\frac{\lambda^c}{2}\psi
\ee
eingef\"uhrt haben. In der von uns verwendeten Normierung ist
$\lambda^0=\sqrt{2/3}\,{\bf 1}$. Man beachte, da\ss\ $1/\sqrt{3}
(\sqrt{2}\lambda^0 +\lambda^8)$ gerade die Einheitsmatrix im 
$SU(2)$-Unterraum ist. Die Str\"ome (\ref{sig0q},\ref{sigcom})
enthalten daher keine Beitr\"age der seltsamen Quarks. Die St\"arke
der chiralen Symmetriebrechung sowie der Isopsinbrechung wird
durch die Parameter
\beq
  \overline{m} &=& \frac{1}{2}(m_u+m_d)  \\
  \delta m     &=& m_u -m_d
\eeq
kontrolliert. Die Zeitkomponente des Sigmakommutators ist im
Fall verschwindender Isopsinbrechung identisch mit der Divergenz des
Axialstroms gegeben.  Das Ergebnis (\ref{sig0q}) reproduziert 
daher das Stromalgebraresultat (\ref{sig0}). Insbesondere
lassen sich Nukleonmatrixelemente der pseudoskalaren Dichte
$\bar{\psi}\gamma_5\lambda^b\psi$ mit Hilfe des Pion-Nukleon
Formfaktors $G_{\pi NN}$ ausdr\"ucken und liefern den bereits
diskutierten Beitrag zum Pionpolterm. 

Die r\"aumlichen Komponenten des Kommutators (\ref{csbcom}) 
liefern dagegen mit den oben definierten Tensorstr\"omen 
einen v\"ollig neuen Beitrag zur Photoproduktionsamplitude.
Dieser Beitrag verschwindet am weichen Punkt $q=0$ und 
bestimmt daher die Extrapolation der Amplitude zur
physikalischen Schwelle.

Die allgemeinste Form des Nukleonmatrixelements der Tensorstr\"ome 
lautet
\beq
  <N(p_2)|\bar{\psi}\sigma_{\mu\nu}\tau^{a}\psi|N(p_1)> &=& 
        \bar{u}(p_2) \left[
     G_T^{a}(t) \sigma_{\mu\nu} + iG_2^{a}(t)
     \frac{\gamma_\mu \Delta_\nu - \Delta_\mu \gamma_\nu}{2M} 
     \right. \\
 & & \mbox{}+ \left. iG_3^{a}(t) 
     \frac{\Delta_\mu P_\nu - P_\mu \Delta_\nu}{M^2}
     + iG_4^{a} \frac{\gamma_\mu P_\nu - P_\mu \gamma_\nu}{M^2}    
     \right] \tau^{a} u(p_1) \nonumber
\eeq
wobei $\Delta_\mu=(p_2-p_1)_\mu$ den Impuls\"ubertrag und $\tau^0 ={\bf 1}$ 
sowie $\tau^{a}\; (a=1,2,3)$ die Paulimatrizen bezeichnet.
Das Matrixelement vereinfacht sich erheblich, wenn man die
r\"aumlichen Komponenten der Str\"ome im Breitsystem des Nukleons
betrachtet
\beq
   <N(\vec{p}\,)|\bar{\psi}\sigma_{jk}\tau^a\psi|N(-\vec{p}\,)>    
  & = & \epsilon_{jkm}\chi^\dagger_f \left[
     \left( G_T(t) +\frac{t}{4M^2} G_2(t)\right)\sigma_{{\mini T}m} 
    \right. \\
 & & \hspace{2.7cm} \mbox{} + \left. G_T(t)\frac{E_p}{M} \sigma_{{\mini L}m}
 \right] \tau^{a} \chi_i  \nonumber
\eeq
Bis auf Korrekturen der Gr\"o\ss enordnung $m_\pi^2$ kann man die
Formfaktoren durch ihren Wert bei $t=0$ ersetzen.
Mit der Definition $g_T=G_T(0)$ ergibt sich schlie\ss lich die 
Korrektur zur Schwellenamplitude f\"ur neutrale Pionen 
\be
\label{delneu}
\Delta E_{0+}(\pi^0 N) = \frac{e}{4\pi f_\pi}\frac{\overline{m}}{m_\pi (1+\mu)}
  \left\{ \left( 1+\frac{\delta m}{6\overline{m}} \right) g_T^0
     \pm \left(\frac{1}{3}+\frac{\delta m}{2\overline{m}}\right) g_T^3
     \right\} \; ,
\ee
wobei das sich das Vorzeichen auf die Produktion am Proton 
bzw.~Neutron bezieht. Verwendet man die oben zitierten  Werte der
Quarkmassen, so ist $\delta m/(2\overline{m}) \simeq -1/3$ und
$\Delta E_{0+}(\pi^0N)$ ist fast vollst\"andig durch die Tensorkopplung
im Singletkanal bestimmt. Man beachte, da\ss\ der Korrekturterm formal
von der Ordnung $m_\pi$ ist, denn nach der GOR Relation gilt $\overline{m}
= m_\pi^2f_\pi^2/|<\bar{u}u+\bar{d}d>|$. 

Die entsprechende Korrektur f\"ur die Produktion geladener Pionen
lautet
\be
\label{delchar}
 \Delta E_{0+}(\pi^-p)=\Delta E_{0+}(\pi^+n) =
  \frac{\sqrt{2}e}{4\pi f_\pi}\frac{\overline{m}}{m_\pi (1+\mu)}
  \,\frac{g_T^3}{3}\; .
\ee  
In diesem Fall tr\"agt der isospinbrechende Term proportional
zu $\delta m$ nicht bei. Der Korrekturterm modifiziert nicht
die Ladungsasymmetrie $|\Epn|-|\Emp|$, liefert aber einen
kleinen Beitrag zum Panofskyverh\"altnis $\Epn/\Emp$.

Die wesentliche Aufgabe bei der Berechnung von $\Delta E_{0+}$
ist nun die Bestimmung der Tensorkopplungskonstanten
$g_T^{a}$ des Nukleons. Diese sind leider experimentell
nicht direkt zug\"anglich, so da\ss\ man in diesem 
Zusammenhang auf Modelle angewiesen bleibt.
Die einfachste M\"oglichkeit ist die Verwendung eines
nichtrelativistischen Konstituentenmodells zur 
Beschreibung der Struktur des Nukleons. In diesem Fall
reduzieren sich die Tensorstr\"ome $\frac{1}{2}\epsilon_{ijk}
\bar{\psi}\sigma_{jk}\psi$ auf Axialstr\"ome $\bar{\psi}
\gamma_i\gamma_5 \psi$. Die Korrektur zur elektrischen 
Dipolamplitude lautet dann
\be
 \DEop = \frac{e}{4\pi f_\pi}\frac{\overline{m}}{m_\pi (1+\mu)}
    (0.90 \cdot g_A^0 + 0.04 \cdot g_A^3) \; .
\ee
In einem nichtrelativistischen Quarkmodell ist $g_A^0=1$ und $g_A^3=5/3$,
so da\ss\ $\DEop = 1.6 \su$. Diese Korrektur ist von derselben 
Gr\"o\ss enordnung wie der f\"uhrende Term des Niederenergietheorems,
$\DEop = -2.3\su$, und besitzt dar\"uber hinaus das umgekehrte Vorzeichen.
        

\section{Eichinvarianz}
Die Forderung nach Eichinvarianz der \"Ubergangsmatrix $T_\mu^{a}$
liefert wichtige Einschr\"ankungen f\"ur die Form der invarianten Amplituden.
Im Falle von Pionen auf der Massenschale ergeben sich diese 
Bedingungen aus der Erhaltung des elektromagnetischen Stroms
im \"Ubergangsmatrixelement
\be
\label{ongi}
k^\mu T_\mu^{a} = ie<\pi^{a}(q)N(p_2)|\partial^\mu V_\mu^{em}(0)|N(p_1)>
=0 \; .
\ee
Die aus dieser Gleichung folgenden Beziehungen (\ref{gaugecond}) haben 
wir bereits im ersten Kapitel angegeben. Man pr\"uft leicht nach, da\ss\ 
die in Abschnitt 2.2 abgeleiteten Amplituden diese Bedingungen erf\"ullen.
Dies gilt jedoch nur f\"ur die Summe von Stromalgebra-, Nukleon- und
Pionpolbeitr\"agen. Keiner dieser Terme ist f\"ur sich genommen 
eichinvariant. 

Die nicht eichinvarianten Terme in den einzelnen Beitr\"agen heben
sich allerdings nur dann gegenseitig weg, wenn keine ph\"anomenologischen 
Formfaktoren an den Vertices verwendet werden. Um die Rolle
der Formfaktoren n\"aher zu untersuchen, wollen wir unsere 
Betrachtungen auf die Elektroproduktion von Pionen erweitern.
In diesem Fall ist das ausgetauschte Photon virtuell und besitzt 
eine nicht verschwindende invariante Masse $k^2$.  Die Kopplung
des Photons wird durch die elektrischen Formfaktoren des 
Nukleons sowie des Pions 
\beq
 \Gamma_\mu^\gamma &=& F_1(k^2) \gamma_\mu + \frac{i\sigma_{\mu\nu}
               k^\nu}{2M} F_2(k^2) \\
 \Gamma_\mu^{\gamma\pi} &=& F_\pi (k^2)(2q-k)_\mu
\eeq
beschrieben. Dar\"uber hinaus liefert der Stromalgebraterm
\be
 C_\mu^{a} = -i\epsilon^{a3c} F_A(t) g_A \bar{u}(p_2)\gamma_\mu
    \gamma_5 \frac{\tau^c}{2} u(p_1)
\ee    
einen Beitrag, welcher den normierten axialen Formfaktor
$F_A(t)=G_A(t)/G_A(0)$ enth\"alt. Wie im Falle reeller 
Photonen lautet die Eichinvarianzbedingung $k^\mu T_\mu^{a}=0$.
Wir zerlegen  die Amplitude in der Form
\be
 T_\mu^{a} = T_\mu^{a(Born)} + \Delta T_\mu^{a} + T_\mu^{a(Res)}
\ee
wobei $T_\mu^{a(Born)}$ die Polterme sowie des Stromalgebrabeitrag
enth\"alt. Der Korrekturterm $\Delta T_\mu^{a}$ ist durch die Bedingung
\be
 k^\mu ( T_\mu^{a(Born)}+\Delta T_\mu^{a}) =0
\ee
definiert, w\"ahrend $T_\mu^{a(Res)}$ eine Untergrundamplitude bezeichnet,
die bis auf die Eichinvarianzforderung $k^\mu T_\mu^{a(Res)}=0$  unbestimmt 
bleibt.

Ber\"ucksichtigt man die Formfaktoren an den Vertices, so ist die
Divergenz des isopsinantisymmetrischen Teils der Bornmaplitude
\beq
\label{ngi}
 k^\mu T_\mu^{(-)(Born)} &=& \frac{ief}{m_\pi} \bar{u}(p_2)\Big(
          2M (2F_1^v(k^2) - F_\pi (k^2) ) \\
   & & \hspace{3cm} \mbox{} - \gamma\cdot k 
	  (2F_1^v(k^2) - F_A(t)) \Big) \gamma_5 u(p_1) \nonumber
\eeq 
Alle anderen Isospinkomponenten erf\"ullen die Eichinvarianzbedingung.
F\"ur die $(-)$-Komponente ist dies nur am Photonpunkt $k^2=0$ der
Fall. Um eine eichinvariante Amplitude zu erhalten, mu\ss\ man einen
Korrekturterm \cite{VZ72,SK91}
\beq
\label{gcor}
\Delta T_\mu^{(-)} &=& -\frac{ief}{m_\pi} \bar{u}(p_2)\left(
          \frac{2Mk_\mu}{k^2} (2F_1^v(k^2) - F_\pi (k^2) ) \right.\\
 & & \hspace{3cm} \mbox{}	  
	  - \left. \frac{k_\mu\gamma\cdot k}{k^2} (2F_1^v(k^2) - F_A(t))
	   \right) \gamma_5 u(p_1) \nonumber
\eeq 
addieren. Dieser Term ist nicht eindeutig bestimmt. Jeder beliebige
Ausdruck, der sich von (\ref{gcor}) nur um einen divergenzfreien
Beitrag unterscheidet, ist ebenfalls ein m\"oglicher Korrekututerm.
Die Summe $T_\mu^{a(Born)}+\Delta T_\mu^{a}$ liefert schlie\ss lich
eine eichinvariante Elektroproduktionsamplitude. 

Es ist instruktiv, die Konsequenzen von Eichinvarianz auch f\"ur 
Pionen abseits der Massenschale zu untersuchen. Dieses Problem
ist vor allem  bei der Bestimmung der Amplitude am weichen Punkt
von Bedeutung. Da sich das Pion nicht in einem asymptotischen
Zustand befindet, kann man zu diesem Zweck allerdings nicht von
Gleichung (\ref{ongi}) Gebrauch machen.  Statt dessen betrachten 
wir die zu (\ref{avward}) analoge Vektorwardidentit\"at
\be
\label{vwi}
ik^\mu \overline{\Pi}_{\nu\mu}^\alpha (q) = - C^\alpha_\nu + 
\frac{i}{m_\pi^2} \big( q_\nu \Sigma_0^\alpha (q) - 
\delta_{\nu 0} k^\rho \Sigma_\rho^\alpha (q) \big) .
\ee
Auch diese Relation beruht auf der Erhaltung des elektromagnetischen 
Stroms. Sie enth\"alt aber keine zus\"atzlichen Annahmen \"uber den
Impuls des Pions. In Verbindung mit der Axialvektorwardidentit\"at
(\ref{avward}) ergibt sich folgender Ausdruck f\"ur die Divergenz
von $T_\mu^{a}$
\be
\label{offgi}
 k^\mu T_\mu^{a} = -i\epsilon^{a3c} \frac{q^2-m_\pi^2}{f_\pi m_\pi^2}
   <N(p_2)|D^c(0)|N(p_1)> \; .
\ee
F\"ur Pionen auf der Massenschale ergibt sich die bekannte Beziehung
$k^\mu T_\mu^{a} =0$. Abseits der Massenschale liefern geladene 
virtuelle Pionen einen zus\"atzlichen Quellterm f\"ur den elektromagnetischen
Strom und bewirken eine nichtverschwindende Divergenz von $T_\mu^{a}$.

Bei der Herleitung der Relation (\ref{offgi}) ben\"otigt man keine 
Annahmen \"uber die modellabh\"angigen Komponenten der symmetriebrechenden
Amplitude $\Sigma_\mu^{a}$. Betrachtet man die einzelnen Beitr\"age 
zur linken Seite von (\ref{offgi}),
\be
\label{divamp}
 k^\mu T_\mu^{a} = \frac{1}{f_\pi} \left\{ ik^\mu q^\nu 
   \overline{\Pi}_{\mu\nu}^{a}
   -k^\mu C_\mu^{a} +\frac{i\omega_\pi}{m_\pi^2} k^\mu \Sigma_\mu 
   \right\}
\ee     
so tragen diese Terme jedoch bei. Die Polterme erf\"ullen in Verbindung
mit dem Stromalgebrabeitrag auch die verallgemeinerte Eichinvarianzbedingung
(\ref{offgi}). Man kann diese Terme daher aus der Gleichung (\ref{divamp})
eliminieren. In der Herleitung des Niederenergietheorems vernachl\"assigt
man Untergrundbeitr\"age zu den Amplituden. Die Gleichung (\ref{divamp})
reduziert sich daher auf eine Beziehung f\"ur die symmetriebrechende
Amplitude: $k^\mu \Sigma_\mu^{(+0)}=0$. Die im Abschnitt 2.5 bestimmten
Beitr\"age erf\"ullen diese Gleichung nicht. Wir definieren daher den 
eichinvarianten Teil von $\Sigma_\mu^{a}$
\be
  \Sigma_\mu^{a(gi)} = \Sigma_\mu^{a} +\Delta\Sigma_\mu^{a}
\ee
durch die Forderung $k^\mu\Sigma_\mu^{(+0)(gi)}=0$. Die isospinungeraden
Komponenten liefern die rechte Seite von (\ref{offgi}). Eine L\"osung
dieser Bedingungen lautet
\beq
 \frac{\omega_\pi}{m_\pi^2}\Sigma_\mu^{(-)(gi)} &=& -
                 \frac{f_\pi}{m_\pi^2-t}\, g_{\pi NN}
                   \, \bar{u}(p_2)i\gamma_5  q_\mu u(p_1) \\
 \frac{\omega_\pi}{m_\pi^2}\Sigma_\mu^{(0)(gi)} &=& \spm
             \frac{4\overline{m}M}{m_\pi^2} \,\frac{g_T^3}{3}
	 \, \bar{u}(p_2)i\gamma_5 \frac{\gamma_\mu \gamma\cdot k}{2M}u(p_1) \\
 \frac{\omega_\pi}{m_\pi^2}\Sigma_\mu^{(+)(gi)} &=& \spm
             \frac{4\overline{m}M}{m_\pi^2} \,
	     \left\{ g_T^0 \left(  1+\frac{\delta m}{6\overline{m}} \right)
	     \pm g_T^3 \frac{\delta m}{2\overline{m}} \right\}
	 \, \bar{u}(p_2)i\gamma_5 \frac{\gamma_\mu \gamma\cdot k}{2M}u(p_1)
\eeq
Auch diese Amplituden sind nicht eindeutig bestimmt. Wir haben sie durch
die Forderung bestimmt, da\ss\ die Schwellenamplitude (\ref{delneu})
unver\"andert bleibt und keine Beitr\"age zum longitudinalen 
Multipol $L_{0+}$ auftreten.	 
   

\section{Resonanzbeitr\"age}
Das Niederenergietheorem zur Pionphotoproduktion beruht auf der
Annahme, da\ss\ sich die Zweipunktfunktion $q^\nu\overline{\Pi}_{\mu\nu}$
in der N\"ahe des weichen Punktes $q^2=0$ durch die Nukleonpolterme
approximieren l\"a\ss t. Dabei vernachl\"assigt man  die Beitr\"age
von Schleifendiagrammen sowie den Austausch von Resonanzen im s- oder
t-Kanal. 

In der Photoproduktion von Pionen bei mittleren Energien 
$\omega^{lab}= 0.3-1.5$ GeV ist die Bedeutung von s-Kanal Resonanzen
in den Multipolamplituden deutlich zu erkennen. 
An der Schwelle sind diese Beitr\"age jedoch durch das
Verh\"altnis $m_\pi/\Delta E_R$ der Pionmasse zur Anregungsenergie
der Resonanz unterdr\"uckt. Der niedrigste Anregungszustand des
Nukleons ist die Deltaresonanz bei $\Delta E_R =294$ MeV. Dieser
Zustand koppelt au\ss erordentlich stark an das Pion-Nukleon System
und dominiert aus diesem Grund die resonante $M_{1+}$-Amplitude
bis in die Schwellenregion. Der niedrigste resonante Beitrag zur
$E_{0+}$ Amplitude stammt vom $N(1535)$ bei einer deutlich h\"oheren
Anregungsenergie $\Delta E_R= 597$ MeV. Im Gegensatz zur Deltaresonanz
zerf\"allt dieser Zustand zu etwa 50\% in $\eta N$ und liefert 
insgesamt nur einen geringen Beitrag zur $E_{0+}$ Amplitude
an der Schwelle.

Um diese Aussagen quantitativ zu belegen, wollen wir die 
Resonanzbeitr\"age mit Hilfe effektiver chiraler Lagrangedichten studieren 
\cite{Pec69,OO75,NB80}. Diese Methode ignoriert die intrinsische 
Struktur der Resonanz, hat aber den wesentlichen Vorteil, mit einem
Minimum an freien Parametern auszukommen. Diese Parameter beschreiben
neben der Masse der Resonanz die Kopplungen $\gamma N\to N^{*}$ 
sowie $N^{*}\to N\pi$ und lassen sich aus den experimentell bestimmten
Helizit\"atsamplituden und Zerfallsbreiten extrahieren.

Zu diesem Zweck betrachten wir resonante Photoproduktion
$\gamma N(\Lambda_i=\frac{1}{2},\frac{3}{2}) \to N^{*} \to \pi N$
mit definierter Helizit\"at $\Lambda_i$ im Eingangskanal. Die
zugeh\"origen Helizit\"atsamplituden $A_{1/2}$ und $A_{3/2}$ sind durch
\beq
\label{helamp}
 A_{l\pm} &=& \mp \alpha C_{N\pi} A_{1/2}  \\
 B_{l\pm} &=& \pm \frac{4\alpha}{\sqrt{(2J-1)(2J+3)}} C_{N\pi} A_{3/2}
\eeq
definiert \cite{PDG90}. Die Helizit\"atskomponenten $(A_{l\pm},B_{l\pm})$ 
sind Linearkombinationen der Multipolamplituden $(E_{l\pm},M_{l\pm})$.
Die entsprecheneden Zusammenh\"ange finden sich im Anhang B. Der
Parameter $\alpha$ lautet
\be
 \alpha = \left[ \frac{1}{\pi} \frac{k}{q} \frac{M\Gamma_\pi}{(2J+1)
    M_R \Gamma^2} \right]^{1/2} \; .
\ee
Dabei bezeichnet $M_R$ die Masse der Resonanz, $J$ ihren Spin
und $\Gamma$ sowie $\Gamma_\pi$ die totalen bzw.~partiellen Zerfallsbreiten.
$C_{N\pi}$ ist der Clebsch Gorden Koeffizient f\"ur den Zerfall der 
Resonanz in den relevanten $N\pi$ Ladungszustand.  Die Definition
(\ref{helamp}) hat den Vorzug, da\ss\ alle Gr\"o\ss en, die mit der
Propagation und dem Zerfall der Resonanz zusammenh\"angen, aus
der  eigentlichen Resonanzamplitude eliminiert werden. Die 
Helizit\"atsamplituden $A_{1/2,3/2}$ liefern daher ein zuverl\"assiges
Ma\ss\ f\"ur die St\"arke des \"Ubergangsmatrixelements $\gamma N\to N^{*}$. 
In Tabelle 1 haben wir die entsprechenden Werte f\"ur die
wichtigsten Resonanzen mit Massen unterhalb 1.6 GeV zusammengefa\ss t. 
    
\begin{table}
\caption{Helizit\"atsmaplituden (in $10^{-3}\,{\rm GeV}^{1/2}$) und
totale und partielle Breiten (in MeV) f\"ur die wichtigsten Nukleonresonanzen
mit Massen unterhalb 1.65 GeV. Alle Angaben nach [PDG90].}
\begin{center}
\begin{tabular}{|l||c|r|r|r|r|} \hline
  Resonanz             & Hel.  &  $A_{1/2,3/2}^p$ & $A_{1/2,3/2}^n$ 
		& $\Gamma_{tot}$ & $\Gamma_\pi$ \\ \hline\hline
 $N(1440)\,P_{11}$ & 1/2   &  $-69\pm 7\;\,$  & $37\pm 19$
                &  200         & 120   \\ 
 $N(1520)\,D_{13}$ & 1/2   &  $-22\pm 10$     & $-65\pm 13$
                &  125         &  70    \\
                       & 3/2   &  $167\pm 10$     & $144\pm 14$
		&              &        \\
 $N(1535)\,S_{11}$ & 1/2   &  $73\pm 14$      & $-76\pm 32$
                &  150         &   65    \\
 $N(1650)\,S_{11}$ & 1/2   &  $48\pm 16$      & $-17\pm 37$ 
                & 150	       &   90    \\
 $\Delta (1232)\,\rm P_{33}$ & 1/2 & $-141\pm 5\;\,$&
                &  115         &  115   \\
		        & 3/2  &  $-258\pm 11$    &          
		&              &        \\ \hline
\end{tabular}
\end{center}
\end{table}

Die dominante Resonanz in der $E_{0+}$ Amplitude ist die $N(1535)S_{11}$
Anregung. Dieser Zustand besitzt wie das Nukleon Spin und Isospin 1/2, 
aber negative Parit\"at. Anregung und Zerfall der Resonanz werden durch
die Kopplungen
\beq
\label{s11coup}
 {\cal L}_{\pi NN^{*}} &=& \frac{f_R}{m_\pi} \bar{\psi}_{N^{*}}
   \gamma_\mu \tau^{a}\psi \partial^\mu \phi^{a} + h.c. \\
 {\cal L}_{\gamma NN^{*}} &=& \frac{e}{4M} \bar{\psi}_{N^{*}} 
   \gamma_5 \sigma_{\mu\nu} (\kappa^s_R +\kappa^v_R \tau^3) \psi
    F^{\mu\nu} + h.c.
\eeq
beschrieben. Die beiden Parameter $f_R$ und $\kappa_R$ werden mit Hilfe
der Beziehungen
\beq
\label{rescoup}
       f_R         &=& \frac{2m_\pi}{M_R-M} 
       \sqrt{\frac{\pi M_R \Gamma_\pi}{ p_1(E_1+M)}} 
       \simeq 0.27  \\
 e\kappa^{p}_R   &=& \frac{(2M)^{3/2}}{\sqrt{(M_R+M)(M_R-M)}} A^{p}_{1/2}
       \simeq 0.51 e
\eeq
festgelegt. Dabei bezeichnen $E_1$ und $p_1$ die Energie sowie den Impuls
des Nukleons im Ruhesystems des angeregten Zustands bei der Resonanzenergie
$\sqrt{s}=M_R$. Unter Verwendung der Vertices (\ref{s11coup}) lassen sich
nun die Borndiagramme zur resonanten Photoproduktion bestimmen. Die 
zugeh\"origen invarianten Amplituden finden sich im Anhang B. Der Beitrag
der s-Kanal Anregung der N(1535) Resonanz zur elektrischen Dipolamplitude
an der Schwelle lautet
\beq
  E_{0+}^{N^{*}}(p\pi^0) &=& \frac{e\kappa_R}{16\pi M}\frac{f_R}{m_\pi}
    \frac{2\mu+\mu^2}{(1+\mu)^{3/2}} 
    \frac{(M_R-M)(M_R+M+m_\pi)}{(M+m_\pi)^2-M_R^2} \\[0.2cm]
    &\simeq& 0.28 \su \, .  \nonumber
\eeq    
Dieses Ergebnis ist formal von der Ordnung $\mu$ und widerspricht daher
der in Abschnitt 2.3 vorgenommenen Absch\"atzung der Untergrundamplitude.
Das liegt darin begr\"undet, da\ss\ die N(1535) Resonanz zus\"atzliche
Kontakterme in den inavrianten Amplitude erzeugt, die in den dort
gemachten Voraussetzungen explizit ausgeschlossen worden sind.

Die leichteste Anregung mit denselben Quantenzahlen wie das Nukleon ist
die Roperesonanz $N(1440)$. Dieser Zustand liefert einen resonanten
Beitrag zur $M_{1-}$ Amplitude, ist in der elektrischen Dipolamplitude 
aber nur als Untergrund pr\"asent. Die effektive Lagrangedichte, welche
die Kopplung des $N(1440)$ an das Nukleon beschreibt, lautet
\beq        
\label{nstarcoup}
 {\cal L}_{\pi NN^{*}} &=& \frac{f_R}{m_\pi} \bar{\psi}_{N^{*}}
   \gamma_\mu \gamma_5\tau^{a}\psi \partial^\mu \phi^{a} + h.c. \\
 {\cal L}_{\gamma NN^{*}} &=& \frac{e}{4M} \bar{\psi}_{N^{*}} 
    \sigma_{\mu\nu} (\kappa^s_R +\kappa^v_R \tau^3) \psi
    F^{\mu\nu} + h.c.
\eeq
Bestimmt man die Kopplungskonstanten aus der Zerfallsbreite und der
Helizit\"atsamplitude bei der Resonanzenergie $\sqrt{s}=m_R$, so
ergibt sich $f_R=0.48$ und $\kappa_R^p=0.58$. Mit diesen Werten 
findet man folgenden Beitrag der s-Kanal Anregung
\beq
 E_{0+}^{N^{*}}(p\pi^0) &=& \frac{e\kappa_R}{16\pi M}\frac{f_R}{m_\pi}
    \frac{2\mu+\mu^2}{(1+\mu)^{3/2}} 
     \frac{m_\pi(M_R-M-m_\pi)}{(M+m_\pi)^2-M_R^2}  
    \\[0.2cm]
    &\simeq& -0.025 \su \, .  \nonumber
\eeq 
Wie erwartet ist die entsprechende Amplitude au\ss erordentlich gering.

Eine gewisse Schwierigkeit stellt die Behandlung der Deltaresonanz
$\Delta (1232)$ dar \cite{DMW91,NS89,NB80}. Dieser Zustand ist
eine $P_{33}$ Anregung und sollte daher nicht zur s-Wellen 
Produktion beitragen. In einer relativistischen Beschreibung
der Deltaresonanz als elementares Spin 3/2 Rarita-Schwinger Feld
enth\"alt der Deltapropagator allerdings abseits der Massenschale auch
Spin 1/2 Komponenten. Die Kopplung dieser Beitr\"age an 
die Zerfallskan\"ale $\gamma N$ und $\pi N$ ist im wesentlichen
unbestimmt.  Je nach Wahl der entsprechenden Parameter findet man
\cite{NS89}
\be
\label{delta}
   E_{0+}^\Delta(\pi^0p) = (-0.10 \ldots 0.34) \su  \; .
\ee
Das angegebene Intervall entspricht der Streuung, die sich aus
verschiedenen Fits der Parameter an die nicht resonanten
Amplituden ergibt.  Das Resultat zeigt deutlich die Grenzen der 
Verwendung effektiver chiraler Lagrangedichten bei der Beschreibung
angeregter Zust\"ande auf. Trotzdem sind auch die Korrekturen
auf Grund der Deltaresonanz letztlich relativ gering. 

Wir haben unsere Untersuchung bislang auf die Rolle von Resonanzen
im s-Kanal beschr\"ankt. Aus dem Studium von Dispersionsrelationen 
ist jedoch bekannt, da\ss\ die Einbeziehung von Vektormesonen als
t-Kanal Resonanzen die Beschreibung der differentiellen Wirkungsquerschnitte 
besonders bei kleinen Energien verbessert \cite{BDW67}. Es scheint daher 
angemessen, die Rolle von Vektormesonen auch direkt an der Schwelle zu 
untersuchen. Dabei beschr\"anken wir uns auf die $\rho$- und $\omega$-Mesonen. 
Das $\phi$-Mesonen sowie die schwereren Vektormesonen liefern nur geringe
Beitr\"age. Die effektive Lagrangedichte lautet
\beq
\label{lvm}
 {\cal L}_{\rho NN} &=& f_{\rho NN} \bar{\psi}
         \left( \gamma_\mu +\frac{\kappa_\rho}{2M}\sigma_{\mu\nu}
	 \partial^\nu \right) \vec{\tau}\cdot\vec{\rho}^{\,\mu} \psi  \\ 
 {\cal L}_{\omega NN} &=& f_{\omega NN} \bar{\psi}
         \left( \gamma_\mu +\frac{\kappa_\omega}{2M}\sigma_{\mu\nu}
	 \partial^\nu \right) \omega^\mu \psi  \\
 {\cal L}_{\rho\pi\gamma} &=& \frac{eg_{\rho\pi\gamma}}{2m_\pi}
         \epsilon_{\alpha\beta\gamma\delta} F^{\alpha\beta}
	 \vec{\phi}\cdot\partial^\gamma\vec{\rho}^{\,\delta} \\
 {\cal L}_{\omega\pi\gamma} &=& \frac{eg_{\omega\pi\gamma}}{2m_\pi}
         \epsilon_{\alpha\beta\gamma\delta} F^{\alpha\beta}
	 \phi_3\cdot\partial^\gamma\omega^\delta \; .
\eeq
Die Kopplung des Photons l\"a\ss t sich aus der gemessen Zerfallsbreite
$\Gamma(\rho,\omega\to\pi\gamma)$ bestimmen. Die Vektormeson-Nukleon
Kopplungskonstante mu\ss\ dagegen indirekt, aus detaillierten Analysen
des Nukleon-Nukleon Potentials gewonnen werden \cite{Dum82}. Die
resultierenden Werte finden sich in Tabelle 2.  
 
\begin{table}
\caption{Parameter f\"ur die wichtigsten t-Kanal Beitr\"age 
zur Pionphotoproduktion.}
\begin{center}
\begin{tabular}{|rcl|rcl|rcl|}\hline
   & $\pi$ &             &  & $\rho$ &             &    &  $\omega$ &  \\ 
                                                                \hline\hline
$m_{\pi^\pm}$&=&139 MeV  & $m_\rho$&=&$770$ MeV    &  $m_\omega$&=&$783$ MeV\\
$f_{\pi NN}$&=&$1.00$    &  $f_{\rho NN}$&=&$2.66$ & $f_{\omega NN}$&=&$7.98$\\
$g_{\pi\pi\gamma}$&=&$1$ &  $g_{\rho\pi\gamma}$&=&$0.125$ 
                                        &   $g_{\omega\pi\gamma}$&=&$0.374$  \\
    & &       &  $\kappa_\rho$&=&$6.6$  &   $\kappa_\omega$&=&$0.$   \\ \hline
\end{tabular}
\end{center}
\end{table}                    

Die invarianten Amplituden, die sich aus der Wechselwirkung (\ref{lvm})
ergeben, haben wir in Anhang B gesammelt. Der Beitrag zur
elektrischen Dipolamplitude an der Schwelle ist
\be
  E_{0+}^V(\pi^0p) = \frac{e}{16\pi} \sum_V\frac{g_V}{m_\pi} 
    f_V (1+\kappa_V)\mu^3
     \frac{2+\mu}{(1+\mu)^{3/2}}\frac{M^2}{m_\pi^2+m_V^2 (1+\mu)}
\ee     	 
wobei $V=\rho,\omega$ zu setzen ist. Dieses Resultat ist explizit
von der Ordnung $\mu^3$ und entspricht daher der Absch\"atzung
aus dem Abschnitt 2.5. Mit den Werten aus Tabelle 2 findet man 
$g_\rho f_\rho (1+\kappa_\rho)=1.50$ und  $g_\omega f_\omega 
(1+\kappa_\omega)=2.87$, so da\ss\ wir schlie\ss lich eine 
Korrektur $E_{0+}=0.024\su$ erhalten.   


 
\section{Abschlie\ss ende Bemerkungen}
blub blub


\chapter{Photoproduktion von Eta-Mesonen}
%revised Jan. 2, 1992
Die Eta-Mesonen $\eta$ und $\eta'$ tragen wie das neutrale Pion
die Quantenzahlen $J^{PC}=0^{-+}$, besitzen aber den Isospin $I=0$.
Sie sind eine Mischung der Flavor-$SU(3)$ Singlet- und 
Oktett-Zust\"ande $\eta_0$ und $\eta_8$:
\newcommand{\thp}{\theta_{\eta\eta'}}
\be
\label{etamix}
\left( \begin{array}{c} \eta \\ \eta' \end{array} \right) =
\left( \begin{array}{cc} 
       \cos\thp  &   -\sin\thp  \\
       \sin\thp  &   \spm\cos\thp 
\end{array} \right)
\left( \begin{array}{c} \eta_8 \\ \eta_0 \end{array} \right)  .       
\ee   
Mit Hilfe quadratischer Massenformeln findet man einen 
Mischungswinkel $\thp =-11^\circ$,
w\"ahrend eine Analyse der Zerf\"alle $\eta\to 2\gamma$ und $\eta'\to 
2\gamma$ den Wert $\thp=-20^\circ$ liefert \cite{PDG90}. Die Masse
des Eta-Mesons betr\"agt $m_\eta=549$ MeV, w\"ahrend das $\eta'$-Meson
mit $m_{\eta'}=958$ MeV deutlich schwerer ist. Diese Tatsache ist 
eine Konsequenz der $U(1)_A$-Anomalie, die die Entartung zwischen 
den $SU(3)$ Singlet- und Oktett-Zust\"anden aufhebt.

Wir wollen in diesem Kapitel die Photoproduktion von Eta-Mesonen in
der N\"ahe der Schwelle untersuchen. Diese Reaktion liefert wesentliche
Informationen \"uber die Kopplung des Eta-Mesons an das Nukleon und 
die $N(1535)$-Resonanz \cite{TDR88}. Dar\"uber hinaus erhoffen wir uns
Hinweise auf die Rolle der chiralen Symmetriebrechung im Eta-Kanal
und den Strangeness-Inhalt des Protons.

Gegenw\"artig existieren keine zuverl\"assigen Daten zur Eta-Photoproduktion
in der Schwellenregion. Diese Situation wird sich in naher Zukunft durch
Experimente, die an den neuen, kontinuierlichen Elektronbeschleunigern
ELSA und CEBAF in Planung sind, deutlich verbessern. Eine Messung der
Eta-Photoproduktionsamplitude ist dar\"uber hinaus am MIT-Bates Labor 
im Gange.  

Wir haben im letzten Kapitel demonstriert, da\ss\  die elektrische 
Dipolamplitude f\"ur die Photoproduktion von Pionen durch die Nukleon- und
Pion-Bornterme dominiert wird. In der Eta-Photoproduktion hingegen
ist die Schwellenenergie mit $\omega^{\em cms}_{th}=448$ MeV deutlich
gr\"o\ss er, so da\ss\ die Energie des Photons vergleichbar ist mit der 
Anregungsenergie der ersten Resonanz in der $E_{0+}$-Amplitude. Aus 
diesem Grund ist es im engeren Sinne nicht m\"oglich, Niederenergietheoreme
zur  Eta-Photoproduktion zu formulieren. Trotzdem kann man die im letzten 
Kapitel entwickelten Methoden auch auf die Photoproduktion von Eta-Mesonen 
anwenden. Die wichtigsten Beitr\"age zu diesem Proze\ss\ stammen vom 
Nukleon-Bornterm, der $N(1535)$-Resonanz sowie dem Austausch
von Vektormesonen. Wir beschreiben diese Terme mit Hilfe der
Lagrangedichte
\beq
 {\cal L}_{\eta NN} &=& \frac{f_{\eta NN}}{m_\eta} 
     \bar{\psi} \gamma_5 \gamma_\mu \psi \partial^\mu \eta \; ,\\
 {\cal L}_{\eta NN^{*}} &=& \frac{f_{\eta NN^{*}}}{m_\eta} 
     \bar{\psi}_{N^{*}} \gamma_\mu \psi \partial^\mu \eta + h.c. \; ,\\
 {\cal L}_{{\mini V}\eta\gamma} &=& \frac{g_{{\mini V}\eta\gamma}}{2m_\eta}
                 \epsilon_{\alpha\beta\gamma\delta}
		  F^{\alpha\beta} \partial^\gamma
		  V^\delta \eta  \; ,
\eeq
wobei $V$ f\"ur die Vektormesonen $\rho,\omega$ steht. \"Uber
die St\"arke der $\eta NN$-Wechselwirkung gibt es nur sehr
widerspr\"uchliche Informationen \cite{Dum82}. Eine einfache
Absch\"atzung gewinnt man mit Hilfe der $SU(3)$-Relation zwischen
den pseudoskalaren Kopplungen
\be
\label{octcoup}
  g_{\eta_8} = \frac{1}{\sqrt{3}}\, \frac{3F/D-1}{F/D+1} \, g_\pi \; .
\ee      
Dabei bezeichnen $D,F$ die symmetrischen bzw.~antisymmetrischen 
$SU(3)$-Kopplungen. Ph\"anomenologische Untersuchungen haben f\"ur diese 
Parameter die Werte $F=0.47\pm 0.04$ und $D=0.81\pm 0.03$ ergeben \cite{JM90}.
Unter Verwendung von $SU(6)$-Spin-Flavor-Wellenfunktionen gilt 
dar\"uber hinaus $g_{\eta_8} = \frac{\sqrt{3}}{5}g_\pi$ und 
$g_{\eta_0} = \frac{\sqrt{6}}{5}g_\pi$. Die Kopplung der physikalischen
Zust\"ande ist recht sensitiv auf die Gr\"o\ss e der $\eta-\eta'$ 
Mischung. F\"ur Mischungswinkel im Bereich $\thp=0^\circ -25^\circ$
finden wir $g_\eta=4.6-7.0$. 

F\"ur unsere Berechnung verwenden wir den Wert $\thp=-12.5^\circ$, 
der die Pseudovektorkopplung $f_{\eta NN}=1.7$ ergibt. 
Der Nukleon-Bornterm liefert in diesem Fall
\beq
 E_{0+}^N (\eta p) &=& -\frac{e}{4\pi} \frac{f_{\eta NN}}{m_\eta}
   \frac{\mu_\eta}{(1+\mu_\eta)^{3/2}}\, \left( 1-\frac{\mu_\eta}{2}
   \kappa_p \right) \\[0.2cm]
   &\simeq& \mbox{} -1.4 \su , \nonumber 
\eeq     
wobei $\mu_\eta=m_\eta/M$ das Verh\"altnis der Eta- zur Nukleonmasse
bezeichnet. Im Gegensatz zum Niederenergietheorem zur 
Pionphotoproduktion ist es hier wenig sinnvoll, den kinematischen 
Faktor in Potenzen von $\mu_\eta$ zu entwickeln. 
 
Die $\eta NN^{*}$-Kopplung l\"a\ss t sich analog zu Gleichung (\ref{rescoup})
aus dem Wert der partiellen Breite $\Gamma (N(1535) \to N\eta)=75$ MeV
zu $f_{\eta NN^{*}}=1.94$ bestimmen. Bei der Photoproduktion von 
Etamesonen mu\ss\ man selbst an der Schwelle die endliche Breite der
Resonanz auf Grund des offenen Zerfallskanals $N^{*}\to N\pi$
ber\"ucksichtigen. Zu diesem Zweck  parametrisieren wir die 
Energieabh\"angigkeit der totalen Breite in der Form
\be
 \Gamma (s) = \Gamma_\pi \left(\frac{q_\pi}{q_\pi^R} \right)
   + \Gamma_\eta \left( \frac{q_\eta}{q_\eta^R} \right) \; ,
\ee
wobei $q_\pi,q_\eta$ die Impulse der Mesonen im Schwerpunktsystem
und $q_\pi^R,q_\eta^R$ die entsprechenden Werte bei $\sqrt{s}=M_R$
bezeichnen. An der $\eta N$-Schwelle ist dann $\Gamma (s_{th})=59$
MeV. Mit Hilfe des im letzten Abschnitt bestimmten anomalen magnetischen
Moments f\"ur den \"Ubergang $\gamma N \to N^{*}$ finden wir
f\"ur den Resonanzbeitrag
\begin{table}
\caption{Vergleich der f\"uhrenden Beitr\"age zur elektrischen
Dipolamplitude f\"ur die Reaktionen $\gamma p \to \pi^0 p$ und 
$\gamma p \to \eta p$. $E_{0+}$ in Einheiten $10^{-3} m_\pi^{-1}$.}
\begin{center}
\begin{tabular}{|l||r|r|}\hline
               & $\gamma p\to \pi^0 p$  &  $\gamma p\to \eta p$ \\ \hline\hline
 Nukleon-Bornterm               & $ -2.32$& $-1.4$              \\
 Vektormesonen ($\rho,\omega$)  & $0.21$  &  3.6	         \\
 Resonanz $N(1535)$             & $0.06$  &   9.4                \\
 Total                          & $-2.05$ &  11.6                 \\ \hline
\end{tabular}
\end{center}    
\end{table}
\be
  {\rm Re} E_{0+}^{N^{*}} (\eta p) = 9.4  \su .
\ee
Auch Vektormesonen spielen in der Etaproduktion eine deutlich
gr\"o\ss ere Rolle, als dies in der Pionproduktion der Fall ist.
Die Vektormeson-Nukleon Kopplungskonstanten haben wir bereits im
letzten Kapitel diskutiert. Am $VNN$-Vertex verwenden wir einen 
Formfaktor der Monopolform
\be
 F(t) = \frac{\Lambda^2-m_V^2}{\Lambda^2-t}
\ee
mit dem Cutoff $\Lambda=1.4$ GeV. Die $V\eta\gamma$-Kopplungen lassen sich
aus den experimentell bestimmten Zerfallsbreiten $\Gamma (V\to\eta
\gamma )$ ermitteln. Mit den Werten  aus \cite{Dum82} finden wir  
\be
 f_{\rho\mini NN}g_{\rho\eta\gamma}(1+\kappa_\rho)
  +f_{\omega\mini NN}g_{\omega\eta\gamma} \simeq 20.63 \; .
\ee     
Da die $\rho$- und $\omega$-Massen praktisch entartet sind, bestimmt
diese effektive Kopplung den Vektormesonbeitrag zur Etaproduktion
an der Schwelle
\beq
 E_{0+}^{V}(\eta p) &=& \frac{eM}{16\pi}
 \sum_V f_{\mini VNN}g_{{\mini V}\eta\gamma}(1+\kappa_V)
 \mu_\eta^2 \frac{2+\mu_\eta}
   {(1+\mu_\eta)^{3/2}} \frac{F(t)}
   {m_\eta^2 +m_V^2(1+\mu_\eta)}  \\[0.2cm]
   &\simeq& 3.58 \su \; . \nonumber
\eeq
Wir haben die bisher diskutierten Beitr\"age in Tabelle 3.1 gesammelt 
und mit den entsprechenden Resultaten in der Pionphotoproduktion 
verglichen. Man erkennt deutlich den sehr unterschiedlichen Charakter
dieser beiden Prozesse. 

Es ist interessant, die m\"oglichen Auswirkungen expliziter chiraler
Symmetriebrechung in der Photoproduktion von Eta-Mesonen  zu
untersuchen. Eine denkbare Konsequenz ist die Tatsache, da\ss\ an der
physikalischen Schwelle die Pseudovektorkopplung des Eta-Mesons
an das Nukleon nicht notwendig bevorzugt ist. Wir betrachten daher
die allgemeinere Wechselwirkung \cite{BM91}
\be
 {\cal L}_{\eta NN} = (1-\epsilon)\frac{f_{\eta NN}}{m_\eta}       
     \bar{\psi} \gamma_\mu \gamma_5 \psi \partial^\mu \eta
     + i\epsilon g_{\eta NN} \bar{\psi}\gamma_5\psi \eta \; ,
\ee
welche so konstruiert ist, da\ss\ die Kopplung f\"ur Nukleonen 
auf der Massenschale unabh\"angig von dem Parameter $\epsilon$ ist.
Dagegen zeigt der Nukleonbeitrag zur elektrischen Dipolamplitude 
an der Schwelle
\beq
 E_{0+}^N (\eta p) &=& -\frac{e}{4\pi} \frac{f_{\eta NN}}{m_\eta}
   \frac{\mu_\eta}{(1+\mu_\eta)^{3/2}}\, \left( 1 +
   \kappa_p \left( \epsilon -(1-\epsilon)\frac{\mu_\eta}{2} \right) \right) 
   \\[0.2cm]
   &\simeq& -(1.4+6.1\epsilon)\su \nonumber
\eeq   
eine starke Abh\"angigkeit von $\epsilon$ . Welcher Wert
von $\epsilon$ die beste Beschreibung der Etaproduktion an der
Schwelle liefert, l\"a\ss t sich letztlich nur durch eine
sorgf\"altige Untersuchung der differentiellen Wirkungsquerschnitte
unterhalb der Resonanzregion entscheiden. In derselben Weise
kann man auch f\"ur die $N(1535)$-Anregung eine skalare anstatt
der oben beschriebenen vektoriellen Kopplung verwenden
\be
 {\cal L}_{\eta NN^*} = (1-\alpha) \frac{f_{\eta NN^*}}{m_\eta}
    \bar{\psi}_{N^*}\gamma_\mu\psi\partial^\mu\eta
    +i\alpha g_{\eta NN^*}\bar{\psi}_{N^*}\psi\eta + h.c. \; .
\ee
Diese Modifikation beeinflu\ss t lediglich den nichtresonanten
Untergrund und hat daher nur geringe Auswirkung auf die
elektrische Dipolamplitude an der Schwelle. 

Wie im Abschnitt 2.5 diskutiert, kann die explizite Brechung der chiralen
Symmetrie, in diesem Fall insbesondere durch die Masse des seltsamen 
Quarks, eine zus\"atzliche Korrektur an die elektrische Dipolamplitude
liefern. Betrachtet man das Eta-Meson als reinen
Oktett-Zustand, dann ergibt sich mit Hilfe der oben verwendeten
Methoden
\beq
\label{sigeta}
 \Delta E_{0+} (\eta p)&=& \frac{e}{4\pi} \frac{1}{1+\mu_\eta}
   \frac{\overline{m}}{f_\eta m_\eta}
    \big( b_0 g_T^{\em sing} + b_3 g_T^3 + \delta b \,(g_T^8-g_T^{\em sing})
   \big) \; ,   \\[0.1cm]
   b_0      &=& \frac{1}{3\sqrt{3}}\; , 
                \hspace{2.5cm} b_3=\frac{1}{\sqrt{3}} \; ,\\
   \delta b &=& \frac{1}{9\sqrt 3}\left( 1 -4\frac{m_s}{\overline m}
                \right)\,  , 	   
\eeq
wobei $g_T^{\em sing}$ die Flavorsingletkopplung im Tensorkanal bezeichnet
und $f_\eta\simeq f_\pi$ die Eta-Meson-Zerfallskonstante ist. 
Die St\"arke der Korrektur h\"angt wesentlich von der Gr\"o\ss e
des flavormischenden Parameters $\delta g_T \equiv g_T^8-g_T^{\em sing}$ ab.

Nach der ph\"anomenologisch erfolgreichen Zweig-Regel verschwindet 
der Strangeness-In\-halt des Protons, und es gilt $\delta g_T=0$.
In diesem Fall ist $\Delta E_{0+}(\eta p)$ proportional zu 
$\overline{m}/m_\eta$ und liefert nur geringf\"ugige Korrekturen.
Verwendet man das in Abschnitt 2.5 diskutierte nichtrelativistische
Quarkmodell, dann ist $g_T^{\em sing}=1$ sowie $g_T^3=5/3$, und wir finden
$\Delta E_{0+}(\eta p)=0.4\su$. Ist dagegen $\delta g_T\neq 0$,
so ist die Korrektur proportional zu $m_s/m_\eta$ und kann einen 
substantiellen Beitrag zur Schwellenamplitude liefern. 
\cld

\chapter{Chirale Modelle des Nukleons}
%revised Jan. 2, 1992 
Wir haben in  Kapitel 2 verschiedene Korrekturen zum
Niederenergietheorem f\"ur die Pionphotoproduktion untersucht. 
Dabei haben wir festgestellt, da\ss\ die explizite Brechung der chiralen 
Symmetrie durch die Quarkmassen in der QCD-Lagrangedichte 
im Fall neutraler Pionen einen wesentlichen Beitrag 
zur elektrischen Dipolamplitude liefern kann.  Diese Feststellung
beruht allerdings auf der Verwendung eines nichtrelativistischen
Quarkmodells des Nukleons, um Matrixelemente des symmetriebrechenden Terms 
abzusch\"atzen. Solche Modelle verletzen die chirale Symmetrie bereits im
Ansatz, so da\ss\ das Resultat aus Abschnitt 2.5, $\Delta E_{0+}
(\pi^0p)=1.6\su$, mit Vorsicht zu betrachten ist. Wir wollen in diesem 
Kapitel versuchen, mit Hilfe chiraler Modelle des Nukleons eine verbesserte 
Bestimmung der symmetriebrechenden Amplitude $\Delta E_{0+}(\pi^0N)$ 
zu gewinnen.

\section{Die Spinstruktur des Nukleons}
Bei der Untersuchung geeigneter Nukleonmodelle wollen wir gro\ss en Wert 
auf eine genaue Beschreibung anderer Matrixelemente des Nukleons legen.
Besonderes Gewicht werden wir in diesem Zusammenhang der axialen Struktur
des Nukleons geben. Die sym\-me\-trie\-brechende Amplitude 
$\Delta E_{0+}$ ist im wesentlichen durch
Matrixelemente eines Flavorsingletstroms bestimmt. Wir wollen daher im 
ersten Abschnitt dieses Kapitels eine kurze Diskussion der experimentellen 
Informationen \"uber die Axialvektorkopplung im Flavorsinglet-Kanal geben.

Im Jahre 1987 berichtete die EMC-Gruppe \cite{EMC89} \"uber eine 
Messung der Spinasymmetrie 
\be
  A = \frac{\sigma (\mu\!\uparrow p\!\uparrow)
            -\sigma (\mu\!\uparrow p\!\downarrow)}
	    {\sigma (\mu\!\uparrow p\!\uparrow)
            +\sigma (\mu\!\uparrow p\!\downarrow)}    
\ee
in der tief-inelastischen Streuung polarisierter Myonen an polarisierten 
Protonen. Dabei bezeichnet $\sigma (\mu\!\uparrow p\!\uparrow\downarrow)$ 
den differentiellen Wirkungsquerschnitt $d\sigma/(dQ^2dx)$ f\"ur die 
Streuung longitudinal polarisierter Myonen an Protonen mit parallel 
bzw.~antiparallel ausgerichtetem Spin. Die Asymmetrie ist f\"ur Werte 
der Bjoerkenvariable $x$ im Bereich $0.01<x<0.7$ bestimmt worden. Die 
entsprechenden Impuls\"ubertr\"age liegen zwischen $Q^2=3.5\,\gev^2$
bei $x=0.01$ und $Q^2=29.5\,\gev^2$ bei $x=0.7$. Im kinematischen 
Bereich des EMC-Experiments ist die Asymmetrie in guter N\"aherung durch das 
Verh\"altnis der spinabh\"angigen Strukturfunktion $g_1^p$ und der 
unpolarisierten Strukturfunktion $F_1^p$
\be
  A =\frac{g_1^p(x,Q^2)}{F_1^p(x,Q^2)}
\ee
gegeben. Die Operatorproduktentwicklung liefert eine Vorhersage f\"ur das 
erste Moment von $g_1^p$ \cite{JM90}: 
\be
 \int_0^1 g_1^p(x,Q^2)dx =\frac{1}{2} \left( \frac{4}{9}\Delta u
   +\frac{1}{9}\Delta d +\frac{1}{9}\Delta s \right)
   \left( 1-\frac{\alpha_s}{\pi} + {\cal O}(\alpha_s^2) \right) \; .
\ee       
Die Spinasymmetrie $\Delta q$  der Quarkflavors $q=u,d,s$ ist durch
Matrixelemente des Axialvektorstroms definiert
\be
   <N(p)|\bar{q}\gamma_\mu\gamma_5 q|N(p)>=\Delta q s_\mu\; ,
\ee    	    
wobei $s_\mu=\bar u(p)\gamma_\mu\gamma_5u(p)$ den kovarianten Spin des 
Nukleons bezeichnet. Mit Hilfe schwacher Zerf\"alle und der 
$SU(3)$-Flavorsymmetrie lassen sich zwei Kombinationen von
$\Delta u,\Delta d$ und $\Delta s$ experimentell festlegen
\beq
  g_A^3 &=& \Delta u-\Delta d = 1.255\pm 0.006  \\
  g_A^8 &=& \Delta u+\Delta d -2\Delta s = 0.6 \pm 0.1
\eeq
Das Resultat des EMC-Experiments, $\int g_1^p dx = 0.114\pm 0.036$, 
liefert eine weitere Kombination von $\Delta u,\Delta d,\Delta s$ und
erm\"oglicht die individuelle Bestimmung der $\Delta q$
\be
  \Delta u  =    0.74\pm 0.10  \hspace{0.5cm}
  \Delta d  =  - 0.54\pm 0.10  \hspace{0.5cm}
  \Delta s  =  - 0.20\pm 0.11  \; .
\ee
\"Uberraschend ist der starke Polarisationsgrad der seltsamen Quarks
im Nukleon. Dar\"uber hinaus implizieren diese Zahlen  einen
sehr kleinen Wert f\"ur die 
Flavorsinglet-Axial\-vek\-tor\-kopplungs\-kon\-stan\-ten des Nukleons
\be
 g_A^{\em sing} = \Delta u + \Delta d+\Delta s = 0.01\pm 0.29 \; .
\ee
Diese Gr\"o\ss e ist ein Ma\ss\ f\"ur den Beitrag der Quarkspins 
zum Spin des Nukleons. Nach dem  EMC-Experiment ist dieser Wert 
mit Null vertr\"aglich. Auch der Anteil der leichten Quarks am 
Nukleonspin $g_A^0 = \Delta u + \Delta d=0.20$ ist erstaunlich 
gering.

Ber\"ucksichtigt man in der Operatorproduktentwicklung die 
Effekte der $U(1)_A$-Anomalie, so ergibt sich ein zus\"atzlicher
gluonischer Beitrag $\Delta \Gamma =\alpha_s/(2\pi)\Delta g$ 
zu den Spinasymmetrien $\Delta q$ \cite{AR88}. Dabei bezeichnet 
$\Delta g=g_+-g_-$ das erste Moment der Differenz der polarisierten
Gluonverteilungen mit Spins parallel bzw.~antiparallel zum Spin des
Protons. Eine obere Grenze f\"ur $\Delta g$ ergibt sich aus der
unpolarisierten Gluonverteilung. Altarelli und Stirling \cite{AS89}
verwenden die Absch\"atzung $\Delta \Gamma = 0.1$ und finden
$g_A^{\em sing}=0.3$ sowie $g_A^0=0.4$.   

\section{Das chirale Bagmodell}
Das chirale Bagmodell beschreibt das Nukleon als ein System
masseloser Quarks, die in einer sph\"arischen Kavit\"at vom Radius 
$R$ eingeschlossen sind. An der Bagoberfl\"ache koppeln die Quarks an 
Mesonenfelder, die die Struktur des Nukleons bei gr\"o\ss eren
Abst\"anden bestimmen . Die  Lagrangedichte des Modells lautet 
\be
\label{LCB}
{\cal L}= \left( \bar{\psi}\frac{i}{2}\gamma^{\mu}\stackrel{\leftrightarrow}
{\partial}_{\mu}\psi -B\right) \Theta (R-r) - \frac{1}{2} \bar{\psi}
e^{i\gamma_5 \vec{\tau}\cdot\vec{\phi}}\psi\delta (r-R)
+{\cal L}_{mes}\Theta (r-R) \; .
\ee
Wir beschr\"anken uns hier auf Flavor-$SU(2)$, so da\ss\ $\psi=(u,d)$ ein 
Isodoublet und $\vec{\phi}$ das Isotriplet Pionfeld bezeichnet. Die Struktur 
des Kopplungsterms ist durch die Forderung nach chiraler 
$SU(2)_L\times SU(2)_R$-Invarianz der Wechselwirkung diktiert.
Um die Beschreibung der elektromagnetischen Eigenschaften des 
Nukleons zu verbessern, enth\"alt das Modell neben den Pionfeldern
auch die Vektormesonen $(\rho,a_1,\omega)$ als zus\"atzliche 
Freiheitsgrade. Die detaillierte Form der mesonischen Lagrangedichte 
${\cal L}_{mes}$ sowie die resultierenden Bewegungsgleichungen finden 
sich in \cite{HTW90}. Wir haben die mesonischen Parameter $f_\pi=93$ MeV 
und $g_{\rho\pi\pi}=5.85$ durch ihre empirischen Werte fixiert. 

Die Bewegungsgleichungen werden mit Hilfe eines Ansatzes maximaler 
Symmetrie, des ''Hedgehog``-Ansatzes 
\be
\label{hedge}
\vec{\phi}(\vec{r}) = \hat{r} f_\pi \Theta (r)
\ee
f\"ur das Pionfeld gel\"ost. Die dimensionslose Funktion $\Theta (r)$
bezeichnet man als chiralen Winkel. Auf Grund des Hedgehog-Ansatzes 
kommutiert der Hamiltonoperator nicht mit dem Drehimpuls- oder
Isospinoperator. Quarkzust\"ande lassen sich daher lediglich durch ihren 
Grandspin $\vec{G}=\vec{L}+\vec{S}+\vec{I}$ und ihre Parit\"at $\pi$
klassifizieren. Wellenfunktionen mit gutem Spin und Isospin werden mit 
Hilfe der semiklassischen Crankingmethode \cite{KJR86} konstruiert. 

Wir kommen nun zur Bestimmung der Matrixelemente des Kommutators
$[\dot Q_5^a,V_i^{em}]$ im chiralen Bagmodell. Im Au\ss enraum der 
Kavit\"at ist
\be
  \dot Q_5^a =  f_\pi m_\pi^2\int d^3r\,\pi^{a}(\vec r\,)\; ,
\ee
wobei $\pi^{a}=\frac{f_\pi \sin\Theta}{3}\tau^{a}\vec{\sigma}\cdot
\hat{r}$ das kanonische Pionfeld bezeichnet. Der Kommutator 
$[\dot Q_5^a,V_i^{em}]$ verschwindet dann aus den bereits im
Zusammenhang mit dem linearen $\sigma$-Modell diskutierten Gr\"unden.
Im Innern des Bags sind die Quarks frei, und wir erhalten wie in 
Abschnitt 2.5
\be
\label{deleop}
\Delta E_{0+}(\pi^0p) = \frac{e}{4\pi f_\pi}\frac{\overline{m}}{m_\pi (1+\mu)}
  \left\{ \left( 1+\frac{\delta m}{6\overline{m}} \right) g_T^0
     + \left(\frac{1}{3}+\frac{\delta m}{2\overline{m}}\right) g_T^3
     \right\} \; ,
\ee
wobei die Tensorkopplungskonstanten im chiralen Bagmodell zu bestimmen 
sind. F\"ur die Isosingletkopplung ergibt sich
\be
\label{gt0cb}
g_T^{0}=\frac{T_{\overline{\sigma}\tau}}{2\Lambda_{tot}}\; ,
\ee
wobei $\Lambda_{tot}$ das Tr\"agheitsmoment des Solitons bezeichnet
\cite{HTW90} und $T_{\overline\sigma\tau}$ durch die Teilchen-Loch
Anregungen des Hedgehog-Grundzustands bestimmt ist. Die detaillierte
Berechnung dieser Gr\"o\ss e findet sich in Anhang C. Die 
Isovektorkopplung hat keinen gro\ss en Einflu\ss\ auf die
Amplitude $\Delta E_{0+}(\pi^0p)$. Wir beschreiben die Bestimmung
von $g_T^3$ in der Ver\"offentlichung \cite{SW90}.

\begin{figure}
\caption{Korrekturen zur elektrischen Dipolamplitude auf Grund expliziter 
Symmetriebrechung im chiralen Bagmodell.}
\vspace{9cm}
\end{figure}
\begin{figure}
\caption{Axiale Kopplungskonstanten  des Nukleons im chiralen Bagmodell.}
\vspace{9cm}
\end{figure}

Die Ergebnisse f\"ur die Korrektur $\Delta E_{0+}$ im chiralen Bagmodell 
finden sich in Abbildung 4.1. Sie h\"angen stark vom
verwendeten Bagradius ab. Insbesondere verschwindet $\Delta E_{0+}$
im Grenzfall $R\to 0$, was dem Resultat in einem rein mesonischen 
Solitonmodell entspricht. Die starke Abh\"angigkeit vom Bagradius
widerspricht an und f\"ur sich einer der grundlegenden Ideen des chiralen 
Bagmodells, da\ss\ n\"amlich physikalische Observable in gewissen Umfang 
unabh\"angig von der Realisierung der Dynamik mit Hilfe
von mesonischen oder Quark-Gluon Freiheitsgraden sein sollte. Sie mag 
daher ein Hinweis auf das Fehlen wichtiger gluonischer Beitr\"age in 
unserer Beschreibung sein.
     
%\begin{table}
%\caption{Korrekturen zur elektrischen Dipolamplitude auf Grund expliziter 
%Symmetriebrechung im chiralen Bag Modell.} 
%\begin{center}
%\begin{tabular}{|c||c|c|c|}\hline
% R [fm]   & $\DEop$         & $\DEon$       & $\DEcn$        \\ \hline\hline
% 0.30     &   0.07          &    0.09       &   0.16         \\ 
% 0.52     &   0.22          &    0.20       &   0.37         \\  
% 0.75     &   0.51          &    0.28       &   0.56         \\  
% 1.00     &   0.89          &    0.27       &   0.61         \\ 
% 1.30     &   1.14          &    0.25       &   0.62        \\ \hline
%\end{tabular}
%\end{center}
%\end{table} 

Eine \"ahnlich starke Abh\"angigkeit vom Bagradius zeigt auch die
axiale Kopplung im Isosinglet-Kanal, siehe Abbildung 4.1.  F\"ur den 
in ph\"anomenologischen Anwendungen des Modells gew\"ohnlich bevorzugten Wert 
des chiralen Winkels $\Theta (R)=
\frac{\pi}{2}\;(R=0.52\,\rm fm)$ ergibt sich $g_A^0=0.12$. 
Konsistent mit dem  EMC-Resultat $g_A^0<0.4$ sind Bagradien bis etwa
$R=0.7$ fm. Das entspricht einer oberen Grenze f\"ur die
symmetriebrechende Amplitude $\Delta E_{0+}(\pi^0p)<0.6 \su$.    

\section{Das nichttopologische Solitonmodell}
Um die Modellabh\"angigkeit von $\Delta E_{0+}$ n\"aher zu untersuchen,
wollen wir den sym\-metrie\-brechenden Kommutator in einem 
nichttopologischen Solitonmodell \cite{BB85} untersuchen. 
Auch dieses Modell beschreibt das Nukleon als ein System von
Quarks, umgeben von einer mesonischen Polarisationswolke.
Im Gegensatz zum chiralen Bagmodell verzichtet das 
nichttopologische Solitonmodell auf eine r\"aumliche Trennung
zwischen diesen Freiheitsgraden. Das Modell basiert auf der
Lagrangedichte des linearen $\sigma$-Modells 
\beq
\label{linsig}
  {\cal L} & = & \bar  \psi [i
\partial  _\mu \gamma _\mu + g ( \sigma  + i \vec \tau \cdot \vec
\pi \gamma  _5 ) ] \psi \nonumber  \\
 & & + {1 \over 2} (\partial
_\mu  \sigma)^2  + {1 \over  2} (\partial  _\mu  \vec  \pi )^2  -
{\lambda^2  \over  4} 
 ( \sigma  ^2 + \vec  \pi  ^2  - \nu  ^2 )^2
 + {\cal L}_{sb}\; ,
\eeq	
wobei $\psi$ wie oben das Doublet der leichten Quarks bezeichnet, die
Mesonfelder $\sigma$ und $\vec\pi$ sich im Gegensatz zum chiralen 
Bagmodell aber nach einer linearen Darstellung von $SU(2)_L\times
SU(2)_R$ transformieren. 

Die Parameter $g$ und $\lambda$ bezeichnen die Quark-Meson und
Meson-Meson Kopplungskonstanten. Die Pionzerfallskonstante 
$f_\pi=93$ MeV und die Pionmasse $m_\pi=139.6$ MeV sind durch
ihre experimentellen Werte bestimmt. Dar\"uber hinaus ist
$\nu^2=f_\pi^2 -m_\pi^2/\lambda^2$, so da\ss\ die chirale 
Symmetrie im Grundzustand durch einen nichtverschwindenden 
Erwartungswert $\langle\sigma\rangle=-f_\pi$ spontan gebrochen ist. 
\begin{figure}
\caption{Korrekturen $\DEop$ (durchgezogene Linie) und $\DEon$
(gestrichelte Linie) im nichttopologischen Solitonmodell.
Die Ergebnisse des Peierls-Yoccoz Verfahrens und der 
semiklassischen Quantisierung sind mit PY bzw.~CM bezeichnet.}
\vspace{9cm}
\end{figure}
\begin{figure}
\caption{Axiale Kopplungskonstanten im nichttopologischen
Solitonmodell. Die Ergebnisse des Peierls-Yoccoz Verfahrens und der 
semiklassischen Quantisierung sind mit PY bzw.~CM bezeichnet.}
\vspace{9cm}
\end{figure}

Die explizite Brechung der chiralen Symmetrie ber\"ucksichtigen 
wir in der Lagrangedichte durch den Beitrag ${\cal L}_{sb} = 
-f_\pi m_\pi^2\sigma -\bar\psi M\psi$ \cite{JJP89}. Dabei enth\"alt das 
skalare Feld den nichtperturbativen  Anteil des Quarkkondensats,
w\"ahrend der zweite Term nur im Valenzquark-Sektor zu 
ber\"ucksichtigen ist. Wie im letzten Abschnitt tragen nur
die Quarkfelder zu dem Kommutator $[\dot Q_5^a,V_i^{em}]$ bei. 
Mit Hilfe der kanonischen Vertauschungsrelationen zwischen den
Fermionfeldern erhalten wir erneut das Resultat (\ref{deleop}),
wobei die Tensorkopplungen $g_T^0$ und $g_T^3$ im nichttopologischen
Solitonmodell zu bestimmen sind.


Zu diesem Zweck haben wir Wellenfunktionen des Nukleons mit Hilfe
der Peierls-Yoccoz-Projektion konstruiert. Dieses Verfahren ist
in Anhang D erl\"autert. In den Abbildungen 4.3 und 4.4 zeigen wir 
die Ergebnisse f\"ur $g_A$ und $\Delta E_{0+}(\pi^0N)$ als Funktion der 
Quark-Meson Kopplung $g$. F\"ur hinreichend gro\ss e Werte der Meson-Meson 
Kopplung sind die Resultate praktisch unabh\"angig von $\lambda$. 
Die korrekte Masse des Nukleons ergibt sich f\"ur die Wahl
$g=5.38$ und $\lambda =10$.  In diesem Fall ist $\DEop=0.9\su$,
ein Wert, der noch im Bereich eines relativistischen Konstituentenmodells
liegt. Man sollte allerdings beachten, da\ss\ im Rahmen der 
gew\"ahlten Methode auch die axialen Kopplungen des Nukleons
mit $g_A^0=0.45$ und $g_A^3=1.64$ recht gro\ss e Werte annehmen.
Dies mag ein Hinweis auf die Tatsache sein, da\ss\ das Modell 
generell die Rolle der Valenzquarks in der Struktur des Nukleons
\"ubersch\"atzt oder gewisse Beitr\"age im mesonischen bzw.~Quarksektor
der Theorie mehrfach ber\"ucksichtigt.


Die im Vergleich zum chiralen Bagmodell abweichenden Ergebnisse 
k\"onnen nat\"urlich auch in den unterschiedlichen 
Projektionsverfahren begr\"undet sein. Um diese Frage n\"aher 
zu untersuchen, haben wir die  semiklassische
Quantisierung auch auf das nichttopologische Solitonmodell
angewendet. Diese Methode ist in den Ver\"offentlichungen
\cite{SW91,CB86} n\"aher beschrieben.


In dieser Arbeit wollen wir lediglich die Ergebnisse kurz 
diskutieren. Sie sind den mit CM gekennzeichneten Kurven
in den Abbildungen 4.3 und 4.4 zu entnehemen. Die Resultate befinden 
sich in der Tat in besserer \"Ubereinstimmung mit den  Ergebnissen
aus dem chiralen Bagmodell als die mit Hilfe der Peierls-Yoccoz
Projektion gefundenen Werte. F\"ur den bevorzugten Parametersatz
$g=5.38,\lambda =10$ finden wir $\DEop=0.6\su$. Auch die axialen
Kopplungen des Nukleons sind mit $g_A^0=0.15$ und $g_A^3=1.41$ 
deutlich kleiner und damit n\"aher an ihren experimentellen 
Werten. 

%\begin{table}
%\caption{Korrekturen zur elektrischen Dipolamplitude auf Grund expliziter
%Symmetriebrechung im nichttopologischen Solitonmodell.}
%\begin{center}
%\begin{tabular}{|c||c|c|c||c|c|c|}\hline
%          & \multicolumn{3}{|c||}{Peierls-Yoccoz Projektion} 
%	  & \multicolumn{3}{c|}{semiklassische Projektion}  \\  
%   g      & $\DEop$         & $\DEon$        & $\DEcn$
%          & $\DEop$         & $\DEon$        & $\DEcn$\\ \hline\hline
%   4      &   1.09 & 0.84 & 0.99 & 1.03 & 0.86 & 0.66 \\ 
%   5      &   0.95 & 0.72 & 0.92 & 0.68 & 0.52 & 0.63 \\ 
%   6      &   0.89 & 0.66 & 0.88 & 0.53 & 0.37 & 0.61 \\ 
%   7      &   0.85 & 0.63 & 0.85 & 0.43 & 0.28 & 0.59 \\ 
%   7.8    &   0.82 & 0.61 & 0.83 & 0.39 & 0.24 & 0.58 \\ \hline
%\end{tabular}   
%\end{center}
%\end{table}

\section{Zusammenfassung}  
Wir haben in diesem Kapitel die symmetriebrechende Amplitude 
in verschiedenen chiralen Modellen des Nukleons bestimmt. 
Je nach verwendetem Parametersatz haben sich dabei Werte
in dem Bereich $\DEop =(0-1.1)\su$ ergeben. Die Resultate
zeigen im Rahmen der untersuchten Modelle eine deutliche 
Korrelation mit der Flavorsinglet-Axialvektorkopplung
des Nukleons. In dieser Eigenschaft sind die betrachteten 
Modelle einfachen Konstituentenbildern des Nukleons verwandt,
allerdings mit einem deutlich reduzierten $g_A^0$. 

Insgesamt scheint eine gute ph\"anomenologische Beschreibung 
der Eigenschaften des Nukleons nur mit symmetriebrechenden
Amplituden $\DEop <0.5 \su$ vertr\"aglich zu sein. Diese Korrektur
ist deutlich kleiner als die im Rahmen eines nichtrelativistischen
Quarkmodells gewonnene Ab\-sch\"a\-tzung $\DEop=1.6\su$. 
\cld

\pagestyle{empty}

\addcontentsline{toc}{chapter}{Teil II

Analyse spektraler Summenregeln in der QCD}

\vspace*{5cm}

\centerline{\bf\huge Teil II}

\vspace*{4cm}

\centerline{\bf\huge Analyse spektraler Summenregeln}

\vspace*{1cm}

\centerline{\bf\huge in der Quantenchromodynamik}

\cld

\chapter[Spektrale Summenregeln in der QCD]{Spektrale Summenregeln 
in der Quantenchromodynamik}
% revised Jan. 1, 1992
\section{Einf\"uhrung}
In Teil I dieser Arbeit haben wir Stromalgebratechniken verwendet,
um die Photoproduktion von Pionen am Nukleon in der N\"ahe der Schwelle 
zu untersuchen. Die Photoproduktionsamplitude ist mit Hilfe
der Reduktionsformel durch die Zweipunktfunktion eines Vektor- und 
eines Axialvektorstroms zwischen Nukleonzust\"anden bestimmt. Diese 
Korrelationsfunktion haben wir approximiert, indem wir nur Zwischenzust\"ande 
betrachtet haben, die einzelne Nukleonen oder Pionen enthalten.  

In diesem Teil der Arbeit wollen wir die Korrelationsfunktionen von
Vektor- und Axialvektorstr\"omen im Vakuum untersuchen. Zwar lassen 
sich auch diese Funktionen f\"ur kleine Impuls\"ubertr\"age durch die
Beitr\"age der niedrigsten Resonanzen approximieren, im Gegensatz 
zur oben betrachteten Situation besteht aber die M\"oglichkeit, die
zugeh\"origen Spektren direkt aus experimentellen Daten 
zu rekonstruieren. 

Auf Grund der Analytizit\"at der Korrelationsfunktion bestimmt die
Spektralfunktion das Verhalten des Korrelators in der gesamten 
komplexen Ebene. Damit ergibt sich die M\"oglichkeit,
die Konsistenz der Daten mit Theorien \"uber das asymptotische 
Verhalten des  Korrelators in QCD zu vergleichen. Historisch ist 
eine solche Analyse das erste Mal im  Charmonium-System durchgef\"uhrt
worden \cite{NOS78}. \"Ahnliche Untersuchungen existieren auch  f\"ur
den Vektorkorrelator \cite{LNT84,CM90,Cap91}, nicht aber f\"ur das 
Spektrum der Axialvektormesonen. Wir werden daher im n\"achsten 
Kapitel eine eingehende Analyse der existierenden Daten \"uber die 
Spektralfunktion im Axialvektorkanal vornehmen. Zun\"achst wollen wir 
jedoch eine kurze Einf\"uhrung in die Methode der QCD-Summenregeln geben. 

\section{Summenregeln und die Operatorprodukt\-ent\-wick\-lung}
Die Struktur der Quantenchromodynamik bei niederen Energien 
wird durch eine Reihe nichtperturbativer Ph\"anomene bestimmt.
Von Bedeutung sind vor allem  Confinement, der permanente
Einschlu\ss\ von Quarks und Gluonen in farbneutralen Hadronen,
sowie das Auftreten von Kondensaten, das hei\ss t 
nichtverschwindenden Vakuumerwartungswerten von  Quark- oder 
Gluonoperatoren. Ein Beispiel f\"ur die Rolle der Kondensate
bei der Bestimmung des Spektrums liefert der Vergleich  
der Korrelationsfunktionen im Vektor- und Axialvektorkanal
\beq
\label{vvcor}
  \Pi_{\mu\nu}^V  &=& -(g_{\mu\nu}q^2-q_\mu q_\nu) \Pi^V(q^2)  \\[0.1cm]
   & &\hspace{0.3cm} = \, i\int d^4x\, e^{iq\cdot x}\, \langle 0|T\left( 
   j_\mu^{(\rho)}(x)j_\nu^{(\rho)}(0) \right) |0\rangle \; , \nonumber \\
\label{aacor}
  \Pi_{\mu\nu}^A &=& -g_{\mu\nu}q^2 \Pi^{A_1}(q^2)
              + q_\mu q_\nu \Pi^{A_2}(q^2)   \\[0.1cm]
& & \hspace{0.3cm}  =\, i\int d^4x\, e^{iq\cdot x} \,\langle 0|T\left( 
   j_\mu^{(a_1)}(x)j_\nu^{(a_1)}(0) \right) |0\rangle \;  . \nonumber
\eeq
Die Str\"ome $j_\mu^{(\rho)} =\frac{1}{2}(\bar u\gamma_\mu u-
\bar d\gamma_\mu d)$ und $j_\mu^{(a_1)}=\frac{1}{2} (\bar u
\gamma_\mu\gamma_5 d-\bar d\gamma_\mu\gamma_5 d)$ tragen die 
Quantenzahlen des $\rho$ bzw.~$a_1$-Mesons. Wir betrachten den
isospinsymmetrischen Fall, das hei\ss t der Vektorstrom $j_\mu^{(\rho)}$
ist erhalten und die entsprechende Korrelationsfunktion transversal. 

Im chiralen Limes sind die St\"orungsentwicklungen der 
beiden Korrelationsfunktionen im Vektor- und Axialvektorkanal
identisch. Dagegen werden wir im n\"achsten Kapitel zeigen, 
da\ss\ die zugeh\"origen Spektralfunktionen eine sehr unterschiedliche 
Gestalt haben. In der Tat ist es ein nichtperturbativer  Effekt, die 
spontane Brechung der chiralen Symmetrie hervorgerufen durch die 
nichtverschwindenden Quarkkondensate, welcher die Aufspaltung der 
$\rho$- und $a_1$-Massen bewirkt.

Von ebenso gro\ss er Bedeutung f\"ur den Niederenergiesektor der QCD ist das
Gluonkondensat $\langle G_{\mu\nu}^{a}G^{a}_{\mu\nu}\rangle $. Dieses 
Matrixelement  bestimmt den gluonischen Beitrag zur Vakuumenergie und
ist ein Ordnungsparameter f\"ur die Brechung der Invarianz der QCD 
unter Skalentransformationen.
  

Wir ber\"ucksichtigen die
Rolle der Kondensate in der Korrelationsfunktion mit Hilfe
der Operatorproduktentwicklung (OPE), g\"ultig im tief
euklidischen Bereich $Q^2=-q^2\to\infty$
\be
\label{ope}
 \Pi (Q^2) =  C_{1\!\!1}  + \sum_{n=2} \frac{1}{Q^{2n}}
        C_n \langle 0| {\cal O}_n|0\rangle  \; .
\ee
Dabei liefert der Koeffizient $C_{1\!\!1}$ des Einheitsoperators
die \"ubliche St\"orungsreihe in Potenzen von $\alpha_s$,
w\"ahrend ${\cal O}_n$ einen Operator der Dimension $d=2n$
bezeichnet, dessen Vakuumerwartungswert durch nichtperturbative
Effekte bestimmt wird. Der kurzreichweitige Teil der 
Wechselwirkung der Str\"ome mit den Operatoren ${\cal O}_n$ ist
in den Wilson-Koeffizienten $C_n$ enthalten, die sich in 
in der \"ublichen Weise aus Feynman-Diagrammen bestimmen lassen. 
Nicht berechenbar auf dem gegenw\"artigen Stand der Theorie sind 
dagegen die  Vakuumerwartungswerte $\langle 0|{\cal O}_n|0\rangle $.

QCD Summenregeln machen von der Analytizit\"at der 
Korrelationsfunktion Gebrauch, um das Verhalten von $\Pi (Q^2)$ 
im tief euklidischen Bereich mit der Gestalt der Spektralfunktion 
bei niederen Energien in Verbindung zu bringen. Die Funktion 
$\Pi (Q^2)$ erf\"ullt die Dispersionsrelation 
\be 
\label{disprel}
 \Pi (Q^2) = (-1)^n \frac{Q^{2n}}{\pi} \int_{s_0}^{\infty}
    \frac{{\rm Im}\Pi (s)}{s^n(s+Q^2)} ds +
    \sum_{k=0}^{n-1} a_k Q^{2k}\, ,
\ee
wobei eine von den Eigenschaften des Stroms abh\"angige Zahl von 
Subtraktionen vorgenommen werden mu\ss .  Die entsprechenden  
Subtraktionskonstanten haben wir mit $a_k$ bezeichnet. 
Der Imagin\"arteil ${\rm Im}\Pi (Q^2)$ l\"a\ss t sich mit
Hilfe von  Unitarit\"atsbeziehungen aus physikalischen
Observablen bestimmen.

Im Prinzip existiert daher eine sehr starke Korrelation zwischen 
dem asymptotischen Verhalten der Funktion $\Pi (Q^2)$, parametrisiert
mit Hilfe der Kondensate $\langle {\cal O}_n\rangle $, und experimentell 
zug\"anglichen Gr\"o\ss en. In der Praxis ist man allerdings 
gezwungen, die OPE bei $n=3$ oder $n=4$ abzubrechen. Auch die
experimentellen Daten sind nur in einem beschr\"ankten Bereich 
bekannt und mit Fehlern behaftet. Wir werden uns daher mit der 
Frage befassen m\"ussen, in welchem Umfang eine Bestimmung der 
niederdimensionalen Kondensate aus den gemessenen Spektralfunktionen 
im Vektor- und Axialvektorkanal m\"oglich ist. 

Auf Grund experimenteller Unsicherheiten in der Bestimmung der Daten
sind verschiedene Formulierungen der Summenregel (\ref{disprel}) nicht 
in gleicher Weise zur Bestimmung der Vakuumparameter geeignet. Besonders 
bew\"ahrt hat sich die Boreltransformierte Dispersionsrelation   
\be
\label{borel}
 \hat L_B \Pi (Q^2) = \frac{1}{\pi} \int_{s_0}^{\infty} ds\,
  {\rm Im}\Pi (s) e^{-s\tau} \, ,
\ee  
wobei $\hat L_B$ den Boreloperator
\be
\label{borelop}
 \hat L_B = \left. \lim_{Q^2,n\to\infty} \frac{1}{(n-1)!}Q^{2n}
   \left( -\frac{d}{dQ^2} \right)^n  \right|_{n/Q^2=\tau} 
\ee
bezeichnet. Die Boreltransformation hat den Vorzug, das experimentell 
zuverl\"assiger bestimmte Niederenergieverhalten der 
Spektralfunktionen st\"arker zu gewichten und m\"ogliche 
Subtraktionskonstanten zu eliminieren. Dar\"uber hinaus werden 
h\"ohere Potenzen von $1/Q^2$ in der OPE durch zus\"atzliche 
Faktoren $1/n!$ unterdr\"uckt. 

Eine weitere Konsequenz der analytischen Struktur der 
Korrelationsfunktionen sind Summenregeln in einem endlichen
Energieintervall ({\em engl.} finite energy sum rule, FESR)
\be
\label{fesr}
 M_n (t_c)  = \frac{t_c^{n+1}}{\pi} \int_0^{2\pi}d\phi\,
     e^{i(n+1)\phi} \Pi (t_ce^{i\phi})\, .
\ee
Die rechte Seite der Summenregel testet das Verhalten der 
Funktion $\Pi (Q^2)$ auf einem Kreis in der komplexen
$q^2$ Ebene und l\"a\ss t sich f\"ur hinreichend gro\ss e 
$t_c$ mit Hilfe der OPE auswerten. Die linke Seite ist durch 
das n-te Moment der Spektralfunktion in einem endlichen
Intervall gegeben
\be
\label{moment}
 M_n (t_c) = \frac{1}{\pi} \int_{s_0}^{t_c}ds \, s^n 
    {\rm Im}\Pi (s)
\ee
und kann aus experimentellen Daten bestimmt werden. Ein 
Vorzug der FESR ist die Tatsache, da\ss\ die Relationen 
(\ref{fesr}) direkt auf Kondensate einer festen Dimension
projizieren. Allerdings sind diese Beziehungen f\"ur h\"ohere
$n$ zunehmend von Fehlern in den Eingabedaten beeintr\"achtigt. 

\section{Entwicklung nach lokalen Operatoren}
In diesem Abschnitt wollen wir die Wilson-Koeffizienten f\"ur die 
Korrelationsfunktionen im Vektor- und Axialvektorkanal angeben. Beginnen 
werden wir mit dem Vektorkorrelator
\beq
 \Pi_{\mu\nu}^V(q^2) &=& \frac{i}{4} \int d^4x \, e^{iq\cdot x}
    \langle 0|T\big[ (\bar u(0)\gamma_\mu u(0)-\bar d(0)\gamma_\mu d(0) )
    \\
    & & \hspace{4.3cm}  \cdot (\bar u(x)\gamma_\nu u(x)
       -\bar d(x)\gamma_\nu d(x) )\big] |0\rangle  \, .\nonumber
\eeq
Entwickelt man das zeitgeordnete Produkt der Feldoperatoren mit
Hilfe des Wick'schen Theorems und vernachl\"assigt die 
normalgeordneten Beitr\"age, so erh\"alt man die \"ubliche 
St\"orungsentwicklung, die den Koeffizienten
$C_{1\!\!1}$ bestimmt. Die Wilson-Koeffizienten der nichtperturbativen
Kondensate ergeben sich aus den normalgeordneten Beitr\"agen. 
Um Produkte von Feldoperatoren an verschiedenen Punkten als
Erwartungswerte lokaler Operatoren zu schreiben, entwickelt
man die Operatorprodukte in  eine  Taylorreihe in der 
Differenz der Argumente. Dieses Verfahren ist konsistent, da der 
Korrelator f\"ur hohe $Q^2$ durch das Produkt der Str\"ome bei
kleinen Abst\"anden bestimmt ist. Fluktuationen auf
gro\ss e L\"angenskalen manifestieren sich in den 
nichtverschwindenden Kondensaten. 


Die Operatorproduktentwicklung der Korrelationsfunktion l\"a\ss t
sich in die Form 
\be
\label{powexp} 
\Pi^V (Q^2) = -h_0^V \ln \left(\frac{Q^2}{\mu^2}\right) 
+ \sum_{n=2} \frac{1}{n}\,\frac{h_n^V}{Q^{2n}}
\ee
bringen, wobei $\mu^2$ die Renormierungsskala kennzeichnet.
Mit Hilfe der geschilderten Methoden findet man die 
Koeffizienten
\beq
 h_0^V &=& \frac{1}{8\pi^2} \left( 1+\frac{\alpha_s}{\pi}\right)\; , \\
 h_2^V &=& m_u\langle \bar{u}u\rangle +m_d\langle \bar{d}d\rangle  + 
  \frac{1}{12}  \langle \frac{\alpha_s}{\pi} 
        G_{\mu\nu}^{a}G^{a}_{\mu\nu} \rangle \; ,   \\
 h_3^V &=& -\frac{3\pi\alpha_s}{2}
        \langle \left(\bar{u}\gamma_\mu\gamma_5\lambda^{a}u -
	       \bar{d}\gamma_\mu\gamma_5\lambda^{a}d \right)^2 \rangle    \\
      & & \;\mbox{}-\frac{\pi\alpha_s}{3} 
       \langle  (\bar{u}\gamma_\mu\lambda^{a}u +
          \bar{d}\gamma_\mu\lambda^{a}d ) 
	  \sum_{q=u,d,s} \bar{q}\gamma_\mu\lambda^{a} q\rangle  \nonumber \, .
\eeq
Dabei bezeichnet $\alpha_s$ die impulsabh\"angige laufende 
Kopplungskonstante. In Ein-Schlei\-fen\-n\"ahe\-rung ist
\be
\label{runcoupl}
  \frac{\alpha_s(Q^2)}{\pi} = \left( -\frac{\beta_1}{2}
      \ln \left(\frac{Q^2}{\Lambda^2}\right) \right)^{-1}
\ee
mit $\beta_1=-\frac{11}{2}+\frac{N_f}{3}$ und dem Skalenparameter
$\Lambda=230\pm 80$ MeV \cite{PDG90}. Im perturbativen Teil
des Resultats haben wir Korrekturen auf Grund der endlichen 
Strommassen der Quarks vernachl\"assigt. St\"orungstheoretische 
Korrekturen zum Einheitsoperator sind bis zur Ordnung $\alpha_s^3$  
bestimmt worden \cite{GKL88}, w\"ahrend die entsprechenden Korrekturen an 
$\langle \bar qq\rangle $ und $\langle G^2\rangle $ nur  bis zur Ordnung $\alpha_s$ bekannt 
sind \cite{Nar89}. Wir haben die Operatorproduktentwicklung bei
Termen der Dimension $d=6$ abgebrochen. In der n\"achst\-h\"oheren 
Ordnung der OPE sind lediglich die Wilson-Koeffizienten f\"ur alle rein 
gluonischen Operatoren berechnet worden \cite{BG85}.

Der Beitrag  der Dimension $d=4$ in der OPE enth\"alt  neben
dem aus der GOR-Relation (\ref{GOR}) bekannten Matrixelement
$m_u\langle \bar uu\rangle +m_d\langle \bar dd\rangle $ auch den rein 
gluonischen Erwartungswert $\langle\frac{\alpha_s}{\pi}G^2\rangle $. Dieses 
Matrixelement ist besonders intensiv im Zusammenhang mit
Charmoniumzust\"anden untersucht worden. In diesem System 
ist die Rolle der Quarkkondensate auf Grund der gro\ss en
Masse des charm-Quarks unterdr\"uckt. Der kanonische
Wert des Gluonkondensats aus einer  Analyse der angeregten
Zust\"ande des $c\bar c$-Systems betr\"agt  $\langle \frac{\alpha_s}{\pi}
G^2\rangle =(360\pm 20{\rm MeV})^4$ \cite{RRY85}.  

Die Quarkkondensate der Dimension $d=6$  lassen sich mit Hilfe
der Faktorisierungshypothese \cite{SVZ79}
\be
\label{fact}
  \langle \bar\psi\Gamma_1\psi\bar\psi\Gamma_2\psi \rangle  =
    \frac{1}{N^2} \big( {\rm Tr}(\Gamma_1)\,{\rm Tr}(\Gamma_2)  
    - {\rm Tr} (\Gamma_1\Gamma_2) \big) \, \langle \bar\psi\psi \rangle ^2
\ee    
absch\"atzen. Dabei bezeichnet $\psi =(u,d,\ldots)$ einen Quarkspinor
in $SU(N_f)$, und die Normierungskonstante ist durch $N=4N_cN_f$
gegeben. Auf diese Weise findet man 
\be
  h_3 = -\frac{112}{27}\xi^V\pi\alpha_s \, \langle \bar qq\rangle^2\; ,
\ee
wobei die Gr\"o\ss e $\xi^V$ Abweichungen von der exakten 
Faktorisierung ($\xi^V=1$) parametrisiert. Da sich die 
anomalen Dimensionen der Kopplungskonstante und des Kondensats
praktisch aufheben, ist $h_3$ nur sehr schwach vom 
Normierungspunkt abh\"angig. Besitzt man eine Absch\"atzung
der Quarkmassen, so l\"a\ss t sich der Wert des Quarkkondensats  
mit Hilfe der GOR-Relation berechnen. Da weder das Kondensat noch die 
Strommassen invariant unter der Renormierungsgruppe sind, ist
es zu diesem Zweck allerdings sehr wichtig, die Skala zu 
kennen, bei der die Quarkmassen bestimmt worden sind. 
Verwendet man den Wert $\left. (m_u+m_d)\right|_{Q^2=1\,
{\rm GeV}^2}=14$ MeV, den wir in Kapitel 2 angegeben haben,
so findet man $\langle \bar qq\rangle =-(230\,{\rm MeV})^3$ bei $Q^2=1\,
{\rm GeV}^2$. 

Auf Grund der expliziten Brechung der chiralen Symmetrie wird
die Korrelationsfunktion im Axialvektorkanal durch zwei 
unabh\"angige invariante Funktionen charakterisiert. 
Diese lassen sich wie in (\ref{powexp}) entwickeln. F\"ur den
Koeffizienten der $q_\mu q_\nu$-Struktur findet man 
\beq
 h_0^{A_2} &=& \frac{1}{8\pi^2} \left( 1+\frac{\alpha_s}{\pi}\right) \; ,\\
 h_2^{A_2} &=& m_u\langle \bar{u}u\rangle + m_d\langle \bar{d}d\rangle  + 
      \frac{1}{12} \langle \frac{\alpha_s}{\pi} 
             G_{\mu\nu}^{a}G^{a}_{\mu\nu} \rangle\; ,    \\
 h_3^{A_2} &=& -\frac{3\pi\alpha_s}{2}
        \langle \left(\bar{u}\gamma_\mu\lambda^{a}u -
	       \bar{d}\gamma_\mu\lambda^{a}d \right)^2 \rangle    \\
     & & \;\mbox{}-\frac{\pi\alpha_s}{3} 
       \langle  (\bar{u}\gamma_\mu\lambda^{a}u +
          \bar{d}\gamma_\mu\lambda^{a}d ) 
	  \sum_{q=u,d,s} \bar{q}\gamma_\mu\lambda^{a} q\rangle  \nonumber \, .
\eeq
Im chiralen Limes sind der Koeffizient des Einheitsoperators 
sowie die Beitr\"age des Gluonkondensats im Vektor- und Axialvektorkanal
identisch. Betrachtet man den Koeffizienten $h_3$, so liefert
der Axialstromkorrelator eine andere Diracstruktur im ersten der beiden 
Beitr\"age. Verwendet man die Faktorisierungshypothese
\be 
 h_3^{A_2} = \left(\frac{11}{7}\right)\frac{112}{27} 
       \xi^A\pi\alpha_s \, \langle \bar qq\rangle ^2 \, ,
\ee
so ergibt sich eine gegen\"uber dem Resultat im Vektorkanal
nur geringf\"ugig modifizierte numerische Konstante, aber ein 
ge\"andertes Vorzeichen. 

Die Differenz der beiden invarianten  Strukturen $\Pi^{A_1}$ und 
$\Pi^{A_2}$ ist in f\"uhrender Ordnung durch 
\be
  \Pi^{A_1} (Q^2) - \Pi^{A_2} (Q^2) = 
      \frac{m_u\langle \bar uu\rangle + m_d\langle \bar dd\rangle }{Q^4} 
\ee
gegeben. Diese Relation steht in engem Zusammenhang zur PCAC-Hypothese.
Saturiert man die Differenz der beiden Spektralfunktionen mit dem
Pionbeitrag, so ergibt sich die in Kapitel 2 ausgiebig diskutierte 
GOR-Relation
\be
 -m_\pi^2 f_\pi^2 = m_u \langle \bar uu\rangle + m_d\langle 
 \bar dd\rangle  .
\ee
  
\section{Spektrale Summenregeln}
Mit Hilfe der im letzten Abschnitt bestimmten Wilson-Koeffizienten
k\"onnen wir nun die explizite Form der Summenregeln f\"ur die
Spektralfunktionen ${\rm Im}\Pi^{V,A}$ angeben. So lautet die
Boreltransformierte Dispersionsrelation im Vektorkanal
\be 
\label{borelsum}
 \frac{1}{\pi} \int_{s_0}^\infty {\rm Im}\Pi^V(s)e^{-s\tau}ds
      = \frac{1}{\tau} \left\{ h_0^V + \frac{h_2^V}{2!} \tau^2
         + \frac{h_3^V}{3!}\tau^3 + \ldots \right\} \, .
\ee	 
H\"ohere Momente der Borelsummenregel ergeben sich durch Differenzieren
von (\ref{borelsum}) nach $\tau$. Der Borelparameter $\tau$ 
kontrolliert die Gewichtung des Spektrums und die relative
Bedeutung h\"oherer Korrekturen in der OPE. Im Prinzip ist 
die Summenregel (\ref{borelsum}) f\"ur beliebige Werte von
$\tau$ g\"ultig. In der Praxis mu\ss\ man allerdings den 
Borelparameter in einem gewissen Bereich w\"ahlen, um die 
Fehler auf Grund der vernachl\"assigten Terme in der 
OPE und der Unsicherheiten in der Gestalt der Spektralfunktion
unter Kontrolle zu halten.

Die zu (\ref{borelsum}) analogen Summenregeln in einem endlichen 
Energieintervall lauten 
\beq
\label{fesr1}
  t_c F_2(t_c) 
     &=&    8\pi \int_{s_0}^{t_c} {\rm Im} \Pi^V(s) \,ds \; , \\
\label{fesr2}
  c_4 + t_c^2 F_4(t_c) 
     &=&    16\pi \int_{s_0}^{t_c} {\rm Im} \Pi^V(s)\, s\, ds \; , \\     
\label{fesr3}
  c_6 - \frac{t_c^3}{2} F_6(t_c) 
     &=&    12\pi \int_{s_0}^{t_c} {\rm Im} \Pi^V(s)\, s^2\, ds  \; .
\eeq
Die erste dieser Beziehungen illustriert sehr anschaulich das 
Konzept der Dualit\"at zwischen den Beschreibungen der 
Korrelationsfunktion bei kleinen bzw.~gro\ss en 
Abst\"anden. Obwohl die Spektralfunktion bei niederen Energien 
nicht durch freie Quark-Antiquark\-zu\-st\"an\-de, sondern durch 
Resonanzbeitr\"age bestimmt ist, liefert ihr Integral den 
st\"orungstheoretischen Wert. Die spektrale St\"arke der gebundenen 
Zust\"ande entspricht daher \"uber ein hinreichend gro\ss es Intervall 
gemittelt dem Resultat f\"ur  asymptotisch 
freie Quarks. H\"ohere Momente der Spektralfunktion enthalten die 
nichtperturbativen Korrekturen  
\be
\label{defcor}
  c_4 = -\frac{h_2^V}{h_0^V} \hspace{1cm} 
  c_6 = -\frac{h_3^V}{2h_0^V} \, .
\ee
Verwendet man den oben angegebenen Wert des Gluonkondensats und 
die Faktorisierungshypothese, so findet man $c_4=-0.07\,\gev^4$ 
und $c_6=0.04\,\gev^6$. Perturbative Korrekturen sind in den 
Funktionen $F_{2n}(t_c)$ ber\"ucksichtigt. In erster Ordnung in $\alpha_s$ ist
$F_{2n}(t_c) = (1+\alpha_s(t_c)/\pi)$, w\"ahrend die  
Korrekturen in h\"oherer Ordnung abh\"angig von $n$ ist. Resultate
bis $\alpha_s^2$ finden sich in \cite{BDL88}.

Um hadronische Parameter mit Hilfe der Summenregeln zu fixieren, 
ist man gezwungen, eine m\"oglichst einfache Parametrisierung der
Spektralfunktion zu verwenden. Im Vektorkanal lautet der Ansatz
\be
\label{zerow}
 \frac{1}{\pi} {\rm Im}\Pi^V(s) = \frac{m_\rho^2}{g_\rho^2}
    \delta (s-m_\rho^2) + \frac{1}{8\pi^2}
     \left( 1+\frac{\alpha_s}{\pi} \right) \Theta (s-s_{th})\; ,
\ee
wobei der erste Term den Beitrag der $\rho$-Resonanz im Grenzfall
verschwindender Breite ber\"ucksichtigt und der zweite 
Term das Quark-Antiquarkkontinuum repr\"asentiert. Die Kopplung
des $\rho$-Mesons ist durch das Matrixelement
\be
  \langle 0| J_\mu^{(\rho)}(0) |\rho (p)\rangle  = \epsilon_\mu \,\frac{g_\rho}{m_\rho^2}
\ee
definiert. Die beiden Parameter $m_\rho$ und $g_\rho$ lassen sich aus 
der Boreltransformierten Summenregel (\ref{borelsum}) und deren 
erstem Moment bestimmen. Zu diesem Zweck fixiert man die 
Kontinuumschwelle $s_{th}$ mit Hilfe der Konsistenzbeziehung
(\ref{fesr1}) und w\"ahlt $\tau =m_\rho^{-2}$, um den Beitrag der
$\rho$-Resonanz zur Summeneregel zu maximieren. 
Shifman, Vainshtein und Sakharov \cite{SVZ79} verwenden
\be
\label{svzval}
 c_4 =-0.07\;{\rm GeV}^4\hspace{1.5cm}  c_6 = 0.09\;{\rm GeV}^6 ,
\ee
wobei sich der abweichende Wert von $c_6$ aus der sehr kleinen 
Skala erkl\"art, bei der SVZ die oben angegebene Absch\"atzung der 
leichten Quarkmassen verwenden. Mit diesen Kondensaten ergibt sich 
$m_\rho = 774$ MeV (exp. 770 MeV) sowie $\frac{g_\rho^2}{4\pi} = 2.3$
(exp. 2.36) in hervorragender \"Ubereinstimmung mit den experimentellen 
Werten. 

Im Axialvektorkanal liefern die beiden Strukturen $\Pi^{A_1}$
und $\Pi^{A_2}$ zwei unabh\"angige Borelsummenregeln
\beq 
 \frac{1}{\pi} \int_{s_0}^\infty {\rm Im}\Pi^{A_1}(s)se^{-s\tau}ds
      &=& \frac{1}{\tau^2} \left\{ h_0^{A_1} - \frac{h_2^{A_1}}{2!} \tau^2
         - \frac{2h_3^{A_1}}{3!}\tau^3 - \ldots \right\}, \\
 \frac{1}{\pi} \int_{s_0}^\infty {\rm Im}\Pi^{A_2}(s)e^{-s\tau}ds
      &=& \,\frac{1}{\tau} \,\left\{ h_0^{A_2} + \frac{h_2^{A_2}}{2!} \tau^2
         + \frac{h_3^{A_2}}{3!}\tau^3 + \ldots \right\}	 .
\eeq
In der verwendeten N\"aherung sind die Koeffizienten $h_n$ in den 
beiden Summenregeln identisch bis auf den Beitrag des Quarkkondensats
\be
 h_2^{A_1} - h_2^{A_2} = 2(m_u\langle \bar uu\rangle \!+\, m_d\langle \bar dd\rangle ) \; .
\ee
Numerisch ist diese Differenz klein im Vergleich mit dem Beitrag des
Gluonkondensats, so da\ss\ in recht guter N\"aherung $h_2^{A_1}
\simeq h_2^{A_2}$ gilt. Die Spektralfunktionen im Axialvektorkanal
definieren wir durch 
\be
 \frac{1}{\pi} {\rm Im} \Pi_{\mu\nu}^{A}(q^2) =
    \rho_A(q^2) ( q_\mu q_\nu -g_{\mu\nu}q^2) +
    \rho_A^{||}(q^2) q_\mu q_\nu\; ,
\ee
wobei PCAC den Pionbeitrag zum longitudinalen Anteil $\rho_A^{||}(q^2)=
f_\pi^2 \delta(q^2-m_\pi^2)$ fixiert. Im chiralen Limes findet man 
damit folgende Summenregeln vom FESR-Typ
\beq
\label{afesr1}
  t_c F_2(t_c) 
     &=&    8\pi^2 \int_{s_0}^{t_c} \rho_A(s) \,ds  + 8\pi^2f_\pi^2 \; ,\\
\label{afesr2}
  c_4 + t_c^2 F_4(t_c) 
     &=&    16\pi^2 \int_{s_0}^{t_c} \rho_A(s)\, s\, ds \; , \\     
\label{afesr3}
  -\frac{11}{7}\frac{\xi^A}{\xi^V} c_6 - \frac{t_c^3}{2} F_6(t_c) 
     &=&    12\pi \int_{s_0}^{t_c}  \rho_A(s)\, s^2\, ds  \; .
\eeq
Kombiniert man diese Ergebnisse mit den Resultaten aus dem 
Vektorkanal, so erh\"alt man die FESR-Varianten der  
Weinberg-Summenregeln \cite{Wei67}. Ber\"ucksichtigt man 
die Effekte der endlichen Stromquarkmassen, so divergieren diese
Summenregeln allerdings im Limes $t_c\to\infty$ \cite{FNR79}. 
Es ist daher in der Regel sinnvoller, die Boreltransformierte Version der 
Weinberg-Relationen zu studieren \cite{PS87}. 

Auch die Summenregeln im Axialvektorkanal sind mit Hilfe des einfachen 
Ansatzes 
\be
\label{zerowa}
 \rho_A (s) = \frac{m_{a_1}^2}{g_{a_1}^2} \delta (s-m_{a_1}^2)
   + \frac{1}{8\pi^2}\left( 1+\frac{\alpha_s}{\pi}\right) \Theta (s-s_{th})
\ee	      
untersucht worden \cite{RRY85}. Die Kontinuumsschwelle l\"a\ss t sich
in diesem Fall entweder aus den FESR-Bedingungen oder aus der
Forderung nach \"Aquivalenz der beiden Boreltransformierten 
Summenregeln bestimmen. Beide Verfahren liefern \"ubereinstimmend
den Wert $s_{th}=2.3\,{\rm GeV}^2$. Verwendet man dieses Resultat,
um die Parameter der $a_1$-Resonanz zu fixieren, so findet
man $m_{a_1}=1230$ MeV (exp. 1260 MeV) und $\frac{g_{a_1}^2}{4\pi}=5.3$ 
(exp. 6.5) in recht guter \"Ubereinstimmung  mit den von der Particle 
Data Group angegebenen Werten \cite{PDG90}. Allerdings besitzt die 
$a_1$-Resonanz mit etwa 450 MeV eine so gro\ss e Breite, da\ss\ die 
Parametrisierung (\ref{zerowa}) wenig sinnvoll ist. Die geschilderte
Bestimmung der $a_1$-Masse erscheint daher weinger zuverl\"assig, als
das entsprechende Resultat f\"ur das $\rho$-Meson.
 
\cld

\chapter{Analyse der Spektralfunktionen}
% revised Jan. 1, 1992
In diesem Kapitel wollen wir zun\"achst auf die Bestimmung der 
Spektralfunktionen aus experimentellen Wirkungsquerschnitten eingehen.
Das Spektrum der Vektormesonen entnehmen wir der $e^+e^-$-Annihilation
w\"ahrend die spektrale Verteilung der Axialvektormesonen aus 
hadronischen Zerf\"allen des $\tau$-Leptons bestimmt werden kann.
Wir benutzen die gewonnenen Spektralfunktionen, 
um die G\"ultigkeit der im letzten Kapitel angegebenen Summenregeln
zu testen und die enthalten Vakuumparameter zu bestimmen. Im
letzten Abschnitt schlie\ss lich kehren wir die Vorgehensweise
um und verwenden Summenregeln, um Vorhersagen \"uber das 
Axialvektorspektrum jenseits der Masse des $\tau$-Leptons zu 
machen.   

\section{Bestimmung der Spektralfunktion im Vektorkanal}
Die Spektralfunktion im Vektorkanal l\"a\ss t sich mit Hilfe 
des optischen Theorems aus dem totalen Wirkungsquerschnitt
f\"ur die Annihilation von $e^+e^-$-Paaren in Hadronen 
mit dem Isospin $I$=1 bestimmen
\be
\label{opttheo}
\rho_V(s) \equiv \frac{1}{\pi}{\rm Im}\Pi^V(s) = \frac{s}{16\pi^3\alpha^2}
   \,\sigma (e^+e^-\to I=1) .
\ee
Diese Relation l\"a\ss t sich vereinfachen, indem man den
hadronischen Wirkungsquerschnitt auf den rein     
elektromagnetischen Proze\ss\ $e^+e^-\to\mu^+\mu^-$
normiert
\be
\label{rhov}
 \rho_V(s) =  \frac{1}{12\pi^2}\, R^{I=1}(s)\, ,
\ee
wobei $R^{I=1}$ das Verh\"altnis $\sigma (e^+e^-\to I=1)/
\sigma (e^+e^-\to\mu^+\mu^-)$ bezeichnet. Dieser Quotient
ist an einer Reihe von Speicherringen mit gro\ss er 
Genauigkeit vermessen worden.

Zwischen der Schwelle $\sqrt s=2m_\pi$ und $\sqrt s=1\,\gev$ wird
$R^{I=1}(s)$ von der resonanten $\pi^+\pi^-$-Produktion dominiert.
In diesem Bereich l\"a\ss t sich die Spektralfunktion mit Hilfe
der Beziehung 
\be
 R^{I=1}(s) = \frac{1}{4} |F_\pi (s)|^2 \left( 
       1-\frac{4m_\pi^2}{s} \right)^{3/2}
\ee       
aus dem elektromagnetischen Formfaktor des Pions extrahieren. 
Pr\"azise Ergebnisse f\"ur den Formfaktor im zeitartigen Bereich 
$\sqrt s<1$ GeV liegen aus Orsay vor \cite{Que78}. Die gewonnenen 
Daten werden gew\"ohnlich mit Hilfe eines Gounaris-Sakurai Fits
beschrieben \cite{Hoe83}. Dieses Modell basiert auf einer 
Parametrisierung der $\pi\pi$-Endzustandswechsel\-wir\-kung, die
f\"ur die analytische Struktur des Pionformfaktors verantwortlich ist.
F\"ur unsere Zwecke gen\"ugt bereits eine noch einfachere 
Beschreibung der Daten, die in Anhang E skizziert ist.
\begin{figure}
\caption{Verh\"altnis $R^{I=1}$ als Funktion der Schwerpunktsenergie. 
Die durchgezogene Linie stellt die von uns verwendete 
Parametrisierung dar.}
\vspace{9cm}
\end{figure}

Im Bereich zwischen  $\sqrt s=1$ GeV und $\sqrt s=2$ GeV ist 
$R^{I=1}(s)$ durch eine Reihe breiter Resonanzen bestimmt. Bis
$\sqrt s=1.4$ GeV verwenden wir die Novosibirsk-Daten \cite{Sid76}
f\"ur die Kan\"ale $e^+e^-\to \pi^+\pi^-\pi^+\pi^-$ und $e^+e^-\to 
\pi^+\pi^-\pi^0\pi^0$. Jenseits von $\sqrt s=1.4$ GeV benutzen wir 
Ergebnisse aus Frascati \cite{Bac79} f\"ur die Reaktion $e^+e^-\to
(\ge 3\pi )$. In beiden F\"allen addieren wir den Kanal $e^+e^-\to\pi^+\pi^-$
aus den Orsay-Daten bei h\"oheren Energien \cite{Bis89}. Das Resultat 
zeigt eine deutlich erkennbare Struktur bei der Schwerpunktsenergie 
1.6 GeV. In dieser \"Uberh\"ohung lassen sich mit Hilfe pr\"aziser
Daten aus dem Pionformfaktor  die Beitr\"age
zweier \"uberlappender Resonanzen, $\rho(1450)$ und $\rho(1700)$, 
isolieren.  

Bei noch h\"oheren Energien zeigt der totale Wirkungsquerschnitt 
$\sigma (e^+e^-\to \rm hadr.)$ vor allem die Schwellen f\"ur schwere
Quarkproduktion sowie die Beitr\"age der Vektormesonen 
$J/\psi,\psi ',\ldots$. Es wird allerdings zunehmend schwieriger, den 
$I$=1 Anteil des Wirkungsquerschnitts zu  identifizieren. Wir 
benutzen daher f\"ur $\sqrt s>2$ GeV das st\"orungstheoretische 
Resultat 
\be
 R^{I=1}(s)= \frac{3}{2}\left( 1+\frac{\alpha_s}{\pi}
  + {\cal O}(\alpha_s^2) \right) .
\ee
Die verwendeten Daten f\"ur $R^{I=1}(s)$ im Bereich bis $\sqrt s=2$
GeV sowie die beschriebene Parametrisierung finden sich  in Abbildung 6.1.
F\"ur $\sqrt s<1$ GeV liefern diese Resultate eine sehr genaue 
Bestimmung der Spektralfunktion. F\"ur h\"ohere Energien sind die
Daten allerdings mit einer Reihe von Unsicherheiten behaftet. So
\begin{itemize}
\item{beinhalten die Daten im Bereich $\sqrt s>1.4$ GeV auch den 
Kanal $\pi^+\pi^-\pi^0$, der im wesentlichen zum Isospin $I\!=\!0$
zu rechnen ist.}
\item{ist es nicht ohne weiteres m\"oglich, den nichtresonanten 
$I\!=\!0$-Untergrund aus den Kan\"alen $\pi^+\pi^-\pi^+\pi^-$ und 
$\pi^+\pi^-\pi^0\pi^0$ abzutrennen.}
\item{fehlen in unserer Analyse alle Kan\"ale, deren Endzust\"ande
 Kaonen oder Eta-Mesonen enthalten.}
\item{zeigt der totale Wirkungsquerschnitt f\"ur $\sqrt s<4$ GeV
eine Reihe von Strukturen, die auf $I\!=\!1$-Resonanzen zur\"uckzuf\"uhren
sein k\"onnten.}
\end{itemize}
W\"ahrend sich diese Unsicherheiten im Bereich bis $\sqrt s=2$ GeV 
in der Gr\"o\ss enordnung der experimentellen Fehler bewegen, ist 
das Fehlen isospinseparierter Daten in der Region jenseits von
$\sqrt s=2$ GeV durchaus problematisch f\"ur die Analyse der
Summenregeln.

\section{Bestimmung der Axialvektorspektralfunktion}
Die Spektralfunktion im Axialvektorkanal l\"a\ss t sich auf Grund der
Tatsache bestimmen, da\ss\ der Zerfall eines schweren Leptons $\tau$ 
in nichtseltsame Hadronen durch die $V\!-\!A$ Wechselwirkung
\be
\label{fermi}
 {\cal L} = -\frac{G}{\sqrt 2} \cos\theta_c \, 
    \bar u_\nu\gamma^\mu (1-\gamma_5) u_\tau\,
    (V_\mu^{1+i2}-A_\mu^{1+i2})
\ee
vermittelt wird. Dabei bezeichnet $G=1.16637\cdot 10^{-5}\,{\rm GeV}^{-2}$
die Fermikonstante und $\cos\theta_c=0.9744$ den Cosinus des Cabbibowinkels.
Mit der Wechselwirkung (\ref{fermi}) ergibt sich die partielle
Zerfallsbreite des $\tau$-Leptons in Hadronen mit der
invarianten Masse $q^2$ \cite{Tsa71,Oku82}      
\beq
\label{tauwid}
 \frac{d\Gamma (\tau\to^{\;s=0}\!{\rm hadr.}+\nu )}{dq^2} &=&
 \frac{G^2\cos^2\theta_c}{8\pi m_\tau^3} (m_\tau^2-q^2)^2
  \Big\{   m_\tau^2 \rho_A^{||} (q^2)\\
 & & \hspace{2.2cm}\mbox{}  +
 (m_\tau^2+2q^2)(\rho_V(q^2)+\rho_A(q^2) ) \Big\} \, .
   \nonumber
\eeq
Dieses Resultat bestimmt die Spektralfunktionen von der Schwelle
bis hinauf zur Masse des $\tau$-Leptons, $m_\tau=1.784$ GeV. In der 
Praxis ist  allerdings die Statistik auf Grund des kleiner werdenden 
Phasenraums bereits f\"ur invariante Massen $\sqrt{q^2}>1.5$ GeV 
unzureichend.

Um die verschiedenen Spektralfunktionen zu separieren, mu\ss\
man die m\"oglichen Endzust\"ande im $\tau$-Zerfall untersuchen. 
Ber\"ucksichtigt  man nur den Beitrag des Pions $\rho_A^{||}(s)= 
f_\pi^2 \delta (s-m_\pi^2)$ in der longitudinalen
Spektralfunktion, so ergibt sich das \"ubliche Resultat f\"ur 
die Zerfallsbreite $\tau\to\nu\pi$  
\be
 \Gamma (\tau\to\nu\pi ) = \frac{G^2\cos^2\theta_c}{8\pi}
    m_\tau^3f_\pi^2 \left( 1-\frac{m_\pi^2}{m_\tau^2} \right)^2 .
\ee
Die n\"achsth\"ohere Anregung mit den Quantenzahlen $J^\pi=0^+$
ist die $\pi (1300)$-Resonanz, die \"uberwiegend in drei Pionen zerf\"allt.
Dieser Zustand konnte im $\tau$-Zerfall noch nicht eindeutig 
identifiziert worden. Mit Hilfe von QCD-Summenregeln f\"ur den 
pseudoskalaren Korrelator gewinnt man jedoch folgende 
Absch\"atzung der Resonanzparameter \cite{GL82}
\be
  r_\pi = \frac{f_{\pi'}^2m_{\pi'}^4}{f_\pi^2 m_\pi^4}
         \simeq 8 .
\ee
Dieser Wert liefert ein Verzweigungsverh\"altnis $B(\tau\to\nu\pi')
=2\cdot 10^{-5}$, das um vier Gr\"o\ss enordnungen kleiner ist als
das gesamte Verzweigungsverh\"altnis f\"ur Zerf\"alle mit drei 
Pionen im Endzustand, $B(\tau\to\nu 3\pi)=1.1\cdot 10^{-1}$. Wir haben 
daher die longitudinale Spektralfunktion in unserer Analyse der
Zerf\"alle mit mehr als einem Pion im Ausgangskanal nicht ber\"ucksichtigt.


Die Vektor- und Axialvektoranteile der transversalen 
Spektralfunktion lassen sich in sehr guter N\"aherung trennen, indem man 
$\tau$-Zerf\"alle in eine gerade bzw.~ungerade Anzahl von Pionen 
betrachtet. Zerfallskan\"ale wie $\tau^\pm\to\nu\rho^\pm\to
\nu\pi^\pm\pi^+\pi^-$, die durch den Vektorstrom vermittelt werden,
aber Endzust\"ande mit drei Pionen liefern, haben nur 
ein verschwindend kleines Verzweigungsverh\"altnis. 

Messungen von $\tau$-Zerf\"allen sind an den Speicherringen 
DORIS \cite{Alb86} und PEP \cite{Ruc86,Ban87} mit Hilfe
der Reaktion $e^+e^-\to\tau^+\tau^-$ durchgef\"uhrt worden.
Wir verwenden die Daten der Argus-Kollaboration \cite{Alb86}
f\"ur den Zerfallskanal $\tau^\pm\to\nu\pi^\pm\pi^+\pi^-$. 
Es gibt keine experimentellen Informationen \"uber das 
Spektrum im anderen $3\pi$-Kanal $\tau^\pm\to\nu\pi^\pm\pi^0\pi^0$.
Eine Analyse der invarianten Massen im $2\pi$-System f\"ur die
Reaktion $\tau^\pm\to\nu\pi^\pm\pi^+\pi^-$ \cite{Alb86}
zeigt allerdings eine klare Pr\"aferenz f\"ur den resonanten 
Proze\ss\
\be
  \tau^\pm\to \nu a_1^\pm \to \nu\rho^0\pi^\pm \to \nu\pi^+\pi^-\pi^\pm .
\ee
F\"ur diese Zerfallssequenz lassen sich die nicht gemessenen 
partiellen Breiten mit Hilfe der Isospinrelation 
\be
 d\Gamma (\tau^\pm\to\nu\pi^\pm\pi^0\pi^0 ) =
 d\Gamma (\tau^\pm\to\nu\pi^\pm\pi^+\pi^- ) 
\ee
absch\"atzen. Diese Absch\"atzung ist konsistent mit den Resultaten der 
MAC-Kollaboration \cite{Ban87} f\"ur die totalen Verzweigungsverh\"altnisse 
$B(\tau^\pm\to\nu\pi^\pm\pi^+\pi^-)=7.0\pm 1.0$ \% und $B(\tau^\pm\to\nu
\pi^\pm\pi^0\pi^0)=8.7\pm 1.5$ \% .
\begin{figure}
\caption{Spektralfunktion $\rho_A(s)$ im Axialvektorkanal.
Die durchgezogene Linie stellt die von uns verwendete 
Parametrisierung dar.}
\vspace{9cm}
\end{figure}

Auch die Massenverteilungen der $5\pi$-Zerf\"alle des $\tau$-Leptons
sind experimentell nicht bestimmt. Das gemessene Verzweigungsverh\"altnis
$B(\tau^-\to 3\pi^-2\pi^+)=6.4\cdot 10^{-4}$ \cite{Alb88} ist jedoch
au\ss erordentlich klein, und es gibt theoretische Hinweise, da\ss\ diese
Aussage auch auf  andere $5\pi$-Zerf\"alle zutrifft \cite{GR85}.
Wir haben daher diese Kan\"ale in unserer Analyse vollst\"andig 
vernachl\"assigt. Die resultierende Spektralfunktion findet sich in 
Abbildung 6.2. Die gezeigte Parametrisierung besteht aus einer 
Breit-Wigner Funktion f\"ur den resonanten Teil, erg\"anzt durch 
ein Polynom zur Beschreibung des Untergrunds. 

Dieser Fit liefert als Nebenprodukt eine einfache Bestimmung der 
Parameter der $a_1$-Resonanz. Wir finden $m_{a_1}=1169$ MeV und
$\Gamma_{a_1}=552$ MeV, in guter \"Ubereinstimmung mit den 
Ergebnissen anderer $\tau$-Zerfallsexperimente, aber in deutlichem
Widerspruch zu den Standardwerten aus rein hadronischen Reaktionen,
$m_{a_1}=1260\pm 30$ MeV und $\Gamma_{a_1}=316\pm 45$ MeV \cite{PDG90}.
Diese Diskrepanz hat eine Reihe vergleichender Analysen der zur Verf\"ugung 
stehenden  Daten angeregt \cite{Bow86,VIO90}. Diese Untersuchungen
zeigen, da\ss\ man die Bestimmung der $a_1$-Masse im $\tau$-Zerfall
mit den Resultaten hadronischer Experimente in Einklang bringen
kann, wenn man energieabh\"angige Kopplungen f\"ur den Zerfall
$a_1\to 3\pi$ zul\"a\ss t. Dagegen bleibt die sehr gro\ss e Breite
des $a_1$ im $\tau$-Zerfall im Widerspruch zu rein hadronischen
Daten.
   
\section{Konsequenzen der Spektralfunktionen}
Wir haben die bestimmten Spektralfunktionen einer Reihe von 
Konsistenztests unterzogen. Zum einen kann man die Ergebnisse 
aus dem Vektorkanal verwenden, um die entsprechenden 
Verzweigungsverh\"altnisse im $\tau$-Zerfall zu bestimmen. 
Wir finden 
\beq
   B(\tau^\pm\to\nu\rho^\pm) &=& 21.8 \,\%  \, , \\
   B(\tau^\pm\to\nu 2n\pi  ) &=& 29.8 \,\%  \, ,
\eeq
in guter \"Ubereinstimmung mit den experimentellen Daten \cite{PDG90}.
Dar\"uber hinaus haben wir die G\"ultigkeit der Weinberg-Summenregeln
\cite{Wei67,PS87} in einem endlichen Intervall untersucht. In 
Abbildung 6.3 zeigen wir die Funktionen 
\begin{figure}
\caption{Test der Weinberg-Summenregeln in einem endlichen 
Energieintervall. Die Funktionen $\Delta_1(t_c),\Delta_2(t_c)$ 
sind im Text definiert. F\"ur die zweite Weinberg-Summenregel 
zeigen wir zum Vergleich den asymptotischen Wert $f_\pi^2$.}
\vspace{19.6cm}
\end{figure} 
\beq
 \Delta_1(t_c) &=& \int_0^{t_c} \big( \rho_V(s)-\rho_A(s)\big)\, ds \; ,\\
 \Delta_2(t_c) &=& \int_0^{t_c} \big( \rho_V(s)-\rho_A(s)\big)\,s\, ds  
\eeq
f\"ur verschiedene Werte des Cutoffs $t_c$. Im chiralen Limes liefert
die OPE keine Korrekturen zu den Stromalgebravorhersagen $\Delta_1
(t_c) \stackrel{t_c\to\infty}{\longrightarrow} f_\pi^2$ und $\Delta_2 (t_c)
\stackrel{t_c\to\infty}{\longrightarrow} 0$. Dagegen zeigen unsere
Ergebnisse, da\ss\ die St\"arke der Axialvektorspektralfunktion
in dem gemessenen Bereich nicht ausreicht, um die Weinberg-Summenregeln 
zu saturieren. Dies mag ein Hinweis auf die Tatsache sein, da\ss\ das
Spektrum im Axialvektorkanal auch jenseits der Masse des $\tau$-Leptons 
noch wichtige Strukturen zeigt.

Stromalgebratechniken erlauben auch die Bestimmung der elektromagnetischen
Massendifferenz der Pionen, $\Delta m_\pi=m_{\pi^\pm}-m_{\pi^0}$, aus
den Spektralfunktionen im Vektor- und Axialvektorkanal. Das Resultat 
von Das et al.~\cite{DGM67} lautet

\be
\label{dmpi}
 \Delta m_\pi (t_c) = \frac{3\alpha}{8\pi m_\pi f_\pi^2}
    \int_0^{t_c} \big( \rho_A(s) -\rho_V(s) \big) s\ln (s)\,ds\, ,
\ee
wobei das Integral auf Grund der zweiten Weinberg-Summenregel unabh\"angig
von der Skala im Logarithmus ist. Saturiert man die Summenregel mit
$\rho$ und $a_1$-Resonanzen verschwindender Breite, so ergibt sich das
klassische Ergebnis $\Delta m_\pi=\frac{3\alpha m_\rho^2 \ln 2}{4\pi m_\pi}
=5.15$ MeV. In Abbildung 6.4 zeigen wir die Resultate unter Verwendung
realistischer Spektralfunktionen. Die Massendifferenz $\Delta m_\pi (t_c)$
zeigt eine sehr starke Abh\"angigkeit vom Cutoff $t_c$. Bei $\sqrt {t_c} 
= 1.4$ GeV finden wir ein Plateau, bei dem das Ergebnis  $\Delta m_\pi (t_c)=
4.2$ MeV (exp. 4.6 MeV) eine sehr gute \"Ubereinstimmung mit dem 
experimentellen Resultat zeigt. Wie bei den Weinberg-Summenregeln 
dominiert f\"ur gr\"o\ss ere Cutoffs $t_c$ die Vektorspektralfunktion, und    
der Wert von $\Delta m_\pi$ wird deutlich zu klein.  

Eine weitere Gr\"o\ss e, die sensitiv auf die Form der Spektralfunktionen 
ist, ist der axiale Formfaktor im strahlungsbegleiteten Pionzerfall
$\pi^\pm\to e^\pm\nu\gamma$ \cite{BDL82}. Die zugeh\"orige Amplitude 
l\"a\ss t sich in die beiden Anteile $T_{SI}$ und $T_{SD}$ zerlegen. 
Die strukturunabh\"angige Amplitude $T_{SI}$ enth\"alt die Beitr\"age
der Leptonbremsstrahlung, des Pionpolterms sowie des Pionkontaktterms.
Diese Amplituden lassen sich ganz analog zur Photoproduktionsamplitude 
in Bornapproximation berechnen. Das Resultat ist strukturunabh\"angig 
in dem Sinne, da\ss\ die Amplitude als einzigen hadronischen Parameter 
die Pionzerfallskonstante $f_\pi$ enth\"alt. 
\begin{figure}
\caption{Bestimmung der Pionmassendifferenz $\Delta m_\pi$ und des axialen 
Pionformfaktors $F_A(0)$ in einem endlichen Energieintervall.}
\vspace{20cm}
\end{figure}

Die strukturabh\"angige Amplitude $T_{SD}$ l\"a\ss t sich auf Grund von
Eich- und Lorentzinvarianz mit Hilfe des Vektorformfaktors $F_V(s)$
und eines Axialvektorformfaktors $F_A(s)$ parametrisieren. Dabei bezeichnet
$s=k\cdot q$ das Produkt der Impulse des Pions und des abgestrahlten 
Photons. Der Vektorformfaktor ist durch die Zerfallsbreite $\pi^0\to 2\gamma$
bestimmt. Stromalgebra liefert eine Vorhersage f\"ur den axialen 
Formfaktor am weichen Punkt \cite{DMO67}
\be
\label{axpiff}
 F_A(0) = -\frac{m_\pi}{f_\pi} \int \frac{ds}{s}\, \left(
  \rho_A(s) -\rho_V(s) \right)  \; +\; \frac{1}{3}f_\pi m_\pi
   \langle r^2_\pi \rangle ,
\ee
wobei $\langle r^2_\pi\rangle=0.46\pm 0.01\,{\rm fm}^2$ \cite{Que78} den 
elektromagnetischen Radius des Pions bezeichnet. Experimentell lassen sich 
die beiden Formfaktoren bestimmen, indem man das Spektrum der 
Zerfallsprodukte f\"ur Photonenergien $E_\gamma>45$ MeV mi\ss t. 
In diesem Bereich dominiert der strukturabh\"angige Prozess
gegen\"uber der Bremsstrahlung, und man findet $F_A(0)=0.83\pm 0.14
\cdot 10^{-2}$ \cite{PDG90}.

Saturiert man die Summenregel (\ref{axpiff}) mit den niedrigsten
Resonanzen, so ergibt sich $F_A(0)=-\frac{3f_\pi m_\pi}{2m_\rho^2}
+\frac{1}{3}f_\pi m_\pi\langle r_\pi^2\rangle =1.8\cdot 10^{-2}$, w\"ahrend 
die realistischen Spektralfunktionen kleinere Formfaktoren liefern. 
Auf Grund der bei kleinen invarianten Massen konzentrierten 
Gewichtsfunktion zeigt sich eine im Vergleich zu $\Delta m_\pi$
deutlich geringere Abh\"angigkeit von $t_c$. Wie oben finden wir ein 
Plateau bei $\sqrt{t_c}=1.4$ GeV, wo $F_A(0)=0.56\cdot 10^{-2}$ 
gut mit dem experimentellen Wert \"ubereinstimmt.  

Zusammenfassend stellen wir fest, da\ss\ die bestimmten Spektralfunktionen
f\"ur kleine und mittlere invariante Massen bis etwa $\mu^2=2.5\,
\gev^2$ konsistent mit Stromalgebrasummenregeln sind. Dagegen scheint
die St\"arke der Axialvektorspektralfunktion f\"ur h\"ohere invariante
Massen nicht auszureichen, um stabile Resultate zu erzielen. 

\section{Borelquotient im Vektorkanal}
Wir kommen nun zur Bestimmung der Vakuumparameter mit Hilfe einer
Analyse der QCD-Summenregeln im Vektorkanal. Zu diesem Zweck
betrachten wir den Quotienten der Borelmomente der Spektralfunktion
\cite{LNT84}
\be
\label{bratio}
 R^V(\tau) \equiv \frac{\displaystyle \frac{1}{\pi}\int_{s_0}^{\infty}
    {\rm Im}\Pi^V(s) e^{-s\tau} s\, ds }
  {\displaystyle \frac{1}{\pi}\int_{s_0}^{\infty}
    {\rm Im}\Pi^V(s) e^{-s\tau}\, ds }\; .
\ee
Die Operatorproduktentwicklung liefert f\"ur diese Gr\"o\ss e
die theoretische Vorhersage
\be
\label{operatio}
 R^V(\tau) = \frac{1}{\tau}\, \big\{ 1+c_4\tau^2+c_6\tau^3+
    c_8\tau^4 + \ldots \big\} \; ,
\ee
wobei wir die Koeffizienten $c_i$ in (\ref{defcor}) definiert haben.
Der Quotient (\ref{bratio}) hat gegen\"uber den individuellen 
Momenten der Boreltransformierten Summenregel den Vorzug, da\ss\
in der OPE keine perturbativen Korrekturen zum Einheitsoperator
auftreten. Diese Korrekturen liefern in den einzelnen 
Momenten Beitr\"age von derselben Gr\"o\ss enordnung wie die
f\"uhrenden Kondensate, k\"urzen sich in dem Verh\"altnis 
(\ref{bratio}) aber heraus. Das theoretische Resultat (\ref{operatio})
besitzt daher eine nur sehr geringe Sensitivit\"at auf den 
Wert des Skalenparameters und die Gr\"o\ss e von perturbativen
Korrekturen h\"oherer Ordnung.
\begin{figure}
\caption{OPE-Parametrisierung des experimentellen Borelquotienten
nach Subtraktion des perturbativen Beitrags $1/\tau$. Die gestrichelte
Kurve zeigt das Resultat f\"ur das einfache Modellspektrum (5.28).}
\vspace{9cm}
\end{figure}

Mit Hilfe der im letzten Abschnitt bestimmten Spektralfunktionen 
l\"a\ss t sich der Borelquotient f\"ur beliebige Werte des 
Parameters $\tau$ bestimmen. Betrachtet man die Funktion
\be
\label{linratio}
 Y^V(\tau) \equiv R^V(\tau)-\frac{1}{\tau}
           = c_4 \tau + c_6\tau^2 + c_8 \tau^3 + \ldots\; ,
\ee
so reduziert sich die Bestimmung der Koeffizienten $c_i$
auf eine einfache Polynomregression. Um die Unsicherheiten
in der Bestimmung der Spektralfunktion zu ber\"ucksichtigen, haben 
wir das Borelverh\"altnis f\"ur verschiedene Parametrisierungen der 
Daten im Bereich der experimentellen Fehler berechnet. 
Die resultierende Variation der Funktion $Y^V(\tau)$ ist 
in Abbildung 6.5 dargestellt. Die Ergebnisse zeigen, da\ss\ der
relative Fehler nach Abzug des st\"orungstheoretischen 
Anteils $1/\tau$ recht erheblich ist. Man erkennt allerdings
auch, da\ss\ sich systematische Fehler in der Spektralfunktion
f\"ur Werte des Borelparameters $\tau\simeq 0.85\,{\rm GeV}^{-2}$
praktisch nicht auf das Borelverh\"altnis $R^V(\tau)$
auswirken. Diese Tatsache ist ein weiterer Vorzug des
Quotienten $R^V(\tau)$ gegen\"uber den individuellen
Momenten der Boreltransformierten Summenregel. 

Wir wollen zun\"achst eine allgemeine Diskussion der Funktion 
$Y^V(\tau)$ vornehmen. Zu diesem Zweck vergleichen wir in 
Abbildung 6.5 das experimentelle Resultat f\"ur $Y^V(\tau)$ mit
dem entsprechenden Resultat f\"ur die einfache Parametrisierung 
(\ref{zerow}) des  Spektrums als Summe einer Delta- und einer
Thetafunktion. Das Modellspektrum liefert zwar ein qualitativ
\"ahnliches Verhalten wie die korrekten Daten, liegt aber
nicht im Bereich der experimentellen Streuung. 

Beide Kurven haben die Eigenschaft, da\ss\ $Y^V(\tau)$ f\"ur 
$\tau\to 0$ gegen Null strebt. Entwickelt man den Borelquotienten 
in eine Potenzreihe in $\tau$, so zeigt sich, da\ss\ diese Tatsache
\"aquivalent mit der ersten FESR-Bedingung 
\be
\label{lfesr}
 t_cF_2(t_c) = 8\pi \int_{s_0}^{t_c} {\rm Im}\Pi^V(s) \, ds
\ee
ist. W\"ahrend wir diese Bedingung f\"ur das Modellspektrum
durch die Wahl der Kontinuumsschwelle $s_{th}$ erzwungen haben,
liefert die Forderung $Y^V(\tau)\stackrel{\tau\to 0}{\longrightarrow}0$
im Fall der experimentellen Spektralfunktion   einen wichtigen Test
f\"ur die Konsistenz der Daten mit der Dualit\"atsforderung. 
F\"ur $\sqrt{t_c}=2\,\gev$ betr\"agt die Diskrepanz zwischen der 
linken und rechten Seite der Summenregel (\ref{lfesr}) 
$\Delta = 0.43\,\gev^2$. Das bedeutet, da\ss\ f\"ur die verwendete 
Spektralfunktion die Dualit\"atsforderung  bis auf Abweichungen von 
der Gr\"o\ss enordnung 10\% erf\"ullt ist. 
\begin{figure}
\caption{Chi-Quadrat-Contour f\"ur die Bestimmung der Koeffizienten 
$c_4$ und $c_6$ aus dem Borelverh\"altnis im Vektorkanal. Das Minimum
liegt bei $\surd (\chi^2/NDF)=0.24$ und eine Contourlinie entspricht
$\Delta\surd (\chi^2/NDF)=18.5$.}
\vspace{9cm}
\end{figure}

F\"ur gr\"o\ss ere Borelparameter $\tau$ ist die Funktion $Y^V(\tau)$ 
zun\"achst negativ, hat aber eine positive Kr\"ummung und \"andert bei 
$\tau\simeq 0.6$ ihr Vorzeichen. Dieses Verhalten entspricht den aus
der theoretischen Absch\"atzung $c_4=-0.07\,\gev^4$ und $c_6=0.04\,\gev^6$ 
erwarteten Vorzeichen der f\"uhrenden Kondensate.

F\"ur die quantitative Analyse des Borelquotienten ist es von 
gro\ss er  Bedeutung, das Intervall $[\tau_{min},\tau_{max}]$
zu bestimmen, in dem man \"Ubereinstimmung zwischen der experimentellen 
und theoretischen Seite der  Summenregel fordert. Asymptotische
Freiheit garantiert die G\"ultigkeit der Summenregel im Grenzfall 
$\tau\to 0$. Da die Daten in diesem Bereich auf Grund der gro\ss en 
Fehler in jedem Fall nicht sehr stark gewichtet werden, setzen wir
die untere Grenze des Intervalls $\tau_{min}=0$. Die obere Grenze
ergibt sich aus der Forderung, da\ss\ sowohl die statistischen
Fehler auf Grund der Unsicherheit der Daten als auch die theoretischen 
Fehler, hervorgerufen durch das Abbrechen der OPE, minimal sind.  

Um das g\"unstige Verhalten des Borelquotienten bez\"uglich 
systematischer Fehler in der Spektralfunktion zu nutzen, sollte
$\tau_{max}>0.8\,{\rm GeV}^{-2}$ gew\"ahlt werden. Eine 
Absch\"atzung des theoretischen Fehlers ergibt sich, indem man
die Koeffizienten $c_4$ und $c_6$ f\"ur verschiedene vorgegebene
Werte des vernachl\"assigten Korrekturglieds $c_8$ bestimmt. Zu 
diesem Zweck haben wir eine obere Grenze f\"ur $|c_8|$ aus der 
Absch\"atzung \cite{CM90}
\be
  |c_8|_{max}  = {\rm max} \left\{ |c_4|m_{sc}^4,
    |c_6|m_{sc}^2 \right\}
\ee
gewonnen. Dabei bezeichnet $m_{sc}\simeq m_\rho$ eine f\"ur das
Zusammenbrechen der OPE charakteristische  Skala. Verwendet
man die von SVZ bevorzugten Werte (\ref{svzval}), so ergibt sich
$|c_8|_{max}\simeq 0.045\,{\rm GeV}^8$.    
\begin{table}
\caption{Bestimmung der Koeffizienten $c_i$ aus dem Borelquotienten
im Vektorkanal. Angegeben sind die Koeffizienten $c_4$ und $c_6$
sowie Absch\"atzungen der experimentellen und theoretischen
Unsicherheiten f\"ur verschiedene Werte des maximalen Borelparameters.}  
\begin{center}
\begin{tabular}{|c||c|c|c||c|c|c||}\hline
 $\tau_m\,[\gev^{-2}]$  &  $c_4\,[\gev^4]$  & $\Delta^{exp} c_4$ &
       $\Delta^{th} c_4$ &  $c_6\,[\gev^6]$  
	     & $\Delta^{exp} c_6$ & $\Delta^{th} c_6$  \\ \hline\hline
    0.8   &$-0.264$ & $\pm 0.150$        & $\pm 0.020$       &
             0.367  & $\pm 0.201$        & $\pm 0.061$   \\
    0.9   &$-0.099$ & $\pm 0.046$        & $\pm 0.030$       &
             0.148  & $\pm 0.053$        & $\pm 0.074$   \\	     
    1.0   &$-0.073$ & $\pm 0.032$        & $\pm 0.034$       &
             0.119  & $\pm 0.036$        & $\pm 0.079$   \\	     
    1.1   &$-0.054$ & $\pm 0.022$        & $\pm 0.039$       &
             0.096  & $\pm 0.025$        & $\pm 0.083$   \\	     
    1.2   &$-0.042$ & $\pm 0.016$        & $\pm 0.041$       &
             0.083  & $\pm 0.019$        & $\pm 0.086$   \\ \hline
\end{tabular}
\end{center}
\end{table}

In Tabelle 6.1 finden sich die Ergebnisse f\"ur die
Koeffizienten $c_4$ und $c_6$ sowie unsere Absch\"atzung der
theoretischen und experimentellen Fehler f\"ur verschiedene
Werte des maximalen Borelparameters $\tau_{max}$. Die Resultate
zeigen keine Stabilit\"at bez\"uglich der Wahl dieses Parameters,
so da\ss\ eine eindeutige Bestimmung der Kondensate nicht m\"oglich
ist. Verwendet man die oben angegebene Absch\"atzung von $|c_8|$,
so beginnt f\"ur
$\tau_{max}>1\,{\rm GeV}^{-2}$ der theoretische Fehler
in der Bestimmung der Koeffizienten gegen\"uber dem experimentellen
zu dominieren. Wir betrachten  diesen Wert daher als optimalen
Kompromi\ss\ zwischen den Forderungen nach ausreichender 
Bestimmtheit des Fits und Kontrolle \"uber den Abbruchfehler
in der OPE. Das entsprechende Resultat
\beq
\label{fitresult}
  c_4 &=&-0.073\pm 0.032\,{\rm GeV}^{4}, \\   
  c_6 &=& \spm 0.119\pm 0.036\,{\rm GeV}^{6}
\eeq
ist konsistent mit der Absch\"atzung von SVZ, beinhaltet allerdings
gro\ss e experimentelle Fehler. In Abbildung 6.6 zeigen wir die
$\chi^2$-Contour f\"ur diesen Fit. Man erkennt eine sehr starke 
Korrelation der beiden Koeffizienten $c_4$ und $c_6$. Das 
Verh\"altnis dieser beiden Gr\"o\ss en ist daher sehr viel besser
bestimmt als ihre absoluten Werte.   Diesen Effekt beobachtet
man auch bei der Variation der Ergebnisse mit dem maximalen 
Borelparameter, siehe Tabelle 6.1. Obwohl $c_4$ und $c_6$ eine
sehr starke Abh\"angigkeit von $\tau_{max}$ aufweisen, ist ihr 
Verh\"altnis praktisch konstant. 

Damit besteht die M\"oglichkeit einer genaueren Bestimmung 
von $c_6$, indem man zus\"atzliche Informationen \"uber 
die Gr\"o\ss e $c_6$ heranzieht. Verwendet man die Analyse
des Gluonkondensats im Charmoniumsystem, so ergibt sich der
im letzten Kapitel angegebene Wert $c_4=-0.07\,\gev^4$. Wir
finden in diesem Fall $c_6=0.11\,\gev^6$, in \"Ubereinstimmung
mit dem Ergebnis des oben angewandten Kriteriums. 

%Alternativ zur Analyse der Spektralfunktion mit Hilfe des
%Borelquotienten lassen sich die Koeffizienten $c_4$ und $c_6$ 
%auch direkt aus den FESR-Gleichungen (\ref{fesr2},\ref{fesr3}) bestimmen
%\cite{BDL88}. Betrachtet man zun\"achst nur die niedrigste
%Bedingung
%\be
%\label{lfesr}
% t_cF_2(t_c) = 8\pi \int_{s_0}^{t_c} {\rm Im}\Pi^V(s) \, ds
%\ee
%f\"ur $\sqrt{t_c}=2$ GeV, so ergibt sich mit unseren Daten eine
%Diskrepanz von $0.43\,{\rm GeV}^2$ zwischen der linken und 
%rechten Seite der Summenregel. Das bedeutet, da\ss\ bei der 
%verwendeten Spektralfunktion die Dualit\"atsforderung im Bereich
%von ca. 10\% erf\"ullt ist.
%
%Die h\"oheren FESR-Gleichungen sind au\ss erordentlich sensitiv 
%auf die G\"ultigkeit dieser Bedingung. Verwendet man (\ref{fesr2}) 
%f\"ur $\sqrt{t_c}=2$ GeV, so findet man $c_4= 1.76\,\gev^4$ und 
%$c_6= -4.01\,\gev^6$, in deutlichem Widerspruch zu (\ref{fitresult}). 
%Fixiert man dagegen $t_c$ durch die Dualit\"atsforderung (\ref{lfesr}), 
%so ergibt sich mit $\sqrt{t_c}=1.46\,{\rm GeV}$ eine Schwellenenergie 
%in der N\"ahe der $\rho'$-Resonanz. Dieser Wert von $\sqrt{t_c}$ 
%liefert die Kondensate 
%\be
%c_4=-0.33\,\gev^4 \hspace{1cm} c_6=0.68\,\gev^6 ,
%\ee
%in besserer \"Ubereinstimmung mit dem Resultat
%der Analyse des Borelquotieten. Insbesondere entspricht auch dieses
%Ergebnis unserer Beobachtung, da\ss\ die verwendeten Summenregeln
%im Wesentlichen das Verh\"altnis $c_4/c_6$ fixieren.

\section{Borelquotient im Axialvektorkanal}
Nachdem wir im letzten Abschnitt eine Analyse der experimentellen
Spektralfunktion im Vektorkanal vorgestellt haben, wollen wir uns 
nun unserem eigentlichen Thema zuwenden und untersuchen, inwieweit
auch die Daten im Axialvektorkanal mit QCD-Summenregeln vertr\"aglich
sind. Insbesondere soll die Frage untersucht werden, ob die Ergebnisse
konsistent mit der Faktorisierungshypothese $\xi^V=\xi^A$ sind.

Zu diesem Zweck betrachten wir den Quotienten der Borelmomente der 
$q_\mu q_\nu$-Struktur in der Korrelationsfunktion
\be
\label{ra2}
 R^{A_2}(\tau) \equiv \frac{\displaystyle \int_{s_0}^{\infty}
    \rho_A (s) e^{-s\tau} s\, ds }
  {\displaystyle \int_{s_0}^{\infty}
    \rho_A (s) e^{-s\tau}\, ds \,+\, f_\pi^2}\; ,
\ee
wobei wir die endliche Masse des Pions vernachl\"assigt haben. Diese
N\"aherung ist konsistent mit unserer Vorgehensweise im perturbativen
Sektor, wo wir die Effekte der Strommassen nicht ber\"ucksichtigt 
haben. Die Operatorproduktentwicklung liefert die Vorhersage 
\be
 Y^{A_2}(\tau) \equiv R^{A_2}(\tau)-\frac{1}{\tau} =
     c_4\tau - \frac{11}{7}\frac{\xi^A}{\xi^V} c_6\tau^2 + \ldots\, ,
\ee
so da\ss\ sich die Koeffizienten $c_4,c_6$ unter der Annahme $\xi^A=\xi^V$ 
ganz analog zur Analyse im Vektorkanal bestimmen lassen.
\begin{figure}
\caption{OPE-Parametrisierung des experimentellen Borelquotienten
f\"ur die $q_\mu q_\nu$-Struktur im Axialvektorkanal 
nach Subtraktion des perturbativen Beitrags $1/\tau$.}
\vspace{9cm}
\end{figure}
       
Das Pion liefert einen sehr genau bestimmten Beitrag zum ersten Moment
der Spektralfunktion, w\"ahrend der restliche Teil des Spektrums mit
erheblichen Fehlern behaftet ist. \"Ahnlich wie im Vektorkanal, wo
wir einen analogen Effekt auf Grund des Zusammenwirkens der $\rho$-
und $\rho'$-Resonanz beobachtet haben, existiert daher ein Bereich 
von Werten des Parameters $\tau$, in dem der Borelquotient $R^{A_2}
(\tau)$ kaum von den experimentellen Unsicherheiten beeinflu\ss t ist
(siehe Abb.~6.7). 

Wie im letzten Abschnitt bestimmen wir die Koeffizienten durch Anpassung an 
die Daten im Intervall $[0,\tau_{max}]$ und studieren die Abh\"angigkeit
der Ergebnisse von $\tau_{max}$. Die Werte von $c_4$ und $c_6$, die
sich unter Verwendung der Faktorisierungshypothese ergeben, finden sich 
in Tabelle 6.2. Erneut finden wir keine Stabilit\"atsregion f\"ur
den maximalen Borelparameter. Fixiert man $\tau_{max}$ mit Hilfe 
des theoretischen Werts f\"ur $c_4$, so ergibt sich $c_6=0.39\,
{\rm GeV}^6$. Dieses Resultat entspricht einer Verletzung der 
exakten Faktorisierung $\xi^A/\xi^V=1$ um den Betrag  $3.2\pm 1.9$, 
wobei der Fehler so gro\ss\ ist, da\ss\ auch $\xi^A/\xi^V=1$ nicht 
ausgeschlossen ist.      
\begin{table}
\caption{Bestimmung der Koeffizienten $c_i$ aus dem Borelquotienten
f\"ur die $q_\mu q_\nu$-Struktur im Axialvektorkanal. Angegeben sind 
die Koeffizienten $c_4$ und $c_6$ sowie die Absch\"atzung der 
experimentellen Unsicherheiten f\"ur verschiedene Werte des maximalen 
Borelparameters.}  
\begin{center}
\begin{tabular}{|c||c|c||c|c||}\hline
 $\tau_m\,[\gev^{-2}]$ &  $c_4\,[\gev^4]$  & $\Delta^{exp} c_4$ &
             $c_6\,[\gev^6]$  & $\Delta^{exp} c_6$   \\ \hline\hline
    0.80  &$ \spm 0.042$ & $\pm 0.077$        &
             0.559  & $\pm 0.169$          \\
    0.85  &$-0.037$ & $\pm 0.063$        & 
             0.447  & $\pm 0.137$          \\	     
    0.90  &$-0.101$ & $\pm 0.052$        &
             0.356  & $\pm 0.112$          \\	     
    0.95  &$-0.154$ & $\pm 0.044$        &
             0.283  & $\pm 0.093$          \\	     
    1.00  &$-0.198$ & $\pm 0.037$        & 
             0.221  & $\pm 0.076$          \\ \hline
\end{tabular}
\end{center}
\end{table}

Um die Zuverl\"assigkeit dieser Resultate einsch\"atzen zu k\"onnen,
mu\ss\ man allerdings beachten, da\ss\ unter Verwendung der beschriebenen
Spektralfunktion die niedrigste FESR-Bedingung
\be
\label{lfesra}
 t_c F_2(t_c) = 8\pi^2 \int_{s_0}^{t_c} \rho_A (s)\, ds \,
 +\, 8\pi^2f_\pi^2
\ee
deutlich verletzt ist. F\"ur $\sqrt{t_c}=m_\tau$ betr\"agt die
Diskrepanz zwischen der linken und rechten Seite der Summenregel
$\Delta =0.98\,{\rm GeV}^2$, ein Wert, der erheblich gr\"o\ss er ist als der
entsprechende Fehlbetrag im Vektorkanal. Tats\"achlich l\"a\ss t
sich die Bedingung erst f\"ur $\sqrt{t_c}=1.3 \,{\rm GeV}$ 
zumindest n\"aherungsweise erf\"ullen.
Dieser Wert ist aber so klein, da\ss\ praktisch der gesamte Resonanzbeitrag 
aus dem Spektrum herausgeschnitten wird.

Die Tatsache, da\ss\ die $a_1$-Resonanz allein die Dualit\"atsbedingung
nicht erf\"ullt, ist ein Hinweis darauf, da\ss\ die Verwendung des
perturbative Spektrums f\"ur $\sqrt s>m_\tau$ ein unrealistisches 
Modell der Spektralfunktion bei mittleren Energien liefert. Diese
Schlu\ss folgerung wird durch eine Analyse des Quotienten der 
Borelmomente der $g_{\mu\nu}$-Struktur der Korrelationsfunktion
\be
\label{ra1}
 R^{A_1}(\tau) \equiv \frac{\displaystyle \int_{s_0}^{\infty}
    \rho_A (s) e^{-s\tau} s^2\, ds }
  {\displaystyle \int_{s_0}^{\infty}
    \rho_A (s) e^{-s\tau}s\, ds }
\ee
untermauert. Dieses Verh\"altnis enth\"alt h\"ohere Momente der
Spektralfunktion und ist daher in st\"arkerem Umfang sensitiv 
auf die Form des Spektrums bei Energien jenseits der $a_1$-Masse.
Die Operatorproduktentwicklung liefert die theoretische
Vorhersage 
\be
\label{ya1}
 Y^{A_1}(\tau) \equiv -R^{A_1}(\tau)+\frac{2}{\tau} = \overline{c}_4
     \tau - \frac{22}{7}\frac{\xi^A}{\xi^V} c_6\tau^2 + \ldots\, ,
\ee
wobei
\be
 \overline{c}_4=c_4 + 16\pi^2 (m_u+m_d)<\bar qq>=c_4-0.026
\,{\rm GeV}^4
\ee
nur eine kleine Korrektur gegen\"uber dem entsprechenden
Koeffizienten im Vektorkanal enth\"alt.  
\begin{figure}
\caption{OPE-Parametrisierung des experimentellen Borelquotienten
f\"ur die $g_{\mu\nu}$-Struktur im Axialvektorkanal 
nach Subtraktion des perturbativen Beitrags $2/\tau$.}
\vspace{9cm}
\end{figure}

Die Resultate f\"ur $Y^{A_1}(\tau)$, die sich unter Verwendung des
Spektrums aus dem $\tau$-Zerfall ergeben, finden sich in Abbildung
6.8. Auf Grund der ung\"unstigeren Gewichtung des Spektrums wird
der systematische Fehler erst bei relativ gro\ss en Werten des
Borelparameters, $\tau\simeq 1.2\,{\rm GeV}^{-2}$, minimal. Wir finden
allerdings ganz unabh\"angig von dem verwendeten Intervall 
$[\tau_{min}=0,\tau_{max}]$ keine \"Ubereinstimmung der Daten mit 
der QCD-Parametrisierung (\ref{ya1}). Verwendet man $\tau_{max}=1.5
\,{\rm GeV}^{-2}$, um den Einflu\ss\ der experimentellen Fehler gering zu
halten, so ergibt sich $c_4=-0.58\pm 0.08\,{\rm GeV}^4$ und
$c_6 = -0.08\pm 0.06\,{\rm GeV}^6$, in klarem Widerspruch  zur 
Analyse der niedrigeren Borelmomente.    

\section{Hinweise auf radiale Anregungen der $a_1$-Re\-so\-nanz}
Wir haben im letzten Abschnitt das Spektrum im Axialvektorkanal mit 
Hilfe von QCD-Summenregeln untersucht. Unter Verwendung von Summenregeln,
die im wesentlichen auf den Pionbeitrag und den niederenergetischen 
Teil des Spektrums sensitiv sind, haben wir \"Ubereinstimmung mit den 
Ergebnissen  aus dem Vektorkanal und der Faktorisierungshypothese 
erzielt. Dagegen zeigen h\"ohere Borelmomente deutliche Abweichungen 
von diesen Vorhersagen. Dar\"uber hinaus ist unser Modell des Spektrums,
in dem jenseits der Masse des $\tau$-Leptons der st\"orungstheoretische
Wert verwendet wird, nicht konsistent mit der asymptotischen 
G\"ultigkeit der Summenregeln. 

Es liegt daher nahe, auch die Beitr\"age h\"oherer Anregungen des 
$a_1$-Mesons zu ber\"ucksichtigen. Die vorhandenen theoretischen 
und experimentellen Informationen \"uber solche Zust\"ande sind 
allerdings sehr fragmentarisch. So enth\"alt die Kompilation der
Particle Data Group \cite{PDG90} neben dem $a_1$ keine weiteren 
Mesonen mit den Quantenzahlen $I^G(J^{PC})=1^-(1^{++})$.

In nichtrelativistischen Konstituendenmodellen des mesonischen 
Spektrums ergeben sich dagegen in nat\"urlicher Weise radiale
Anregungen des $a_1$-Mesons. Ein typisches Beispiel ist das
Potentialmodell von Metsch et al.~\cite{Met90}. Diese Autoren 
finden den Grundzustand im $1^{++}$-Kanal bei einer Energie
von 1220 MeV und die erste radiale Anregung bei $m_{a_1'}=
1923$ MeV. Auch die bereits angesprochene gro\ss e Breite der
$a_1$-Resonanz ist als Hinweis auf die Gegenwart h\"oherer Anregungen
gedeutet worden. Beschreibt man das Spektrum aus dem $\tau$-Zerfall
mit Hilfe zweier interferierender Resonanzen \cite{IKM89},
so reduziert sich die resultierende $a_1$-Breite auf 
$\Gamma_{a_1}=380\pm 20$ MeV, in \"Ubereinstimmung mit den
Resultaten hadronischer Experimente.  Die entsprechenden 
Parameter der $a_1'$-Resonanz lauten $m_{a_1'}=1550\pm 40$ MeV
und $\Gamma_{a_1'}=525\pm 25$ MeV.

Die Frage, ob die im $\tau$-Zerfall beobachtete Struktur in 
Wahrheit durch mehrere \"uberlappende Resonanzen hervorgerufen
wird, l\"a\ss t sich  nicht mit Hilfe von Summenregeln    
kl\"aren. Wir wollen uns daher im folgenden auf die Gestalt des 
Spektrums jenseits der Masse des $\tau$-Leptons konzentrieren. 
Zu diesem Zweck parametrisieren wir die Spektralfunktion 
in der Form
\beq
\label{rhoap}
  \rho_A(s) &=& \rho_a^\tau (s) +\frac{1}{4\pi} \Big( 
   \frac{a'm_{a_1'}^2\Gamma_{a_1'}^2}{(s-m_{a_1'}^2)^2
   +m_{a_1'}^2\Gamma_{a_1'}^2} + b\sqrt{s-s_0} \Big)
   \Theta (s_{th}-s)   \\[0.1cm]
   & &  \hspace{1.3cm} \mbox{}+\frac{1}{8\pi^2} 
   \Big( 1+\frac{\alpha_s}{\pi} \Big) \Theta (s-s_{th}), \nonumber
\eeq
wobei $\rho_a^\tau (s)$ die Spektralfunktion aus dem $\tau$-Zerfall 
bezeichnet. Die Gr\"o\ss en $a',m_{a_1'}$ und $\Gamma_{a_1'}$ 
charakterisieren die $a_1'$-Resonanz, w\"ahrend der Term 
$b\sqrt{s-s_0}$ einen langsam wachsenden Untergrund beschreibt. 
Dieser Untergrund geht f\"ur Energien oberhalb der 
Kontinuumsschwelle $s_{th}$ in das st\"orungstheoretische Resultat 
\"uber.

\begin{figure}
\caption{Vergleich der Spektralfunktionen im Vektor- und 
Axialvektorkanal. Die Axialvektorspektralfunktion enth\"alt 
die im Text bestimmten Beitr\"age der $a_1'$-Resonanz.}
\vspace{9cm}
\end{figure} 
Wir bestimmen die freien Parameter, indem wir in einem gegebenen 
Intervall $[\tau_{min},\tau_{max}]$ die \"Ubereinstimmung der 
beiden Borelquotienten $R^{A_1}(\tau)$ und $R^{A_2}(\tau)$ mit 
den Vorhersagen der OPE optimieren. Zu diesem Zweck minimieren 
wir die Funktion
\be
\chi^2 =\sum_i \left| \frac{ Y^{A_1}(\tau_i)-Y^{A_1}_{th}(\tau_i) }
                    { \Delta Y^{A_1}(\tau_i) } \right|^2 +
               \left| \frac{ Y^{A_2}(\tau_i)-Y^{A_2}_{th}(\tau_i) }
	            { \Delta Y^{A_2}(\tau_i) } \right|^2 
\ee
in dem gegebenen Parameterraum. In der Praxis haben wir dabei die 
Werte von $s_0,s_{th}$ und $\Gamma_{a_1'}$ fixiert, da diese 
Gr\"o\ss en durch die verwendeten Summenregelen nicht sehr stark 
eingeschr\"ankt sind. Mit $s_0=2.5\,\gev^2$, $s_{th}=4.65\,\gev^2$
und $\Gamma_{a_1'}=0.25\,\gev$ finden wir
\be
\label{ma1p}
m_{a_1'} = 1729^{+11}_{-67}\,{\rm MeV}
\ee
sowie $a'=0.274$ und $b=5.84\cdot 10^{-2}\,\gev^{-1}$. Die angegebenen
Fehler sind ein Ma\ss\ f\"ur die Variation der Ergebnisse bei 
verschiedenen Werten von $s_0,s_{th}$ und $\Gamma_{a_1'}$. 

Die resultierende Spektralfunktion ist in Abbildung 6.9 dargestellt. 
Die Position der $a_1'$-Resonanz liegt noch unterhalb der Masse des 
$\tau$-Leptons, allerdings in einem Bereich, wo die Daten aus dem 
$\tau$-Zerfall mit sehr gro\ss en Fehlern behaftet sind (siehe Abb. 6.2).
Die modifizierte Spektralfunktion liefert ein Verzweigungsverh\"altnis
\be
 B(\tau\to\nu (3\pi)) = 13.4\%,
\ee
das etwas oberhalb der von der Argus-Kollaboration angebenen experimentellen
Unsicherheit, $B(\tau\to\nu (3\pi)) = 11.2\pm 1.4\%$, aber noch unterhalb des 
Resultats des MAC-Experiments, $B(\tau\to\nu (3\pi))=15.7\pm 2.5\%$, liegt. 

Auch die Dualit\"atsforderung l\"a\ss t sich unter Einbeziehung der 
$a_1'$-Resonanz erf\"ullen. Untersucht man die G\"ultigkeit der 
FESR-Bedingung   
\be
 t_c F_2(t_c) = 8\pi^2 \int_{s_0}^{t_c} \rho_A (s)\, ds \,
 +\, 8\pi^2f_\pi^2
\ee
f\"ur $\sqrt{t_c}=2.15$ GeV, so ergibt sich eine Diskrepanz $<1\%$
zwischen den beiden Seiten der Summenregel. Wir haben dar\"uber hinaus
die Auswirkung  der $a_1'$-Resonanz auf die in Abschnitt 6.2 
diskutierten Gr\"o\ss en $\Delta m_\pi$ und $F_A(0)$ untersucht. 
Wir finden
\beq
    \Delta m_\pi &=& 4.1^{+0.3}_{-0.2} \,\mev , \\
    F_A(0)       &=& 0.6^{+0.2}_{-0.1}\cdot 10^{-2}
\eeq
in \"Ubereinstimmung mit den Werten aus 6.3, aber mit einer deutlich 
kleineren Abh\"angigkeit von dem verwendeten Cutoff.
\cld

\pagestyle{plain}

\addcontentsline{toc}{chapter}{Zusammenfassung}
% revised Jan. 1, 1992
\vspace*{4cm}
{\Huge \bf Zusammenfassung} 
\\[4cm]
Experimente, die an den Elektronenbeschleunigern ALS in Saclay und
MAMI A in Mainz durchgef\"uhrt wurden, haben zu einem starken 
Wiederaufleben des Interesses an der Pionphotoproduktion gef\"uhrt.
Stellvertretend f\"ur diese Experimente haben wir im ersten Kapitel 
die Daten aus Mainz diskutiert. Dabei haben wir gezeigt, da\ss\ die
experimentellen Ergebnisse direkt an der Schwelle nicht im Widerspruch
zu der Vorhersage des Niederenergietheorems, $\Eop=-2.3\su$, stehen. 
Unerwartet ist allerdings die starke Energieabh\"angigkeit der
elektrischen Dipolamplitude in der Schwellenregion. 

Vor diesem Hintergrund haben wir die theoretischen Grundlagen der
Stromalgebravorhersage untersucht. Das Niederenergietheorem liefert
nur im Grenzfall verschwindender Pionmasse eine eindeutige Bestimmung 
der Schwellenamplitude. Um die elektrische Dipolamplitude f\"ur 
massive Pionen zu bestimmen, entwickelt man die \"Ubergangsmatrix in 
eine Potenzreihe in $\mu=m_\pi/M$. Das oben zitierte Ergebnis
$\Eop=-2.3\su$ beruht auf der Berechnung der Photoproduktionsamplitude
bis zur Ordnung $\mu^2$. Um die Eindeutigkeit dieses Ergebnisses zu
demonstrieren, ben\"otigt man allerdings zus\"atzliche Annahmen, 
die \"uber Stromalgebra und PCAC hinausgehen. 

Wir haben daher die Bedeutung verschiedener Korrekturen zum 
Niederenergietheorem studiert. W\"ahrend Nukleonresonanzen und 
Vektormesonen keine wesentlichen Beitr\"age zur Schwellenamplitude 
liefern, ergeben sich im Rahmen eines einfachen Quarkmodells
bedeutende Korrekturen auf Grund der expliziten Symmetriebrechung
durch die nichtverschwindenden Strommassen der Quarks. 

Allerdings verletzen nichtrelativistische Quarkmodelle die chirale
Symmetrie, auf der die Niederenergietheoreme basieren, bereits im
Ansatz. Es ist daher von gro\ss er Bedeutung, die Effekte der 
endlichen Strommassen auch in chiralen Modellen des Nukleons zu
untersuchen. Dabei haben wir gezeigt, da\ss\ Modelle, die eine 
gute ph\"anomenologische Beschreibung insbesondere der axialen
Struktur des Nukleons beinhalten, Korrekturen $\DEop <0.5\su$
liefern. Diese Schranke ist deutlich kleiner als die in 
nichtrelativistischen Quarkmodellen gewonnene Absch\"atzung
$\DEop =1.6\su$, so da\ss\ wir keine wesentliche Modifikation 
des Niedernenergietheorems finden.

Wichtige Aufschl\"usse \"uber die Rolle der expliziten Symmetriebrechung 
erhoffen wir uns aus der Bestimmung der elektrischen Dipolamplitude
f\"ur die Produktion neutraler Pi\-onen am Neutron. Eine experimentelle
Untersuchung dieser Reaktion mit Hilfe der quasifreien Pionproduktion
am Deuteron befindet sich in Mainz in Planung. Ein deutliches
Abweichen der Schwellenamplitude von der Vorhersage des 
Niederenergietheorems, $\Eon =-0.5\su$, w\"are ein Hinweis auf eine
starke Brechung der Isospinsymmetrie in der Pionphotoproduktion.

Ebenfalls im Planungsstadium befinden sich Experimente zur Photoproduktion
von Eta-Mesonen. Die theoretische Bestimmung der zugeh\"origen 
Schwellenamplitude liefert das Resultat $E_{0+}(\eta p)=11.6\su$.
Diese Amplitude ist im wesentlichen durch die $N(1535)$-Resonanz 
dominiert, deren Photoanregung in der Pionproduktion studiert werden 
kann.  Die wichtigste Unsicherheit bei der Bestimmung von $E_{0+}
(\eta p)$ besteht daher nicht in der Festlegung der Resonanzparameter,
sondern in der Form der Eta-Nukleon-Kopplung.
\\

Im zweiten Teil der Arbeit haben wir die Spektren der Vektor- und 
Axialvektormesonen mit Hilfe von QCD-Summenregeln untersucht. Das
Spektrum der Vektormesonen  haben wir aus den Daten zur $e^+e^-$
Annihilation in Hadronen entnommen, w\"ahrend die spektrale Dichte
der Axialvektormesonen mit Hilfe der invarianten Massenverteilungen
im $\tau$-Zerfall bestimmt wurde.

Auf Grund seines g\"unstigen Verhaltens bez\"uglich der experimentellen
Unsicherheiten haben wir den Borelquotienten der Spektralfunktionen 
verwendet, um die G\"ultigkeit der Summenregeln zu untersuchen.
Im Vektorkanal zeigt sich ein gute \"Ubereinstimmung der Daten mit der
QCD-Parametrisierung. Da die Ergebnisse nicht stabil bez\"uglich der
Wahl des Borelparameters sind, erm\"oglichen die Resultate aber keine
eindeutige Bestimmung der Quark- und Gluonkondensate. Diese 
Feststellung befindet sich im Widerspruch zu fr\"uheren Arbeiten zu
diesem Thema \cite{LNT84}. Zuverl\"assig bestimmt ist lediglich das 
Verh\"altnis der Quark- und Gluonkondensate. Verwendet man den
Standardwert des Gluonkondensats, $\langle\frac{\alpha_s}{\pi}G^2
\rangle =(360\pm 20\,\mev)^4$, so finden wir eine Verletzung der
Faktorisierunghypothese f\"ur die Kondensate $\langle\bar\psi
\Gamma_1\psi\bar\psi\Gamma_2\psi\rangle$ um den Faktor $\xi^V =2.4$. 

Das Spektrum der Axialvektormesonen haben wir zun\"achst im Rahmen 
eines Modells studiert, in dem die Daten aus dem $\tau$-Zerfall 
jenseits der Masse $m_\tau =1.784\,\gev$ des $\tau$-Leptons mit 
Hilfe des st\"orungstheoretischen Resultats fortgesetzt wurden.
Dieses Modell liefert allerdings eine Spektralfunktion, die nicht
in der Lage ist, die Summenregeln  zu saturieren. 

Wir haben daher das Verfahren umgekehrt und QCD-Summenregeln eingesetzt,
um die Parameter der zweiten Resonanz im Spektrum zu bestimmen.
Die Summenregeln liefern eine Resonanzmasse $m_{a_1'}=1.7\,\gev$,
die im Bereich der $\tau$-Masse liegt. Diese Tatsache, zusammen 
mit der gro\ss en Kopplung der $a_1'$-Resonanz, die wir gefunden haben,
sollte es erm\"oglichen, diese Anregung in zuk\"unftigen Experimenten 
zu identifizieren.


\cld

\begin{appendix}

\include{an1}

\include{an2}

\include{an3}

\include{an4}

\include{an5}

\end{appendix}

\cld

\begin{thebibliography}{ABC.99}
\bibitem[0]{0} \underline{\LARGE\bf Teil I} 
\bibitem[Ada76]{Ada76} M. I. Adamovich, Proceedings of the P. N. Lebedeev
              Institute 71 (1976); siehe auch : M. N. Nagels {\em et al.}
	      Physics Data, Compilation of Coupling Constants and Low 
	      Energy Parameters, Nucl. Phys. {\bf 147} (1979) 189 
\bibitem[AD66]{AD66}  S. L. Adler, Y. Dothan, Phys. Rev. {\bf 151}
              (1966) 1267	 
\bibitem[AD68]{AD68} S. L. Adler, R. F. Dashen, Current Algebra and
              Applications to Particle Physics, W. A. Benjamin Inc.,
	      (1968) 	      
\bibitem[AFF73]{AFF73} V. de Alfaro, S. Fubini, G. Furlan, C. Rosetti, 
              Currents in Hadron Physics, 
	      North Holland Publishing Company (1973)	      
\bibitem[AFF79]{AFF79} E. Amaldi, S. Fubini, G. Furlan, Pion
              Electroproduction, Springer Tracts in Modern Physics 83,
	      Springer Verlag (1979) 	           
\bibitem[AG66]{AG66}  S. L. Adler, F. J. Gilman, Phys. Rev.
              {\bf 152} (1966) 1460
\bibitem[AR88]{AR88} G. Altarelli, G. G. Ross, \plb{212}{88}{391}	      
\bibitem[AS89]{AS89} G. Altarelli, W. J. Stirling, Particle World, 
              Vol. 1.2 (1989) 40 	      
\bibitem[Bae70]{Bae70} P. de Baenst, Nucl. Phys. 
              {\bf B24} (1970) 633	      
\bibitem[BB85]{BB85} M. C. Birse, M. K. Banerjee, Phys. Rev. 
              {\bf D31} (1985) 118
\bibitem[BDW67]{BDW67} F. A. Berends, A. Donnachie, D. L. Weaver,
              \npb{5}{67}{1}	      
\bibitem[Bec89]{Bec89} R. Beck, Dissertation, Institut f\"ur 
              Kernphysik der Unversit\"at Mainz, Mainz (1989)	      
\bibitem[Bec90]{Bec90} R. Beck {\em et al.}, Phys. Rev. Lett. 
              {\bf 65} (1990) 1841
\bibitem[Ber91]{Ber91} J. C. Bergstrom, Vorabdruck, Sasketchawan
              Accelarator Laboratory (1991)	      
\bibitem[BH91]{BH91} A. M. Bernstein, B. R. Holstein, Comments on 
              Nuclear and Particle Physics (1991), im Druck	      
\bibitem[Bir86]{Bir86} M. C. Birse, Phys. Rev. 
              {\bf D33} (1986) 1934
\bibitem[BKG91]{BKG91} V. Bernard, N. Kaiser, J. Gasser, U.-G. Meissner,
              \plb{268}{91}{291} 
\bibitem[BM91]{BM91} M. Benmerouche, N. C. Mukhopadhyay, \prl{67}{91}{1070}
\bibitem[BPP71]{BPP71} L. S. Brown, W. J. Pardee, R. D. Peccei, Phys.
              Rev. {\bf D4} (1971) 2801
\bibitem[BR88]{BR88} G. E. Brown, M. Rho, Comm. Nucl. Part. Phys. 18 (1988) 1      	      
\bibitem[CB86]{CB86} T. D. Cohen, W. Broniowski, Phys. Rev. 
              {\bf D34 } (1986) 3472
\bibitem[CGL57]{CGL57} G. F. Chew, M. L. Goldberger, F. E. Low, Y. Nambu,
              \pr{106}{57}{1345}	      
\bibitem[Col67]{Col67} S. Coleman, Soft Pions, Erice Lectures 1967, 
              in : Aspects of Symmetry, Cambridge University Press 	      
\bibitem[Don72]{Don72} A. Donnachie, Photo and Electroproduction Processes,
              in: High Energy Physics, Hgb.: A. S. Burhop, Academic Press
	      (1972)	
\bibitem[DMW91]{DMW91} R. M. Davidson, N. C. Mukhopadhyay, R. S. Wittman,
              \prd{43}{91}{71} 	            	      
\bibitem[Dum82]{Dum82} O. Dumbrais et al., Compilation of Coupling Constants
              and Low-Energy Parameters, 1982-Edition, \npb{216}{83}{277}      
\bibitem[DR87]{DR87} C. A. Dominguez, E. de Rafael, Ann. Phys. (N.Y.)
              {\bf 147} (1987) 372	 
\bibitem[DT91]{DT91} D. Drechsel, L. Tiator, Vorabdruck, Institut f\"ur 
              Kernphysik der Universit\"at Mainz, MKPH-91-6 	    
\bibitem[EMC89]{EMC89} EMC Collaboration, J. Ashman {\em et al.}, Nucl. Phys. 
              {\bf B328} (1989) 1
\bibitem[EW88]{EW88} T. Ericson, W. Weise, Pions and Nuclei, Clarendon 
              Press, Oxford (1988)	      
\bibitem[FPV69]{FPV69} G. Furlan, N. Paver, G. Verzegnassi, Nuo. Cim. {\bf 62A} 
              (1969) 519
\bibitem[FPV74]{FPV74} G. Furlan, N. Paver, G. Verzegnassi, Nuo. Cim. {\bf 20A}
              (1974) 295
\bibitem[GL82]{GL82} J. Gasser, H. Leutwyler, Phys. Rep. {\bf 87} (1982) 77
\bibitem[GMG54]{GMG54} M. Gell-Mann, M. L. Goldberger, \pr{96}{54}{1433}
\bibitem[GOR68]{GOR68} M. Gell-Mann, R. J. Oakes, B. Renner, \pr{175}{68}{2195}
\bibitem[HSK90]{HSK90} A. Hosaka, T. Sch\"afer, U. Kalmbach, \zpa{337}{90}{447}     
\bibitem[HTW90]{HTW90} A. Hosaka, H. Toki, W. Weise, \npa{506}{90}{501}
\bibitem[HW89]{HW89} A. Hosaka, W. Weise, \plb{232}{89}{442}  
\bibitem[JM90]{JM90} R. L. Jaffe, A. V. Manohar,\npb{337}{90}{509} 
\bibitem[JJP89]{JJP89} P. Jain, R. Johnson, N. W. Park, J. Schechter,
              H. Weigel, \prd{40}{89}{855}
\bibitem[Kam89]{Kam89} A. N. Kamal, Phys. Rev. Lett. 
              {\bf 63} (1989) 2346;
\bibitem[Kam90]{Kam90} A. N. Kamal, Vorabdruck, University of Alberta,
              Alberta Thy-8-90
\bibitem[KJR86]{KJR86} D. E. Kahana, A. D. Jackson, G. Ripka,
              \npa{459}{86}{663}  	      
\bibitem[KR54]{KR54} N. M. Kroll, M. A. Ruderman, \pr{93}{54}{233}
\bibitem[Low54]{Low54} F. E. Low, \pr{96}{54}{1428}	      	      
\bibitem[Low58]{Low58} F. E. Low, \pr{110}{58}{974}
\bibitem[LP71]{LP71} L.-F. Li, H. Pagels, \prl{26}{71}{1204}
\bibitem[LYL91]{LYL91} C. Lee, S. N. Yang, T.-S. H. Lee, J. Phys. 
              {\bf G17} (1991) 131      
\bibitem[Maz86]{Maz86} E. Mazzucato {\em et al.}, Phys. Rev. Lett.
              {\bf 57} (1986) 3144
\bibitem[MS76]{MS76} J. T. MacMullen, M. D. Scadron, Phys. Rev. {\bf D20}
              (1979)  1069; {\em ibid.} 1081
\bibitem[Nau91]{Nau91} H. W. L. Naus, Phys. Rev. {\bf C43} (1991) R365; 
              \prc{44}{91}{531}
\bibitem[NB80]{NB80} L. M. Nath, B. K. Bhattacharyya, \zpc{5}{80}{9}
\bibitem[NS89]{NS89} L. M. Nath, S. K. Singh, Phys. Rev. 
              {\bf C39} (1989) 1207 
\bibitem[NSV74]{NSV74} M. Nigro, P. Spillantini, V. Valente, Nucl. Phys.
              {\bf B84} (1974) 201	      	      
\bibitem[NK87]{NK87}   H. W. L. Naus, J. H. Koch, \prc{36}{87}{2459}	      
\bibitem[NKF90]{NKF90} H. W. L. Naus, J. H. Koch, J. L. Friar,
              Phys. Rev. {\bf C41}(1990) 2852	      
\bibitem[NLB90]{NLB90} S. Nozawa, T.-S. H. Lee, B. Blankleider, 
              Phys. Rev. {\bf C41}(1990) 213, {\em ibid.} 1306	      
\bibitem[OO75]{OO75} M. G. Olson, E. T. Ossypowski, \npb{87}{75}{399} 	      
\bibitem[PDG90]{PDG90} Particle Data Group, J.J. Hernandez et al.,
              \plb{239}{90}{1}	      
\bibitem[Pec69]{Pec69} R. D. Peccei, \pr{181}{69}{1902}
\bibitem[PP71]{PP71} H. Pagels, W. J. Pardee, \prd{4}{71}{3335}
\bibitem[PV89]{PV89} B.-Y. Park, V. Vento, \npa{504}{89}{829}	      
\bibitem[RST76]{RST76}G. M. Radutskii, V. A. Sverdutskii, 
              A. N. Tabachenko, Yad. Fiz. {\bf 24} (1976) 400
\bibitem[Sch91]{Sch91} B. Schoch, Private Mitteilung, (1991) 	      
\bibitem[SK91]{SK91} S. Scherer, J. H. Koch, \npa{534}{91}{461}
\bibitem[ST77]{ST77} S. S. Shei, H. S. Tsao, \prd{15}{77}{3049}
\bibitem[Str90]{Str90} H. Str\"oher, Habilitationsschrift, 
              Giessen (1990)
\bibitem[SW90]{SW90} T. Sch\"afer, W. Weise, \plb{250}{90}{6}
\bibitem[SW91]{SW91} T. Sch\"afer, W. Weise, \npa{531}{91}{520}     	       
\bibitem[Tab90]{Tab90} A. N. Tabachenko, Yad. Fiz. 51 (1990) 1623,
              Sov. J. Nucl. Phys. 51 (1990) 1026 	      
\bibitem[TD90]{TD90} L. Tiator, D. Drechsel, Nucl. Phys. 
              {\bf A508} (1990) 541c	      
\bibitem[TDR88]{TDR88} F. Tabakin, S. A. Dytman, A. S. Rosenthal, in:
              Excited Baryons 88, Hgb.: G. Adams, N. C. Mukhopadhyay,
	      P. Stoler, World Scientific (Singapore)   	      
\bibitem[VJG84]{VJG84} L. Vepstas, A. D. Jackson, A. S. Goldhaber, Phys. Lett.
              {\bf B140} (1984) 280	                          
\bibitem[VZ72]{VZ72}  A. I. Vainshtein, V. I. Zakharov, Nucl. Phys.
              {\bf B36} (1972) 589
\bibitem[Yan89]{Yan89} S. N. Yang,
	      Phys. Rev. {\bf C40} (1989) 1810\\
	      \\  \\
	      \underline{\LARGE\bf Teil II}
\bibitem[Alb86]{Alb86} H. Albrecht et al., ARGUS collaboration, 
              \zpc{33}{86}{7}	      
\bibitem[Alb88]{Alb88} H. Albrecht et al., ARGUS collaboration, 
              \plb{202}{88}{149}	      	      
\bibitem[Bac79]{Bac79} C. Bacci et al., Adone $\gamma\gamma 2$
              collaboration, \plb{86}{79}{234}
\bibitem[Ban87]{Ban87} H. R. Band et al., MAC collaboration,
              \plb{198}{87}{297}
\bibitem[BDL82]{BDL82} D. A. Bryman, P. Depommier, C. Leroy, 
              \prp{88}{82}{152}	      	      	      
\bibitem[BDL88]{BDL88} R. A. Bertlmann, C. A. Dominguez, M. Loewe,
              M. Perrotet, E. de Raffael,  \zpc{39}{88}{231}
\bibitem[BG85]{BG85} D. Broadhurst, S. C. Generalis, \plb{165}{85}{175}
\bibitem[Bis89]{Bis89} D. Bisello et al., DCI-DM2 collaboration, 
              \plb{220}{89}{321} 
\bibitem[Bow86]{Bow86} M. G. Bowler, \plb{182}{86}{400} 	      
\bibitem[Cap91]{Cap91} I. Caprini, \prd{44}{91}{1569}
\bibitem[CM90]{CM90} M. B. Causse, G. Menessier, \zpc{47}{90}{611}
\bibitem[Cor82]{Cor82} A. Cordier et al., Orsay DCI collaboration,
              \plb{109}{82}{129}
\bibitem[DGM67]{DGM67} T. Das, G. Guralnik, V. Mathur, F. Low, J. Young,
              \prl{19}{67}{759}
\bibitem[DMO67]{DMO67} T. Das, V. Mathur, S. Okubo, \prl{19}{67}{859}  	      
\bibitem[FNR79]{FNR79} E. G. Floratos, S. Narison, E. de Rafael,
              \npb{155}{79}{115}
\bibitem[GKL88]{GKL88} S. G. Gorishny, L. Kataev, A. Larin,
              \plb{212}{88}{238}
\bibitem[GR85]{GR85} F. Gilman, S. H. Rhie, \prd{31}{85}{1066}    
\bibitem[Hoe83]{Hoe83} G. H\"ohler, Pion-Nucleon Scattering,
              Landolt B\"ornstein, Bd. I/9b2, Springer Verlag (1983)        
\bibitem[IKM89]{IKM89} J. Iizuka, H. Koibuchi, F. Masuda, \prd{39}{89}{3357}
\bibitem[LNT84]{LNT84} G. Launer, S. Narison, R. Tarrach, \zpc{26}{84}{433}
\bibitem[Met90]{Met90} B. Metsch, Verhandlungsberichte des Arbeitstreffens
              Mittelenergiephysik der DPG, Hambach (1990), Seite 159
\bibitem[Nar89]{Nar89} S. Narison, QCD Spectral Sum Rules, World
              Scientific, Singapore (1989)
\bibitem[NOS78]{NOS78} V. Novikov, L. Okun, M. Shifman, A. Vainshtein, 
              M. Voloshin, V. Zakharov, \prp{41}{78}{1}	 
\bibitem[Oku82]{Oku82} L. B. Okun, Leptons and Quarks, North Holland (1982)           
\bibitem[PS87]{PS87} R. D. Peccei, J. Sola, \npb{281}{87}{1}	      
\bibitem[PT84]{PT84} P. Pascual, R. Tarrach, QCD: Renormalization 
              for the Practioner, Springer Lecture Notes in Physics,
	      Volume 194, Springer Verlag (1984)
\bibitem[Que78]{Que78} A. Quenzer et al., Orsay DCI collaboration,
             \plb{76}{78}{512}	      	      	      
\bibitem[RRY85]{RRY85} L. J. Reinders, H. Rubinstein, S. Yazaki,
              \prp{127}{85}{1}
\bibitem[Ruc86]{Ruc86} W. Ruckstuhl et al., DELCO collaboration, 
              \prl{56}{86}{2132}	      
\bibitem[Sid76]{Sid76} V. A. Sidorov, VEPP-2M collaboration, 
              Proceedings of the XVIII conference on High Energy
	      Physics, Tbilissi (1976) 	      
\bibitem[SVZ79]{SVZ79} M. A. Shifman, A. I. Vainshtein, V. I. Zakharov,
              \npb{147}{79}{385,448}
\bibitem[Tsa71]{Tsa71} Y. S. Tsai, \prd{4}{71}{2821}
\bibitem[VIO90]{VIO90} M. K. Volkov, Yu. P. Ivanov, A. A. Osipov,
              Yad. Fiz. {\bf 52} (1990) 129, Sov. J. Nucl. Phys. {\bf 52}
	      (1990) 82 	      
\bibitem[Wei67]{Wei67} S. Weinberg, \prl{18}{67}{507}	       	      
\end{thebibliography} 


\pagestyle{empty}

\cleardoublepage

{\Large\bf  Danksagung}\\[2.0cm]
Mein besonderer Dank gilt Herrn Prof.~Dr.~Wolfram Weise f\"ur
die Aufnahme in sein Institut, seine unerm\"udliche
Diskussionsbereitschaft und f\"ur das best\"andige 
Interesse am Fortgang dieser Arbeit.\\

W\"ahrend der Zeit, in der diese Arbeit entstanden ist, 
habe ich von vielen Diskussionen mit Kollegen und Freunden 
profitiert. Ich m\"ochte insbesondere nennen A.~Hosaka, 
S.~Klimt, M.~Lutz, G.~Piller, B.~Schoch, H.~Str\"oher, L.~Tiator
und U.~Vogl.  \\

Einen besonderen Dank an Claudia, Margit und Volker f\"ur die
Durchsicht der Arbeit und an Franz Stadler f\"ur die Anfertigung
der Zeichnungen. 


\end{document}

