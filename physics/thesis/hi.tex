\addcontentsline{toc}{chapter}{Einleitung}
% revised Jan. 1, 1992  
\vspace*{4cm}
{\Huge \bf Einleitung} 
\\[3.5cm]
Das Pion nimmt als leichtestes aller stark wechselwirkenden Teilchen
eine zentrale Stellung in der Kern- und Elementarteilchenphysik ein.
Ein besonders geeigneter Proze\ss\ zum Studium der Piondynamik ist die
Photoproduktion, das hei\ss t die Erzeugung von Pionen mittels
energetischer Photonen.

Den angemessenen theoretischen Rahmen f\"ur die Untersuchung der
Pionphotoproduktion liefert die Quantenchromodynamik, die
inzwischen anerkannte Eichtheorie der starken Wechselwirkung.
Auf Grund des nichtperturbativen Charakters der Theorie bei kleinen
Energien ist allerdings die Beschreibung der Pionphotoproduktion 
auf der Basis der QCD  bis heute nicht m\"oglich. Einen 
sehr erfolgreichen Ansatz liefert dagegen die Verwendung von
Stromalgebratechniken zur Bestimmung der Photoproduktionsamplitude 
an der Schwelle. Diese Methode verwendet die chirale Invarianz 
der QCD-Lagrangedichte, um die Schwellenamplitude mit Hilfe 
elementarer hadronischer Parameter, wie der Pionzerfallskonstante
und der axialen Kopplung des Nukleons, auszudr\"ucken.     

Neue Experimente zur Pionphotoproduktion an der Schwelle wurden an
den Elektronenbeschleunigern ALS in Saclay und MAMI A in Mainz 
durchgef\"uhrt. Diese Messungen haben zu Diskrepanzen mit den 
klassischen Stromalgebravorhersagen gef\"uhrt. Wir wollen daher
im ersten Teil der Arbeit eine kritische Analyse der experimentellen
Resultate und des theoretischen Niederenergietheorems vornehmen.
Konzentrieren werden  wir uns in diesem Zusammenhang auf m\"ogliche 
Korrekturen zum Niederenergietheorem, die sich aus der expliziten 
Brechung der chiralen Symmetrie durch die Stromquarkmassen in der 
QCD-Lagrangedichte ergeben. \newpage

Wichtige R\"uckschl\"usse auf die Rolle der expliziten Symmetriebrechung
erm\"oglicht auch das Studium der Photoproduktion von Eta-Mesonen, mit
dem wir uns in Kapitel 3 dieser Arbeit befassen wollen.  Das
Eta-Meson ist wie das Pion ein pseudoskalares Meson, besitzt aber eine
deutlich gr\"o\ss ere Masse. Aus diesem Grund haben Korrekturen
zum einfachen Niederenergietheorem eine deutlich gr\"o\ss ere Bedeutung,
als dies in der Photoproduktion von Pionen der Fall ist.

Im zweiten Teil der Arbeit wollen wir eine der Stromlagebra entlehnte
Technik, die Analyse spektraler Summenregeln, auf das Spektrum der 
Vektor- und Axialvektormesonen anwenden. Diese Spektren lassen sich 
experimentell in der $e^+e^-$-Annihilation bzw.~in hadronischen 
Zerf\"allen des $\tau$-Leptons bestimmen. 

Bei niedrigen Massen zeigen die Spektren die komplexe Struktur der 
Quark-Antiquark Bindungszust\"ande in der Quantenchromodynamik.
Dagegen beschreiben die Spektralfunktionen im Bereich gro\ss er invarianter
Massen die Propagation eines nur schwach wechselwirkenden 
Quark-Antiquark Paares im QCD-Vakuum. Auch der Grundzustand der Theorie
ist au\ss erordentlich kompliziert, insbesondere erwartet man die
Kondensation von Gluonen und Quark-Antiquark Paaren. Diese Kondensate 
lassen sich bislang nur in einfachen Modellen des QCD-Vakuums 
bestimmen. Ihr Einflu\ss\ auf das Spektrum bei sehr hohen Massen 
kann aber in st\"orungstheoretischen Rechnungen systematisch 
ber\"ucksichtigt werden.

QCD-Summenregeln verkn\"upfen das experimentell bestimmte Spektrum bei
kleinen Massen mit der asymptotischen Form des Spektrums in 
st\"orungstheoretischer QCD. Sie er-m\"og\-li\-chen damit einen wichtigen
Test f\"ur die oben angesprochenen Modelle der Vakuumstruktur.

Im Vektormesonkanal gibt es sowohl f\"ur Systeme schwerer Quarks als
auch f\"ur leichte Quarks eingehende Untersuchungen \"uber die 
Konsistenz der Daten mit QCD-Sum\-men\-re\-geln. Wir wollen in dieser 
Arbeit eine solche Analyse auch f\"ur den Axialvektorkanal vornehmen.
Theoretisch unterscheiden sich diese beiden Kan\"ale bei hohen 
invarianten Massen durch die Form der auftretenden Quarkkondensate.  
Eine Untersuchung des Spektrums der Axialvektormesonen erm\"oglicht daher
Aussagen \"uber die Faktorisierbarkeit der Quarkkondensate.
