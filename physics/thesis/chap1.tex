\chapter{Pionphotoproduktion}
\section{Einleitung}
%revised Jan. 2, 1992
Das Pion spielt eine herausragende Rolle als Bindeglied zwischen 
der klassischen Kernphysik als Theorie der Wechselwirkung von 
Nukleonen im Kernverband und der heute anerkannten fundamentalen
Theorie der starken Wechselwirkung, der Quantenchromodynamik.  
Diese Bedeutung ergibt sich aus der Sonderstellung der Pionen
$(\pi^\pm,\pi^0)$, welche mit einer Masse von 139 MeV (bzw.~135 MeV)
die bei weitem leichtesten aller stark wechselwirkenden
Teilchen sind. 

Aus diesem Grund dominiert das Pion mit einer Comptonwellenl\"ange
von etwa 1.4 fm den langreichweitigen Teil der Nukleon-Nukleon
Wechselwirkung. Ebenso entstehen Pionen beim Zerfall der meisten
Anregungszust\"ande des Nukleons und liefern daher wichtige Informationen 
\"uber die Struktur der Hadronen bei niederen Energien.

Auf der anderen Seite versteht man die geringe Masse des Pions
als Manifestation der im Grundzustand spontan 
gebrochenen chiralen Symmetrie der Quantenchromodynamik.
Das Pion wird in diesem Zusammenhang als Goldstoneboson der
gebrochenen Symmetrie interpretiert und dominiert daher
den Niederenergiesektor der QCD. Dieser Bereich der Theorie
ist theoretisch nur schwer zug\"anglich, da die QCD-Kopplungskonstante
bei kleinen Energien sehr gro\ss\ wird. St\"orungstheoretische
Berechnungen mit Hilfe der fundamentalen Konstituenden der Theorie, den
Quarks und Gluonen, sind daher wenig sinnvoll. Pionische 
Reaktionen  sind ein wichtiges Feld, um 
unser Verst\"andnis dieses Teils der Theorie zu verbessern.

Im folgenden betrachten wir speziell die Photoproduktion 
von Pionen am Nukleon. Der Vorzug elektromagnetischer  
Prozesse liegt in der Tatsache, da\ss\  diese Wechselwirkung
besonders gut verstanden ist. Insbesondere erm\"oglicht die
kleine Kopplung $\frac{e^2}{4\pi}=\frac{1}{137}$ die Behandlung
der elektromagnetischen Wechselwirkung in niedrigster Ordnung. 
Die schwache Kopplung erweist sich auch als Vorteil
beim Studium von Photoproduktionsreaktionen an Kernen.
Photonen unterliegen nur einer geringen Absorption und erm\"oglichen 
die Produktion von Pionen \"uber das ganze Kernvolumen.

Das Studium der elementaren Produktion von Pionen am Nukleon ist vor 
allem aus zwei Gr\"unden bedeutsam. Zum einen ist das 
Verst\"andnis dieses Prozesses eine grundlegende Voraussetzung,
um mit Hilfe photoinduzierter Reaktionen am Kern das 
Verhalten von Pionen und nukleonischen Anregungen in 
Kernmaterie zu studieren. In dieser Arbeit wollen wir uns 
aber auf einen anderen Aspekt der Pionphotoproduktion
konzentrieren. In der N\"ahe der Schwelle erlaubt die Anwendung
von Niederenergietheoremen \cite{AD68}, die Photoproduktionsamplitude
in modellunabh\"angiger Weise als Funktion elementarer
hadronischer Parameter auszudr\"ucken. Diese Theoreme beruhen 
ausschlie\ss lich auf der Anwendung fundamentaler Symmetrien der
QCD und sind daher ein wichtiges Bindeglied zwischen der zugrunde 
liegenden Theorie und ph\"anomenologischen Modellen zur Beschreibung
des Pion-Nukleon Systems.
             

\section{Definition der invarianten Amplituden}
Die Photoproduktionsreaktion $\gamma(k)+N(p_1)\to\pi(q)+N(p_2)$
wird  durch die \"Ubergangsmatrix
\be
\label{tmat}
T^{a}=-e \epsilon^\mu \langle N(p_2)\pi^{a}(q)|V_\mu(0)|N(p_1)\rangle 
\ee
beschrieben. Dabei bezeichnen $k,q$ die Impulse des Photons sowie
des erzeugten Pions und $p_1,p_2$ die Impulse der ein- und 
auslaufenden Nukleonen. Dar\"uber hinaus ist $e$ die elektrische Ladung
und $\epsilon_\mu$ der Polarisationsvektor des einlaufenden Photons.
Der Index $a$ beschreibt den Isospin des produzierten Pions. Die 
\"Ubergangsmatrix l\"a\ss t sich wie folgt als Summe lorentzinvarianter
Amplituden schreiben
\be
\label{invamp}
T^{a} = \bar{u}(p_2)\,\sum_{\lambda=1}^{n_\lambda} A_{\lambda}(s,t)
   {\cal M}_\lambda \, u(p_1) \, .
\ee    
Die invarianten Amplituden $A_\lambda(s,t)$ sind Funktionen der 
Mandelstamvariablen $s$ und $t$, die wir in Anhang A definieren.
Sie liefern die Koeffizienten der 
Diracoperatoren ${\cal M}_\lambda$, deren Matrixelemente zwischen
freien Nukleonspinoren genommen sind. In der Literatur findet man
verschiedene m\"ogliche Entscheidungen bez\"uglich der Wahl dieser
Operatoren. Wir folgen hier de Baenst \cite{Bae70} und definieren
\newpage
\beq
\label{baeamp}
T^{a} &=& \frac{ie}{2M}\,\epsilon^\mu\, \bar{u}(p_2)\gamma_5 \left\{
    \frac{q_\mu}{2M} A_1^{a} + \frac{P_\mu}{2M} A_2^{a}
    + \gamma_\mu A_3^{a} \right .  \\
    & & \hspace{2.5cm} + \left. \left( \frac{q_\mu}{2M} A_4^{a}
     + \frac{P_\mu}{2M} A_5^{a} + \gamma_\mu A_6^{a} \right)
     \frac{\gamma\cdot k}{2M} \right\} u(p_1)\, , \nonumber 
\eeq  
wobei $P_\mu = \frac{1}{2} (p_1+p_2)_\mu$ den mittleren Impuls 
der Nukleonen bezeichnet.
Die Amplituden $A_\lambda^a$ sind  dimensionslos und frei von 
kinematischen Singularit\"aten. Dar\"uber hinaus haben sie den 
Vorzug, einen besonders kompakten Ausdruck f\"ur die 
Schwellenamplitude zu liefern.
 
Auf Grund der zus\"atzlichen Bedingungen, die sich aus der Forderung
nach Eichinvarianz der \"Ubergangsmatrix ergeben, sind  nicht
alle sechs Amplituden unabh\"angig voneinander. Verwendet man die
Erhaltung des elektromagnetischen Stroms $\partial_\mu V^\mu (x)=0$,
so ergibt sich
\be
\label{curcon}
  k^\mu T_\mu^{a} = -ek^\mu \langle N(p_2)\pi^{a}(q)|V_\mu(0)|N(p_1)\rangle =0 \; .
\ee   
Diese Bedingung liefert  zwei Gleichungen f\"ur 
die sechs Amplituden  $A_\lambda$ 
\be
\label{gaugecond}
\begin{array}{rcl} 
   2\nu_1 A_1^{a} + \nu A_2^{a}  &=& 0             \; , \\[0.2cm]
   4 A_3^{a} + 2 \nu_1 A_4^{a} + \nu A_5^{a} &=& 0 \; .
\end{array}   
\ee
Dabei haben wir die  dimensionslosen kinematischen 
Variablen 
\be
\label{dimvar}
  \nu = \frac{P\cdot k}{M^2} \hspace{1.5cm}
  \nu_1 = \frac{k\cdot q}{2 M^2}
\ee
eingef\"uhrt.  
Man kann die beiden Eichinvarianzbedingungen (\ref{gaugecond}) verwenden, 
um die Zahl der invarianten Amplituden auf vier zu reduzieren. Wir
geben eine solche Wahl von Operatoren in Appendix B an. Auf Grund
der genannten Vorz\"uge wollen wir im Hauptteil der Arbeit
aber bei der Definition (\ref{baeamp}) verbleiben.

Betrachtet man  die Elektroproduktion von Pionen in der Reaktion
$e(k_1)+N(p_1)\to e(k_2)+N(p_2)+\pi(q)$, so ist das ausgetauschte
Photon virtuell und besitzt eine nicht verschwindende invariante Masse
$k^2$. Man ben\"otigt daher  zwei
weitere Amplituden, um den hadronischen Teil der \"Ubergangsmatrix
zu spezifizieren. Eine denkbare Wahl f\"ur die entsprechenden
Diracoperatoren lautet
\beq
\label{elamp}
{\cal M}_7 &=& \frac{ie}{2M}\gamma_5 \frac{\epsilon\cdot k}{2M} \; ,\\
{\cal M}_8 &=& \frac{ie}{2M}\gamma_5 \frac{\epsilon\cdot k}{2M}
               \frac{\gamma\cdot k}{2M}  \; .  
\eeq
Neben der Zerlegung im Dirac-Raum ist es sinnvoll, die \"Ubergangsmatrix
auch nach verschiedenen Isospinkomponenten zu entwickeln.
Im Isospinraum transformiert sich der elektromagnetische Strom wie
die Summe eines Isoskalars und der dritten Komponente eines 
Isovektors
\be
\label{emcur}
 V_\mu = V_\mu^S\,+\, V_\mu^{3} \; ,
\ee
w\"ahrend die Pionquellfunktion ein reiner Isovektor ist. Aus der
Definition der \"Ubergangsmatrix ergibt sich daher folgende Isospinstruktur
f\"ur die invarianten Amplituden $A_\lambda$
\be
\label{isodec}
A_\lambda^{a} = A_\lambda^{(+)} \delta^{a3} + A_\lambda^{(-)}
\frac{1}{2} [\tau^{a},\tau^{3}] + A_\lambda^{(0)} \tau^{a} \; ,
\ee
wobei die Paulimatrizen $\tau^{a}$ auf den Isospinanteil der
Nukeonspinoren wirken. Amplituden, die zu bestimmten Isospins des
$\pi N$-System geh\"oren, lassen sich durch
\beq
  A_\lambda^{(\frac{1}{2})} &=& A_\lambda^{(+)}+2A_\lambda^{(-)} ,\\
  A_\lambda^{(\frac{3}{2})} &=& A_\lambda^{(+)}-A_\lambda^{(-)} 
\eeq
definieren. Die Amplitude $A_\lambda^{(0)}$ f\"uhrt auschlie\ss lich
zu $\pi N$-Zust\"anden mit dem Isospin $1/2$. Die Amplituden f\"ur die
vier physikalischen Kan\"ale sind durch
\beq
A_\lambda(\gamma p \to \pi^{+}n) &=&
           \sqrt{2} (A_\lambda^{(0)}+A_\lambda^{(-)} ), \\
A_\lambda(\gamma n \to \pi^{-}p) &=&
           \sqrt{2} (A_\lambda^{(0)}-A_\lambda^{(-)} ), \\
A_\lambda(\gamma p \to \pi^{0\,}p) &=&
            A_\lambda^{(+)}+A_\lambda^{(0)},  \\
A_\lambda(\gamma n \to \pi^{0\,}n) &=&
            A_\lambda^{(+)}-A_\lambda^{(0)}   
\eeq
gegeben. Ber\"ucksichtigt man isospinbrechende Effekte, wie sie durch die 
elektromagnetische Massendifferenz der Pionen und die kleine 
Differenz der up und down Stromquarkmassen in der QCD-Lagrangedichte
gegeben sind, so ist die Zerlegung (\ref{isodec}) nicht mehr
m\"oglich. In diesem Fall sind die Amplituden f\"ur die vier
Isospinkan\"ale unabh\"angig und eine experimentelle Bestimmung
der \"Ubergangsmatrix erfordert die Messung aller Reaktionskan\"ale.

Aus der Invarianz der \"Ubergangsmatrix unter Ladungskonjugation folgt, 
da\ss\ die invarianten Amplituden ein wohldefiniertes Verhalten unter
Austausch der Mandelstamvariablen $s$ und $u$ besitzen
\be
\label{cross}
 A_\lambda^{(+0,-)}(s,t,u) = \eta_\lambda^{(+0,-)} A_\lambda^{(+0,-)}
 (u,t,s) \; .
\ee  
Die Phasen $\eta_\lambda^{(+0,-)}$ sind durch die Gleichung
\be
\label{defphase}
 {\cal M}_\lambda (P,k,q) =-\eta_\lambda^{(+0)} C^{-1}
 {\cal M}_\lambda (-P,k,q) C
\ee
sowie $\eta_\lambda^{(-)}=\eta_\lambda^{(+0)}$ bestimmt.  Dabei
bezeichnet $C$ den Operator der Ladungskonjugation. Mit Hilfe
von (\ref{defphase}) findet man
\be
\label{cphase}
  \eta_\lambda^{(+0)} = \{ -1,+1,-1,-1,+1,+1 \}\, .
\ee   
Das Verhalten der dimensionslosen Gr\"o\ss en $\nu,\nu_1$ unter
dem Austausch der  Variablen im s- und u-Kanal lautet
$(\nu,\nu_1)\to (-\nu,\nu_1)$.  

\section[Multipolanalyse]{Multipolanalyse des differentiellen
 Wirkungs\-quer\-schnitts}
Um eine Multipolanalyse des differentiellen Wirkungsquerschnitts
durchf\"uhren zu k\"onnen, ist es zun\"achst sinnvoll, die 
\"Ubergangsamplitude nach Paulimatrizen und zweikomponentigen
Spinoren zu entwickeln. Eine solche Zerlegung ist nat\"urlich von
der Wahl des Bezugsystems abh\"angig. Wir definieren im 
Pion-Nukleon Schwerpunktsystem
\be
\label{famp}
\bar{u}(p_2)\, \sum_{\lambda} A_\lambda(s,t) {\cal M}_\lambda u(p_1)
 = \frac{4\pi\sqrt{s}}{M} \, \chi^{\dagger}_2 {\cal F} \chi_1\; ,
\ee
wobei $\chi_1$ und $\chi_2$ die zweikomponentigen Paulispinoren der ein- 
bzw.~auslaufenden Nu\-kle\-onen sind. Der kinematische Faktor in 
(\ref{famp}) hat 
den Zweck, die Definition der Amplitude ${\cal F}$ an die \"ublichen 
Konventionen f\"ur nichtrelativistische \"Ubergangsamplituden anzupassen.
Damit ergibt sich f\"ur den differentiellen Wirkungsquerschnitt
\be
\label{xdiff}
\frac{d \sigma}{d \Omega} = \frac{q}{k} \,
\left|\chi^\dagger_2 {\cal F} \chi_1 \right|^{2} \; .
\ee
Dabei bezeichnen $q=|\vec{q}\,|$ und $k=|\vec{k}|$ die Betr\"age der
Dreierimpulse des Pions bzw.~Photons. Wir werden diese Bezeichnung immer 
dann verwenden, wenn keine Verwechslungsgefahr mit den entsprechenden 
Vierervektoren besteht.
Der differentielle Wirkungsquerschnitt f\"ur unpolarisierte Teilchen
ergibt sich wie \"ublich,  indem man \"uber
die Spins im Endkanal summiert und im Eingangskanal mittelt. 

In Coulombeichung $\vec{\epsilon}\cdot\vec{k}=0$ l\"a\ss t sich der
Operator ${\cal F}$ nach vier linear unabh\"angigen Paulimatrizen
entwickeln 
\beq
\label{fdec}
{\cal F} &=& i F_1 (\vec{\sigma}\cdot\vec{\epsilon}\,) + F_2
             (\vec{\sigma}\cdot\hat{q}) (\vec{\sigma}\cdot
	     (\hat{k}\times\vec{\epsilon}\,))  \\
	 & &\mbox{} + i F_3 (\vec{\sigma}\cdot\hat{k})(\hat{q}\cdot
	     \vec{\epsilon}\,) + i F_4 (\vec{\sigma}\cdot\hat{q})
	     (\hat{q}\cdot\vec{\epsilon}\,) \nonumber \; .
\eeq
Die Amplituden $F_i$ sind komplexwertige Funktionen der
Mandelstamvariablen $s$ und $t$. Ihre Abh\"angigkeit vom
Streuwinkel $x=\hat{q}\cdot\hat{k}$ 
l\"a\ss t sich durch eine Entwicklung nach Legendrepolynomen 
extrahieren:
\beq
\label{f1mult}
F_1 &=& \sum_{l=0}^{\infty} \left[ l M_{l+} + E_{l+} \right] P^{'}_{l+1}(x)
	             +  \left[ (l+1) M_{l-} + E_{l-} \right] P^{'}_{l-1}(x), 
		     \\  
F_2 &=& \sum_{l=1}^{\infty} \left[ (l+1) M_{l+} + l M_{l-} \right] P^{'}_{l}(x)
		    , \\	  
F_3 &=& \sum_{l=1}^{\infty} \left[ E_{l+} - M_{l+} \right] P^{''}_{l+1}(x)
	             +  \left[ E_{l-} + M_{l-} \right] P^{''}_{l-1}(x), 
		     \\	  
\label{f4mult}		     
F_4 &=& \sum_{l=2}^{\infty} \left[ M_{l+} - E_{l+} - M_{l-} - E_{l-} \right] 
			     P^{''}_{l}(x) .
\eeq			     
Die energieabh\"angigen Multipolamplituden 
$E_{l\pm}$ und $M_{l\pm}$ geh\"oren zu Eigenzust\"anden des $\pi$N-Systems
mit dem Gesamtdrehimpuls $j=l\pm 1/2$. Elektrische bzw.~magnetische 
\"Ubergangsamplituden sind durch ihre Parit\"at $\pi=\pm (-1)^{l}$ 
charakterisiert. 

Die Multipolentwicklung dient vor allem zur experimentellen 
Bestimmung von Mul\-ti\-pol\-am\-plituden aus den gemessenen 
Winkelverteilungen. Um theoretisch bestimmte Spi\-nor\-am\-pli\-tuden
$F_i$ auf gegebene Multipolarit\"aten zu projizieren, ben\"otigt
man die Inversen der Beziehungen (\ref{f1mult}-\ref{f4mult}). 
Als Beispiel zitieren wir das Resultat f\"ur die elektrische 
Dipolamplitude \cite{BDW67}
\be
\label{eop}
E_{0+}(s) = \frac{1}{2}\int_{-1}^{1} dx\, \left(
  F_1 - x F_2 + \frac{1}{3} (1-P_{2}(x)) F_4 \right) .
\ee
Auch der totale Wirkungsquerschnitt l\"a\ss t sich als Funktion der
Multipolamplituden angeben. Mit Hilfe von (\ref{xdiff}) und 
(\ref{f1mult}-\ref{f4mult}) findet man 
\begin{figure}
\caption{Totaler Wirkungsquerschnitt f\"ur die Reaktionen 
$\gamma p\to\pi^{+}n$ und $\gamma p\to \pi^{0}p$ im Bereich 
von Laborenergien bis $\omega_{lab}=800$ MeV.}
\vspace{9cm}
\end{figure}   
\beq
\label{xtot}
 \sigma_{\rm tot} &=& 2\pi \frac{q}{k} \sum_{l=1}^{\infty}
 \left\{ l(l+1)^2 \left[ |M_{l+}|^2 +|E_{(l+1)-}|^2 \right] \right. \\
 & & \hspace{3cm} + l^2(l+1) \left.\left[ |M_{l-}|^2+|E_{(l-1)+}|^2 \right]
 \right\} \nonumber \; .
\eeq
Die Entwicklung nach Multipolen erweist sich als besonders
hilfreich, um die Beitr\"age von Nukleonresonanzen zur Photoproduktion
zu diskutieren. Solche Resonanzen besitzen wohl definierten Spin und Parit\"at
und tragen daher im $s$-Kanal selektiv zu bestimmten Multipolamplituden 
bei.

Zur Illustration zeigen wir in Abbildung 1.1 den totalen Wirkungsquerschnitt 
f\"ur die Produktion geladener und neutraler Pionen 
am Proton im Bereich von Photonenergien im
Laborsystem bis maximal $\omega_{lab}=800$ MeV. Als charakteristische
Eigenschaft der Reaktion in diesem Bereich erkennt man
die Anregung der $\Delta(1232)$-Resonanz. Auf Grund ihrer Quantenzahlen
$I(J^\pi)=\frac{3}{2}(\frac{3}{2}^{+})$ tr\"agt diese Resonanz vor 
allem zur $M_{1+}$-Amplitude bei und bewirkt eine ausgepr\"agte 
p-Wellenstruktur des differentiellen Wirkungsquerschnitts in der 
Resonanzregion. Die relative Gr\"o\ss e des Resonanzquerschnitts im
neutralen  und geladenen Kanal sollte wegen des Isospins der Anregung
genau $2:1$ betragen. Die Abweichung der experimentellen Daten von 
dieser Vorhersage ist eine Konsequenz der unterschiedlichen St\"arke 
des s-Wellen-Untergrunds in den beiden Kan\"alen.
 
\section{Photoproduktion an der Schwelle}
Wir haben bereits in der Einleitung angemerkt, da\ss\ die Photoproduktion
von Pionen an der Schwelle von gro\ss er Bedeutung im Zusammenhang mit 
chiralen Niederenergietheoremen ist. Wir wollen daher in diesem Aschnitt die 
spezielle Situation an der Schwelle etwas n\"aher untersuchen.

An der Produktionsschwelle ist  die Gesamtenergie
im Schwerpunktsystem durch die Sum\-me der Ruhemassen der Teilchen
im Endzustand gegeben. Die Einschu\ss energie des Photons im 
Laborsystem betr\"agt in diesem Fall
\be
\label{wthr}
\omega^{th}_{lab} = \frac{1}{2M_1} \left( (M_2 + m_{\pi^{a}})^2
   - M_1^2 \right) \; .
\ee
Dabei bezeichnet $M_1$ die Masse des Targetnukleons, $M_2$ die Masse
des auslaufenden Nukleons und $m_{\pi^{a}}$ die Masse des produzierten
Pions. Auf Grund der unterschiedlichen Massen 
\beq
\Delta m_\pi &=& m_\pi^\pm -m_\pi^0=4.6\,\mev ,  \\
\Delta M_N   &=& M_n-M_p = 1.3\, \mev
\eeq
der verschiedenen Ladungszust\"ande der Pionen und Nukleonen sind die
Schwellenenergien der Kan\"ale nicht identisch (siehe Tabelle 1.1). Von 
Bedeutung f\"ur die Behandlung von Endzustandskorrekturen ist vor allem die
Tatsache, da\ss\ bei gegebenem Isospin des Targets die Produktionsschwelle
f\"ur geladene Pionen einige MeV \"uber derjenigen f\"ur neutrale Pionen
liegt.
    
\begin{table}
\caption{Schwellenenergie des einlaufenden Photons im Labor und 
 Schwerpunktsystem}
 \begin{center}
 \begin{tabular}{|c||c|c|} \hline
         &  $\omega_{th}^{lab} $[MeV]  & $\omega_{th}^{\em cms}$ [MeV] \\
	 \hline \hline
 $\gamma p\to \pi^+ n$   &  151.43     &   131.67  \\ 
 $\gamma n\to \pi^- p$   &  148.45     &   129.40  \\
 $\gamma p\to \pi^0 p$   &  144.68     &   126.49  \\
 $\gamma n\to \pi^0 n$   &  144.67     &   126.50  \\ \hline
 \end{tabular}
 \end{center}
 \end{table}

Das Verhalten der Multipolamplituden in der N\"ahe der Schwelle l\"a\ss t
sich aus der Darstellung der Amplituden mit Hilfe von Dispersionsrelationen 
entnehmen. Verwendet man die Analytizit\"at und Austauschsymmetrie 
der Spinoramplituden, so ergibt sich   
\be
\label{thramp}
\begin{array}{cclc}
E_{l+}   & \sim & q^l k^l     & l\ge 0  \, ,\\
E_{l-}   & \sim & q^l k^{l-2} & l\ge 2  \, ,\\
M_{l\pm} & \sim & q^l k^l     & l\ge 1  \, . 
\end{array}
\ee
An der Schwelle wird der differentielle Wirkungsquerschnitt von der
elektrischen Di\-pol\-am\-plitude $E_{0+}$ dominiert und zeigt ein typisches
s-Wellenverhalten 
\be
\label{thrxdiff}
\left. \frac{d\sigma}{d\Omega}\right|_{thr} = 
\frac{q}{k} \left|E_{0+}\right|^2 \, .
\ee
Direkt an der Schwelle verschwindet $q$, w\"ahrend $k\simeq m_\pi$. 
Die elektrische Dipolamplitude bestimmt daher die Steigung des 
differentiellen Wirkungsquerschnitts in der N\"ahe der Schwelle als 
Funktion von $q/k \sim (\omega^{lab}-\omega_{th}^{lab})^{1/2}$. In der Praxis
ist dieses Verhalten allerdings nur schwer zu beobachten, da der
Wirkungsquerschnitt bereits bei sehr kleinen Pionenergien von
p-Wellen-Multipolen mit einer charakteristischen Energieabh\"angigkeit
$(\omega^{lab}-\omega^{lab}_{th})^{3/2}$ dominiert wird.

Die elektrische Dipolamplitude $E_{0+}$ an der Schwelle l\"a\ss t sich 
mit Hilfe der Gleichungen (\ref{famp}) und (\ref{eop}) aus den
invarianten Amplituden $A_\lambda$ bestimmen.  Zu diesem Zweck 
wertet man Matrixelemente
der Operatoren $M_\lambda$ zwischen freien Nukleonspinoren an der
Schwelle im Schwerpunktsystem aus. Identifiziert man den  
Koeffizienten der $(\vec{\epsilon}\cdot\vec{\sigma})$-Struktur, so
findet man 
\be
\label{thrinvamp}
 \left. E_{0+} \right|_{thr} = \frac{e}{16\pi M} \frac{2+\mu}{(1+\mu)^{3/2}}
  \left. ( A_3 + \frac{\mu}{2}  A_6 ) \right|_{thr}\, ,
\ee
wobei $\mu=m_\pi/M$ das Verh\"altnis der Pion- und Nukleonmassen bezeichnet.
 
Eine einfache Absch\"atzung der relativen St\"arke der Dipolamplitude
in den einzelnen Ladungskan\"alen 
kann man aus folgender \"Uberlegung gewinnen. An der Schwelle ist die
Wellenl\"ange des Photons $k^{-1}\simeq 1.4$ fm so gro\ss , da\ss\ es nicht 
in der Lage ist, die detaillierte Struktur des Nukleons aufzul\"osen. Die 
Dipolamplitude ist daher allein vom klassischen Dipolmoment des
Pion-Nukleon Systems im Endzustand abh\"angig \cite{EW88}. Aus dieser
Betrachtung ergibt sich
\be
\label{elpred}
\left.
\begin{array}{c}
 \Epn \\[0.2cm] \Emp \\[0.2cm] \Eop \\[0.2cm] \Eon 
\end{array}
\right\}
= \frac{ef}{4\pi m_\pi} \left( 1+\mu \right)^{-\frac{3}{2}}
\left\{
\begin{array}{c}
\sqrt{2} \\[0.2cm]
-\sqrt{2} ( 1 + \mu ) \\[0.2cm]
-\mu \\[0.2cm]
0
\end{array}
\right.  ,
\ee
wobei wir h\"ohere Ordnungen in $\mu=m_\pi/M$ vernachl\"assigt haben. Die
St\"arke der elektrischen Dipolamplitude ist durch den Faktor $f/m_\pi$
bestimmt, wobei $f$ die pseudovektorielle Pion-Nukleon-Kopplungskonstante
bezeichnet, $f^2/(4\pi) = 0.08$.  Das Resultat zeigt, da\ss\ die 
Produktion neutraler Pionen am Proton um etwa eine Gr\"o\ss enordnung 
gegen\"uber dem geladenen Kanal unterdr\"uckt ist. Insbesondere 
verschwindet die $\pi^0$-Produktionsamplitude, wenn die Pionmasse gegen 
Null geht. Die
Produktion neutraler Pionen ist daher besonders sensitiv auf die 
explizite Brechung der chiralen Symmetrie, die sich in einer von Null
verschiedenen Pionmasse widerspiegelt.  Diese Feststellung gilt 
in besonderer Weise f\"ur den $\pi^0 n$-Kanal, in dem die 
klassische N\"aherung eine verschwindende elektrische Dipolamplitude 
liefert. 


\section{Zur experimentellen Situation}
In diesem Abschnitt wollen wir eine detaillierte Diskussion der 
experimentellen Ergebnisse zur Pionphotoproduktion an der Schwelle
vornehmen. Bis vor einigen Jahren existierten zuverl\"assige Daten in
diesem Bereich nur in den beiden geladenen Kan\"alen. 
W\"ahrend diese Ergebnisse im Fall der Reaktion $\Rpn$ direkt an einem 
Wasserstofftarget gewonnen werden k\"onnen, stammen die Daten f\"ur die
Reaktion $\Rmp$ entweder aus der quasifreien Produktion am Deuteron oder dem 
inversen Proze\ss\ $\pi^- p \to \gamma n$. Die Ergebnisse finden sich
in Tabelle 1.2 und zeigen eine gute \"Ubereinstimmung mit den Vorhersagen
der chiralen Niederenergietheoreme. 

\begin{table}
\caption{Theoretische Vorhersagen und experimentelle Ergebnisse
f\"ur die Schwellenamplitude $ E_{0+}^{th}$ in Einheiten von
$10^{-3}m_\pi^{-1}$.}
\begin{center}
\begin{tabular}{|c||c|rl|} \hline
Kanal                  & $E_{0+}$(LET) & $E_{0+}$(Exp)  & Referenz \\ \hline
                                                                      \hline 
$\gamma p \to \pi^+ n$ &$\spm 26.6$    &  $28.3\pm 0.5$ & [Ada76] \\ \hline
$\gamma n \to \pi^- p$ &    $-31.7$    & $-31.9\pm 0.5$ & [Ada76] \\ \hline
$\gamma p \to \pi^0 p$ &    $-2.3$     &  $-2.7\pm 0.4$ & [NSV74] \\
                       &               &  $-0.5\pm 0.3$ & [Maz86] \\
		       &               & $-0.35\pm 0.1$ & [Bec89] \\ \hline  
$\gamma n \to \pi^0 n$ &    $-0.5$     & $-\hspace{1cm}$  &        \\ \hline
\end{tabular}
\end{center}
\end{table}

Gleiches gilt f\"ur den bis zum Jahre 1986 allgemein akzeptierten 
Wert f\"ur die $\pi^0 p$ Amplitude an der Schwelle, $\Eop=
2.7\pm 0.4 \su$ \cite{NSV74}. Dieses Ergebnis stammt allerdings aus
einer ph\"anomenologischen Multipolanalyse, die sich im wesentlichen
auf Daten aus der Resonanzregion st\"utzt. Dabei werden die Realteile
der niedrigsten Multipole ($l$=0,1) bei einer gegebenen  Energie mit
Hilfe eines freien Fits an die gemessenen Winkelverteilungen bestimmt. 
Die zugeh\"origen Imagin\"arteile fixiert man  aus
den bekannten $\pi N$-Streuphasen. H\"ohere Multipolamplituden werden
aus Dispersionsanalysen oder theoretischen Modellen \"ubernommen. 
Gew\"ohnlich ergeben sich bei
dieser Prozedur verschiedene S\"atze von Multipolamplituden, die eine
\"ahnlich gute \"Ubereinstimmung mit den Daten zeigen. Aus diesem Satz
von Amplituden w\"ahlt man mit Hilfe zus\"atzlicher Kriterien, wie
zum Beispiel der Forderung nach einer m\"oglichst stetigen 
Energieabh\"angigkeit, die beste L\"osung aus. Es sind vor allem diese
Kriterien, welche die Extrapolation zur Schwelle beeinflussen.

Inzwischen konnten jedoch unter Verwendung moderner 
Dau\-er\-strich-Elek\-tro\-nen\-be\-schleu\-niger in Saclay \cite{Maz86} und
Mainz \cite{Bec90} bedeutende Fortschritte bei der Bestimmung der
Photoproduktionsamplitude direkt an der Schwelle gemacht werden. 
Der wesentliche Vorzug dieser Maschinen besteht in der Tatsache, 
da\ss\ dank der hohen kontinuierlichen Intensit\"at
Mehrfach-Koinzidenzexperimente mit markierten (``tagged'')
Photonen m\"oglich sind. Diese Technik gestattet die Produktion von
Photonen mit sehr genau definierter Energie ($\sim 200$ keV) und 
erm\"oglicht pr\"azise Experimente direkt an der Schwelle. 
Es war daher eine betr\"achtliche \"Uberraschung, als die ersten 
Datenanalysen mit Resultaten $\Eop = (-0.5\pm 0.3) \su$ \cite{Maz86}
bzw. $(-0.35 \pm 0.1) \su$ \cite{Bec89} signifikante Abweichungen
vom Niederenergietheorem zeigten. 

Bei der Analyse dieser Resultate wollen wir uns im folgenden auf 
das Mainzer Experiment \cite{Bec90,Str90} beschr\"anken. Dieses Experiment 
zeichnet sich im Vergleich zu den Messungen aus Saclay durch eine deutlich 
bessere Statistik aus. Dar\"uber hinaus besteht das Problem,
da\ss\ an den Saclay-Daten, ebenso wie an den ersten, unver\"offentlichten 
Daten aus Mainz \cite{Bec89}, bereits eine theoretische Korrektur f\"ur 
Endzustandswechselwirkungen vorgenommen worden ist.
\begin{figure}
\caption{Differentielle Wirkungsquerschnitte f\"ur die Reaktion
$\gamma p\to \pi^0 p$. Gezeigt sind die Resultate aus Mainz
[Bec90] f\"ur verschiedene Laborenergien des einlaufenden
Photons.}
\vspace{20cm}
\end{figure}

In der N\"ahe der Schwelle ist es ausreichend, in der Multipolentwicklung
(\ref{f1mult}-\ref{f4mult}) nur Drehimpulse bis $l=1$ zu ber\"ucksichtigen. 
Auf Grund der Dominanz der resonanten $M_{1+}$-Amplitude ist diese N\"aherung
auch jenseits der unmittelbaren Schwellenregion in vielen F\"allen
ausreichend. Mit Hilfe von (\ref{xdiff}) ergibt sich dann f\"ur den 
differentiellen Wirkungsquerschnitt 
\be
\label{xang}
\frac{d\sigma}{d\Omega} = \frac{q}{k} \left(
 A + B\cos \Theta + C \cos^2\Theta \right) 
\ee
mit den Koeffizienten
\beq
 A & = & |E_{0+}|^2 + \frac{1}{2} ( |P_1|^2 + |P_3|^2 ), \nonumber \\ 
 \label{angcoef}
 B & = & 2 {\rm Re} (E_{0+}P_1^* ), \\
 C & = & |P_1|^2 - \frac{1}{2} ( |P_2|^2 + |P_3|^2 ) ,\nonumber
\eeq
wobei wir die p-Wellen-Multipole
\beq
\label{pmult}
 P_{1,2} &=& 3 E_{1+} \pm M_{1+} \mp M_{1-} \; , \\
 P_3     &=& 2 M_{1+} + M_{1-}
\eeq
eingef\"uhrt haben. Die in Mainz gemessenen differentiellen 
Wirkungsquerschnitte in der Schwellenregion zeigen wir in 
Abbildung 1.2. Ebenfalls dargestellt ist die Parametrisierung
(\ref{xang}). Die zugeh\"origen Koeffizienten $A,B,C$ finden
sich in der Referenz \cite{Bec90}. Um die Multipolamplituden 
aus diesen Daten zu extrahieren, schreiben wir (\ref{angcoef}) 
als ein System von Gleichungen f\"ur $E_{0+}$ und $P_1$ :
\begin{table}
\label{e0tab}
\caption{Bestimmung der elektrischen Dipolamplitude $\Eop$ aus 
den Mainzer Daten. Winkelverteilungskoeffizienten in Einheiten von
$nb\,sr^{-1}$, Dipolamplituden in $10^{-3} m_\pi^{-1}$.}
\begin{center}
\begin{tabular}{|c||r|c|c|c|} \hline
 $E_{\gamma}$ 
          & $A+B+C$      & $ E_{0+}^I $     &  $E_{0+}^{II}$   
	                 & $E_{0+}^f$         \\ \hline \hline
 146.8    & $36\pm 26$   & $-0.25\pm 0.20$  & $-1.59\pm 0.38$ 
                         & $-1.64\pm 0.10$     \\ 
 149.1    & $19\pm 27$   & $-0.72\pm 0.29$  & $-1.69\pm 0.45$ 
                         & $-1.26\pm 0.15$     \\  	      
 151.4    & $40\pm 31$   & $-0.56\pm 0.21$  & $-0.56\pm 0.21$ 
                         & $-0.79\pm 0.26$     \\  
 153.7    & $90\pm 34$   & $-0.25\pm 0.17$  & $-0.25\pm 0.17$ 
                         & $-0.54\pm 0.33$     \\ 
 156.1    & $189\pm 40$  & $-0.35\pm 0.14$  & $-0.35\pm 0.14$ 
                         & $-1.02\pm 0.39$     \\  \hline
\end{tabular}
\end{center}  
\end{table}
\be
\begin{array}{rcl}
A+C &=& |E_{0+}|^2 + |P_1|^2  \\[0.2cm]
 B\hspace{0.5cm}  &=& 2 {\rm Re} (E_{0+}P_1^* )  \; .
\end{array} 
\ee
Man erkennt bereits an dieser Stelle, da\ss\ die Gleichungen
v\"ollig symmetrisch in $E_{0+}$ und $P_1$ sind. Es sind daher
zus\"atzliche Kriterien erforderlich, um die korrekte
L\"osung auszuw\"ahlen. Unterhalb der $\pi^+ n$-Schwelle sind keine 
weiteren Kan\"ale offen und alle Multipole bis auf einen kleinen 
Beitrag aus der $\pi^0 p$-Streuphase reell. Vernachl\"assigt man 
die Imagin\"arteile, so ergeben sich vier L\"osungen f\"ur die Amplituden
$E_{0+}$ und $P_1$. Stellt man dar\"uber hinaus die physikalischen
Bedingungen $E_{0+}<0$ und $P_1>0$, so reduziert sich diese Zahl auf nur 
noch zwei L\"osungen\footnote{Bei der Auswahl der L\"osungen
haben wir von der Tatsache Gebrauch gemacht, da\ss\ f\"ur die 
gemessenen Winkelverteilungen $A>0$ sowie $B,C<0$ gilt.}
\beq
 E_{0+} &=& \frac{1}{2} \left( -\sqrt{A-B+C} \pm \sqrt{A+B+C} \right) \; ,\\
 P_1    &=& \frac{1}{2} \left( +\sqrt{A-B+C} \pm \sqrt{A+B+C} \right) \; . 
\eeq
Dabei ist in beiden Gleichungen jeweils dasselbe Vorzeichen zu w\"ahlen. 
Je nach dieser Wahl erh\"alt man $|E_{0+}|<|P_1|$ (L\"osung I) oder
$|E_{0+}|>|P_1|$ (L\"osung II). Nach Gleichung (\ref{thramp}) ist
das Schwellenverhalten der Amplituden durch $E_{0+} \sim {\rm const}$
und $P_1 \sim qk$ gegeben. Direkt an der Schwelle verschwindet $P_1$,
so da\ss\ die physikalisch korrekten Amplituden durch L\"osung II
gegeben sind. In der Resonanzregion dagegen ist die $M_{1+}$-Amplitude deutlich
gr\"o\ss er als die elektrische Dipolamplitude, so da\ss\ in diesem 
Bereich L\"osung I zu w\"ahlen ist. Das Problem besteht daher in der
Bestimmung der kritischen Energie, bei der von der einen L\"osung zur
anderen zu wechseln ist. Wir wollen im folgenden zu diesem Zweck 
sowohl die Winkelverteilungen als auch die totalen Wirkungsquerschnitte 
zu Rate ziehen.
 
\begin{table}
\label{p1tab}
\caption{Bestimmung der p-Wellen-Amplitude $P_1$ aus den Mainzer Daten. Impulse 
$kq$ in Einheiten $fm^{-2}$, Multipolamplituden in $10^{-3} m_\pi^{-1}$.}
\begin{center}
\begin{tabular}{|c||c|c|c|c|} \hline
$E_{\gamma}$ & $(kq)$ & $P_1^I$        & $P_1^{II}$      & $P_1^f$    \\ \hline
                                                                         \hline
 146.8  & 0.069     & $1.59\pm 0.38$   & $0.25\pm 0.20$  & $1.10\pm 0.07$ \\
 149.1  & 0.101     & $1.69\pm 0.45$   & $0.72\pm 0.29$  & $1.62\pm 0.10$ \\  
 151.4  & 0.127     & $1.97\pm 0.39$   & $1.97\pm 0.39$  & $2.03\pm 0.13$ \\
 153.7  & 0.149     & $2.42\pm 0.21$   & $2.42\pm 0.21$  & $2.39\pm 0.15$ \\
 156.1  & 0.170     & $3.41\pm 0.27$   & $3.41\pm 0.27$  & $2.73\pm 0.17$ \\
 \hline
\end{tabular}
\end{center}
\end{table}
 

Bei der kritischen Energie ist $|E_{0+}|=|P_1|$ und daher auch
$A+B+C=0$. Wir haben in Tabelle 1.3 die Summe der gemessenen 
Winkelverteilungskoeffizienten angegeben. Dabei erkennt man ein deutliches
Minimum bei der Photonenergie $E_\gamma=149.1$ MeV, wo die Summe der
Koeffizienten mit Null vertr\"aglich ist. Wir haben daher diesen Wert
mit der kritischen Energie identifiziert und dabei (etwas willk\"urlich)
den Punkt $E_\gamma=149.1$ MeV noch in den Bereich von L\"osung II genommen.
Das entsprechende Resultat findet sich in der vierten Spalte von Tabelle
1.3 und ist mit $E_{0+}^{II}$ bezeichnet. Die Experimentatoren
schlie\ss en jedoch ein Szenario, in dem die kritische Energie kleiner 
als $146.8$ MeV ist, nicht v\"ollig aus \cite{Bec90}.
In diesem Fall ergeben sich die mit $E_{0+}^{I}$ bezeichneten
L\"osungen. 

Die beiden L\"osungen f\"ur $E_{0+}$ sind in Abbildung 1.3 dargestellt.
W\"ahrend $E_{0+}^{II}$ im Rahmen der Fehler mit dem Niederenergietheorem
direkt an der Schwelle vertr\"aglich erscheint, steht das Szenario $E_{0+}^I$
in deutlichem Widerspruch zu der LET-Vorhersage. Wir wollen nun 
demonstrieren, da\ss\ eine Betrachtung der totalen Wirkungsquerschnitte 
eine klare Pr\"aferenz f\"ur die mit dem Niederenergietheorem
vertr\"agliche L\"osung $E_{0+}^{II}$ ergibt \cite{Ber91,Sch91}.

\begin{figure}
\caption{Ergebnis der Mainzer Multipolanalyse f\"ur die
s- und p-Wellen-Multipole $E_{0+}$ und $P_1$. Gezeigt sind
die Resultate f\"ur die beiden Szenarien I (kein Vorzeichenwechsel) und II
(Vorzeichenwechsel bei $E_\gamma=149.1$ MeV) als Funktion von $(qk)$.}
\vspace{9cm}
\end{figure}
\begin{figure}
\caption{Totaler Wirkungsquerschnitt f\"ur die Reaktion $\Rop$ nach
[Str90]. Aus den Daten wurde der kinematische Faktor $4\pi\frac{q}{k}$
herausskaliert. Die durchgezogene Linie ist der Beitrag der 
p-Wellen-Amplitude $2f_0^2(qk)^2$ f\"ur $f_0=8\cdot 10^{-3}m_\pi^{-3}$.}
\vspace{9cm}
\end{figure}     

Ber\"ucksichtigt man wie oben nur die Beitr\"age der Multipole mit
$l=0$ und $l=1$, so ist der totale Wirkungsquerschnitt durch
\beq
 \sigma_{\rm tot} &=& 4\pi \left( \frac{q}{k} \right) 
 \left( |E_{0+}|^2 + 2|M_{1+}|^2 + |M_{1-}|^2 + 6|E_{1+}|^2 \right)
  \nonumber \\
\label{xfit}  
 & \simeq & 4\pi \left( \frac{q}{k} \right) 
 \left( ({\rm Re} E_{0+})^2 + 2f_0^2 (qk)^2 \right) 
\eeq
gegeben, wobei wir in der zweiten Zeile den Imagin\"arteil von $E_{0+}$
vernachl\"assigt und das Schwellenverhalten der Dipolamplituden 
verwendet haben. Diese Annahme ist zumindest f\"ur die resonante 
$M_{1+}$-Amplitude im Bereich der gemessenen Daten sicher gerechtfertigt.
F\"ur die anderen beiden p-Wellen-Amplituden erwartet man jedoch
Korrekturen an dieser einfachen Parametrisierung. 

Die einfachste M\"oglichkeit zur Bestimmung von $f_0$ besteht darin, 
diesen Parameter mit der $M_{1+}$-Amplitude zu identifizieren. 
Ph\"anomenologische Analysen ergeben den Wert $M_{1+}\simeq 8qk 
\cdot 10^{-3} m_\pi^{-3}$ \cite{NSV74}. Alternativ l\"a\ss t sich
$f_0$ auch direkt aus den gemessenen Daten extrahieren, indem 
man den totalen Wirkungsquerschnitt in der Form
\be
 \frac{1}{4\pi} \left( \frac{k}{q} \right) \sigma_{tot}\simeq
 c_0^2 + 2f_0^2 (qk)^2 
\ee
parametrisiert. Die Ergebnisse zeigen wir in Abbildung 1.4.
Unter Verwendung der Daten im Bereich $E_\gamma=144.7$ MeV bis 
$E_\gamma=156.3$ MeV finden  wir $f_0=7.1\cdot 10^{-3}m_\pi^{-3}$.
F\"ur $c_0=0$ ergibt sich $f_0=8.0\cdot 10^{-3}m_\pi^{-3}$, in 
\"Ubereinstimmung mit dem ph\"anomenologischen Wert der $M_{1+}$-Amplitude. 
Man erkennt, da\ss\ die totalen Wirkungsquerschnitte bei h\"oheren
Energien sehr gut durch den Beitrag der p-Wellen-Amplitude allein
beschrieben werden. Dagegen ergibt sich direkt an der Schwelle eine 
Diskrepanz, die  auf eine nichtverschwindende $E_{0+}$-Amplitude 
hinweist.
 
Benutzt man den Wert $f_0=8\cdot 10^{-3}m_\pi^{-3}$, um mit Hilfe von 
(\ref{xfit}) den Realteil von $E_{0+}$ zu extrahieren, so ergeben sich die 
in der letzten Spalte der Tabelle 1.3 angegebenen Werte. Die zitierten
Fehler ber\"ucksichtigen lediglich den experimentellen Fehler in der
Bestimmung des totalen Wirkungsquerschnitts und beinhalten nicht die 
Unsicherheit im Wert von $f_0$. Die resultierenden Dipolamplituden $E_{0+}^f$
zeigen bei kleinen $E_\gamma$ eine deutliche Pr\"aferenz f\"ur die 
mit dem Niederenergietheorem vertr\"agliche L\"osung $E_{0+}^{II}$. 
Bei der h\"ochsten Energie ist die \"Ubereinstimmung nicht mehr 
so gut. Wir werten diese Tatsache als einen deutlichen Hinweis auf 
Abweichungen von der naiven $qk$-Abh\"angigkeit der p-Wellen-Amplitude. 

Auf Grund der besseren Statistik sind die totalen Wirkungsquerschnitte
in kleineren Energieintervallen ($\Delta E_\gamma \simeq 0.3$ MeV)
bestimmt worden als die Winkelverteilungen. In Tabelle 1.3 haben wir 
jeweils \"uber mehrere Intervalle gemittelt, um die mit Hilfe der 
Formel (\ref{xfit}) bestimmten elektrischen Dipolamplituden  
mit der Multipolanalyse vergleichen zu k\"onnen. Der niedrigste Datenpunkt
liegt in diesem Fall bei $E_\gamma = 146.8$ MeV, also etwa 2 MeV 
oberhalb der Schwelle. N\"aher an die Schwelle heran kommt man, indem
man auf die Mittelwertbildung verzichtet \cite{Ber91}.
Verwendet man den  Datensatz in  Abbildung 1.4, um $E_{0+}^f$ bis an die 
Schwelle zu  extrapolieren, so findet man den Wert $E_{0+}^f =-(2.1\pm 0.2) 
\su$. Dieses Resultat ist im Bereich der experimentellen Unsicherheiten
konsistent mit der Vorhersage des Niederenergietheorems.   

In Tabelle 1.4 vergleichen wir die verschiedenen Resutate 
f\"ur die Amplitude $P_1$. Keine der beiden L\"osungen der
Multipolanalyse ist mit einer einfachen $qk$-Abh\"angigkeit
ver\-tr\"ag\-lich. Die von uns bevorzugte L\"osung $P_1^{II}$ zeigt
eine signifikante p-Wellen-Unterdr\"uckung in der N\"ahe der
Schwelle.

Zusammenfassend stellen wir fest, da\ss\ die Bestimmung der 
$E_{0+}$-Am\-pli\-tu\-de an der Schwel\-le von Mehrdeutigkeiten betroffen 
ist. Sorgf\"altige Analysen zeigen allerdings keine signifikante Verletzung
des Niederenergietheorems direkt an der Schwelle. 
Dagegen findet man aber eine \"uberraschend starke
Energieabh\"angigkeit der $E_{0+}$-Amplitude und eine Abweichung
der p-Wellen-Amplitude vom erwarteten $qk$-Verhalten.
