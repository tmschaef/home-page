\section{Ableitung des Niederenergietheorems}
Die Herleitung von Niederenergietheoremen zur Pionphotoproduktion
verl\"auft analog zu der im letzten Abschnitt geschilderten Ableitung 
der GOR-Relation. Ausgangspunkt ist die Darstellung der Streumatrix mit Hilfe
der LSZ-Reduktionsformel
\beq
\label{LSZ}
 S^{a} &=& -(2\pi)^4 \,\delta^4 (p_1+k-p_2+q)\, Z_\gamma^{-1/2}
   Z_\pi^{-1/2} \\
   & & \mbox{}\cdot \int d^4x\, e^{iq\cdot x} (\Box +m_\pi^2)
   <N(p_2)|T\left(\epsilon^\mu V_\mu^{em}(0) \phi^{a}(x)\right)|N(p_1)>
   \nonumber\; .
\eeq
Dabei bezeichnet $\phi^{a}$ das kanonische Pionfeld, $Z_\pi=(2\pi)^3
2\omega_\pi$ dessen kovariante Normierung und $Z_\gamma=(2\pi)^3
2\omega$ die Normierung des elektromagnetischen Feldes. Die
\"Ubergangsmatrix nach der Definition aus dem ersten Kapitel ist
durch
\be
\label{deft}
 S^{a} = i(2\pi)^4\,\delta^4 (p_1+k-p_2+q) Z_\gamma^{-1/2}
  Z_\pi^{-1/2} \epsilon^\mu T_\mu^{a}
\ee
gegeben. Wir betrachten  die Zweipunktfunktion
\be
\label{Pimunu}
\overline{\Pi}^a_{\mu\nu}(q) = \int d^4 x\, e^{iq\cdot x}<N(p_2)| 
T\left( V_\mu^{em} (0) B_\nu^{a}(x) \right) |N(p_1)> \; .
\ee
des elektromagnetischen Stroms $V_\mu^{em}$ und des 'transversalen` Axialstroms
\be
B_\mu^{a}(x) =A_\mu^{a}(x)+\frac{1}{m_\pi^2}\partial_\mu D^{a}(x)
\ee
wobei $D^{a}(x)=\partial^\mu A_\mu^{a}(x)$ die Divergenz des Axialstroms
bezeichnet. Mit Hilfe der PCAC-Relation und der Definition der
Pionquellfunktion findet man
\be
\label{defb}
\partial^\mu B_\mu^a (x) = (\Box +m_\pi^2)\phi^{a}(x) =-j_\pi^{a}(x)\, .
\ee
Einmaliges Differenzieren des zeitgeordneten Produktes in der 
Zweipunktfunktion $\overline{\Pi}_{\mu\nu}$ liefert schlie\ss lich 
die gesuchte Wardidentit\"at f\"ur $T_\mu^{a}$
\be
\label{avward}
T_\mu^a (q) = \frac{1}{f_\pi}\left\{
iq^\nu \overline{\Pi}_{\mu\nu}^a (q) \, - \, C_\mu^a (q)  \, - \,
\frac{i \omega_\pi}{m_\pi^2} \Sigma^a_\mu (q) \right\} \; .
\ee
Dabei haben wir die LSZ-Formel (\ref{LSZ}) verwendet, um die 
Photoproduktionsamplitude $T_\mu^{a}$ zu identifizieren. Die Wirkung 
der Ableitung auf den Zeitordnungsoperator liefert die Kommutatoren
\beq
\label{cmua}
 C_\mu^{a}(q) &=& \int d^4x\, e^{iq\cdot x}\delta (x^0)
   <N(p_2)|[A^{a}_0(x),V_\mu^{em}(x)]|N(p_1)> \\
\label{sig}     
 \Sigma_\mu^{a}(q) &=& \int d^4x\, e^{iq\cdot x}\delta (x^0)
   <N(p_2)|[D^{a}(x),V_\mu^{em}(x)]|N(p_1)>\;
\eeq
wobei wir das Resultat in den Stromalgebraterm $C_\mu^{a}$
und den Kommutator der Divergenz des Axialstroms zerlegt haben.  
In Analogie zum Sigmaterm in der Pion-Nukleon-Streuung bezeichnet
man diesen Beitrag auch als $(\gamma,\pi)$-Sigmaterm. Wie der
$\pi N$-Sigmaterm liefert er eine zus\"atzliche Korrektur, die direkt
proportional zu den Stromquarkmassen in der QCD-Lagrangedichte ist.

Es ist instruktiv, die Wardidentit\"at zu studieren, 
die sich aus der Zweipunktfunktion $\Pi^{a}_{\mu\nu}$ des
\"ublichen Axialstroms ergibt. Analog zu (\ref{avward})
erh\"alt man
\be
\label{avward2}
\frac{m_\pi^2}{q^2-m_\pi^2} T_\mu^a (q) = \frac{1}{f_\pi}\Big\{
iq^\nu \Pi_{\mu\nu}^a (q) \, - \, C_\mu^a \Big\} \; .
\ee
Auf Grund des Pionpropagators vor der Amplitude $T_\mu^{a}$
liefert diese Beziehung die Photoproduktionsamplitude zun\"achst
nur am unphysikalischen ''weichen`` Punkt $q_\mu=0$.  Um die physikalische 
Schwelle $q^2=m_\pi^2$ zu erreichen, ist es notwendig, den Pionpol
explizit abzuseparieren. Diesem Zweck dient der transversale Axialstrom
$B_\mu^{a}$. Tats\"achlich l\"a\ss t sich $B_\mu^{a}$ mit Hilfe der
PCAC Relation als der nichtpionische Anteil des Axialstroms interpretieren
\be
 B_\mu^{a}(x) = A_\mu^{a}(x) +f_\pi\partial_\mu \phi^{a}(x)
    = A_\mu^{a}(x) - A_{\mu}^{a\, (\pi)} (x) \, .
\ee
Wir wollen nun die verschiedenen Beitr\"age zur Photoproduktionsamplitude
$T_\mu^{a}$ im Einzelnen studieren. Beginnen werden wir dabei mit
der Zweipunktfunktion $q^\nu\overline{\Pi}_{\mu\nu}^{a}$. Da dieser
Term proportional zum Impuls $q$ ist, tragen im Grenzfall weicher Pionen
nur die Polterme in $\overline{\Pi}_{\mu\nu}^{a}$ zur Amplitude bei.
Die Summe der Nukleonpolterme im direkten und im Austauschkanal
lautet
\beq
\label{nborn}
f_\pi T_\mu^{a\,(N)} &=& \bar{u}(p_2) \Big( iq^\nu \Gamma_\nu^{B^{a}}
   (p_1-k,-q,p_2) S_F(p_1+k) \Gamma_\mu^\gamma (p_1,k,p_1+k)
      \\[0.2cm]
   & & \hspace{0.5cm} \mbox{}+ \Gamma_\mu^\gamma (p_1-q,k,p_2)
   S_F(p_1-q) iq^\nu \Gamma_\mu^{B^{a}}(p_1,-q,p_1-q) \Big) u(p_1)
   \; .\nonumber
\eeq      
Dabei bezeichnet $S_F(p)$ den Nukleonpropagator und $\Gamma_\mu^\gamma$
bzw. $\Gamma_\nu^{B^{a}}$ die Vertexfunktionen f\"ur die Kopplung
des Nukleons an das elektromagnetische Feld und den Axialstrom $B_\nu^{a}$.
In den Poltermen sind alle Teilchen mit Ausnahme des
intermedi\"aren Nukleons auf der Massenschale. Die allgemeine Gestalt der
Vertexfunktion lautet in diesem Fall
\beq
\label{emvert}
\Gamma_\mu^\gamma (p_1,k,p_1+k)u(p_1) &=& \left( \gamma_\mu F_1 
   +  \frac{M+(p_1+k)\cdot\gamma}{2M} \frac{i\sigma_\mu\nu k^\nu}{2M}F_2^+ 
   \right. \\
 & & \hspace{1.5cm}\left. \mbox{}
   +  \frac{M-(p_1+k)\cdot\gamma}{2M} \frac{i\sigma_\mu\nu k^\nu}{2M}F_2^-
   \right) u(p_1)  \nonumber \\
\label{bavert}
\Gamma_\nu^{B^{a}} (p_1,-q,p_1-q)u(p_1) &=& \left( \gamma_\nu \overline{G}_A 
   +  \frac{M+(p_1-q)\cdot\gamma}{2M} \frac{q_\nu}{2M} \overline{G}_P^{\, +}
   \right. \\
  & & \hspace{1.5cm}\left. \mbox{}  
   +  \frac{M-(p_1-q)\cdot\gamma}{2M} \frac{q_\nu}{2M} \overline{G}_P^{\, -} 
   \right)\gamma_5\frac{\tau^{a}}{2} u(p_1) \; .\nonumber
\eeq     
Analoge Ausdr\"ucke ergeben sich, wenn der Impuls $p_2$ die 
Massenschalenbedingung erf\"ullt. Die Formfaktoren $F_i$ sind Funktionen
des Impuls\"ubertages $k^2$ und des off-shell Parameters $\delta^2=(p_1+k)^2
-M^2$. Ihre Isospinstruktur lautet
\be
 F_i=F_i^{s}+F_i^{v}\tau^3 \; .
\ee
Entsprechend h\"angen die Formfaktoren am $NNB^{a}$-Vertex von den 
Variablen $q^2$ und ${\delta '}^2=(p_1-q)^2-M^2$ ab. Wir haben diese 
Formfaktoren mit einem Querstrich gekennzeichnet, um sie von den 
entsprechenden Funktionen am  Vertex des Axialvektorstroms zu 
unterscheiden.

Die Abh\"angigkeit der Photoproduktionsamplitude vom off-shell
Verhalten der Formfaktoren wurde in einer Arbeit von Naus, Koch
und Friar \cite{NKF90} untersucht. Die Autoren zeigen, da\ss\ 
off-shell Korrekturen erst in derselben Ordnung in $m_\pi$ 
wie andere modellabh\"angige Korrekturen auftreten. Wir verwenden
daher von nun an die on-shell Vertices
\beq
\Gamma_\mu^\gamma &=& \gamma_\mu F_1 + 
          \frac{i\sigma_{\mu\nu}k^\nu}{2M} F_2 \\
\Gamma_\nu^{B^{a}}&=& \left( \gamma_\mu \overline{G}_A
         + \frac{q_\mu}{2M}\overline{G}_P \right) \gamma_5 
	 \frac{\tau^{a}}{2}
\eeq
wobei die auftretenden Formfaktoren nur mehr Funktionen der
Impuls\"ubertr\"age $k^2$ bzw.~$q^2$ sind. F\"ur reelle
Photonen ist
\be
\begin{array}{rclcrcl}
  F_1^{s}&=& 1/2        &\hspace{1cm}& F_1^{v}&=& 1/2     \\[0.2cm]
  F_2^{s}&=&\kappa^s    &            & F_2^{v}&=&\kappa^v  
\end{array}
\ee
mit den anomalen magnetischen Momenten $\kappa^{s,v}=\frac{1}{2}
(\kappa_p\pm \kappa_n)$. Die Vertexfunktionen f\"ur die beiden 
Str\"ome $A_\nu^{a}$ und $B_\nu^{a}$ unterscheiden sich nur 
um die Matrixelemente des pionischen Beitrags 
$A_\nu^{a(\pi)}=-f_\pi \partial_\nu \phi^{a}$. 
Dieser Term erzeugt den Pionpol im induzierten 
pseudoskalaren Formfaktor
\be
  G_P^{\pi -Pol} (q^2)=\frac{4Mf_\pi}{m_\pi^2-q^2} G_{\pi NN}(q^2)
\ee
wobei $G_{\pi NN}$ den Pion-Nukleon Formfaktor
\be
  <N(p_2)|j_\pi^{a}(0)|N(p_1)> = G_{\pi NN}(t) \bar{u}(p_2)i\gamma_5
         \tau^{a}u(p_1)
\ee
bezeichnet. Der Zusammenhang der Formfaktoren an den Vertices ist 
daher durch $\overline{G}_A=G_A$ und $\overline{G}_P=G_P-G_P^{\pi -Pol}$ 
gegeben. Empirische Untersuchungen zeigen, da\ss\ $G_P$ in sehr guter 
N\"aherung durch den Polterm beschrieben wird. Wir setzen daher 
$\overline{G}_P=0$ und erhalten
\beq
\label{nborn2}
f_\pi T_\mu^{a\,(N)} &=& \bar{u}(p_2) \left\{ g_A iq\cdot\gamma \gamma_5 
 \,\frac{\tau^{a}}{2} \frac{i}{(p_1+k)\cdot\gamma -M} \,\Gamma_\mu^\gamma
 \right. \\
 & & \hspace{1cm}\left. \mbox{} + \Gamma_\mu^\gamma 
     \,\frac{i}{(p_1-q)\cdot\gamma -M}\,
  g_A iq\cdot\gamma \gamma_5 \frac{\tau^{a}}{2} \right\} u(p_1), \nonumber 
\eeq
wobei wir daueber hinaus den axialen Formfaktor $G_A(q^2)$ durch den
Wert beim Impuls\"ubertrag $g_A=G_A(q^2=0)$ ersetzt haben. Die axiale
Kopplung l\"a\ss t sich mit Hilfe der Goldberger-Treiman Relation
\be
\label{GT}
\frac{g_A}{2f_\pi} = \frac{f}{m_\pi}
\ee
durch die pseudovektorielle Pion-Nukleon Kopplungskonstante $f$ ausdr\"ucken.
Das Resultat (\ref{nborn2}) entspricht daher der Born-Approximation f\"ur
eine effektive Pion-Nukleon Lagrangedichte mit dem Kopplungsterm
\be
\label{pv}
{\cal L} = \frac{f}{m_\pi} \bar{\psi}\gamma_5\gamma_\mu \tau^{a}\psi
   \partial^\mu \phi^{a}\; .
\ee    
Die geschilderte Herleitung f\"uhrt also in nat\"urlicher Weise
auf eine pseudovektorielle Kopplung des Pions an das Nukleon. Dies 
steht im Gegensatz zu vielen klassischen Arbeiten, in denen 
gew\"ohnlich mit einer pseudoskalaren Kopplung gerechnet wird.
Um Konsistenz mit der Wardidentit\"at (\ref{avward}) zu erzielen,
m\"ussen in diesem Fall zus\"atzliche Korrekturterme zur Bornamplitude
addiert werden.

Der Beitrag des Kommutators $C_\mu^{a}$ l\"a\ss t sich mit Hilfe 
der Stromalgebraregeln berechnen
\be
\label{curcom}
 C_\mu^{a} = -\epsilon^{a3c} <N(p_2)|A_\mu^{c}(0)|N(p_1)>\; .
\ee
Vernachl\"assigt man den Hintergrundbeitrag im induzierten 
pseudoskalaren Formfaktor, so ergibt sich
\be
\label{kr}
\epsilon^\mu C_\mu^{a} = -\epsilon^{a3c} \bar{u}(p_2)
  \left\{ G_A(t)\epsilon\cdot\gamma + G_{\pi NN}(t) f_\pi  
   \frac{\epsilon\cdot (k-q)}{m_\pi^2-t} \right\}
   \gamma_5 \tau^{c}u(p_1) 
\ee   	 	  
als Funktion der Mandelstamvariable $t=(q-k)^2$.
Das Resultat ist antisymmetrisch in den Isospinindices und
tr\"agt daher nur zur Produktion geladener Pionen bei. 
Der erste Term liefert den f\"uhrenden Beitrag zur 
Pho\-to\-pro\-duk\-ti\-ons\-amplitude im Grenzfall weicher Pionen
\be
\label{krtheo}
\lim_{q,k\to 0} T^{a}(q) =\frac{g_A}{f_\pi}\epsilon^{a3c}
   \bar{u}(p_2) \epsilon\cdot\gamma\gamma_5\tau^{a}u(p_1)\; ,
\ee
den sogenannten Kroll-Ruderman-Term \cite{KR54}. 

Der zweite Kommutator enth\"alt die Divergenz des Axialstroms
und l\"a\ss t sich daher mit Ausnahme der Zeitkomponente 
$\Sigma_0^{a}$ nicht modellunabh\"angig bestimmen. Mit Hilfe
des Stromalgebraresultats
\be
 \,[Q_5^{a},V_\mu^{em}(0)]=-i\epsilon^{a3c}A_\mu^{c}(0)
\ee
und der Erhaltung des elektromagnetischen Stroms findet man
\be
\label{sig0}
  \int d^4x \,\delta (x^0)\, [\partial^\mu A_\mu^{a}(x),V_0^{em}
  (0)] = -i\epsilon^{a3c} \partial^\mu A_\mu^{c} (0) \; .
\ee     
Das Matrixelement der Divergenz des Axialstroms zwischen
Nukleonzust\"anden $|N(p)>$ ist die axialen Formfaktoren des
Nukleons bestimmt:
\be
<N(p_2)| D^{a}(x) |N(p_1)> = \bar{u}(p_2) \left[ M G_A (t)
  + \frac{t}{4M} G_P(t) \right] \gamma_5 \tau^{a} u(p_1) \; .
\ee
Ber\"ucksichtigt man wie oben nur den f\"uhrenden Pionbeitrag,
so l\"a\ss t sich die Summe der beiden Kommutatoren 
an der Schwelle in die Form
\be
 \epsilon^\mu T_\mu^{a\,(\pi)}  = \epsilon^{a3c} g_{\pi NN}
   \,\frac{\epsilon\cdot (k-2q)}{m_\pi^2 -t} \,
   \bar{u}(p_2)\gamma_5 \tau^{a} u(p_1)
\ee
bringen. 
Die Raumkomponenten des Sigmaterms $\Sigma_\mu^{a}$ enthalten die 
Information \"uber die explizite Brechung der chiralen Symmetrie 
durch die Quarkmassen in der QCD-Lagrangedichte. Ihre Bestimmung
ist jedoch modellabh\"angig und wird uns in den n\"achsten 
Abschnitten noch besch\"aftigen. Vernachl\"assigt man diese
Korrektur, so ergeben die oben diskutierten Beitr\"age (\ref{nborn2},
\ref{kr},\ref{sig0}) folgende Bestimmung der invarianten Amplituden       
\beq
\label{letamp}
A^{(+0,-)}_1 &=&  \frac{2f}{\mu} \spm
      \left\{ -\frac{1}{\nu+\nu_1} \mp \frac{1}{\nu-\nu_1} 
      + \frac{1\mp 1}{\nu_1} \right\} \\
A^{(+0,-)}_2 &=&  \frac{2f}{\mu} \spm
      \left\{ -\frac{2}{\nu+\nu_1} \pm \frac{2}{\nu-\nu_1} \right\} \\      
A^{(+0,-)}_3 &=& \; \frac{2f}{\mu} \kappa \; (-1 \pm 1)   \\
A^{(+0,-)}_4 &=& \frac{2f}{\mu}\;\kappa
      \left\{ \frac{2}{\nu+\nu_1} \pm \frac{2}{\nu-\nu_1} \right\} \\ 
A^{(+0,-)}_5 &=&  \frac{2f}{\mu}\;\kappa
      \left\{ \frac{4}{\nu+\nu_1} \mp \frac{4}{\nu-\nu_1} \right\} \\       
A^{(+0,-)}_6 &=&  \frac{2f}{\mu}(1+2\kappa)
      \left\{ \frac{1}{\nu+\nu_1} \mp \frac{1}{\nu-\nu_1} \right\} 
      + \frac{2f}{\mu} \kappa\, (1\pm 1) \; .
\eeq
Dabei haben wir der \"Ubersichtlichkeit halber den Isospinindex
der anomalen magnetischen Momente unterdr\"uckt. Es gilt
$\kappa^{(\pm)}=\kappa^v$ und $\kappa^{(0)}=\kappa^s$. 
Mit Hilfe der im ersten Kapitel abgeleiteten Formel f\"ur die
Schwellenamplitude,
\be
 \left. E_{0+}\right|_{thr} = \frac{e}{16\pi M}
 \frac{2+\mu}{(1+\mu)^{3/2}} \, \left. \left(
   A_3 + \frac{\mu}{2} A_6 \right) \right|_{thr}
\ee                
und der in Anhang A diskutierten Kinematik
ergibt sich schlie\ss lich folgende Bestimmung der elektrischen
Dipolamplitude f\"ur die vier physikalischen Kan\"ale 
\beq
\label{LET}
\Epn &=& \frac{e}{4\pi} \frac{\sqrt{2}f}{m_\pi}
    \left\{ 1 - \frac{3}{2}\mu + {\cal O}(\mu^2) \right\}
    \cong 26.6 \su \\[0.1cm]
\Emp &=& \frac{e}{4\pi} \frac{\sqrt{2}f}{m_\pi}
     \left\{ -1 + \frac{1}{2}\mu + {\cal O}(\mu^2) \right\}
    \cong -31.7 \su \\[0.1cm]
\Eop &=& \frac{e}{4\pi} \frac{f}{m_\pi}
     \left\{ -\mu + \frac{\mu^2}{2}(3+\kappa_p ) +
  {\cal O}(\mu^3) \right\}    \cong -2.32
  \su \\[0.1cm]
\Eon &=& \frac{e}{4\pi} \frac{f}{m_\pi}
     \left\{  \frac{\mu^2}{2}\kappa_n  +
  {\cal O}(\mu^3) \right\}  \cong -0.51 \su 
\eeq
wobei wir die Werte $\kappa_p=1.79$ und $\kappa_n=-1.91$ verwendet 
haben. Dieses Resultat liefert den Inhalt des Niederenergietheorems
\cite{Bae70,VZ72}. Die relative Ordnung der nicht bestimmten
Korrekturen wurde mit Hilfe verschiedener Annahmen \"uber
das Verhalten der Untergrundamplitude festgelegt. Wir werden
diese Annahmen und ihre Rechtfertigung im n\"achsten Abschnitt
diskutieren.

\begin{figure}
\label{feyn}
\caption{Diagrammatische Darstellung der f\"uhrenden Beitr\"age zur
Pionphotoproduktionsamplitude}
\vspace{8.5cm}
\end{figure}

Niederenergietheoreme zur Pionphotoproduktion lassen sich auch
direkt aus der Bestimmung der Bornterme in effektiven 
chiralen Meson-Nukleon Theorien ableiten \cite{Pec69}. Die
entsprechende Lagrangedichte unter Einbeziehung der 
elektromagnetischen Wechselwirkung lautet
\beq
\label{leff}
{\cal L} & =& \bar{\psi}(i\gamma\cdot{\cal D}-M)\psi 
  +\frac{1}{2}({\cal D}_\mu\phi^{a})^2 - \frac{1}{2}m_\pi^2
  (\phi^{a})^2  \\
 & & \mbox{} + \frac{f}{m_\pi} \bar{\psi}\gamma_5 \gamma_\mu
 \tau^{a} {\cal D}^\mu \phi^{a}\psi 
  + \frac{e}{4m}\bar{\psi} (\kappa^s +\kappa^v \tau^3)
  \sigma_{\mu\nu}\psi F^{\mu\nu} \nonumber 
\eeq
wobei ${\cal D}_\mu=\partial_\mu+iQ{\cal A}_\mu$ die kovariante
Ableitung, ${\cal A}_\mu$ das elektromagnetische Potential und
$Q$ den Ladungsoperator bezeichnet. Die Eichung der pseudovektoriellen
Pion-Nukleon-Kopplung erzeugt eine $\gamma\pi NN$-Kontaktwechselwirkung,
\be
{\cal L}_{\gamma\pi NN} = \frac{ef}{m_\pi}\epsilon^{3ab}
  \bar{\psi}\gamma_5 \gamma_\mu \tau^{a}\psi {\cal A}^\mu \phi^b
\ee  
die im Rahmen der effektiven Theorie die Kroll-Ruderman-Amplitude
liefert. Die Wirkung der kovarianten Ableitung auf das Pionfeld
bestimmt die Kopplung des elektromagnetischen Feldes an die
geladenen Pionen. Die $\gamma\pi\pi$-Wechselwirkung 
\be  
{\cal L}_{\gamma\pi\pi} = e\epsilon^{3ab}\phi^{a}\partial_\mu
 \phi^{b} {\cal A}^\mu
\ee
liefert schlie\ss lich den Pionpol in der Photoproduktionsamplitude
f\"ur geladene Pionen. 
   
    
\section{Absch\"atzung der vernachl\"assigten Beitr\"age}
Um die Modellabh\"angigkeit des im letzten Abschnitt
vorgestellten Niederenergietheorems zu studieren, ist es
hilfreich, die \"Ubergangsmatrix in der Form 
\be
 T_\mu^{a} = T_\mu^{a({\rm LET})} + \delta T_\mu^{a}
\ee
zu zerlegen. Dabei bezeichnet $T_\mu^{a(\rm LET)}$ die 
T-Matrix, die zu den invarianten Amplituden  (\ref{letamp})
geh\"ort, und $\delta T_\mu^{a}$ die vernachl\"assigte 
Hintergrundamplitude. Nach dem Kroll-Ruderman Theorem gilt
\be
  \lim_{q,k\to 0} \delta T_\mu^{a} =0 \, ,
\ee
so da\ss\ $\delta T_\mu^{a}$ am ''weichen`` Punkt $q_\mu=0$ verschwindet.
Um zu untersuchen, in welcher Ordnung in der Pionmasse $\delta T_\mu^{a}$ 
Korrekturen zur elektrischen Dipolamplitude an der physikalischen Schwelle
liefert, definieren wir
\be
  \delta T_\mu^{a} = \bar{u}(p_2) \sum_{\lambda} 
   \delta A_\lambda^{a}(\nu,\nu_1) {\cal M}_\lambda u(p_1)
\ee
und nehmen an, da\ss\ sich die die invarianten Amplituden 
$\delta A_\lambda (\nu,\nu_1)$ in eine Taylorreihe um den Punkt
$\nu=\nu_1=0$ entwickeln lassen:
\be
 \delta A_\lambda^{a} (\nu,\nu_1) = a^{a}_{\lambda\, 00}
    + a^{a}_{\lambda\, 10} \nu + a^{a}_{\lambda\, 01}\nu_1
    + \ldots
\ee
Diese Annahme ist gerechtfertigt, da lediglich die Pion- und 
Nukleonpolterme Singularit\"aten bei $\nu=0$ oder $\nu_1=0$ 
enthalten. Diese Terme haben wir aber explizit in $T_\mu^{a(
{\rm LET})}$ ber\"ucksichtigt. Dar\"uber hinaus wollen wir 
in den folgenden Betrachtungen voraussetzen, da\ss\ alle
Koeffizienten $a^{a}_{\lambda\, ij}$ im Limes $m_\pi\to 0$
regul\"ar sind. Das bedeutet, da\ss\ sich diese Koeffizienten
beim Abz\"ahlen von Potenzen in $\mu$ als Gr\"o\ss en der
Ordnung ${\cal O}(1)$ betrachten lassen. 

Diese Annahme ist vermutlich unzutreffend, denn Pion-Schleifendiagramme 
k\"onnen Beitr\"age liefern, die nichtanalytisch in $m_\pi$ sind 
\cite{LP71,PP71}. Die Gegenwart solcher Terme ist von Bernard et al.~durch 
eine explizite Rechnung im Rahmen der chiralen St\"orungstheorie best\"atigt
worden\footnote{Dagegen bestreitet Naus \cite{Nau91} auf Grund von
\"Uberlegungen allgemeiner Natur die Existenz 
nichtanalytischer Terme in der Pionphotoproduktionaamplitude}.
Trotzdem ist es von Interesse, die Gr\"o\ss enordnung der analytischen
Beitr\"age in $\delta T_\mu^{a}$ zu studieren. 
 
Die Amplitude $T_\mu^{a(\rm LET)}$ enth\"alt im isospinsymmetrischen 
Fall neben den Nukleon- und Pion-Polen auch einen Kontaktterm. 
Die Gegenwart dieses Terms unterscheidet die Bornamplituden in
pseudovektorieller bzw.~pseudoskalarer Kopplung und ist deshalb
eine Konsequenz der PCAC-Relation. Dieses Resultat l\"a\ss t sich
als eine Bedingung f\"ur die invariante Amplitude $A_6^{(+0)}$
formulieren \cite{AG66}
\be
\label{FFR}
 \lim_{\nu\to 0} \lim_{\nu_1\to 0} A_6^{(+0)} (\nu,\nu_1)
   =  \frac{4f}{\mu} \kappa^{v,s} \; .
\ee
Die Reihenfolge der beiden Grenz\"uberg\"ange in (\ref{FFR})
ist nicht beliebig. Sie ist so gew\"ahlt, da\ss\ der Polterm
keinen Beitrag zum Grenzwert liefert. 
Da der Kontaktterm (\ref{FFR}) bereits in $T_\mu^{a(\rm LET)}$
enthalten ist, verschwindet $a^{(+0)}_{6\,00}$ im Grenzfall $q_\mu
\to 0$. De Baenst \cite{Bae70} verwendet daher in seiner
Diskussion der Amplitude $\delta T_\mu^{a}$ die zus\"atzliche
Annahme $a^{(+0)}_{6\,00}=0$. 
          
Nur $\delta A_3$ und $\delta A_6$ tragen zur elektrischen 
Dipolamplitude an der Schwelle bei. Mit Hilfe der 
Eichinvarianzbedingung (\ref{gaugecond}) und der Forderung nach korrektem
Verhalten der Amplituden unter der Austauschtransformation
$(\nu,\nu_1)\to(-\nu,\nu_1)$ l\"a\ss t sich die m\"ogliche
Form der Taylorentwicklungen f\"ur $\delta A_{3,6}$ erheblich
einschr\"anken. F\"ur die isospinsymmetrischen Komponenten
findet man
\beq
 \delta A_{3}^{(+0)} &=& a_{3\, 11}^{(+0)} \nu\nu_1 + \ldots \\
 \delta A_{6}^{(+0)} &=& a_{6\, 01}^{(+0)} \nu_1
               + a_{6\, 20}^{(+0)} \nu^2
	       + a_{6\, 02}^{(+0)} \nu_1^2 + \ldots \; .
\eeq
An der Schwelle ist $\nu={\cal O}(\mu)$ und $\nu_1={\cal O}(\mu^2)$,
so da\ss\ die Austauschsymmetrie im wesentlichen das 
Transformationsverhalten der Amplitude unter $m_\pi\to -m_\pi$
spezifiziert. Auf Grund der Beziehung 
\be
\delta E_{0+} \sim \delta A_3 + \frac{\mu}{2} \delta A_6
\ee
folgt, da\ss\ die Hintergrundamplitude $\delta E_{0+}^{(+0)}$
an der Schwelle von der Ordnung $\mu^3$ ist. Verwendet man
an Stelle der Annahme $a_{6\,00}^{(+0)}=0$ die Absch\"atzung 
$a_{6\,00}^{(+0)}={\cal O}(\mu)$, so ergibt sich das schw\"achere 
Resultat $\delta E_{0+}^{(+0)}={\cal O}(\mu^2)$. Eine analoge
Argumentation l\"a\ss t sich auch f\"ur die isospinungeraden Komponenten 
durchf\"uhren. In diesem Fall findet man $\delta E_{0+}^{(-)}= 
{\cal O}(\mu^2)$.

\section{Die Methode von Furlan, Paver und Verzegnassi}
Die Ableitung des Niederenergietheorems im  Abschnitt 2.2
basierte im wesentlichen auf der Reduktionsformel und 
Wardidentit\"aten f\"ur die Zweipunktfunktion $\overline{\Pi}_{\mu\nu}^{a}$.
In diesem Abschnitt wollen wir auf eine andere Methode eingehen, die
direkt mit Stromalgebrakommutatoren und Vollst\"andigkeitsrelationen
arbeitet. Im Rahmen dieses Verfahrens wurde erstmals darauf hingewiesen, 
da\ss\ die explizite Brechung der chiralen Symmetrie Korrekturen an 
das Standard-Niederenergietheorem liefern kann \cite{FPV74,NS89}.

Allerdings wird in der \"ublichen Diskussion der Methode nur der
Nukleonbeitrag in der Vollst\"andigkeitssumme explizit ber\"ucksichtigt.
In diesem Fall fehlt der f\"uhrende Beitrag zur Photoproduktion 
neutraler Pionen und man mu\ss\ nachtr\"aglich Eichinvarianz 
erzwingen, um die korrekte Schwellenamplitude zu reproduzieren.
 
In diesem Abschnitt wollen wir demonstrieren, da\ss\ unter Einbeziehung
der Beitr\"age von Antinukleonen im Zwischenzustand auch das Verfahren
von Furlan et al.~die komplette Schwellenamplitude liefert. Zu diesem
Zweck betrachten wir die zu dem Strom $B_\mu^{a}$ geh\"orende Ladung
\be
 \overline{Q}^{a}_5(t) = Q^{a}_5(t) +  \frac{1}{m_\pi^2}\,
 \frac{d}{dt} \, \int d^3x\, D^{a} (\vec{x},t)\, .
\ee
Zwischen physikalischen Zust\"anden reduziert sich dieser Operator
auf die Ladung $\qfl$,
\beq
\label{q5l}
 \qfl (t) &=& Q_5^{a}(t) +\frac{i}{m_\pi}\dot{Q}^{a}_5(t)\, , \\
\label{q5r} 
 \qfr (t) &=& \left( \qfl (t)\right)^\dagger 
                =  Q_5^{a}(t) -\frac{i}{m_\pi}\dot{Q}^{a}_5(t)  \, .
\eeq
Den zugeh\"origen hermitesch konjugierten Operator haben wir mit
$\qfr$ bezeichnet. Die  Pionmatrixelemente dieser Operatoren lauten:  
\beq
  <0|\qfl |\pi^{b}(q)> &=& \spm 2if_\pi m_\pi \,\delta^{ab}
                           (2\pi)^3  \delta^3 (\vec{q}\,)\, , \\[0.2cm]  
  <\pi^{b}(q)|\qfr |0> &=& -2if_\pi m_\pi \,\delta^{ab}
                            (2\pi)^3 \delta^3 (\vec{q}\,)\, , \\[0.2cm]
 <\pi^{b}(q)|\qfl |0>&=& <0|\qfr |\pi^{b}(q)> = 0 \; .
\eeq
Die axialen Ladungen $Q_{5\,{\mini L,R}}^{a}$ sind nicht hermitesch
und haben die Eigenschaft, zwischen Pionen im Eingangs- und Ausgangskanal 
zu unterscheiden. Diese Tatsache erweist sich als besonders n\"utzlich 
bei der  Konstruktion von Summenregeln, da sie es erm\"oglicht,
bestimmte Prozesse in der Vollst\"andigkeitssumme zu selektieren. 
Im folgenden betrachten wir Summenregeln f\"ur das Matrixelement 
\be
 M_\mu^{a} = <N(p_2)| [\qfl ,V_\mu^{em}(0)]|N(p_1)>\, ,
\ee
in dem sich mit Hilfe der oben angegebenen Matrixelemente die 
Photoproduktionsamplitude identifizieren l\"a\ss t. Das Resultat
besitzt eine sehr \"ubersichtliche Struktur als Summe von Poltermen 
und einem Dispersionsintegral, das die Hintergrundamplitude 
repr\"asentiert. 

Da der Operator $\qfl$ nur Pionen in Ruhe produziert, 
verlangt die Berechnung der Schwellenamplitude die Kenntnis
von $M_\mu^{a}$ im Schwerpunktsystem. Um die folgende Rechnung
etwas zu vereinfachen, werden wir $M_\mu^{a}$ statt dessen an
der Breitschwelle berechnen, das hei\ss t f\"ur Pionen, die
im Breitsystem des Nukleons ruhen\footnote{An der Schwelle ruht
das Pion im Schwerpunktsystem. Der Impuls des Pions im Breitsystem 
des Nukleons dann tats\"achlich sehr klein, $|\vec{q}|=m_\pi^2
+{\cal O}(m_\pi^3)$.}:
\be
\begin{array}{rclcrcl}
  \vec{p}_2 &=&\spm \vec{p}, &\hspace{1cm} & \vec{q} &=& 0, \\[0.2cm]
  \vec{p}_1 &=&-\vec{p}    , &\hspace{1cm} & \vec{k} &=& 2\vec{p}.
\end{array}
\ee
Die eine Seite der Summenregel f\"ur $M_\mu^{a}$ ergibt sich, indem 
man das Matrixelement
\beq
\label{comqfl}
\lefteqn{<\!N(\vec p\,)|[\qfl,V_\mu^{em}(0)] |N(-\vec p\,)\!>= } \\
&\hspace{0.5cm} & <\! N(\vec p\,)|[Q_5^{a},V_\mu^{em}(0)]|N(-\vec p)\!>
 +\frac{i}{m_\pi}<\!N(\vec p\,)| [\dot{Q}_5^{a},V_\mu^{em}(0)]
 |N(-\vec p\,)\!> \nonumber 
\eeq
direkt ausgwertet. Dabei findet man den  Kroll-Ruderman Term sowie
den bereits diskutierten Beitrag der expliziten Symmetriebrechung. Die 
andere Seite der Summenregel ergibt sich aus der Vollst\"andigkeitsrelation 
f\"ur den Kommutator (\ref{comqfl}). Die Clusterzerlegung \cite{AFF73} 
ist eine systematische Methode, um die verschiedenen Beitr\"age zur 
Vollst\"andigkeitssumme
\beq
\sum_n <N(\vec{p}\,)|\qfl |n><n|V_\mu^{em}|N(-\vec{p}\,)>
\eeq
zu identifizieren. Sie tr\"agt der Tatsache Rechnung, da\ss\
in einer relativistischen Theorie Beitr\"age mit unterschiedlichen 
Teilchenzahlen auftreten k\"onnen. Konkret zerlegt man $M_\mu^{a}$ 
in der Form    
\begin{figure}
\label{diag}
\caption{Beitr\"age zur Vollst\"andigkeitssumme f\"ur das
Operatorprodukt $V_\mu^{em}\qfl$.}
\vspace{7.5cm}
\end{figure}
\beq
\label{cluster}
M_\mu^{a\;\;}  &=& M_\mu^{a\,I}+M_\mu^{a\,II}  \\
M_\mu^{a\,I\,} &=& \sum_\alpha <N(\vec{p})|\qfl |\alpha>_c
                             <\alpha|V_\mu^{em}|N(-\vec{p}\,)>_c \\   
 & &        \hspace{0.5cm} -  \sum_\beta <0|\qfl |N(-\vec{p}\,)\beta>
              <N(\vec{p}\,)\beta|V_\mu^{em}|0>\; +\; {\em c.~t.} \nonumber \\
M_\mu^{a\,II} &=& \sum_{\gamma_1} <N(\vec{p}\,)|\qfl |N(-\vec{p}\,)\gamma_1>_c
                             <\gamma_1|V_\mu^{em}|0> \\   
 & &       \hspace{0.5cm} +  \sum_{\gamma_2} <0|\qfl |\gamma_2>
       <\gamma_2 N(\vec{p}\,)|V_\mu^{em}|N(-\vec{p}\,)>\; +\; 
       {\em c.~t.} \nonumber
\eeq
wobei der Index $c$  den zusammenh\"angenden Teil des Matrixelements
und $c.t.$  die Voll\-st\"an\-dig\-keitssumme mit den Operatoren in der
anderen Reihenfolge bezeichnet. Der erste Teil der Clusterzerlegung
beinhaltet solche Zust\"ande, die Baryonenzahl tragen. Die
f\"uhrenden Terme in diesem Beitrag stammen von Nukleonen
$|\alpha>=|N(\vec{p}\,)>$ und Antinukleonen $|\beta>=|\bar{N}
(\vec{p}\,)>$. Der zweite Teil von (\ref{cluster}) beschreibt die Produktion
eines Zustands $\gamma_{1,2}$ aus dem Vakuum, gefolgt von der Reaktion
$\gamma_1 +N(p_1) \to \qfl + N(p_2)$ bzw.~ $V_\mu^{em}+N(p_1)
\to \gamma_2 + N(p_2)$. Insbesondere findet man f\"ur Pionzust\"ande
$|\gamma_2>=|\pi^{a}(\vec{q}\,)>$ die Photoproduktionsamplitude
\be
\sum_{\pi^{b}(\vec{q})} <0|\qfl |\pi^{b}(\vec{q}\,)><\pi^{b}(\vec{q}\,)
  N(\vec{p}\,)|V_\mu^{em}|N(-\vec{p}\,)> = f_\pi T_\mu^{a}(\vec{q}=0)\, .
\ee
Auf Grund der speziellen Eigenschaften des Operators  $\qfl$
enth\"alt die Vollst\"andigkeitssumme keine Beitr\"age von der
inversen Reaktion $\pi^{a}(q)+N(p_1)\to V_\mu^{em}+N(p_2)$. Isoliert
man die Photoproduktionsamplitude $T_\mu^{a}$ und separiert 
die Nukleonbeitr"age in $M_{\mu}^{a\, I}$, so ergibt sich 
schlie\ss lich folgender Ausdruck f\"ur $T_\mu^{a}$
\beq
\label{fpv}
f_{\pi} T_{\mu}^{\alpha}(\vec{q}=0) &=&
 i \epsilon^{\alpha 3 \gamma} <N(\vec{p}\,)|A_{\mu}^{\gamma}|N(-\vec{p}\,)> 
                   \\[0.3cm]
   & &\mbox{}-\sum_{N(\vec{p}_n)} <N(\vec{p}\,)|\qfl |N(\vec{p}_n)>
   <N(\vec{p}_n)|V_{\mu}^{em}|N(-\vec{p}\,)> 
           \;+ \; {\em c.~t.} \nonumber \\
   & &\mbox{}  + \frac{i}{m_\pi}<N(\vec{p}\,)|[\dot{Q}_5^{\alpha},V_{\mu}^{em}]
    |N(-\vec{p}\,)> 
    \; + \;  f_\pi \delta T_\mu^{a}  \nonumber ,
\eeq
wobei $\delta T_\mu^{a}$ die vernachl\"assigten Beitr\"age in der
Vollst\"andigkeitssumme bezeichnet. Dabei handelt es sich vor allem um
$\pi N$-Kontinuumszust\"ande und Antinukleonen in $M_\mu^{a\, I}$ 
sowie Vektormesonen in $M_\mu^{a\, II}$.
Die beiden Kommutatoren in (\ref{fpv}) liefern wie in (\ref{avward})
den Kroll-Ruderman Term und die Korrekturen auf Grund der expliziten
Symmetriebrechung.
Mit Hilfe der Eigenschaften des Operators $\qfl$ l\"a\ss t sich folgende
Darstellung der Hintergrundamplitude ableiten \cite{AFF73}
\be
\label{ressum}
\delta T_\mu^{a} = -im_\pi \sum_{n\neq\pi,N} (2\pi)^3 \delta^3 
  (\vec{p}-\vec{p}_n) 
  \frac{ <N(\vec{p}\,)|j_\pi^{a}|n><n|V_\mu^{em}|N(-\vec{p}\,)> }{ 
       (E_p-E_n)(E_p+m_\pi-E_n+i\epsilon) }
  \;-\; c.t.
\ee
Die Summation l\"auft \"uber beliebige intermedi\"are Zust\"ande 
mit Ausnahme von Nukleonen und Pionen. $E_p=(\vec{p}^{\,2}+M^2)^{1/2}$
bezeichnet die Energie des auslaufenden Nukleons, $E_n$ die Energie
des Zwischenzustands. Der Energienenner in (\ref{ressum}) verschwindet
nur f\"ur Nukleonzust\"ande, so da\ss\ alle anderen Beitr\"age im Limes 
$m_\pi \to 0$ unterdr\"uckt sind. 

Wir betrachten nun im Einzelnen die verschiedenen Beitr\"age zur
Photoproduktionsamplitude (\ref{fpv}). Den Kroll-Ruderman Term 
und den Sigmakommutator haben wir bereits im Abschnitt 2.2 
diskutiert. Um den Nukleonterm zu berechnen, ben\"otigen wir die
Matrixelemente 
\beq
  <N(\vec{p}\,)|A_0^{a}(0)|N(\vec{p}\,)> &=&
     \frac{g_A}{M} \,\chi^\dagger_f (\vec{\sigma}\cdot\vec{p}\,)
     \frac{\tau^{a}}{2} \chi_i\, ,  \\  
 <N(\vec{p}\,)|V_0^{em}(0)|N(-\vec{p}\,)> &=&
     e \,\chi^\dagger_f (G_E^s (t) + \tau^3 G_E^v (t) ) \chi_i \, ,\\[0.1cm]
 <N(\vec{p}\,)|\vec{V}^{em}(0)|N(-\vec{p}\,)> &=&
     \frac{e}{M} \,\chi^\dagger_f (G_M^s(t) + \tau^3 G_M^v(t))  
     i(\vec{\sigma}\times\vec{p}) \chi_i \, .
\eeq     
Im Breitsystem treten die elektrischen und magnetischen Formfaktoren
des Nukleons,
\beq
  G_E(t) &=& F_1(t)+\frac{t}{4M^2}F_2(t) \, , \\[0.1cm]
  G_M(t) &=& F_1(t)+F_2(t)\, ,
\eeq
auf. Da wir bereits ein spezielles Bezugssystem gew\"ahlt haben,
ist es sinnvoll, die $T$-Matrix in einer nicht kovarianten Form
anzugeben. Der Nukleonbeitrag zur Photoproduktionsamplitude an  
der Breitschwelle $t=m_\pi^2$ ergibt sich schlie\ss lich zu
\beq
   T_0^{a}  &=& -\frac{g_A}{f_\pi}\,\frac{1}{E_p}
        \chi^\dagger_f (G_E^s(t) \tau^{a} + G_E^v \delta^{a3})
	(\vec{\sigma}\cdot\vec{p}\,)\chi_i \, , \\
\vec{T}^{a} &=& \spm \frac{g_A}{f_\pi} \, \frac{\vec{p}^{\, 2}}{mE_p}
       G_M^v(t) \,\chi^\dagger_f \frac{1}{4} [\tau^{a},\tau^3] 
       \,\vec{\sigma}_{\mini T}\, \chi_i\, ,
\eeq
wobei wir die transversalen und longitudinalen Spins 
\beq
   \vec{\sigma}_{\mini T} &=& \vec{\sigma} - \hat{p}(\vec{\sigma}
              \cdot\hat{p}) \\
   \vec{\sigma}_{\mini L} &=&  \hat{p}(\vec{\sigma}\cdot\hat{p})	      
\eeq
eingef\"uhrt haben. Nur der transversale Anteil liefert einen Beitrag
zur Photoproduktion  mit reellen Photonen. Im Falle des Nukleonterms
ist dieser Beitrag von der Ordnung $m_\pi^2$ und proportional zum
magnetischen Moment des Nukleons.   	             
      
Als pseudoskalares Teilchen koppelt das Pion stark an die unteren
Komponenten der Nukleonspinoren. Der f\"uhrende Beitrag zur Produktion
neutraler Pionen kommt daher von Zwischenzust\"anden, die propagierende 
Antinukleonen enthalten ('Z-Graphen`). Mit Hilfe der Darstellung
(\ref{ressum}) findet man
\be
 \vec{T}^{a} = -\frac{g_A}{f_\pi} \frac{m_\pi}{2E_p} \,
     \chi^\dagger_f (G_E^s(t)\tau^{a} + G_E^v(t) \delta^{a3})
     \vec{\sigma} \chi_i
\ee
wobei wir h\"ohere Ordnungen in $m_\pi/M$ vernachl\"assigt haben.
Damit ist die elektrische Dipolamplitude bis zur Ordnung $m_\pi^2$
bestimmt. Vernachl\"assigt man den Beitrag aus der expliziten
chiralen Symmetriebrechung, so ergibt sich
\beq
\label{LET2}
\Epn &=& \frac{e}{4\pi} \frac{g_A}{\sqrt{2}f_\pi}
    \left\{ 1 - \frac{3}{2}\mu + {\cal O}(\mu^2) \right\}
    \cong 24.1  \su \\
\Emp &=& \frac{e}{4\pi} \frac{g_A}{\sqrt{2}f_\pi}
     \left\{ -1 + \frac{1}{2}\mu + {\cal O}(\mu^2) \right\}
    \cong -29.6  \su \\
\Eop &=& \frac{e}{4\pi} \,\frac{g_A}{2f_\pi}\;
     \bigg\{ -\mu + {\cal O}(\mu^2) \bigg\}  \cong -3.3 \su
\eeq
Die elektrische Dipolamplitude f\"ur die Produktion neutraler
Pionen an Neutron verschwindet in dieser Ordnung. Die von
(\ref{LET}) abweichenden Werte in den geladenen Kan\"alen 
sind eine Konsequenz der Tatsache, da\ss\ die 
Goldberger-Treiman-Relation $\frac{g_A}{2f_\pi}=\frac{f}{m_\pi}$
experimentell um ca.~6\% verletzt ist. Diese Abweichung ist 
formal von der Ordnung $m_\pi^2$ und entspricht daher der 
oben vorgenommenen Absch\"atzung. 
