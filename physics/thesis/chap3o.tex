\chapter{Etaphotoproduktion}
Wir haben im letzten Kapitel verifiziert, da\ss\ Resonanzen im
s- oder t-Kanal nur geringe Beitr\"age zur Pion-Photoproduktionsamplitude
an der Schwelle liefern, und damit die in der Herleitung der
Niederenergietheoreme gemachten Annahmen best\"atigt. Zum
Vergleich wollen wir im Folgenden die Photoproduktion von 
$\eta$-Mesonen betrachten. In dieser Reaktion betr\"agt die 
Schwellenenergie $\omega^{lab}_{th}=709$ MeV, so da\ss\ die
Energie des Photons vergleichbar ist mit der Anregungsenergie der
ersten Resonanz in der $E_{0+}$ Amplitude. Aus diesem Grund 
kann man  im engeren Sinne keine Niederenergietheoreme zur 
$\eta$-Photoproduktion formulieren.

Trotzdem kann man die im letzten Kapitel entwickelten Methoden auch auf
die Photoproduktion von Etamesonen anwenden. Dabei lassen sich wertvolle
Informationen \"uber die Wechselwirkung von Etas mit Nukleonen
und Nukleonresonanzen, \"uber die $\eta-\eta'$ Mischung 
sowie die Struktur der expliziten Symmetriebrechung im Oktett-Kanal
gewinnen. 

Die wichtigsten Beitr\"age zur Etaphotoprodukion stammen vom 
Nukleon-Bornterm, der $N(1535)$-Resonanz sowie dem Austausch
von Vektormesonen. Wir beschreiben diese Terme mit Hilfe der
Lagrangedichte
\beq
 {\cal L}_{\eta NN} &=& \frac{f_{\eta NN}}{m_\eta} 
     \bar{\psi} \gamma_5 \gamma_\mu \psi \partial^\mu \eta \\
 {\cal L}_{\eta NN^{*}} &=& \frac{f_{\eta NN^{*}}}{m_\eta} 
     \bar{\psi}_{N^{*}} \gamma_\mu \psi \partial^\mu \eta + h.c. \\
 {\cal L}_{{\mini V}\eta\gamma} &=& \frac{g_{{\mini V}\eta\gamma}}{2m_\eta}
                 \epsilon_{\alpha\beta\gamma\delta}
		  F^{\alpha\beta} \partial^\gamma
		  V^\delta \eta  
\eeq
wobei $V$ f\"ur die Vektormesonen $\rho,\omega$ steht. \"Uber
die St\"arke der $\eta NN$-Wechselwirkung gibt es nur sehr
widerspr\"uchliche Informationen \cite{Dum82}. Eine einfache
Absch\"atzung gewinnt man mit Hilfe der $SU(3)$-Relation zwischen
den pseudoskalaren Kopplungen
\be
\label{octcoup}
  g_{\eta_8} = \frac{1}{\sqrt{3}}\, \frac{3F/D-1}{F/D+1} \, g_\pi \; .
\ee      
Dabei bezeichnen $D,F$ die symmetrischen bzw.~antisymmetrischen 
$SU(3)$-Kopplungen. Unter Verwendung von $SU(6)$ Spin-Flavor
Wellenfunktionen gilt dar\"uber hinaus $g_{\eta_8} = 
\frac{\sqrt{3}}{5}g_\pi$ und $g_{\eta_0} = \frac{\sqrt{6}}{5}g_\pi$.
Die Kopplungen der physikalische  Zust\"ande sind recht sensitiv
auf die Gr\"o\ss e der $\eta-\eta'$ Mischung. Mit Hilfe von 
quadratischen Massenformeln findet man einen Mischungswinkel
$\theta_{\eta\eta'} =-11^\circ$. Eine Analyse der Zerf\"alle 
$\eta\to 2\gamma$ und $\eta'\to 2\gamma$ liefert dagegen Winkel 
in dem Bereich $\theta_{\eta\eta'}=-(20^\circ -25^\circ)$.

Der Einfachheit halber verwenden wir deshalb eine reines Okttet-Eta mit 
$f_{\eta NN}=1.36$. Der Nukleon-Bornterm liefert dann
\beq
 E_{0+}^N (\eta p) &=& \frac{e}{4\pi} \frac{f_\eta}{m_\eta}
   \frac{\mu_\eta}{(1+\mu_\eta)^{3/2}}\, \left( 1-\frac{\mu_\eta}{2}
   \kappa_p \right) \\[0.2cm]
   &\simeq& \mbox{} -1.1 \su , \nonumber 
\eeq     
wobei $\mu_\eta=m_\eta/M$ das Verh\"altnis der Eta- zur Nukleonmasse
bezeichnet. Im Gegensatz zum Niederenergietheorem zur 
Pionphotoproduktion ist es hier wenig sinnvoll, den kinematischen 
Faktor in Potenzen von $\mu_\eta$ zu entwickeln. 
 
Die $\eta NN^{*}$-Kopplung l\"a\ss t sich analog zu Gleichung (\ref{rescoup})
aus der partiellen Breite $\Gamma (N(1535)\to N\eta)=75$ MeV
zu $f_{\eta NN^{*}}=1.94$ bestimmen. Bei der Photoproduktion von 
Etamesonen mu\ss\ man selbst an der Schwelle die endliche Breite der
Resonanz auf Grund des offenen Zerfallskanals $N^{*}\to N\pi$
ber\"ucksichtigen. Zu diesem Zweck  parametrisieren wir die 
Energieabh\"angigkeit der totalen Breite in der Form
\be
 \Gamma (s) = \Gamma_\pi \left(\frac{q_\pi}{q_\pi^R} \right)
   + \Gamma_\eta \left( \frac{q_\eta}{q_\eta^R} \right) 
\ee
wobei $q_\pi,q_\eta$ die Impulse der Mesonen im Schwerpunktsystem
und $q_\pi^R,q_\eta^R$ die entsprechenden Werte bei $\sqrt{s}=M_R$
bezeichnen. An der $\eta N$-Schwelle ist dann $\Gamma (s_{th})=59$
MeV. Mit Hilfe des im letzten Abschnitt bestimmten anomalen magnetischen
Moments f\"ur den \"Ubergang $\gamma N \to N^{*}$ finden wir
f\"ur den Resonanzbeitrag
\be
  {\rm Re} E_{0+}^{N^{*}} (\eta p) = 11.6  \su
\ee
\begin{table}
\caption{Vergleich der verschiedenen Beitr\"age zur elektrischen
Dipolamplitude f\"ur die Reaktionen $\gamma p \to \pi^0 p$ und 
$\gamma p \to \eta p$. $E_{0+}$ in Einheiten $10^{-3} m_\pi^{-1}$.}
\begin{center}
\begin{tabular}{|l||r|r|}\hline
               & $\gamma p\to \pi^0 p$  &  $\gamma p\to \eta p$ \\ \hline\hline
 Nukleon Born  & $ -2.30$                &   $ -1.1$              \\
 Vektor Mesonen ($\rho,\omega$) & $0.02$  &  3.8	         \\
 Resonanz $N(1535)$             & $0.04$  &  12.4                \\
 Total                          & $-2.24$ &                      \\ \hline
\end{tabular}
\end{center}    
\end{table}
Auch Vektormesonen spielen in der Etaproduktion eine deutlich
gr\"o\ss ere Rolle, als dies in der Pionproduktion der Fall ist.
Die Vektormeson-Nukleon Kopplungskonstanten haben wir bereits im
letzten Kapitel diskutiert. Am $VNN$-Vertex verwenden wir einen 
Formfaktor der Monopolgestalt
\be
 F(t) = \left(\frac{\Lambda^2-m_V^2}{\Lambda^2-t}\right)
\ee
mit dem cutoff $\Lambda=1.4$ GeV. Die $V\eta\gamma$-Kopplungen lassen sich
aus den experimentell bestimmten Zerfallsbreiten $\Gamma (V\to\eta
\gamma )$ ermitteln. Mit den Werten  aus \cite{Dum82} finden wir  
\be
 f_{\rho NN}g_{\rho\eta\gamma}(1+\kappa_\rho)
  +f_{\omega NN}g_{\omega\eta\gamma} \simeq 20.63 \; .
\ee     
Da die $\rho$ und $\omega$ Massen praktisch entartet sind, bestimmt
diese effektive Kopplung den Vektormesonbeitrag zur Etaproduktion
an der Schwelle
\beq
 E_{0+}^{V}(\eta p) &=& \sum_V\frac{eM}{16\pi}\frac{\mu_\eta^2(2+\mu_\eta)}
   {(1+\mu_\eta)^{3/2}} \frac{f_{VNN}g_{V\eta\gamma}(1+\kappa_V)}
   {m_\eta^2 +m_V^2(1+\mu_\eta)}  \\[0.2cm]
   &\simeq& 3.58 \su \; . \nonumber
\eeq
Es ist interessant, \"uber die Auswirkungen expliziter chiraler
Symmetriebrechung bei der Photoproduktion von Etamesonen  zu
spekulieren. Dies k\"onnte zum einen dazu f\"uhren, da\ss\ an der
physikalischen Schwelle die Pseudovektorkopplung des Etamesons
an das Nukleon nicht notwendig bevorzugt ist. Wir betrachten daher
die allgemeinere Wechselwirkung \cite{BM91}
\be
 {\cal L}_{\eta NN} = (1-\epsilon)\frac{f_{\eta NN}}{m_\eta}       
     \bar{\psi} \gamma_\mu \gamma_5 \psi \partial^\mu \eta
     + i\epsilon g_{\eta NN} \bar{\psi}\gamma_5\psi \eta \; ,
\ee
welche so konstruiert ist, da\ss\ die Kopplung f\"ur Nukleonen 
auf der Massenschale unabh\"angig von dem Parameter $\epsilon$ ist.
Dagegen zeigt der Nukleonbeitrag zur elektrischen Dipolamplitude 
an der Schwelle
\beq
 E_{0+}^N (\eta p) &=& \frac{e}{4\pi} \frac{f_\eta}{m_\eta}
   \frac{\mu_\eta}{(1+\mu_\eta)^{3/2}}\, \left( 1-
   \kappa_p \left( \epsilon -(1-\epsilon)\frac{\mu_\eta}{2} \right) \right) 
   \\[0.2cm]
   &\simeq& -(1.1+4.9\epsilon)\su 
\eeq   
eine starke Abh\"angigkeit von $\epsilon$ . Welcher Wert
von $\epsilon$ die beste Beschreibung der Etaproduktion an der
Schwelle liefert, l\"a\ss t sich letztlich nur durch eine
sorgf\"altige Untersuchung der differentiellen Wirkungsquerschnitte
unterhalb der Resonanzregion entscheiden. In derselben Weise
kann man auch f\"ur die $N(1535)$-Anregung eine skalare anstatt
der oben beschriebenen vektoriellen Kopplung verwenden
\be
 {\cal L}_{\eta NN^*} = (1-\alpha) \frac{f_{\eta NN^*}}{m_\eta}
    \bar{\psi}_{N^*}\gamma_\mu\psi\partial^\mu\eta
    +i\alpha g_{\eta NN^*}\bar{\psi}_{N^*}\psi\eta 
\ee
Diese Modifikation beeinflu\ss t lediglich den nichtresonanten
Untergrund und hat daher nur geringe Auswirkung auf die
elektrische Dipolamplitude an der Schwelle. 

Explizite chirale Symmetriebrechung auf Grund der Masse des
seltsamen Quarks k\"onnte aber vor allem zum Auftreten eines
Sigmaterms f\"uhren. Betrachtet man das Etameson als reines
Okttet-Teilchen, dann ergibt sich mit Hilfe der Methoden aus
Abschnitt 2.4
\beq
\label{sigeta}
 \Delta E_{0+} (\eta p)&=& \frac{e}{4\pi} \frac{1}{1+\mu_\eta}
   \frac{\overline{m}}{f_\eta m_\eta} ( b_0 g_T^0 + b_3 g_T^3) \\[0.1cm]
   b_0 &=& \frac{1}{3\sqrt{3}} \left( 1 +\frac{g^8_T-g_T^0}{3}
     \left( 1-\frac{4m_s}{\overline{m}} \right) \right)  \\
   b_3 &=& \frac{1}{\sqrt{3}}
\eeq
wobei $g_T^0$ die Flavorsingletkopplung im Tensorkanal bezeichnet
und $f_\eta\simeq f_\pi$ die Etamesonzerfallskonstante ist. 
Die St\"arke der Korrektur h\"angt wesentlich von der Gr\"o\ss e
des flavormischenden Parameters $\delta g_T \equiv g_T^8-g_T^0$ ab.

Ist $\delta g_T=0$, dann ist das Resultat proportional zu 
$\overline{m}/m_\eta$ und liefert nur geringf\"ugige Korrekturen.
Konkret finden wir mit $g_T^0=1$ und $g_T^3=5/3$ den Wert
$\Delta E_{0+}(\eta p)=0.4\su$. Ist dagegen $\delta g_T\neq 0$,
so ist die Korrektur proportional zu $m_s/m_\eta$ und kann einen 
substantiellen Beitrag zur Schwellenamplitude liefern. 
 
