\section{Explizite chirale Symmetriebrechung}
Die r\"aumlichen Komponenten des Beitrags aus der expliziten
chiralen Symmetriebrechung 
\be
\label{csbcom}
 \Sigma_\mu^{a}(\vec{q}=0) = 
  \int d^4 x \,\delta (x^0) [\partial^\nu A_\nu^{a}(x),
  V_\mu^{em}(0)]
\ee
sind nicht durch Stromalgebra festgelegt. Aus diesem Grund haben 
wir Ihren Beitrag zur Photoproduktionsamplitude bislang 
vernachl\"assigt. Repr\"asentiert man jedoch die Str\"ome
durch Quarkfelder, so ist auch dieser Kommutator durch die
kanonischen Vertauschungsregeln der Felder bestimmt. 
Der Vollst\"andigkeit halber arbeiten wir in Flavor-$SU(3)$,
so da\ss\
\beq
   \partial^\nu A_\nu^{a} &=& \frac{i}{2} \bar{\psi} \gamma_5
      \left\{ M,\lambda^{a} \right\} \psi  \\
    V_\mu^{em}            &=& \frac{1}{2} \bar{\psi} \gamma_\mu
      ( \lambda^3 + 1/\sqrt{3} \lambda^8 ) \psi
\eeq
mit $M={\rm diag}(m_u,m_d,m_s)$. Es wird sich allerdings zeigen, da\ss\
die Masse der seltsamen Quarks nicht in das Resultat eingeht.
Der Kommutator der beiden Bilinearformen l\"a\ss t sich mit Hilfe 
der Relation              
\beq
\label{bilcom}
 \lefteqn{\delta (x^0-y^0) [\psi^\dagger (y)\frac{\lambda^{a}}{2}
      \Gamma \psi (y),\psi^\dagger (x)\frac{\lambda^{b}}{2}
      \Gamma' \psi (x)] = }  \\
    & & \hspace{1cm}   \frac{1}{2} \delta^4 (x-y) 
      \psi^\dagger (x) \left( if^{abc} \{\Gamma,\Gamma'\} 
      + id^{abc} [\Gamma,\Gamma' ] \right) \frac{\lambda^c}{2}
      \psi (x) \nonumber
\eeq
auswerten. Dabei bezeichnen $\Gamma$ und $\Gamma'$ die Diracoperatoren,
$f^{abc}$ und $d^{abc}$ die antisymmetrschen bzw.~symmetrischen $SU(3)$
Strukturkonstanten. Mit Hilfe von (\ref{bilcom}) ergibt sich f\"ur
$a=1,2,3$
\be
\label{sig0q}
\int d^4x\, \delta (x^0) [\partial^{\nu}A_{\nu}^{a}(x),
   V_{0}^{em}(0)] = i\,\overline{m} \,\epsilon^{3ab} \bar{\psi}
   \gamma_5\lambda^b \psi\hspace{4cm}
\ee
und
\beq   
\label{sigcom}
\int d^4 x\, \delta (x^0) [\partial^{\nu}A_{\nu}^{a}(x),
        V_{i}^{em}(0)] & =&
i\,\overline{m} \,\epsilon_{ijk} \left\{  \delta^{a3}
\frac{1}{\sqrt{3}}\left( \sqrt{2} J_{jk}^{0}+J_{jk}^{8} \right) +
 \frac{1}{3} J_{jk}^{a}\right\}    \\
& &\mbox{} + i\frac{\delta m}{2}\epsilon_{ijk}\delta^{a3} \left\{
 \frac{1}{3\sqrt{3}}\left(\sqrt{2} J_{jk}^{0}+ J_{jk}^{8} \right)
  + J_{jk}^{3}  \right\}  \nonumber
\eeq
wobei wir die Tensorstr\"ome
\be
 J_{\mu\nu}^{c} = \bar{\psi}\sigma_{\mu\nu}\frac{\lambda^c}{2}\psi
\ee
eingef\"uhrt haben. In der von uns verwendeten Normierung ist
$\lambda^0=\sqrt{2/3}\,{\bf 1}$. Man beachte, da\ss\ $1/\sqrt{3}
(\sqrt{2}\lambda^0 +\lambda^8)$ gerade die Einheitsmatrix im 
$SU(2)$-Unterraum ist. Die Str\"ome (\ref{sig0q},\ref{sigcom})
enthalten daher keine Beitr\"age der seltsamen Quarks. Die St\"arke
der chiralen Symmetriebrechung sowie der Isopsinbrechung wird
durch die Parameter
\beq
  \overline{m} &=& \frac{1}{2}(m_u+m_d)  \\
  \delta m     &=& m_u -m_d
\eeq
kontrolliert. Die Zeitkomponente des Sigmakommutators ist im
Fall verschwindender Isopsinbrechung identisch mit der Divergenz des
Axialstroms gegeben.  Das Ergebnis (\ref{sig0q}) reproduziert 
daher das Stromalgebraresultat (\ref{sig0}). Insbesondere
lassen sich Nukleonmatrixelemente der pseudoskalaren Dichte
$\bar{\psi}\gamma_5\lambda^b\psi$ mit Hilfe des Pion-Nukleon
Formfaktors $G_{\pi NN}$ ausdr\"ucken und liefern den bereits
diskutierten Beitrag zum Pionpolterm. 

Die r\"aumlichen Komponenten des Kommutators (\ref{csbcom}) 
liefern dagegen mit den oben definierten Tensorstr\"omen 
einen v\"ollig neuen Beitrag zur Photoproduktionsamplitude.
Dieser Beitrag verschwindet am weichen Punkt $q=0$ und 
bestimmt daher die Extrapolation der Amplitude zur
physikalischen Schwelle.

Die allgemeinste Form des Nukleonmatrixelements der Tensorstr\"ome 
lautet
\beq
  <N(p_2)|\bar{\psi}\sigma_{\mu\nu}\tau^{a}\psi|N(p_1)> &=& 
        \bar{u}(p_2) \left[
     G_T^{a}(t) \sigma_{\mu\nu} + iG_2^{a}(t)
     \frac{\gamma_\mu \Delta_\nu - \Delta_\mu \gamma_\nu}{2M} 
     \right. \\
 & & \mbox{}+ \left. iG_3^{a}(t) 
     \frac{\Delta_\mu P_\nu - P_\mu \Delta_\nu}{M^2}
     + iG_4^{a} \frac{\gamma_\mu P_\nu - P_\mu \gamma_\nu}{M^2}    
     \right] \tau^{a} u(p_1) \nonumber
\eeq
wobei $\Delta_\mu=(p_2-p_1)_\mu$ den Impuls\"ubertrag und $\tau^0 ={\bf 1}$ 
sowie $\tau^{a}\; (a=1,2,3)$ die Paulimatrizen bezeichnet.
Das Matrixelement vereinfacht sich erheblich, wenn man die
r\"aumlichen Komponenten der Str\"ome im Breitsystem des Nukleons
betrachtet
\beq
   <N(\vec{p}\,)|\bar{\psi}\sigma_{jk}\tau^a\psi|N(-\vec{p}\,)>    
  & = & \epsilon_{jkm}\chi^\dagger_f \left[
     \left( G_T(t) +\frac{t}{4M^2} G_2(t)\right)\sigma_{{\mini T}m} 
    \right. \\
 & & \hspace{2.7cm} \mbox{} + \left. G_T(t)\frac{E_p}{M} \sigma_{{\mini L}m}
 \right] \tau^{a} \chi_i  \nonumber
\eeq
Bis auf Korrekturen der Gr\"o\ss enordnung $m_\pi^2$ kann man die
Formfaktoren durch ihren Wert bei $t=0$ ersetzen.
Mit der Definition $g_T=G_T(0)$ ergibt sich schlie\ss lich die 
Korrektur zur Schwellenamplitude f\"ur neutrale Pionen 
\be
\label{delneu}
\Delta E_{0+}(\pi^0 N) = \frac{e}{4\pi f_\pi}\frac{\overline{m}}{m_\pi (1+\mu)}
  \left\{ \left( 1+\frac{\delta m}{6\overline{m}} \right) g_T^0
     \pm \left(\frac{1}{3}+\frac{\delta m}{2\overline{m}}\right) g_T^3
     \right\} \; ,
\ee
wobei das sich das Vorzeichen auf die Produktion am Proton 
bzw.~Neutron bezieht. Verwendet man die oben zitierten  Werte der
Quarkmassen, so ist $\delta m/(2\overline{m}) \simeq -1/3$ und
$\Delta E_{0+}(\pi^0N)$ ist fast vollst\"andig durch die Tensorkopplung
im Singletkanal bestimmt. Man beachte, da\ss\ der Korrekturterm formal
von der Ordnung $m_\pi$ ist, denn nach der GOR Relation gilt $\overline{m}
= m_\pi^2f_\pi^2/|<\bar{u}u+\bar{d}d>|$. 

Die entsprechende Korrektur f\"ur die Produktion geladener Pionen
lautet
\be
\label{delchar}
 \Delta E_{0+}(\pi^-p)=\Delta E_{0+}(\pi^+n) =
  \frac{\sqrt{2}e}{4\pi f_\pi}\frac{\overline{m}}{m_\pi (1+\mu)}
  \,\frac{g_T^3}{3}\; .
\ee  
In diesem Fall tr\"agt der isospinbrechende Term proportional
zu $\delta m$ nicht bei. Der Korrekturterm modifiziert nicht
die Ladungsasymmetrie $|\Epn|-|\Emp|$, liefert aber einen
kleinen Beitrag zum Panofskyverh\"altnis $\Epn/\Emp$.

Die wesentliche Aufgabe bei der Berechnung von $\Delta E_{0+}$
ist nun die Bestimmung der Tensorkopplungskonstanten
$g_T^{a}$ des Nukleons. Diese sind leider experimentell
nicht direkt zug\"anglich, so da\ss\ man in diesem 
Zusammenhang auf Modelle angewiesen bleibt.
Die einfachste M\"oglichkeit ist die Verwendung eines
nichtrelativistischen Konstituentenmodells zur 
Beschreibung der Struktur des Nukleons. In diesem Fall
reduzieren sich die Tensorstr\"ome $\frac{1}{2}\epsilon_{ijk}
\bar{\psi}\sigma_{jk}\psi$ auf Axialstr\"ome $\bar{\psi}
\gamma_i\gamma_5 \psi$. Die Korrektur zur elektrischen 
Dipolamplitude lautet dann
\be
 \DEop = \frac{e}{4\pi f_\pi}\frac{\overline{m}}{m_\pi (1+\mu)}
    (0.90 \cdot g_A^0 + 0.04 \cdot g_A^3) \; .
\ee
In einem nichtrelativistischen Quarkmodell ist $g_A^0=1$ und $g_A^3=5/3$,
so da\ss\ $\DEop = 1.6 \su$. Diese Korrektur ist von derselben 
Gr\"o\ss enordnung wie der f\"uhrende Term des Niederenergietheorems,
$\DEop = -2.3\su$, und besitzt dar\"uber hinaus das umgekehrte Vorzeichen.
        

\section{Eichinvarianz}
Die Forderung nach Eichinvarianz der \"Ubergangsmatrix $T_\mu^{a}$
liefert wichtige Einschr\"ankungen f\"ur die Form der invarianten Amplituden.
Im Falle von Pionen auf der Massenschale ergeben sich diese 
Bedingungen aus der Erhaltung des elektromagnetischen Stroms
im \"Ubergangsmatrixelement
\be
\label{ongi}
k^\mu T_\mu^{a} = ie<\pi^{a}(q)N(p_2)|\partial^\mu V_\mu^{em}(0)|N(p_1)>
=0 \; .
\ee
Die aus dieser Gleichung folgenden Beziehungen (\ref{gaugecond}) haben 
wir bereits im ersten Kapitel angegeben. Man pr\"uft leicht nach, da\ss\ 
die in Abschnitt 2.2 abgeleiteten Amplituden diese Bedingungen erf\"ullen.
Dies gilt jedoch nur f\"ur die Summe von Stromalgebra-, Nukleon- und
Pionpolbeitr\"agen. Keiner dieser Terme ist f\"ur sich genommen 
eichinvariant. 

Die nicht eichinvarianten Terme in den einzelnen Beitr\"agen heben
sich allerdings nur dann gegenseitig weg, wenn keine ph\"anomenologischen 
Formfaktoren an den Vertices verwendet werden. Um die Rolle
der Formfaktoren n\"aher zu untersuchen, wollen wir unsere 
Betrachtungen auf die Elektroproduktion von Pionen erweitern.
In diesem Fall ist das ausgetauschte Photon virtuell und besitzt 
eine nicht verschwindende invariante Masse $k^2$.  Die Kopplung
des Photons wird durch die elektrischen Formfaktoren des 
Nukleons sowie des Pions 
\beq
 \Gamma_\mu^\gamma &=& F_1(k^2) \gamma_\mu + \frac{i\sigma_{\mu\nu}
               k^\nu}{2M} F_2(k^2) \\
 \Gamma_\mu^{\gamma\pi} &=& F_\pi (k^2)(2q-k)_\mu
\eeq
beschrieben. Dar\"uber hinaus liefert der Stromalgebraterm
\be
 C_\mu^{a} = -i\epsilon^{a3c} F_A(t) g_A \bar{u}(p_2)\gamma_\mu
    \gamma_5 \frac{\tau^c}{2} u(p_1)
\ee    
einen Beitrag, welcher den normierten axialen Formfaktor
$F_A(t)=G_A(t)/G_A(0)$ enth\"alt. Wie im Falle reeller 
Photonen lautet die Eichinvarianzbedingung $k^\mu T_\mu^{a}=0$.
Wir zerlegen  die Amplitude in der Form
\be
 T_\mu^{a} = T_\mu^{a(Born)} + \Delta T_\mu^{a} + T_\mu^{a(Res)}
\ee
wobei $T_\mu^{a(Born)}$ die Polterme sowie des Stromalgebrabeitrag
enth\"alt. Der Korrekturterm $\Delta T_\mu^{a}$ ist durch die Bedingung
\be
 k^\mu ( T_\mu^{a(Born)}+\Delta T_\mu^{a}) =0
\ee
definiert, w\"ahrend $T_\mu^{a(Res)}$ eine Untergrundamplitude bezeichnet,
die bis auf die Eichinvarianzforderung $k^\mu T_\mu^{a(Res)}=0$  unbestimmt 
bleibt.

Ber\"ucksichtigt man die Formfaktoren an den Vertices, so ist die
Divergenz des isopsinantisymmetrischen Teils der Bornmaplitude
\beq
\label{ngi}
 k^\mu T_\mu^{(-)(Born)} &=& \frac{ief}{m_\pi} \bar{u}(p_2)\Big(
          2M (2F_1^v(k^2) - F_\pi (k^2) ) \\
   & & \hspace{3cm} \mbox{} - \gamma\cdot k 
	  (2F_1^v(k^2) - F_A(t)) \Big) \gamma_5 u(p_1) \nonumber
\eeq 
Alle anderen Isospinkomponenten erf\"ullen die Eichinvarianzbedingung.
F\"ur die $(-)$-Komponente ist dies nur am Photonpunkt $k^2=0$ der
Fall. Um eine eichinvariante Amplitude zu erhalten, mu\ss\ man einen
Korrekturterm \cite{VZ72,SK91}
\beq
\label{gcor}
\Delta T_\mu^{(-)} &=& -\frac{ief}{m_\pi} \bar{u}(p_2)\left(
          \frac{2Mk_\mu}{k^2} (2F_1^v(k^2) - F_\pi (k^2) ) \right.\\
 & & \hspace{3cm} \mbox{}	  
	  - \left. \frac{k_\mu\gamma\cdot k}{k^2} (2F_1^v(k^2) - F_A(t))
	   \right) \gamma_5 u(p_1) \nonumber
\eeq 
addieren. Dieser Term ist nicht eindeutig bestimmt. Jeder beliebige
Ausdruck, der sich von (\ref{gcor}) nur um einen divergenzfreien
Beitrag unterscheidet, ist ebenfalls ein m\"oglicher Korrekututerm.
Die Summe $T_\mu^{a(Born)}+\Delta T_\mu^{a}$ liefert schlie\ss lich
eine eichinvariante Elektroproduktionsamplitude. 

Es ist instruktiv, die Konsequenzen von Eichinvarianz auch f\"ur 
Pionen abseits der Massenschale zu untersuchen. Dieses Problem
ist vor allem  bei der Bestimmung der Amplitude am weichen Punkt
von Bedeutung. Da sich das Pion nicht in einem asymptotischen
Zustand befindet, kann man zu diesem Zweck allerdings nicht von
Gleichung (\ref{ongi}) Gebrauch machen.  Statt dessen betrachten 
wir die zu (\ref{avward}) analoge Vektorwardidentit\"at
\be
\label{vwi}
ik^\mu \overline{\Pi}_{\nu\mu}^\alpha (q) = - C^\alpha_\nu + 
\frac{i}{m_\pi^2} \big( q_\nu \Sigma_0^\alpha (q) - 
\delta_{\nu 0} k^\rho \Sigma_\rho^\alpha (q) \big) .
\ee
Auch diese Relation beruht auf der Erhaltung des elektromagnetischen 
Stroms. Sie enth\"alt aber keine zus\"atzlichen Annahmen \"uber den
Impuls des Pions. In Verbindung mit der Axialvektorwardidentit\"at
(\ref{avward}) ergibt sich folgender Ausdruck f\"ur die Divergenz
von $T_\mu^{a}$
\be
\label{offgi}
 k^\mu T_\mu^{a} = -i\epsilon^{a3c} \frac{q^2-m_\pi^2}{f_\pi m_\pi^2}
   <N(p_2)|D^c(0)|N(p_1)> \; .
\ee
F\"ur Pionen auf der Massenschale ergibt sich die bekannte Beziehung
$k^\mu T_\mu^{a} =0$. Abseits der Massenschale liefern geladene 
virtuelle Pionen einen zus\"atzlichen Quellterm f\"ur den elektromagnetischen
Strom und bewirken eine nichtverschwindende Divergenz von $T_\mu^{a}$.

Bei der Herleitung der Relation (\ref{offgi}) ben\"otigt man keine 
Annahmen \"uber die modellabh\"angigen Komponenten der symmetriebrechenden
Amplitude $\Sigma_\mu^{a}$. Betrachtet man die einzelnen Beitr\"age 
zur linken Seite von (\ref{offgi}),
\be
\label{divamp}
 k^\mu T_\mu^{a} = \frac{1}{f_\pi} \left\{ ik^\mu q^\nu 
   \overline{\Pi}_{\mu\nu}^{a}
   -k^\mu C_\mu^{a} +\frac{i\omega_\pi}{m_\pi^2} k^\mu \Sigma_\mu 
   \right\}
\ee     
so tragen diese Terme jedoch bei. Die Polterme erf\"ullen in Verbindung
mit dem Stromalgebrabeitrag auch die verallgemeinerte Eichinvarianzbedingung
(\ref{offgi}). Man kann diese Terme daher aus der Gleichung (\ref{divamp})
eliminieren. In der Herleitung des Niederenergietheorems vernachl\"assigt
man Untergrundbeitr\"age zu den Amplituden. Die Gleichung (\ref{divamp})
reduziert sich daher auf eine Beziehung f\"ur die symmetriebrechende
Amplitude: $k^\mu \Sigma_\mu^{(+0)}=0$. Die im Abschnitt 2.5 bestimmten
Beitr\"age erf\"ullen diese Gleichung nicht. Wir definieren daher den 
eichinvarianten Teil von $\Sigma_\mu^{a}$
\be
  \Sigma_\mu^{a(gi)} = \Sigma_\mu^{a} +\Delta\Sigma_\mu^{a}
\ee
durch die Forderung $k^\mu\Sigma_\mu^{(+0)(gi)}=0$. Die isospinungeraden
Komponenten liefern die rechte Seite von (\ref{offgi}). Eine L\"osung
dieser Bedingungen lautet
\beq
 \frac{\omega_\pi}{m_\pi^2}\Sigma_\mu^{(-)(gi)} &=& -
                 \frac{f_\pi}{m_\pi^2-t}\, g_{\pi NN}
                   \, \bar{u}(p_2)i\gamma_5  q_\mu u(p_1) \\
 \frac{\omega_\pi}{m_\pi^2}\Sigma_\mu^{(0)(gi)} &=& \spm
             \frac{4\overline{m}M}{m_\pi^2} \,\frac{g_T^3}{3}
	 \, \bar{u}(p_2)i\gamma_5 \frac{\gamma_\mu \gamma\cdot k}{2M}u(p_1) \\
 \frac{\omega_\pi}{m_\pi^2}\Sigma_\mu^{(+)(gi)} &=& \spm
             \frac{4\overline{m}M}{m_\pi^2} \,
	     \left\{ g_T^0 \left(  1+\frac{\delta m}{6\overline{m}} \right)
	     \pm g_T^3 \frac{\delta m}{2\overline{m}} \right\}
	 \, \bar{u}(p_2)i\gamma_5 \frac{\gamma_\mu \gamma\cdot k}{2M}u(p_1)
\eeq
Auch diese Amplituden sind nicht eindeutig bestimmt. Wir haben sie durch
die Forderung bestimmt, da\ss\ die Schwellenamplitude (\ref{delneu})
unver\"andert bleibt und keine Beitr\"age zum longitudinalen 
Multipol $L_{0+}$ auftreten.	 
   

\section{Resonanzbeitr\"age}
Das Niederenergietheorem zur Pionphotoproduktion beruht auf der
Annahme, da\ss\ sich die Zweipunktfunktion $q^\nu\overline{\Pi}_{\mu\nu}$
in der N\"ahe des weichen Punktes $q^2=0$ durch die Nukleonpolterme
approximieren l\"a\ss t. Dabei vernachl\"assigt man  die Beitr\"age
von Schleifendiagrammen sowie den Austausch von Resonanzen im s- oder
t-Kanal. 

In der Photoproduktion von Pionen bei mittleren Energien 
$\omega^{lab}= 0.3-1.5$ GeV ist die Bedeutung von s-Kanal Resonanzen
in den Multipolamplituden deutlich zu erkennen. 
An der Schwelle sind diese Beitr\"age jedoch durch das
Verh\"altnis $m_\pi/\Delta E_R$ der Pionmasse zur Anregungsenergie
der Resonanz unterdr\"uckt. Der niedrigste Anregungszustand des
Nukleons ist die Deltaresonanz bei $\Delta E_R =294$ MeV. Dieser
Zustand koppelt au\ss erordentlich stark an das Pion-Nukleon System
und dominiert aus diesem Grund die resonante $M_{1+}$-Amplitude
bis in die Schwellenregion. Der niedrigste resonante Beitrag zur
$E_{0+}$ Amplitude stammt vom $N(1535)$ bei einer deutlich h\"oheren
Anregungsenergie $\Delta E_R= 597$ MeV. Im Gegensatz zur Deltaresonanz
zerf\"allt dieser Zustand zu etwa 50\% in $\eta N$ und liefert 
insgesamt nur einen geringen Beitrag zur $E_{0+}$ Amplitude
an der Schwelle.

Um diese Aussagen quantitativ zu belegen, wollen wir die 
Resonanzbeitr\"age mit Hilfe effektiver chiraler Lagrangedichten studieren 
\cite{Pec69,OO75,NB80}. Diese Methode ignoriert die intrinsische 
Struktur der Resonanz, hat aber den wesentlichen Vorteil, mit einem
Minimum an freien Parametern auszukommen. Diese Parameter beschreiben
neben der Masse der Resonanz die Kopplungen $\gamma N\to N^{*}$ 
sowie $N^{*}\to N\pi$ und lassen sich aus den experimentell bestimmten
Helizit\"atsamplituden und Zerfallsbreiten extrahieren.

Zu diesem Zweck betrachten wir resonante Photoproduktion
$\gamma N(\Lambda_i=\frac{1}{2},\frac{3}{2}) \to N^{*} \to \pi N$
mit definierter Helizit\"at $\Lambda_i$ im Eingangskanal. Die
zugeh\"origen Helizit\"atsamplituden $A_{1/2}$ und $A_{3/2}$ sind durch
\beq
\label{helamp}
 A_{l\pm} &=& \mp \alpha C_{N\pi} A_{1/2}  \\
 B_{l\pm} &=& \pm \frac{4\alpha}{\sqrt{(2J-1)(2J+3)}} C_{N\pi} A_{3/2}
\eeq
definiert \cite{PDG90}. Die Helizit\"atskomponenten $(A_{l\pm},B_{l\pm})$ 
sind Linearkombinationen der Multipolamplituden $(E_{l\pm},M_{l\pm})$.
Die entsprecheneden Zusammenh\"ange finden sich im Anhang B. Der
Parameter $\alpha$ lautet
\be
 \alpha = \left[ \frac{1}{\pi} \frac{k}{q} \frac{M\Gamma_\pi}{(2J+1)
    M_R \Gamma^2} \right]^{1/2} \; .
\ee
Dabei bezeichnet $M_R$ die Masse der Resonanz, $J$ ihren Spin
und $\Gamma$ sowie $\Gamma_\pi$ die totalen bzw.~partiellen Zerfallsbreiten.
$C_{N\pi}$ ist der Clebsch Gorden Koeffizient f\"ur den Zerfall der 
Resonanz in den relevanten $N\pi$ Ladungszustand.  Die Definition
(\ref{helamp}) hat den Vorzug, da\ss\ alle Gr\"o\ss en, die mit der
Propagation und dem Zerfall der Resonanz zusammenh\"angen, aus
der  eigentlichen Resonanzamplitude eliminiert werden. Die 
Helizit\"atsamplituden $A_{1/2,3/2}$ liefern daher ein zuverl\"assiges
Ma\ss\ f\"ur die St\"arke des \"Ubergangsmatrixelements $\gamma N\to N^{*}$. 
In Tabelle 1 haben wir die entsprechenden Werte f\"ur die
wichtigsten Resonanzen mit Massen unterhalb 1.6 GeV zusammengefa\ss t. 
    
\begin{table}
\caption{Helizit\"atsmaplituden (in $10^{-3}\,{\rm GeV}^{1/2}$) und
totale und partielle Breiten (in MeV) f\"ur die wichtigsten Nukleonresonanzen
mit Massen unterhalb 1.65 GeV. Alle Angaben nach [PDG90].}
\begin{center}
\begin{tabular}{|l||c|r|r|r|r|} \hline
  Resonanz             & Hel.  &  $A_{1/2,3/2}^p$ & $A_{1/2,3/2}^n$ 
		& $\Gamma_{tot}$ & $\Gamma_\pi$ \\ \hline\hline
 $N(1440)\,P_{11}$ & 1/2   &  $-69\pm 7\;\,$  & $37\pm 19$
                &  200         & 120   \\ 
 $N(1520)\,D_{13}$ & 1/2   &  $-22\pm 10$     & $-65\pm 13$
                &  125         &  70    \\
                       & 3/2   &  $167\pm 10$     & $144\pm 14$
		&              &        \\
 $N(1535)\,S_{11}$ & 1/2   &  $73\pm 14$      & $-76\pm 32$
                &  150         &   65    \\
 $N(1650)\,S_{11}$ & 1/2   &  $48\pm 16$      & $-17\pm 37$ 
                & 150	       &   90    \\
 $\Delta (1232)\,\rm P_{33}$ & 1/2 & $-141\pm 5\;\,$&
                &  115         &  115   \\
		        & 3/2  &  $-258\pm 11$    &          
		&              &        \\ \hline
\end{tabular}
\end{center}
\end{table}

Die dominante Resonanz in der $E_{0+}$ Amplitude ist die $N(1535)S_{11}$
Anregung. Dieser Zustand besitzt wie das Nukleon Spin und Isospin 1/2, 
aber negative Parit\"at. Anregung und Zerfall der Resonanz werden durch
die Kopplungen
\beq
\label{s11coup}
 {\cal L}_{\pi NN^{*}} &=& \frac{f_R}{m_\pi} \bar{\psi}_{N^{*}}
   \gamma_\mu \tau^{a}\psi \partial^\mu \phi^{a} + h.c. \\
 {\cal L}_{\gamma NN^{*}} &=& \frac{e}{4M} \bar{\psi}_{N^{*}} 
   \gamma_5 \sigma_{\mu\nu} (\kappa^s_R +\kappa^v_R \tau^3) \psi
    F^{\mu\nu} + h.c.
\eeq
beschrieben. Die beiden Parameter $f_R$ und $\kappa_R$ werden mit Hilfe
der Beziehungen
\beq
\label{rescoup}
       f_R         &=& \frac{2m_\pi}{M_R-M} 
       \sqrt{\frac{\pi M_R \Gamma_\pi}{ p_1(E_1+M)}} 
       \simeq 0.27  \\
 e\kappa^{p}_R   &=& \frac{(2M)^{3/2}}{\sqrt{(M_R+M)(M_R-M)}} A^{p}_{1/2}
       \simeq 0.51 e
\eeq
festgelegt. Dabei bezeichnen $E_1$ und $p_1$ die Energie sowie den Impuls
des Nukleons im Ruhesystems des angeregten Zustands bei der Resonanzenergie
$\sqrt{s}=M_R$. Unter Verwendung der Vertices (\ref{s11coup}) lassen sich
nun die Borndiagramme zur resonanten Photoproduktion bestimmen. Die 
zugeh\"origen invarianten Amplituden finden sich im Anhang B. Der Beitrag
der s-Kanal Anregung der N(1535) Resonanz zur elektrischen Dipolamplitude
an der Schwelle lautet
\beq
  E_{0+}^{N^{*}}(p\pi^0) &=& \frac{e\kappa_R}{16\pi M}\frac{f_R}{m_\pi}
    \frac{2\mu+\mu^2}{(1+\mu)^{3/2}} 
    \frac{(M_R-M)(M_R+M+m_\pi)}{(M+m_\pi)^2-M_R^2} \\[0.2cm]
    &\simeq& 0.28 \su \, .  \nonumber
\eeq    
Dieses Ergebnis ist formal von der Ordnung $\mu$ und widerspricht daher
der in Abschnitt 2.3 vorgenommenen Absch\"atzung der Untergrundamplitude.
Das liegt darin begr\"undet, da\ss\ die N(1535) Resonanz zus\"atzliche
Kontakterme in den inavrianten Amplitude erzeugt, die in den dort
gemachten Voraussetzungen explizit ausgeschlossen worden sind.

Die leichteste Anregung mit denselben Quantenzahlen wie das Nukleon ist
die Roperesonanz $N(1440)$. Dieser Zustand liefert einen resonanten
Beitrag zur $M_{1-}$ Amplitude, ist in der elektrischen Dipolamplitude 
aber nur als Untergrund pr\"asent. Die effektive Lagrangedichte, welche
die Kopplung des $N(1440)$ an das Nukleon beschreibt, lautet
\beq        
\label{nstarcoup}
 {\cal L}_{\pi NN^{*}} &=& \frac{f_R}{m_\pi} \bar{\psi}_{N^{*}}
   \gamma_\mu \gamma_5\tau^{a}\psi \partial^\mu \phi^{a} + h.c. \\
 {\cal L}_{\gamma NN^{*}} &=& \frac{e}{4M} \bar{\psi}_{N^{*}} 
    \sigma_{\mu\nu} (\kappa^s_R +\kappa^v_R \tau^3) \psi
    F^{\mu\nu} + h.c.
\eeq
Bestimmt man die Kopplungskonstanten aus der Zerfallsbreite und der
Helizit\"atsamplitude bei der Resonanzenergie $\sqrt{s}=m_R$, so
ergibt sich $f_R=0.48$ und $\kappa_R^p=0.58$. Mit diesen Werten 
findet man folgenden Beitrag der s-Kanal Anregung
\beq
 E_{0+}^{N^{*}}(p\pi^0) &=& \frac{e\kappa_R}{16\pi M}\frac{f_R}{m_\pi}
    \frac{2\mu+\mu^2}{(1+\mu)^{3/2}} 
     \frac{m_\pi(M_R-M-m_\pi)}{(M+m_\pi)^2-M_R^2}  
    \\[0.2cm]
    &\simeq& -0.025 \su \, .  \nonumber
\eeq 
Wie erwartet ist die entsprechende Amplitude au\ss erordentlich gering.

Eine gewisse Schwierigkeit stellt die Behandlung der Deltaresonanz
$\Delta (1232)$ dar \cite{DMW91,NS89,NB80}. Dieser Zustand ist
eine $P_{33}$ Anregung und sollte daher nicht zur s-Wellen 
Produktion beitragen. In einer relativistischen Beschreibung
der Deltaresonanz als elementares Spin 3/2 Rarita-Schwinger Feld
enth\"alt der Deltapropagator allerdings abseits der Massenschale auch
Spin 1/2 Komponenten. Die Kopplung dieser Beitr\"age an 
die Zerfallskan\"ale $\gamma N$ und $\pi N$ ist im wesentlichen
unbestimmt.  Je nach Wahl der entsprechenden Parameter findet man
\cite{NS89}
\be
\label{delta}
   E_{0+}^\Delta(\pi^0p) = (-0.10 \ldots 0.34) \su  \; .
\ee
Das angegebene Intervall entspricht der Streuung, die sich aus
verschiedenen Fits der Parameter an die nicht resonanten
Amplituden ergibt.  Das Resultat zeigt deutlich die Grenzen der 
Verwendung effektiver chiraler Lagrangedichten bei der Beschreibung
angeregter Zust\"ande auf. Trotzdem sind auch die Korrekturen
auf Grund der Deltaresonanz letztlich relativ gering. 

Wir haben unsere Untersuchung bislang auf die Rolle von Resonanzen
im s-Kanal beschr\"ankt. Aus dem Studium von Dispersionsrelationen 
ist jedoch bekannt, da\ss\ die Einbeziehung von Vektormesonen als
t-Kanal Resonanzen die Beschreibung der differentiellen Wirkungsquerschnitte 
besonders bei kleinen Energien verbessert \cite{BDW67}. Es scheint daher 
angemessen, die Rolle von Vektormesonen auch direkt an der Schwelle zu 
untersuchen. Dabei beschr\"anken wir uns auf die $\rho$- und $\omega$-Mesonen. 
Das $\phi$-Mesonen sowie die schwereren Vektormesonen liefern nur geringe
Beitr\"age. Die effektive Lagrangedichte lautet
\beq
\label{lvm}
 {\cal L}_{\rho NN} &=& f_{\rho NN} \bar{\psi}
         \left( \gamma_\mu +\frac{\kappa_\rho}{2M}\sigma_{\mu\nu}
	 \partial^\nu \right) \vec{\tau}\cdot\vec{\rho}^{\,\mu} \psi  \\ 
 {\cal L}_{\omega NN} &=& f_{\omega NN} \bar{\psi}
         \left( \gamma_\mu +\frac{\kappa_\omega}{2M}\sigma_{\mu\nu}
	 \partial^\nu \right) \omega^\mu \psi  \\
 {\cal L}_{\rho\pi\gamma} &=& \frac{eg_{\rho\pi\gamma}}{2m_\pi}
         \epsilon_{\alpha\beta\gamma\delta} F^{\alpha\beta}
	 \vec{\phi}\cdot\partial^\gamma\vec{\rho}^{\,\delta} \\
 {\cal L}_{\omega\pi\gamma} &=& \frac{eg_{\omega\pi\gamma}}{2m_\pi}
         \epsilon_{\alpha\beta\gamma\delta} F^{\alpha\beta}
	 \phi_3\cdot\partial^\gamma\omega^\delta \; .
\eeq
Die Kopplung des Photons l\"a\ss t sich aus der gemessen Zerfallsbreite
$\Gamma(\rho,\omega\to\pi\gamma)$ bestimmen. Die Vektormeson-Nukleon
Kopplungskonstante mu\ss\ dagegen indirekt, aus detaillierten Analysen
des Nukleon-Nukleon Potentials gewonnen werden \cite{Dum82}. Die
resultierenden Werte finden sich in Tabelle 2.  
 
\begin{table}
\caption{Parameter f\"ur die wichtigsten t-Kanal Beitr\"age 
zur Pionphotoproduktion.}
\begin{center}
\begin{tabular}{|rcl|rcl|rcl|}\hline
   & $\pi$ &             &  & $\rho$ &             &    &  $\omega$ &  \\ 
                                                                \hline\hline
$m_{\pi^\pm}$&=&139 MeV  & $m_\rho$&=&$770$ MeV    &  $m_\omega$&=&$783$ MeV\\
$f_{\pi NN}$&=&$1.00$    &  $f_{\rho NN}$&=&$2.66$ & $f_{\omega NN}$&=&$7.98$\\
$g_{\pi\pi\gamma}$&=&$1$ &  $g_{\rho\pi\gamma}$&=&$0.125$ 
                                        &   $g_{\omega\pi\gamma}$&=&$0.374$  \\
    & &       &  $\kappa_\rho$&=&$6.6$  &   $\kappa_\omega$&=&$0.$   \\ \hline
\end{tabular}
\end{center}
\end{table}                    

Die invarianten Amplituden, die sich aus der Wechselwirkung (\ref{lvm})
ergeben, haben wir in Anhang B gesammelt. Der Beitrag zur
elektrischen Dipolamplitude an der Schwelle ist
\be
  E_{0+}^V(\pi^0p) = \frac{e}{16\pi} \sum_V\frac{g_V}{m_\pi} 
    f_V (1+\kappa_V)\mu^3
     \frac{2+\mu}{(1+\mu)^{3/2}}\frac{M^2}{m_\pi^2+m_V^2 (1+\mu)}
\ee     	 
wobei $V=\rho,\omega$ zu setzen ist. Dieses Resultat ist explizit
von der Ordnung $\mu^3$ und entspricht daher der Absch\"atzung
aus dem Abschnitt 2.5. Mit den Werten aus Tabelle 2 findet man 
$g_\rho f_\rho (1+\kappa_\rho)=1.50$ und  $g_\omega f_\omega 
(1+\kappa_\omega)=2.87$, so da\ss\ wir schlie\ss lich eine 
Korrektur $E_{0+}=0.024\su$ erhalten.   


 
\section{Abschlie\ss ende Bemerkungen}
blub blub
