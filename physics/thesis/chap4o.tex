\chapter{Chirale Modelle des Nukleons}
Wir haben in  Kapitel 2 verschiedene Korrekturen an das
Standard-Niederenergietheorem zur Pionphotoproduktion untersucht. 
Dabei hat sich gezeigt, da\ss\ die explizite Brechung der chiralen 
Symmetrie durch die Quarkmassen in der QCD Lagrangedichte einen
Beittrag von derselben Gr\"o\ss enordnung wie der f\"uhrende Term
der Niederenergieentwicklung der elektrischen Dipolamplitude
f\"ur neutrale Pionen zu liefern vermag.  Dieses Resultat beruht
freilich wesentlich auf der Verwendung eines nicht-relativistischen
Konstituendenmodells des Nukleons, um die Matrixelemente des 
symmetriebrechenden Terms abzusch\"atzen. Solche Modelle verletzen 
die chirale Symmetrie bereits in der Abwesenheit
von Quarkstrommassen. Wir wollen daher in diesem Kapitel versuchen,
mit Hilfe chiraler Modelle des Nukleons eine verbesserte Bestimmung
der symmetriebrechenden Amplitude $\Delta E_{0+}$ zu gewinnen.

Konkret betrachten wir im Folgenden das chirale Bagmodell sowie
ein nichttopologisches Solitonmodell. Beiden ist gemeinsam, da\ss\
sie das Nukleon als ein System von Valenzquarks und einer mesonischen
Polarisationswolke betrachten. Unterschiedlich ist die Art und Weise,
wie die Aufteilung zwischen diesen Freiheitsgraden vorgenommen wird.
W\"ahrend im chiralen Bagmodell die Quarks im Innern des 
Bags eingeschlossen sind und die Mesonenfelder nur ausserhalb des
Confinementvolumens existieren, verzichtet das nichttopologische
Solitonmodell auf eine scharfe Trennung zwischen diesen beiden 
Realisierungen der zu Grunde liegenden Dynamik.
  
  
\section{Die Spinstruktur des Nukleons}
Bei der Bestimmung der symmetriebrechenden Amplitude $\Delta E_{0+}$
ist es selbstverst\"andlich erforderlich, gleichzeitig eine
m\"oglichst gute Beschreibung der bekannten Eigenschaften des 
Nukleons zu gew\"ahrleisten. In diesem Zusammenhang wollen wir 
uns vor allem auf die axiale Struktur des Nukleons konzentrieren. 
W\"ahrend die Isovektor-Kopplung $g_A^3$ durch den schwachen Zerfalls
des Neutrons sehr genau bestimmt werden kann, haben sich erst 
in letzter Zeit Information \"uber den Flavorsinglet-Kanal ergeben.

Im Jahre 1987 berichtet die EMC-Gruppe \cite{EMC89} \"uber eine 
Messung der Spinasymmetrie 
\be
  A = \frac{\sigma (\mu\!\uparrow p\!\uparrow)
            -\sigma (\mu\!\uparrow p\!\downarrow)}
	    {\sigma (\mu\!\uparrow p\!\uparrow)
            +\sigma (\mu\!\uparrow p\!\downarrow)}    
\ee
in polarisierter tiefinelastischer  Streuung. $\sigma (\mu\!\uparrow
p\!\uparrow\downarrow)$ bezeichnet den doppeltdifferentiellen
Wirkungsquerschnitt $d\sigma/(dQ^2dx)$ f\"ur die Streuung longitudinal
polarisierter Muonen an Protonen mit parallel bzw.~antiparallel 
ausgerichtetem Spin. Im kinematischen Bereich des EMC-Experiments
ist die Aymmetrie in guter N\"aherung durch das Verh\"altnis der
spinabh\"angigen Strukturfunktion $g_1^p$ und der unpolarisierten
Strukturfunktion $F_1^p$
\be
  A =\frac{g_1^p(x,Q^2)}{F_1^p(x,Q^2)}
\ee
gegeben. Im Rahmen des Partonmodells lautet  das erste Moment von
$g_1^p$ 
\be
 \int_0^1 g_1^p(x,Q^2)dx =\frac{1}{2} \left( \frac{4}{9}\Delta u
   +\frac{1}{9}\Delta d +\frac{1}{9}\Delta s \right)
   \left( 1-\frac{\alpha_s}{\pi} + {\cal O}(\alpha_s^2) \right) \; .
\ee       
Die Spinmomente $\Delta q$ der Quarks lassen sich durch
Matrixelemente des Axialvektorstroms definieren
\be
   <N(p)|\bar{q}\gamma_\mu\gamma_5 q|N(p)>=\Delta q s_\mu\; ,
\ee    	    
wobei $s_\mu$ den kovarianten Spin des Nukleons bezeichnet.
Unter Verwendung schwacher Zerf\"alle und der $SU(3)$
Flavorsymmetrie lassen sich zwei Kombinationen der Momente
$\Delta q$ festlegen
\beq
  g_A^3 &=& \Delta u-\Delta d = 1.255\pm 0.006  \\
  g_A^8 &=& \Delta u+\Delta d -2\Delta s = 0.6 \pm 0.1
\eeq
Das Resultat des EMC-Experiments, $\int g_1^p dx =
0.114\pm 0.036$, erm\"oglicht schlie\ss lich die individuelle
Bestimmung der $\Delta q$
\be
  \Delta u  =    0.74\pm 0.10  \hspace{0.5cm}
  \Delta d  =  - 0.54\pm 0.10  \hspace{0.5cm}
  \Delta s  =  - 0.20\pm 0.11  \; .
\ee
\"Uberraschend ist der starke Polarisationsgrad der seltsamen Quarks
im Nukleon. Dar\"uber hinaus implizieren diese Zahlen  einen
sehr kleinen Werte f\"ur die 
Flavorsinglet-Axial\-vek\-tor\-kopplungs\-kon\-stan\-ten des Nukleons
\be
 g_A^{\lambda_0} = \Delta u + \Delta d+\Delta s = 0.01\pm 0.29 \; .
\ee
Diese Gr\o"\ss e  ist ein Ma\ss\ f\"ur den Beitrag der Quarkspins 
zum Spin des Nukleons. Nach dem  EMC-Experiment ist dieser Wert 
vertr\"aglich mit Null. Auch der Anteil der leichten Quarks am 
Nukleonspin $g_A^0 = \Delta u + \Delta d=0.20$ ist erstaunlich 
gering.

Ber\"ucksichtigt man im Partonmodell die Effekte der $U(1)_A$
Anomalie, so m\"ussen die Spinmomente $\Delta q$ um den 
gluonischen Beitrag $\Delta \Gamma =\alpha_s/(2\pi)\Delta g$ 
korrigiert werden \cite{AR88}. Nach einer vorsichtigen Absch\"atzung    
\cite{AS89} ist $\Delta \Gamma = 0.1$ ein Wert, der mit den 
Informationen \"uber die unpolarisierte Gluonverteiliung 
vertr\"aglich ist. In diesem Fall ergibt sich $g_A^{\lambda_0}
=0.3$ und $g_A^0=0.4$.   

\section{Das chirale Bagmodell}
Das chirale Bagmodell beschreibt ein System masseloser Quarks,
die in einem sph\"arischen Bag vom Radius $R$ eingeschlossen
sind. An der Bagoberfl\"ache koppeln die Quarks an Mesonenfelder,
die auf den Au\ss enraum des Bags eingeschr\"ankt sind. Die 
Lagrangedichte des Modells lautet 
\be
\label{LCB}
{\cal L}= \left( \bar{\psi}\frac{i}{2}\gamma^{\mu}\stackrel{\leftrightarrow}
{\partial}_{\mu}\psi -B\right) \Theta (R-r) - \frac{1}{2} \bar{\psi}
e^{i\gamma_5 \vec{\tau}\cdot\vec{\phi}}\psi\delta (r-R)
+{\cal L}_{mes}\Theta (r-R) \; .
\ee
Wir bechr\"anken uns hier auf Flavor-$SU(2)$, so da\ss\ $\psi$ einen 
Isodoublet Quarkspinor und $\vec{phi}$ das Isotriplet Pionfeld 
bezeichnet. Die Struktur des Kopplungsterms ist durch die Forderung 
nach $SU(2)_L\times SU(2)_R$ Invarianz der Wechselwirkung diktiert.
Um die Beschreibung der elektromagnetischen Eigenschaften des 
Nukleons zu verbessern, enth\"alt das Modell neben den Pionfeldern
die Vektormesonen $(\rho,a_1,\omega)$, die durch eine massive 
Yang-Mills Lagrangedichte beschrieben werden. Die detaillierte 
Form der mesonischen Lagrangedichte ${\cal L}_{mes}$ sowie die
resultierenden Bewegungsgleichungen finden sich in \cite{HTW90}.
Wir haben die mesonischen Parameter $f_\pi=93$ MeV und $g_{\rho\pi\pi}
=5.85$ durch ihre empirischen Werte fixiert und eine Vakuumenergie
$B^{1/4}=150$ MeV verwendet. Die Abh\"angigkeit der Resultate vom
Bagradius $R$ werden wir noch im Einzelnen diskutieren.

Die Bewegungsgleichungen werden mit Hilfe des Hedgehogansatzes 
\be
\label{hedge}
\vec{\phi}(\vec{r}) = \hat{r} f_\pi \Theta (r)
\ee
f\"ur das Pionfeld gel\"\ss t. Die dimensionslose Funktion $\Theta (r)$
bezeichnet man als chiralen Winkel. Der Wert $\Theta_R\equiv \Theta (R)$
des chiralen Winkels am Bagrand bestimmt \"uber die chirale Randbedingung
\be
 \left. n^\mu\gamma_\mu \psi\right|_R = \left. \exp [i\gamma_5\vec\tau\cdot
 \hat r \Theta ] \psi \right|_R
\ee 
das Spektrum des Dirac-Hamiltonoperators. Auf Grund des 
Hedgehogansatzes kommutiert dieser nicht mit dem Drehimpuls- oder
Isospinoperator. Quarkzust\"ande lassen sich daher lediglich durch ihren 
Grandspin $\vec{G}=\vec{L}+\vec{S}+\vec{I}$ und ihre Parit\"at $\pi$
klassifizieren.  F\"ur $G^\pi=0^+$ haben die Wellenfunktionen die 
Gestalt   
\beq
\label{zeroplus}
\psi_{0+}  &=& \frac{1}{\sqrt{4\pi}} \left( \begin{array}{c}
                 j_0(\epsilon r)  \\[0.2cm]
                 i\vec \sigma \cdot \hat{r} j_1(\epsilon r)
                 \end{array} \right) \chi_H  \\
 \chi_H &=&  \frac{1}{\sqrt 2}
            (\vert u\downarrow \rangle - \vert d\uparrow \rangle ) 
	    \nonumber
\eeq
Zust\"ande mit gutem Spin und Isopspin werden mit Hilfe der 
semiklassischen Crankingmethode \cite{KJR86} konstruiert. In 
f\"uhrender der Ordnung der Crankingfrequenz $\vec{\omega}$ lautet
die resultierende Wellenfunktion 	    
\be
\label{crankwave}
|\psi>=|H>-\sum_{ph} \frac{|ph><ph|\vec{Q}|H>}{E_p -E_h}
\cdot \vec{\omega}
\ee
wobei 
\be
 \vec{Q} = \int d^3r \bar{\psi}\gamma_0\frac{\vec{\tau}}{2}\psi
\ee
die Isovektorladung im Quarksektor ist. Ferner bezeichnet $|H>$ den
Hedgehogzustand  (\ref{zeroplus}) und $|ph>$ dessen Teilchen-Loch
Anregungen. $E_p$ und $E_h$ sind die Energien der Teilchen 
bzw.~Lochzust\"ande. 

Im Innern des Bags sind die Quarks frei, so da\ss\ der symmetriebrechende
Kommutator durch den Ausdruck (\ref{sigcom}) bestimmt ist. Die 
zugeh\"orige ergibt sich aus Matrixelementen der Tensorstr\"ome 
zwischen den kollektiven Wellenfunktionen (\ref{crankwave}). Im
Au\ss enraum ist die Divergenz des Axialstroms
\be
 \partial^\mu A_\mu^{a} =  f_\pi m_\pi^2\pi^{a}
\ee
wobei $\pi^{a}=f_\pi \sin\Theta\frac{1}{3}\tau^{a}\vec{\sigma}\cdot
\hat{r}$ das kanonische Pionfeld bezeichnet. Da die r\"aumlichen 
Komponenten des elektromagnetischen Stroms nicht das kanonisch
konjugierte Feld $\dot{\pi}^{a}$ enthalten, verschwindet der 
Kommutaor au\ss erhalb des Bagvolumens.

Die Bestimmung der Tensorstrommatrixelemente erfolgt mit Hilfe von
Standardmethoden der semiklassischen Quantisierung. F\"ur die
Singletkopplung ergibt sich   
\be
\label{gt0cb}
g_T^{0}=\frac{T_{\overline{\sigma}\tau}}{2\Lambda_{tot}}
\ee
mit
\be
\label{tsigbtau}
T_{\overline{\sigma}\tau}=\frac{2N_c}{3}\sum_{ph}
\frac{<H|\bar{\psi}\gamma_0\vec{\sigma}\psi|ph>\cdot<ph|\vec{Q}|H>}{E_p-E_h}.
\ee
wobei $\Lambda_{tot}=\Lambda_{qu}+\Lambda_{mes}$ das gesamte (Quarks und
Mesonen) Tr\"agheitsmoment des Solitons bezeichnet.  Die detaillierte 
Gestalt der Matrixelemente findet sich in Anhang C, der mesonische
Beitrag zum  Tr\"agheitsmoment in Referenz \cite{HTW90}.

Das Resultat (\ref{gt0cb}) zeigt gro\ss e \"Ahnlichkeit mit dem 
entsprechenden Ergebnis f\"ur die axiale Kopplung im Isosingletkanal
\be
\label{ga0cb}
 g_A^0 = \frac{T_{\sigma\tau}}{2\Lambda_{tot}}
\ee
wobei $T_{\sigma\tau}$ analog zu (\ref{tsigbtau}) definiert ist, aber
anstelle von Matrixelementen des Tensorstroms die \"Ubergangselemente
$<H|\bar{\psi}\vec{\sigma}\psi|ph>$ enth\"alt. Ber\"ucksichtigt man 
im Grundzustand nur das $0^+$-Valenzorbital, so ist $g_A^0=g_T^0$
genau dann  erf\"ullt, wenn die unteren Komponenten der zugeh\"origen
Wellenfunktion verschwinden. Diese Situation ist bei dem speziellen
Anschlu\ss winkel $\Theta_R$ gegeben. In diesem Fall verhalten sich 
die Quarks wie nichtrelativistische Konstituenden, liefern aber 
$g_A^0<1$.
\begin{figure}
\caption{Vergleich der Quarkbeitr\"age zu den axialen und 
Tensorkopplungskonstanten im chiralen Bag Model als Funktion 
des Bagradius $R$ in fm.}
\vspace{14cm}
\end{figure} 

Bei der Bestimmung der Matrixelementen von Isovektoroperatoren 
tr\"agt bis zur ersten Ordnung in der kollektiven Variable 
$\vec{\omega}$ nur der Grundzustand bei. Das $0^+$-Valenzorbital
liefert
\be
\label{gt3cb}
  g_T^3 = \frac{1}{3} \frac{2+2y^2-3y/\Omega}{1+y^2-2y/\Omega}
\ee
f\"ur die Tensorkopplung, sowie
\be
\label{ga3cb}
  g_A^3 = \frac{1}{3} \frac{1+y^2}{1+y^2-2y/\Omega}
\ee  
f\"ur die axiale Kopplungskonstante. Dabei bezeichnet $\Omega=
ER$ den Energieeigenwert in Einheiten des Bagradius und 
$y=j_0(\Omega)/j_1(\Omega)$. Die beiden Kopplungen bleiben
endlich im Grenzfall eines verschwindenden Bagradius. Das bedeutet
unter anderem, da\ss\ in diesem Limes die Kontinuit\"at des 
Axialstroms durch die Bagoberfl\"ache nicht gew\"ahrleistet 
werden kann. Es ist daher erforderlich, auch die Beitr\"age
von Zust\"anden im Diracsee zu ber\"ucksichtigen. 
Wir regularisieren den  Axialstrom durch eine infinitesimale 
Aufspaltung der Zeitargumente
\be
\vec A_{reg}^{a} =\frac{1}{2}
 [\bar{\psi}(x^0,\vec{x}),\vec{\gamma}\gamma_5\tau^{a}
\psi (x^0+i\tau,\vec{x})] \; .
\ee
Die Seebeitr\"age zur axialen Kopplung lassen sich dann in die
Form
\be
\label{mode}
 g_A^3(\Theta_R,\tau) = -\frac{1}{2} \sum_n {\rm sgn}(\Omega_n)
   g_A^3(\Omega_n) e^{-\tau |\Omega_n|} + \Theta (\Omega_0)
   g_A^3(\Omega_0)
\ee
bringen. Dabei bezeichnet $g_A^3(\Omega_n)$ den Beitrag des 
Zustands mit dem Eigenwert $\Omega_n$ und $\Omega_0$ das
Valenzniveau. Die Summe divergiert im Limes $\tau\to 0$, so
da\ss\ man eine Renormierungsvorschrift ben\"otigt. Im Falle
des Axialstroms lassen sich die notwendigen Gegenterme durch
die Forderung nach Stromerhaltung fixieren \cite{ZWM85}. Das
Ergebnis lautet 
\be
\label{gareg}
g_A^3(\Theta_R) = \lim_{\tau\to 0} \left\{ g_A^3(\Theta_R,\tau)
  -\frac{1}{2} \sin (2\Theta_R) \left.
  \frac{dg_A^3(\Theta_R,\tau)}{d\Theta_R}
  \right|_{\Theta_R=0} \right\} \; .
\ee   
F\"ur den Tensorstrom gibt es keine Erhaltungss\"atze, mit deren
Hilfe man die Form der Gegenterme begr\"unden k\"onnte. Auf Grund
der gro\ss en \"Ahnlichkeit der beiden Matrixelemente erscheint es
allerdings sinnvoll, dieselben Subtraktionen vorzunehmen. 

In den praktischen Rechnungen haben wir uns bei der Bestimmung  der
Modesummen (\ref{mode}) auf Zust\"ande mit den Quantenzahlen
$G^\pi=0^\pm$ beschr\"ankt. Auf Grund der Symmetrieeigenschaften des
Spektrums verursacht diese Einschr\"ankung keine Fehler in den 
Spezialf\"allen $\Theta_R=0,\frac{\pi}{2},\pi$. Dar\"uber hinaus 
kann man f\"ur die axiale Kopplung, wo gute Parametrisierungen des
vollst\"andigen  Resultats vorliegen \cite{VJG84}, nachpr\"ufen, da\ss\
auch f\"ur andere Werte des chiralen Winkels $\Theta_R$ keine gro\ss en
Abweichungen auftreten.

\begin{figure}
\caption{Korrekturen zur elektrischen Dipolamplitude auf Grund expliziter 
Symmetriebrechung im chiralen Bag Modell.}
\vspace{8cm}
\end{figure}

Die Ergebnisse f\"ur die symmetriebrechende Amplitude $\Delta E_{0+}$
im chiralen Bagmodell finden sich in Abbildung 2. Sie h\"angen stark vom
verwendeten Bagradius ab. Insbesondere verschwindet $\Delta E_{0+}$
im Grenzfall $R\to 0$, was dem Resultat in einem rein mesonischen 
Solitonmodell entspricht. Die starke Abh\"angigkeit vom Bagradius
widerspricht allerdings einer der grundlegenden Ideen des chiralen 
Bagmodells, da\ss\ n\"amlich physikalische Observable zumindest in
gewissen Umfang unabh\"angig von der Realisierung der Dynamik mit Hilfe
von mesonischen oder Quark-Gluon Freiheitsgraden sein sollte. Sie mag 
daher ein Hinweis auf das Fehlen wichtiger gluonischer Beitr\"age in 
unserer Beschreibung sein.
     
\begin{table}
\caption{Korrekturen zur elektrischen Dipolamplitude auf Grund expliziter 
Symmetriebrechung im chiralen Bag Modell.} 
\begin{center}
\begin{tabular}{|c||c|c|c|}\hline
 R [fm]   & $\DEop$         & $\DEon$       & $\DEcn$        \\ \hline\hline
 0.30     &   0.07          &    0.09       &   0.16         \\ 
 0.52     &   0.22          &    0.20       &   0.37         \\  
 0.75     &   0.51          &    0.28       &   0.56         \\  
 1.00     &   0.89          &    0.27       &   0.61         \\ 
 1.30     &   1.14          &    0.25       &   0.62        \\ \hline
\end{tabular}
\end{center}
\end{table} 

Eine \"ahnlich starke Abh\"angigkeit vom Bagradius zeigt auch die
axiale Kopplung im Isosinglet-Kanal\footnote{Im Fall von $g_A^0$
existieren allerdings Versuche, die Radiusabh\"angigkeit mit Hilfe
gluonischer Effekte zu mildern, siehe \cite{PV89}. Keine gro\ss en
Effekte liefert dagegen die Einbeziehung eines mesonischen Beitrags
vom $\eta '$ Meson \cite{HW89}.}.  F\"ur den in ph\"anomenologischen
Anwendungen des Modells gew\"ohnlich bevorzugten Wert $\Theta_R=
\frac{\pi}{2}\;(R=0.52\,\rm fm)$ ergibt sich $g_A^0=0.12$. 
Konsistent mit dem  EMC-Resultat $g_A^0<0.4$ sind Bagradien bis etwa
$R=0.7$ fm. Das entspricht einer oberen Grenze f\"ur die
symmetriebrechende Amplitude $\Delta E_{0+}(\pi^0p)<0.6 \su$.    

\section{Das nichttopologische Solitonmodell}
Um die Modellabh\"angigkeit von $\Delta E_{0+}$ n\"aher zu untersuchen,
wollen wir den symmetriebrechenden Kommutator in einem 
nichttopologischen Solitonmodell \cite{BB85} untersuchen. Wir
verwenden die Lagrangedichte des linearen $\sigma$-Modells 
\beq
\label{linsig}
  {\cal L} & = & \bar  \psi [i
\partial  _\mu \gamma _\mu + g ( \sigma  + i \vec \tau \cdot \vec
\pi \gamma  _5 ) ] \psi \nonumber  \\
 & & + {1 \over 2} (\partial
_\mu  \sigma)^2  + {1 \over  2} (\partial  _\mu  \vec  \pi )^2  -
{\lambda^2  \over  4} 
 ( \sigma  ^2 + \vec  \pi  ^2  - \nu  ^2 )^2
 + {\cal L}_{sb}
\eeq	
wobei $\psi$ wie oben das Doublet der leichten Quarks bezeichnet, die
Mesonfelder $\sigma$ und $\vec\pi$ sich im Gegensatz zum chiralen 
Bagmodell aber nach einer linearen Darstellung von $SU(2)_L\times
SU(2)_R$ transformieren. 

Die Parameter $g$ und $\lambda$ bezeichnen die Quark-Meson und
Meson-Meson Kopplungskonstanten. Die Pionzerfallskonstante 
$f_\pi=93$ MeV und die Pionmasse $m_\pi=139.6$ MeV sind durch
ihre experimentellen Werte bestimmt. Dar\"uber hinaus ist
$\nu^2=f_\pi^2 -m_\pi^2/\lambda^2$, so da\ss\ die chirale 
Symmetrie im Grundzustand durch einen nichtverschwindenden 
Erwartungswert $<\sigma>=-f_\pi$ spontan gebrochen ist. 

Die explizite Brechung der chiralen Symmetrie ber\"ucksichtigen 
wir in der Lagrangedichte durch den Beitrag ${\cal L}_{sb} = 
-f_\pi m_\pi^2\sigma -\bar\psi M\psi$. Dabei enth\"alt das 
skalare Feld den nichtperturbativen  Anteil des Quarkkondensats,
w\"ahrend der zweite Term nur im Valenzquark-Sektor zu 
ber\"ucksichtigen ist. Aus den im letzten Abschnitt geschilderten 
Gr\"unden tragen nur die Quarkfelder zum symmetriebrechenden 
Kommutator bei. Auf diese Weise reduziert sich die Auswertung 
von $\Delta E_{0+}$ erneut auf die Bestimmung der Tensorkopplungen
des Nukleons. Eine genau umgekehrte Situation ergibt sich im Rahmen
des Modells f\"ur den $\pi N$-Sigmaterm, bei dem  der wesentliche
Beitrag vom $\sigma$-Feld stammt, w\"ahrend die Valenzquarks
nur geringf\"ugige Korrekturen liefern. 

Wir verwenden erneut den Hedgehogansatz, um die klassischen 
Bewegungsgleichungen zu l\"osen :
\beq
\psi_H(\vec r) &=& \left( \begin{array}{c}
                 G(r) , \\
                 i\vec \sigma \cdot \hat r F(r)
                 \end{array} \right)  \chi_H  \\
\pi^{a}(\vec r)  &=& \hat r^{a} h(r)  \nonumber \\
\sigma (\vec r)  &=&  \sigma (r) \nonumber 
\eeq
Der wesentliche Unterschied zum chiralen Bagmodell besteht darin, da\ss\
die verschiedenen Freiheitsgrade nicht durch scharfe Grenzen voneinander
getrennt sind. Es ist daher m\"oglich, anstelle der im letzten Abschnitt
verwendeten semiklassischen Quantisierung, Peierls-Yoccoz Projektion
zu verwenden, um Zust\"ande mit wohl definiertem Spin und Isospin zu
konstruieren. Die Wellenfunktion eines Protons mit Spinprojektion
$+\frac{1}{2}$ lautet   
\be
\vert p \uparrow \rangle \sim \int d[ \Omega ] \,
        D^{1/2 *}_{1/2,-1/2}(\Omega) \, R(\Omega) \vert h \rangle ,
\ee
wobei $\Omega$ die Eulerwinkel bezeichnet, $R(\Omega)$ den 
Rotationsoperator und $D^{1/2}$ die Spin $\frac{1}{2}$
Wigner D-Funktion. 

Um den mesonischen Teil der Hedgehog-Wellenfunktion $|h>$ zu 
konstruieren, interpretieren wir die klassischen Mesonfelder   
als Erwartungswerte der Feldoperatoren in einem koh\"arenten
Fockzustand \cite{Bir86}
\beq
  <h_{mes}|\pi^{a}(\vec r\,)|h_{mes}> &=& \hat{r}^{a} h(r) \\
  <h_{mes}|\sigma (\vec r\,)|h_{mes}> &=& \sigma (r) \; .
\eeq  
Dieser Zustand erm\"oglicht die Bestimmung des Erwartungswerts
des Pionanzahloperators im Hedgehogsoliton
\be
 \overline{\rm N}_\pi = <h_{mes}|\hat{N}_\pi|h_{mes}> \; .
\ee
Diese Gr\"o\ss e bestimmt den Einflu\ss\ der Pionfelder auf
Matrixelemente fermionischer Operatoren in einem projezierten
Zustand.  Die detaillierte Gestalt dieser Matrixelemente ist 
abh\"angig vom Isospin des Operators. Mit Hilfe der im Anhang 
D erl\"auterten Drehimpulsalgebra finden wir
\be
\label{ga0qs}
g_{A,T}^{0} = \frac{N_c}{2}\,\frac{1}{{\cal N}}
\int_0^{\pi}du\,\int_0^1 dt\,t^{N_c +2}\cos^{N_c +1}{\kl (}u{\kl )}\sin^2
{\kl (}u{\kl )}
\,e^{-\frac{4}{3}\overline{{\rm N}}_{\pi}(1-t^2 \cos^2 u)}\,\cdot I_{A,T}
\ee
f\"ur den Isosinglet-Anteil, sowie
\be
\label{ga3qs}
g_{A,T}^{3} =  \frac{N_c}{3} \left\{ 1\,+\, \frac{1}{{\cal N}}
\int_0^{\pi} du\, \int_0^1 dt\, t^{N_c}(1-t^2 )\cos^{N_c -1}{\kl (}u{\kl )}
\,e^{-\frac{4}{3}\overline{{\rm N}}_{\pi}(1-t^2 \cos^2 u)} \right\}
\cdot I_{A,T}
\ee
f\"ur die Isovektor-Kopplungen. Der Normierungsfaktor lautet
\be
\label{norm}
{\cal N} = \int_0^{\pi} du\, \int_0^1 dt\, t^{N_c +2}\cos^{N_c +1}{\kl(}u{\kl)}
\,e^{-\frac{4}{3}\overline{{\rm N}}_{\pi}(1-t^2 \cos^2 u)}  
\ee
und $I_{A,T}$ bezeichnet die Radialintegrale
\be
\label{red}
I_{A,T}=\int_0^{\infty} r^2 dr\, ( G^2 \pm\frac{1}{3} F^2 )\;.
\ee
Im Grenzfall verschwindender Pionfelder ist $\overline{\rm N}_\pi=0$
und wir erhalten $g_{A,T}^0=I_{A,T}$ sowie $g_{A,T}^3=\frac{5}{3}
I_{A,T}$, wie man es im Rahmen eines relativistischen 
Konstituendenmodells des Nukleons erwarten w\"urde.

F\"ur realistische Modellparameter ist $\overline{\rm N}_\pi=1.2$ und 
die Kopplungskonstanten werden durch die mesonischen Matrixelemente
reduziert. In den Abbildungen 3 und 4 zeigen wir die Ergebnisse 
f\"ur $g_A$ und $\Delta E_{0+}$ als Funktion der Quark-Meson 
Kopplung $g$. F\"ur hinreichend gro\ss e Werte der Meson-Meson 
Kopplung sind die Resultate praktisch unabh\"angig von $\lambda$. 
Die korrekte Masse des Nukleons ergibt sich f\"ur die Wahl
$g=5.38$ und $\lambda =10$.  In diesem Fall ist $\DEop=0.9\su$,
ein Wert, der noch im Bereich eines relativistischen Konstituendenmodells
liegt. Man sollte allerdings beachten, da\ss\ im Rahmen der 
gew\"ahlten Methode auch die axialen Kopplungen des Nukleons
mit $g_A^0=0.45$ und $g_A^3=1.64$ recht gro\ss e Werte annehmen.
Dies mag ein Hinweis auf die Tatsache sein, da\ss\ das Modell 
generell die Rolle der Valenzquarks in der Struktur des Nukleons
\"ubersch\"atzt oder gewisse Beitr\"age im mesonischen bzw.~Quarksektor
der Theorie mehrfach ber\"ucksichtigt.

\begin{figure}
\caption{Axiale Kopplungskonstanten im nichttopologischen
Solitonmodell. Die durchgezogenen Linien zeigen die mit Hilfe
von Peierls-Yoccoz Projektion gewonnenen Resultate, die 
gestrichelten Linien die entsprechenden Ergebnisse der 
semiklassischen Quantisiserung.}
\vspace{8cm}
\end{figure}
Die im Vergleich zum chiralen Bagmodell abweichenden Ergebnisse 
k\"onnten nat\"urlich auch in den unterschiedlichen 
Projektionsverfahren begr\"undet sein. Um diese Frage n\"aher 
zu untersuchen, haben wir die  semiklassische
Quantisierung auch auf das nichttopologischen Solitonmodell
angewendet. Bezeichnet   $H=\vec\alpha\cdot\vec p -g\beta
(\sigma+i\gamma_5\vec\tau\cdot\hat r h)$ den Dirac-Hamiltonoperator 
und $\psi_H$ die Grundzustandswellenfunktion mit dem
zugeh\"origen Eigenwert $\epsilon$, so ist die St\"orung 
der Wellenfunktion auf Grund der kollektiven Rotation mit
der Winkelgeschwiundigkeit $\omega$ durch die Gleichung 
\be
\label{crankeq}
(H-\epsilon)\delta\psi=-\frac{\vec{\omega}\cdot\vec{\tau}}{2}\psi_H .
\ee
bestimmt. F\"uhrt man einen vollst\"andigen Satz Eigenfunktionen 
von $H$ ein, so l\"a\ss t sich  Gleichung (\ref{crankeq})
auch in die \"aquivalente Form (\ref{crankwave}) aus dem letzten
Abschnitt bringen. Da wir im nicht-topologischen Solitonmodell keine 
expliziten Ausdr\"ucke f\"ur  die Eigenfunktion zur Verf\"ugung
haben, ist es jedoch sinnvoller, die Cranking-Gleichung 
(\ref{crankeq}) direkt zu l\"osen. Zu diesem Zweck parametrisieren
wir $\delta\psi$ wie in \cite{CB86}  
\be
\delta\psi(\vec{r}) = \frac{1}{\sqrt{4\pi}} \left( 
\begin{array}{c}
 A(r)\vec{\Omega}\cdot\vec{\sigma} \;+\; B(r) (\frac{1}{3}\vec{\Omega}\cdot
\vec{\sigma}-(\vec{\Omega}\cdot\hat{r})(\vec{\sigma}\cdot\hat{r})) \\
iC(r)\vec{\Omega}\cdot\hat{r} \;-\; D(r) (\vec{\Omega}\times\hat{r})
\cdot\vec{\sigma}
\end{array}   \right) |\chi_H>\;\;,
\ee

\begin{figure}
\caption{Korrekturen zur elektrischen Dipolamplitude auf Grund expliziter 
Symmetriebrechung im nichttopologischen Solitonmodell.}
\vspace{8cm}
\end{figure}
\begin{table}
\caption{Korrekturen zur elektrischen Dipolamplitude auf Grund expliziter
Symmetriebrechung im nichttopologischen Solitonmodell.}
\begin{center}
\begin{tabular}{|c||c|c|c||c|c|c|}\hline
          & \multicolumn{3}{|c||}{Peierls-Yoccoz Projektion} 
	  & \multicolumn{3}{c|}{semiklassische Projektion}  \\  
   g      & $\DEop$         & $\DEon$        & $\DEcn$
          & $\DEop$         & $\DEon$        & $\DEcn$\\ \hline\hline
   4      &   1.09 & 0.84 & 0.99 & 1.03 & 0.86 & 0.66 \\ 
   5      &   0.95 & 0.72 & 0.92 & 0.68 & 0.52 & 0.63 \\ 
   6      &   0.89 & 0.66 & 0.88 & 0.53 & 0.37 & 0.61 \\ 
   7      &   0.85 & 0.63 & 0.85 & 0.43 & 0.28 & 0.59 \\ 
   7.8    &   0.82 & 0.61 & 0.83 & 0.39 & 0.24 & 0.58 \\ \hline
\end{tabular}   
\end{center}
\end{table}
und l\"osen die resultierenden gekoppelten Gleichungen f\"ur die 
Funktionen $A,B,C,D$ numerisch. Matrixelemente von Bilinearformen 
ergeben sich, indem man die Feldoperatoren durch die rotierenden
Wellenfunktionen $\psi_h+\delta\psi$ ersetzt.  Die Operatoren sind 
sind dann Funktionen der kollektiven Variablen, welche die 
Rotation beschreiben. Nimmt man deren Matrixelemente zwischen
geeignet konstruierten Nukleonwellenfunktionen, so findet man 
\be
 g_{A,T}^{0}= \frac{N_c}{2}\frac{1}{\Lambda_{tot}} \int dr\, r^2
 ( AG \pm \frac{1}{3}(C+2D)F)
\ee
f\"ur die Isosinglet-Operatoren, sowie
\be
g_{A,T}^{3}=\frac{N_c}{3} \int dr\, r^2 (G^2\mp\frac{1}{3}F^2)
\ee
f\"ur die Isovektor-Kopplungen des Nukleons. Die Ergebnisse finden
sich in den Abbildungen 3 und 4 sowie in Tabelle 2. Sie befinden 
sich in der Tat in besserer \"Ubereinstimmung mit den Resultaten
aus dem chiralen Bagmodell als die mit Hilfe der Peierls-Yoccoz
Projektion gefundenen Werte. F\"ur den bevorzugten Parametersatz
$g=5.38,\lambda =10$ finden wir $\DEop=0.6\su$. Auch die axialen
Kopplungen des Nukleons sind mit $g_A^0=0.15$ und $g_A^3=1.41$ 
deutlich kleiner und damit n\"aher an ihren experimentellen 
Werten. 



\section{Zusammenfassung}  
Wir haben in diesem Kapitel die symmetriebrechende Amplitude 
in verschiedenen chiralen Modellen des Nukleons bestimmt. 
Je nach verwendetem Parametersatz haben sich dabei Werte
in dem Bereich $\DEop =(0.0-1.1)\su$ ergeben. Die Resultate
zeigen im Rahmen der untersuchten Modelle eine deutliche 
Korrelation mit der Flavorsinglet-Axialvektorkopplungkonstante
des Nukleons. In dieser Eigenschaft sind die betrachteten 
Modelle einfachen Konstituendenbildern des Nukleons verwandt,
allerdings mit einem deutlich reduzierten $g_A^0$. 

Insgesamt scheint eine gute ph\"anomenologiosche Beschreibung 
der Eigenschaften des Nukleons nur mit symmetriebrechenden
Amplituden $\DEop <0.5 \su$ vertr\"aglich zu sein. Die 
signifikante Unterdr\"uckung dieser Amplitude im Vergleich
zur naiven Absch\"atzung $\DEop = 1.6\su$ aus dem Kapitel
2 ist unser wesentliches Resultat. 
