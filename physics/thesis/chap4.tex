\chapter{Chirale Modelle des Nukleons}
%revised Jan. 2, 1992 
Wir haben in  Kapitel 2 verschiedene Korrekturen zum
Niederenergietheorem f\"ur die Pionphotoproduktion untersucht. 
Dabei haben wir festgestellt, da\ss\ die explizite Brechung der chiralen 
Symmetrie durch die Quarkmassen in der QCD-Lagrangedichte 
im Fall neutraler Pionen einen wesentlichen Beitrag 
zur elektrischen Dipolamplitude liefern kann.  Diese Feststellung
beruht allerdings auf der Verwendung eines nichtrelativistischen
Quarkmodells des Nukleons, um Matrixelemente des symmetriebrechenden Terms 
abzusch\"atzen. Solche Modelle verletzen die chirale Symmetrie bereits im
Ansatz, so da\ss\ das Resultat aus Abschnitt 2.5, $\Delta E_{0+}
(\pi^0p)=1.6\su$, mit Vorsicht zu betrachten ist. Wir wollen in diesem 
Kapitel versuchen, mit Hilfe chiraler Modelle des Nukleons eine verbesserte 
Bestimmung der symmetriebrechenden Amplitude $\Delta E_{0+}(\pi^0N)$ 
zu gewinnen.

\section{Die Spinstruktur des Nukleons}
Bei der Untersuchung geeigneter Nukleonmodelle wollen wir gro\ss en Wert 
auf eine genaue Beschreibung anderer Matrixelemente des Nukleons legen.
Besonderes Gewicht werden wir in diesem Zusammenhang der axialen Struktur
des Nukleons geben. Die sym\-me\-trie\-brechende Amplitude 
$\Delta E_{0+}$ ist im wesentlichen durch
Matrixelemente eines Flavorsingletstroms bestimmt. Wir wollen daher im 
ersten Abschnitt dieses Kapitels eine kurze Diskussion der experimentellen 
Informationen \"uber die Axialvektorkopplung im Flavorsinglet-Kanal geben.

Im Jahre 1987 berichtete die EMC-Gruppe \cite{EMC89} \"uber eine 
Messung der Spinasymmetrie 
\be
  A = \frac{\sigma (\mu\!\uparrow p\!\uparrow)
            -\sigma (\mu\!\uparrow p\!\downarrow)}
	    {\sigma (\mu\!\uparrow p\!\uparrow)
            +\sigma (\mu\!\uparrow p\!\downarrow)}    
\ee
in der tief-inelastischen Streuung polarisierter Myonen an polarisierten 
Protonen. Dabei bezeichnet $\sigma (\mu\!\uparrow p\!\uparrow\downarrow)$ 
den differentiellen Wirkungsquerschnitt $d\sigma/(dQ^2dx)$ f\"ur die 
Streuung longitudinal polarisierter Myonen an Protonen mit parallel 
bzw.~antiparallel ausgerichtetem Spin. Die Asymmetrie ist f\"ur Werte 
der Bjoerkenvariable $x$ im Bereich $0.01<x<0.7$ bestimmt worden. Die 
entsprechenden Impuls\"ubertr\"age liegen zwischen $Q^2=3.5\,\gev^2$
bei $x=0.01$ und $Q^2=29.5\,\gev^2$ bei $x=0.7$. Im kinematischen 
Bereich des EMC-Experiments ist die Asymmetrie in guter N\"aherung durch das 
Verh\"altnis der spinabh\"angigen Strukturfunktion $g_1^p$ und der 
unpolarisierten Strukturfunktion $F_1^p$
\be
  A =\frac{g_1^p(x,Q^2)}{F_1^p(x,Q^2)}
\ee
gegeben. Die Operatorproduktentwicklung liefert eine Vorhersage f\"ur das 
erste Moment von $g_1^p$ \cite{JM90}: 
\be
 \int_0^1 g_1^p(x,Q^2)dx =\frac{1}{2} \left( \frac{4}{9}\Delta u
   +\frac{1}{9}\Delta d +\frac{1}{9}\Delta s \right)
   \left( 1-\frac{\alpha_s}{\pi} + {\cal O}(\alpha_s^2) \right) \; .
\ee       
Die Spinasymmetrie $\Delta q$  der Quarkflavors $q=u,d,s$ ist durch
Matrixelemente des Axialvektorstroms definiert
\be
   <N(p)|\bar{q}\gamma_\mu\gamma_5 q|N(p)>=\Delta q s_\mu\; ,
\ee    	    
wobei $s_\mu=\bar u(p)\gamma_\mu\gamma_5u(p)$ den kovarianten Spin des 
Nukleons bezeichnet. Mit Hilfe schwacher Zerf\"alle und der 
$SU(3)$-Flavorsymmetrie lassen sich zwei Kombinationen von
$\Delta u,\Delta d$ und $\Delta s$ experimentell festlegen
\beq
  g_A^3 &=& \Delta u-\Delta d = 1.255\pm 0.006  \\
  g_A^8 &=& \Delta u+\Delta d -2\Delta s = 0.6 \pm 0.1
\eeq
Das Resultat des EMC-Experiments, $\int g_1^p dx = 0.114\pm 0.036$, 
liefert eine weitere Kombination von $\Delta u,\Delta d,\Delta s$ und
erm\"oglicht die individuelle Bestimmung der $\Delta q$
\be
  \Delta u  =    0.74\pm 0.10  \hspace{0.5cm}
  \Delta d  =  - 0.54\pm 0.10  \hspace{0.5cm}
  \Delta s  =  - 0.20\pm 0.11  \; .
\ee
\"Uberraschend ist der starke Polarisationsgrad der seltsamen Quarks
im Nukleon. Dar\"uber hinaus implizieren diese Zahlen  einen
sehr kleinen Wert f\"ur die 
Flavorsinglet-Axial\-vek\-tor\-kopplungs\-kon\-stan\-ten des Nukleons
\be
 g_A^{\em sing} = \Delta u + \Delta d+\Delta s = 0.01\pm 0.29 \; .
\ee
Diese Gr\"o\ss e ist ein Ma\ss\ f\"ur den Beitrag der Quarkspins 
zum Spin des Nukleons. Nach dem  EMC-Experiment ist dieser Wert 
mit Null vertr\"aglich. Auch der Anteil der leichten Quarks am 
Nukleonspin $g_A^0 = \Delta u + \Delta d=0.20$ ist erstaunlich 
gering.

Ber\"ucksichtigt man in der Operatorproduktentwicklung die 
Effekte der $U(1)_A$-Anomalie, so ergibt sich ein zus\"atzlicher
gluonischer Beitrag $\Delta \Gamma =\alpha_s/(2\pi)\Delta g$ 
zu den Spinasymmetrien $\Delta q$ \cite{AR88}. Dabei bezeichnet 
$\Delta g=g_+-g_-$ das erste Moment der Differenz der polarisierten
Gluonverteilungen mit Spins parallel bzw.~antiparallel zum Spin des
Protons. Eine obere Grenze f\"ur $\Delta g$ ergibt sich aus der
unpolarisierten Gluonverteilung. Altarelli und Stirling \cite{AS89}
verwenden die Absch\"atzung $\Delta \Gamma = 0.1$ und finden
$g_A^{\em sing}=0.3$ sowie $g_A^0=0.4$.   

\section{Das chirale Bagmodell}
Das chirale Bagmodell beschreibt das Nukleon als ein System
masseloser Quarks, die in einer sph\"arischen Kavit\"at vom Radius 
$R$ eingeschlossen sind. An der Bagoberfl\"ache koppeln die Quarks an 
Mesonenfelder, die die Struktur des Nukleons bei gr\"o\ss eren
Abst\"anden bestimmen . Die  Lagrangedichte des Modells lautet 
\be
\label{LCB}
{\cal L}= \left( \bar{\psi}\frac{i}{2}\gamma^{\mu}\stackrel{\leftrightarrow}
{\partial}_{\mu}\psi -B\right) \Theta (R-r) - \frac{1}{2} \bar{\psi}
e^{i\gamma_5 \vec{\tau}\cdot\vec{\phi}}\psi\delta (r-R)
+{\cal L}_{mes}\Theta (r-R) \; .
\ee
Wir beschr\"anken uns hier auf Flavor-$SU(2)$, so da\ss\ $\psi=(u,d)$ ein 
Isodoublet und $\vec{\phi}$ das Isotriplet Pionfeld bezeichnet. Die Struktur 
des Kopplungsterms ist durch die Forderung nach chiraler 
$SU(2)_L\times SU(2)_R$-Invarianz der Wechselwirkung diktiert.
Um die Beschreibung der elektromagnetischen Eigenschaften des 
Nukleons zu verbessern, enth\"alt das Modell neben den Pionfeldern
auch die Vektormesonen $(\rho,a_1,\omega)$ als zus\"atzliche 
Freiheitsgrade. Die detaillierte Form der mesonischen Lagrangedichte 
${\cal L}_{mes}$ sowie die resultierenden Bewegungsgleichungen finden 
sich in \cite{HTW90}. Wir haben die mesonischen Parameter $f_\pi=93$ MeV 
und $g_{\rho\pi\pi}=5.85$ durch ihre empirischen Werte fixiert. 

Die Bewegungsgleichungen werden mit Hilfe eines Ansatzes maximaler 
Symmetrie, des ''Hedgehog``-Ansatzes 
\be
\label{hedge}
\vec{\phi}(\vec{r}) = \hat{r} f_\pi \Theta (r)
\ee
f\"ur das Pionfeld gel\"ost. Die dimensionslose Funktion $\Theta (r)$
bezeichnet man als chiralen Winkel. Auf Grund des Hedgehog-Ansatzes 
kommutiert der Hamiltonoperator nicht mit dem Drehimpuls- oder
Isospinoperator. Quarkzust\"ande lassen sich daher lediglich durch ihren 
Grandspin $\vec{G}=\vec{L}+\vec{S}+\vec{I}$ und ihre Parit\"at $\pi$
klassifizieren. Wellenfunktionen mit gutem Spin und Isospin werden mit 
Hilfe der semiklassischen Crankingmethode \cite{KJR86} konstruiert. 

Wir kommen nun zur Bestimmung der Matrixelemente des Kommutators
$[\dot Q_5^a,V_i^{em}]$ im chiralen Bagmodell. Im Au\ss enraum der 
Kavit\"at ist
\be
  \dot Q_5^a =  f_\pi m_\pi^2\int d^3r\,\pi^{a}(\vec r\,)\; ,
\ee
wobei $\pi^{a}=\frac{f_\pi \sin\Theta}{3}\tau^{a}\vec{\sigma}\cdot
\hat{r}$ das kanonische Pionfeld bezeichnet. Der Kommutator 
$[\dot Q_5^a,V_i^{em}]$ verschwindet dann aus den bereits im
Zusammenhang mit dem linearen $\sigma$-Modell diskutierten Gr\"unden.
Im Innern des Bags sind die Quarks frei, und wir erhalten wie in 
Abschnitt 2.5
\be
\label{deleop}
\Delta E_{0+}(\pi^0p) = \frac{e}{4\pi f_\pi}\frac{\overline{m}}{m_\pi (1+\mu)}
  \left\{ \left( 1+\frac{\delta m}{6\overline{m}} \right) g_T^0
     + \left(\frac{1}{3}+\frac{\delta m}{2\overline{m}}\right) g_T^3
     \right\} \; ,
\ee
wobei die Tensorkopplungskonstanten im chiralen Bagmodell zu bestimmen 
sind. F\"ur die Isosingletkopplung ergibt sich
\be
\label{gt0cb}
g_T^{0}=\frac{T_{\overline{\sigma}\tau}}{2\Lambda_{tot}}\; ,
\ee
wobei $\Lambda_{tot}$ das Tr\"agheitsmoment des Solitons bezeichnet
\cite{HTW90} und $T_{\overline\sigma\tau}$ durch die Teilchen-Loch
Anregungen des Hedgehog-Grundzustands bestimmt ist. Die detaillierte
Berechnung dieser Gr\"o\ss e findet sich in Anhang C. Die 
Isovektorkopplung hat keinen gro\ss en Einflu\ss\ auf die
Amplitude $\Delta E_{0+}(\pi^0p)$. Wir beschreiben die Bestimmung
von $g_T^3$ in der Ver\"offentlichung \cite{SW90}.

\begin{figure}
\caption{Korrekturen zur elektrischen Dipolamplitude auf Grund expliziter 
Symmetriebrechung im chiralen Bagmodell.}
\vspace{9cm}
\end{figure}
\begin{figure}
\caption{Axiale Kopplungskonstanten  des Nukleons im chiralen Bagmodell.}
\vspace{9cm}
\end{figure}

Die Ergebnisse f\"ur die Korrektur $\Delta E_{0+}$ im chiralen Bagmodell 
finden sich in Abbildung 4.1. Sie h\"angen stark vom
verwendeten Bagradius ab. Insbesondere verschwindet $\Delta E_{0+}$
im Grenzfall $R\to 0$, was dem Resultat in einem rein mesonischen 
Solitonmodell entspricht. Die starke Abh\"angigkeit vom Bagradius
widerspricht an und f\"ur sich einer der grundlegenden Ideen des chiralen 
Bagmodells, da\ss\ n\"amlich physikalische Observable in gewissen Umfang 
unabh\"angig von der Realisierung der Dynamik mit Hilfe
von mesonischen oder Quark-Gluon Freiheitsgraden sein sollte. Sie mag 
daher ein Hinweis auf das Fehlen wichtiger gluonischer Beitr\"age in 
unserer Beschreibung sein.
     
%\begin{table}
%\caption{Korrekturen zur elektrischen Dipolamplitude auf Grund expliziter 
%Symmetriebrechung im chiralen Bag Modell.} 
%\begin{center}
%\begin{tabular}{|c||c|c|c|}\hline
% R [fm]   & $\DEop$         & $\DEon$       & $\DEcn$        \\ \hline\hline
% 0.30     &   0.07          &    0.09       &   0.16         \\ 
% 0.52     &   0.22          &    0.20       &   0.37         \\  
% 0.75     &   0.51          &    0.28       &   0.56         \\  
% 1.00     &   0.89          &    0.27       &   0.61         \\ 
% 1.30     &   1.14          &    0.25       &   0.62        \\ \hline
%\end{tabular}
%\end{center}
%\end{table} 

Eine \"ahnlich starke Abh\"angigkeit vom Bagradius zeigt auch die
axiale Kopplung im Isosinglet-Kanal, siehe Abbildung 4.1.  F\"ur den 
in ph\"anomenologischen Anwendungen des Modells gew\"ohnlich bevorzugten Wert 
des chiralen Winkels $\Theta (R)=
\frac{\pi}{2}\;(R=0.52\,\rm fm)$ ergibt sich $g_A^0=0.12$. 
Konsistent mit dem  EMC-Resultat $g_A^0<0.4$ sind Bagradien bis etwa
$R=0.7$ fm. Das entspricht einer oberen Grenze f\"ur die
symmetriebrechende Amplitude $\Delta E_{0+}(\pi^0p)<0.6 \su$.    

\section{Das nichttopologische Solitonmodell}
Um die Modellabh\"angigkeit von $\Delta E_{0+}$ n\"aher zu untersuchen,
wollen wir den sym\-metrie\-brechenden Kommutator in einem 
nichttopologischen Solitonmodell \cite{BB85} untersuchen. 
Auch dieses Modell beschreibt das Nukleon als ein System von
Quarks, umgeben von einer mesonischen Polarisationswolke.
Im Gegensatz zum chiralen Bagmodell verzichtet das 
nichttopologische Solitonmodell auf eine r\"aumliche Trennung
zwischen diesen Freiheitsgraden. Das Modell basiert auf der
Lagrangedichte des linearen $\sigma$-Modells 
\beq
\label{linsig}
  {\cal L} & = & \bar  \psi [i
\partial  _\mu \gamma _\mu + g ( \sigma  + i \vec \tau \cdot \vec
\pi \gamma  _5 ) ] \psi \nonumber  \\
 & & + {1 \over 2} (\partial
_\mu  \sigma)^2  + {1 \over  2} (\partial  _\mu  \vec  \pi )^2  -
{\lambda^2  \over  4} 
 ( \sigma  ^2 + \vec  \pi  ^2  - \nu  ^2 )^2
 + {\cal L}_{sb}\; ,
\eeq	
wobei $\psi$ wie oben das Doublet der leichten Quarks bezeichnet, die
Mesonfelder $\sigma$ und $\vec\pi$ sich im Gegensatz zum chiralen 
Bagmodell aber nach einer linearen Darstellung von $SU(2)_L\times
SU(2)_R$ transformieren. 

Die Parameter $g$ und $\lambda$ bezeichnen die Quark-Meson und
Meson-Meson Kopplungskonstanten. Die Pionzerfallskonstante 
$f_\pi=93$ MeV und die Pionmasse $m_\pi=139.6$ MeV sind durch
ihre experimentellen Werte bestimmt. Dar\"uber hinaus ist
$\nu^2=f_\pi^2 -m_\pi^2/\lambda^2$, so da\ss\ die chirale 
Symmetrie im Grundzustand durch einen nichtverschwindenden 
Erwartungswert $\langle\sigma\rangle=-f_\pi$ spontan gebrochen ist. 
\begin{figure}
\caption{Korrekturen $\DEop$ (durchgezogene Linie) und $\DEon$
(gestrichelte Linie) im nichttopologischen Solitonmodell.
Die Ergebnisse des Peierls-Yoccoz Verfahrens und der 
semiklassischen Quantisierung sind mit PY bzw.~CM bezeichnet.}
\vspace{9cm}
\end{figure}
\begin{figure}
\caption{Axiale Kopplungskonstanten im nichttopologischen
Solitonmodell. Die Ergebnisse des Peierls-Yoccoz Verfahrens und der 
semiklassischen Quantisierung sind mit PY bzw.~CM bezeichnet.}
\vspace{9cm}
\end{figure}

Die explizite Brechung der chiralen Symmetrie ber\"ucksichtigen 
wir in der Lagrangedichte durch den Beitrag ${\cal L}_{sb} = 
-f_\pi m_\pi^2\sigma -\bar\psi M\psi$ \cite{JJP89}. Dabei enth\"alt das 
skalare Feld den nichtperturbativen  Anteil des Quarkkondensats,
w\"ahrend der zweite Term nur im Valenzquark-Sektor zu 
ber\"ucksichtigen ist. Wie im letzten Abschnitt tragen nur
die Quarkfelder zu dem Kommutator $[\dot Q_5^a,V_i^{em}]$ bei. 
Mit Hilfe der kanonischen Vertauschungsrelationen zwischen den
Fermionfeldern erhalten wir erneut das Resultat (\ref{deleop}),
wobei die Tensorkopplungen $g_T^0$ und $g_T^3$ im nichttopologischen
Solitonmodell zu bestimmen sind.


Zu diesem Zweck haben wir Wellenfunktionen des Nukleons mit Hilfe
der Peierls-Yoccoz-Projektion konstruiert. Dieses Verfahren ist
in Anhang D erl\"autert. In den Abbildungen 4.3 und 4.4 zeigen wir 
die Ergebnisse f\"ur $g_A$ und $\Delta E_{0+}(\pi^0N)$ als Funktion der 
Quark-Meson Kopplung $g$. F\"ur hinreichend gro\ss e Werte der Meson-Meson 
Kopplung sind die Resultate praktisch unabh\"angig von $\lambda$. 
Die korrekte Masse des Nukleons ergibt sich f\"ur die Wahl
$g=5.38$ und $\lambda =10$.  In diesem Fall ist $\DEop=0.9\su$,
ein Wert, der noch im Bereich eines relativistischen Konstituentenmodells
liegt. Man sollte allerdings beachten, da\ss\ im Rahmen der 
gew\"ahlten Methode auch die axialen Kopplungen des Nukleons
mit $g_A^0=0.45$ und $g_A^3=1.64$ recht gro\ss e Werte annehmen.
Dies mag ein Hinweis auf die Tatsache sein, da\ss\ das Modell 
generell die Rolle der Valenzquarks in der Struktur des Nukleons
\"ubersch\"atzt oder gewisse Beitr\"age im mesonischen bzw.~Quarksektor
der Theorie mehrfach ber\"ucksichtigt.


Die im Vergleich zum chiralen Bagmodell abweichenden Ergebnisse 
k\"onnen nat\"urlich auch in den unterschiedlichen 
Projektionsverfahren begr\"undet sein. Um diese Frage n\"aher 
zu untersuchen, haben wir die  semiklassische
Quantisierung auch auf das nichttopologische Solitonmodell
angewendet. Diese Methode ist in den Ver\"offentlichungen
\cite{SW91,CB86} n\"aher beschrieben.


In dieser Arbeit wollen wir lediglich die Ergebnisse kurz 
diskutieren. Sie sind den mit CM gekennzeichneten Kurven
in den Abbildungen 4.3 und 4.4 zu entnehemen. Die Resultate befinden 
sich in der Tat in besserer \"Ubereinstimmung mit den  Ergebnissen
aus dem chiralen Bagmodell als die mit Hilfe der Peierls-Yoccoz
Projektion gefundenen Werte. F\"ur den bevorzugten Parametersatz
$g=5.38,\lambda =10$ finden wir $\DEop=0.6\su$. Auch die axialen
Kopplungen des Nukleons sind mit $g_A^0=0.15$ und $g_A^3=1.41$ 
deutlich kleiner und damit n\"aher an ihren experimentellen 
Werten. 

%\begin{table}
%\caption{Korrekturen zur elektrischen Dipolamplitude auf Grund expliziter
%Symmetriebrechung im nichttopologischen Solitonmodell.}
%\begin{center}
%\begin{tabular}{|c||c|c|c||c|c|c|}\hline
%          & \multicolumn{3}{|c||}{Peierls-Yoccoz Projektion} 
%	  & \multicolumn{3}{c|}{semiklassische Projektion}  \\  
%   g      & $\DEop$         & $\DEon$        & $\DEcn$
%          & $\DEop$         & $\DEon$        & $\DEcn$\\ \hline\hline
%   4      &   1.09 & 0.84 & 0.99 & 1.03 & 0.86 & 0.66 \\ 
%   5      &   0.95 & 0.72 & 0.92 & 0.68 & 0.52 & 0.63 \\ 
%   6      &   0.89 & 0.66 & 0.88 & 0.53 & 0.37 & 0.61 \\ 
%   7      &   0.85 & 0.63 & 0.85 & 0.43 & 0.28 & 0.59 \\ 
%   7.8    &   0.82 & 0.61 & 0.83 & 0.39 & 0.24 & 0.58 \\ \hline
%\end{tabular}   
%\end{center}
%\end{table}

\section{Zusammenfassung}  
Wir haben in diesem Kapitel die symmetriebrechende Amplitude 
in verschiedenen chiralen Modellen des Nukleons bestimmt. 
Je nach verwendetem Parametersatz haben sich dabei Werte
in dem Bereich $\DEop =(0-1.1)\su$ ergeben. Die Resultate
zeigen im Rahmen der untersuchten Modelle eine deutliche 
Korrelation mit der Flavorsinglet-Axialvektorkopplung
des Nukleons. In dieser Eigenschaft sind die betrachteten 
Modelle einfachen Konstituentenbildern des Nukleons verwandt,
allerdings mit einem deutlich reduzierten $g_A^0$. 

Insgesamt scheint eine gute ph\"anomenologische Beschreibung 
der Eigenschaften des Nukleons nur mit symmetriebrechenden
Amplituden $\DEop <0.5 \su$ vertr\"aglich zu sein. Diese Korrektur
ist deutlich kleiner als die im Rahmen eines nichtrelativistischen
Quarkmodells gewonnene Ab\-sch\"a\-tzung $\DEop=1.6\su$. 
