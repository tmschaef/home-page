\chapter{Analyse der Spektralfunktionen}
% revised Jan. 1, 1992
In diesem Kapitel wollen wir zun\"achst auf die Bestimmung der 
Spektralfunktionen aus experimentellen Wirkungsquerschnitten eingehen.
Das Spektrum der Vektormesonen entnehmen wir der $e^+e^-$-Annihilation
w\"ahrend die spektrale Verteilung der Axialvektormesonen aus 
hadronischen Zerf\"allen des $\tau$-Leptons bestimmt werden kann.
Wir benutzen die gewonnenen Spektralfunktionen, 
um die G\"ultigkeit der im letzten Kapitel angegebenen Summenregeln
zu testen und die enthalten Vakuumparameter zu bestimmen. Im
letzten Abschnitt schlie\ss lich kehren wir die Vorgehensweise
um und verwenden Summenregeln, um Vorhersagen \"uber das 
Axialvektorspektrum jenseits der Masse des $\tau$-Leptons zu 
machen.   

\section{Bestimmung der Spektralfunktion im Vektorkanal}
Die Spektralfunktion im Vektorkanal l\"a\ss t sich mit Hilfe 
des optischen Theorems aus dem totalen Wirkungsquerschnitt
f\"ur die Annihilation von $e^+e^-$-Paaren in Hadronen 
mit dem Isospin $I$=1 bestimmen
\be
\label{opttheo}
\rho_V(s) \equiv \frac{1}{\pi}{\rm Im}\Pi^V(s) = \frac{s}{16\pi^3\alpha^2}
   \,\sigma (e^+e^-\to I=1) .
\ee
Diese Relation l\"a\ss t sich vereinfachen, indem man den
hadronischen Wirkungsquerschnitt auf den rein     
elektromagnetischen Proze\ss\ $e^+e^-\to\mu^+\mu^-$
normiert
\be
\label{rhov}
 \rho_V(s) =  \frac{1}{12\pi^2}\, R^{I=1}(s)\, ,
\ee
wobei $R^{I=1}$ das Verh\"altnis $\sigma (e^+e^-\to I=1)/
\sigma (e^+e^-\to\mu^+\mu^-)$ bezeichnet. Dieser Quotient
ist an einer Reihe von Speicherringen mit gro\ss er 
Genauigkeit vermessen worden.

Zwischen der Schwelle $\sqrt s=2m_\pi$ und $\sqrt s=1\,\gev$ wird
$R^{I=1}(s)$ von der resonanten $\pi^+\pi^-$-Produktion dominiert.
In diesem Bereich l\"a\ss t sich die Spektralfunktion mit Hilfe
der Beziehung 
\be
 R^{I=1}(s) = \frac{1}{4} |F_\pi (s)|^2 \left( 
       1-\frac{4m_\pi^2}{s} \right)^{3/2}
\ee       
aus dem elektromagnetischen Formfaktor des Pions extrahieren. 
Pr\"azise Ergebnisse f\"ur den Formfaktor im zeitartigen Bereich 
$\sqrt s<1$ GeV liegen aus Orsay vor \cite{Que78}. Die gewonnenen 
Daten werden gew\"ohnlich mit Hilfe eines Gounaris-Sakurai Fits
beschrieben \cite{Hoe83}. Dieses Modell basiert auf einer 
Parametrisierung der $\pi\pi$-Endzustandswechsel\-wir\-kung, die
f\"ur die analytische Struktur des Pionformfaktors verantwortlich ist.
F\"ur unsere Zwecke gen\"ugt bereits eine noch einfachere 
Beschreibung der Daten, die in Anhang E skizziert ist.
\begin{figure}
\caption{Verh\"altnis $R^{I=1}$ als Funktion der Schwerpunktsenergie. 
Die durchgezogene Linie stellt die von uns verwendete 
Parametrisierung dar.}
\vspace{9cm}
\end{figure}

Im Bereich zwischen  $\sqrt s=1$ GeV und $\sqrt s=2$ GeV ist 
$R^{I=1}(s)$ durch eine Reihe breiter Resonanzen bestimmt. Bis
$\sqrt s=1.4$ GeV verwenden wir die Novosibirsk-Daten \cite{Sid76}
f\"ur die Kan\"ale $e^+e^-\to \pi^+\pi^-\pi^+\pi^-$ und $e^+e^-\to 
\pi^+\pi^-\pi^0\pi^0$. Jenseits von $\sqrt s=1.4$ GeV benutzen wir 
Ergebnisse aus Frascati \cite{Bac79} f\"ur die Reaktion $e^+e^-\to
(\ge 3\pi )$. In beiden F\"allen addieren wir den Kanal $e^+e^-\to\pi^+\pi^-$
aus den Orsay-Daten bei h\"oheren Energien \cite{Bis89}. Das Resultat 
zeigt eine deutlich erkennbare Struktur bei der Schwerpunktsenergie 
1.6 GeV. In dieser \"Uberh\"ohung lassen sich mit Hilfe pr\"aziser
Daten aus dem Pionformfaktor  die Beitr\"age
zweier \"uberlappender Resonanzen, $\rho(1450)$ und $\rho(1700)$, 
isolieren.  

Bei noch h\"oheren Energien zeigt der totale Wirkungsquerschnitt 
$\sigma (e^+e^-\to \rm hadr.)$ vor allem die Schwellen f\"ur schwere
Quarkproduktion sowie die Beitr\"age der Vektormesonen 
$J/\psi,\psi ',\ldots$. Es wird allerdings zunehmend schwieriger, den 
$I$=1 Anteil des Wirkungsquerschnitts zu  identifizieren. Wir 
benutzen daher f\"ur $\sqrt s>2$ GeV das st\"orungstheoretische 
Resultat 
\be
 R^{I=1}(s)= \frac{3}{2}\left( 1+\frac{\alpha_s}{\pi}
  + {\cal O}(\alpha_s^2) \right) .
\ee
Die verwendeten Daten f\"ur $R^{I=1}(s)$ im Bereich bis $\sqrt s=2$
GeV sowie die beschriebene Parametrisierung finden sich  in Abbildung 6.1.
F\"ur $\sqrt s<1$ GeV liefern diese Resultate eine sehr genaue 
Bestimmung der Spektralfunktion. F\"ur h\"ohere Energien sind die
Daten allerdings mit einer Reihe von Unsicherheiten behaftet. So
\begin{itemize}
\item{beinhalten die Daten im Bereich $\sqrt s>1.4$ GeV auch den 
Kanal $\pi^+\pi^-\pi^0$, der im wesentlichen zum Isospin $I\!=\!0$
zu rechnen ist.}
\item{ist es nicht ohne weiteres m\"oglich, den nichtresonanten 
$I\!=\!0$-Untergrund aus den Kan\"alen $\pi^+\pi^-\pi^+\pi^-$ und 
$\pi^+\pi^-\pi^0\pi^0$ abzutrennen.}
\item{fehlen in unserer Analyse alle Kan\"ale, deren Endzust\"ande
 Kaonen oder Eta-Mesonen enthalten.}
\item{zeigt der totale Wirkungsquerschnitt f\"ur $\sqrt s<4$ GeV
eine Reihe von Strukturen, die auf $I\!=\!1$-Resonanzen zur\"uckzuf\"uhren
sein k\"onnten.}
\end{itemize}
W\"ahrend sich diese Unsicherheiten im Bereich bis $\sqrt s=2$ GeV 
in der Gr\"o\ss enordnung der experimentellen Fehler bewegen, ist 
das Fehlen isospinseparierter Daten in der Region jenseits von
$\sqrt s=2$ GeV durchaus problematisch f\"ur die Analyse der
Summenregeln.

\section{Bestimmung der Axialvektorspektralfunktion}
Die Spektralfunktion im Axialvektorkanal l\"a\ss t sich auf Grund der
Tatsache bestimmen, da\ss\ der Zerfall eines schweren Leptons $\tau$ 
in nichtseltsame Hadronen durch die $V\!-\!A$ Wechselwirkung
\be
\label{fermi}
 {\cal L} = -\frac{G}{\sqrt 2} \cos\theta_c \, 
    \bar u_\nu\gamma^\mu (1-\gamma_5) u_\tau\,
    (V_\mu^{1+i2}-A_\mu^{1+i2})
\ee
vermittelt wird. Dabei bezeichnet $G=1.16637\cdot 10^{-5}\,{\rm GeV}^{-2}$
die Fermikonstante und $\cos\theta_c=0.9744$ den Cosinus des Cabbibowinkels.
Mit der Wechselwirkung (\ref{fermi}) ergibt sich die partielle
Zerfallsbreite des $\tau$-Leptons in Hadronen mit der
invarianten Masse $q^2$ \cite{Tsa71,Oku82}      
\beq
\label{tauwid}
 \frac{d\Gamma (\tau\to^{\;s=0}\!{\rm hadr.}+\nu )}{dq^2} &=&
 \frac{G^2\cos^2\theta_c}{8\pi m_\tau^3} (m_\tau^2-q^2)^2
  \Big\{   m_\tau^2 \rho_A^{||} (q^2)\\
 & & \hspace{2.2cm}\mbox{}  +
 (m_\tau^2+2q^2)(\rho_V(q^2)+\rho_A(q^2) ) \Big\} \, .
   \nonumber
\eeq
Dieses Resultat bestimmt die Spektralfunktionen von der Schwelle
bis hinauf zur Masse des $\tau$-Leptons, $m_\tau=1.784$ GeV. In der 
Praxis ist  allerdings die Statistik auf Grund des kleiner werdenden 
Phasenraums bereits f\"ur invariante Massen $\sqrt{q^2}>1.5$ GeV 
unzureichend.

Um die verschiedenen Spektralfunktionen zu separieren, mu\ss\
man die m\"oglichen Endzust\"ande im $\tau$-Zerfall untersuchen. 
Ber\"ucksichtigt  man nur den Beitrag des Pions $\rho_A^{||}(s)= 
f_\pi^2 \delta (s-m_\pi^2)$ in der longitudinalen
Spektralfunktion, so ergibt sich das \"ubliche Resultat f\"ur 
die Zerfallsbreite $\tau\to\nu\pi$  
\be
 \Gamma (\tau\to\nu\pi ) = \frac{G^2\cos^2\theta_c}{8\pi}
    m_\tau^3f_\pi^2 \left( 1-\frac{m_\pi^2}{m_\tau^2} \right)^2 .
\ee
Die n\"achsth\"ohere Anregung mit den Quantenzahlen $J^\pi=0^+$
ist die $\pi (1300)$-Resonanz, die \"uberwiegend in drei Pionen zerf\"allt.
Dieser Zustand konnte im $\tau$-Zerfall noch nicht eindeutig 
identifiziert worden. Mit Hilfe von QCD-Summenregeln f\"ur den 
pseudoskalaren Korrelator gewinnt man jedoch folgende 
Absch\"atzung der Resonanzparameter \cite{GL82}
\be
  r_\pi = \frac{f_{\pi'}^2m_{\pi'}^4}{f_\pi^2 m_\pi^4}
         \simeq 8 .
\ee
Dieser Wert liefert ein Verzweigungsverh\"altnis $B(\tau\to\nu\pi')
=2\cdot 10^{-5}$, das um vier Gr\"o\ss enordnungen kleiner ist als
das gesamte Verzweigungsverh\"altnis f\"ur Zerf\"alle mit drei 
Pionen im Endzustand, $B(\tau\to\nu 3\pi)=1.1\cdot 10^{-1}$. Wir haben 
daher die longitudinale Spektralfunktion in unserer Analyse der
Zerf\"alle mit mehr als einem Pion im Ausgangskanal nicht ber\"ucksichtigt.


Die Vektor- und Axialvektoranteile der transversalen 
Spektralfunktion lassen sich in sehr guter N\"aherung trennen, indem man 
$\tau$-Zerf\"alle in eine gerade bzw.~ungerade Anzahl von Pionen 
betrachtet. Zerfallskan\"ale wie $\tau^\pm\to\nu\rho^\pm\to
\nu\pi^\pm\pi^+\pi^-$, die durch den Vektorstrom vermittelt werden,
aber Endzust\"ande mit drei Pionen liefern, haben nur 
ein verschwindend kleines Verzweigungsverh\"altnis. 

Messungen von $\tau$-Zerf\"allen sind an den Speicherringen 
DORIS \cite{Alb86} und PEP \cite{Ruc86,Ban87} mit Hilfe
der Reaktion $e^+e^-\to\tau^+\tau^-$ durchgef\"uhrt worden.
Wir verwenden die Daten der Argus-Kollaboration \cite{Alb86}
f\"ur den Zerfallskanal $\tau^\pm\to\nu\pi^\pm\pi^+\pi^-$. 
Es gibt keine experimentellen Informationen \"uber das 
Spektrum im anderen $3\pi$-Kanal $\tau^\pm\to\nu\pi^\pm\pi^0\pi^0$.
Eine Analyse der invarianten Massen im $2\pi$-System f\"ur die
Reaktion $\tau^\pm\to\nu\pi^\pm\pi^+\pi^-$ \cite{Alb86}
zeigt allerdings eine klare Pr\"aferenz f\"ur den resonanten 
Proze\ss\
\be
  \tau^\pm\to \nu a_1^\pm \to \nu\rho^0\pi^\pm \to \nu\pi^+\pi^-\pi^\pm .
\ee
F\"ur diese Zerfallssequenz lassen sich die nicht gemessenen 
partiellen Breiten mit Hilfe der Isospinrelation 
\be
 d\Gamma (\tau^\pm\to\nu\pi^\pm\pi^0\pi^0 ) =
 d\Gamma (\tau^\pm\to\nu\pi^\pm\pi^+\pi^- ) 
\ee
absch\"atzen. Diese Absch\"atzung ist konsistent mit den Resultaten der 
MAC-Kollaboration \cite{Ban87} f\"ur die totalen Verzweigungsverh\"altnisse 
$B(\tau^\pm\to\nu\pi^\pm\pi^+\pi^-)=7.0\pm 1.0$ \% und $B(\tau^\pm\to\nu
\pi^\pm\pi^0\pi^0)=8.7\pm 1.5$ \% .
\begin{figure}
\caption{Spektralfunktion $\rho_A(s)$ im Axialvektorkanal.
Die durchgezogene Linie stellt die von uns verwendete 
Parametrisierung dar.}
\vspace{9cm}
\end{figure}

Auch die Massenverteilungen der $5\pi$-Zerf\"alle des $\tau$-Leptons
sind experimentell nicht bestimmt. Das gemessene Verzweigungsverh\"altnis
$B(\tau^-\to 3\pi^-2\pi^+)=6.4\cdot 10^{-4}$ \cite{Alb88} ist jedoch
au\ss erordentlich klein, und es gibt theoretische Hinweise, da\ss\ diese
Aussage auch auf  andere $5\pi$-Zerf\"alle zutrifft \cite{GR85}.
Wir haben daher diese Kan\"ale in unserer Analyse vollst\"andig 
vernachl\"assigt. Die resultierende Spektralfunktion findet sich in 
Abbildung 6.2. Die gezeigte Parametrisierung besteht aus einer 
Breit-Wigner Funktion f\"ur den resonanten Teil, erg\"anzt durch 
ein Polynom zur Beschreibung des Untergrunds. 

Dieser Fit liefert als Nebenprodukt eine einfache Bestimmung der 
Parameter der $a_1$-Resonanz. Wir finden $m_{a_1}=1169$ MeV und
$\Gamma_{a_1}=552$ MeV, in guter \"Ubereinstimmung mit den 
Ergebnissen anderer $\tau$-Zerfallsexperimente, aber in deutlichem
Widerspruch zu den Standardwerten aus rein hadronischen Reaktionen,
$m_{a_1}=1260\pm 30$ MeV und $\Gamma_{a_1}=316\pm 45$ MeV \cite{PDG90}.
Diese Diskrepanz hat eine Reihe vergleichender Analysen der zur Verf\"ugung 
stehenden  Daten angeregt \cite{Bow86,VIO90}. Diese Untersuchungen
zeigen, da\ss\ man die Bestimmung der $a_1$-Masse im $\tau$-Zerfall
mit den Resultaten hadronischer Experimente in Einklang bringen
kann, wenn man energieabh\"angige Kopplungen f\"ur den Zerfall
$a_1\to 3\pi$ zul\"a\ss t. Dagegen bleibt die sehr gro\ss e Breite
des $a_1$ im $\tau$-Zerfall im Widerspruch zu rein hadronischen
Daten.
   
\section{Konsequenzen der Spektralfunktionen}
Wir haben die bestimmten Spektralfunktionen einer Reihe von 
Konsistenztests unterzogen. Zum einen kann man die Ergebnisse 
aus dem Vektorkanal verwenden, um die entsprechenden 
Verzweigungsverh\"altnisse im $\tau$-Zerfall zu bestimmen. 
Wir finden 
\beq
   B(\tau^\pm\to\nu\rho^\pm) &=& 21.8 \,\%  \, , \\
   B(\tau^\pm\to\nu 2n\pi  ) &=& 29.8 \,\%  \, ,
\eeq
in guter \"Ubereinstimmung mit den experimentellen Daten \cite{PDG90}.
Dar\"uber hinaus haben wir die G\"ultigkeit der Weinberg-Summenregeln
\cite{Wei67,PS87} in einem endlichen Intervall untersucht. In 
Abbildung 6.3 zeigen wir die Funktionen 
\begin{figure}
\caption{Test der Weinberg-Summenregeln in einem endlichen 
Energieintervall. Die Funktionen $\Delta_1(t_c),\Delta_2(t_c)$ 
sind im Text definiert. F\"ur die zweite Weinberg-Summenregel 
zeigen wir zum Vergleich den asymptotischen Wert $f_\pi^2$.}
\vspace{19.6cm}
\end{figure} 
\beq
 \Delta_1(t_c) &=& \int_0^{t_c} \big( \rho_V(s)-\rho_A(s)\big)\, ds \; ,\\
 \Delta_2(t_c) &=& \int_0^{t_c} \big( \rho_V(s)-\rho_A(s)\big)\,s\, ds  
\eeq
f\"ur verschiedene Werte des Cutoffs $t_c$. Im chiralen Limes liefert
die OPE keine Korrekturen zu den Stromalgebravorhersagen $\Delta_1
(t_c) \stackrel{t_c\to\infty}{\longrightarrow} f_\pi^2$ und $\Delta_2 (t_c)
\stackrel{t_c\to\infty}{\longrightarrow} 0$. Dagegen zeigen unsere
Ergebnisse, da\ss\ die St\"arke der Axialvektorspektralfunktion
in dem gemessenen Bereich nicht ausreicht, um die Weinberg-Summenregeln 
zu saturieren. Dies mag ein Hinweis auf die Tatsache sein, da\ss\ das
Spektrum im Axialvektorkanal auch jenseits der Masse des $\tau$-Leptons 
noch wichtige Strukturen zeigt.

Stromalgebratechniken erlauben auch die Bestimmung der elektromagnetischen
Massendifferenz der Pionen, $\Delta m_\pi=m_{\pi^\pm}-m_{\pi^0}$, aus
den Spektralfunktionen im Vektor- und Axialvektorkanal. Das Resultat 
von Das et al.~\cite{DGM67} lautet

\be
\label{dmpi}
 \Delta m_\pi (t_c) = \frac{3\alpha}{8\pi m_\pi f_\pi^2}
    \int_0^{t_c} \big( \rho_A(s) -\rho_V(s) \big) s\ln (s)\,ds\, ,
\ee
wobei das Integral auf Grund der zweiten Weinberg-Summenregel unabh\"angig
von der Skala im Logarithmus ist. Saturiert man die Summenregel mit
$\rho$ und $a_1$-Resonanzen verschwindender Breite, so ergibt sich das
klassische Ergebnis $\Delta m_\pi=\frac{3\alpha m_\rho^2 \ln 2}{4\pi m_\pi}
=5.15$ MeV. In Abbildung 6.4 zeigen wir die Resultate unter Verwendung
realistischer Spektralfunktionen. Die Massendifferenz $\Delta m_\pi (t_c)$
zeigt eine sehr starke Abh\"angigkeit vom Cutoff $t_c$. Bei $\sqrt {t_c} 
= 1.4$ GeV finden wir ein Plateau, bei dem das Ergebnis  $\Delta m_\pi (t_c)=
4.2$ MeV (exp. 4.6 MeV) eine sehr gute \"Ubereinstimmung mit dem 
experimentellen Resultat zeigt. Wie bei den Weinberg-Summenregeln 
dominiert f\"ur gr\"o\ss ere Cutoffs $t_c$ die Vektorspektralfunktion, und    
der Wert von $\Delta m_\pi$ wird deutlich zu klein.  

Eine weitere Gr\"o\ss e, die sensitiv auf die Form der Spektralfunktionen 
ist, ist der axiale Formfaktor im strahlungsbegleiteten Pionzerfall
$\pi^\pm\to e^\pm\nu\gamma$ \cite{BDL82}. Die zugeh\"orige Amplitude 
l\"a\ss t sich in die beiden Anteile $T_{SI}$ und $T_{SD}$ zerlegen. 
Die strukturunabh\"angige Amplitude $T_{SI}$ enth\"alt die Beitr\"age
der Leptonbremsstrahlung, des Pionpolterms sowie des Pionkontaktterms.
Diese Amplituden lassen sich ganz analog zur Photoproduktionsamplitude 
in Bornapproximation berechnen. Das Resultat ist strukturunabh\"angig 
in dem Sinne, da\ss\ die Amplitude als einzigen hadronischen Parameter 
die Pionzerfallskonstante $f_\pi$ enth\"alt. 
\begin{figure}
\caption{Bestimmung der Pionmassendifferenz $\Delta m_\pi$ und des axialen 
Pionformfaktors $F_A(0)$ in einem endlichen Energieintervall.}
\vspace{20cm}
\end{figure}

Die strukturabh\"angige Amplitude $T_{SD}$ l\"a\ss t sich auf Grund von
Eich- und Lorentzinvarianz mit Hilfe des Vektorformfaktors $F_V(s)$
und eines Axialvektorformfaktors $F_A(s)$ parametrisieren. Dabei bezeichnet
$s=k\cdot q$ das Produkt der Impulse des Pions und des abgestrahlten 
Photons. Der Vektorformfaktor ist durch die Zerfallsbreite $\pi^0\to 2\gamma$
bestimmt. Stromalgebra liefert eine Vorhersage f\"ur den axialen 
Formfaktor am weichen Punkt \cite{DMO67}
\be
\label{axpiff}
 F_A(0) = -\frac{m_\pi}{f_\pi} \int \frac{ds}{s}\, \left(
  \rho_A(s) -\rho_V(s) \right)  \; +\; \frac{1}{3}f_\pi m_\pi
   \langle r^2_\pi \rangle ,
\ee
wobei $\langle r^2_\pi\rangle=0.46\pm 0.01\,{\rm fm}^2$ \cite{Que78} den 
elektromagnetischen Radius des Pions bezeichnet. Experimentell lassen sich 
die beiden Formfaktoren bestimmen, indem man das Spektrum der 
Zerfallsprodukte f\"ur Photonenergien $E_\gamma>45$ MeV mi\ss t. 
In diesem Bereich dominiert der strukturabh\"angige Prozess
gegen\"uber der Bremsstrahlung, und man findet $F_A(0)=0.83\pm 0.14
\cdot 10^{-2}$ \cite{PDG90}.

Saturiert man die Summenregel (\ref{axpiff}) mit den niedrigsten
Resonanzen, so ergibt sich $F_A(0)=-\frac{3f_\pi m_\pi}{2m_\rho^2}
+\frac{1}{3}f_\pi m_\pi\langle r_\pi^2\rangle =1.8\cdot 10^{-2}$, w\"ahrend 
die realistischen Spektralfunktionen kleinere Formfaktoren liefern. 
Auf Grund der bei kleinen invarianten Massen konzentrierten 
Gewichtsfunktion zeigt sich eine im Vergleich zu $\Delta m_\pi$
deutlich geringere Abh\"angigkeit von $t_c$. Wie oben finden wir ein 
Plateau bei $\sqrt{t_c}=1.4$ GeV, wo $F_A(0)=0.56\cdot 10^{-2}$ 
gut mit dem experimentellen Wert \"ubereinstimmt.  

Zusammenfassend stellen wir fest, da\ss\ die bestimmten Spektralfunktionen
f\"ur kleine und mittlere invariante Massen bis etwa $\mu^2=2.5\,
\gev^2$ konsistent mit Stromalgebrasummenregeln sind. Dagegen scheint
die St\"arke der Axialvektorspektralfunktion f\"ur h\"ohere invariante
Massen nicht auszureichen, um stabile Resultate zu erzielen. 

\section{Borelquotient im Vektorkanal}
Wir kommen nun zur Bestimmung der Vakuumparameter mit Hilfe einer
Analyse der QCD-Summenregeln im Vektorkanal. Zu diesem Zweck
betrachten wir den Quotienten der Borelmomente der Spektralfunktion
\cite{LNT84}
\be
\label{bratio}
 R^V(\tau) \equiv \frac{\displaystyle \frac{1}{\pi}\int_{s_0}^{\infty}
    {\rm Im}\Pi^V(s) e^{-s\tau} s\, ds }
  {\displaystyle \frac{1}{\pi}\int_{s_0}^{\infty}
    {\rm Im}\Pi^V(s) e^{-s\tau}\, ds }\; .
\ee
Die Operatorproduktentwicklung liefert f\"ur diese Gr\"o\ss e
die theoretische Vorhersage
\be
\label{operatio}
 R^V(\tau) = \frac{1}{\tau}\, \big\{ 1+c_4\tau^2+c_6\tau^3+
    c_8\tau^4 + \ldots \big\} \; ,
\ee
wobei wir die Koeffizienten $c_i$ in (\ref{defcor}) definiert haben.
Der Quotient (\ref{bratio}) hat gegen\"uber den individuellen 
Momenten der Boreltransformierten Summenregel den Vorzug, da\ss\
in der OPE keine perturbativen Korrekturen zum Einheitsoperator
auftreten. Diese Korrekturen liefern in den einzelnen 
Momenten Beitr\"age von derselben Gr\"o\ss enordnung wie die
f\"uhrenden Kondensate, k\"urzen sich in dem Verh\"altnis 
(\ref{bratio}) aber heraus. Das theoretische Resultat (\ref{operatio})
besitzt daher eine nur sehr geringe Sensitivit\"at auf den 
Wert des Skalenparameters und die Gr\"o\ss e von perturbativen
Korrekturen h\"oherer Ordnung.
\begin{figure}
\caption{OPE-Parametrisierung des experimentellen Borelquotienten
nach Subtraktion des perturbativen Beitrags $1/\tau$. Die gestrichelte
Kurve zeigt das Resultat f\"ur das einfache Modellspektrum (5.28).}
\vspace{9cm}
\end{figure}

Mit Hilfe der im letzten Abschnitt bestimmten Spektralfunktionen 
l\"a\ss t sich der Borelquotient f\"ur beliebige Werte des 
Parameters $\tau$ bestimmen. Betrachtet man die Funktion
\be
\label{linratio}
 Y^V(\tau) \equiv R^V(\tau)-\frac{1}{\tau}
           = c_4 \tau + c_6\tau^2 + c_8 \tau^3 + \ldots\; ,
\ee
so reduziert sich die Bestimmung der Koeffizienten $c_i$
auf eine einfache Polynomregression. Um die Unsicherheiten
in der Bestimmung der Spektralfunktion zu ber\"ucksichtigen, haben 
wir das Borelverh\"altnis f\"ur verschiedene Parametrisierungen der 
Daten im Bereich der experimentellen Fehler berechnet. 
Die resultierende Variation der Funktion $Y^V(\tau)$ ist 
in Abbildung 6.5 dargestellt. Die Ergebnisse zeigen, da\ss\ der
relative Fehler nach Abzug des st\"orungstheoretischen 
Anteils $1/\tau$ recht erheblich ist. Man erkennt allerdings
auch, da\ss\ sich systematische Fehler in der Spektralfunktion
f\"ur Werte des Borelparameters $\tau\simeq 0.85\,{\rm GeV}^{-2}$
praktisch nicht auf das Borelverh\"altnis $R^V(\tau)$
auswirken. Diese Tatsache ist ein weiterer Vorzug des
Quotienten $R^V(\tau)$ gegen\"uber den individuellen
Momenten der Boreltransformierten Summenregel. 

Wir wollen zun\"achst eine allgemeine Diskussion der Funktion 
$Y^V(\tau)$ vornehmen. Zu diesem Zweck vergleichen wir in 
Abbildung 6.5 das experimentelle Resultat f\"ur $Y^V(\tau)$ mit
dem entsprechenden Resultat f\"ur die einfache Parametrisierung 
(\ref{zerow}) des  Spektrums als Summe einer Delta- und einer
Thetafunktion. Das Modellspektrum liefert zwar ein qualitativ
\"ahnliches Verhalten wie die korrekten Daten, liegt aber
nicht im Bereich der experimentellen Streuung. 

Beide Kurven haben die Eigenschaft, da\ss\ $Y^V(\tau)$ f\"ur 
$\tau\to 0$ gegen Null strebt. Entwickelt man den Borelquotienten 
in eine Potenzreihe in $\tau$, so zeigt sich, da\ss\ diese Tatsache
\"aquivalent mit der ersten FESR-Bedingung 
\be
\label{lfesr}
 t_cF_2(t_c) = 8\pi \int_{s_0}^{t_c} {\rm Im}\Pi^V(s) \, ds
\ee
ist. W\"ahrend wir diese Bedingung f\"ur das Modellspektrum
durch die Wahl der Kontinuumsschwelle $s_{th}$ erzwungen haben,
liefert die Forderung $Y^V(\tau)\stackrel{\tau\to 0}{\longrightarrow}0$
im Fall der experimentellen Spektralfunktion   einen wichtigen Test
f\"ur die Konsistenz der Daten mit der Dualit\"atsforderung. 
F\"ur $\sqrt{t_c}=2\,\gev$ betr\"agt die Diskrepanz zwischen der 
linken und rechten Seite der Summenregel (\ref{lfesr}) 
$\Delta = 0.43\,\gev^2$. Das bedeutet, da\ss\ f\"ur die verwendete 
Spektralfunktion die Dualit\"atsforderung  bis auf Abweichungen von 
der Gr\"o\ss enordnung 10\% erf\"ullt ist. 
\begin{figure}
\caption{Chi-Quadrat-Contour f\"ur die Bestimmung der Koeffizienten 
$c_4$ und $c_6$ aus dem Borelverh\"altnis im Vektorkanal. Das Minimum
liegt bei $\surd (\chi^2/NDF)=0.24$ und eine Contourlinie entspricht
$\Delta\surd (\chi^2/NDF)=18.5$.}
\vspace{9cm}
\end{figure}

F\"ur gr\"o\ss ere Borelparameter $\tau$ ist die Funktion $Y^V(\tau)$ 
zun\"achst negativ, hat aber eine positive Kr\"ummung und \"andert bei 
$\tau\simeq 0.6$ ihr Vorzeichen. Dieses Verhalten entspricht den aus
der theoretischen Absch\"atzung $c_4=-0.07\,\gev^4$ und $c_6=0.04\,\gev^6$ 
erwarteten Vorzeichen der f\"uhrenden Kondensate.

F\"ur die quantitative Analyse des Borelquotienten ist es von 
gro\ss er  Bedeutung, das Intervall $[\tau_{min},\tau_{max}]$
zu bestimmen, in dem man \"Ubereinstimmung zwischen der experimentellen 
und theoretischen Seite der  Summenregel fordert. Asymptotische
Freiheit garantiert die G\"ultigkeit der Summenregel im Grenzfall 
$\tau\to 0$. Da die Daten in diesem Bereich auf Grund der gro\ss en 
Fehler in jedem Fall nicht sehr stark gewichtet werden, setzen wir
die untere Grenze des Intervalls $\tau_{min}=0$. Die obere Grenze
ergibt sich aus der Forderung, da\ss\ sowohl die statistischen
Fehler auf Grund der Unsicherheit der Daten als auch die theoretischen 
Fehler, hervorgerufen durch das Abbrechen der OPE, minimal sind.  

Um das g\"unstige Verhalten des Borelquotienten bez\"uglich 
systematischer Fehler in der Spektralfunktion zu nutzen, sollte
$\tau_{max}>0.8\,{\rm GeV}^{-2}$ gew\"ahlt werden. Eine 
Absch\"atzung des theoretischen Fehlers ergibt sich, indem man
die Koeffizienten $c_4$ und $c_6$ f\"ur verschiedene vorgegebene
Werte des vernachl\"assigten Korrekturglieds $c_8$ bestimmt. Zu 
diesem Zweck haben wir eine obere Grenze f\"ur $|c_8|$ aus der 
Absch\"atzung \cite{CM90}
\be
  |c_8|_{max}  = {\rm max} \left\{ |c_4|m_{sc}^4,
    |c_6|m_{sc}^2 \right\}
\ee
gewonnen. Dabei bezeichnet $m_{sc}\simeq m_\rho$ eine f\"ur das
Zusammenbrechen der OPE charakteristische  Skala. Verwendet
man die von SVZ bevorzugten Werte (\ref{svzval}), so ergibt sich
$|c_8|_{max}\simeq 0.045\,{\rm GeV}^8$.    
\begin{table}
\caption{Bestimmung der Koeffizienten $c_i$ aus dem Borelquotienten
im Vektorkanal. Angegeben sind die Koeffizienten $c_4$ und $c_6$
sowie Absch\"atzungen der experimentellen und theoretischen
Unsicherheiten f\"ur verschiedene Werte des maximalen Borelparameters.}  
\begin{center}
\begin{tabular}{|c||c|c|c||c|c|c||}\hline
 $\tau_m\,[\gev^{-2}]$  &  $c_4\,[\gev^4]$  & $\Delta^{exp} c_4$ &
       $\Delta^{th} c_4$ &  $c_6\,[\gev^6]$  
	     & $\Delta^{exp} c_6$ & $\Delta^{th} c_6$  \\ \hline\hline
    0.8   &$-0.264$ & $\pm 0.150$        & $\pm 0.020$       &
             0.367  & $\pm 0.201$        & $\pm 0.061$   \\
    0.9   &$-0.099$ & $\pm 0.046$        & $\pm 0.030$       &
             0.148  & $\pm 0.053$        & $\pm 0.074$   \\	     
    1.0   &$-0.073$ & $\pm 0.032$        & $\pm 0.034$       &
             0.119  & $\pm 0.036$        & $\pm 0.079$   \\	     
    1.1   &$-0.054$ & $\pm 0.022$        & $\pm 0.039$       &
             0.096  & $\pm 0.025$        & $\pm 0.083$   \\	     
    1.2   &$-0.042$ & $\pm 0.016$        & $\pm 0.041$       &
             0.083  & $\pm 0.019$        & $\pm 0.086$   \\ \hline
\end{tabular}
\end{center}
\end{table}

In Tabelle 6.1 finden sich die Ergebnisse f\"ur die
Koeffizienten $c_4$ und $c_6$ sowie unsere Absch\"atzung der
theoretischen und experimentellen Fehler f\"ur verschiedene
Werte des maximalen Borelparameters $\tau_{max}$. Die Resultate
zeigen keine Stabilit\"at bez\"uglich der Wahl dieses Parameters,
so da\ss\ eine eindeutige Bestimmung der Kondensate nicht m\"oglich
ist. Verwendet man die oben angegebene Absch\"atzung von $|c_8|$,
so beginnt f\"ur
$\tau_{max}>1\,{\rm GeV}^{-2}$ der theoretische Fehler
in der Bestimmung der Koeffizienten gegen\"uber dem experimentellen
zu dominieren. Wir betrachten  diesen Wert daher als optimalen
Kompromi\ss\ zwischen den Forderungen nach ausreichender 
Bestimmtheit des Fits und Kontrolle \"uber den Abbruchfehler
in der OPE. Das entsprechende Resultat
\beq
\label{fitresult}
  c_4 &=&-0.073\pm 0.032\,{\rm GeV}^{4}, \\   
  c_6 &=& \spm 0.119\pm 0.036\,{\rm GeV}^{6}
\eeq
ist konsistent mit der Absch\"atzung von SVZ, beinhaltet allerdings
gro\ss e experimentelle Fehler. In Abbildung 6.6 zeigen wir die
$\chi^2$-Contour f\"ur diesen Fit. Man erkennt eine sehr starke 
Korrelation der beiden Koeffizienten $c_4$ und $c_6$. Das 
Verh\"altnis dieser beiden Gr\"o\ss en ist daher sehr viel besser
bestimmt als ihre absoluten Werte.   Diesen Effekt beobachtet
man auch bei der Variation der Ergebnisse mit dem maximalen 
Borelparameter, siehe Tabelle 6.1. Obwohl $c_4$ und $c_6$ eine
sehr starke Abh\"angigkeit von $\tau_{max}$ aufweisen, ist ihr 
Verh\"altnis praktisch konstant. 

Damit besteht die M\"oglichkeit einer genaueren Bestimmung 
von $c_6$, indem man zus\"atzliche Informationen \"uber 
die Gr\"o\ss e $c_6$ heranzieht. Verwendet man die Analyse
des Gluonkondensats im Charmoniumsystem, so ergibt sich der
im letzten Kapitel angegebene Wert $c_4=-0.07\,\gev^4$. Wir
finden in diesem Fall $c_6=0.11\,\gev^6$, in \"Ubereinstimmung
mit dem Ergebnis des oben angewandten Kriteriums. 

%Alternativ zur Analyse der Spektralfunktion mit Hilfe des
%Borelquotienten lassen sich die Koeffizienten $c_4$ und $c_6$ 
%auch direkt aus den FESR-Gleichungen (\ref{fesr2},\ref{fesr3}) bestimmen
%\cite{BDL88}. Betrachtet man zun\"achst nur die niedrigste
%Bedingung
%\be
%\label{lfesr}
% t_cF_2(t_c) = 8\pi \int_{s_0}^{t_c} {\rm Im}\Pi^V(s) \, ds
%\ee
%f\"ur $\sqrt{t_c}=2$ GeV, so ergibt sich mit unseren Daten eine
%Diskrepanz von $0.43\,{\rm GeV}^2$ zwischen der linken und 
%rechten Seite der Summenregel. Das bedeutet, da\ss\ bei der 
%verwendeten Spektralfunktion die Dualit\"atsforderung im Bereich
%von ca. 10\% erf\"ullt ist.
%
%Die h\"oheren FESR-Gleichungen sind au\ss erordentlich sensitiv 
%auf die G\"ultigkeit dieser Bedingung. Verwendet man (\ref{fesr2}) 
%f\"ur $\sqrt{t_c}=2$ GeV, so findet man $c_4= 1.76\,\gev^4$ und 
%$c_6= -4.01\,\gev^6$, in deutlichem Widerspruch zu (\ref{fitresult}). 
%Fixiert man dagegen $t_c$ durch die Dualit\"atsforderung (\ref{lfesr}), 
%so ergibt sich mit $\sqrt{t_c}=1.46\,{\rm GeV}$ eine Schwellenenergie 
%in der N\"ahe der $\rho'$-Resonanz. Dieser Wert von $\sqrt{t_c}$ 
%liefert die Kondensate 
%\be
%c_4=-0.33\,\gev^4 \hspace{1cm} c_6=0.68\,\gev^6 ,
%\ee
%in besserer \"Ubereinstimmung mit dem Resultat
%der Analyse des Borelquotieten. Insbesondere entspricht auch dieses
%Ergebnis unserer Beobachtung, da\ss\ die verwendeten Summenregeln
%im Wesentlichen das Verh\"altnis $c_4/c_6$ fixieren.

\section{Borelquotient im Axialvektorkanal}
Nachdem wir im letzten Abschnitt eine Analyse der experimentellen
Spektralfunktion im Vektorkanal vorgestellt haben, wollen wir uns 
nun unserem eigentlichen Thema zuwenden und untersuchen, inwieweit
auch die Daten im Axialvektorkanal mit QCD-Summenregeln vertr\"aglich
sind. Insbesondere soll die Frage untersucht werden, ob die Ergebnisse
konsistent mit der Faktorisierungshypothese $\xi^V=\xi^A$ sind.

Zu diesem Zweck betrachten wir den Quotienten der Borelmomente der 
$q_\mu q_\nu$-Struktur in der Korrelationsfunktion
\be
\label{ra2}
 R^{A_2}(\tau) \equiv \frac{\displaystyle \int_{s_0}^{\infty}
    \rho_A (s) e^{-s\tau} s\, ds }
  {\displaystyle \int_{s_0}^{\infty}
    \rho_A (s) e^{-s\tau}\, ds \,+\, f_\pi^2}\; ,
\ee
wobei wir die endliche Masse des Pions vernachl\"assigt haben. Diese
N\"aherung ist konsistent mit unserer Vorgehensweise im perturbativen
Sektor, wo wir die Effekte der Strommassen nicht ber\"ucksichtigt 
haben. Die Operatorproduktentwicklung liefert die Vorhersage 
\be
 Y^{A_2}(\tau) \equiv R^{A_2}(\tau)-\frac{1}{\tau} =
     c_4\tau - \frac{11}{7}\frac{\xi^A}{\xi^V} c_6\tau^2 + \ldots\, ,
\ee
so da\ss\ sich die Koeffizienten $c_4,c_6$ unter der Annahme $\xi^A=\xi^V$ 
ganz analog zur Analyse im Vektorkanal bestimmen lassen.
\begin{figure}
\caption{OPE-Parametrisierung des experimentellen Borelquotienten
f\"ur die $q_\mu q_\nu$-Struktur im Axialvektorkanal 
nach Subtraktion des perturbativen Beitrags $1/\tau$.}
\vspace{9cm}
\end{figure}
       
Das Pion liefert einen sehr genau bestimmten Beitrag zum ersten Moment
der Spektralfunktion, w\"ahrend der restliche Teil des Spektrums mit
erheblichen Fehlern behaftet ist. \"Ahnlich wie im Vektorkanal, wo
wir einen analogen Effekt auf Grund des Zusammenwirkens der $\rho$-
und $\rho'$-Resonanz beobachtet haben, existiert daher ein Bereich 
von Werten des Parameters $\tau$, in dem der Borelquotient $R^{A_2}
(\tau)$ kaum von den experimentellen Unsicherheiten beeinflu\ss t ist
(siehe Abb.~6.7). 

Wie im letzten Abschnitt bestimmen wir die Koeffizienten durch Anpassung an 
die Daten im Intervall $[0,\tau_{max}]$ und studieren die Abh\"angigkeit
der Ergebnisse von $\tau_{max}$. Die Werte von $c_4$ und $c_6$, die
sich unter Verwendung der Faktorisierungshypothese ergeben, finden sich 
in Tabelle 6.2. Erneut finden wir keine Stabilit\"atsregion f\"ur
den maximalen Borelparameter. Fixiert man $\tau_{max}$ mit Hilfe 
des theoretischen Werts f\"ur $c_4$, so ergibt sich $c_6=0.39\,
{\rm GeV}^6$. Dieses Resultat entspricht einer Verletzung der 
exakten Faktorisierung $\xi^A/\xi^V=1$ um den Betrag  $3.2\pm 1.9$, 
wobei der Fehler so gro\ss\ ist, da\ss\ auch $\xi^A/\xi^V=1$ nicht 
ausgeschlossen ist.      
\begin{table}
\caption{Bestimmung der Koeffizienten $c_i$ aus dem Borelquotienten
f\"ur die $q_\mu q_\nu$-Struktur im Axialvektorkanal. Angegeben sind 
die Koeffizienten $c_4$ und $c_6$ sowie die Absch\"atzung der 
experimentellen Unsicherheiten f\"ur verschiedene Werte des maximalen 
Borelparameters.}  
\begin{center}
\begin{tabular}{|c||c|c||c|c||}\hline
 $\tau_m\,[\gev^{-2}]$ &  $c_4\,[\gev^4]$  & $\Delta^{exp} c_4$ &
             $c_6\,[\gev^6]$  & $\Delta^{exp} c_6$   \\ \hline\hline
    0.80  &$ \spm 0.042$ & $\pm 0.077$        &
             0.559  & $\pm 0.169$          \\
    0.85  &$-0.037$ & $\pm 0.063$        & 
             0.447  & $\pm 0.137$          \\	     
    0.90  &$-0.101$ & $\pm 0.052$        &
             0.356  & $\pm 0.112$          \\	     
    0.95  &$-0.154$ & $\pm 0.044$        &
             0.283  & $\pm 0.093$          \\	     
    1.00  &$-0.198$ & $\pm 0.037$        & 
             0.221  & $\pm 0.076$          \\ \hline
\end{tabular}
\end{center}
\end{table}

Um die Zuverl\"assigkeit dieser Resultate einsch\"atzen zu k\"onnen,
mu\ss\ man allerdings beachten, da\ss\ unter Verwendung der beschriebenen
Spektralfunktion die niedrigste FESR-Bedingung
\be
\label{lfesra}
 t_c F_2(t_c) = 8\pi^2 \int_{s_0}^{t_c} \rho_A (s)\, ds \,
 +\, 8\pi^2f_\pi^2
\ee
deutlich verletzt ist. F\"ur $\sqrt{t_c}=m_\tau$ betr\"agt die
Diskrepanz zwischen der linken und rechten Seite der Summenregel
$\Delta =0.98\,{\rm GeV}^2$, ein Wert, der erheblich gr\"o\ss er ist als der
entsprechende Fehlbetrag im Vektorkanal. Tats\"achlich l\"a\ss t
sich die Bedingung erst f\"ur $\sqrt{t_c}=1.3 \,{\rm GeV}$ 
zumindest n\"aherungsweise erf\"ullen.
Dieser Wert ist aber so klein, da\ss\ praktisch der gesamte Resonanzbeitrag 
aus dem Spektrum herausgeschnitten wird.

Die Tatsache, da\ss\ die $a_1$-Resonanz allein die Dualit\"atsbedingung
nicht erf\"ullt, ist ein Hinweis darauf, da\ss\ die Verwendung des
perturbative Spektrums f\"ur $\sqrt s>m_\tau$ ein unrealistisches 
Modell der Spektralfunktion bei mittleren Energien liefert. Diese
Schlu\ss folgerung wird durch eine Analyse des Quotienten der 
Borelmomente der $g_{\mu\nu}$-Struktur der Korrelationsfunktion
\be
\label{ra1}
 R^{A_1}(\tau) \equiv \frac{\displaystyle \int_{s_0}^{\infty}
    \rho_A (s) e^{-s\tau} s^2\, ds }
  {\displaystyle \int_{s_0}^{\infty}
    \rho_A (s) e^{-s\tau}s\, ds }
\ee
untermauert. Dieses Verh\"altnis enth\"alt h\"ohere Momente der
Spektralfunktion und ist daher in st\"arkerem Umfang sensitiv 
auf die Form des Spektrums bei Energien jenseits der $a_1$-Masse.
Die Operatorproduktentwicklung liefert die theoretische
Vorhersage 
\be
\label{ya1}
 Y^{A_1}(\tau) \equiv -R^{A_1}(\tau)+\frac{2}{\tau} = \overline{c}_4
     \tau - \frac{22}{7}\frac{\xi^A}{\xi^V} c_6\tau^2 + \ldots\, ,
\ee
wobei
\be
 \overline{c}_4=c_4 + 16\pi^2 (m_u+m_d)<\bar qq>=c_4-0.026
\,{\rm GeV}^4
\ee
nur eine kleine Korrektur gegen\"uber dem entsprechenden
Koeffizienten im Vektorkanal enth\"alt.  
\begin{figure}
\caption{OPE-Parametrisierung des experimentellen Borelquotienten
f\"ur die $g_{\mu\nu}$-Struktur im Axialvektorkanal 
nach Subtraktion des perturbativen Beitrags $2/\tau$.}
\vspace{9cm}
\end{figure}

Die Resultate f\"ur $Y^{A_1}(\tau)$, die sich unter Verwendung des
Spektrums aus dem $\tau$-Zerfall ergeben, finden sich in Abbildung
6.8. Auf Grund der ung\"unstigeren Gewichtung des Spektrums wird
der systematische Fehler erst bei relativ gro\ss en Werten des
Borelparameters, $\tau\simeq 1.2\,{\rm GeV}^{-2}$, minimal. Wir finden
allerdings ganz unabh\"angig von dem verwendeten Intervall 
$[\tau_{min}=0,\tau_{max}]$ keine \"Ubereinstimmung der Daten mit 
der QCD-Parametrisierung (\ref{ya1}). Verwendet man $\tau_{max}=1.5
\,{\rm GeV}^{-2}$, um den Einflu\ss\ der experimentellen Fehler gering zu
halten, so ergibt sich $c_4=-0.58\pm 0.08\,{\rm GeV}^4$ und
$c_6 = -0.08\pm 0.06\,{\rm GeV}^6$, in klarem Widerspruch  zur 
Analyse der niedrigeren Borelmomente.    

\section{Hinweise auf radiale Anregungen der $a_1$-Re\-so\-nanz}
Wir haben im letzten Abschnitt das Spektrum im Axialvektorkanal mit 
Hilfe von QCD-Summenregeln untersucht. Unter Verwendung von Summenregeln,
die im wesentlichen auf den Pionbeitrag und den niederenergetischen 
Teil des Spektrums sensitiv sind, haben wir \"Ubereinstimmung mit den 
Ergebnissen  aus dem Vektorkanal und der Faktorisierungshypothese 
erzielt. Dagegen zeigen h\"ohere Borelmomente deutliche Abweichungen 
von diesen Vorhersagen. Dar\"uber hinaus ist unser Modell des Spektrums,
in dem jenseits der Masse des $\tau$-Leptons der st\"orungstheoretische
Wert verwendet wird, nicht konsistent mit der asymptotischen 
G\"ultigkeit der Summenregeln. 

Es liegt daher nahe, auch die Beitr\"age h\"oherer Anregungen des 
$a_1$-Mesons zu ber\"ucksichtigen. Die vorhandenen theoretischen 
und experimentellen Informationen \"uber solche Zust\"ande sind 
allerdings sehr fragmentarisch. So enth\"alt die Kompilation der
Particle Data Group \cite{PDG90} neben dem $a_1$ keine weiteren 
Mesonen mit den Quantenzahlen $I^G(J^{PC})=1^-(1^{++})$.

In nichtrelativistischen Konstituendenmodellen des mesonischen 
Spektrums ergeben sich dagegen in nat\"urlicher Weise radiale
Anregungen des $a_1$-Mesons. Ein typisches Beispiel ist das
Potentialmodell von Metsch et al.~\cite{Met90}. Diese Autoren 
finden den Grundzustand im $1^{++}$-Kanal bei einer Energie
von 1220 MeV und die erste radiale Anregung bei $m_{a_1'}=
1923$ MeV. Auch die bereits angesprochene gro\ss e Breite der
$a_1$-Resonanz ist als Hinweis auf die Gegenwart h\"oherer Anregungen
gedeutet worden. Beschreibt man das Spektrum aus dem $\tau$-Zerfall
mit Hilfe zweier interferierender Resonanzen \cite{IKM89},
so reduziert sich die resultierende $a_1$-Breite auf 
$\Gamma_{a_1}=380\pm 20$ MeV, in \"Ubereinstimmung mit den
Resultaten hadronischer Experimente.  Die entsprechenden 
Parameter der $a_1'$-Resonanz lauten $m_{a_1'}=1550\pm 40$ MeV
und $\Gamma_{a_1'}=525\pm 25$ MeV.

Die Frage, ob die im $\tau$-Zerfall beobachtete Struktur in 
Wahrheit durch mehrere \"uberlappende Resonanzen hervorgerufen
wird, l\"a\ss t sich  nicht mit Hilfe von Summenregeln    
kl\"aren. Wir wollen uns daher im folgenden auf die Gestalt des 
Spektrums jenseits der Masse des $\tau$-Leptons konzentrieren. 
Zu diesem Zweck parametrisieren wir die Spektralfunktion 
in der Form
\beq
\label{rhoap}
  \rho_A(s) &=& \rho_a^\tau (s) +\frac{1}{4\pi} \Big( 
   \frac{a'm_{a_1'}^2\Gamma_{a_1'}^2}{(s-m_{a_1'}^2)^2
   +m_{a_1'}^2\Gamma_{a_1'}^2} + b\sqrt{s-s_0} \Big)
   \Theta (s_{th}-s)   \\[0.1cm]
   & &  \hspace{1.3cm} \mbox{}+\frac{1}{8\pi^2} 
   \Big( 1+\frac{\alpha_s}{\pi} \Big) \Theta (s-s_{th}), \nonumber
\eeq
wobei $\rho_a^\tau (s)$ die Spektralfunktion aus dem $\tau$-Zerfall 
bezeichnet. Die Gr\"o\ss en $a',m_{a_1'}$ und $\Gamma_{a_1'}$ 
charakterisieren die $a_1'$-Resonanz, w\"ahrend der Term 
$b\sqrt{s-s_0}$ einen langsam wachsenden Untergrund beschreibt. 
Dieser Untergrund geht f\"ur Energien oberhalb der 
Kontinuumsschwelle $s_{th}$ in das st\"orungstheoretische Resultat 
\"uber.

\begin{figure}
\caption{Vergleich der Spektralfunktionen im Vektor- und 
Axialvektorkanal. Die Axialvektorspektralfunktion enth\"alt 
die im Text bestimmten Beitr\"age der $a_1'$-Resonanz.}
\vspace{9cm}
\end{figure} 
Wir bestimmen die freien Parameter, indem wir in einem gegebenen 
Intervall $[\tau_{min},\tau_{max}]$ die \"Ubereinstimmung der 
beiden Borelquotienten $R^{A_1}(\tau)$ und $R^{A_2}(\tau)$ mit 
den Vorhersagen der OPE optimieren. Zu diesem Zweck minimieren 
wir die Funktion
\be
\chi^2 =\sum_i \left| \frac{ Y^{A_1}(\tau_i)-Y^{A_1}_{th}(\tau_i) }
                    { \Delta Y^{A_1}(\tau_i) } \right|^2 +
               \left| \frac{ Y^{A_2}(\tau_i)-Y^{A_2}_{th}(\tau_i) }
	            { \Delta Y^{A_2}(\tau_i) } \right|^2 
\ee
in dem gegebenen Parameterraum. In der Praxis haben wir dabei die 
Werte von $s_0,s_{th}$ und $\Gamma_{a_1'}$ fixiert, da diese 
Gr\"o\ss en durch die verwendeten Summenregelen nicht sehr stark 
eingeschr\"ankt sind. Mit $s_0=2.5\,\gev^2$, $s_{th}=4.65\,\gev^2$
und $\Gamma_{a_1'}=0.25\,\gev$ finden wir
\be
\label{ma1p}
m_{a_1'} = 1729^{+11}_{-67}\,{\rm MeV}
\ee
sowie $a'=0.274$ und $b=5.84\cdot 10^{-2}\,\gev^{-1}$. Die angegebenen
Fehler sind ein Ma\ss\ f\"ur die Variation der Ergebnisse bei 
verschiedenen Werten von $s_0,s_{th}$ und $\Gamma_{a_1'}$. 

Die resultierende Spektralfunktion ist in Abbildung 6.9 dargestellt. 
Die Position der $a_1'$-Resonanz liegt noch unterhalb der Masse des 
$\tau$-Leptons, allerdings in einem Bereich, wo die Daten aus dem 
$\tau$-Zerfall mit sehr gro\ss en Fehlern behaftet sind (siehe Abb. 6.2).
Die modifizierte Spektralfunktion liefert ein Verzweigungsverh\"altnis
\be
 B(\tau\to\nu (3\pi)) = 13.4\%,
\ee
das etwas oberhalb der von der Argus-Kollaboration angebenen experimentellen
Unsicherheit, $B(\tau\to\nu (3\pi)) = 11.2\pm 1.4\%$, aber noch unterhalb des 
Resultats des MAC-Experiments, $B(\tau\to\nu (3\pi))=15.7\pm 2.5\%$, liegt. 

Auch die Dualit\"atsforderung l\"a\ss t sich unter Einbeziehung der 
$a_1'$-Resonanz erf\"ullen. Untersucht man die G\"ultigkeit der 
FESR-Bedingung   
\be
 t_c F_2(t_c) = 8\pi^2 \int_{s_0}^{t_c} \rho_A (s)\, ds \,
 +\, 8\pi^2f_\pi^2
\ee
f\"ur $\sqrt{t_c}=2.15$ GeV, so ergibt sich eine Diskrepanz $<1\%$
zwischen den beiden Seiten der Summenregel. Wir haben dar\"uber hinaus
die Auswirkung  der $a_1'$-Resonanz auf die in Abschnitt 6.2 
diskutierten Gr\"o\ss en $\Delta m_\pi$ und $F_A(0)$ untersucht. 
Wir finden
\beq
    \Delta m_\pi &=& 4.1^{+0.3}_{-0.2} \,\mev , \\
    F_A(0)       &=& 0.6^{+0.2}_{-0.1}\cdot 10^{-2}
\eeq
in \"Ubereinstimmung mit den Werten aus 6.3, aber mit einer deutlich 
kleineren Abh\"angigkeit von dem verwendeten Cutoff.
