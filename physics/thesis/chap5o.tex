\chapter{Spektrale Summenregeln in QCD}
\section{Einf\"uhrung}
Die Struktur der Quantenchromodynamik bei niederen Energien 
wird durch eine Reihe nichtperturbativer Ph\"anomene bestimmt.
Von Bedeutung sind vor allem  Confinement, der permanente
Einschlu\ss\ von Quarks und Gluonen in farbneutralen Hadronen,
sowie das Auftreten von Kondensaten, das hei\ss t nicht
verschwindenden Vakuumerwartungswerten von  Quark- oder 
Gluonoperatoren. Ein Beispiel f\"ur die Auswirkung der Kondensate
auf die Gestalt des Spektrums liefert der Vergleich  
der Korrelationsfunktionen im Vektor und Axialvektorkanal
\beq
\label{vvcor}
  \Pi_{\mu\nu}^V &=& -(g_{\mu\nu}q^2-q_\mu q_\nu) \Pi^V(q^2) \\
   & & \hspace{0.3cm} =\;\, i\int d^4x\, e^{iq\cdot x} <0|T\left( 
   j_\mu^{(\rho)}(x)j_\nu^{(\rho)}(0) \right) |0> \nonumber \\[0.1cm]
\label{aacor}
  \Pi_{\mu\nu}^A &=& -g_{\mu\nu}q^2 \Pi^{A_1}(q^2)
              + q_\mu q_\nu \Pi^{A_2}(q^2) \\
   & & \hspace{0.3cm}=\;\, i\int d^4x\, e^{iq\cdot x} <0|T\left( 
   j_\mu^{(a_1)}(x)j_\nu^{(a_1)}(0) \right) |0> \nonumber \, .
\eeq
Die Str\"ome $j_\mu^{(\rho)} =\frac{1}{2}(\bar u\gamma_\mu u-
\bar d\gamma_\mu d)$ und $j_\mu^{(a_1)}=\frac{1}{2} (\bar u
\gamma_\mu\gamma_5 d-\bar d\gamma_\mu\gamma_5 d)$ tragen die 
Quantenzahlen des $\rho$ bzw.~$a_1$-Mesons. Wir betrachten den
isospinsymmetrischen Fall, da\ss\ hei\ss t der Strom $j_\mu^{(\rho)}$
ist erhalten und die entsprechende Korrelationsfunktion transversal. 

Im chiralen Limes sind die St\"orungsentwicklungen der 
beiden Korrelationsfunktionen identisch. Dieser Sachverhalt steht 
im klaren Widerspruch zu der sehr unterschieldlichen Form der 
Spektralfunktionen im Vektor  bzw.~Axialvektorkanal. In der Tat ist 
es ein nichtperturbativer  Effekt, die spontane Brechung der chiralen 
Symmetrie hervorgerufen durch die nichtverschwindenden Quarkkondensate, 
welcher die Aufspaltung der $\rho$ und $a_1$ Massen bewirkt.

Ganz analog signalisiert das Gluonkondensat $<G_{\mu\nu}^{a}
G^{a}_{\mu\nu}>$ die Brechung der konformen Symmetrie und ist 
damit f\"ur das Auftreten von Massenparametern in der 
Quantenchromodynamik verantwortlich. Wir ber\"ucksichtigen die
Rolle der Kondensate in der Korrelationsfunktion mit Hilfe
der Operatorproduktentwicklung (OPE), g\"ultig im tief
euklidischen Bereich $Q^2=-q^2\to\infty$
\be
\label{ope}
 \Pi (Q^2) =  C_{1\!\!1}  + \sum_{n=2} \frac{1}{Q^{2n}}
        C_n <0| {\cal O}_n|0> \; .
\ee
Dabei liefert der Koeffizient $C_{1\!\!1}$ des Einheitsoperators
die \"ubliche St\"orungsreihe in Potenzen von $\alpha_s$,
w\"ahrend ${\cal O}_n$ einen Operator der Dimension $d=2n$
bezeichnet, dessen Vakuumerwartungswert durch nichtperturbative
Effekte bestimmt wird. Der kurzreichweitige Teil der 
Wechselwirkung der Str\"ome mit den Operatoren ${\cal O}_n$ ist
in den Wilson-Koeffizienten $C_n$ enthalten, die sich in 
in der \"ublichen Weise aus Feynman-Diagrammen bestimmen lassen. 
Nicht berechenbar auf dem gegenw\"artigen Stand der Theorie sind 
dagegen die  Vakuumerwartungswerte $<0|{\cal O}_n|0>$.

QCD Summenregeln machen daher von der Analytizit\"at der 
Korrelationsfunktion Gebrauch, um das Verhalten von $\Pi (Q^2)$ 
im tief euklidischen Bereich mit der Gestalt der Spektralfunktion 
bei kleinen Energien in Verbindung zu bringen. Die Funktion 
$\Pi (Q^2)$ erf\"ullt die Dispersionsrelation 
\be 
\label{disprel}
 \Pi (Q^2) = (-1)^n \frac{Q^{2n}}{\pi} \int_{s_0}^{\infty}
    \frac{{\rm Im}\Pi (s)}{s^n(s+Q^2)} ds +
    \sum_{k=0}^{n-1} a_k Q^{2k}\, ,
\ee
wobei eine von den Eigenschaften des Stroms abh\"angige Zahl von 
Subtraktionen vorgenommen werden mu\ss .  Die entsprechenden  
Subtraktionskonstanten haben wir mit $a_k$ bezeichnet. 
Der Imagin\"arteil ${\rm Im}\Pi (Q^2)$ l\"a\ss t sich mit
Hilfe von  Unitarit\"atbeziehungen aus physikalischen
Observablen bestimmen.

Im Prinzip existiert daher eine sehr starke Korrelation zwischen 
dem asymptotischen Verhalten der Funktion $\Pi (Q^2)$, parametrisiert
mit Hilfe der Kondensate $<{\cal O}_n>$, und experimentell 
zug\"anglichen Gr\"o\ss en. In der Praxis ist man allerdings 
gezwungen, die OPE bei $n=3$ oder $n=4$ abzubrechen. Auch die
experimentellen Daten sind nur in einem beschr\"ankten Bereich 
bekannt und mit Fehlern behaftet. In diesem Teil der Arbeit wollen
wir untersuchen, in welchem Umfang eine Bestimmung der 
niederdimensionalen Kondensate aus den gemessenen Spektralfunktionen 
im Vektor- und Axialvektorkanal m\"oglich ist. 

In der Gegenwart von Fehlern sind verschiedene Formulierungen 
der Summenregel (\ref{disprel}) nicht in gleicher Weise zur
Bestimmung der Vakuumparameter geeignet. Besonders bew\"ahrt 
hat sich die Boreltransformierte Dispersionsrelation   
\be
\label{borel}
 \hat L_B \Pi (Q^2) = \frac{1}{\pi} \int_{s_0}^{\infty} ds\,
  {\rm Im}\Pi (s) e^{-s\tau} \, ,
\ee  
wobei $\hat L_B$ den Boreloperator
\be
\label{borelop}
 \hat L_B = \left. \lim_{Q^2,n\to\infty} \frac{1}{(n-1)!}Q^{2n}
   \left( -\frac{d}{dQ^2} \right)^n  \right|_{n/Q^2=\tau} 
\ee
bezeichnet. Die Boreltransformation hat den Vorzug, das experimentell 
zuverl\"assiger bestimmte Niederenergieverhalten der 
Spektralfunktionen st\"arker zu gewichten und m\"ogliche 
Subtraktionskonstanten zu eliminieren. Dar\"uber hinaus werden 
h\"ohere Potenzen von $1/Q^2$ in der OPE durch zus\"atzliche 
Faktoren $1/n!$ unterdr\"uckt. 

Eine weitere Konsequenz der analytischen Struktur der 
Korrelationsfunktionen sind Summenregeln in einem endlichen
Energieintervall ({\em engl.} finite energy sum rule, FESR)
\be
\label{fesr}
 M_n (t_c)  = \frac{t_c^{n+1}}{\pi} \int_0^{2\pi}d\phi\,
     e^{i(n+1)\phi} \Pi (t_ce^{i\phi})\, .
\ee
Die rechte Seite der Summenregel testet das Verhalten der 
der Funktion $\Pi (Q^2)$ auf einem Kreis in der komplexen
$q^2$ Ebene und l\"a\ss t sich f\"ur hinreichend gro\ss e 
$t_c$ mit Hilfe der OPE auswerten. Die linke Seite ist durch 
das n-te Momente der Spektralfunktion in einem endlichen
Intervall gegeben
\be
\label{moment}
 M_n (t_c) = \frac{1}{\pi} \int_{s_0}^{t_c}ds \, s^n 
    {\rm Im}\Pi (s)
\ee
und kann aus experimentellen Daten bestimmt werden. Ein 
Vorzug der FESR ist die Tatsache, da\ss\ die Relationen 
(\ref{fesr}) direkt auf Kondensate einen festen Dimension
projezieren. Allerdings sind diese Beziehungen f\"ur h\"ohere
$n$ zunehmend von Fehlern in den Eingabedaten beeintr\"achtigt. 

\section{Entwicklung nach lokalen Operatoren}
Wir beschreiben in diesem Abschnitt die Bestimmung der 
Wilson-Koeffizienten f\"ur die Korrelationsfunktion 
im Vektor- und Axialvektorkanal. Beginnen wollen wir
mit dem Vektorkorrelator
\beq
 \Pi_{\mu\nu}^V(q^2) &=& \frac{i}{4} \int d^4x \, e^{iq\cdot x}
    <0|T\big( (\bar u(0)\gamma_\mu u(0)-\bar d(0)\gamma_\mu d(0) )
    \\
    & & \hspace{4.3cm}  \cdot (\bar u(x)\gamma_\nu u(x)
       -\bar d(x)\gamma_\nu d(x) )\big) |0> \, .\nonumber
\eeq
Entwickelt man das zeitgeordnete Produkt der Feldoperatoren mit
Hilfe des Wick'schen Theorems und vernachl\"assigt die 
normalgeordneten Beitr\"age, so erh\"alt man die \"ubliche 
St\"orungsentwicklung f\"ur $\Pi_{\mu\nu}^V$. Um normalgeordnete    
Produkte von Feldoperatoren an verschiedenen Punkten als
Erwartungswerte lokaler Operatoren zu schreiben, entwickelt
man die Operatorprodukte in  eine  Taylorreihe in der 
Differenz der Argumente. Dieses Verfahren ist konsistent, da der 
Korrelator f\"ur hohe $Q^2$ durch das Produkt der Str\"ome bei
kleinen Abst\"anden bestimmt ist. Fluktuationen auf
gro\ss e L\"angenskalen manifestieren sich in den 
nichtverschwindenden Kondensaten. 

Die oben beschriebene Vorgehensweise l\"a\ss t sich besonders
einfach in der Fock-Schwinger Eichung 
\be
\label{fsgauge}
 (x-x_0)^\mu A_\mu^{a}(x) =0 
\ee
anwenden. Diese Eichung hat den Vorzug, da\ss\ sich die Taylorentwicklung 
der Felder allein mit Hilfe eichkovarianter Operatoren ausdr\"ucken 
l\"a\ss t
\beq
  A_\mu (x) &=& \sum_n \frac{1}{n!(n+2)}\, x^\alpha x^{\nu_1} \left.
       \cdot\ldots\cdot x^{\nu_n} [ {\cal D}_{\nu_1},
       [ \ldots ,[ {\cal D}_{\nu_n}, G_{\alpha\mu}] \ldots ]] \right|_{x=0}
       \\
  \psi (x)  &=& \sum_n \frac{1}{n!}\, x^{\nu_1} \cdot\ldots\cdot
       x^{\nu_n} \left. \big( {\cal D}_{\nu_1} \ldots
       {\cal D}_{\nu_n} \psi (x) \big) \right|_{x=0} \; .
\eeq
Verwendet man diese Beziehungen, so ergeben sich folgende Erwartungswerte
f\"ur die Produkte der Feldoperatoren \cite{PT84}
\beq
 <A_\mu (x) A_\nu (y)> &=& \frac{1}{48}\, x^\rho y^\sigma
      (g_{\rho\sigma}g_{\mu\nu}-g_{\rho\mu}g_{\nu\sigma} )
       <G_{\alpha\beta}^{a}G^{a}_{\alpha\beta}>  + \ldots \\
  <\bar q^\alpha_i (x) q^\beta_j (0)> &=& \frac{\delta^{\alpha\beta}}{4N_c}
             \left( \delta_{ij} +\frac{i}{4} m_q x^\mu (\gamma_\mu)_{ij}
	     \right) <\bar qq> + \ldots
\eeq
wobei $\alpha,\beta$ die Farbindices und $i,j$ die Diracindices der 
Quarks bezeichnet. Das Produkt der Eichfelder l\"a\ss t sich vollst\"andig 
mit Hilfe des Feldst\"arketensors ausdr\"ucken. Dagegen liefert 
das Produkt der Fermionfelder in h\"oheren Ordnungen auch gemischte 
Kondensate. 

Die Fock-Schwinger Technik ist au\ss erordentlich effektiv bei 
der Bestimmung der Wilson-Koeffizienten komplizierter 
gluonischer Operatoren. Dieser Vorzug wird allerdings durch 
den Nachteil erkauft, da\ss\ die Berechnung perturbativer
Korrekturen sehr viel aufwendiger als in kovarianten
Eichungen ist.

Die Operatorproduktentwicklung der Korrelationsfunktion l\"a\ss t
sich in die Form 
\be
\label{powexp} 
\Pi^V (Q^2) = -h_0^V \ln \left(\frac{Q^2}{\mu^2}\right) 
+ \sum_{n=2} \frac{1}{n}\,\frac{h_n^V}{Q^{2n}}
\ee
bringen, wobei $\mu^2$ die Renormierungsskala kennzeichnet.
Mit Hilfe der geschilderten Methoden findet man die 
Koeffizienten
\beq
 h_0^V &=& \frac{1}{8\pi^2} \left( 1+\frac{\alpha_s}{\pi}\right) \\
 h_2^V &=& m_u<\bar{u}u>\!+\,m_d<\bar{d}d> + \frac{1}{12}
    <\frac{\alpha_s}{\pi} G_{\mu\nu}^{a}G^{a}_{\mu\nu} >   \\
 h_3^V &=& -\frac{3\pi\alpha_s}{2}
        <\left(\bar{u}\gamma_\mu\gamma_5\lambda^{a}u -
	       \bar{d}\gamma_\mu\gamma_5\lambda^{a}d \right)^2 >   \\
      & & \;\mbox{}-\frac{\pi\alpha_s}{3} 
       < (\bar{u}\gamma_\mu\lambda^{a}u +
          \bar{d}\gamma_\mu\lambda^{a}d ) 
	  \sum_{q=u,d,s} \bar{q}\gamma_\mu\lambda^{a} q> \nonumber \, .
\eeq
Dabei bezeichnet $\alpha_s$ die impulsabh\"angige laufende 
Kopplungskonstante. In Ein-Schleifenn\"aherung ist
\be
\label{runcoupl}
  \frac{\alpha_s(Q^2)}{\pi} = \left( -\frac{\beta_1}{2}
      \ln \left(\frac{Q^2}{\Lambda^2}\right) \right)^{-1}
\ee
mit $\beta_1=-\frac{11}{2}+\frac{N_f}{3}$ und dem Skalenparameter
$\Lambda=230\pm 80$ MeV. Im perturbativen Teil
des Resultats haben wir Korrekturen auf Grund der endlichen 
Strommassen der Quarks vernachl\"assigt. St\"orungstheoretische 
Korrekturen zum Einheitsoperator sind bis zur Ordnung $\alpha_s^3$  
bestimmt worden \cite{GKL88}, w\"ahrend die entsprechenden Korrekturen an 
$<\bar qq>$ und $<G^2>$ nur  bis zur Ordnung $\alpha_s$ bekannt 
sind \cite{Nar89}. Wir haben die Operatorproduktentwicklung bei
Termen der Dimension $d=6$ abgebrochen. In der n\"achst h\"oheren 
Ordnung der OPE sind lediglich die Wilson-Koeffizienten f\"ur alle rein 
gluonischen Operatoren berechnet worden \cite{BG85}.

Der Beitrag  der Dimension $d=4$ in der OPE enth\"alt  neben
dem aus der GOR-Relation (\ref{GOR}) bekannten Matrixelement
$m_u\!<\!\bar uu>\!+\,m_d\!<\!\bar dd\!>$ auch den rein gluonischen
Erwartungswert $<\!\frac{\alpha_s}{\pi}G^2\!>$. Dieses 
Matrixelement ist besonders intensiv im Zusammenhang mit
Charmoniumzust\"anden untersucht worden. In diesem System 
ist die Rolle der Quarkkondensate auf Grund der gro\ss e
Massen des Charmquarks unterdr\"uckt. Der kanonische
Wert des Gluonkondensats aus einer  Analyse der angeregten
Zust\"ande des $c\bar c$-Systems betr\"agt  $<\!\frac{\alpha_s}{\pi}
G^2\!>=(360\pm 20{\rm MeV})^4$ \cite{RRY85}.  

Die Quarkkondensate der Dimension $d=6$  lassen sich mit Hilfe
der Faktorisierungshypothese \cite{SVZ79}
\be
\label{fact}
  <\bar\psi\Gamma_1\psi\bar\psi\Gamma_2\psi > =
    \frac{1}{N^2} \big( {\rm Tr}(\Gamma_1)\,{\rm Tr}(\Gamma_2)  
    - {\rm Tr} (\Gamma_1\Gamma_2) \big) \, <\bar\psi\psi >^2
\ee    
vereinfachen. Dabei bezeichnet $\psi =(u,d,\ldots)$ einen Quarkspinor
in $SU(N_f)$ und die Normierungskonstante ist durch $N=4N_cN_f$
gegeben. Auf diese Weise findet man 
\be
  h_3 = -\frac{112}{27}\xi^V\pi\alpha_s \, <\bar qq>^2
\ee
wobei die Gr\"o\ss e $\xi^V$ Abweichungen von der exakten 
Faktorisierung ($\xi^V=1$) parametrisiert. Da sich die 
anomalen Dimensionen der Kopplungskonstante und des Kondensats
praktisch aufheben, ist $h_3$ nur sehr schwach vom 
Normierungspunkt abh\"angig. Besitzt man eine Absch\"atzung
der Quarkamssen, so l\"a\ss t sich der Wert des Quarkkondensats  
mit Hilfe der GOR Relation berechnen. Da weder das Kondensat noch die 
Strommassen invariant unter der Renormierungsgruppe sind, ist
es zu diesem Zweck allerdings sehr wichtig, die Skala zu 
kennen, bei der die Quarkmassen bestimmt worden sind. 
Verwendet man den Wert $\left. (m_u+m_d)\right|_{Q^2=1\,
{\rm GeV}^2}=14$ MeV, den wir in Kapitel 2 angegeben haben,
so findet man $<\bar qq>=(230\,{\rm MeV})^3$ bei $Q^2=1\,
{\rm GeV}^2$. 

Auf Grund der expliziten Brechung der chiralen Symmetrie wird
die Korrelationsfunktion im Axialvektorkanal durch zwei 
unabh\"angige invariante Funktionen charakterisiert. 
Diese lassen sich wie in (\ref{powexp}) entwickeln. F\"ur den
Koeffizienten der $q_\mu q_\nu$-Struktur findet man 
\beq
 h_0^{A_2} &=& \frac{1}{8\pi^2} \left( 1+\frac{\alpha_s}{\pi}\right) \\
 h_2^{A_2} &=& m_u<\bar{u}u>\!+\,m_d<\bar{d}d> + \frac{1}{12}
               <\frac{\alpha_s}{\pi} G_{\mu\nu}^{a}G^{a}_{\mu\nu} >   \\
 h_3^{A_2} &=& -\frac{3\pi\alpha_s}{2}
        <\left(\bar{u}\gamma_\mu\lambda^{a}u -
	       \bar{d}\gamma_\mu\lambda^{a}d \right)^2 >   \\
     & & \;\mbox{}-\frac{\pi\alpha_s}{3} 
       < (\bar{u}\gamma_\mu\lambda^{a}u +
          \bar{d}\gamma_\mu\lambda^{a}d ) 
	  \sum_{q=u,d,s} \bar{q}\gamma_\mu\lambda^{a} q> \nonumber \, .
\eeq
Im chiralen Limes sind der Koeffizient des Einheitsoperators 
sowie die Beitr\"age des Gluonkondensats im Vektor- und Axialvektorkanal
identisch. Betrachtet man den Koeffizienten $h_3$, so liefert
der Axialstromkorrelator eine andere Diracstruktur im ersten der beiden 
Beitr\"age. Verwendet man die Faktorisierungshypothese
\be 
 h_3^{A_2} = \left(\frac{11}{7}\right)\frac{112}{27} 
       \xi^A\pi\alpha_s \, <\bar qq>^2 \, ,
\ee
so ergibt sich eine gegen\"uber dem Resultat im Vektorkanal
nur geringf\"ugig modifizierte numerische Konstante, aber ein 
ge\"andertes Vorzeichen. 

Die Differenz der beiden invarianten  Strukturen $\Pi^{A_1}$ und 
$\Pi^{A_2}$ ist in f\"uhrender Ordnung durch 
\be
  \Pi^{A_1} (Q^2) - \Pi^{A_2} (Q^2) = 
      \frac{m_u<\bar uu>\!+\,m_d<\bar dd>}{Q^4} \; .
\ee
gegeben. Saturiert man die zugeh\"origen Spektralfunktionen mit
Pionzust\"anden, so ist diese Beziehung \"aquivalent mit der
GOR Relation. 

\section{Summenregeln}
Mit Hilfe der im letzten Abschnitt bestimmten Wilson-Koeffizienten
k\"onne wir nun die explizite Form der Summenregeln f\"ur die
Spektralfunktionen ${\rm Im}\Pi^{V,A}$ angeben. So lautet die
Boreltransformierte Dispersionsrelation im Vektorkanal
\be 
\label{borelsum}
 \frac{1}{\pi} \int_{s_0}^\infty {\rm Im}\Pi^V(s)e^{-s\tau}ds
      = \frac{1}{\tau} \left\{ h_0^V + \frac{h_2^V}{2!} \tau^2
         + \frac{h_3^V}{3!}\tau^3 + \ldots \right\} \, .
\ee	 
H\"ohere Momente der Borelsummenregel ergeben sich durch differenzieren
von (\ref{borelsum}) nach $\tau$. Der Borelparameter $\tau$ 
kontrolliert die Gewichtung des Spektrums und die relative
Bedeutung h\"oherer Korrekturen in der OPE. Im Prinzip ist 
die Summenregel (\ref{borelsum}) f\"ur beliebige Werte von
$\tau$ g\"ultig. In der Praxis mu\ss\ man allerdings den 
Borelparameter in einem gewissen Bereich w\"ahlen, um die 
Fehler auf Grund der vernachl\"assigten Terme in der 
OPE und der Unsicherheiten in der Gestalt der Spektralfunktion
unter Kontrolle zu halten.

Die zu (\ref{borelsum}) analogen Summenregeln in einem endlichen 
Energieintervall lauten 
\beq
\label{fesr1}
  t_c F_2(t_c) 
     &=&    8\pi \int_{s_0}^{t_c} {\rm Im} \Pi^V(s) \,ds  \\
\label{fesr2}
  c_4 + t_c^2 F_4(t_c) 
     &=&    16\pi \int_{s_0}^{t_c} {\rm Im} \Pi^V(s)\, s\, ds  \\     
\label{fesr3}
  c_6 - \frac{t_c^3}{2} F_6(t_c) 
     &=&    12\pi \int_{s_0}^{t_c} {\rm Im} \Pi^V(s)\, s^2\, ds  \, ,
\eeq
wobei wir folgende Bezeichnung f\"ur die nivhtperturbativen
Korrekturen eingef\"uhrt haben 
\be
\label{defcor}
  c_4 = -\frac{h_2^V}{h_0^V} \hspace{1cm} 
  c_6 = -\frac{h_3^V}{2h_0^V} \, .
\ee  
Perturbative Korrekturen sind in den Funktionen $F_{2n}(t_c)$ 
ber\"ucksichtigt. In erster Ordnung in $\alpha_s$ ist
$F_{2n}(t_c) = (1+\frac{\alpha_s(t_c)}{\pi})$, w\"ahrend die  
Korrekturen in h\"oherer Ordnung abh\"angig von $n$ ist. Resultate
bis $\alpha_s^2$ finden sich in \cite{BDL88}.

Um hadronische Parameter mit Hilfe der Summenregeln zu fixieren, 
ist man gezwungen, eine m\"oglicht einfache Parametrisierung der
Spektralfunktion zu verwenden. Im Vektorkanal lautet der Ansatz
\be
\label{vansatz}
 \frac{1}{\pi} {\rm Im}\Pi^V(s) = \frac{m_\rho^2}{g_\rho^2}
    \delta (s-m_\rho^2) + \frac{1}{8\pi^2}
     \left( 1+\frac{\alpha_s}{\pi} \right) \Theta (s-s_{th})
\ee
wobei der erste Term den Beitrag der $\rho$-Resonanz im Grenzfall
verschwindender Breite ber\"ucksichtigt und der zweite 
Term das QCD-Kontinuum repr\"asentiert. Die Kopplung
des $\rho$-Mesons ist durch das Matrixelement
\be
  <0| J_\mu^{(\rho)}(x) |\rho (p)> = \frac{g_\rho}{m_\rho^2}
    \epsilon_\mu e^{ip\cdot x}
\ee
definiert. Die beiden Parameter $m_\rho$ und $g_\rho$ lassen sich aus 
der Boreltransformierten Summenregel (\ref{borelsum}) und deren 
erstem Moment bestimmen. Zu diesem Zweck fixiert man die 
Kontinuumschwelle $s_{th}$ mit Hilfe der Konsistenzbeziehung
(\ref{fesr1}) und w\"ahlt zum Beispiel $\tau =m_\rho^{-2}$. 
Shifman, Vainshtein und Sakharov \cite{SVZ79} verwenden
die Werte
\be
\label{svzval}
 c_4 =-0.07\;{\rm GeV}^4\hspace{1.5cm}  c_6 = 0.09\;{\rm GeV}^6
\ee
und finden  $m_\rho = 783$ MeV sowie $\frac{g_\rho^2}{4\pi} = 2.6$
in hervoragender \"Ubereinstimmung mit den experimentellen Werten. 
Wir wollen im n\"achsten Kapitel  auf die Frage eingehen, in 
wie weit dieser Erfolg ein  Ergebnis der naiven 
Parametrisierung der Spektralfunktion ist. Bereits an dieser Stelle
sei angemerkt, da\ss\ das Resultat f\"ur $m_\rho$ nur sehr schwach
vom Wert der Kondensate abh\"angt, solange das Verh\"altnis 
$c_4/c_6$ in etwa konstant gew\"ahlt wird.

Im Axialvektorkanal liefern die beiden Strukturen $\Pi^{A_1}$
und $\Pi^{A_2}$ zwei unabh\"angige Borelsummenregeln
\beq 
 \frac{1}{\pi} \int_{s_0}^\infty {\rm Im}\Pi^{A_1}(s)se^{-s\tau}ds
      &=& \frac{1}{\tau^2} \left\{ h_0^{A_1} - \frac{h_2^{A_1}}{2!} \tau^2
         - \frac{2h_3^{A_1}}{3!}\tau^3 - \ldots \right\} \\
 \frac{1}{\pi} \int_{s_0}^\infty {\rm Im}\Pi^{A_2}(s)e^{-s\tau}ds
      &=& \,\frac{1}{\tau} \,\left\{ h_0^{A_2} + \frac{h_2^{A_2}}{2!} \tau^2
         + \frac{h_3^{A_2}}{3!}\tau^3 + \ldots \right\}	 
\eeq
In der verwendeten N\"aherung sind die Koeffizienten $h_n$ in den 
beiden Summenregeln identisch bis auf den Beitrag des Quarkkondensats
\be
 h_2^{A_1} - h_2^{A_2} = 2(m_u<\bar uu>\!+\, m_d<\bar dd>) \; .
\ee
Numerisch ist diese Differenz klein im Vergleich mit dem Beitrag des
Gluonkondensats, so da\ss\ in recht guter N\"aherung $h_2^{A_1}
\simeq h_2^{A_2}$ gilt. Die Spektralfunktionen im Axialvektorkanal
definieren wir durch 
\be
 \frac{1}{\pi} {\rm Im} \Pi_{\mu\nu}^{A}(q^2) =
    \rho_A(q^2) ( q_\mu q_\nu -g_{\mu\nu}q^2) +
    \rho_A^{||}(q^2) q_\mu q_\nu
\ee
wobei PCAC den Pionbeitrag zum longitudinalen Anteil $\rho_A^{||}(q^2)=
f_\pi^2 \delta(q^2-m_\pi^2)$ fixiert. Im chiralen Limes findet man 
damit folgende Summenregeln vom FESR-Typ
\beq
\label{afesr1}
  t_c F_2(t_c) 
     &=&    8\pi^2 \int_{s_0}^{t_c} \rho_A(s) \,ds  + 8\pi^2f_\pi^2 \\
\label{afesr2}
  c_4 + t_c^2 F_4(t_c) 
     &=&    16\pi^2 \int_{s_0}^{t_c} \rho_A(s)\, s\, ds  \\     
\label{afesr3}
  -\frac{11}{7}\frac{\xi^A}{\xi^V} c_6 - \frac{t_c^3}{2} F_6(t_c) 
     &=&    12\pi \int_{s_0}^{t_c}  \rho_A(s)\, s^2\, ds  \, .
\eeq
Kombiniert man diese Ergebnisse mit den Resultaten aus dem 
Vektorkanal, so erh\"alt man die FESR-Varianten der  
Weinberg-Summenregeln \cite{Wei67}. Ber\"ucksichtigt man 
die Effekte der endlichen Stromquarkmassen, so divergieren diese
Summenregeln allerdings im Limes $t_c\to\infty$ \cite{FNR79}. 
Es ist daher in der Regel sinnvoller, die Boreltransformierte Version der 
Weinberg-Relationen zu studieren \cite{PS87}. 

Auch die Summenregeln im Axialvektorkanal sind mit Hilfe des einfachen 
Ansatzes 
\be
 \rho_A (s) = \frac{m_{a_1}^2}{g_{a_1}^2} \delta (s-m_{a_1}^2)
   + \frac{1}{8\pi^2}\left( 1+\frac{\alpha_s}{\pi}\right) \Theta (s-s_{th})
\ee	      
untersucht worden \cite{RRY85}. Die Kontinuumsschwelle l\"a\ss t sich
in diesem Fall entweder aus den FESR-Bedingungen oder aus der
Forderung nach \"Aquivalenz der beiden boreltransformierten 
Summenregeln bestimmen. Beide Verfahren liefern \"ubereinstimmend
den Wert $s_{th}=2.3\,{\rm GeV}^2$. Verwendet man diese Resultat,
um die Parameter der $a_1$-Resonanz zu fixieren, so findet
man $m_{a_1}=1230$ MeV und $\frac{g_{a_1}^2}{4\pi}=5.3$ in recht
guter \"Ubereinstimmung  mit den experimentellen Werten. Dieses
Resultat ist umso \"uberraschender, als die experimentelle
Breite der $a_1$-Resonanz im $\tau$-Zerfall mit $\Gamma_{a_1}=
521$ MeV \cite{Alb86} sehr gro\ss\ ist.
