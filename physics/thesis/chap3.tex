\chapter{Photoproduktion von Eta-Mesonen}
%revised Jan. 2, 1992
Die Eta-Mesonen $\eta$ und $\eta'$ tragen wie das neutrale Pion
die Quantenzahlen $J^{PC}=0^{-+}$, besitzen aber den Isospin $I=0$.
Sie sind eine Mischung der Flavor-$SU(3)$ Singlet- und 
Oktett-Zust\"ande $\eta_0$ und $\eta_8$:
\newcommand{\thp}{\theta_{\eta\eta'}}
\be
\label{etamix}
\left( \begin{array}{c} \eta \\ \eta' \end{array} \right) =
\left( \begin{array}{cc} 
       \cos\thp  &   -\sin\thp  \\
       \sin\thp  &   \spm\cos\thp 
\end{array} \right)
\left( \begin{array}{c} \eta_8 \\ \eta_0 \end{array} \right)  .       
\ee   
Mit Hilfe quadratischer Massenformeln findet man einen 
Mischungswinkel $\thp =-11^\circ$,
w\"ahrend eine Analyse der Zerf\"alle $\eta\to 2\gamma$ und $\eta'\to 
2\gamma$ den Wert $\thp=-20^\circ$ liefert \cite{PDG90}. Die Masse
des Eta-Mesons betr\"agt $m_\eta=549$ MeV, w\"ahrend das $\eta'$-Meson
mit $m_{\eta'}=958$ MeV deutlich schwerer ist. Diese Tatsache ist 
eine Konsequenz der $U(1)_A$-Anomalie, die die Entartung zwischen 
den $SU(3)$ Singlet- und Oktett-Zust\"anden aufhebt.

Wir wollen in diesem Kapitel die Photoproduktion von Eta-Mesonen in
der N\"ahe der Schwelle untersuchen. Diese Reaktion liefert wesentliche
Informationen \"uber die Kopplung des Eta-Mesons an das Nukleon und 
die $N(1535)$-Resonanz \cite{TDR88}. Dar\"uber hinaus erhoffen wir uns
Hinweise auf die Rolle der chiralen Symmetriebrechung im Eta-Kanal
und den Strangeness-Inhalt des Protons.

Gegenw\"artig existieren keine zuverl\"assigen Daten zur Eta-Photoproduktion
in der Schwellenregion. Diese Situation wird sich in naher Zukunft durch
Experimente, die an den neuen, kontinuierlichen Elektronbeschleunigern
ELSA und CEBAF in Planung sind, deutlich verbessern. Eine Messung der
Eta-Photoproduktionsamplitude ist dar\"uber hinaus am MIT-Bates Labor 
im Gange.  

Wir haben im letzten Kapitel demonstriert, da\ss\  die elektrische 
Dipolamplitude f\"ur die Photoproduktion von Pionen durch die Nukleon- und
Pion-Bornterme dominiert wird. In der Eta-Photoproduktion hingegen
ist die Schwellenenergie mit $\omega^{\em cms}_{th}=448$ MeV deutlich
gr\"o\ss er, so da\ss\ die Energie des Photons vergleichbar ist mit der 
Anregungsenergie der ersten Resonanz in der $E_{0+}$-Amplitude. Aus 
diesem Grund ist es im engeren Sinne nicht m\"oglich, Niederenergietheoreme
zur  Eta-Photoproduktion zu formulieren. Trotzdem kann man die im letzten 
Kapitel entwickelten Methoden auch auf die Photoproduktion von Eta-Mesonen 
anwenden. Die wichtigsten Beitr\"age zu diesem Proze\ss\ stammen vom 
Nukleon-Bornterm, der $N(1535)$-Resonanz sowie dem Austausch
von Vektormesonen. Wir beschreiben diese Terme mit Hilfe der
Lagrangedichte
\beq
 {\cal L}_{\eta NN} &=& \frac{f_{\eta NN}}{m_\eta} 
     \bar{\psi} \gamma_5 \gamma_\mu \psi \partial^\mu \eta \; ,\\
 {\cal L}_{\eta NN^{*}} &=& \frac{f_{\eta NN^{*}}}{m_\eta} 
     \bar{\psi}_{N^{*}} \gamma_\mu \psi \partial^\mu \eta + h.c. \; ,\\
 {\cal L}_{{\mini V}\eta\gamma} &=& \frac{g_{{\mini V}\eta\gamma}}{2m_\eta}
                 \epsilon_{\alpha\beta\gamma\delta}
		  F^{\alpha\beta} \partial^\gamma
		  V^\delta \eta  \; ,
\eeq
wobei $V$ f\"ur die Vektormesonen $\rho,\omega$ steht. \"Uber
die St\"arke der $\eta NN$-Wechselwirkung gibt es nur sehr
widerspr\"uchliche Informationen \cite{Dum82}. Eine einfache
Absch\"atzung gewinnt man mit Hilfe der $SU(3)$-Relation zwischen
den pseudoskalaren Kopplungen
\be
\label{octcoup}
  g_{\eta_8} = \frac{1}{\sqrt{3}}\, \frac{3F/D-1}{F/D+1} \, g_\pi \; .
\ee      
Dabei bezeichnen $D,F$ die symmetrischen bzw.~antisymmetrischen 
$SU(3)$-Kopplungen. Ph\"anomenologische Untersuchungen haben f\"ur diese 
Parameter die Werte $F=0.47\pm 0.04$ und $D=0.81\pm 0.03$ ergeben \cite{JM90}.
Unter Verwendung von $SU(6)$-Spin-Flavor-Wellenfunktionen gilt 
dar\"uber hinaus $g_{\eta_8} = \frac{\sqrt{3}}{5}g_\pi$ und 
$g_{\eta_0} = \frac{\sqrt{6}}{5}g_\pi$. Die Kopplung der physikalischen
Zust\"ande ist recht sensitiv auf die Gr\"o\ss e der $\eta-\eta'$ 
Mischung. F\"ur Mischungswinkel im Bereich $\thp=0^\circ -25^\circ$
finden wir $g_\eta=4.6-7.0$. 

F\"ur unsere Berechnung verwenden wir den Wert $\thp=-12.5^\circ$, 
der die Pseudovektorkopplung $f_{\eta NN}=1.7$ ergibt. 
Der Nukleon-Bornterm liefert in diesem Fall
\beq
 E_{0+}^N (\eta p) &=& -\frac{e}{4\pi} \frac{f_{\eta NN}}{m_\eta}
   \frac{\mu_\eta}{(1+\mu_\eta)^{3/2}}\, \left( 1-\frac{\mu_\eta}{2}
   \kappa_p \right) \\[0.2cm]
   &\simeq& \mbox{} -1.4 \su , \nonumber 
\eeq     
wobei $\mu_\eta=m_\eta/M$ das Verh\"altnis der Eta- zur Nukleonmasse
bezeichnet. Im Gegensatz zum Niederenergietheorem zur 
Pionphotoproduktion ist es hier wenig sinnvoll, den kinematischen 
Faktor in Potenzen von $\mu_\eta$ zu entwickeln. 
 
Die $\eta NN^{*}$-Kopplung l\"a\ss t sich analog zu Gleichung (\ref{rescoup})
aus dem Wert der partiellen Breite $\Gamma (N(1535) \to N\eta)=75$ MeV
zu $f_{\eta NN^{*}}=1.94$ bestimmen. Bei der Photoproduktion von 
Etamesonen mu\ss\ man selbst an der Schwelle die endliche Breite der
Resonanz auf Grund des offenen Zerfallskanals $N^{*}\to N\pi$
ber\"ucksichtigen. Zu diesem Zweck  parametrisieren wir die 
Energieabh\"angigkeit der totalen Breite in der Form
\be
 \Gamma (s) = \Gamma_\pi \left(\frac{q_\pi}{q_\pi^R} \right)
   + \Gamma_\eta \left( \frac{q_\eta}{q_\eta^R} \right) \; ,
\ee
wobei $q_\pi,q_\eta$ die Impulse der Mesonen im Schwerpunktsystem
und $q_\pi^R,q_\eta^R$ die entsprechenden Werte bei $\sqrt{s}=M_R$
bezeichnen. An der $\eta N$-Schwelle ist dann $\Gamma (s_{th})=59$
MeV. Mit Hilfe des im letzten Abschnitt bestimmten anomalen magnetischen
Moments f\"ur den \"Ubergang $\gamma N \to N^{*}$ finden wir
f\"ur den Resonanzbeitrag
\begin{table}
\caption{Vergleich der f\"uhrenden Beitr\"age zur elektrischen
Dipolamplitude f\"ur die Reaktionen $\gamma p \to \pi^0 p$ und 
$\gamma p \to \eta p$. $E_{0+}$ in Einheiten $10^{-3} m_\pi^{-1}$.}
\begin{center}
\begin{tabular}{|l||r|r|}\hline
               & $\gamma p\to \pi^0 p$  &  $\gamma p\to \eta p$ \\ \hline\hline
 Nukleon-Bornterm               & $ -2.32$& $-1.4$              \\
 Vektormesonen ($\rho,\omega$)  & $0.21$  &  3.6	         \\
 Resonanz $N(1535)$             & $0.06$  &   9.4                \\
 Total                          & $-2.05$ &  11.6                 \\ \hline
\end{tabular}
\end{center}    
\end{table}
\be
  {\rm Re} E_{0+}^{N^{*}} (\eta p) = 9.4  \su .
\ee
Auch Vektormesonen spielen in der Etaproduktion eine deutlich
gr\"o\ss ere Rolle, als dies in der Pionproduktion der Fall ist.
Die Vektormeson-Nukleon Kopplungskonstanten haben wir bereits im
letzten Kapitel diskutiert. Am $VNN$-Vertex verwenden wir einen 
Formfaktor der Monopolform
\be
 F(t) = \frac{\Lambda^2-m_V^2}{\Lambda^2-t}
\ee
mit dem Cutoff $\Lambda=1.4$ GeV. Die $V\eta\gamma$-Kopplungen lassen sich
aus den experimentell bestimmten Zerfallsbreiten $\Gamma (V\to\eta
\gamma )$ ermitteln. Mit den Werten  aus \cite{Dum82} finden wir  
\be
 f_{\rho\mini NN}g_{\rho\eta\gamma}(1+\kappa_\rho)
  +f_{\omega\mini NN}g_{\omega\eta\gamma} \simeq 20.63 \; .
\ee     
Da die $\rho$- und $\omega$-Massen praktisch entartet sind, bestimmt
diese effektive Kopplung den Vektormesonbeitrag zur Etaproduktion
an der Schwelle
\beq
 E_{0+}^{V}(\eta p) &=& \frac{eM}{16\pi}
 \sum_V f_{\mini VNN}g_{{\mini V}\eta\gamma}(1+\kappa_V)
 \mu_\eta^2 \frac{2+\mu_\eta}
   {(1+\mu_\eta)^{3/2}} \frac{F(t)}
   {m_\eta^2 +m_V^2(1+\mu_\eta)}  \\[0.2cm]
   &\simeq& 3.58 \su \; . \nonumber
\eeq
Wir haben die bisher diskutierten Beitr\"age in Tabelle 3.1 gesammelt 
und mit den entsprechenden Resultaten in der Pionphotoproduktion 
verglichen. Man erkennt deutlich den sehr unterschiedlichen Charakter
dieser beiden Prozesse. 

Es ist interessant, die m\"oglichen Auswirkungen expliziter chiraler
Symmetriebrechung in der Photoproduktion von Eta-Mesonen  zu
untersuchen. Eine denkbare Konsequenz ist die Tatsache, da\ss\ an der
physikalischen Schwelle die Pseudovektorkopplung des Eta-Mesons
an das Nukleon nicht notwendig bevorzugt ist. Wir betrachten daher
die allgemeinere Wechselwirkung \cite{BM91}
\be
 {\cal L}_{\eta NN} = (1-\epsilon)\frac{f_{\eta NN}}{m_\eta}       
     \bar{\psi} \gamma_\mu \gamma_5 \psi \partial^\mu \eta
     + i\epsilon g_{\eta NN} \bar{\psi}\gamma_5\psi \eta \; ,
\ee
welche so konstruiert ist, da\ss\ die Kopplung f\"ur Nukleonen 
auf der Massenschale unabh\"angig von dem Parameter $\epsilon$ ist.
Dagegen zeigt der Nukleonbeitrag zur elektrischen Dipolamplitude 
an der Schwelle
\beq
 E_{0+}^N (\eta p) &=& -\frac{e}{4\pi} \frac{f_{\eta NN}}{m_\eta}
   \frac{\mu_\eta}{(1+\mu_\eta)^{3/2}}\, \left( 1 +
   \kappa_p \left( \epsilon -(1-\epsilon)\frac{\mu_\eta}{2} \right) \right) 
   \\[0.2cm]
   &\simeq& -(1.4+6.1\epsilon)\su \nonumber
\eeq   
eine starke Abh\"angigkeit von $\epsilon$ . Welcher Wert
von $\epsilon$ die beste Beschreibung der Etaproduktion an der
Schwelle liefert, l\"a\ss t sich letztlich nur durch eine
sorgf\"altige Untersuchung der differentiellen Wirkungsquerschnitte
unterhalb der Resonanzregion entscheiden. In derselben Weise
kann man auch f\"ur die $N(1535)$-Anregung eine skalare anstatt
der oben beschriebenen vektoriellen Kopplung verwenden
\be
 {\cal L}_{\eta NN^*} = (1-\alpha) \frac{f_{\eta NN^*}}{m_\eta}
    \bar{\psi}_{N^*}\gamma_\mu\psi\partial^\mu\eta
    +i\alpha g_{\eta NN^*}\bar{\psi}_{N^*}\psi\eta + h.c. \; .
\ee
Diese Modifikation beeinflu\ss t lediglich den nichtresonanten
Untergrund und hat daher nur geringe Auswirkung auf die
elektrische Dipolamplitude an der Schwelle. 

Wie im Abschnitt 2.5 diskutiert, kann die explizite Brechung der chiralen
Symmetrie, in diesem Fall insbesondere durch die Masse des seltsamen 
Quarks, eine zus\"atzliche Korrektur an die elektrische Dipolamplitude
liefern. Betrachtet man das Eta-Meson als reinen
Oktett-Zustand, dann ergibt sich mit Hilfe der oben verwendeten
Methoden
\beq
\label{sigeta}
 \Delta E_{0+} (\eta p)&=& \frac{e}{4\pi} \frac{1}{1+\mu_\eta}
   \frac{\overline{m}}{f_\eta m_\eta}
    \big( b_0 g_T^{\em sing} + b_3 g_T^3 + \delta b \,(g_T^8-g_T^{\em sing})
   \big) \; ,   \\[0.1cm]
   b_0      &=& \frac{1}{3\sqrt{3}}\; , 
                \hspace{2.5cm} b_3=\frac{1}{\sqrt{3}} \; ,\\
   \delta b &=& \frac{1}{9\sqrt 3}\left( 1 -4\frac{m_s}{\overline m}
                \right)\,  , 	   
\eeq
wobei $g_T^{\em sing}$ die Flavorsingletkopplung im Tensorkanal bezeichnet
und $f_\eta\simeq f_\pi$ die Eta-Meson-Zerfallskonstante ist. 
Die St\"arke der Korrektur h\"angt wesentlich von der Gr\"o\ss e
des flavormischenden Parameters $\delta g_T \equiv g_T^8-g_T^{\em sing}$ ab.

Nach der ph\"anomenologisch erfolgreichen Zweig-Regel verschwindet 
der Strangeness-In\-halt des Protons, und es gilt $\delta g_T=0$.
In diesem Fall ist $\Delta E_{0+}(\eta p)$ proportional zu 
$\overline{m}/m_\eta$ und liefert nur geringf\"ugige Korrekturen.
Verwendet man das in Abschnitt 2.5 diskutierte nichtrelativistische
Quarkmodell, dann ist $g_T^{\em sing}=1$ sowie $g_T^3=5/3$, und wir finden
$\Delta E_{0+}(\eta p)=0.4\su$. Ist dagegen $\delta g_T\neq 0$,
so ist die Korrektur proportional zu $m_s/m_\eta$ und kann einen 
substantiellen Beitrag zur Schwellenamplitude liefern. 
