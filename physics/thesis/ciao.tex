\addcontentsline{toc}{chapter}{Zusammenfassung}
% revised Jan. 1, 1992
\vspace*{4cm}
{\Huge \bf Zusammenfassung} 
\\[4cm]
Experimente, die an den Elektronenbeschleunigern ALS in Saclay und
MAMI A in Mainz durchgef\"uhrt wurden, haben zu einem starken 
Wiederaufleben des Interesses an der Pionphotoproduktion gef\"uhrt.
Stellvertretend f\"ur diese Experimente haben wir im ersten Kapitel 
die Daten aus Mainz diskutiert. Dabei haben wir gezeigt, da\ss\ die
experimentellen Ergebnisse direkt an der Schwelle nicht im Widerspruch
zu der Vorhersage des Niederenergietheorems, $\Eop=-2.3\su$, stehen. 
Unerwartet ist allerdings die starke Energieabh\"angigkeit der
elektrischen Dipolamplitude in der Schwellenregion. 

Vor diesem Hintergrund haben wir die theoretischen Grundlagen der
Stromalgebravorhersage untersucht. Das Niederenergietheorem liefert
nur im Grenzfall verschwindender Pionmasse eine eindeutige Bestimmung 
der Schwellenamplitude. Um die elektrische Dipolamplitude f\"ur 
massive Pionen zu bestimmen, entwickelt man die \"Ubergangsmatrix in 
eine Potenzreihe in $\mu=m_\pi/M$. Das oben zitierte Ergebnis
$\Eop=-2.3\su$ beruht auf der Berechnung der Photoproduktionsamplitude
bis zur Ordnung $\mu^2$. Um die Eindeutigkeit dieses Ergebnisses zu
demonstrieren, ben\"otigt man allerdings zus\"atzliche Annahmen, 
die \"uber Stromalgebra und PCAC hinausgehen. 

Wir haben daher die Bedeutung verschiedener Korrekturen zum 
Niederenergietheorem studiert. W\"ahrend Nukleonresonanzen und 
Vektormesonen keine wesentlichen Beitr\"age zur Schwellenamplitude 
liefern, ergeben sich im Rahmen eines einfachen Quarkmodells
bedeutende Korrekturen auf Grund der expliziten Symmetriebrechung
durch die nichtverschwindenden Strommassen der Quarks. 

Allerdings verletzen nichtrelativistische Quarkmodelle die chirale
Symmetrie, auf der die Niederenergietheoreme basieren, bereits im
Ansatz. Es ist daher von gro\ss er Bedeutung, die Effekte der 
endlichen Strommassen auch in chiralen Modellen des Nukleons zu
untersuchen. Dabei haben wir gezeigt, da\ss\ Modelle, die eine 
gute ph\"anomenologische Beschreibung insbesondere der axialen
Struktur des Nukleons beinhalten, Korrekturen $\DEop <0.5\su$
liefern. Diese Schranke ist deutlich kleiner als die in 
nichtrelativistischen Quarkmodellen gewonnene Absch\"atzung
$\DEop =1.6\su$, so da\ss\ wir keine wesentliche Modifikation 
des Niedernenergietheorems finden.

Wichtige Aufschl\"usse \"uber die Rolle der expliziten Symmetriebrechung 
erhoffen wir uns aus der Bestimmung der elektrischen Dipolamplitude
f\"ur die Produktion neutraler Pi\-onen am Neutron. Eine experimentelle
Untersuchung dieser Reaktion mit Hilfe der quasifreien Pionproduktion
am Deuteron befindet sich in Mainz in Planung. Ein deutliches
Abweichen der Schwellenamplitude von der Vorhersage des 
Niederenergietheorems, $\Eon =-0.5\su$, w\"are ein Hinweis auf eine
starke Brechung der Isospinsymmetrie in der Pionphotoproduktion.

Ebenfalls im Planungsstadium befinden sich Experimente zur Photoproduktion
von Eta-Mesonen. Die theoretische Bestimmung der zugeh\"origen 
Schwellenamplitude liefert das Resultat $E_{0+}(\eta p)=11.6\su$.
Diese Amplitude ist im wesentlichen durch die $N(1535)$-Resonanz 
dominiert, deren Photoanregung in der Pionproduktion studiert werden 
kann.  Die wichtigste Unsicherheit bei der Bestimmung von $E_{0+}
(\eta p)$ besteht daher nicht in der Festlegung der Resonanzparameter,
sondern in der Form der Eta-Nukleon-Kopplung.
\\

Im zweiten Teil der Arbeit haben wir die Spektren der Vektor- und 
Axialvektormesonen mit Hilfe von QCD-Summenregeln untersucht. Das
Spektrum der Vektormesonen  haben wir aus den Daten zur $e^+e^-$
Annihilation in Hadronen entnommen, w\"ahrend die spektrale Dichte
der Axialvektormesonen mit Hilfe der invarianten Massenverteilungen
im $\tau$-Zerfall bestimmt wurde.

Auf Grund seines g\"unstigen Verhaltens bez\"uglich der experimentellen
Unsicherheiten haben wir den Borelquotienten der Spektralfunktionen 
verwendet, um die G\"ultigkeit der Summenregeln zu untersuchen.
Im Vektorkanal zeigt sich ein gute \"Ubereinstimmung der Daten mit der
QCD-Parametrisierung. Da die Ergebnisse nicht stabil bez\"uglich der
Wahl des Borelparameters sind, erm\"oglichen die Resultate aber keine
eindeutige Bestimmung der Quark- und Gluonkondensate. Diese 
Feststellung befindet sich im Widerspruch zu fr\"uheren Arbeiten zu
diesem Thema \cite{LNT84}. Zuverl\"assig bestimmt ist lediglich das 
Verh\"altnis der Quark- und Gluonkondensate. Verwendet man den
Standardwert des Gluonkondensats, $\langle\frac{\alpha_s}{\pi}G^2
\rangle =(360\pm 20\,\mev)^4$, so finden wir eine Verletzung der
Faktorisierunghypothese f\"ur die Kondensate $\langle\bar\psi
\Gamma_1\psi\bar\psi\Gamma_2\psi\rangle$ um den Faktor $\xi^V =2.4$. 

Das Spektrum der Axialvektormesonen haben wir zun\"achst im Rahmen 
eines Modells studiert, in dem die Daten aus dem $\tau$-Zerfall 
jenseits der Masse $m_\tau =1.784\,\gev$ des $\tau$-Leptons mit 
Hilfe des st\"orungstheoretischen Resultats fortgesetzt wurden.
Dieses Modell liefert allerdings eine Spektralfunktion, die nicht
in der Lage ist, die Summenregeln  zu saturieren. 

Wir haben daher das Verfahren umgekehrt und QCD-Summenregeln eingesetzt,
um die Parameter der zweiten Resonanz im Spektrum zu bestimmen.
Die Summenregeln liefern eine Resonanzmasse $m_{a_1'}=1.7\,\gev$,
die im Bereich der $\tau$-Masse liegt. Diese Tatsache, zusammen 
mit der gro\ss en Kopplung der $a_1'$-Resonanz, die wir gefunden haben,
sollte es erm\"oglichen, diese Anregung in zuk\"unftigen Experimenten 
zu identifizieren.
