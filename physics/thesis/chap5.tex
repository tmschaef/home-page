\chapter[Spektrale Summenregeln in der QCD]{Spektrale Summenregeln 
in der Quantenchromodynamik}
% revised Jan. 1, 1992
\section{Einf\"uhrung}
In Teil I dieser Arbeit haben wir Stromalgebratechniken verwendet,
um die Photoproduktion von Pionen am Nukleon in der N\"ahe der Schwelle 
zu untersuchen. Die Photoproduktionsamplitude ist mit Hilfe
der Reduktionsformel durch die Zweipunktfunktion eines Vektor- und 
eines Axialvektorstroms zwischen Nukleonzust\"anden bestimmt. Diese 
Korrelationsfunktion haben wir approximiert, indem wir nur Zwischenzust\"ande 
betrachtet haben, die einzelne Nukleonen oder Pionen enthalten.  

In diesem Teil der Arbeit wollen wir die Korrelationsfunktionen von
Vektor- und Axialvektorstr\"omen im Vakuum untersuchen. Zwar lassen 
sich auch diese Funktionen f\"ur kleine Impuls\"ubertr\"age durch die
Beitr\"age der niedrigsten Resonanzen approximieren, im Gegensatz 
zur oben betrachteten Situation besteht aber die M\"oglichkeit, die
zugeh\"origen Spektren direkt aus experimentellen Daten 
zu rekonstruieren. 

Auf Grund der Analytizit\"at der Korrelationsfunktion bestimmt die
Spektralfunktion das Verhalten des Korrelators in der gesamten 
komplexen Ebene. Damit ergibt sich die M\"oglichkeit,
die Konsistenz der Daten mit Theorien \"uber das asymptotische 
Verhalten des  Korrelators in QCD zu vergleichen. Historisch ist 
eine solche Analyse das erste Mal im  Charmonium-System durchgef\"uhrt
worden \cite{NOS78}. \"Ahnliche Untersuchungen existieren auch  f\"ur
den Vektorkorrelator \cite{LNT84,CM90,Cap91}, nicht aber f\"ur das 
Spektrum der Axialvektormesonen. Wir werden daher im n\"achsten 
Kapitel eine eingehende Analyse der existierenden Daten \"uber die 
Spektralfunktion im Axialvektorkanal vornehmen. Zun\"achst wollen wir 
jedoch eine kurze Einf\"uhrung in die Methode der QCD-Summenregeln geben. 

\section{Summenregeln und die Operatorprodukt\-ent\-wick\-lung}
Die Struktur der Quantenchromodynamik bei niederen Energien 
wird durch eine Reihe nichtperturbativer Ph\"anomene bestimmt.
Von Bedeutung sind vor allem  Confinement, der permanente
Einschlu\ss\ von Quarks und Gluonen in farbneutralen Hadronen,
sowie das Auftreten von Kondensaten, das hei\ss t 
nichtverschwindenden Vakuumerwartungswerten von  Quark- oder 
Gluonoperatoren. Ein Beispiel f\"ur die Rolle der Kondensate
bei der Bestimmung des Spektrums liefert der Vergleich  
der Korrelationsfunktionen im Vektor- und Axialvektorkanal
\beq
\label{vvcor}
  \Pi_{\mu\nu}^V  &=& -(g_{\mu\nu}q^2-q_\mu q_\nu) \Pi^V(q^2)  \\[0.1cm]
   & &\hspace{0.3cm} = \, i\int d^4x\, e^{iq\cdot x}\, \langle 0|T\left( 
   j_\mu^{(\rho)}(x)j_\nu^{(\rho)}(0) \right) |0\rangle \; , \nonumber \\
\label{aacor}
  \Pi_{\mu\nu}^A &=& -g_{\mu\nu}q^2 \Pi^{A_1}(q^2)
              + q_\mu q_\nu \Pi^{A_2}(q^2)   \\[0.1cm]
& & \hspace{0.3cm}  =\, i\int d^4x\, e^{iq\cdot x} \,\langle 0|T\left( 
   j_\mu^{(a_1)}(x)j_\nu^{(a_1)}(0) \right) |0\rangle \;  . \nonumber
\eeq
Die Str\"ome $j_\mu^{(\rho)} =\frac{1}{2}(\bar u\gamma_\mu u-
\bar d\gamma_\mu d)$ und $j_\mu^{(a_1)}=\frac{1}{2} (\bar u
\gamma_\mu\gamma_5 d-\bar d\gamma_\mu\gamma_5 d)$ tragen die 
Quantenzahlen des $\rho$ bzw.~$a_1$-Mesons. Wir betrachten den
isospinsymmetrischen Fall, das hei\ss t der Vektorstrom $j_\mu^{(\rho)}$
ist erhalten und die entsprechende Korrelationsfunktion transversal. 

Im chiralen Limes sind die St\"orungsentwicklungen der 
beiden Korrelationsfunktionen im Vektor- und Axialvektorkanal
identisch. Dagegen werden wir im n\"achsten Kapitel zeigen, 
da\ss\ die zugeh\"origen Spektralfunktionen eine sehr unterschiedliche 
Gestalt haben. In der Tat ist es ein nichtperturbativer  Effekt, die 
spontane Brechung der chiralen Symmetrie hervorgerufen durch die 
nichtverschwindenden Quarkkondensate, welcher die Aufspaltung der 
$\rho$- und $a_1$-Massen bewirkt.

Von ebenso gro\ss er Bedeutung f\"ur den Niederenergiesektor der QCD ist das
Gluonkondensat $\langle G_{\mu\nu}^{a}G^{a}_{\mu\nu}\rangle $. Dieses 
Matrixelement  bestimmt den gluonischen Beitrag zur Vakuumenergie und
ist ein Ordnungsparameter f\"ur die Brechung der Invarianz der QCD 
unter Skalentransformationen.
  

Wir ber\"ucksichtigen die
Rolle der Kondensate in der Korrelationsfunktion mit Hilfe
der Operatorproduktentwicklung (OPE), g\"ultig im tief
euklidischen Bereich $Q^2=-q^2\to\infty$
\be
\label{ope}
 \Pi (Q^2) =  C_{1\!\!1}  + \sum_{n=2} \frac{1}{Q^{2n}}
        C_n \langle 0| {\cal O}_n|0\rangle  \; .
\ee
Dabei liefert der Koeffizient $C_{1\!\!1}$ des Einheitsoperators
die \"ubliche St\"orungsreihe in Potenzen von $\alpha_s$,
w\"ahrend ${\cal O}_n$ einen Operator der Dimension $d=2n$
bezeichnet, dessen Vakuumerwartungswert durch nichtperturbative
Effekte bestimmt wird. Der kurzreichweitige Teil der 
Wechselwirkung der Str\"ome mit den Operatoren ${\cal O}_n$ ist
in den Wilson-Koeffizienten $C_n$ enthalten, die sich in 
in der \"ublichen Weise aus Feynman-Diagrammen bestimmen lassen. 
Nicht berechenbar auf dem gegenw\"artigen Stand der Theorie sind 
dagegen die  Vakuumerwartungswerte $\langle 0|{\cal O}_n|0\rangle $.

QCD Summenregeln machen von der Analytizit\"at der 
Korrelationsfunktion Gebrauch, um das Verhalten von $\Pi (Q^2)$ 
im tief euklidischen Bereich mit der Gestalt der Spektralfunktion 
bei niederen Energien in Verbindung zu bringen. Die Funktion 
$\Pi (Q^2)$ erf\"ullt die Dispersionsrelation 
\be 
\label{disprel}
 \Pi (Q^2) = (-1)^n \frac{Q^{2n}}{\pi} \int_{s_0}^{\infty}
    \frac{{\rm Im}\Pi (s)}{s^n(s+Q^2)} ds +
    \sum_{k=0}^{n-1} a_k Q^{2k}\, ,
\ee
wobei eine von den Eigenschaften des Stroms abh\"angige Zahl von 
Subtraktionen vorgenommen werden mu\ss .  Die entsprechenden  
Subtraktionskonstanten haben wir mit $a_k$ bezeichnet. 
Der Imagin\"arteil ${\rm Im}\Pi (Q^2)$ l\"a\ss t sich mit
Hilfe von  Unitarit\"atsbeziehungen aus physikalischen
Observablen bestimmen.

Im Prinzip existiert daher eine sehr starke Korrelation zwischen 
dem asymptotischen Verhalten der Funktion $\Pi (Q^2)$, parametrisiert
mit Hilfe der Kondensate $\langle {\cal O}_n\rangle $, und experimentell 
zug\"anglichen Gr\"o\ss en. In der Praxis ist man allerdings 
gezwungen, die OPE bei $n=3$ oder $n=4$ abzubrechen. Auch die
experimentellen Daten sind nur in einem beschr\"ankten Bereich 
bekannt und mit Fehlern behaftet. Wir werden uns daher mit der 
Frage befassen m\"ussen, in welchem Umfang eine Bestimmung der 
niederdimensionalen Kondensate aus den gemessenen Spektralfunktionen 
im Vektor- und Axialvektorkanal m\"oglich ist. 

Auf Grund experimenteller Unsicherheiten in der Bestimmung der Daten
sind verschiedene Formulierungen der Summenregel (\ref{disprel}) nicht 
in gleicher Weise zur Bestimmung der Vakuumparameter geeignet. Besonders 
bew\"ahrt hat sich die Boreltransformierte Dispersionsrelation   
\be
\label{borel}
 \hat L_B \Pi (Q^2) = \frac{1}{\pi} \int_{s_0}^{\infty} ds\,
  {\rm Im}\Pi (s) e^{-s\tau} \, ,
\ee  
wobei $\hat L_B$ den Boreloperator
\be
\label{borelop}
 \hat L_B = \left. \lim_{Q^2,n\to\infty} \frac{1}{(n-1)!}Q^{2n}
   \left( -\frac{d}{dQ^2} \right)^n  \right|_{n/Q^2=\tau} 
\ee
bezeichnet. Die Boreltransformation hat den Vorzug, das experimentell 
zuverl\"assiger bestimmte Niederenergieverhalten der 
Spektralfunktionen st\"arker zu gewichten und m\"ogliche 
Subtraktionskonstanten zu eliminieren. Dar\"uber hinaus werden 
h\"ohere Potenzen von $1/Q^2$ in der OPE durch zus\"atzliche 
Faktoren $1/n!$ unterdr\"uckt. 

Eine weitere Konsequenz der analytischen Struktur der 
Korrelationsfunktionen sind Summenregeln in einem endlichen
Energieintervall ({\em engl.} finite energy sum rule, FESR)
\be
\label{fesr}
 M_n (t_c)  = \frac{t_c^{n+1}}{\pi} \int_0^{2\pi}d\phi\,
     e^{i(n+1)\phi} \Pi (t_ce^{i\phi})\, .
\ee
Die rechte Seite der Summenregel testet das Verhalten der 
Funktion $\Pi (Q^2)$ auf einem Kreis in der komplexen
$q^2$ Ebene und l\"a\ss t sich f\"ur hinreichend gro\ss e 
$t_c$ mit Hilfe der OPE auswerten. Die linke Seite ist durch 
das n-te Moment der Spektralfunktion in einem endlichen
Intervall gegeben
\be
\label{moment}
 M_n (t_c) = \frac{1}{\pi} \int_{s_0}^{t_c}ds \, s^n 
    {\rm Im}\Pi (s)
\ee
und kann aus experimentellen Daten bestimmt werden. Ein 
Vorzug der FESR ist die Tatsache, da\ss\ die Relationen 
(\ref{fesr}) direkt auf Kondensate einer festen Dimension
projizieren. Allerdings sind diese Beziehungen f\"ur h\"ohere
$n$ zunehmend von Fehlern in den Eingabedaten beeintr\"achtigt. 

\section{Entwicklung nach lokalen Operatoren}
In diesem Abschnitt wollen wir die Wilson-Koeffizienten f\"ur die 
Korrelationsfunktionen im Vektor- und Axialvektorkanal angeben. Beginnen 
werden wir mit dem Vektorkorrelator
\beq
 \Pi_{\mu\nu}^V(q^2) &=& \frac{i}{4} \int d^4x \, e^{iq\cdot x}
    \langle 0|T\big[ (\bar u(0)\gamma_\mu u(0)-\bar d(0)\gamma_\mu d(0) )
    \\
    & & \hspace{4.3cm}  \cdot (\bar u(x)\gamma_\nu u(x)
       -\bar d(x)\gamma_\nu d(x) )\big] |0\rangle  \, .\nonumber
\eeq
Entwickelt man das zeitgeordnete Produkt der Feldoperatoren mit
Hilfe des Wick'schen Theorems und vernachl\"assigt die 
normalgeordneten Beitr\"age, so erh\"alt man die \"ubliche 
St\"orungsentwicklung, die den Koeffizienten
$C_{1\!\!1}$ bestimmt. Die Wilson-Koeffizienten der nichtperturbativen
Kondensate ergeben sich aus den normalgeordneten Beitr\"agen. 
Um Produkte von Feldoperatoren an verschiedenen Punkten als
Erwartungswerte lokaler Operatoren zu schreiben, entwickelt
man die Operatorprodukte in  eine  Taylorreihe in der 
Differenz der Argumente. Dieses Verfahren ist konsistent, da der 
Korrelator f\"ur hohe $Q^2$ durch das Produkt der Str\"ome bei
kleinen Abst\"anden bestimmt ist. Fluktuationen auf
gro\ss e L\"angenskalen manifestieren sich in den 
nichtverschwindenden Kondensaten. 


Die Operatorproduktentwicklung der Korrelationsfunktion l\"a\ss t
sich in die Form 
\be
\label{powexp} 
\Pi^V (Q^2) = -h_0^V \ln \left(\frac{Q^2}{\mu^2}\right) 
+ \sum_{n=2} \frac{1}{n}\,\frac{h_n^V}{Q^{2n}}
\ee
bringen, wobei $\mu^2$ die Renormierungsskala kennzeichnet.
Mit Hilfe der geschilderten Methoden findet man die 
Koeffizienten
\beq
 h_0^V &=& \frac{1}{8\pi^2} \left( 1+\frac{\alpha_s}{\pi}\right)\; , \\
 h_2^V &=& m_u\langle \bar{u}u\rangle +m_d\langle \bar{d}d\rangle  + 
  \frac{1}{12}  \langle \frac{\alpha_s}{\pi} 
        G_{\mu\nu}^{a}G^{a}_{\mu\nu} \rangle \; ,   \\
 h_3^V &=& -\frac{3\pi\alpha_s}{2}
        \langle \left(\bar{u}\gamma_\mu\gamma_5\lambda^{a}u -
	       \bar{d}\gamma_\mu\gamma_5\lambda^{a}d \right)^2 \rangle    \\
      & & \;\mbox{}-\frac{\pi\alpha_s}{3} 
       \langle  (\bar{u}\gamma_\mu\lambda^{a}u +
          \bar{d}\gamma_\mu\lambda^{a}d ) 
	  \sum_{q=u,d,s} \bar{q}\gamma_\mu\lambda^{a} q\rangle  \nonumber \, .
\eeq
Dabei bezeichnet $\alpha_s$ die impulsabh\"angige laufende 
Kopplungskonstante. In Ein-Schlei\-fen\-n\"ahe\-rung ist
\be
\label{runcoupl}
  \frac{\alpha_s(Q^2)}{\pi} = \left( -\frac{\beta_1}{2}
      \ln \left(\frac{Q^2}{\Lambda^2}\right) \right)^{-1}
\ee
mit $\beta_1=-\frac{11}{2}+\frac{N_f}{3}$ und dem Skalenparameter
$\Lambda=230\pm 80$ MeV \cite{PDG90}. Im perturbativen Teil
des Resultats haben wir Korrekturen auf Grund der endlichen 
Strommassen der Quarks vernachl\"assigt. St\"orungstheoretische 
Korrekturen zum Einheitsoperator sind bis zur Ordnung $\alpha_s^3$  
bestimmt worden \cite{GKL88}, w\"ahrend die entsprechenden Korrekturen an 
$\langle \bar qq\rangle $ und $\langle G^2\rangle $ nur  bis zur Ordnung $\alpha_s$ bekannt 
sind \cite{Nar89}. Wir haben die Operatorproduktentwicklung bei
Termen der Dimension $d=6$ abgebrochen. In der n\"achst\-h\"oheren 
Ordnung der OPE sind lediglich die Wilson-Koeffizienten f\"ur alle rein 
gluonischen Operatoren berechnet worden \cite{BG85}.

Der Beitrag  der Dimension $d=4$ in der OPE enth\"alt  neben
dem aus der GOR-Relation (\ref{GOR}) bekannten Matrixelement
$m_u\langle \bar uu\rangle +m_d\langle \bar dd\rangle $ auch den rein 
gluonischen Erwartungswert $\langle\frac{\alpha_s}{\pi}G^2\rangle $. Dieses 
Matrixelement ist besonders intensiv im Zusammenhang mit
Charmoniumzust\"anden untersucht worden. In diesem System 
ist die Rolle der Quarkkondensate auf Grund der gro\ss en
Masse des charm-Quarks unterdr\"uckt. Der kanonische
Wert des Gluonkondensats aus einer  Analyse der angeregten
Zust\"ande des $c\bar c$-Systems betr\"agt  $\langle \frac{\alpha_s}{\pi}
G^2\rangle =(360\pm 20{\rm MeV})^4$ \cite{RRY85}.  

Die Quarkkondensate der Dimension $d=6$  lassen sich mit Hilfe
der Faktorisierungshypothese \cite{SVZ79}
\be
\label{fact}
  \langle \bar\psi\Gamma_1\psi\bar\psi\Gamma_2\psi \rangle  =
    \frac{1}{N^2} \big( {\rm Tr}(\Gamma_1)\,{\rm Tr}(\Gamma_2)  
    - {\rm Tr} (\Gamma_1\Gamma_2) \big) \, \langle \bar\psi\psi \rangle ^2
\ee    
absch\"atzen. Dabei bezeichnet $\psi =(u,d,\ldots)$ einen Quarkspinor
in $SU(N_f)$, und die Normierungskonstante ist durch $N=4N_cN_f$
gegeben. Auf diese Weise findet man 
\be
  h_3 = -\frac{112}{27}\xi^V\pi\alpha_s \, \langle \bar qq\rangle^2\; ,
\ee
wobei die Gr\"o\ss e $\xi^V$ Abweichungen von der exakten 
Faktorisierung ($\xi^V=1$) parametrisiert. Da sich die 
anomalen Dimensionen der Kopplungskonstante und des Kondensats
praktisch aufheben, ist $h_3$ nur sehr schwach vom 
Normierungspunkt abh\"angig. Besitzt man eine Absch\"atzung
der Quarkmassen, so l\"a\ss t sich der Wert des Quarkkondensats  
mit Hilfe der GOR-Relation berechnen. Da weder das Kondensat noch die 
Strommassen invariant unter der Renormierungsgruppe sind, ist
es zu diesem Zweck allerdings sehr wichtig, die Skala zu 
kennen, bei der die Quarkmassen bestimmt worden sind. 
Verwendet man den Wert $\left. (m_u+m_d)\right|_{Q^2=1\,
{\rm GeV}^2}=14$ MeV, den wir in Kapitel 2 angegeben haben,
so findet man $\langle \bar qq\rangle =-(230\,{\rm MeV})^3$ bei $Q^2=1\,
{\rm GeV}^2$. 

Auf Grund der expliziten Brechung der chiralen Symmetrie wird
die Korrelationsfunktion im Axialvektorkanal durch zwei 
unabh\"angige invariante Funktionen charakterisiert. 
Diese lassen sich wie in (\ref{powexp}) entwickeln. F\"ur den
Koeffizienten der $q_\mu q_\nu$-Struktur findet man 
\beq
 h_0^{A_2} &=& \frac{1}{8\pi^2} \left( 1+\frac{\alpha_s}{\pi}\right) \; ,\\
 h_2^{A_2} &=& m_u\langle \bar{u}u\rangle + m_d\langle \bar{d}d\rangle  + 
      \frac{1}{12} \langle \frac{\alpha_s}{\pi} 
             G_{\mu\nu}^{a}G^{a}_{\mu\nu} \rangle\; ,    \\
 h_3^{A_2} &=& -\frac{3\pi\alpha_s}{2}
        \langle \left(\bar{u}\gamma_\mu\lambda^{a}u -
	       \bar{d}\gamma_\mu\lambda^{a}d \right)^2 \rangle    \\
     & & \;\mbox{}-\frac{\pi\alpha_s}{3} 
       \langle  (\bar{u}\gamma_\mu\lambda^{a}u +
          \bar{d}\gamma_\mu\lambda^{a}d ) 
	  \sum_{q=u,d,s} \bar{q}\gamma_\mu\lambda^{a} q\rangle  \nonumber \, .
\eeq
Im chiralen Limes sind der Koeffizient des Einheitsoperators 
sowie die Beitr\"age des Gluonkondensats im Vektor- und Axialvektorkanal
identisch. Betrachtet man den Koeffizienten $h_3$, so liefert
der Axialstromkorrelator eine andere Diracstruktur im ersten der beiden 
Beitr\"age. Verwendet man die Faktorisierungshypothese
\be 
 h_3^{A_2} = \left(\frac{11}{7}\right)\frac{112}{27} 
       \xi^A\pi\alpha_s \, \langle \bar qq\rangle ^2 \, ,
\ee
so ergibt sich eine gegen\"uber dem Resultat im Vektorkanal
nur geringf\"ugig modifizierte numerische Konstante, aber ein 
ge\"andertes Vorzeichen. 

Die Differenz der beiden invarianten  Strukturen $\Pi^{A_1}$ und 
$\Pi^{A_2}$ ist in f\"uhrender Ordnung durch 
\be
  \Pi^{A_1} (Q^2) - \Pi^{A_2} (Q^2) = 
      \frac{m_u\langle \bar uu\rangle + m_d\langle \bar dd\rangle }{Q^4} 
\ee
gegeben. Diese Relation steht in engem Zusammenhang zur PCAC-Hypothese.
Saturiert man die Differenz der beiden Spektralfunktionen mit dem
Pionbeitrag, so ergibt sich die in Kapitel 2 ausgiebig diskutierte 
GOR-Relation
\be
 -m_\pi^2 f_\pi^2 = m_u \langle \bar uu\rangle + m_d\langle 
 \bar dd\rangle  .
\ee
  
\section{Spektrale Summenregeln}
Mit Hilfe der im letzten Abschnitt bestimmten Wilson-Koeffizienten
k\"onnen wir nun die explizite Form der Summenregeln f\"ur die
Spektralfunktionen ${\rm Im}\Pi^{V,A}$ angeben. So lautet die
Boreltransformierte Dispersionsrelation im Vektorkanal
\be 
\label{borelsum}
 \frac{1}{\pi} \int_{s_0}^\infty {\rm Im}\Pi^V(s)e^{-s\tau}ds
      = \frac{1}{\tau} \left\{ h_0^V + \frac{h_2^V}{2!} \tau^2
         + \frac{h_3^V}{3!}\tau^3 + \ldots \right\} \, .
\ee	 
H\"ohere Momente der Borelsummenregel ergeben sich durch Differenzieren
von (\ref{borelsum}) nach $\tau$. Der Borelparameter $\tau$ 
kontrolliert die Gewichtung des Spektrums und die relative
Bedeutung h\"oherer Korrekturen in der OPE. Im Prinzip ist 
die Summenregel (\ref{borelsum}) f\"ur beliebige Werte von
$\tau$ g\"ultig. In der Praxis mu\ss\ man allerdings den 
Borelparameter in einem gewissen Bereich w\"ahlen, um die 
Fehler auf Grund der vernachl\"assigten Terme in der 
OPE und der Unsicherheiten in der Gestalt der Spektralfunktion
unter Kontrolle zu halten.

Die zu (\ref{borelsum}) analogen Summenregeln in einem endlichen 
Energieintervall lauten 
\beq
\label{fesr1}
  t_c F_2(t_c) 
     &=&    8\pi \int_{s_0}^{t_c} {\rm Im} \Pi^V(s) \,ds \; , \\
\label{fesr2}
  c_4 + t_c^2 F_4(t_c) 
     &=&    16\pi \int_{s_0}^{t_c} {\rm Im} \Pi^V(s)\, s\, ds \; , \\     
\label{fesr3}
  c_6 - \frac{t_c^3}{2} F_6(t_c) 
     &=&    12\pi \int_{s_0}^{t_c} {\rm Im} \Pi^V(s)\, s^2\, ds  \; .
\eeq
Die erste dieser Beziehungen illustriert sehr anschaulich das 
Konzept der Dualit\"at zwischen den Beschreibungen der 
Korrelationsfunktion bei kleinen bzw.~gro\ss en 
Abst\"anden. Obwohl die Spektralfunktion bei niederen Energien 
nicht durch freie Quark-Antiquark\-zu\-st\"an\-de, sondern durch 
Resonanzbeitr\"age bestimmt ist, liefert ihr Integral den 
st\"orungstheoretischen Wert. Die spektrale St\"arke der gebundenen 
Zust\"ande entspricht daher \"uber ein hinreichend gro\ss es Intervall 
gemittelt dem Resultat f\"ur  asymptotisch 
freie Quarks. H\"ohere Momente der Spektralfunktion enthalten die 
nichtperturbativen Korrekturen  
\be
\label{defcor}
  c_4 = -\frac{h_2^V}{h_0^V} \hspace{1cm} 
  c_6 = -\frac{h_3^V}{2h_0^V} \, .
\ee
Verwendet man den oben angegebenen Wert des Gluonkondensats und 
die Faktorisierungshypothese, so findet man $c_4=-0.07\,\gev^4$ 
und $c_6=0.04\,\gev^6$. Perturbative Korrekturen sind in den 
Funktionen $F_{2n}(t_c)$ ber\"ucksichtigt. In erster Ordnung in $\alpha_s$ ist
$F_{2n}(t_c) = (1+\alpha_s(t_c)/\pi)$, w\"ahrend die  
Korrekturen in h\"oherer Ordnung abh\"angig von $n$ ist. Resultate
bis $\alpha_s^2$ finden sich in \cite{BDL88}.

Um hadronische Parameter mit Hilfe der Summenregeln zu fixieren, 
ist man gezwungen, eine m\"oglichst einfache Parametrisierung der
Spektralfunktion zu verwenden. Im Vektorkanal lautet der Ansatz
\be
\label{zerow}
 \frac{1}{\pi} {\rm Im}\Pi^V(s) = \frac{m_\rho^2}{g_\rho^2}
    \delta (s-m_\rho^2) + \frac{1}{8\pi^2}
     \left( 1+\frac{\alpha_s}{\pi} \right) \Theta (s-s_{th})\; ,
\ee
wobei der erste Term den Beitrag der $\rho$-Resonanz im Grenzfall
verschwindender Breite ber\"ucksichtigt und der zweite 
Term das Quark-Antiquarkkontinuum repr\"asentiert. Die Kopplung
des $\rho$-Mesons ist durch das Matrixelement
\be
  \langle 0| J_\mu^{(\rho)}(0) |\rho (p)\rangle  = \epsilon_\mu \,\frac{g_\rho}{m_\rho^2}
\ee
definiert. Die beiden Parameter $m_\rho$ und $g_\rho$ lassen sich aus 
der Boreltransformierten Summenregel (\ref{borelsum}) und deren 
erstem Moment bestimmen. Zu diesem Zweck fixiert man die 
Kontinuumschwelle $s_{th}$ mit Hilfe der Konsistenzbeziehung
(\ref{fesr1}) und w\"ahlt $\tau =m_\rho^{-2}$, um den Beitrag der
$\rho$-Resonanz zur Summeneregel zu maximieren. 
Shifman, Vainshtein und Sakharov \cite{SVZ79} verwenden
\be
\label{svzval}
 c_4 =-0.07\;{\rm GeV}^4\hspace{1.5cm}  c_6 = 0.09\;{\rm GeV}^6 ,
\ee
wobei sich der abweichende Wert von $c_6$ aus der sehr kleinen 
Skala erkl\"art, bei der SVZ die oben angegebene Absch\"atzung der 
leichten Quarkmassen verwenden. Mit diesen Kondensaten ergibt sich 
$m_\rho = 774$ MeV (exp. 770 MeV) sowie $\frac{g_\rho^2}{4\pi} = 2.3$
(exp. 2.36) in hervorragender \"Ubereinstimmung mit den experimentellen 
Werten. 

Im Axialvektorkanal liefern die beiden Strukturen $\Pi^{A_1}$
und $\Pi^{A_2}$ zwei unabh\"angige Borelsummenregeln
\beq 
 \frac{1}{\pi} \int_{s_0}^\infty {\rm Im}\Pi^{A_1}(s)se^{-s\tau}ds
      &=& \frac{1}{\tau^2} \left\{ h_0^{A_1} - \frac{h_2^{A_1}}{2!} \tau^2
         - \frac{2h_3^{A_1}}{3!}\tau^3 - \ldots \right\}, \\
 \frac{1}{\pi} \int_{s_0}^\infty {\rm Im}\Pi^{A_2}(s)e^{-s\tau}ds
      &=& \,\frac{1}{\tau} \,\left\{ h_0^{A_2} + \frac{h_2^{A_2}}{2!} \tau^2
         + \frac{h_3^{A_2}}{3!}\tau^3 + \ldots \right\}	 .
\eeq
In der verwendeten N\"aherung sind die Koeffizienten $h_n$ in den 
beiden Summenregeln identisch bis auf den Beitrag des Quarkkondensats
\be
 h_2^{A_1} - h_2^{A_2} = 2(m_u\langle \bar uu\rangle \!+\, m_d\langle \bar dd\rangle ) \; .
\ee
Numerisch ist diese Differenz klein im Vergleich mit dem Beitrag des
Gluonkondensats, so da\ss\ in recht guter N\"aherung $h_2^{A_1}
\simeq h_2^{A_2}$ gilt. Die Spektralfunktionen im Axialvektorkanal
definieren wir durch 
\be
 \frac{1}{\pi} {\rm Im} \Pi_{\mu\nu}^{A}(q^2) =
    \rho_A(q^2) ( q_\mu q_\nu -g_{\mu\nu}q^2) +
    \rho_A^{||}(q^2) q_\mu q_\nu\; ,
\ee
wobei PCAC den Pionbeitrag zum longitudinalen Anteil $\rho_A^{||}(q^2)=
f_\pi^2 \delta(q^2-m_\pi^2)$ fixiert. Im chiralen Limes findet man 
damit folgende Summenregeln vom FESR-Typ
\beq
\label{afesr1}
  t_c F_2(t_c) 
     &=&    8\pi^2 \int_{s_0}^{t_c} \rho_A(s) \,ds  + 8\pi^2f_\pi^2 \; ,\\
\label{afesr2}
  c_4 + t_c^2 F_4(t_c) 
     &=&    16\pi^2 \int_{s_0}^{t_c} \rho_A(s)\, s\, ds \; , \\     
\label{afesr3}
  -\frac{11}{7}\frac{\xi^A}{\xi^V} c_6 - \frac{t_c^3}{2} F_6(t_c) 
     &=&    12\pi \int_{s_0}^{t_c}  \rho_A(s)\, s^2\, ds  \; .
\eeq
Kombiniert man diese Ergebnisse mit den Resultaten aus dem 
Vektorkanal, so erh\"alt man die FESR-Varianten der  
Weinberg-Summenregeln \cite{Wei67}. Ber\"ucksichtigt man 
die Effekte der endlichen Stromquarkmassen, so divergieren diese
Summenregeln allerdings im Limes $t_c\to\infty$ \cite{FNR79}. 
Es ist daher in der Regel sinnvoller, die Boreltransformierte Version der 
Weinberg-Relationen zu studieren \cite{PS87}. 

Auch die Summenregeln im Axialvektorkanal sind mit Hilfe des einfachen 
Ansatzes 
\be
\label{zerowa}
 \rho_A (s) = \frac{m_{a_1}^2}{g_{a_1}^2} \delta (s-m_{a_1}^2)
   + \frac{1}{8\pi^2}\left( 1+\frac{\alpha_s}{\pi}\right) \Theta (s-s_{th})
\ee	      
untersucht worden \cite{RRY85}. Die Kontinuumsschwelle l\"a\ss t sich
in diesem Fall entweder aus den FESR-Bedingungen oder aus der
Forderung nach \"Aquivalenz der beiden Boreltransformierten 
Summenregeln bestimmen. Beide Verfahren liefern \"ubereinstimmend
den Wert $s_{th}=2.3\,{\rm GeV}^2$. Verwendet man dieses Resultat,
um die Parameter der $a_1$-Resonanz zu fixieren, so findet
man $m_{a_1}=1230$ MeV (exp. 1260 MeV) und $\frac{g_{a_1}^2}{4\pi}=5.3$ 
(exp. 6.5) in recht guter \"Ubereinstimmung  mit den von der Particle 
Data Group angegebenen Werten \cite{PDG90}. Allerdings besitzt die 
$a_1$-Resonanz mit etwa 450 MeV eine so gro\ss e Breite, da\ss\ die 
Parametrisierung (\ref{zerowa}) wenig sinnvoll ist. Die geschilderte
Bestimmung der $a_1$-Masse erscheint daher weinger zuverl\"assig, als
das entsprechende Resultat f\"ur das $\rho$-Meson.
 
