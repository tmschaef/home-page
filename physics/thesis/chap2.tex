\chapter{Niederenergietheoreme zur Pionphotoproduktion}
%revised Jan. 2, 1992
\section{Einleitung}
Nachdem wir uns im letzten Kapitel mit der experimentellen 
Bestimmung der elektrischen Dipolamplitude an der Schwelle
befa\ss t haben, wollen wir uns nun auf die theoretische 
Bestimmung von $E_{0+}$ mit Hilfe von Niederenergietheoremen
konzentrieren. Die spezielle Bedeutung der Photoproduktion
neutraler Pionen ergibt sich dabei aus der Tatsache, da\ss\
die entsprechende Schwellenamplitude in einer hypothetischen Welt
mit masselosen Pionen  verschwinden w\"urde.
Dieser Kanal ist daher besonders sensitiv auf die Rolle der
expliziten chiralen Symmetriebrechung, welche sich in  dem
nur approximativen Charakter des Pions als Goldstoneboson 
widerspiegelt. 

Die physikalische Grundidee der Niederenergietheoreme
({\em engl.} Low Energy Theorem, LET) l\"a\ss t sich besonders
\"ubersichtlich an rein elektromagnetischen Reaktionen
diskutieren. Die Anwendung  von Niederenergietheoremen wird 
in diesem Fall durch zwei physikalische Kriterien kontrolliert. 
Die beiden Forderungen lauten, da\ss\ die Wellenl\"ange des 
Photons gro\ss\ ist im Vergleich zur Ausdehnung des Streuzentrums
und die Energie des Photons klein ist gegen die typische 
Anregungsenergie. Sind diese Voraussetzungen erf\"ullt, so ist das 
Photon nicht in der Lage, die innere Struktur des Targets aufzul\"osen. 
Der differentielle Wirkungsquerschnitt ist daher ausschlie\ss lich durch 
globale elektromagnetische Eigenschaften des Streuzentrums bestimmt.
Betrachtet man die Comptonstreuung niederenergetischer
Photonen an einem hadronischen Target, so folgt aus dieser
\"Uberlegung, da\ss\ der differentielle Wirkungsquerschnitt 
nur von der Gesamtladung $Ze$  und der Targetmasse $M$ 
abh\"angt. Insbesondere ergibt 
sich im Grenzfall $k_\mu=(\omega,\vec{k})\to 0$ die klassische
Thomson-Streuung
\be
\label{thomson}
 \lim_{\omega \to 0} \frac{d\sigma}{d\Omega} =
  \frac{Z^2e^2}{4\pi M^2} (\vec{\epsilon}_1 \cdot\vec{\epsilon}_2)^2 ,
\ee     
wobei $\vec{\epsilon}_1$ und $\vec{\epsilon}_2$ die 
Polarisationsvektoren der ein- und auslaufenden Photonen
bezeichnen. Low, Gell-Mann und Goldberger \cite{Low54,Low58,GMG54}
konnten dar\"uber hinaus zeigen, da\ss\ auf Grund von 
Eich- und Lorentzinvarianz die Amplitude f\"ur Comptonstreuung 
in Vorw\"artsrichtung an einem Spin-1/2 Target mit der Ladung $e$ sogar 
bis auf Terme linear in der Laborenergie $\omega$ bestimmt ist
\be
 \lim_{\omega \to 0}  T(\omega) =- \frac{e^2}{M}  
  (\vec{\epsilon}_1 \cdot\vec{\epsilon}_2) -i \frac{e^2}{2 M^2}
  \kappa^2 \omega  (\vec{\epsilon}_1 \times\vec{\epsilon}_2)
  \cdot \vec{\sigma} \; .
\ee
Der erste Term beschreibt  die Thomsonamplitude, w\"ahrend der
zweite Term eine Korrektur liefert, die proportional zum
anomalen magnetischen Moment $\kappa$ des Streuzentrums ist.

Die Anwendung von Niederenergietheoremen auf pionische 
Reaktionen wird durch die endliche Masse des Pions 
erschwert. Die Pionmasse setzt eine untere Grenze f\"ur 
die Energie  $\omega_\pi=(\vec{q}^{\, 2} +m_\pi^2)^{1/2}$
des Pions. W\"ahrend daher die Wellenl\"ange beliebig gro\ss\
gemacht werden kann, existiert eine prinzipielle Schranke 
f\"ur die Energie. Die Anwendbarkeit von Niederenergietheoremen
setzt daher voraus, da\ss\ die Masse des Pions klein gegen die 
charakteristische Energieskala der Reaktion ist. In hadronischen 
Prozessen ist eine solche Skala durch die Masse des Nukeons gegeben. 
Das Verh\"altnis $m_\pi/M\simeq 1/7$ ist daher ein nat\"urlicher
Parameter, der die Abweichung der  Amplitude vom unphysikalischen
Grenzfall $q_\mu \to 0$ kontrolliert.

Formal basieren Niederenergietheoreme f\"ur ''weiche`` Pionen auf
der G\"ultigkeit von Strom\-al\-ge\-bra, chiraler Symmetrie und
PCAC. Historisch wurden diese Konzepte in den sechziger Jahren 
als Hypothesen \"uber das Transformationsverhalten der in
hadronische Reaktionen eingehenden Str\"ome entwickelt
\cite{AD68,AFF73}. Sie erwiesen sich als au\ss erordentlich
fruchtbar, um die  experimentellen 
Informationen \"uber hadronische Reaktionen zu verstehen. 
Die klassischen Anwendungen liegen im Bereich
der $\pi\pi$- und $\pi N$-Streuung, sowie der Photo- und schwachen
Produktion pseudoskalarer Mesonen. Wichtige Vorhersagen ergeben
sich dar\"uber hinaus f\"ur leptonische und semileptonische Zerf\"alle
stark wechselwirkender Teilchen.    
  
Nach der Gr\"underphase trat die Anwendung von Stromalgebra und
chiraler Symmetrie zun\"achst in den Hintergrund gegen\"uber
der Entwicklung der Quantenchromodynamik als fundamentaler
Eichtheorie der starken Wechselwirkung. In diesem Zusammenhang 
zeigte sich allerdings, da\ss\ die G\"ultigkeit der erw\"ahnten Konzepte 
eine direkte Konsequenz der QCD ist. Dar\"uber hinaus liefert die chirale 
Symmetrie  einen wichtigen Zugang zur starken Wechselwirkung
in einem Bereich, in dem die direkte Anwendung der QCD
bislang noch gro\ss en Schwierigkeiten gegen\"ubersteht.

\section[Quantenchromodynamik, Stromalgebra \ldots]{Quantenchromodynamik,
Stromalgebra und chirale Symmetrie}
Die Quantenchromodynamik ist eine nichtabelsche Eichtheorie, 
beschrieben durch die Lagrangedichte
\be
\label{lqcd}
{\cal L}_{QCD} = -\frac{1}{4} G_{\mu\nu}^{\;\;a} G^{\mu\nu\, a}
 + \sum_{j=1}^{n_f} \bar{\psi}^{\alpha}_{j}( i\gamma^{\mu}
 {\cal D}_\mu^{\alpha\beta} - \delta^{\alpha\beta} m_j )
 \psi_j^\beta \; ,
\ee 
wobei die  Summation \"uber $n_f$ verschiedene Quarkarten (flavors)
ausgef\"uhrt wird. W\"ahrend sich die Quarkspinoren $\psi^{\alpha}$ nach
der fundamentalen Darstellung der Eichgruppe $SU(3)$ transformieren,
sind die Gluonen $A_\mu^{a}$ Vektorfelder und tragen einen Index 
in der adjungierten Darstellung der $SU(3)$. Die zugeh\"origen
Elemente der Lie-Algebra sind
\be
 A_\mu(x) = A_\mu^{a}(x)\frac{\lambda^{a}}{2}\; ,
\ee
wobei $\lambda^{a}$ die Generatoren der Algebra bezeichnet.
Sie erf\"ullen die fundamentalen Vertauschungsrelationen
\be
 [\lambda^{a},\lambda^b] = 2if^{abc}\lambda^c
\ee
und k\"onnen durch
\be
 Tr \,\lambda^{a}\lambda^b = 2\delta^{ab}
\ee
normiert werden. Dabei bezeichnet $f^{abc}$ die Strukturkonstanten
von $SU(3)$. Der Yang-Mills-Feldst\"arketensor ist durch
\be
\label{fmunu}
 G_{\mu\nu}^{\;\; a} = \partial_\mu A_\nu^{a} -\partial_\nu A_\mu^{a} 
 + g f^{abc} A_\mu^b A_\nu^c
\ee
gegeben, w\"ahrend die kovariante Ableitung durch
\be
\label{kovd}
 {\cal D}_\mu^{\alpha\beta} = \delta^{\alpha\beta}\partial_\mu
  + i\frac{g}{2} (\lambda^{a})^{\alpha\beta} A_\mu^{a}
\ee
definiert ist. Entscheidend f\"ur die Eichinvarianz der Theorie
ist die Tatsache, da\ss\ die Quark-Gluon-Wechselwirkung durch dieselbe
Kopplungskonstante $g$ wie die gluonische Selbstwechselwirkung 
bestimmt ist.
   
Neben der lokalen $SU(3)$ Eichsymmetrie besitzt die $QCD$-Lagrangedichte 
noch eine Reihe kontinuierlicher globaler Symmetrien.
So ist ${\cal L}_{QCD}$ invariant unter der globalen 
$U(1)_V$-Transformation
\be
\label{uone}
\psi_j(x) \to \exp (-i\theta) \psi_j (x) \, .
\ee
Der dazugeh\"orige erhaltene Vektorstrom ist der Baryonenstrom
\be
 j_\mu(x) = \sum_{j=1}^{n_f} \bar{\psi}_j \gamma_\mu \psi_j
\ee
mit der baryonischen Ladung 
\be
B=\int d^3x\, j_0(\vec{x},t)\, .
\ee
F\"ur masselose Quarks ist ${\cal L}_{QCD}$ ebenfalls invariant
unter axialen $U(1)_A$-Transformationen
\be
\label{uaone}
\psi_j(x) \to \exp (-i\theta\gamma_5) \psi_j (x) \; .
\ee
Der entsprechende Strom besitzt allerdings eine anomale Divergenz
\be
\label{axanom}
\partial^\mu j_{\mu\, 5}(x) = \frac{g^2}{4\pi}\frac{n_f}{8}
 \epsilon^{\mu\nu\rho\sigma} G_{\mu\nu}^{\;\; a}G_{\rho\sigma}^{\;\; a}
 \; ,
\ee
so da\ss\  die axiale Ladung $Q_5=\int d^3x\, j_{0\,5}(\vec{x},t)$
nur in  Abwesenheit instantonartiger L\"osungen erhalten ist.

Vernachl\"assigt man die Quarkmassen, so ist ${\cal L}_{QCD}$ dar\"uber 
hinaus invariant unter Skalentransformationen. Diese Symmetrie wird in 
der quantisierten Theorie durch die Notwendigkeit der Renormierung
gebrochen. Dabei tritt ein dimensionsbehafteter Parameter, der
QCD-Skalenparameter $\Lambda_{\mini QCD}$, auf. Sein Wert ist unter anderem 
aus der Skalenbrechung in der tief-inelastischen Lepton-Nukleon-Streuung zu
$\Lambda_{\mini QCD}^{\mini\overline{MS}} =230\pm 80$ MeV bestimmt worden
\cite{PDG90}. Der Index ${\kl \overline{MS}}$ bezeichnet eine spezielle
Renormierungsvorschrift, die sogenannte modifizierte minimale 
Subtraktion.

Ebenfalls auf Grund der Renormierung sind  auch die Werte der 
Quarkmassen von der experimentellen Skala abh\"angig.
Im Bereich typischer hadronischer Prozesse lassen sich 
die Stromquarkmassen mit Hilfe von QCD-Summenregeln 
extrahieren. Bei $\mu^2=1\,{\rm GeV}^2$ findet man
\cite{GL82,DR87}
\beq
   m_u &=& 5.1 \pm 1.5 \;{\rm MeV}, \nonumber  \\
   m_d &=& 8.9 \pm 2.6 \;{\rm MeV}, \\
   m_s &=& 175 \pm 55 \;{\rm MeV}.  \nonumber
\eeq   
Alle anderen bekannten Flavors haben Massen \"uber einem GeV. Die
up- und down-Quarks sind au\ss erordentlich leicht verglichen mit dem
QCD-Skalenparameter, w\"ahrend das seltsame Quark eine Zwischenstellung
einnimmt.   

Wir wollen daher im folgenden die drei leichten Quarks zun\"achst als
masselos betrachten. In diesem Fall besitzt ${\cal L}_{QCD}$
eine chirale $SU(3)_L \times SU(3)_R$ Symmetrie im Raum der $u$-,
$d$- und $s$-Flavors
\beq
\label{suv}
\psi_i(x) &\to& \exp (-i\theta^{a}\lambda^{a})_{ij} \psi_j (x)\, ,\\
\label{sua}
\psi_i(x) &\to& \exp (-i\phi^{a}\lambda^{a}\gamma_5)_{ij} \psi_j (x)\, ,
\eeq
wobei die $SU(3)$ Matrizen $\lambda^{a}$ auf den Flavorindex der
Quarks wirken. Die zugeh\"origen N\"otherstr\"ome sind die
Vektor- und Axialvektorstr\"ome
\beq
   V_\mu^{a}(x) &=& \bar{\psi}_i \gamma_\mu \frac{\lambda^{a}_{ij}}{2}
     \psi_j \, , \\
   A_\mu^{a}(x) &=& \bar{\psi}_i \gamma_\mu \gamma_5
   \frac{\lambda^{a}_{ij}}{2}   \psi_j  \, , 
\eeq
deren Zeitkomponenten auf die erhaltenen Ladungen 
\beq
 Q^{a}(t)   &=& \int d^3x \, V_0^{a}(\vec{x},t) \, , \\
 Q^{a}_5(t) &=& \int d^3x \, A_0^{a}(\vec{x},t) 
\eeq
f\"uhren. Die Struktur der zugeh\"origen Lie-Algebra erkennt man 
am einfachsten, indem man zu den 
Linearkombinationen $Q^{a}_{L,R}=\frac{1}{2}(Q^{a}\mp Q^{a}_5)$ \"ubergeht.
Sie erf\"ullen die Vertauschungsrelationen
\beq
\label{chalg}
\,[Q_{L}^{a}(t),Q_{L}^{b}(t)] &=& i f^{abc} Q_{L}^{c}(t)\, ,  \\
\,[Q_{R}^{a}(t),Q_{R}^{b}(t)] &=& i f^{abc} Q_{R}^{c}(t)\, ,  \\
\,[Q_{L}^{a},Q_{R}^{b}]     &=& 0\, ,
\eeq
charakteristisch f\"ur die Lie-Algebra $SU(3)_L \times SU(3)_R$.  
Das Transformationsverhalten der Str\"ome unter der chiralen
$SU(3)_L\times SU(3)_R$ ergibt sich aus den kanonischen
Vertauschungsregeln f\"ur die Quarkfelder
\beq
\label{curalg}
\,[ Q^{a}(t),V_\mu^b (\vec{x},t)] &=&
               \!\!\spm i f^{abc}V_\mu^{c}(\vec{x},t)\, ,\\
\,[ Q^{a}(t),A_\mu^b (\vec{x},t)] &=&
               \!\!\spm i f^{abc}A_\mu^{c}(\vec{x},t)\, ,\\  
\,[ Q^{a}_5(t),V_\mu^b (\vec{x},t)] &=&
               \!\! -i f^{abc}A_\mu^{c}(\vec{x},t)\, ,\\ 
\,[ Q^{a}_5(t),A_\mu^b (\vec{x},t)] &=&
               \!\!\spm i f^{abc}V_\mu^{c}(\vec{x},t) \; .
\eeq 
Diese Relationen legen  die $SU(3)_L\times SU(3)_R$ Darstellung
fest, nach der sich die Str\"ome transformieren. Sie wurden von 
Gell-Mann auf Grund rein ph\"anomenologischer \"Uberlegungen 
postuliert \cite{AD68}. 

Soll die QCD im Grenzfall verschwindender Massen 
der leichten Quarks eine sinnvolle 
N\"aherung an die vollst\"andige Theorie darstellen, dann kann der 
QCD-Grundzustand nicht $SU(3)_L\times SU(3)_R$ symmetrisch sein. 
W\"are n\"amlich
\be
\label{symvac}
 Q_L^{a}|0\rangle  = Q_R^{a}|0\rangle  = 0 \; ,
\ee
dann sollten auch die Zweipunktfunktionen der Vektor- und Axialvektorstr\"ome
identisch sein
\be
\label{symspec}
\langle 0|T(A_\mu^{a}(x)A_\nu^{b}(0))|0\rangle  =\langle 0|T(V_\mu^{a}(x)V_\nu^{b}(0))|0\rangle  \; .
\ee  
Zum Beweis zerlegt man die beiden Str\"ome in ihre links- und rechtsh\"andigen
Komponenten 
\beq
 V_\mu^{a} &=& J_{\mu\, R}^{a} + J_{\mu\, L}^{a}\, , \\
 A_\mu^{a} &=& J_{\mu\, R}^{a} - J_{\mu\, L}^{a}
\eeq
und folgert aus (\ref{symvac}) und dem Transformationsverhalten der Str\"ome,
da\ss\  die nichtdiagonale Kombination $\langle 0|T(J_{\mu\, L}^{a}J_{\nu\,R}^{a})|0\rangle $
verschwindet.

Die spektralen Dichten zu den beiden Zweipunktfunktionen (\ref{symspec}) sind
experimentell zu\-g\"ang\-lich und zeigen eine sehr verschiedenartige Gestalt.
W\"ahrend der Vektorkanal vor allem durch die $\rho$-Resonanz bei 
$m_\rho=770$ MeV dominiert wird, ist die Axialvektorspektralfunktion in 
diesem Bereich klein und zeigt erst im Bereich der $a_1$-Resonanz bei
$m_{a_1}=1260$ MeV eine ausgepr\"agte Struktur. Wir folgern daraus, da\ss\ 
der QCD Grundzustand nicht $\chs$ symmetrisch ist. Dieses Ph\"anomen, 
da\ss\  der Grundzustand der Theorie
nicht die volle Symmetrie der Lagrangedichte besitzt, bezeichnet man
als spontane Symmetriebrechung.

Nach dem Goldstone-Theorem impliziert die spontane Brechung einer 
kontinuierlichen Symmetrie
das Auftreten masseloser Bosonen. Ist $Q^{a}$ ein  $\chs$ Generator,
der das Vakuum nicht invariant l\"a\ss t, dann gibt es einen physikalischen
Zustand $Q^{a}|0\rangle $, der mit dem Vakuum energetisch entartet ist.  
Handelt es sich bei $Q^{a}$ um eine Vektorladung, dann beschreibt
$Q^{a}|0\rangle $ ein masseloses skalares Teilchen. Ist $Q^{a}$ dagegen eine
axiale Ladung, so fordert das Goldstone-Theorem das Auftreten masseloser
pseudoskalarer Bosonen.

In der Natur sind die acht leichtesten Hadronen $(\pi,K,\eta)$ pseudoskalar.
Dagegen sind die leichtesten skalaren Teilchen schwerer als das Nukleon.
Die chirale Symmetrie ist daher in der Form
\beq
   Q_V^{a}|0\rangle  &=& 0 \, , \\
   Q_A^{a}|0\rangle  &\neq& 0
\eeq
realisiert. Die Vektorsymmetrie bleibt erhalten, so da\ss\  sich Hadronen
nach irreduziblen Darstellungen von $SU(3)_V$ klassifizieren lassen. Im
$SU(2)$-Sektor der Theorie folgt daraus die Isospinsymmetrie der 
starken Wechselwirkung.

Endliche Quarkmassen brechen die chirale Symmetrie explizit. Die 
Divergenz der Vektor- und Axialvektorstr\"ome lautet 
\beq
\label{divv}
\partial^\mu V_\mu^{a} &=& \frac{i}{2} 
               \bar{\psi}\left[M,\lambda^{a}\right]\psi\, , \\ 
\label{diva}
\partial^\mu A_\mu^{a} &=& \frac{i}{2} 
               \bar{\psi}\left\{M,\lambda^{a}\right\}\psi \, ,
\eeq
wobei $M={\rm diag}(m_u,m_d,m_s)$ die Massenmatrix f\"ur die drei 
leichten Flavors bezeichnet. Die rechte Seite von (\ref{divv},\ref{diva})
l\"a\ss t sich besonders \"ubersichtlich darstellen, indem man $M$ nach 
Gell-Mann-Matrizen entwickelt, $M=\epsilon_0\lambda^0+\epsilon_3\lambda^3+
\epsilon_8\lambda^8$. Dabei ist
\beq
\epsilon_0 &=& \frac{1}{\sqrt{2}} (m_u+m_d+m_s)\, , \\
\epsilon_3 &=& \frac{1}{2} (m_u-m_d)\, ,  \\
\epsilon_8 &=& \frac{1}{2\sqrt{3}} (m_u+m_d-2m_s)\, .
\eeq
Im Fall entarteter Quarkmassen $m_u=m_d=m_s$ ist $\epsilon_3=\epsilon_8=0$,
und die $SU(3)_V$ Flavor-Symmetrie bleibt ungebrochen.
 
Im Hadronenspektrum manifestieren sich die nichtverschwindenden 
Quarkmassen in einer endlichen Masse f\"ur die Goldstonebosonen 
$(\pi,K,\eta)$. Der Zusammenhang dieser Massen mit denen der Quarks
ist durch die Gell-Mann, Oakes, Renner (GOR)-Relation gegeben
\cite{GOR68}. Um die bei der Herleitung von Niederenergietheoremen
verwendeten Methoden vorzustellen, wollen wir im folgenden einen
kurzen Beweis der GOR-Relation vorstellen. Zu diesem Zweck definieren
wir die Pionzerfallskonstante durch das Matrixelement
\be
\label{fpi}
 \langle 0|A_\mu^{a}(x)|\pi^{b}(q)\rangle  = 
 i\delta^{ab} f_\pi q_\mu e^{-iq\cdot x} .
\ee
Diese \"Ubergangsmatrix kontrolliert den schwachen Zerfall der
geladenen Pionen $\pi^\pm \to \mu^\pm \nu_\mu$. F\"ur das
Matrixelement der Divergenz des Axialstroms findet man
\be
 \langle 0|\partial^\mu A_\mu^{a}(x)|\pi^{b}(q)\rangle  = 
         \delta^{ab} f_\pi m_\pi^2 e^{-iq\cdot x}\; ,
\ee	     
so da\ss\ sich ein interpolierendes Feld f\"ur das Pion durch
die Beziehung
\be
\label{PCAC}
\partial^\mu A_\mu^{a}(x) = f_\pi m_\pi^2 \phi^{a} (x)
\ee
definieren l\"a\ss t \cite{Col67}. Diese Gleichung wird als PCAC (partially
conserved axial current) Relation bezeichnet. Sie ist 
\"aquivalent zur Divergenzbeziehung (\ref{diva}) und dr\"uckt wie
diese die Tatsache aus, da\ss\  der Axialvektorstrom im chiralen 
Limes $m_\pi \to 0$ erhalten ist.

Wir wollen im folgenden Wardidentit\"aten f\"ur die Zweipunktfunktionen
\beq
\label{axtwop}
\Pi_{5\, \mu\nu}^{ab}(q) &=& i\int d^4x\, e^{iq\cdot x}
          \langle 0|T(A_\mu^{a}(x)A_\nu^{b}(0))|0\rangle  \; ,   \\
\label{divtwop}	  
\psi_{5}^{ab} (q) &=& i\int d^4x\, e^{iq\cdot x}
          \langle 0|T(\partial^\mu A_\mu^{a}(x)\partial^\nu A_\nu^{b}(0))|0\rangle 
\eeq
konstruieren. Zweimaliges Differenzieren des zeitgeordneten Produkts 
(\ref{axtwop}) liefert die Beziehung
\beq
\label{wi}
q^\mu q^\nu  \Pi^{ab}_{5\,\mu\nu} (q) &=& \psi_5^{ab}(q)
   -q^\nu \int d^4x\, e^{iq\cdot x} 
   \delta (x^0) \langle 0|[A_0^{a}(x),A_\nu^{b}(0)]|0\rangle  \\
   & & \mbox{} -i\int d^4x\,  e^{iq\cdot x} 
   \delta (x^0) \langle 0|[A_0^{a}(x),\partial^\mu A_\mu^{b}(0)]|0\rangle \, , \nonumber
\eeq
die sich im Limes $q_\mu \to 0$ auf die Gleichung
\be
\label{psi0}
  \psi_5^{ab}(0) = -i \int d^4x \, \delta(x^0) 
        \langle 0|[A_0^{a}(x),\partial^\mu A_\mu^{b}(0)]|0\rangle     	   	       
\ee	
reduziert. Die linke Seite dieser Gleichung folgt aus der PCAC Relation, 
w\"ahrend man die rechte Seite mit Hilfe von (\ref{diva}) und
den kanonischen Vertauschungsregeln f\"ur die Quarkfelder 
bestimmen kann. Das Resultat ergibt schlie\ss lich die GOR-Relation
\be
\label{GOR}
 -f_\pi^2 m_\pi^2 = m_u\langle 0|\bar{u}u|0\rangle + 
 m_d \langle 0|\bar{d}d|0\rangle \; .
\ee
Sie verbindet die hadronischen Parameter $m_\pi$ und $f_\pi$ mit
den Quarkmassen und dem Ordnungsparameter f\"ur die spontane 
Brechung der chiralen Symmetrie, dem Quarkkondensat $\langle 0|\bar uu
+\bar dd|0\rangle $. Wir haben in der Herleitung von dem Grenz\"ubergang 
$q_\mu \to 0$ Gebrauch gemacht. Die GOR-Relation ergibt daher 
eine Aussage \"uber die Pionzerfallskonstante im chiralen Limes.
F\"ur den physikalischen Wert von $f_\pi$ liefert die
GOR-Relation nur den f\"uhrenden Term in einer Entwicklung
in Potenzen der Quarkmasse.     

%Die Wardidentit\"at (\ref{wi}) gilt unabh\"angig von den 
%physikalischen Zust\"anden, zwischen denen die Matrixelemente
%genommen sind. Ersetzt man die Vakuumzust\"ande durch ein und
%auslaufende Nukleonen, so l\"a\ss t sich die Divergenzamplitude 
%$\psi_5^{ab}$  mit der Pion-Nukleon-Streumatrix
%$T_{\pi N}^{ab}$ identifizieren. Auf diese
%Weise k\"onnen wir die Rolle der expliziten Symmetriebrechung
%in physikalischen Streuprozessen studieren. Durch zweimaliges
%Differenzieren der Zweipunktfunktion $\Pi^{ab}_{5\,\mu\nu}$ ergibt sich 
%folgende Wardidentit\"at f\"ur $T_{\pi N}^{ab}$ \cite{BPP71}:
%\beq
%\label{tpin}
% T_{\pi N}^{ab}(p_1,q_1;p_2,q_2) &=& T_{PV}^{ab}(p_1,q_1;p_2,q_2)
%   +\frac{q_1^2+q_2^2-m_\pi^2}{f_\pi^2 m_\pi^2} \Sigma^{ab}(p_1,p_2) \\
%   & &\mbox{} +\frac{1}{2f_\pi^2} (q_1+q_2)^\mu V_\mu^{ab}(p_1,p_2)
%   +q_1^\mu q_2^\nu R^{ab}_{\mu\nu}(p_1,q_1;p_2,q_2)\, . \nonumber
%\eeq
%Dabei haben wir $q_1^\mu q_2^\nu\Pi^{ab}_{5\,\mu\nu}$ in die 
%Beitr\"age der Pseudovektor-Bornterme $T_{PV}^{ab}$ und die 
%Hintergrundamplitude $q_1^\mu q_2^\nu R_{\mu\nu}^{ab}$ zerlegt.
%Sie enth\"alt weder Nukleon- noch Pionpole
%und verschwindet daher im Niederenergielimes $q_1,q_2 \to 0$.
%Des weiteren bezeichnet $V_\mu^{ab}(p_1,p_2)$ den Stromalgebraterm
%\be
% V_\mu^{ab}(p_1,p_2) = i\epsilon^{abc}\langle N(p_2)|V_\mu^c(0)|N(p_1)\rangle \, .
%\ee
%Unser spezielles Augenmerk liegt auf dem Pion-Nukleon Sigmaterm
%$\Sigma^{ab}(p_1,p_2)$, der ein Ma\ss\ ist f\"ur die St\"arke der 
%expliziten Symmetriebrechung. Wie bei der Herleitung der 
%GOR-Relation ergibt sich
%\be
%\label{pinsig}
%\Sigma^{ab}(p_1,p_2) = \frac{\delta^{ab}}{2}(m_u+m_d)
%    \langle N(p_2)|\bar{u}u+\bar{d}d|N(p_1)\rangle \; .
%\ee
%Der Sigmaterm liefert an der Schwelle den f\"uhrenden Beitrag zum
%isospinsymmetrischen Teil der Streuamplitude. Dar\"uber hinaus kann
%man den  Wert von
%$\Sigma^{ab}(p_1,p_2)$ am unphysikalischen Punkt $p_1=p_2$
%als Beitrag der Strommassen der leichten Quarks zur Nukleonmasse
%interpretieren. Diese Gr\"o\ss e l\"a\ss t sich prinzipiell aus dem
%beobachteten Baryonspektrum ermitteln. In chiraler St\"orungstheorie
%findet man \cite{GL80}
%\be
% \sigma =\frac{1}{2}(m_u+m_d)\langle p|\bar{u}u+\bar{d}d|p\rangle 
%    = \frac{35\pm 5}{1+y}\; {\rm MeV}\, ,              
%\ee
%wobei $y=\langle p|\bar{s}s|p\rangle /\langle p|\bar{u}u|p\rangle $ das Verh\"altnis des
%Kondensats der seltsamen Quarks zu dem  der leichten Quarks im
%Nukleon bezeichnet. Um den Wert von $\sigma$ aus
%Pion-Nukleon Streuphasen zu extrahieren, ist ein aufwendiges
%Extrapolationsverfahren notwendig. Im Gegensatz zu fr\"uheren
%Auswertungen liefern neuere Dispersionsanalysen einen 
%Wert $\sigma=45 \pm 5$ MeV \cite{GLS91}, der vertr\"aglich
%ist mit der Annahme eines verschwindenden Kondensats seltsamer
%Quarks im Nukleon.

\section{Ableitung des Niederenergietheorems}
Die Herleitung von Niederenergietheoremen zur Pionphotoproduktion
verl\"auft analog zu der im letzten Abschnitt geschilderten Ableitung 
der GOR-Relation. Ausgangspunkt ist die Darstellung der Streumatrix mit Hilfe
der LSZ-Reduktionsformel
\beq
\label{LSZ}
 S^{a} &=& -(2\pi)^4 \,\delta^4 (p_1+k-p_2-q)\, Z_\gamma^{-1/2}
   Z_\pi^{-1/2} \\
   & & \mbox{}\cdot \int d^4x\, e^{iq\cdot x} (\Box +m_\pi^2)
   \langle N(p_2)|T\left(\epsilon^\mu V_\mu^{em}(0) \phi^{a}(x)\right)|N(p_1)\rangle 
   \nonumber\; .
\eeq
Dabei bezeichnet $\phi^{a}$ das kanonische Pionfeld, $Z_\pi=(2\pi)^3
2\omega_\pi$ dessen kovariante Normierung und $Z_\gamma=(2\pi)^3
2\omega$ die Normierung des elektromagnetischen Feldes. Die
\"Ubergangsmatrix nach der Definition aus dem ersten Kapitel ist
durch
\be
\label{deft}
 S^{a} = i(2\pi)^4\,\delta^4 (p_1+k-p_2-q) Z_\gamma^{-1/2}
  Z_\pi^{-1/2} \epsilon^\mu T_\mu^{a}
\ee
gegeben. Wir betrachten  die Zweipunktfunktion
\be
\label{Pimunu}
\overline{\Pi}^a_{\mu\nu}(q) = \int d^4 x\, e^{iq\cdot x}\langle N(p_2)| 
T\left( V_\mu^{em} (0) B_\nu^{a}(x) \right) |N(p_1)\rangle  \; .
\ee
des elektromagnetischen Stroms $V_\mu^{em}$ und des 'transversalen` Axialstroms
\be
B_\mu^{a}(x) =A_\mu^{a}(x)+\frac{1}{m_\pi^2}\partial_\mu D^{a}(x)\; ,
\ee
wobei $D^{a}(x)=\partial^\mu A_\mu^{a}(x)$ die Divergenz des Axialstroms
bezeichnet. Mit Hilfe der PCAC-Relation und der Definition der
Pionquellfunktion findet man
\be
\label{defb}
\partial^\mu B_\mu^a (x) = f_\pi (\Box +m_\pi^2)\phi^{a}(x) =
     -f_\pi j_\pi^{a}(x)\, .
\ee
Einmaliges Differenzieren des zeitgeordneten Produktes in der 
Zweipunktfunktion $\overline{\Pi}_{\mu\nu}^{a}$ liefert schlie\ss lich 
die gesuchte Wardidentit\"at f\"ur $T_\mu^{a}$ \cite{RST76}
\be
\label{avward}
T_\mu^a (q) = \frac{1}{f_\pi}\left\{
q^\nu \overline{\Pi}_{\mu\nu}^a (q) \, - \,i C_\mu^a (q)  \, - \,
\frac{\omega_\pi}{m_\pi^2} \Sigma^a_\mu (q) \right\} \; .
\ee
Dabei haben wir die LSZ-Formel (\ref{LSZ}) verwendet, um die 
Photoproduktionsamplitude $T_\mu^{a}$ zu identifizieren. Die Wirkung 
der Ableitung auf den Zeitordnungsoperator liefert die Kommutatoren
\beq
\label{cmua}
 C_\mu^{a}(q) &=& \int d^4x\, e^{iq\cdot x}\delta (x^0)
   \langle N(p_2)|[A^{a}_0(x),V_\mu^{em}(x)]|N(p_1)\rangle \; , \\
\label{sig}     
 \Sigma_\mu^{a}(q) &=& \int d^4x\, e^{iq\cdot x}\delta (x^0)
   \langle N(p_2)|[D^{a}(x),V_\mu^{em}(x)]|N(p_1)\rangle \; ,
\eeq
wobei wir das Resultat in den Stromalgebraterm $C_\mu^{a}$
und den Kommutator der Divergenz des Axialstroms zerlegt haben.  
In Analogie zum Sigmaterm in der Pion-Nukleon-Streuung bezeichnet
man diesen Beitrag auch als $(\gamma,\pi)$-Sigmaterm. Wie der
$\pi N$-Sigmaterm liefert er eine zus\"atzliche Korrektur, die direkt
proportional zu den Stromquarkmassen in der QCD-Lagrangedichte ist.

Es ist instruktiv, die Wardidentit\"at zu studieren, 
die sich aus der Zweipunktfunktion $\Pi^{a}_{\mu\nu}$ des
\"ublichen Axialstroms ergibt. Analog zu (\ref{avward})
erh\"alt man
\be
\label{avward2}
-\frac{m_\pi^2}{q^2-m_\pi^2} T_\mu^a (q) = \frac{1}{f_\pi}\Big\{
q^\nu \Pi_{\mu\nu}^a (q) \, - \, iC_\mu^a \Big\} \; .
\ee
Auf Grund des Pionpropagators vor der Amplitude $T_\mu^{a}$
liefert diese Beziehung die Photoproduktionsamplitude zun\"achst
nur am unphysikalischen ''weichen`` Punkt $q_\mu=0$.  Um die physikalische 
Schwelle $q^2=m_\pi^2$ zu erreichen, ist es notwendig, den Pionpol
explizit abzuseparieren. Diesem Zweck dient der transversale Axialstrom
$B_\mu^{a}$. Tats\"achlich l\"a\ss t sich $B_\mu^{a}$ mit Hilfe der
PCAC-Relation als der nichtpionische Anteil des Axialstroms interpretieren
\be
 B_\mu^{a}(x) = A_\mu^{a}(x) +f_\pi\partial_\mu \phi^{a}(x)
    = A_\mu^{a}(x) - A_{\mu}^{a\, (\pi)} (x) \, .
\ee
Wir wollen nun die verschiedenen Beitr\"age zur Photoproduktionsamplitude
$T_\mu^{a}$ im einzelnen studieren. Beginnen werden wir dabei mit
der Zweipunktfunktion $q^\nu\overline{\Pi}_{\mu\nu}^{a}$. Da dieser
Term proportional zum Impuls $q$ ist, tragen im Grenzfall weicher Pionen
nur die Polterme in $\overline{\Pi}_{\mu\nu}^{a}$ zur Amplitude bei.
Die Summe der Nukleonpolterme im direkten und im Austauschkanal
lautet
\beq
\label{nborn}
f_\pi T_\mu^{a\,(N)} &=& \bar{u}(p_2) \Big( q^\nu \Gamma_\nu^{B^{a}}
   (p_1-k,-q,p_2) S_F(p_1+k) \Gamma_\mu^\gamma (p_1,k,p_1+k)
      \\[0.2cm]
   & & \hspace{0.5cm} \mbox{}+ \Gamma_\mu^\gamma (p_1-q,k,p_2)
   S_F(p_1-q) q^\nu \Gamma_\nu^{B^{a}}(p_1,-q,p_1-q) \Big) u(p_1)
   \; .\nonumber
\eeq      
Dabei bezeichnet $S_F(p)$ den Nukleonpropagator und $\Gamma_\mu^\gamma$
bzw. $\Gamma_\nu^{B^{a}}$ die Vertexfunktionen f\"ur die Kopplung
des Nukleons an das elektromagnetische Feld und den Axialstrom $B_\nu^{a}$.
In den Poltermen sind alle Teilchen mit Ausnahme des
intermedi\"aren Nukleons auf der Massenschale. Die allgemeine Gestalt der
Vertexfunktion lautet in diesem Fall \cite{NK87}
\beq
\label{emvert}
i\Gamma_\mu^\gamma (p_1,k,p_1+k)u(p_1) &=& \left( e\gamma_\mu F_1 
   +  \frac{M+(p_1+k)\cdot\gamma}{2M} \frac{ie\sigma_{\mu\nu} k^\nu}{2M}F_2^+ 
   \right. \\
 & & \hspace{1.5cm}\left. \mbox{}
   +  \frac{M-(p_1+k)\cdot\gamma}{2M} \frac{ie\sigma_{\mu\nu} k^\nu}{2M}F_2^-
   \right) u(p_1),  \nonumber 
\eeq
\newpage
\beq   
\label{bavert}
i\Gamma_\nu^{B^{a}} (p_1,-q,p_1-q)u(p_1) &=& \left( \gamma_\nu \overline{G}_A 
   +  \frac{M+(p_1-q)\cdot\gamma}{2M} \frac{q_\nu}{2M} \overline{G}_P^{\, +}
   \right. \\
  & & \hspace{1.5cm}\left. \mbox{}  
   +  \frac{M-(p_1-q)\cdot\gamma}{2M} \frac{q_\nu}{2M} \overline{G}_P^{\, -} 
   \right)\gamma_5\frac{\tau^{a}}{2} u(p_1) \; .\nonumber
\eeq     
Analoge Ausdr\"ucke ergeben sich f\"ur den Vertex des auslaufenden 
Nukleons. Die Formfaktoren $F_i$ sind Funktionen
des Impuls\"ubertrages $k^2$ und des off-shell-Parameters $\delta^2=(p_1+k)^2
-M^2$. Ihre Isospinstruktur lautet
\be
 F_i=F_i^{s}+F_i^{v}\tau^3 \; .
\ee
Entsprechend h\"angen die Formfaktoren $\overline G_A$ und 
$\overline G_P^{\,\pm}$ am $NNB^{a}$-Vertex von den 
Variablen $q^2$ und ${\delta '}^2=(p_1-q)^2-M^2$ ab. Wir haben diese 
Formfaktoren mit einem Querstrich gekennzeichnet, um sie von den 
entsprechenden Funktionen am  Vertex des Axialvektorstroms zu 
unterscheiden.

Die Abh\"angigkeit der Photoproduktionsamplitude vom off-shell-Verhalten 
der Formfaktoren wurde in einer Arbeit von Naus, Koch
und Friar \cite{NKF90} untersucht. Die Autoren zeigen, da\ss\ 
off-shell-Korrekturen erst in derselben Ordnung in $m_\pi$ 
wie andere mo\-dell\-ab\-h\"angige Korrekturen auftreten. Wir verwenden
daher im folgenden die on-shell Vertices
\beq
i\Gamma_\mu^\gamma &=& e\gamma_\mu F_1 + 
          \frac{ie\sigma_{\mu\nu}k^\nu}{2M} F_2\; , \\
i\Gamma_\nu^{B^{a}}&=& \left( \gamma_\nu \overline{G}_A
         + \frac{q_\mu}{2M}\overline{G}_P \right) \gamma_5 
	 \frac{\tau^{a}}{2}\; ,
\eeq
wobei die auftretenden Formfaktoren nur mehr Funktionen der
Impuls\"ubertr\"age $k^2$ bzw.~$q^2$ sind. F\"ur reelle
Photonen ist
\be
\begin{array}{rclcrcl}
  F_1^{s}(0)&=& 1/2,        &\hspace{1cm}& F_1^{v}(0)&=& 1/2,     \\[0.2cm]
  F_2^{s}(0)&=&\kappa^s,    &            & F_2^{v}(0)&=&\kappa^v,  
\end{array}
\ee
mit den anomalen magnetischen Momenten $\kappa^{s,v}=\frac{1}{2}
(\kappa_p\pm \kappa_n)$. Die Vertexfunktionen f\"ur die beiden 
Str\"ome $A_\nu^{a}$ und $B_\nu^{a}$ unterscheiden sich nur 
um die Matrixelemente des pionischen Beitrags 
$A_\nu^{a(\pi)}=-f_\pi \partial_\nu \phi^{a}$. 
Dieser Term erzeugt den Pionpol im induzierten 
pseudoskalaren Formfaktor
\be
  G_P^{\pi -Pol} (q^2)=\frac{4Mf_\pi}{m_\pi^2-q^2} G_{\pi NN}(q^2)\; ,
\ee
wobei $G_{\pi NN}$ den Pion-Nukleon Formfaktor
\be
  \langle N(p_2)|j_\pi^{a}(0)|N(p_1)\rangle  = G_{\pi NN}(t) 
  \bar{u}(p_2)i\gamma_5 \tau^{a}u(p_1)
\ee
bezeichnet. Der Zusammenhang der Formfaktoren an den Vertices ist 
daher durch $\overline{G}_A=G_A$ und $\overline{G}_P=G_P-G_P^{\pi -Pol}$ 
gegeben. Empirische Untersuchungen zeigen, da\ss\ $G_P$ in sehr guter 
N\"aherung durch den Polterm alleine beschrieben wird. Wir setzen daher 
$\overline{G}_P=0$ und erhalten
\beq
\label{nborn2}
f_\pi T_\mu^{a\,(N)} &=& \bar{u}(p_2) \left\{ g_A q\cdot\gamma \gamma_5 
 \,\frac{\tau^{a}}{2} \frac{i}{(p_1+k)\cdot\gamma -M} \,\Gamma_\mu^\gamma
 \right. \\
 & & \hspace{1cm}\left. \mbox{} + \Gamma_\mu^\gamma 
     \,\frac{i}{(p_1-q)\cdot\gamma -M}\,
  g_A q\cdot\gamma \gamma_5 \frac{\tau^{a}}{2} \right\} u(p_1), \nonumber 
\eeq
wobei wir dar\"uber hinaus den axialen Formfaktor $G_A(q^2)$ durch den
Wert bei $q^2=0$, $g_A=G_A(q^2=0)$, ersetzt haben. Die axiale
Kopplung l\"a\ss t sich mit Hilfe der Goldberger-Treiman-Relation
\be
\label{GT}
\frac{g_A}{2f_\pi} = \frac{f}{m_\pi}
\ee
durch die pseudovektorielle Pion-Nukleon-Kopplungskonstante $f$ ausdr\"ucken.
Das Resultat (\ref{nborn2}) entspricht daher der Born-Approximation f\"ur
eine effektive Pion-Nukleon Lagrangedichte mit dem Kopplungsterm
\be
\label{pv}
{\cal L} = \frac{f}{m_\pi} \bar{\psi}\gamma_5\gamma_\mu \tau^{a}\psi
   \partial^\mu \phi^{a}\; .
\ee    
Die geschilderte Herleitung f\"uhrt also in nat\"urlicher Weise
auf eine pseudovektorielle Kopplung des Pions an das Nukleon. Dies 
steht im Gegensatz zu vielen klassischen Arbeiten, in denen 
gew\"ohnlich mit einer pseudoskalaren Kopplung gerechnet wird.
Um Konsistenz mit der Wardidentit\"at (\ref{avward}) zu erzielen,
m\"ussen in diesem Fall zus\"atzliche Korrekturterme zur Bornamplitude
addiert werden.

Der Beitrag des Kommutators $C_\mu^{a}$ l\"a\ss t sich mit Hilfe 
der Stromalgebraregeln berechnen
\be
\label{curcom}
 C_\mu^{a} = -i\epsilon^{a3c} \langle N(p_2)|A_\mu^{c}(0)|N(p_1)\rangle \; .
\ee
Vernachl\"assigt man den Hintergrundbeitrag im induzierten 
pseudoskalaren Formfaktor, so ergibt sich
\be
\label{kr}
\epsilon^\mu C_\mu^{a} = -i\epsilon^{a3c} \bar{u}(p_2)
  \left\{ G_A(t)\epsilon\cdot\gamma + 2f_\pi G_{\pi NN}(t)   
   \frac{\epsilon\cdot (k-q)}{m_\pi^2-t} \right\}
   \gamma_5 \frac{\tau^{c}}{2}u(p_1) 
\ee   	 	  
als Funktion der Mandelstamvariable $t=(q-k)^2$.
Das Resultat ist antisymmetrisch in den Isospinindices und
tr\"agt daher nur zur Produktion geladener Pionen bei. 
Der erste Term liefert den f\"uhrenden Beitrag zur 
Pho\-to\-pro\-duk\-ti\-ons\-amplitude im Grenzfall weicher Pionen
\be
\label{krtheo}
\lim_{q,k\to 0} T^{a}(q) =\frac{eg_A}{f_\pi}\epsilon^{a3c}
   \bar{u}(p_2) \epsilon\cdot\gamma\gamma_5
   \frac{\tau^{c}}{2}u(p_1)\; ,
\ee
den sogenannten Kroll-Ruderman-Term \cite{KR54}. 
Der zweite Kommutator enth\"alt die Divergenz des Axialstroms
und l\"a\ss t sich daher mit Ausnahme der Zeitkomponente 
$\Sigma_0^{a}$ nicht modellunabh\"angig bestimmen. Mit Hilfe
des Stromalgebraresultats
\be
 \,[Q_5^{a},V_\mu^{em}(0)]=-i\epsilon^{a3c}A_\mu^{c}(0)
\ee
und der Erhaltung des elektromagnetischen Stroms, 
$\partial^\mu V_\mu^{em}=0$, findet man
\be
\label{sig0}
  \int d^4x \,\delta (x^0)\, [\partial^\mu A_\mu^{a}(x),V_0^{em}
  (0)] = -i\epsilon^{a3c} \partial^\mu A_\mu^{c} (0) \; .
\ee     
Das Matrixelement der Divergenz des Axialstroms zwischen
Nukleonzust\"anden $|N(p)\rangle $ ist durch die axialen Formfaktoren des
Nukleons bestimmt:
\be
\langle N(p_2)| D^{a}(x) |N(p_1)\rangle  = \bar{u}(p_2) \left[ M G_A (t)
  + \frac{t}{4M} G_P(t) \right] \gamma_5 \tau^{a} u(p_1) \; .
\ee
Ber\"ucksichtigt man wie oben nur den Pionpolbeitrag,
so l\"a\ss t sich die Summe der beiden Kommutatoren 
an der Schwelle in die Form
\be
\label{picont}
 \epsilon^\mu T_\mu^{a\,(\pi)}  = eg_{\pi NN} \epsilon^{a3c} 
   \,\frac{\epsilon\cdot (k-2q)}{m_\pi^2 -t} \,
   \bar{u}(p_2)\gamma_5 \tau^{c} u(p_1)
\ee
bringen. 
Die Raumkomponenten des Sigmaterms $\Sigma_\mu^{a}$ enthalten die 
Information \"uber die explizite Brechung der chiralen Symmetrie 
durch die Quarkmassen in der QCD-Lagrangedichte. Ihre Bestimmung
ist jedoch modellabh\"angig und wird uns in den n\"achsten 
Abschnitten noch besch\"aftigen. Vernachl\"assigt man diese
Korrektur, so ergeben die oben diskutierten Beitr\"age 
(\ref{nborn2},\ref{kr},\ref{picont}) folgende Bestimmung der invarianten 
Amplituden       
\beq
\label{let1}
A^{(+0,-)}_1 &=&  \frac{2f}{\mu} \spm
      \left\{ -\frac{1}{\nu+\nu_1} \mp \frac{1}{\nu-\nu_1} 
      + \frac{1\mp 1}{\nu_1} \right\}\, , \\
A^{(+0,-)}_2 &=&  \frac{2f}{\mu} \spm
      \left\{ -\frac{2}{\nu+\nu_1} \pm \frac{2}{\nu-\nu_1} \right\}\, , \\    
A^{(+0,-)}_3 &=& \; \frac{2f}{\mu} \kappa \; (-1 \pm 1) \, ,  \\
A^{(+0,-)}_4 &=& \frac{2f}{\mu}\;\kappa
      \left\{ \frac{2}{\nu+\nu_1} \pm \frac{2}{\nu-\nu_1} \right\}\, , \\ 
A^{(+0,-)}_5 &=&  \frac{2f}{\mu}\;\kappa
      \left\{ \frac{4}{\nu+\nu_1} \mp \frac{4}{\nu-\nu_1} \right\}\, , \\
\label{let6}       
A^{(+0,-)}_6 &=&  \frac{2f}{\mu}(1+2\kappa)
      \left\{ \frac{1}{\nu+\nu_1} \mp \frac{1}{\nu-\nu_1} \right\} 
      + \frac{2f}{\mu} \kappa\, (1\pm 1) \; .
\eeq
Dabei haben wir der \"Ubersichtlichkeit halber den Isospinindex
der anomalen magnetischen Momente unterdr\"uckt. Es gilt
$\kappa^{(\pm)}=\kappa^v$ und $\kappa^{(0)}=\kappa^s$. 
Mit Hilfe der im ersten Kapitel abgeleiteten Formel f\"ur die
Schwellenamplitude
\be
 \left. E_{0+}\right|_{thr} = \frac{e}{16\pi M}
 \frac{2+\mu}{(1+\mu)^{3/2}} \, \left. \left(
   A_3 + \frac{\mu}{2} A_6 \right) \right|_{thr}
\ee                
und der in Anhang A diskutierten Kinematik
ergibt sich schlie\ss lich folgendes Resultat f\"ur die elektrische
Dipolamplitude in den vier physikalischen Kan\"alen 
\beq
\label{LET1}
\Epn &=& \frac{e}{4\pi} \frac{\sqrt{2}f}{m_\pi}
    \left\{ 1 - \frac{3}{2}\mu + {\cal O}(\mu^2) \right\}
    \cong 26.6 \su , \\[0.1cm]
\label{LET2}    
\Emp &=& \frac{e}{4\pi} \frac{\sqrt{2}f}{m_\pi}
     \left\{ -1 + \frac{1}{2}\mu + {\cal O}(\mu^2) \right\}
    \cong -31.7 \su , \\[0.1cm]
\label{LET3}    
\Eop &=& \frac{e}{4\pi} \frac{f}{m_\pi}
     \left\{ -\mu + \frac{\mu^2}{2}(3+\kappa_p ) +
  {\cal O}(\mu^3) \right\}    \cong -2.32
  \su , \\[0.1cm]
\label{LET4}  
\Eon &=& \frac{e}{4\pi} \frac{f}{m_\pi}
     \left\{  \frac{\mu^2}{2}\kappa_n  +
  {\cal O}(\mu^3) \right\}  \cong -0.51 \su ,
\eeq
wobei wir die Werte $\kappa_p=1.79$ und $\kappa_n=-1.91$ sowie
$f^2/(4\pi)=0.08$ verwendet 
haben. Dieses Resultat liefert den Inhalt des Niederenergietheorems
\cite{Bae70,VZ72}. Die relative Ordnung der nicht bestimmten
Korrekturen wurde mit Hilfe verschiedener Annahmen \"uber
das Verhalten der Hintergrundamplitude festgelegt. Wir werden
diese Annahmen und ihre Rechtfertigung im n\"achsten Abschnitt
diskutieren.

\begin{figure}
\label{feyn}
\caption{Feynman-Diagramme f\"ur die Bornterme in der 
Pionphotoproduktionsamplitude.}
\vspace{9cm}
\end{figure}

Niederenergietheoreme zur Pionphotoproduktion lassen sich auch
direkt aus der Bestimmung der Bornterme in effektiven 
chiralen Meson-Nukleon Theorien ableiten \cite{Pec69}. Die
entsprechende Lagrangedichte unter Einbeziehung der 
elektromagnetischen Wechselwirkung lautet
\beq
\label{leff}
{\cal L} & =& \bar{\psi}(i\gamma\cdot{\cal D}-M)\psi 
  +\frac{1}{2}({\cal D}_\mu\phi^{a})^2 - \frac{1}{2}m_\pi^2
  (\phi^{a})^2  \\
 & & \mbox{} + \frac{f}{m_\pi} \bar{\psi}\gamma_5 \gamma_\mu
 \tau^{a} {\cal D}^\mu \phi^{a}\psi 
  + \frac{e}{4M}\bar{\psi} (\kappa^s +\kappa^v \tau^3)
  \sigma_{\mu\nu}\psi F^{\mu\nu}\; , \nonumber  
\eeq
wobei ${\cal D}_\mu=\partial_\mu+iQ{\cal A}_\mu$ die kovariante
Ableitung, ${\cal A}_\mu$ das elektromagnetische Potential und
$Q$ den Ladungsoperator bezeichnet. Die Eichung der pseudovektoriellen
Pion-Nukleon-Kopplung erzeugt eine $\gamma\pi NN$-Kontaktwechselwirkung
\be
{\cal L}_{\gamma\pi NN} = \frac{ef}{m_\pi}\epsilon^{3ab}
  \bar{\psi}\gamma_5 \gamma_\mu \tau^{a}\psi {\cal A}^\mu \phi^b\; ,
\ee  
die im Rahmen der effektiven Theorie die Kroll-Ruderman-Amplitude
liefert. Die Wirkung der kovarianten Ableitung auf das Pionfeld
bestimmt die Kopplung des elektromagnetischen Feldes an die
geladenen Pionen. Die $\gamma\pi\pi$-Wechselwirkung 
\be  
{\cal L}_{\gamma\pi\pi} = e\epsilon^{3ab}\phi^{a}\partial_\mu
 \phi^{b} {\cal A}^\mu
\ee
liefert schlie\ss lich den Pionpol in der Photoproduktionsamplitude
f\"ur geladene Pionen. Wir haben die entsprechenden Diagramme in 
Abbildung 2.1 zusammengestellt. Zus\"atzlich sind dort die 
Beitr\"age von Resonanzen im s- und t-Kanal gezeigt, die 
wir in Abschnitt 2.7 diskutieren werden. 
   
    
\section{Absch\"atzung der vernachl\"assigten Beitr\"age}
Um die Modellabh\"angigkeit des im letzten Abschnitt
vorgestellten Niederenergietheorems zu studieren, ist es
hilfreich, die \"Ubergangsmatrix in der Form 
\be
 T_\mu^{a} = T_\mu^{a({\rm LET})} + \delta T_\mu^{a}
\ee
zu zerlegen. Dabei bezeichnet $T_\mu^{a(\rm LET)}$ die 
T-Matrix, die zu den invarianten Amplituden  (\ref{let1}-\ref{let6})
geh\"ort und $\delta T_\mu^{a}$ die vernachl\"assigte 
Hintergrundamplitude. Nach dem Kroll-Ruderman-Theorem gilt
\be
  \lim_{q,k\to 0} \delta T_\mu^{a} =0 \, ,
\ee
so da\ss\ $\delta T_\mu^{a}$ am ''weichen`` Punkt $q_\mu=0$ verschwindet.
Um zu untersuchen, in welcher Ordnung in der Pionmasse 
die Amplitude $\delta T_\mu^{a}$ 
Korrekturen zur elektrischen Dipolamplitude an der physikalischen Schwelle
liefert, definieren wir
\be
  \delta T_\mu^{a} = \bar{u}(p_2) \sum_{\lambda} 
   \delta A_\lambda^{a}(\nu,\nu_1) {\cal M}_\lambda u(p_1)
\ee
und nehmen an, da\ss\ sich die die invarianten Amplituden 
$\delta A_\lambda (\nu,\nu_1)$ in eine Taylorreihe um den Punkt
$\nu=\nu_1=0$ entwickeln lassen
\be
 \delta A_\lambda^{a} (\nu,\nu_1) = a^{a}_{\lambda\, 00}
    + a^{a}_{\lambda\, 10} \nu + a^{a}_{\lambda\, 01}\nu_1
    + \ldots \; .
\ee
Diese Voraussetzung ist gerechtfertigt, da lediglich die Pion- und 
Nukleonpolterme Singularit\"aten bei $\nu=0$ oder $\nu_1=0$ 
enthalten. Diese Terme haben wir aber explizit in $T_\mu^{a(
{\rm LET})}$ ber\"ucksichtigt. Dar\"uber hinaus wollen wir 
in den folgenden Betrachtungen voraussetzen, da\ss\ alle
Koeffizienten $a^{a}_{\lambda\, ij}$ im Limes $m_\pi\to 0$
regul\"ar sind. Das bedeutet, da\ss\ sich diese Koeffizienten
beim Abz\"ahlen von Potenzen in $\mu$ als Gr\"o\ss en der
Ordnung ${\cal O}(1)$ betrachten lassen. 

Diese Annahme ist vermutlich unzutreffend, denn Pion-Schleifendiagramme 
k\"onnen Bei\-tr\"a\-ge liefern, die nichtanalytisch in $m_\pi$ sind 
\cite{LP71,PP71}. Die Gegenwart solcher Terme ist von Bernard et 
al.~\cite{BKG91} durch 
eine explizite Rechnung im Rahmen der chiralen St\"orungstheorie best\"atigt
worden\footnote{Dagegen bestreitet Naus \cite{Nau91} auf Grund von
\"Uberlegungen allgemeiner Natur die Existenz 
nichtanalytischer Terme in der Pionphotoproduktionsamplitude.}.
Trotzdem ist es von Interesse, die Gr\"o\ss enordnung der analytischen
Beitr\"age in $\delta T_\mu^{a}$ zu studieren. 
 
Die Amplitude $T_\mu^{a(\rm LET)}$ enth\"alt im isospinsymmetrischen 
Fall neben den Nukleon- und Pion-Polen auch einen Kontaktterm. 
Die Gegenwart dieses Terms unterscheidet die Born\-am\-pli\-tuden in
pseudovektorieller bzw.~pseudoskalarer Kopplung und ist deshalb
eine Konsequenz der PCAC-Relation. Dieses Resultat l\"a\ss t sich
als eine Bedingung f\"ur die invariante Amplitude $A_6^{(+0)}$
formulieren \cite{AG66}
\be
\label{FFR}
 \lim_{\nu\to 0} \lim_{\nu_1\to 0} A_6^{(+0)} (\nu,\nu_1)
   =  \frac{4f}{\mu} \kappa^{v,s} \; .
\ee
Die Reihenfolge der beiden Grenz\"uberg\"ange in (\ref{FFR})
ist nicht beliebig. Sie ist so gew\"ahlt, da\ss\ der Polterm
keinen Beitrag zum Grenzwert liefert. 
Da der Kontaktterm (\ref{FFR}) bereits in $T_\mu^{a(\rm LET)}$
enthalten ist, verschwindet $a^{(+0)}_{6\,00}$ im Grenzfall $q_\mu
\to 0$. De Baenst \cite{Bae70} verwendet daher in seiner
Diskussion der nicht bestimmten Amplitude $\delta T_\mu^{a}$ die 
zus\"atzliche Annahme $a^{(+0)}_{6\,00}=0$. 
          
Nur $\delta A_3$ und $\delta A_6$ tragen zur elektrischen 
Dipolamplitude an der Schwelle bei. Mit Hilfe der 
Eichinvarianzbedingung (\ref{gaugecond}) und der Forderung nach korrektem
Verhalten der Amplituden unter der Austauschtransformation
$(\nu,\nu_1)\to(-\nu,\nu_1)$ l\"a\ss t sich die m\"ogliche
Form der Taylorentwicklungen f\"ur $\delta A_{3,6}$ erheblich
einschr\"anken. F\"ur die isospinsymmetrischen Komponenten
findet man
\beq
 \delta A_{3}^{(+0)} &=& a_{3\, 11}^{(+0)} \nu\nu_1 + \ldots \; ,\\
 \delta A_{6}^{(+0)} &=& a_{6\, 01}^{(+0)} \nu_1
               + a_{6\, 20}^{(+0)} \nu^2
	       + a_{6\, 02}^{(+0)} \nu_1^2 + \ldots \; .
\eeq
An der Schwelle ist $\nu={\cal O}(\mu)$ und $\nu_1={\cal O}(\mu^2)$,
so da\ss\ die Austauschsymmetrie im wesentlichen das 
Transformationsverhalten der Amplitude unter $m_\pi\to -m_\pi$
spezifiziert. Auf Grund der Beziehung 
\be
\delta E_{0+} \sim \delta A_3 + \frac{\mu}{2} \delta A_6
\ee
folgt, da\ss\ die Hintergrundamplitude $\delta E_{0+}^{(+0)}$
an der Schwelle von der Ordnung $\mu^3$ ist. Verwendet man
an Stelle der Annahme $a_{6\,00}^{(+0)}=0$ die Absch\"atzung 
$a_{6\,00}^{(+0)}={\cal O}(\mu)$, so ergibt sich das schw\"achere 
Resultat $\delta E_{0+}^{(+0)}={\cal O}(\mu^2)$. Eine analoge
Argumentation l\"a\ss t sich auch f\"ur die isospinungeraden Komponenten 
durchf\"uhren. In diesem Fall findet man $\delta E_{0+}^{(-)}= 
{\cal O}(\mu^2)$.

\section{Die Methode von Furlan, Paver und Verzegnassi}
Die Ableitung des Niederenergietheorems im  Abschnitt 2.2
basierte im wesentlichen auf der Reduktionsformel und auf 
Wardidentit\"aten f\"ur die Zweipunktfunktion $\overline{\Pi}_{\mu\nu}^{a}$.
In diesem Abschnitt wollen wir auf eine andere Methode eingehen, die
direkt mit Ladungskommutatoren und Vollst\"andigkeitsrelationen
arbeitet. Im Rahmen dieses Verfahrens wurde erstmals darauf hingewiesen, 
da\ss\ die explizite Brechung der chiralen Symmetrie Korrekturen an 
das Standard-Niederenergietheorem liefern kann \cite{FPV74,NS89}.

Allerdings wird in der \"ublichen Diskussion der Methode nur der
Nukleonbeitrag in der Vollst\"andigkeitssumme explizit ber\"ucksichtigt.
In diesem Fall fehlt der f\"uhrende Beitrag zur Photoproduktion 
neutraler Pionen, und man mu\ss\ nachtr\"aglich Eichinvarianz 
erzwingen, um die korrekte Schwellenamplitude zu reproduzieren.
 
In diesem Abschnitt wollen wir demonstrieren, da\ss\ unter Einbeziehung
der Beitr\"age von Antinukleonen im Zwischenzustand auch das Verfahren
von Furlan et al.~die komplette Schwellenamplitude liefert. Zu diesem
Zweck betrachten wir die zu dem Strom $B_\mu^{a}$ geh\"orende Ladung
\be
 \overline{Q}^{a}_5(t) = Q^{a}_5(t) +  \frac{1}{m_\pi^2}\,
 \frac{d}{dt} \, \int d^3x\, D^{a} (\vec{x},t)\, .
\ee
Zwischen physikalischen Zust\"anden reduziert sich dieser Operator
auf die Ladung $\qfl$:
\beq
\label{q5l}
 \qfl (t) &=& Q_5^{a}(t) +\frac{i}{m_\pi}\dot{Q}^{a}_5(t)\, , \\
\label{q5r} 
 \qfr (t) &=& \left( \qfl (t)\right)^\dagger 
                =  Q_5^{a}(t) -\frac{i}{m_\pi}\dot{Q}^{a}_5(t)  \, .
\eeq
Den zugeh\"origen hermitesch konjugierten Operator haben wir mit
$\qfr$ bezeichnet. Die  Pionmatrixelemente dieser Operatoren lauten:  
\beq
  \langle\, 0\,|\,\qfl |\pi^{b}(q)\rangle  &=& \spm 2if_\pi m_\pi \,\delta^{ab}
                           (2\pi)^3  \delta^3 (\vec{q}\,)\, , \\[0.2cm]  
  \langle \pi^{b}(q)|\,\qfr |\,0\,\rangle  &=& -2if_\pi m_\pi \,\delta^{ab}
                            (2\pi)^3 \delta^3 (\vec{q}\,)\, , \\[0.2cm]
  \langle\pi^b(q)|\,\qfl |\,0\,\rangle &=& \langle\,0\,|\,\qfr |
  \pi^{b}(q)\rangle  = 0 \; .
\eeq
Die axialen Ladungen $Q_{5\,{\mini L,R}}^{a}$ sind nicht hermitesch
und haben die Eigenschaft, zwischen Pionen im Eingangs- und Ausgangskanal 
zu unterscheiden. Diese Tatsache erweist sich als besonders n\"utzlich 
bei der  Konstruktion von Summenregeln, da sie es erm\"oglicht,
bestimmte Prozesse in der Vollst\"andigkeitssumme zu selektieren. 
Im folgenden betrachten wir Summenregeln f\"ur das Matrixelement 
\be
 M_\mu^{a} = \langle N(p_2)| [\qfl ,V_\mu^{em}(0)]|N(p_1)\rangle \, ,
\ee
in dem sich mit Hilfe der oben angegebenen Matrixelemente die 
Photoproduktionsamplitude identifizieren l\"a\ss t. Das Resultat
besitzt eine sehr \"ubersichtliche Struktur als Summe von Poltermen 
und einem Dispersionsintegral, das die Hintergrundamplitude 
repr\"asentiert. 

Da der Operator $\qfl$ nur Pionen in Ruhe produziert, 
verlangt die Berechnung der Schwellenamplitude die Kenntnis
von $M_\mu^{a}$ im Schwerpunktsystem. Um die folgende Rechnung
etwas zu vereinfachen, werden wir $M_\mu^{a}$ statt dessen an
der Breitschwelle berechnen, das hei\ss t f\"ur Pionen, die
im Breitsystem des Nukleons ruhen\footnote{An der physikalischen Schwelle 
$\sqrt s=M+m_\pi$ ruht das Pion im Schwerpunktsystem. Der Impuls des Pions 
im Breitsystem des Nukleons ist dann tats\"achlich sehr klein, $|\vec{q}\,|
=\frac{1}{4}m_\pi\mu^2+{\cal O}(m_\pi^4)$.}:
\be
\begin{array}{rclcrcl}
  \vec{p}_2 &=&\spm \vec{p}, &\hspace{1cm} & \vec{q} &=& 0, \\[0.2cm]
  \vec{p}_1 &=&-\vec{p}    , &\hspace{1cm} & \vec{k} &=& 2\vec{p}.
\end{array}
\ee
Die eine Seite der Summenregel f\"ur $M_\mu^{a}$ ergibt sich, indem 
man das Matrixelement
\beq
\label{comqfl}
\lefteqn{\langle N(\vec p\,)|[\qfl,V_\mu^{em}(0)] |N(-\vec p\,)\rangle \;\;= } \\
&\hspace{1.0cm} & \langle  N(\vec p\,)|[Q_5^{a},V_\mu^{em}(0)]|N(-\vec p)\rangle 
 +\frac{i}{m_\pi}\langle N(\vec p\,)| [\dot{Q}_5^{a},V_\mu^{em}(0)]
 |N(-\vec p\,)\rangle  \nonumber 
\eeq
direkt auswertet. Dabei findet man den  Kroll-Ruderman-Term sowie
die bereits diskutierten Beitr\"age der expliziten Symmetriebrechung. Die 
andere Seite der Summenregel ergibt sich aus der Vollst\"andigkeitsrelation 
f\"ur den Kommutator (\ref{comqfl}). Die Clusterzerlegung \cite{AFF73} 
ist eine systematische Methode, um die verschiedenen Beitr\"age zur 
Vollst\"andigkeitssumme
\beq
\sum_n \langle N(\vec{p}\,)|\qfl |n\rangle \langle n|V_\mu^{em}|N(-\vec{p}\,)\rangle 
\eeq
zu identifizieren. Sie tr\"agt der Tatsache Rechnung, da\ss\
in einer relativistischen Theorie Beitr\"age mit unterschiedlichen 
Teilchenzahlen auftreten k\"onnen. Konkret zerlegt man $M_\mu^{a}$ 
in der Form    
\begin{figure}
\label{diag}
\caption{Beitr\"age zur Vollst\"andigkeitssumme f\"ur das
Operatorprodukt $V_\mu^{em}\qfl$.}
\vspace{9cm}
\end{figure}
\be
\label{cluster}
M_\mu^{a\;\;}  = M_\mu^{a\,I}+M_\mu^{a\,II} \; ,
\ee
\newpage
\beq
M_\mu^{a\,I\,} &=& \sum_\alpha \langle N(\vec{p})|\qfl |\alpha\rangle _c
                        \langle \alpha|V_\mu^{em}|N(-\vec{p}\,)\rangle _c \\   
 & &  \hspace{0.5cm} -  \sum_\beta \langle 0|\qfl |N(-\vec{p}\,)\beta\rangle 
 \langle N(\vec{p}\,)\beta|V_\mu^{em}|0\rangle \; +\; {\em c.~t.}\; ,\nonumber \\
M_\mu^{a\,II} &=& \sum_{\gamma_1} \langle N(\vec{p}\,)|\qfl |N(-\vec{p}\,)\gamma_1\rangle _c
                             \langle \gamma_1|V_\mu^{em}|0\rangle  \\   
 & &       \hspace{0.5cm} +  \sum_{\gamma_2} \langle 0|\qfl |\gamma_2\rangle 
       \langle \gamma_2 N(\vec{p}\,)|V_\mu^{em}|N(-\vec{p}\,)\rangle \; +\; 
       {\em c.~t.}\; , \nonumber
\eeq
wobei der Index $c$  den zusammenh\"angenden Teil des Matrixelements
und $c.t.$  die Voll\-st\"an\-dig\-keitssumme mit den Operatoren in der
anderen Reihenfolge bezeichnet. Wir haben die verschiedenen Terme 
schematisch in Abbildung 2.2 dargestellt. Der erste Teil der 
Clusterzerlegung beinhaltet solche Zust\"ande, die Baryonenzahl tragen. 
Die f\"uhrenden Terme in diesem Beitrag stammen von Nukleonen
$|\alpha\rangle \,=|N(\vec{p}\,)\rangle $ und Antinukleonen $|\beta\rangle 
\,=|\bar{N}(\vec{p}\,)\rangle $. Der zweite Teil von (\ref{cluster}) 
beschreibt die Produktion
eines Zustands $\gamma_{1,2}$ aus dem Vakuum, gefolgt von der Reaktion
$\gamma_1 +N(p_1) \to \qfl + N(p_2)$ bzw.~ $V_\mu^{em}+N(p_1)
\to \gamma_2 + N(p_2)$. Insbesondere findet man f\"ur Pionzust\"ande
$|\gamma_2\rangle \,=|\pi^{a}(\vec{q}\,)\rangle $ die Photoproduktionsamplitude
\be
\sum_{\pi^{b}(\vec{q})} \langle 0|\qfl |\pi^{b}(\vec{q}\,)\rangle \langle \pi^{b}(\vec{q}\,)
  N(\vec{p}\,)|V_\mu^{em}|N(-\vec{p}\,)\rangle \; = f_\pi S_\mu^{a}(\vec{q}=0)\, .
\ee
Auf Grund der speziellen Eigenschaften des Operators  $\qfl$
enth\"alt die Vollst\"andigkeitssumme keine Beitr\"age von der
inversen Reaktion $\pi^{a}(q)+N(p_1)\to V_\mu^{em}+N(p_2)$. Isoliert
man die Photoproduktionsamplitude $S_\mu^{a}$ und separiert 
die Nukleonbeitr"age in $M_{\mu}^{a\, I}$, so ergibt sich 
schlie\ss lich folgender Ausdruck f\"ur $S_\mu^{a}$
\beq
\label{fpv}
f_{\pi} S_{\mu}^{a}(\vec{q}=0) &=&
 i \epsilon^{a3c} \langle N(\vec{p}\,)|A_{\mu}^{c}|N(-\vec{p}\,)\rangle  
                   \\[0.3cm]
   & &\mbox{}-\sum_{N(\vec{p}_n)} \langle N(\vec{p}\,)|\qfl |N(\vec{p}_n)\rangle 
   \langle N(\vec{p}_n)|V_{\mu}^{em}|N(-\vec{p}\,)\rangle  
           \;+ \; {\em c.~t.} \nonumber \\
   & &\mbox{}  + \frac{i}{m_\pi}\langle N(\vec{p}\,)|[\dot{Q}_5^{a},V_{\mu}^{em}]
    |N(-\vec{p}\,)\rangle  
    \; + \;  f_\pi \delta S_\mu^{a}  \nonumber ,
\eeq
wobei $\delta S_\mu^{a}$ die vernachl\"assigten Beitr\"age in der
Vollst\"andigkeitssumme bezeichnet. Dabei handelt es sich vor allem um
$\pi N$-Kontinuumszust\"ande und Antinukleonen in $M_\mu^{a\, I}$, 
sowie Vektormesonen in $M_\mu^{a\, II}$.
Mit Hilfe der Eigenschaften des Operators $\qfl$ l\"a\ss t sich folgende
Darstellung der Hintergrundamplitude ableiten \cite{AFF73}
\be
\label{ressum}
\delta S_\mu^{a} = -im_\pi \sum_{n\neq\pi,N} (2\pi)^3 \delta^3 
  (\vec{p}-\vec{p}_n) 
  \frac{ \langle N(\vec{p}\,)|j_\pi^{a}|\,n\,\rangle 
  \langle \,n\,|V_\mu^{em}|N(-\vec{p}\,)\rangle  }{ 
       (E_p-E_n)(E_p+m_\pi-E_n+i\epsilon) }
  \;-\; c.t. \; . 
\ee
Die Summation l\"auft \"uber beliebige intermedi\"are Zust\"ande 
mit Ausnahme von Nukleonen und Pionen. $E_p=(\vec{p}^{\,2}+M^2)^{1/2}$
bezeichnet die Energie des auslaufenden Nukleons, $E_n$ die Energie
des Zwischenzustands. Der Energienenner in (\ref{ressum}) verschwindet
nur f\"ur Nukleonzust\"ande, so da\ss\ alle anderen Beitr\"age im Limes 
$m_\pi \to 0$ unterdr\"uckt sind. 

Wir betrachten nun im einzelnen die verschiedenen Beitr\"age zur
Photoproduktionsamplitude (\ref{fpv}). Den Kroll-Ruderman-Term 
sowie den Sigmakommutator haben wir bereits in Abschnitt 2.2 
diskutiert. Um den Nukleonterm zu berechnen, ben\"otigen wir die
Matrixelemente 
\beq
  \langle N(\vec{p}\,)|A_0^{a}(0)|\,N(\vec{p}\,)\,\rangle\,  &=&
     \frac{g_A}{M} \,\chi^\dagger_f (\vec{\sigma}\cdot\vec{p}\,)
     \frac{\tau^{a}}{2} \chi_i\, ,  \\  
 \langle N(\vec{p}\,)|V_0^{em}(0)|N(-\vec{p}\,)\rangle  &=&
     e \,\chi^\dagger_f (G_E^s (t) + \tau^3 G_E^v (t) ) \chi_i \, ,\\[0.1cm]
 \langle N(\vec{p}\,)|\vec{V}^{em}(0)|N(-\vec{p}\,)\rangle  &=&
     \frac{e}{M} \,\chi^\dagger_f (G_M^s(t) + \tau^3 G_M^v(t))  
     i(\vec{\sigma}\times\vec{p}) \chi_i \, .
\eeq     
Im Breitsystem treten die elektrischen und magnetischen Formfaktoren
des Nukleons,
\beq
  G_E(t) &=& F_1(t)+\frac{t}{4M^2}F_2(t) \, , \\[0.1cm]
  G_M(t) &=& F_1(t)+F_2(t)\, ,
\eeq
auf. Da wir bereits ein spezielles Bezugssystem gew\"ahlt haben,
ist es sinnvoll, die $T$-Matrix in einer nicht kovarianten Form
anzugeben. Der Nukleonbeitrag zur Photoproduktionsamplitude an  
der Breitschwelle $t=m_\pi^2$ ergibt sich schlie\ss lich zu
\beq
   T_0^{a}  &=& \spm i\frac{g_A}{f_\pi}\,\frac{1}{E_p}
        \chi^\dagger_f (G_E^s(t) \tau^{a} + G_E^v \delta^{a3})
	(\vec{\sigma}\cdot\vec{p}\,)\chi_i \, , \\
\vec{T}^{a} &=& -i \frac{g_A}{f_\pi} \, \frac{\vec{p}^{\, 2}}{mE_p}
       G_M^v(t) \,\chi^\dagger_f \frac{1}{4} [\tau^{a},\tau^3] 
       \,\vec{\sigma}_{\mini T}\, \chi_i\, ,
\eeq
wobei wir die transversalen und longitudinalen Spins 
\beq
   \vec{\sigma}_{\mini T} &=& \vec{\sigma} - \hat{p}(\vec{\sigma}
              \cdot\hat{p}), \\
   \vec{\sigma}_{\mini L} &=&  \hat{p}(\vec{\sigma}\cdot\hat{p})	      
\eeq
eingef\"uhrt haben. Nur der transversale Anteil liefert einen Beitrag
zur Photoproduktion  mit reellen Photonen. Im Falle des Nukleonterms
ist dieser Beitrag von der Ordnung $m_\pi^2$ und proportional zum
magnetischen Moment des Nukleons.   	             
      
Als pseudoskalares Teilchen koppelt das Pion stark an die unteren
Komponenten der Nukleonspinoren. Der f\"uhrende Beitrag zur Produktion
neutraler Pionen kommt daher von Zwischenzust\"anden, die propagierende 
Antinukleonen enthalten ('Z-Graphen`). Mit Hilfe der Darstellung
(\ref{ressum}) findet man
\be
 \vec{T}^{a} = i\frac{g_A}{f_\pi} \frac{m_\pi}{2E_p} \,
     \chi^\dagger_f (G_E^s(t)\tau^{a} + G_E^v(t) \delta^{a3})
     \vec{\sigma} \chi_i\; ,
\ee
wobei wir h\"ohere Ordnungen in $m_\pi/M$ vernachl\"assigt haben.
Damit ist die elektrische Dipolamplitude bis zur Ordnung $m_\pi^2$
bestimmt. Vernachl\"assigt man den Beitrag aus der expliziten
chiralen Symmetriebrechung, so ergibt sich
\beq
\label{LETa1}
\Epn &=& \frac{e}{4\pi} \frac{g_A}{\sqrt{2}f_\pi}
    \left\{ 1 - \frac{3}{2}\mu + {\cal O}(\mu^2) \right\}
    \cong 24.1  \su , \\
\Emp &=& \frac{e}{4\pi} \frac{g_A}{\sqrt{2}f_\pi}
     \left\{ -1 + \frac{1}{2}\mu + {\cal O}(\mu^2) \right\}
    \cong -29.6  \su  ,\\
\Eop &=& \frac{e}{4\pi} \,\frac{g_A}{2f_\pi}\;
     \bigg\{ -\mu + {\cal O}(\mu^2) \bigg\}  \cong -3.3 \su . 
\eeq
Die elektrische Dipolamplitude f\"ur die Produktion neutraler
Pionen am Neutron verschwindet in dieser Ordnung. Die von
(\ref{LET1},\ref{LET2}) abweichenden Werte in den geladenen Kan\"alen 
sind eine Konsequenz der Tatsache, da\ss\ die 
Goldberger-Treiman-Relation $\frac{g_A}{2f_\pi}=\frac{f}{m_\pi}$
experimentell um ca.~6\% verletzt ist. Diese Abweichung ist 
formal von der Ordnung $m_\pi^2$ und entspricht daher der 
oben vorgenommenen Absch\"atzung. 

\section{Explizite chirale Symmetriebrechung}
Die r\"aumlichen Komponenten des Beitrags aus der expliziten
chiralen Symmetriebrechung 
\be
\label{csbcom}
 \Sigma_\mu^{a}(\vec{q}=0) = 
  \int d^4 x \,\delta (x^0)\,\langle N(p_2)| [\partial^\nu A_\nu^{a}(x),
  V_\mu^{em}(0)] |N(p_1)\rangle
\ee
sind nicht durch Stromalgebra festgelegt. Aus diesem Grund haben 
wir ihren Beitrag zur Photoproduktionsamplitude bislang 
vernachl\"assigt. Repr\"asentiert man jedoch die Str\"ome
durch Quarkfelder, so ist auch dieser Kommutator durch die
kanonischen Vertauschungsregeln der Felder bestimmt. 
Der Vollst\"andigkeit halber arbeiten wir in Flavor-$SU(3)$,
so da\ss\
\beq
   \partial^\nu A_\nu^{a} &=& \frac{i}{2} \bar{\psi} \gamma_5
      \left\{ M,\lambda^{a} \right\} \psi\; ,  \\
    V_\mu^{em}            &=& \frac{1}{2} \bar{\psi} \gamma_\mu
      ( \lambda^3 + \frac{1}{\sqrt{3}} \lambda^8 ) \psi
\eeq
mit $M={\rm diag}(m_u,m_d,m_s)$. Es wird sich allerdings zeigen, da\ss\
die Masse der seltsamen Quarks nicht in das Resultat eingeht.
Der Kommutator der beiden Bilinearformen l\"a\ss t sich mit Hilfe 
der Relation              
\beq
\label{bilcom}
 \lefteqn{\delta (x^0-y^0) [\psi^\dagger (y)\frac{\lambda^{a}}{2}
      \Gamma \psi (y),\psi^\dagger (x)\frac{\lambda^{b}}{2}
      \Gamma' \psi (x)] = }  \\
    & & \hspace{1cm}   \frac{1}{2} \delta^4 (x-y) 
      \psi^\dagger (x) \left( if^{abc} \{\Gamma,\Gamma'\} 
      + d^{abc} [\Gamma,\Gamma' ] \right) \frac{\lambda^c}{2}
      \psi (x) \nonumber
\eeq
auswerten. Dabei bezeichnen $\Gamma$ und $\Gamma'$ die Diracoperatoren,
$f^{abc}$ und $d^{abc}$ die antisymmetrischen bzw.~symmetrischen 
$SU(3)$-Strukturkonstanten. Mit Hilfe von (\ref{bilcom}) ergibt sich f\"ur
$a=1,2,3$
\beq
\label{sig0q}
\int d^4x\, \delta (x^0) [\partial^{\nu}A_{\nu}^{a}(x),
   V_{0}^{em}(0)] &=& \,\,\overline{m} \,\epsilon^{3ab} \bar{\psi}
   \gamma_5\lambda^b \psi ,  \\  
\label{sigcom}
\int d^4 x\, \delta (x^0) [\partial^{\nu}A_{\nu}^{a}(x),
        V_{i}^{em}(0)] &=&
i\,\overline{m} \,\epsilon_{ijk} \left\{  \delta^{a3}
\frac{1}{\sqrt{3}}\left( \sqrt{2} J_{jk}^{0}+J_{jk}^{8} \right) +
 \frac{1}{3} J_{jk}^{a}\right\}    \\
& &\mbox{} + i\frac{\delta m}{2}\epsilon_{ijk}\delta^{a3} \left\{
 \frac{1}{3\sqrt{3}}\left(\sqrt{2} J_{jk}^{0}+ J_{jk}^{8} \right)
  + J_{jk}^{3}  \right\} , \nonumber
\eeq
wobei wir die Tensorstr\"ome
\be
 J_{\mu\nu}^{a} = \bar{\psi}\sigma_{\mu\nu}\frac{\lambda^a}{2}\psi
\ee
eingef\"uhrt haben. In der von uns verwendeten Normierung ist
$\lambda^0=\sqrt{2/3}\,1\!\! 1$. Man beachte, da\ss\ $(1/\sqrt{3})
(\sqrt{2}\lambda^0 +\lambda^8)$ gerade die Einheitsmatrix im 
$SU(2)$-Unterraum ist. Die Str\"ome (\ref{sig0q},\ref{sigcom})
enthalten daher keine Beitr\"age der seltsamen Quarks. Die St\"arke
der chiralen Symmetriebrechung sowie der Isospinbrechung wird
durch die Parameter
\beq
  \overline{m} &=& \frac{m_u+m_d}{2}\; ,  \\
  \delta m     &=& m_u -m_d
\eeq
kontrolliert. Die Zeitkomponente des Sigmakommutators ist bis auf
Korrekturen aus der Isospinbrechung durch die Divergenz des
Axialstroms gegeben.  Das Ergebnis (\ref{sig0q}) reproduziert 
daher das in Abschnitt 2.2 diskutierte Stromalgebraresultat (\ref{sig0}). 

Die r\"aumlichen Komponenten des Kommutators (\ref{csbcom}) 
liefern dagegen einen zus\"atzlichen Beitrag zur Photoproduktionsamplitude.
Dieser Beitrag verschwindet am ''weichen`` Punkt $q_\mu=0$ und 
bestimmt daher die Extrapolation der Amplitude zur
physikalischen Schwelle.

Die Berechnung dieses Terms beruht allein auf kanonischen 
Vertauschungsrelationen, die durch die Wechselwirkung der Quarks nicht 
modifiziert werden. In der quantisierten Theorie besteht allerdings
die Notwendigkeit, das Produkt der Str\"ome zu regularisieren. 
Shei und Tsao \cite{ST77} haben darauf hingewiesen, da\ss\ diese
Tatsache zu anomalen Beitr\"agen in den Vertauschungsrelationen 
f\"uhren kann. Das oben angegebene Resultat (\ref{sigcom}) beruht
daher strenggenommen auf einem freien Quarkmodell.

Ein weiteres Modell, in dem sich der Kommutator (\ref{csbcom}) 
bestimmen l\"a\ss t, ist das lineare $\sigma$-Modell. Dieses
Modell liefert die Str\"ome
\beq
  \partial^\nu A_\nu^{a} &=& f_\pi m_\pi^2 \pi^{a} \, ,\\
  V_\mu^{em} &=& \frac{1}{2}\bar\Psi (1+\tau_3)\gamma_\mu\Psi
    + \epsilon^{3ab}\pi^{a}\partial_\mu\pi^{b}   \, ,
\eeq      
wobei $\Psi$ einen $SU(2)$-Nukleonspinor und $\pi^{a}$ das
Triplet der Pionfelder bezeichnet. Im linearen $\sigma$-Modell
postuliert man kanonische Vertauschungsrelationen f\"ur das
Pionfeld $\pi^{a}$. In diesem Fall ergibt sich $\vec{\Sigma}^{a}
=0$, da der elektromagnetische Strom $\vec{V}^{em}$ nicht das 
zu $\pi^{a}$ konjugierte Feld $\dot\pi^{a}$ enth\"alt. 

Die QCD liefert also eine kompliziertere Form der Symmetriebrechung
als die Meson-Nukleon-Lagrangedichte des linearen $\sigma$-Modells.
Um Matrixelemente des Kommutators (\ref{sigcom}) zu studieren, 
f\"uhren wir Formfaktoren f\"ur die Tensorstr\"ome $J_{\mu\nu}^{a}$
ein \cite{FPV74,MS76}
\beq
  \langle N(p_2)|\bar{\psi}\sigma_{\mu\nu}\tau^{a}\psi|N(p_1)\rangle  &=& 
        \bar{u}(p_2) \left[
     G_T^{a}(t) \sigma_{\mu\nu} + iG_2^{a}(t)
     \frac{\gamma_\mu \Delta_\nu - \Delta_\mu \gamma_\nu}{2M} 
     \right. \\
 & & \mbox{}+ \left. iG_3^{a}(t) 
     \frac{\Delta_\mu P_\nu - P_\mu \Delta_\nu}{M^2}
     + iG_4^{a} \frac{\gamma_\mu P_\nu - P_\mu \gamma_\nu}{M^2}    
     \right] \tau^{a} u(p_1) \nonumber \, ,
\eeq
wobei $\Delta_\mu=(p_2-p_1)_\mu$ den Impuls\"ubertrag und $\tau^0 ={\bf 1}$ 
sowie $\tau^{a}\; (a=1,2,3)$ die Paulimatrizen bezeichnet.
Das Matrixelement vereinfacht sich erheblich, wenn man die
r\"aumlichen Komponenten der Str\"ome im Breitsystem des Nukleons
betrachtet
\beq
   \langle N(\vec{p}\,)|\bar{\psi}\sigma_{jk}\tau^a\psi|N(-\vec{p}\,)\rangle     
  & = & \epsilon_{jkm}\chi^\dagger_f \left[
     \left( G_T^{a}(t) +\frac{t}{4M^2} G_2^{a}(t)\right)\sigma_{{\mini T}m} 
    \right. \\
 & & \hspace{2.7cm} \mbox{} + \left. G_T^{a}(t)\frac{E_p}{M} 
 \sigma_{{\mini L}m} \right] \tau^{a} \chi_i \; . \nonumber
\eeq
Bis auf Korrekturen der Gr\"o\ss enordnung $m_\pi^2$ kann man die
Formfaktoren durch ihren Wert bei $t=0$ ersetzen.
Mit der Definition $g_T^{a}=G_T^{a}(0)$ ergibt sich schlie\ss lich  
folgende Korrektur zur Schwellenamplitude f\"ur neutrale Pionen 
\be
\label{delneu}
\Delta E_{0+}(\pi^0 N) = \frac{e}{4\pi f_\pi}\frac{\overline{m}}{m_\pi (1+\mu)}
  \left\{ \left( 1+\frac{\delta m}{6\overline{m}} \right) g_T^0
     \pm \left(\frac{1}{3}+\frac{\delta m}{2\overline{m}}\right) g_T^3
     \right\} \; ,
\ee
wobei sich die unterschiedlichen  Vorzeichen auf die Produktion am Proton 
bzw.~Neutron beziehen. Verwendet man die oben zitierten  Werte der
Quarkmassen, so ist $\delta m/(2\overline{m}) \simeq -1/3$, und
$\Delta E_{0+}(\pi^0N)$ ist fast vollst\"andig durch die Tensorkopplung
im Singletkanal bestimmt. Man beachte, da\ss\ der Korrekturterm formal
von der Ordnung $m_\pi$ ist, denn nach der GOR-Relation gilt $\overline{m}
= m_\pi^2f_\pi^2/|\langle \bar{u}u+\bar{d}d\rangle |$. Dagegen ergibt sich 
in der chiralen St\"orungstheorie kein zus\"atzlicher Beitrag in dieser 
Ordnung in $m_\pi$ \cite{BKG91}. Diese Tatsache ist konsistent mit der oben 
gemachten Feststellung, da\ss\ $\vec{\Sigma}^{a}$ in einer 
Meson-Nukleon-Theorie verschwindet.

Die entsprechende Korrektur f\"ur die Produktion geladener Pionen
lautet
\be
\label{delchar}
 \Delta E_{0+}(\pi^-p)=\Delta E_{0+}(\pi^+n) =
  \frac{\sqrt{2}e}{4\pi f_\pi}\frac{\overline{m}}{m_\pi (1+\mu)}
  \,\frac{g_T^3}{3}\; .
\ee  
In diesem Fall tr\"agt der isospinbrechende Term proportional
zu $\delta m$ nicht bei. Der Korrekturterm modifiziert nicht
die Ladungsasymmetrie $|\Epn|-|\Emp|$, liefert aber einen
kleinen Beitrag zum Panofskyverh\"altnis $\Epn/\Emp$.

Die wesentliche Aufgabe bei der Berechnung von $\Delta E_{0+}$
ist nun die Bestimmung der Tensorkopplungen $g_T^{a}$ des Nukleons. 
Diese sind experimentell nicht direkt zug\"anglich, so da\ss\ man 
in diesem  Zusammenhang auf Modelle angewiesen bleibt.
Die einfachste M\"oglichkeit ist die Verwendung eines
nichtrelativistischen Konstituentenmodells zur 
Beschreibung der Struktur des Nukleons. In diesem Fall
reduzieren sich die Tensorstr\"ome $\frac{1}{2}\epsilon_{ijk}
\bar{\psi}\sigma_{jk}\psi$ auf Axialstr\"ome $\bar{\psi}
\gamma_i\gamma_5 \psi$. Deren Nukleonmatrixelemente sind
durch die axialen Kopplungen
\be
 \langle N(p_2)|\bar\psi\gamma_\mu\gamma_5\tau^{a}\psi|N(p_1)\rangle =g_A^{a}\bar u(p_2)
  \gamma_\mu\gamma_5\tau^{a}u(p_1) + \ldots
\ee
bestimmt. Die Korrektur zur elektrischen Dipolamplitude lautet dann
\be
 \DEop = \frac{e}{4\pi f_\pi}\frac{\overline{m}}{m_\pi (1+\mu)}
    (0.90 \cdot g_A^0 + 0.04 \cdot g_A^3) \; .
\ee
In einem nichtrelativistischen Quarkmodell findet man $g_A^0=1$ und 
$g_A^3=5/3$, so da\ss\ $\DEop = 1.6 \su$. Diese Korrektur reduziert
die elektrische Dipolamplitude an der Schwelle auf den Wert 
$E_{0+}(\pi^0p)=-0.7\su$, in \"Ubereinstimmung mit den ersten 
Analysen des MAMI A Experiments \cite{NS89,TD90}.  

Allerdings bricht das verwendete nichtrelativistische Quarkmodell 
die chirale Symmetrie bereits im Ansatz, so da\ss\ die oben 
vorgenommene Absch\"atzung von $\Delta E_{0+}$ mit Vorsicht zu
betrachten ist. Wir werden eine sorgf\"altigere Bestimmung dieser
Korrektur in Kapitel 4 in Angriff nehmen. 

\section{Eichinvarianz}
Die Forderung nach Eichinvarianz der \"Ubergangsmatrix $T_\mu^{a}$
liefert wichtige Einschr\"ankungen f\"ur die Form der invarianten Amplituden.
Im Falle von Pionen auf der Massenschale ergeben sich diese 
Bedingungen aus der Erhaltung des elektromagnetischen Stroms
im \"Ubergangsmatrixelement
\be
\label{ongi}
k^\mu T_\mu^{a} = ie\langle \pi^{a}(q)N(p_2)|\partial^\mu V_\mu^{em}(0)|N(p_1)\rangle 
=0 \; .
\ee
Die aus dieser Gleichung folgenden Beziehungen (\ref{gaugecond}) haben 
wir bereits im ersten Kapitel angegeben. Man pr\"uft leicht nach, da\ss\ 
die in Abschnitt 2.2 abgeleiteten Amplituden diese Bedingungen erf\"ullen.
Diese Feststellung gilt jedoch nur f\"ur die Summe von Stromalgebra-, 
Nukleon- und Pionpolbeitr\"agen. Keiner dieser Terme ist f\"ur sich genommen 
eichinvariant. 

%Die nicht eichinvarianten Terme in den einzelnen Beitr\"agen heben
%sich allerdings nur dann gegenseitig weg, wenn keine ph\"anomenologischen 
%Formfaktoren an den Vertices verwendet werden. Um die Rolle
%der Formfaktoren n\"aher zu untersuchen, wollen wir unsere 
%Betrachtungen auf die Elektroproduktion von Pionen erweitern.
%In diesem Fall ist das ausgetauschte Photon virtuell und besitzt 
%eine nicht verschwindende invariante Masse $k^2$.  Die Kopplung
%des Photons wird durch die elektrischen Formfaktoren des 
%Nukleons sowie des Pions 
%\beq
% \Gamma_\mu^\gamma &=& F_1(k^2) \gamma_\mu + \frac{i\sigma_{\mu\nu}
%               k^\nu}{2M} F_2(k^2) \\
% \Gamma_\mu^{\gamma\pi} &=& F_\pi (k^2)(2q-k)_\mu
%\eeq
%beschrieben. Dar\"uber hinaus liefert der Stromalgebraterm
%\be
% C_\mu^{a} = -i\epsilon^{a3c} F_A(t) g_A \bar{u}(p_2)\gamma_\mu
%    \gamma_5 \frac{\tau^c}{2} u(p_1)
%\ee    
%einen Beitrag, welcher den normierten axialen Formfaktor
%$F_A(t)=G_A(t)/G_A(0)$ enth\"alt. Wie im Falle reeller 
%Photonen lautet die Eichinvarianzbedingung $k^\mu T_\mu^{a}=0$.
%Wir zerlegen  die Amplitude in der Form
%\be
% T_\mu^{a} = T_\mu^{a(Born)} + \Delta T_\mu^{a} + T_\mu^{a(Res)}
%\ee
%wobei $T_\mu^{a(Born)}$ die Polterme sowie des Stromalgebrabeitrag
%enth\"alt. Der Korrekturterm $\Delta T_\mu^{a}$ ist durch die Bedingung
%\be
% k^\mu ( T_\mu^{a(Born)}+\Delta T_\mu^{a}) =0
%\ee
%definiert, w\"ahrend $T_\mu^{a(Res)}$ eine Untergrundamplitude bezeichnet,
%die bis auf die Eichinvarianzforderung $k^\mu T_\mu^{a(Res)}=0$  unbestimmt 
%bleibt.
%
%Ber\"ucksichtigt man die Formfaktoren an den Vertices, so ist die
%Divergenz des isopsinantisymmetrischen Teils der Bornmaplitude
%\beq
%\label{ngi}
% k^\mu T_\mu^{(-)(Born)} &=& \frac{ief}{m_\pi} \bar{u}(p_2)\Big(
%          2M (2F_1^v(k^2) - F_\pi (k^2) ) \\
%   & & \hspace{3cm} \mbox{} - \gamma\cdot k 
%	  (2F_1^v(k^2) - F_A(t)) \Big) \gamma_5 u(p_1) \nonumber
%\eeq 
%Alle anderen Isospinkomponenten erf\"ullen die Eichinvarianzbedingung.
%F\"ur die $(-)$-Komponente ist dies nur am Photonpunkt $k^2=0$ der
%Fall. Um eine eichinvariante Amplitude zu erhalten, mu\ss\ man einen
%Korrekturterm \cite{VZ72,SK91}
%\beq
%\label{gcor}
%\Delta T_\mu^{(-)} &=& -\frac{ief}{m_\pi} \bar{u}(p_2)\left(
%          \frac{2Mk_\mu}{k^2} (2F_1^v(k^2) - F_\pi (k^2) ) \right.\\
% & & \hspace{3cm} \mbox{}	  
%	  - \left. \frac{k_\mu\gamma\cdot k}{k^2} (2F_1^v(k^2) - F_A(t))
%	   \right) \gamma_5 u(p_1) \nonumber
%\eeq 
%addieren. Dieser Term ist nicht eindeutig bestimmt. Jeder beliebige
%Ausdruck, der sich von (\ref{gcor}) nur um einen divergenzfreien
%Beitrag unterscheidet, ist ebenfalls ein m\"oglicher Korrekututerm.
%Die Summe $T_\mu^{a(Born)}+\Delta T_\mu^{a}$ liefert schlie\ss lich
%eine eichinvariante Elektroproduktionsamplitude. 
%
Es ist instruktiv, die Konsequenzen von Eichinvarianz auch f\"ur 
Pionen abseits der Massenschale zu untersuchen. Dieses Problem
ist vor allem  bei der Bestimmung der Amplitude am weichen Punkt
von Bedeutung. Da sich das Pion nicht in einem asymptotischen
Zustand befindet, kann man zu diesem Zweck allerdings nicht von
Gleichung (\ref{ongi}) Gebrauch machen.  Statt dessen betrachten 
wir die zu (\ref{avward}) analoge Vektorwardidentit\"at
\be
\label{vwi}
ik^\mu \overline{\Pi}_{\nu\mu}^\alpha (q) = - C^\alpha_\nu + 
\frac{i}{m_\pi^2} \Big\{ q_\nu \Sigma_0^\alpha (q) - 
 g_{\nu 0} k^\rho \Sigma_\rho^\alpha (q) \Big\} .
\ee
Auch diese Relation beruht auf der Erhaltung des elektromagnetischen 
Stroms. Sie enth\"alt aber keine zus\"atzlichen Annahmen \"uber den
Impuls des Pions. In Verbindung mit der Axialvektorwardidentit\"at
(\ref{avward}) ergibt sich folgender Ausdruck f\"ur die Divergenz
von $T_\mu^{a}$
\be
\label{offgi}
 k^\mu T_\mu^{a} = -i\epsilon^{a3c} \frac{q^2-m_\pi^2}{f_\pi m_\pi^2}
   \langle N(p_2)|D^c(0)|N(p_1)\rangle  \; .
\ee
F\"ur Pionen auf der Massenschale findet man die bekannte Beziehung
$k^\mu T_\mu^{a} =0$. Abseits der Massenschale liefern geladene 
virtuelle Pionen einen zus\"atzlichen Quellterm f\"ur den elektromagnetischen
Strom und bewirken eine nichtverschwindende Divergenz von $T_\mu^{a}$.

Bei der Herleitung der Relation (\ref{offgi}) ben\"otigt man keine 
Annahmen \"uber die modell\-ab\-h\"angigen Komponenten der symmetriebrechenden
Amplitude $\Sigma_\mu^{a}$. Betrachtet man die einzelnen Beitr\"age 
zur linken Seite von (\ref{offgi})
\be
\label{divamp}
 k^\mu T_\mu^{a} = \frac{1}{f_\pi} \left\{ k^\mu q^\nu 
   \overline{\Pi}_{\mu\nu}^{a}
   -ik^\mu C_\mu^{a} -\frac{\omega_\pi}{m_\pi^2} k^\mu \Sigma_\mu 
   \right\}\, ,
\ee     
so tragen diese Terme jedoch bei. Die Polterme erf\"ullen in Verbindung
mit dem Stromalgebrabeitrag auch die verallgemeinerte Eichinvarianzbedingung
(\ref{offgi}). Man kann diese Terme daher aus der Gleichung (\ref{divamp})
eliminieren. In der Herleitung des Niederenergietheorems vernachl\"assigt
man Hintergrundbeitr\"age zu den Amplituden. Die Gleichung (\ref{divamp})
reduziert sich daher auf eine Beziehung f\"ur die symmetriebrechende
Amplitude: $k^\mu \Sigma_\mu^{(+0)}=0$. Die im Abschnitt 2.5 bestimmten
Beitr\"age erf\"ullen diese Gleichung nicht. Wir definieren daher den 
eichinvarianten Teil von $\Sigma_\mu^{a}$
\be
  \Sigma_\mu^{a(gi)} = \Sigma_\mu^{a} +\Delta\Sigma_\mu^{a}
\ee
durch die Forderung $k^\mu\Sigma_\mu^{(+0)(gi)}=0$. Die isospinungeraden
Komponenten sind durch die Bedingung
\be
  k^\mu T_\mu^{(-)}= -i\frac{q^2-m_\pi^2}{f_\pi m_\pi^2}
    \langle N(p_2)|D(0)|N(p_1)\rangle 
\ee
bestimmt. Eine m\"ogliche L\"osung dieser Gleichungen lautet
\beq
 \frac{\omega_\pi}{m_\pi^2}\Sigma_\mu^{(-)(gi)} &=& -
                 \frac{f_\pi}{m_\pi^2-t}\, g_{\pi NN}
                   \, \bar{u}(p_2)i\gamma_5  q_\mu u(p_1)\, , \\
 \frac{\omega_\pi}{m_\pi^2}\Sigma_\mu^{(0)(gi)} &=& \spm
             \frac{4\overline{m}M}{m_\pi^2} \,\frac{g_T^3}{3}
	     \, \bar{u}(p_2)i\gamma_5 \frac{\gamma_\mu \gamma\cdot k}{2M}u(p_1)\, , \\
 \frac{\omega_\pi}{m_\pi^2}\Sigma_\mu^{(+)(gi)} &=& \spm
             \frac{4\overline{m}M}{m_\pi^2} \,
	     \left\{ g_T^0 \left(  1+\frac{\delta m}{6\overline{m}} \right)
	     \pm g_T^3 \frac{\delta m}{2\overline{m}} \right\}
	 \, \bar{u}(p_2)i\gamma_5 \frac{\gamma_\mu \gamma\cdot k}{2M}u(p_1)\, .
\eeq
Diese Amplituden sind durch die Forderung $k^\mu\Sigma_\mu^{(+0)(gi)}=0$
nicht eindeutig bestimmt. Wir haben sie deshalb durch die zus\"atzliche
Bedingung festgelegt, da\ss\ $\Delta\Sigma_\mu$ die Schwellenamplitude 
(\ref{delneu}) nicht ver\"andert und auch keine Beitr\"age zu der 
longitudinalen Multipolamplitude $L_{0+}$ auftreten. Diese Amplitude 
l\"a\ss t sich in der Elektroproduktion von Pionen bestimmen \cite{SK91}.
Obwohl wir kein eichinvariantes, mikroskopisches Modell f\"ur die 
Photoproduktion von Pionen am Nukleon besitzen, haben wir damit eine 
manifest eichinvariante Amplitude konstruiert, die die Beitr\"age der 
expliziten Symmetriebrechung auf dem Quarkniveau ber\"ucksichtigt.	 
   

\section{Resonanzbeitr\"age}
Das Niederenergietheorem zur Pionphotoproduktion beruht auf der
Annahme, da\ss\ sich die Zweipunktfunktion $q^\nu\overline\Pi^a_{\mu\nu}$
in der N\"ahe des ''weichen`` Punktes $q_\mu=0$ durch die Nukleonpolterme
approximieren l\"a\ss t. Dabei vernachl\"assigt man  die Beitr\"age
von Schleifendiagrammen sowie den Austausch von Resonanzen im s- oder
t-Kanal. 

In der Photoproduktion von Pionen bei mittleren Energien 
$\omega^{lab}= 0.3-1.5$ GeV ist die Bedeutung von s-Kanal-Resonanzen
in den Multipolamplituden deutlich zu erkennen. 
An der Schwelle sind diese Beitr\"age jedoch durch das
Verh\"altnis $m_\pi/\Delta E_R$ der Pionmasse zur Anregungsenergie
der Resonanz unterdr\"uckt. Der niedrigste Anregungszustand des
Nukleons ist die Deltaresonanz bei $\Delta E_R =294$ MeV. Dieser
Zustand koppelt au\ss erordentlich stark an das Pion-Nukleon-System
und dominiert aus diesem Grund die resonante $M_{1+}$-Amplitude
bis in die Schwellenregion. Der niedrigste resonante Beitrag zur
$E_{0+}$-Amplitude stammt vom $N(1535)$ bei einer deutlich h\"oheren
Anregungsenergie $\Delta E_R= 597$ MeV. Im Gegensatz zur Deltaresonanz
zerf\"allt dieser Zustand zu etwa 50\% in $\eta N$ und liefert 
insgesamt nur einen geringen Beitrag zur $E_{0+}$-Amplitude
an der Schwelle.

Um diese Aussagen quantitativ zu belegen, wollen wir die 
Resonanzbeitr\"age mit Hilfe effektiver chiraler Lagrangedichten studieren 
\cite{Pec69,OO75,NB80}. Diese Methode ignoriert die intrinsische 
Struktur der Resonanz, hat aber den wesentlichen Vorteil, mit einem
Minimum an freien Parametern auszukommen. Diese Parameter beschreiben
neben der Masse der Resonanz die Kopplungen $\gamma N\to N^{*}$ 
sowie $N^{*}\to N\pi$ und lassen sich aus den experimentell bestimmten
Helizit\"atsamplituden und Zerfallsbreiten extrahieren.

Zu diesem Zweck betrachten wir die resonante Photoproduktion
$\gamma N(\Lambda_i=\frac{1}{2},\frac{3}{2}) \to N^{*} \to \pi N$
mit definierter Helizit\"at $\Lambda_i$ im Eingangskanal. Die
zugeh\"origen Helizit\"atsamplituden $A_{1/2}$ und $A_{3/2}$ sind durch
\beq
\label{helamp}
 A_{l\pm} &=& \mp \alpha C_{N\pi} A_{1/2}\; ,  \\
 B_{l\pm} &=& \pm \frac{4\alpha}{\sqrt{(2J-1)(2J+3)}} C_{N\pi} A_{3/2}
\eeq
definiert \cite{PDG90}. Die Helizit\"atskomponenten $(A_{l\pm},B_{l\pm})$ 
sind Linearkombinationen der Multipolamplituden $(E_{l\pm},M_{l\pm})$.
Die entsprechenden Zusammenh\"ange finden sich im Anhang B. Der
Parameter $\alpha$ lautet
\be
 \alpha = \left[ \frac{1}{\pi} \frac{k}{q} \frac{M\Gamma_\pi}{(2J+1)
    M_R \Gamma^2} \right]^{1/2} \; .
\ee
Dabei bezeichnet $M_R$ die Masse der Resonanz, $J$ ihren Spin
und $\Gamma$ sowie $\Gamma_\pi$ die totalen bzw.~partiellen Zerfallsbreiten.
$C_{N\pi}$ ist der Clebsch-Gordan-Koeffizient f\"ur den Zerfall der 
Resonanz in den relevanten $N\pi$-Ladungszustand.  Die Definition
(\ref{helamp}) hat den Vorzug, da\ss\ alle Gr\"o\ss en, die mit der
Propagation und dem Zerfall der Resonanz zusammenh\"angen, aus
der  eigentlichen Resonanzamplitude eliminiert werden. Die 
Helizit\"atsamplituden $A_{1/2,3/2}$ liefern daher ein zuverl\"assiges
Ma\ss\ f\"ur die St\"arke des \"Ubergangsmatrixelements $\gamma N\to N^{*}$. 
In Tabelle 2.1 haben wir die entsprechenden Werte f\"ur die
wichtigsten Resonanzen mit Massen unterhalb 1.65 GeV zusammengefa\ss t. 
    
\begin{table}
\caption{Helizit\"atsamplituden (in $10^{-3}\,{\rm GeV}^{1/2}$) sowie
totale und partielle Breiten (in MeV) f\"ur die wichtigsten Nukleonresonanzen
mit Massen unterhalb 1.65 GeV. Alle Angaben nach [PDG90].}
\begin{center}
\begin{tabular}{|l||c|r|r|r|r|} \hline
  Resonanz             & Hel.  &  $A_{1/2,3/2}^p$ & $A_{1/2,3/2}^n$ 
		& $\Gamma_{tot}$ & $\Gamma_\pi$ \\ \hline\hline
 $N(1440)\,P_{11}$ & 1/2   &  $-69\pm 7\;\,$  & $37\pm 19$
                &  200         & 120   \\ 
 $N(1520)\,D_{13}$ & 1/2   &  $-22\pm 10$     & $-65\pm 13$
                &  125         &  70    \\
                       & 3/2   &  $167\pm 10$     & $144\pm 14$
		&              &        \\
 $N(1535)\,S_{11}$ & 1/2   &  $73\pm 14$      & $-76\pm 32$
                &  150         &   65    \\
 $N(1650)\,S_{11}$ & 1/2   &  $48\pm 16$      & $-17\pm 37$ 
                & 150	       &   90    \\
 $\Delta (1232)\,\rm P_{33}$ & 1/2 & $-141\pm 5\;\,$&
                &  115         &  115   \\
		        & 3/2  &  $-258\pm 11$    &          
		&              &        \\ \hline
\end{tabular}
\end{center}
\end{table}

Die dominante Resonanz in der $E_{0+}$-Amplitude ist die 
$N(1535)S_{11}$-Anregung. Dieser Zustand besitzt wie das Nukleon Spin und 
Isospin 1/2, aber negative Parit\"at. Anregung und Zerfall der Resonanz 
werden durch die Kopplungen
\beq
\label{s11coup}
 {\cal L}_{\pi NN^{*}} &=& \frac{f_R}{m_\pi} \bar{\psi}_{N^{*}}
   \gamma_\mu \tau^{a}\psi \partial^\mu \phi^{a} + h.c. \; ,\\
 {\cal L}_{\gamma NN^{*}} &=& \frac{e}{4M} \bar{\psi}_{N^{*}} 
   \gamma_5 \sigma_{\mu\nu} (\kappa^s_R +\kappa^v_R \tau^3) \psi
    F^{\mu\nu} + h.c.
\eeq
beschrieben. Die beiden Parameter $f_R$ und $\kappa_R$ werden mit  
Hilfe der Beziehungen
\beq
\label{rescoup}
       f_R         &=& \frac{2m_\pi}{M_R-M} 
       \sqrt{\frac{\pi M_R \Gamma_\pi}{ p_1(E_1+M)}} 
       \simeq 0.27 , \\
 e\kappa^{p}_R   &=& \frac{(2M)^{3/2}}{\sqrt{(M_R+M)(M_R-M)}} A^{p}_{1/2}
       \simeq 0.51 e
\eeq
bestimmt. Dabei bezeichnen $E_1$ und $p_1$ die Energie sowie den Impuls
des Nukleons im Ruhesystem des angeregten Zustands bei der Resonanzenergie
$\sqrt{s}=M_R$. Unter Verwendung der Vertices (\ref{s11coup}) lassen sich
nun die Borndiagramme zur resonanten Photoproduktion berechnen. Die 
zugeh\"origen invarianten Amplituden finden sich im Anhang B. Der Beitrag
der s-Kanal-Anregung der N(1535)-Resonanz zur elektrischen Dipolamplitude
an der Schwelle lautet
\beq
  E_{0+}^{N^{*}}(p\pi^0) &=& \frac{e\kappa_R}{16\pi M}\frac{f_R}{m_\pi}
    \mu^2\frac{2+\mu}{(1+\mu)^{3/2}} 
    \frac{M(M_R+M+m_\pi)}{M_R^2-(M+m_\pi)^2} \\[0.2cm]
    &\simeq& 0.06 \su \, .  \nonumber
\eeq    
Dieses Ergebnis ist formal von der Ordnung $\mu^2$ und widerspricht daher
der in Abschnitt 2.3 vorgenommenen Absch\"atzung der vernachl\"assigten
Amplitude. Das liegt darin begr\"undet, da\ss\ der s-Kanal-Beitrag f\"ur
sich genommen nicht Austauschinvariant ist. 

Die leichteste Anregung mit denselben Quantenzahlen wie das Nukleon ist
die Roperesonanz $N(1440)$. Dieser Zustand liefert einen resonanten
Beitrag zur $M_{1-}$-Amplitude, ist in der elektrischen Dipolamplitude 
aber nur als Untergrund pr\"asent. Die effektive Lagrangedichte, welche
die Kopplung des $N(1440)$ an das Nukleon beschreibt, lautet
\beq        
\label{nstarcoup}
 {\cal L}_{\pi NN^{*}} &=& \frac{f_R}{m_\pi} \bar{\psi}_{N^{*}}
   \gamma_\mu \gamma_5\tau^{a}\psi \partial^\mu \phi^{a} + h.c. \; ,\\
 {\cal L}_{\gamma NN^{*}} &=& \frac{e}{4M} \bar{\psi}_{N^{*}} 
    \sigma_{\mu\nu} (\kappa^s_R +\kappa^v_R \tau^3) \psi
    F^{\mu\nu} + h.c. \; .
\eeq
Bestimmt man die Kopplungskonstanten aus der Zerfallsbreite und der
Helizit\"atsamplitude bei der Resonanzenergie $\sqrt{s}=m_R$, so
ergibt sich $f_R=0.48$ und $\kappa_R^p=0.58$. Mit diesen Werten 
findet man folgenden Beitrag der s-Kanal-Anregung
%\be
% E_{0+}^{N^{*}}(p\pi^0) &=& \frac{e\kappa_R}{16\pi M}\frac{f_R}{m_\pi}
%    \frac{2\mu+\mu^2}{(1+\mu)^{3/2}} 
%     \frac{m_\pi(M_R-M-m_\pi)}{(M+m_\pi)^2-M_R^2}  
%\ee    
\be    
   E_{0+}^{N^{*}}(p\pi^0)  \simeq -0.025 \su \, . 
\ee 
Wie erwartet ist die entsprechende Amplitude au\ss erordentlich klein.
Eine gewisse Schwierigkeit stellt die Behandlung der Deltaresonanz
$\Delta (1232)$ dar \cite{DMW91,NS89,NB80}. Dieser Zustand ist
eine $P_{33}$-Anregung und sollte daher nicht zur s-Wellen-Produktion 
beitragen. In einer relativistischen Beschreibung
der Deltaresonanz als elementares Spin-3/2 Rarita-Schwinger-Feld
enth\"alt der Deltapropagator allerdings abseits der Massenschale auch
Spin-1/2 Komponenten. Die Kopplung dieser Beitr\"age an 
die Zerfallskan\"ale $\gamma N$ und $\pi N$ ist im wesentlichen
unbestimmt.  Je nach Wahl der entsprechenden Parameter findet man
\cite{NS89}
\be
\label{delta}
   E_{0+}^\Delta(\pi^0p) = (-0.10 \ldots 0.34) \su  \; .
\ee
Das angegebene Intervall entspricht der Streuung, die sich aus
verschiedenen Fits der Parameter an die nicht resonanten
Amplituden ergibt.  Das Resultat zeigt deutlich die Grenzen der 
Verwendung effektiver chiraler Lagrangedichten bei der Beschreibung
angeregter Zust\"ande auf. Trotzdem sind auch die Korrekturen
auf Grund der Deltaresonanz letztlich relativ gering. 

Wir haben unsere Untersuchung bislang auf die Rolle von Resonanzen
im s-Kanal beschr\"ankt. Aus dem Studium von Dispersionsrelationen 
ist jedoch bekannt, da\ss\ die Einbeziehung von Vektormesonen als
t-Kanal-Resonanzen die Beschreibung der differentiellen Wirkungsquerschnitte 
besonders bei kleinen Energien verbessert \cite{BDW67}. Es scheint daher 
angemessen, die Rolle von Vektormesonen auch direkt an der Schwelle zu 
untersuchen. Dabei beschr\"anken wir uns auf die $\rho$- und $\omega$-Mesonen. 
Das $\phi$-Meson sowie die schweren Vektormesonen liefern nur geringe
Beitr\"age. Die relevanten Terme in der effektiven Lagrangedichte lauten:
\beq
\label{lvm}
 {\cal L}_{\rho NN} &=& f_{\rho NN} \bar{\psi}
         \left( \gamma_\mu +\frac{\kappa_\rho}{2M}\sigma_{\mu\nu}
	 \partial^\nu \right) \vec{\tau}\cdot\vec{\rho}^{\,\mu} \psi , \\ 
 {\cal L}_{\omega NN} &=& f_{\omega NN} \bar{\psi}
         \left( \gamma_\mu +\frac{\kappa_\omega}{2M}\sigma_{\mu\nu}
	 \partial^\nu \right) \omega^\mu \psi  ,\\
 {\cal L}_{\rho\pi\gamma} &=& \frac{eg_{\rho\pi\gamma}}{2m_\pi}
         \epsilon_{\alpha\beta\gamma\delta} F^{\alpha\beta}
	 \vec{\phi}\cdot\partial^\gamma\vec{\rho}^{\,\delta} ,\\
 {\cal L}_{\omega\pi\gamma} &=& \frac{eg_{\omega\pi\gamma}}{2m_\pi}
         \epsilon_{\alpha\beta\gamma\delta} F^{\alpha\beta}
	 \phi_3\cdot\partial^\gamma\omega^\delta \; .
\eeq
Die Kopplung des Photons l\"a\ss t sich aus den gemessenen Zerfallsbreiten
$\Gamma(\rho,\omega\to\pi\gamma)$ bestimmen. Die 
Vektormeson-Nukleon-Kopplungskonstante mu\ss\ dagegen indirekt, aus 
detaillierten Analysen
des Nukleon-Nukleon-Potentials gewonnen werden \cite{Dum82}. Die
resultierenden Werte finden sich in Tabelle 2.2.  
 
\begin{table}
\caption{Parameter f\"ur die wichtigsten t-Kanal-Beitr\"age 
zur Pionphotoproduktion.}
\begin{center}
\begin{tabular}{|rcl|rcl|rcl|}\hline
   & $\pi$ &             &  & $\rho$ &             &    &  $\omega$ &  \\ 
                                                                \hline\hline
$m_{\pi^\pm}$&=&139 MeV  & $m_\rho$&=&$770$ MeV    &  $m_\omega$&=&$783$ MeV\\
$f_{\pi NN}$&=&$1.00$    &  $f_{\rho NN}$&=&$2.66$ & $f_{\omega NN}$&=&$7.98$\\
$g_{\pi\pi\gamma}$&=&$1$ &  $g_{\rho\pi\gamma}$&=&$0.125$ 
                                        &   $g_{\omega\pi\gamma}$&=&$0.374$  \\
    & &       &  $\kappa_\rho$&=&$6.6$  &   $\kappa_\omega$&=&$0.$   \\ \hline
\end{tabular}
\end{center}
\end{table}                    

Die invarianten Amplituden, die sich aus der Wechselwirkung (\ref{lvm})
ergeben, haben wir in Anhang B gesammelt. Der Beitrag zur
elektrischen Dipolamplitude an der Schwelle ist
\be
  E_{0+}^V(\pi^0p) = \frac{e}{16\pi} \sum_V\frac{g_V}{m_\pi} 
    f_V (1+\kappa_V)\mu^3
     \frac{2+\mu}{(1+\mu)^{3/2}}\frac{M^2}{m_\pi^2+m_V^2 (1+\mu)}\, ,
\ee     	 
wobei $V=\rho,\omega$ zu setzen ist. Dieses Resultat ist explizit
von der Ordnung $\mu^3$ und entspricht daher der Absch\"atzung
aus dem Abschnitt 2.5. Mit den Werten aus Tabelle 2.2 findet man 
$g_\rho f_\rho (1+\kappa_\rho)=2.53$ und  $g_\omega f_\omega 
(1+\kappa_\omega)=2.98$, so da\ss\ sich schlie\ss lich die 
Korrektur $E_{0+}^V(\pi^0p)=0.21\su$ ergibt. Auch dieser Beitrag 
liefert also keine wesentliche Modifikation der elektrischen 
Dipolamplitude an der Schwelle.  
 
\section{Zusammenfassung}
Wir haben in diesem Kapitel Stromalgebratechniken zur Bestimmung der 
Pionphotoproduktionsamplitude an der Schwelle eingesetzt. Das klassische
Niederenergietheorem f\"ur die elektrische Dipolamplitude $E_{0+}$
ergibt sich, indem man die \"Ubergangsmatrix durch die Nukleon- und
Pion-Bornterme approximiert. Wir haben gezeigt, da\ss\ die \"ubliche 
Absch\"atzung der Korrekturen zu diesem Resultat auf zus\"atzlichen 
Annahmen \"uber das Verhalten der Photoproduktionsamplitude beruhen, 
die nicht notwendig gerechtfertigt sind. 

Im Abschnitt 2.5 sind wir auf die Rolle der expliziten Symmetriebrechung
durch die Quarkmassen in der QCD-Lagrangedichte eingegangen. Wir haben 
gezeigt, da\ss\ dieser Effekt in einem Modell freier Quarks zu einem 
Korrekturterm f\"uhrt, der proportional zu den Massen der leichten 
Quarks und der Tensorkopplungskonstante des Nukleons ist. Einfache 
Absch\"atzungen f\"uhren zu dem Schlu\ss , da\ss\ dieser Term einen 
erheblichen Beitrag zur elektrischen Dipolamplitude f\"ur neutrale Pionen 
liefern kann.

Abschlie\ss end haben wir demonstriert, da\ss\ Resonanzen im s- oder 
t-Kanal keine wesentlichen Korrekturen zur elektrischen Dipolamplitude
hervorrufen. Eine wichtige Frage, auf die wir in dieser Arbeit nicht 
eingehen k\"onen, betrifft die Bedeutung der Endzustandswechselwirkung 
im Pion-Nukleon-System. Insbesondere bei der Produktion neutraler Pionen,
wo der elementare Proze\ss\ deutlich unterdr\"uckt ist, kann die Reaktion
$\gamma p\to \pi^+n\to\pi^0p$ bedeutende Korrekturen liefern. Die 
theoretische Bestimmung dieser Beitr\"age wird allerdings gegenw\"artig 
noch sehr kontrovers diskutiert \cite{DT91,Kam89,NLB90,LYL91,Ber91,BKG91}.
