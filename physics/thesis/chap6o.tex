\chapter{Analyse der Spektralfunktionen}
\section{Bestimmung der Spektralfunktion im Vektorkanal}
Die Spektralfunktion im Vektorkanal l\"a\ss t sich mit Hilfe 
des optischen Theorems aus dem totalen Wirkungsquerschnitt
f\"ur die Annihilation von $e^+e^-$-Paaren in Hadronen 
mit dem Isospin $I=1$ bestimmen
\be
\label{opttheo}
 \frac{1}{\pi}{\rm Im}\Pi^V(s) = \frac{s}{16\pi^3\alpha^2}
   \,\sigma (e^+e^-\to I=1) .
\ee
Diese Relation l\"a\ss t sich vereinfachen, indem man den
hadronischen Wirkungsquerschnitt auf den rein     
elektromagnetischen Prozess $e^+e^-\to\mu^+\mu^-$
normiert
\be
\label{rhov}
 \rho_V(s) \equiv \frac{1}{\pi}{\rm Im}\Pi^V(s) =
       \frac{1}{12\pi^2}\, R^{I=1}(s)\, ,
\ee
wobei $R^{I=1}$ das Verh\"altnis $\sigma (e^+e^-\to I=1)/
\sigma (e^+e^-\to\mu^+\mu^-)$ bezeichnet. Dieser Quotient
ist an einer Reihe von Speicherringen mit gro\ss er 
Genauigkeit vermessen worden.

Zwischen der Schwelle $\sqrt s=2m_\pi$ und $\sqrt s=1$ wird
$R^{I=1}$ von der resonanten $\pi^+\pi^-$-Produktion dominiert.
In diesem Bereich l\"a\ss t sich die Spektralfunktion mit Hilfe
der Beziehung 
\be
 R^{I=1}(s) = \frac{1}{4} |F_\pi (s)|^2 \left( 
       1-\frac{4m_\pi^2}{s} \right)^{3/2}
\ee       
aus dem elektromagnetischen Formfaktor des Pions extrahieren. 
Pr\"azise Ergebnisse f\"ur den Formfaktor im zeitartigen Bereich 
$\sqrt s<1$ GeV liegen aus Orsay vor \cite{Que78}. Die gewonnenen 
Daten werden gew\"ohnlich mit Hilfe eines Gounaris-Sakurai Fits
beschrieben \cite{Hoe83}. Dieses Modell basiert auf einer 
Parametrisierung der $\pi\pi$-Endzustandswechselwirkung, die
f\"ur die analytische Struktur des Pionformfaktors verantwortlich ist.
F\"ur unsere Zwecke gen\"ugt bereits eine noch einfachere 
Beschreibung der Daten, die in Anhang E skizziert ist.
\begin{figure}
\caption{Verh\"altnis $R^{I=1}$ als Funktion der Schwerpunktsenergie. 
Die durchgezogene Linie stellt die von uns verwendete 
Parametrisierung dar.}
\vspace{8cm}
\end{figure}

Im Bereich zwischen  $\sqrt s=1$ GeV und $\sqrt s=2$ GeV ist 
$R^{I=1}$ durch eine Reihe breiter Resonanzen bestimmt. Bis
$\sqrt s=1.4$ GeV verwenden wir die Novosibirsk-Daten \cite{Sid76}
f\"ur die Kan\"ale $e^+e^-\to \pi^+\pi^-\pi^+\pi^-$ und $e^+e^-\to 
\pi^+\pi^-\pi^0\pi^0$. Jenseits von $\sqrt s=1.4$ GeV benutzen wir 
Ergebnisse aus Frascati \cite{Bac79} f\"ur die Reaktion $e^+e^-\to
(\ge 3\pi )$. In beiden F\"allen addieren wir den Kanal $e^+e^-\to\pi^+\pi^-$
aus den Orsay-Daten bei h\"oheren Energien \cite{Bis89}. Das Resultat 
zeigt eine recht deutliche Struktur bei der Schwerpunktsenergie 
1.6 GeV. In dieser \"Uberh\"ohung lassen sich mit Hilfe pr\"aziser
Daten \"uber den Pionformfaktor in dem angesprochenen Bereich die Beitr\"age
zweier \"uberlappender Resonanzen, $\rho(1450)$ und $\rho(1700)$, 
isolieren.  

Bei noch h\"oheren Energien zeigt der totalen Wirkungsquerschnitt 
$\sigma (e^+e^-\to \rm hadr.)$ vor allem die Schwellen f\"ur schwere
Quarkproduktion sowie die Beitr\"age der Vektormesonen 
$J/\psi,\psi ',\ldots$. Es wird allerdings zunehmend schwieriger, den 
$I=1$ Anteil des Wirkungsquerschnitts zu  identifizieren. Wir 
benutzen daher f\"ur $\sqrt s>2$ GeV das st\"orungstheoretische 
Resultat 
\be
 R^{I=1}(s)= \frac{3}{2}\left( 1+\frac{\alpha_s}{\pi}
  + {\cal O}(\alpha_s^2) \right) .
\ee
Die verwendeten Daten f\"ur $R^{I=1}(s)$ im Bereich bis $\sqrt s=2$
GeV sowie die beschriebene Parametrisierung finden sich  in Abbildung 1.
F\"ur $\sqrt s<1$ GeV liefern diese Resultate eine sehr genaue 
Bestimmung der Spektralfunktion. F\"ur h\"ohere Energien sind die
Daten allerdings mit einer Reihe von Unsicherheiten behaftet. So
\begin{itemize}
\item{beinhalten die Daten im Bereich $\sqrt s>1.4$ GeV auch den 
Kanal $\pi^+\pi^-\pi^0$, der im Wesentlichen zum Isospin $I=0$
zu rechnen ist.}
\item{ist es nicht ohne weiteres m\"oglich, den nichtresonanten 
$I=0$ Untergrund aus den Kan\"alen $\pi^+\pi^-\pi^+\pi^-$ und 
$\pi^+\pi^-\pi^0\pi^0$ abzutrennen.}
\item{fehlen in unserer Analyse alle Kan\"ale, deren Endzust\"ande
 Kaonen oder Etamesonen enthalten.}
\item{zeigt der totale Wirkungsquerschnitt f\"ur $\sqrt s<4$ GeV
eine Reihe von Strukturen, die auf $I=1$-Resonanzen zur\"uckzuf\"uhren
sein k\"onnten.}
\end{itemize}
W\"ahrend sich diese Unsicherheiten im Bereich bis $\sqrt s=2$ GeV 
in der Gr\"o\ss enordnung der experimentellen Fehler bewegen, ist 
das Fehlen isospinseparierter Daten in der Region jenseits von
$\sqrt s=2$ GeV durchaus problematisch f\"ur die Analyse der
Summenregeln.

\section{Bestimmung der Axialvektorspektralfunktion}
Die Spektralfunktion im Axialvektorkanal l\"a\ss t sich auf Grund der
Tatsache bestimmen, da\ss\ der Zerfall eines schweren Leptons $\tau$ 
in nichtseltsame Hadronen durch die $V\!-\!A$ Wechselwirkung
\be
\label{fermi}
 {\cal L} = -\frac{G}{\sqrt 2} \cos\theta_c \, 
    \bar u_\nu\gamma^\mu (1-\gamma_5) u_\tau\,
    (V_\mu^{1+i2}-A_\mu^{1+i2})
\ee
vermittelt wird. Dabei bezeichnet $G=1.16637\cdot 10^{-5}\,{\rm GeV}^{-2}$
die Fermikonstante und $\cos\theta_c=0.9744$ den Cosinus des Cabbibowinkels.
Mit der Wechselwirkung (\ref{fermi}) ergibt sich die partielle
Zerfallsbreite des $\tau$-Leptons in Hadronen mit der
invarianten Masse $q^2$ \cite{Tsa71,Oku82}      
\beq
\label{tauwid}
 \frac{d\Gamma (\tau\to^{\;s=0}\!{\rm hadr.}+\nu )}{dq^2} &=&
 \frac{G^2\cos^2\theta_c}{8\pi m_\tau^3} (m_\tau^2-q^2)^2
  \Big\{   m_\tau^2 \rho_A^{||} (q^2)\\
 & & \hspace{2.2cm}\mbox{}  +
 (m_\tau^2+2q^2)(\rho_V(q^2)+\rho_A(q^2) ) \Big\} \, .
   \nonumber
\eeq
Dieses Resultat bestimmt die Spektralfunktionen von der Schwelle
bis hinauf zur Masse des $\tau$-Leptons, $m_\tau=1.784$ GeV. In der 
Praxis ist  allerdings die Statistik auf Grund des kleiner werdenden 
Phasenraums bereits f\"ur invariante Massen $\sqrt{q^2}>1.5$ GeV 
unzureichend.\\

Um die verschiedenen Spektralfunktionen zu separieren, mu\ss\
man die m\"oglichen Endzust\"ande im $\tau$-Zerfall untersuchen. 
Ber\"ucksichtigt  man nur den Pionbeitrag $\rho_A^{||}(s)= 
f_\pi^2 \delta (s-m_\pi^2)$ in der longitudinalen
Spektralfunktion, so ergibt sich das \"ubliche Resultat f\"ur 
die Zerfallsbreite $\tau\to\nu\pi$  
\be
 \Gamma (\tau\to\nu\pi ) = \frac{G^2\cos^2\theta_c}{8\pi}
    m_\tau^3f_\pi^2 \left( 1-\frac{m_\pi^2}{m_\tau^2} \right)^2 .
\ee
Die n\"achst h\"ohere Anregung mit den Quantenzahlen $J^\pi=0^+$
ist das $\pi '(1300)$, das \"uberwiegend in drei Pionen zerf\"allt.
Diese Resonanz ist im $\tau$-Zerfall noch nicht eindeutig 
identifiziert worden. Mit Hilfe von QCD-Summenregeln f\"ur den 
pseudoskalaren Korrelator gewinnt man jedoch folgende 
Absch\"atzung der Resonanzparameter \cite{GL82}
\be
  r_\pi '= \frac{f_{\pi'}^2m_{\pi'}^4}{f_\pi^2 m_\pi^4}
         \simeq 8 .
\ee
Dieser Wert liefert ein Verzweigungsverh\"altnis $B(\tau\to\nu\pi')
=2\cdot 10^{-5}$, das um vier Gr\"o\ss enordnungen kleiner ist als
das gesamte Verzweigungsverh\"altnis in Zerf\"alle mit drei 
Pionen $B(\tau\to\nu 3\pi)=1.1\cdot 10^{-1}$. Wir haben daher 
die longitudinale Spektralfunktion in unserer Analyse von
Endzust\"anden, die mehr als ein Pion enthalten, vernachl\"assigt.


Die Vektor- und Axialvektoranteile der transversalen 
Spektralfunktion lassen sich in sehr guter N\"aherung trennen, indem man 
$\tau$-Zerf\"alle in eine gerade bzw.~ungerade Anzahl von Pionen 
betrachtet. Zerfallskan\"ale wie $\tau^\pm\to\nu\rho^\pm\to
\nu\pi^\pm\pi^+\pi^-$, die durch den Vektorstrom vermittelt werden,
aber Endzust\"ande mit drei Pionen liefern, haben nur 
ein verschwindend kleines Verzweigungsverh\"altnis. 

Messungen von $\tau$-Zerf\"allen sind an den Speicherringen 
Doris \cite{Alb86} und PEP \cite{Ruc86,Ban87} mit Hilfe
der Reaktion $e^+e^-\to\tau^+\tau^-$ durchgef\"uhrt worden.
Wir verwenden die Daten der Argus-Kollaboration \cite{Alb86}
f\"ur den Zerfallskanal $\tau^\pm\to\nu\pi^\pm\pi^+\pi^-$. 
Es gibt keine experimentellen Informationen \"uber das 
Spektrum im anderen $3\pi$-Kanal $\tau^\pm\to\nu\pi^\pm\pi^0\pi^0$.
Eine Analyse der invarianten Massen im $2\pi$-System f\"ur die
Reaktion $\tau^\pm\to\nu\pi^\pm\pi^+\pi^-$ \cite{Alb86}
zeigt allerdings eine klare Pr\"aferenz f\"ur den resonanten 
Prozess
\be
  \tau^\pm\to \nu a_1^\pm \to \nu\rho^0\pi^\pm \to \nu\pi^+\pi^-\pi^\pm .
\ee
F\"ur diese Zerfallssequenz lassen sich die nicht gemessenen 
partiellen Breiten mit Hilfe der Isospinrelation 
\be
 d\Gamma (\tau^\pm\to\nu\pi^\pm\pi^0\pi^0 ) =
 d\Gamma (\tau^\pm\to\nu\pi^\pm\pi^+\pi^- ) \, .
\ee
absch\"atzen. Diese Absch\"atzung ist konsistent mit den Resultaten der 
MAC-Kollaboration \cite{Ban87} f\"ur die totalen Verzweigungsverh\"altnisse 
$B(\tau^\pm\to\nu\pi^\pm\pi^+\pi^-)=7.0\pm 1.0$ \% und $B(\tau^\pm\to\nu
\pi^\pm\pi^0\pi^0)=8.7\pm 1.5$ \% .
\begin{figure}
\caption{Spektralfunktion $\rho_A(s)$ im Axialvektorkanal.
Die durchgezogene Linie stellt die von uns verwendete 
Parametrisierung dar.}
\vspace{8cm}
\end{figure}

Auch die Massenverteilungen der $5\pi$-Zerf\"alle des $\tau$-Leptons
sind experimentell nicht bestimmt. Das gemessene Verzweigungsverh\"altnis
$B(\tau^-\to 3\pi^-2\pi^+)=6.4\cdot 10^{-4}$ \cite{Alb88} ist jedoch
au\ss erordentlich klein und es gibt theoretische Hinweise, da\ss\ diese
Aussage auch auf die anderen $5\pi$-Zerf\"alle zutrifft \cite{GR85}.
Wir haben daher diese Kan\"ale in unserer Analyse vollst\"andig 
vernachl\"assigt. Die resultierende Spektralfunktion findet sich in 
Abbildung 2. Die gezeigte Parametrisierung besteht aus einer 
Breit-Wigner Funktion f\"ur den resonanten Teil, erg\"anzt durch 
ein Polynom zur Beschreibung des Untergrunds. 

Dieser Fit liefert nebenbei auch eine einfache Bestimmung der 
Parameter der $a_1$-Resonanz. Wir finden $m_{a_1}=1169$ MeV und
$\Gamma_{a_1}=552$ MeV, in guter \"Ubereinstimmung mit den 
Ergebnissen anderer $\tau$-Zerfallsexperimente, aber in deutlichem
Widerspruch zu den Standardwerten aus rein hadronischen Reaktionen,
$m_{a_1}=1260\pm 30$ MeV und $\Gamma_{a_1}=316\pm 45$ MeV \cite{PDG90}.
Diese Diskrepanz hat eine Reihe von Neuanalysen der zur Verf\"ugung 
stehenden  Daten angeregt \cite{Bow86,VIO90}. Diese Untersuchungen
zeigen, da\ss\ sich die Bestimmung der $a_1$-Masse im $\tau$-Zerfall
mit den Resulaten hadronischer Experimente in Einklang bringen
l\"a\ss t, wenn man energieabh\"angige Kopplungen f\"ur den Zerfall
$a_1\to 3\pi$ verwendet. Dagegen bleibt die sehr gro\ss e Breite
des $a_1$ im $\tau$-Zerfall im Widerspruch zu rein hadronischen
Daten.
   
\begin{figure}
\caption{Test der Weinberg-Summenregeln und der DMO-Formel
in einem endlichen Energieintervall. Die Funktionen 
$\Delta_1(t_c),\Delta_2(t_c)$ und $\Delta m_\pi (t_c)$ sind
im Text definiert. }
\vspace{18cm}
\end{figure}
Wir haben die bestimmten Spektralfunktionen einer Reihe von 
Konsistenztests unterzogen. Zum einen kann man die Ergebnisse 
aus dem Vektorkanal verwenden, um die entsprechenden 
Verzweigungsverh\"altnisse im $\tau$-Zerfall zu bestimmen. 
Wir finden 
\beq
   B(\tau^\pm\to\nu\rho^\pm) &=& 21.8 \,\%   \\
   B(\tau^\pm\to\nu 2n\pi  ) &=& 29.8 \,\%
\eeq
in guter \"Ubereinstimmung mit den experimentellen Daten \cite{PDG90}.
Dar\"uber hinaus kann man die G\"ultigkeit der Weinberg-Summenregeln
\cite{Wei67,PS87} in einem endlichen Intervall untersuchen. In 
Abbildung 3 zeigen wir die Funktionen 
\beq
 \Delta_1(t_c) &=& \int_0^{t_c} \big( \rho_V(s)-\rho_A(s)\big)\, ds \\
 \Delta_2(t_c) &=& \int_0^{t_c} \big( \rho_V(s)-\rho_A(s)\big)\,s\, ds  
\eeq
sowie die elektromagnetische Massendifferenz der Pionen, berechnet 
mit Hilfe der Stromalgebraformel von Das, Matur und Okubo 
\be
 \Delta m_\pi (t_c) = \frac{3\alpha}{8\pi m_\pi^2 f_\pi^2}
    \int_0^{t_c} \big( \rho_V(s) -\rho_A(s) \big) \ln (s)\,ds
\ee
Die Ergebnisse zeigen, da\ss\ die St\"arke der Axialvektorspektralfunktion
in dem gemessenen Bereich nicht ausreicht, um die Weinberg-Summenregeln 
zu saturieren. Dies mag ein Hinweis auf die Tatsache sein, da\ss\ das
Spektrum im Axialvektorkanal auch jenseits der Masse des $\tau$-Leptons 
noch wichtige Strukturen zeigt. Auch die Pionmassendifferenz ist 
optimal f\"ur relativ kleine $\sqrt {t_c} = 1.4$ GeV, w\"ahrend das
Ergebnis f\"ur h\"ohere Cutoffs stark von $t_c$ abh\"angt und  
im Vergleich mit dem experimentellen Wert $\Delta m_\pi = 4.6$
zu klein wird.

\section{Borelquotient im Vektorkanal}
Wir kommen nun zur Bestimmung der Vakuumparameter mit Hilfe einer
Analyse der QCD-Summenregeln im Vektorkanal. Zu diesem Zweck
betrachten wir den Quotienten der Borelmomente der Spektralfunktion
\be
\label{bratio}
 R^V(\tau) \equiv \frac{\displaystyle \frac{1}{\pi}\int_{s_0}^{\infty}
    {\rm Im}\Pi^V(s) e^{-s\tau} s\, ds }
  {\displaystyle \frac{1}{\pi}\int_{s_0}^{\infty}
    {\rm Im}\Pi^V(s) e^{-s\tau}\, ds }
\ee
Die Operatorproduktentwicklung liefert f\"ur diese Gr\"o\ss e
die theoretische Vorhersage
\be
\label{operatio}
 R^V(\tau) = \frac{1}{\tau}\, \big\{ 1+c_4\tau^2+c_6\tau^3+
    c_8\tau^4 + \ldots \big\}
\ee
wobei wir die Koeffizienten $c_i$ in (\ref{defcor}) definiert haben.
Der Quotient (\ref{bratio}) hat gegen\"uber den individuellen 
Momenten der boreltransformierten Summenregel den Vorzug, da\ss\
in der OPE keine perturbativen Korrekturen zum Einheitsoperator
auftreten. Diese Korrekturen liefern in den einzelnen 
Momenten Beitr\"age von derselben Gr\"o\ss enordnung wie die
f\"uhrenden Kondensate, k\"urzen sich in dem Ver\"altnis 
(\ref{bratio}) aber heraus. Das theoretische Resultat (\ref{operatio})
besitzt daher eine nur sehr geringe Sensitivit\"at auf den 
Wert des Skalenparameters und die Gr\"o\ss e von perturbativen
Korrekturen h\"oherer Ordnung.

\begin{figure}
\caption{OPE-Parametrisierung des experimentellen Borelquotienten
nach Subtraktion des perturbativen Beitrags $1/\tau$.}
\vspace{7cm}
\end{figure}
\begin{figure}
\caption{$\chi^2$-Contour f\"ur die Bestimmung der Koeffizienten
$c_4$ und $c_6$ aus dem Borelverh\"altnis im Vektorkanal.}
\vspace{7cm}
\end{figure}
Mit Hilfe der im letzten Abschnitt bestimmten Spektralfunktionen 
l\"a\ss t sich der Borelquotient f\"ur beliebige Werte des 
Parameters $\tau$ bestimmen. Betrachtet man die Funktion
\be
\label{linratio}
 Y^V(\tau) \equiv R^V(\tau)-\frac{1}{\tau}
           = c_4 \tau + c_6\tau^2 + c_8 \tau^3 + \ldots
\ee
so reduziert sich die Bestimmung der Koeffizienten $c_i$
auf eine einfache Polynomregression. Um die Unsicherheiten
in der Bestimmung der Spektralfunktion zu ber\"ucksichtigen, haben 
wir das Borelverh\"altnis f\"ur verschiedene Parametrisierungen der 
Daten im Bereich der experimentellen Fehler berechnet. 
Die resultierende Variation der Funktion $Y^V(\tau)$ ist 
in Abbildung 6.4 dargestellt. Die Ergebnisse zeigen, da\ss\ der
relative Fehler nach Abzug des st\"orungstheoretischen 
Anteils $1/\tau$ recht erheblich ist. Man erkennt allerdings
auch, da\ss\ sich systematische Fehler in der Spektralfunktion
f\"ur Werte des Borelaparameters $\tau\simeq 0.85\,{\rm GeV}^{-2}$
praktisch nicht auf das Borelverh\"altnis $R^V(\tau)$
auswirken. Diese Tatsache ist ein weiterer Vorzug des
Quotienten $R^V(\tau)$ gegen\"uber den individuellen
Momenten der Boreltransformierten Summenregel. 

Zum Vergleich zeigen wir in Abbildung 6.4 auch das Resultat f\"ur
$Y^V(\tau)$, das sich unter Verwendung der einfachen Parametrisierung
(\ref{zerow}) ergibt. Das Ergebnis zeigt ein qualitativ \"ahnliches
Verhalten wie die experimentellen Daten, liegt aber eindeutig nicht 
innerhalb der Fehlergrenzen. 

Sowohl die experimentellen Resultate als auch die Parametrisierung 
(\ref{zerow}) haben die Eigenschaft, da\ss\ $Y^V(\tau)$ f\"ur 
$\tau\to 0$ gegen Null strebt. Entwickelt man den Borelquotienten 
in eine Potenzreihe in $\tau$, so zeigt sich, da\ss\ diese Tatsache
\"aquivalent mit der G\"ultigkeit der ersten FESR-Bedingung (\ref{fesr1})
ist. W\"ahrend wir diese Bedingung f\"ur die Parametrisierung der 
Daten durch die Wahl der Kontinuumsschwelle ber\"ucksichtigt haben,
liefert die Forderung $Y^V(\tau)\stackrel{\tau\to 0}{\rightarrow}0$
im Fall der experimentellen Spektralfunktion   einen wichtigen Test
f\"ur die Konsistenz der Daten mit der Dualit\"atsforderung. 

Die Operatorproduktentwicklung liefert nur eine asymptotische 
Darstellung der Funktion $Y^V(\tau)$ f\"ur kleine Werte 
des Borelparameters $\tau$. Es ist daher von gro\ss er 
praktischer Bedeutung, das Intervall $[\tau_{min},\tau_{max}]$
zu bestimmen, in dem man \"Ubereinstimmung zwischen der experimentellen 
und theoretischen Seite der  Summenregel fordert. Asymptotische
Freiheit garantiert die G\"ultigkeit der Summenregel im Grenzfall 
$\tau\to 0$. Da die Daten in diesem Bereich auf Grund der gro\ss en 
Fehler in jedem Fall nicht sehr stark gewichtet werden, setzen wir
die untere Grenze des Intervalls $\tau_{min}=0$. Die obere Grenze
ergibt sich aus der Forderung, da\ss\ sowohl die statistischen
Fehler auf Grund der Unsicherheit der Daten als auch die theoretischen 
Fehler, hervorgerufen durch das Abbrechen der OPE minimal sind.  

Um das g\"unstige Verhalten des Borelquotienten bez\"uglich 
systematischer Fehler in der Spektralfunktion zu nutzen, sollte
$\tau_{max}>0.8\,{\rm GeV}^{-2}$ gew\"ahlt werden. Eine 
Absch\"atzung des theoretischen Fehlers ergibt sich, indem man
die Koeffizienten $c_4$ und $c_6$ f\"ur verschiedene vogegebene
Werte des vernachl\"assigten Korrekturglieds $c_8$ bestimmt. Zu 
diesem Zweck haben wir eine obere Grenze f\"ur $|c|_8$ aus der 
Absch\"atzung \cite{Cin91}
\be
  |c_8|_{max} m_{sc}^8 = {\rm max} \left\{ |c_4|m_{sc}^4,
    |c_6|m_{sc}^6 \right\}
\ee
gewonnen. Dabei bezeichnet $m_{sc}\simeq 1$ GeV eine f\"ur das
Zusammenbrechen der OPE charakteristische  Skala. Verwendet
man die von SVZ \cite{SVZ79} bevorzugten Werte, so ergibt sich
$|c_8|_{max}\simeq 0.09\,{\rm GeV}^8$. Dieses Resultat liefert 
sicher nur eine sehr grobe Absch\"atzung der vernachl\"assigten
Beitr\"age in der OPE, andere Werte von $|c_8|_{max}$ f\"uhren
aber zu ganz \"ahnlichen qualitativen Schlu\ss folgerungen.   
\begin{table}
\caption{Bestimmung der Koeffizienten $c_i$ aus dem Borelquotienten
im Vektorkanal. Angegeben sind die Koeffizienten $c_4$ und $c_6$
sowie Absch\"atzungen der experimentellen und theoretischen
Unsicherheiten f\"ur verschiedene Werte des maximalen Borelparameters.}  
\begin{center}
\begin{tabular}{|c||c|c|c||c|c|c||}\hline
 $\tau_m\,[\gev^{-2}]$  &  $c_4\,[\gev^4]$  & $\Delta^{exp} c_4$ &
       $\Delta^{th} c_4$ &  $c_6\,[\gev^6]$  
	     & $\Delta^{exp} c_6$ & $\Delta^{th} c_6$  \\ \hline\hline
    0.8   &$-0.264$ & $\pm 0.150$        & $\pm 0.020$       &
             0.367  & $\pm 0.201$        & $\pm 0.061$   \\
    0.9   &$-0.099$ & $\pm 0.046$        & $\pm 0.030$       &
             0.148  & $\pm 0.053$        & $\pm 0.074$   \\	     
    1.0   &$-0.073$ & $\pm 0.032$        & $\pm 0.034$       &
             0.119  & $\pm 0.036$        & $\pm 0.079$   \\	     
    1.1   &$-0.054$ & $\pm 0.022$        & $\pm 0.039$       &
             0.096  & $\pm 0.025$        & $\pm 0.083$   \\	     
    1.2   &$-0.042$ & $\pm 0.016$        & $\pm 0.041$       &
             0.083  & $\pm 0.019$        & $\pm 0.086$   \\ \hline
\end{tabular}
\end{center}
\end{table}

In Tabelle 6.1 finden sich die Ergebnisse f\"ur die
Koeffizienten $c_4$ und $c_6$ sowie unsere Absch\"atzung der
theoretischen und experimentellen Fehler f\"ur verschiedene
Werte des maximalen Borelparameters $\tau_{max}$. F\"ur
$\tau_{max}>1\,{\rm GeV}^{-2}$ beginnt der theoretische Fehler
in der Bestimmung der Koeffizienten gegen\"uber dem experimentellen
zu dominieren. Wir betrachten daher diesen Wert als optimalen
Kompromi\ss\ zwischen den Forderungen nach ausreichender 
Bestimmtheit des Fits und Kontrolle \"uber den Abbruchfehler
in der OPE. Das entsprechende Resultat
\beq
\label{fitresult}
  c_4 &=&-0.073\pm 0.032\,{\rm GeV}^{4} \\   
  c_6 &=& \spm 0.119\pm 0.036\,{\rm GeV}^{6}
\eeq
ist konsitsent mit der Absch\"atzung von SVZ, beinhaltet allerdings
gro\ss e experimentelle Fehler. In Abbildung 6.4 zeigen wir die
$\chi^2$-Contour f\"ur diesen Fit. Man erkennt eine sehr starke 
Korrelation der beiden Koeffizienten $c_4$ und $c_6$. Das 
Verh\"altnis dieser beiden Gr\"o\ss en ist daher sehr viel besser
bestimmt als ihre absoluten Werte.   Diesen Effekt beobachtet
man auch bei der Variation der Ergebnisse mit dem maximalen 
Borelparameter, siehe Tabelle 6.1. Obwohl $c_4$ und $c_6$ eine
sehr starke Abh\"angigkeit von $\tau_{max}$ aufweisen, ist ihr 
Verh\"altnis praktisch konstant. Fixiert man $c_4$ durch die 
Resultate aus der Analyse des Charmoniumsystems, so ergibt sich
$c_6=0.11\,{\rm GeV}^{6}$, in \"Ubereinstimmung mit dem 
Ergebnis (\ref{fitresult}).

Alternativ zur Analyse der Spektralfunktion mit Hilfe des
Borelquotienten lassen sich die Koeffizienten $c_4$ und $c_6$ 
auch direkt aus den FESR-Gleichungen (\ref{fesr2},\ref{fesr3}) bestimmen
\cite{BDL88}. Betrachtet man zun\"achst nur die niedrigste
Bedingung
\be
\label{lfesr}
 t_cF_2(t_c) = 8\pi \int_{s_0}^{t_c} {\rm Im}\Pi^V(s) \, ds
\ee
f\"ur $\sqrt{t_c}=2$ GeV, so ergibt sich mit unseren Daten eine
Diskrepanz von $0.43\,{\rm GeV}^2$ zwischen der linken und 
rechten Seite der Summenregel. Diese Differenz ist hinreichen klein,
um zu gew\"ahrleisten, da\ss\ $Y^V(\tau)$ im Bereich der Fehler 
f\"ur $\tau\to 0$ gegen Null strebt. 

Die h\"oheren FESR-Gleichungen sind allerdings au\ss erordentlich 
sensitiv auf die G\"ultigkeit der Bedingung (\ref{lfesr}). 
Verwendet man (\ref{fesr2}) und (\ref{fesr2}) f\"ur $\sqrt{t_c}=2$
GeV, so findet man $c_4= 1.76\,\gev^4$ und $c_6= -4.01\,\gev^6$, 
in deutlichem Widerspruch zu (\ref{fitresult}). Fixiert man dagegen 
$t_c$ durch die Forderung (\ref{lfesr}), so ergibt sich mit $\sqrt{t_c}=
1.46\,{\rm GeV}$ eine Schwellenenergie in der N\"ahe der $\rho'$-Resonanz.
Dieser Wert von $\sqrt{t_c}$ liefert die Kondensate 
\be
c_4=-0.33\,\gev^4 \hspace{1cm} c_6=0.68\,\gev^6 ,
\ee
in deutlich besserer \"Ubereinstimmung mit dem Resultat
der Analyse des Borelquotieten. Insbesondere entsrpricht auch dieses
Ergebnis unserer Beobachtung, da\ss\ die verwendeten Summenregeln
im wesentlichen das Verh\"altnis $c_4/c_6$ fixieren.

\section{Borelquotient im Axialvektorkanal}
Nachdem wir im letzten Abschnitt eine Neuanalyse der experimentellen
Spektralfunktion im Vektorkanal vorgestellt haben, wollen wir uns 
nun unserem eigentlichen Thema zuwenden und untersuchen, in wie weit
auch die Daten im Axialvektorkanal mit QCD-Summenregeln vertr\"aglich
sind. Insbesondere soll die Frage untersucht werden, in wie weit die
Daten konsistent mit der Faktorisierungshypothese $\xi^V=\xi^A$ sind.

Zu diesem Zweck betrachten wir den Quotienten der Borelmomente der 
$q_\mu q_\nu$-Struktur in der Korrelationsfunktion
\be
\label{ra2}
 R^{A_2}(\tau) \equiv \frac{\displaystyle \int_{s_0}^{\infty}
    \rho_A (s) e^{-s\tau} s\, ds }
  {\displaystyle \int_{s_0}^{\infty}
    \rho_A (s) e^{-s\tau}\, ds \,+\, f_\pi^2}\; ,
\ee
wobei wir die endliche Masse des Pions vernachl\"assigt haben. Diese
N\"aherung ist konsistent mit unserem Vorgehensweise im perturbativen
Sektor, wo wir die Effekte der Strommassen nicht ber\"ucksichtigt 
haben. Die Operatorproduktentwicklung liefert die Vorhersage 
\be
 Y^{A_2}(\tau) \equiv R^{A_2}(\tau)-\frac{1}{\tau} =
     c_4\tau - \frac{11}{7}\frac{\xi^A}{\xi^V} c_6\tau^2 + \ldots\, ,
\ee
so da\ss\ sich die Koeffizienten unter der Annahme $\xi^A=\xi^V$ ganz
ananlog zu der Analyse im Vektorkanal bestimmen lassen.
\begin{figure}
\caption{OPE-Parametrisierung des experimentellen Borelquotienten
f\"ur die $q_mu q_\nu$-Struktur im Axialvektorkanal 
nach Subtraktion des perturbativen Beitrags $1/\tau$.}
\vspace{7cm}
\end{figure}
       
Das Pion liefert einen sehr genau bestimmten Beitrag zum ersten Moment
der Spektralfunktion, w\"ahrend der restliche Teil des Spektrums mit
erheblichen Fehlern behaftet ist. \"Ahnlich wie im Vektorkanal, wo
wir einen \"ahnlichen Effekt auf Grund des Zusammenwirkens der $\rho$
und $\rho'$-Resonanz beobachtet haben, existiert daher ein Bereich 
von Werten des Parameters $\tau$, in dem der Borelquotient $R^{A_2}
(\tau)$ kaum von den experimentellen Unsicherheiten beeinflu\ss t ist. 

Wie im letzten Abschnitt bestimmen wir die Koeffizienten durch Anpassung an 
die Daten im Intervall $[0,\tau_{max}]$ und studieren die Abh\"angigkeit
der Ergebnisse von $\tau_{max}$. Die Werte von $c_4$ und $c_6$, die
sich unter Verwendung der Faktorisierungshypothese ergeben, finden sich 
in Tabelle 6.2. Erneut finden wir keine Stabilit\"atsregion f\"ur
den maximalen Borelparameter. Fixiert man $\tau_{max}$ mit Hilfe 
des theoretischen Werts f\"ur $c_4$, so ergibt sich $c_6=0.39\,
{\rm GeV}^6$. Diese Resultat entspricht einer Verletzung der 
exakten Faktorisierung $\xi^A/\xi^V=1$ um den Betrag  $3.2\pm 1.9$, 
wobei der Fehler so gro\ss\ ist, da\ss\ auch $\xi^A/\xi^V=1$ nicht 
ausgeschlossen ist.      
\begin{table}
\caption{Bestimmung der Koeffizienten $c_i$ aus dem Borelquotienten
f\"ur die $q_\mu q_\nu$-Struktur im Axialvektorkanal. Angegeben sind 
die Koeffizienten $c_4$ und $c_6$ sowie die Absch\"atzung der 
experimentellen Unsicherheiten f\"ur verschiedene Werte des maximalen 
Borelparameters.}  
\begin{center}
\begin{tabular}{|c||c|c||c|c||}\hline
 $\tau_m\,[\gev^{-2}]$ &  $c_4\,[\gev^4]$  & $\Delta^{exp} c_4$ &
             $c_6\,[\gev^6]$  & $\Delta^{exp} c_6$   \\ \hline\hline
    0.80  &$ \spm 0.042$ & $\pm 0.077$        &
             0.559  & $\pm 0.169$          \\
    0.85  &$-0.037$ & $\pm 0.063$        & 
             0.447  & $\pm 0.137$          \\	     
    0.90  &$-0.101$ & $\pm 0.052$        &
             0.356  & $\pm 0.112$          \\	     
    0.95  &$-0.154$ & $\pm 0.044$        &
             0.283  & $\pm 0.093$          \\	     
    1.00  &$-0.198$ & $\pm 0.037$        & 
             0.221  & $\pm 0.076$          \\ \hline
\end{tabular}
\end{center}
\end{table}

Um die Zuverl\"assigkeit dieser Resultate einsch\"atzen zu k\"onnen,
mu\ss\ man allerdings beachten, da\ss\ unter Verwendung der beschriebenen
Spektralfunktion die niedrigste FESR-Bedingung
\be
\label{lfesra}
 t_c F_2(t_c) = 8\pi^2 \int_{s_0}^{t_c} \rho_a (s)\, ds \,
 +\, 8\pi^2f_\pi^2
\ee
deutlich verletzt ist. F\"ur $\sqrt{t_c}=m_\tau$ betr\"agt die
Diskrepanz zwischen der linken und rechten Seite der Summenregel
$0.98\,{\rm GeV}^2$, ein Wert der erheblich gr\"o\ss er ist als der
entsprechende Fehlbetrag im Vektorkanal. Tats\"achlich l\"a\ss t
sich die Bedingung erst f\"ur $\sqrt{t_c}=1.3 \,{\rm GeV}^2$ 
zumindest n\"aherungsweise erf\"ullen.
Dieser Wert ist aber so klein, da\ss\ praktisch der gesamte Resonanzbeitrag 
aus dem Spektrum herausgeschnitten wird.

Die Tatsache, da\ss\ die $a_1$-Resonanz allein die Dualit\"atsbedingung
nicht erf\"ullt ist ein Hinweis darauf, da\ss\ die Verwendung des
perturbative Spektrums f\"ur $\sqrt s>m_\tau$ ein unrealistisches 
Modell der Spektralfunktion bei mittleren Energien liefert. Diese
Schlu\ss folgerung wird durch eine Analyse des Quotienten der 
Borelmomente der $g_{\mu\nu}$-Struktur der Korrelationsfunktion
\be
\label{ra1}
 R^{A_1}(\tau) \equiv \frac{\displaystyle \int_{s_0}^{\infty}
    \rho_A (s) e^{-s\tau} s^2\, ds }
  {\displaystyle \int_{s_0}^{\infty}
    \rho_A (s) e^{-s\tau}s\, ds }
\ee
untermauert. Dieses Verh\"altnis enth\"alt h\"ohere Momente der
Spektralfunktion und ist daher in st\"arkerem Umfang sensitiv 
auf die Form des Spektrums bei Energien jenseits der $a_1$-Masse.
Die Operatorproduktentwicklung liefert die theoretische
Vorhersage 
\be
\label{ya1}
 Y^{A_1}(\tau) \equiv -R^{A_1}(\tau)+\frac{2}{\tau} = \overline{c}_4
     \tau - \frac{22}{7}\frac{\xi^A}{\xi^V} c_6\tau^2 + \ldots\, ,
\ee
wobei $\overline{c}_4=c_4 + 16\pi^2 (m_u+m_d)<\bar qq>=c_4-0.026
\,{\rm GeV}^4$ nur eine kleine Korrektur gegen\"uber dem entsprechenden
Koeffizienten im Vektorkanal enth\"alt.  
\begin{figure}
\caption{OPE-Parametrisierung des experimentellen Borelquotienten
f\"ur die $g_{\mu\nu}$-Struktur im Axialvektorkanal 
nach Subtraktion des perturbativen Beitrags $2/\tau$.}
\vspace{7cm}
\end{figure}

Die Resultate f\"ur $Y^{a_1}(\tau)$, die sich unter Verwendung des
Spektrums aus dem $\tau$-Zerfall ergeben, finden sich in Abbildung
6.7. Auf Grund der ung\"unstigeren Gewichtung des Spektrums wird
der systematische Fehler erst bei relativ gro\ss en Werten des
Borelparameters, $\tau\simeq 1.2\,{\rm GeV}^{-2}$, minimal. Wir finden
allerdings ganz unabh\"angig von dem verwendeten Intervall 
$[\tau_{min}=0,\tau_{max}]$ keine \"Ubereinstimmung der Daten mit 
der QCD-Parametrisierung (\ref{ya1}). Verwendet man $\tau_{max}=1.5
\,{\rm GeV}{-2}$, um den Einflu\ss\ der experimentellen Fehler gering zu
halten, so ergibt sich $c_4=-0.58\pm 0.08\,{\rm GeV}^4$ und
$c_6 = -0.08\pm 0.06\,{\rm GeV}^6$, in klarem Widerspruch  zur 
Analyse der niedrigeren Borelmomente.    

\section{Hinweise auf radiale Anregungen der $a_1$-Resonanz}
Wir haben im letzten Abschnitt das Spektrum im Axialvektorkanal mit 
Hilfe von QCD-Summenregeln untersucht. Unter Verwendung von Summenregeln,
die im wesentlich auf den Pionbeitrag und den niederenergetischen 
Teil des Spektrums sensitiv sind, haben wir \"Ubereinstimmung mit den 
Ergebnissen  aus dem Vektorkanal und der Faktorisierungshypothese 
erzielt. Dagegen zeigen h\"ohere Borelmomente deutliche Abweichungen 
von diesen Vorhersagen. Dar\"uber hinaus ist unser Modell des Spektrums,
in dem jenseits der Masse des $\tau$-Leptons der st\"orungstheoretische
Wert verwendet wird, nicht konsistent mit der asymptotischen 
G\"ultigkeit der Summenregeln. 

Es liegt daher nahe, auch die Beitr\"age h\"oherer Anregungen des 
$a_1$-Mesons zu ber\"ucksichtigen. Die vorhandenen theoretischen 
und experimentellen Informationen \"uber solche Zust\"ande sind 
allerdings sehr fragmentarisch. So enth\"alt die Kompilation der
Particle Data Group \cite{PDG90} neben dem $a_1$ keine weiteren 
Mesonen mit den Quantenzahlen $I^G(J^{PC})=1^-(1^{++})$.

In nichtrelativistischen Konstituendenmodellen des mesonischen 
Spektrums ergeben sich dagegen in nat\"urlicher Weise radiale
Anregungen des $a_1$-Mesons. Ein typisches Beispiel ist das
Potentialmodell von Huber et al. \cite{HMP  }. Diese Autoren 
finden den Grundzustand im $1^{++}$-Kanal bei einer Energie
von 1200 MeV und die erste radiale Anregung bei $m_{a_1'}=
1900$ MeV. Auch die bereits angesprochene gro\ss e Breite der
$a_1$-Resonanz ist als Hinweis auf die Gegenwart h\"oherer 
gedeutet worden. Beschreibt man das Spektrum aus dem $\tau$-Zerfall
mit Hilfe zweier interferierender Resonanzen \cite{IKM89},
so reduziert sich die resultierende $a_1$-Breite auf 
$\Gamma_{a_1}=380\pm 20$ MeV, in \"Ubereinstimmung mit den
Resultaten hadronischer Experimente.  Die entsprechenden 
Parameter der $a_1'$-Resonanz lauten $m_{a_1'}=1550\pm 40$ MeV
und $\Gamma_{a_1'}=525\pm 25$ MeV.

Die Frage, ob die im $\tau$-Zerfall beobachtete Struktur in 
Wahrheit durch mehrere \"uberlappende Resonanzen hervorgerufen
wird, l\"a\ss t sich nat\"urlich nicht mit Hilfe von Summenregeln    
kl\"aren. Wir wollen uns daher im Folgenden auf die Gestalt des 
Spektrums jenseits der Masse des $\tau$-Leptons konzentrieren. 
Zu diesem Zweck parametrisieren wir die Spektralfunktion 
in der Form
\beq
\label{rhoap}
  \rho_A(s) &=& \rho_a^\tau (s) +\frac{1}{4\pi} \Big( 
   \frac{a'm_{a_1'}^2\Gamma_{a_1'}^2}{(s-m_{a_1'}^2)^2
   +m_{a_1'}^2\Gamma_{a_1'}^2} + b\sqrt{s-s_0} \Big)
   \Theta (s_{th}-s)  \nonumber \\[0.1cm]
   & &  \hspace{1.3cm} \mbox{}+\frac{1}{8\pi^2} 
   \Big( 1+\frac{\alpha_s}{\pi} \Big) \Theta (s-s_{th}),
\eeq
wobei $\rho_a^\tau (s)$ die Spektralfunktion aus dem $\tau$-Zerfall 
bezeichnet. Die Gr\"o\ss en $a',m_{a_1'}$ und $\Gamma_{a_1'}$ 
charakterisieren die $a_1'$-Resonanz, w\"ahrend der Term 
$b\sqrt{s-s_0}$ einen langsam wachsenden Untergrund beschreibt. 
Dieser Untergrund geht f\"ur Energien oberhalb der 
Kontinuumsschwelle in das st\"orungstheoretische Resultat 
\"uber.

\begin{figure}
\caption{Vergleich der Spektralfunktionen im Vektor- und 
Axialvektorkanal. Die Axialvektorspektralfunktion enth\"alt 
die im Text bestimmten Beitr\"age der $a_1'$-Resonanz.}
\vspace{7cm}
\end{figure} 
Wir bestimmen die freien Parameter, indem wir in einem gegebenen 
Intervall $[\tau_{min},\tau_{max}]$ die \"Ubereinstimmung der 
beiden Borelquotienten $R^{A_1}(\tau)$ und $R^{A_2}(\tau)$ mit 
den Vorhersagen der OPE optimieren. Zu diesem Zweck minimieren 
wir die Funktion
\be
\chi^2 =\sum_i \left| \frac{ Y^{A_1}(\tau_i)-Y^{A_1}_{th}(\tau_i) }
                    { \Delta Y^{A_1}(\tau_i) } \right|^2 +
               \left| \frac{ Y^{A_2}(\tau_i)-Y^{A_2}_{th}(\tau_i) }
	            { \Delta Y^{A_2}(\tau_i) } \right|^2 
\ee
in dem gegebenen Parameterraum. In der Praxis haben wir dabei die 
Werte von $s_0,s_{th}$ und $\Gamma_{a_1'}$ fixiert, da diese 
Gr\"o\ss en durch die verwendeten Summenregelen nicht sehr stark 
eingeschr\"ankt sind. Mit $s_0=2.5\,\gev$, $s_{th}=4.65\,\gev^2$
und $\Gamma_{a_1'}=0.25\,\gev$ finden wir
\be
\label{ma1p}
m_{a_1'} = 1729^{+11}_{-67}\,{\rm MeV},
\ee
sowie $a'=0.274$ und $b=5.84\cdot 10^{-2}\,\gev^{-1}$. Die angegebenen
Fehler sind ein Ma\ss\ f\"ur die Variation der Ergebnisse bei 
verschiedenen Werten von $s_0,s_{th}$ und $\Gamma_{a_1'}$. 

Wir haben das modifizierte Spektrum analog zur Vorgehensweise in
Abschnitt 6.2 einer Reihe von Konsistenztests unterzogen.
