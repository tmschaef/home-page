\chapter{Niederenergietheoreme zur Pionphotoproduktion}
Nachdem wir uns im letzten Kapitel mit der experimentellen 
Bestimmung der elektrischen Dipolamplitude an der Schwelle
befa\ss t haben, wollen wir uns nun auf die theoretische 
Bestimmung von $E_{0+}$ mit Hilfe von Niederenergietheoremen
konzentrieren. Die spezielle Bedeutung der Photoproduktion
neutraler Pionen ergibt sich dabei aus der Tatsache, da\ss\
die entsprechende Schwellenamplitude in einer hypothetischen Welt
mit masselosen Pionen  verschwinden w\"urde.
Dieser Kanal ist daher besonders sensitiv auf die Rolle der
expliziten chiralen Symmetriebrechung, welche sich in  dem
nur approximativen Charakter des Pions als Goldstoneboson 
widerspiegelt. 

Die physikalische Grundidee der Niederenergietheoreme
({\em engl.} Low Energy Theorem, LET) l\"a\ss t sich besonders
\"ubersichtlich an rein elektromagnetischen Reaktionen
diskutieren. Die Anwendung  von Niederenergietheoremen wird 
in diesem Fall durch zwei physikalische Kriterien kontrolliert. 
Die beiden Forderungen lauten, da\ss\ die Wellenl\"ange des 
Photons gro\ss\ ist im Vergleich zur Ausdehnung des Streuzentrums
w\"ahrend die Energie des Photons klein ist gegen die typische 
Anregungsenergie. Sind diese Voraussetzungen erf\"ullt, so ist das 
Photon nicht in der Lage, die innere Struktur des Targets aufzul\"osen. 
Der differentielle Wirkungsquerschnitt ist daher ausschlie\ss lich durch 
globale elektromagnetische Eigenschaften des Streuzentrums bestimmt.
Betrachtet man die Comptonstreuung niederenergetischer
Photonen an einem hadronischen Target, so folgt aus dieser
\"Uberlegung, da\ss\ der differentielle Wirkungsquerschnitt 
nur von der Gesamtladung $Ze$ abh\"angt und der Targetmasse $M$ 
abh\"angt. Insbesondere ergibt 
sich im Grenzfall $k_\mu=(\omega,\vec{k})\to 0$ die klassische
Thomson-Streuung
\be
\label{thomson}
 \lim_{\omega \to 0} \frac{d\sigma}{d\Omega} =
  \frac{Z^2e^2}{4\pi M^2} (\vec{\epsilon}_1 \cdot\vec{\epsilon}_2)
\ee     
wobei $\vec{\epsilon}_1$ und $\vec{\epsilon}_2$ die 
Polarisationsvektoren der ein- und auslaufenden Photonen
bezeichnen. Low, Gell-Mann und Goldberger \cite{Low54,Low58,GMG54}
konnten dar\"uber hinaus zeigen, da\ss\ auf Grund von 
Eich- und Lorentzinvarianz die Amplitude f\"ur Comptonstreuung 
in Forw\"artsrichtung an einem Spin-1/2 Target mit der Ladung sogar bis
auf Terme linear in der Laborenergie $\omega$ bestimmt ist
\be
 \lim_{\omega \to 0}  T(\omega) =- \frac{e^2}{M}  
  (\vec{\epsilon}_1 \cdot\vec{\epsilon}_2) -i \frac{e^2}{8 M^2}
  \kappa^2 \omega  (\vec{\epsilon}_1 \times\vec{\epsilon}_2)
  \cdot \vec{\sigma} \; .
\ee
Der erste Term beschreibt  die Thomsonamplitude, w\"ahrend der
zweite Term eine Korrektur liefert, die proportional zum
anomalen magnetischen Moment $\kappa$ des Streuzentrums ist.

Die Anwendung von Niederenergietheoremen auf pionische 
Reaktionen wird durch die endliche Masse des Pions 
erschwert. Die Pionmasse setzt eine untere Grenze f\"ur 
die Energie  $\omega_\pi=(\vec{q}^{\, 2} +m_\pi^2)^{1/2}$
des Pions. W\"ahrend daher die Wellenl\"ange beliebig gro\ss\
gemacht werden kann, gibt es eine prinzipielle Schranke 
f\"ur die Energie. Die Anwendbarkeit von Niederenergietheoremen
setzt daher voraus,
da\ss\ die Masse des Pions klein gegen die charakteristische
Energieskala der Reaktion ist. In hadronischen Prozessen ist
eine solche Skala durch die Masse des Nukeons gegeben. 
Das Verh\"altnis $m_\pi/M\simeq 1/7$ ist daher ein nat\"urlicher
Parameter, der die Abweichung der  Amplitude vom unphysikalischen
Grenzfall $q_\mu \to 0$ kontrolliert.

Formal basieren Niederenergietheoreme f\"ur ''weiche`` Pionen auf
der G\"ultigkeit von Stromalgebra, chiraler Symmetrie und
PCAC. Historisch wurden diese Konzepte in den sechziger Jahren 
als Hypothesen \"uber das Transformationsverhalten der in
hadronische Reaktionen eingehenden Str\"ome entwickelt
\cite{AD68,AFF73}. Sie erwiesen sich als au\ss erordentlich
fruchtbar, um die  experimentellen 
Informationen \"uber hadronische Reaktionen zu verstehen. 
Die klassischen Anwendungen liegen im Bereich
der $\pi\pi$- und $\pi N$-Streuung, sowie der Photo- und schwachen
Produktion pseudoskalarer Mesonen. Wichtige Vorhersagen ergeben
sich dar\"uber hinaus f\"ur leptonische und semileptonische Zerf\"alle
stark wechsewirkender Teilchen.    
  
Nach der Gr\"underphase trat die Anwendung von Stromalgebra und
chiraler Symmetrie zun\"achst in den Hintergrund gegen\"uber
der Entwicklung von Quantenchromodynamik als fundamentaler
Eichtheorie der starken Wechselwirkung. In diesem Zusammenhang 
zeigte sich allerdings, da\ss\ die G\"ultigkeit dieser Konzepte eine direkte
Konsequenz von QCD ist. Dar\"uber hinaus liefert die chirale 
Symmetrie  einen wichtigen Zugang zur starken Wechselwirkung
in einem Bereich, in dem die direkte Anwendung von QCD
bislang noch gro\ss en Schwierigkeiten gegen\"uber steht.

\section[Quantenchromodynamik, Stromalgebra \ldots]{Quantenchromodynamik,
 Stromalgebra und chirale Symmetrie}
Die Quantenchromodynamik ist eine nichtabelsche Eichtheorie, 
beschrieben durch die Lagrangedichte
\be
\label{lqcd}
{\cal L}_{QCD} = -\frac{1}{4} F_{\mu\nu}^{\;\;a} F^{\mu\nu\, a}
 + \sum_{j=1}^{n_f} \bar{\psi}^{\alpha}_{j}( i\gamma^{\mu}
 {\cal D}_\mu^{\alpha\beta} - \delta^{\alpha\beta} m_j )
 \psi_j^\beta \; ,
\ee 
wobei die  Summation \"uber $n_f$ verschiedene Quarkarten (flavors)
ausgef\"uhrt wird. W\"ahrend sich die Quarkspinoren $\psi^{\alpha}$ nach
der fundamentalen Darstellung der Eichgruppe $SU(3)$ transformieren,
sind die Gluonen $A_\mu^{a}$ Vektorfelder und tragen einen Index 
in der adjungierten Darstellung der $SU(3)$. Die zugeh\"origen
Elemente der Lie-Algebra sind
\be
 A_\mu(x) = A_\mu^{a}(x)\frac{\lambda^{a}}{2}\; ,
\ee
wobei $\lambda^{a}$ die Generatoren der Algebra bezeichnet.
Sie erf\"ullen die fundamentalen Vertauschungsrelationen
\be
 [\lambda^{a},\lambda^b] = 2if^{abc}\lambda^c
\ee
und k\"onnen durch
\be
 Tr \,\lambda^{a}\lambda^b = 2\delta^{ab}
\ee
normiert werden. Dabei bezeichnet $f^{abc}$ die Strukturkonstanten
von $SU(3)$. Der Yang-Mills-Feldst\"arketensor ist durch
\be
\label{fmunu}
 F_{\mu\nu}^{\;\; a} = \partial_\mu A_\nu^{a} -\partial_\nu A_\mu^{a} 
 + g f^{abc} A_\mu^b A_\nu^c
\ee
gegeben, w\"ahrend die kovariante Ableitung durch
\be
\label{kovd}
 {\cal D}_\mu^{\alpha\beta} = \delta^{\alpha\beta}\partial_\mu
  + i\frac{g}{2} (\lambda^{a})^{\alpha\beta} A_\mu^{a}
\ee
definiert ist. Entscheidend f\"ur die nichtabelsche Eichsymmetrie
ist die Tatsache, da\ss\ die Quark-Gluon-Wechselwirkung durch dieselbe
Kopplungskonstante $g$ wie die gluonische Selbstwechselwirkung 
bestimmt ist.
   
Neben der lokalen $SU(3)$ Eichsymmetrie besitzt die $QCD$
Lagrangedichte noch eine Reihe kontinuierlicher globaler Symmetrien.
So ist ${\cal L}_{QCD}$ invariant unter der globalen 
$U(1)_V$ Transformation
\be
\label{uone}
\psi_j(x) \to \exp (-i\theta) \psi_j (x) \, .
\ee
Der dazugeh\"orige erhaltene Vektorstrom ist der Baryonenstrom
\be
 j_\mu(x) = \sum_{j=1}^{n_f} \bar{\psi}_j \gamma_\mu \psi_j
\ee
mit der baryonischen Ladung 
\be
B=\int d^3x\, j_0(\vec{x},t)\, .
\ee
F\"ur masselose Quarks ist ${\cal L}_{QCD}$ ebenfalls invariant
unter der axialen $U(1)_A$ Transformation
\be
\label{uaone}
\psi_j(x) \to \exp (-i\theta\gamma_5) \psi_j (x) \; .
\ee
Der entsprechende Strom besitzt allerdings eine anomale Divergenz
\be
\label{axanom}
\partial^\mu j_{\mu\, 5}(x) = \frac{g^2}{4\pi}\frac{n_f}{8}
 \epsilon^{\mu\nu\rho\sigma} F_{\mu\nu}^{\;\; a}F_{\rho\sigma}^{\;\; a}
 \; ,
\ee
so da\ss\  die axiale Ladung $Q_5=\int d^3x\, j_{0\,5}(\vec{x},t)$
nur in  Abwesenheit instantonartiger L\"osungen erhalten ist.

Vernachl\"assigt man die Quarkmassen, so ist ${\cal L}_{QCD}$ auch
invariant unter Skalentransformationen. Diese Symmetrie wird in der
quantisierten Theorie durch die Notwendigkeit der Renormierung
gebrochen. Dabei tritt ein dimensionsbehafteter Parameter, der
QCD Skalenparameter $\Lambda_{\mini QCD}$, auf. Sein Wert ist unter anderem 
aus der Skalenbrechung in der tief-inelastischen Lepton-Nukleon Streung zu
$\Lambda_{\mini QCD}^{\mini\overline{MS}} =230\pm 80$ MeV bestimmt worden
\cite{PDG90}. Der Index ${\kl \overline{MS}}$ bezeichnet eine spezielle
Renormierungsvorschrift, die sogenannte modifizierte minimale 
Subtraktion.

Ebenfalls auf Grund der Renormierung sind  auch die Werte der 
Quarkmassen von der experimentellen Skala abh\"angig.
Im Bereich typischer hadronischer Prozesse lassen sich 
die Stromquarkmassen mit Hilfe von QCD-Summenregeln 
extrahieren \cite{GL82}. Bei $\mu^2=1\,{\rm GeV}^2$ findet man
\cite{Leu89}
\beq
   m_u &=& 5.1 \pm 1.5 \;{\rm MeV}, \nonumber  \\
   m_d &=& 8.9 \pm 2.6 \;{\rm MeV}, \\
   m_s &=& 175 \pm 55 \;{\rm MeV}.  \nonumber
\eeq   
Alle anderen bekannten Flavors haben Massen \"uber einem GeV. Die
up und down Quarks sind ausserordentlich leicht verglichen mit dem
QCD Skalenparameter, w\"ahrend das seltsame Quark eine Zwischenstellung
einnimmt.   

Wir wollen daher im folgenden die drei leichten Quarks zun\"achst als
masselos betrachten. In diesem Fall besitzt ${\cal L}_{QCD}$
eine chirale $SU(3)_L \times SU(3)_R$ Symmetrie im Raum der $u$-,
$d$- und $s$-Flavors
\beq
\label{suv}
\psi_i(x) &\to& \exp (-i\theta^{a}\lambda^{a})_{ij} \psi_j (x)\, ,\\
\label{sua}
\psi_i(x) &\to& \exp (-i\phi^{a}\lambda^{a}\gamma_5)_{ij} \psi_j (x)\, ,
\eeq
wobei die $SU(3)$ Matrizen $\lambda^{a}$ auf den Flavorindex der
Quarks wirken. Die zugeh\"origen N\"otherstr\"ome sind die
Vektor- und Axialvektorstr\"ome
\beq
   V_\mu^{a}(x) &=& \bar{\psi}_i \gamma_\mu \frac{\lambda^{a}_{ij}}{2}
     \psi_j \, , \\
   A_\mu^{a}(x) &=& \bar{\psi}_i \gamma_\mu \gamma_5
   \frac{\lambda^{a}_{ij}}{2}   \psi_j  \, , 
\eeq
deren Zeitkomponenten auf die erhaltenen Ladungen 
\beq
 Q^{a}(t)   &=& \int d^3x \, V_0^{a}(\vec{x},t) \, , \\
 Q^{a}_5(t) &=& \int d^3x \, A_0^{a}(\vec{x},t) 
\eeq
f\"uhren. Die Struktur der zugeh\"origen Lie-Algebra erkennt man 
am einfachsten, indem man zu den 
Linearkombinationen $Q^{a}_{L,R}=Q^{a}\pm Q^{a}_5$ \"ubergeht.
Sie erf\"ullen die Vertauschungsrelationen
\beq
\label{chalg}
\,[Q_{L}^{a}(t),Q_{L}^{b}(t)] &=& i f^{abc} Q_{L}^{c}(t)\, ,  \\
\,[Q_{R}^{a}(t),Q_{R}^{b}(t)] &=& i f^{abc} Q_{R}^{c}(t)\, ,  \\
\,[Q_{L}^{a},Q_{R}^{b}]     &=& 0\, ,
\eeq
chrakteristisch f\"ur die Lie-Algebra $SU(3)_L \times SU(3)_R$.  
Das Transformationsverhalten der Str\"ome unter der chiralen
$SU(3)_L\times SU(3)_R$ ergibt sich aus den kanonischen
Vertauschungsregeln f\"ur die Quarkfelder
\beq
\label{curalg}
\,[ Q^{a}(t),V_\mu^b (\vec{x},t)] &=&\!\spm i f^{abc}V_\mu^{c}(\vec{x},t)\, ,\\
\,[ Q^{a}(t),A_\mu^b (\vec{x},t)] &=&\!\spm i f^{abc}A_\mu^{c}(\vec{x},t)\, ,\\  
\,[ Q^{a}_5(t),V_\mu^b (\vec{x},t)] &=&\! -i f^{abc}A_\mu^{c}(\vec{x},t)\, ,\\ 
\,[ Q^{a}_5(t),A_\mu^b (\vec{x},t)] &=&\!\spm i f^{abc}V_\mu^{c}(\vec{x},t) \; .
\eeq 
Diese Relationen legen  die $SU(3)_L\times SU(3)_R$ Darstellung
fest, nach der sich die Str\"ome transformieren. Sie wurden von 
Gell-Mann auf Grund rein ph\"anomenologischer \"Uberlegungen 
postuliert \cite{AD68}. 

Soll die Quantenchromodynamik im Grenzfall verschwindender Massen 
der leichten Quarks eine sinnvolle 
N\"aherung an die vollst\"andige Theorie darstellen, dann kann der 
QCD-Grundzustand nicht $SU(3)_L\times SU(3)_R$ symmetrisch sein. 
W\"are n\"amlich
\be
\label{symvac}
 Q_L^{a}|0> = Q_R^{a}|0> = 0 \; ,
\ee
dann sollten auch die Zweipunktfunktionen der Vektor und Axialvektorstr\"ome
identisch sein
\be
\label{symspec}
<0|T(A_\mu^{a}(x)A_\nu^{b}(0))|0> =<0|T(V_\mu^{a}(x)V_\nu^{b}(0))|0> \; .
\ee  
Zum Beweis zerlegt man die beiden Str\"ome in ihre links- und rechtsh\"andigen
Komponenten 
\beq
 V_\mu^{a} &=& J_{\mu\, R}^{a} + J_{\mu\, L}^{a} \\
 A_\mu^{a} &=& J_{\mu\, R}^{a} - J_{\mu\, L}^{a}
\eeq
und folgert aus (\ref{symvac}) und dem Transformationsverhalten der Str\"ome,
da\ss\  die nichtdiagonale Kombination $<0|T(J_{\mu\, L}^{a}J_{\nu\,R}^{a})|0>$
verschwindet.

Die spektrale Dichten zu den beiden Zweipunktfunktionen (\ref{symspec}) sind
experimentell zu\-g\"ang\-lich und zeigen ein sehr verschiedenartige Gestalt.
W\"ahrend der Vektorkanal vor allem durch die $\rho$-Resonanz bei 
$m_\rho=770$ MeV dominiert wird, ist die Axialvektorspektralfunktion in 
diesem Bereich klein und zeigt erst im Bereich der $a_1$-Resonanz bei
$m_{a_1}=1260$ MeV eine ausgepr\"agte Struktur. Wir folgern daraus, da\ss\ 
der QCD Grundzustand nicht $\chs$ symmetrisch ist. Dieses Ph\"anomen, 
da\ss\  n\"amlich der Grundzustand der Theorie
nicht die volle Symmetrie der Lagrangedichte besitzt, bezeichnet man
als spontane Symmetriebrechung.

Nach dem Goldstone-Theorem impliziert die spontane Brechung einer Symmetrie
das Auftreten masseloser Bosonen. Ist $Q^{a}$ ein  $\chs$ Generator,
der das Vakuum nicht invariant l\"a\ss t, dann gibt es einen physikalischen
Zustand $Q^{a}|0>$, der mit dem Vakuum energetisch entartet ist.  
Handelt es sich bei $Q^{a}$ um eine Vektorladung, dann beschreibt
$Q^{a}|0>$ ein masseloses skalares Teilchen. Ist $Q^{a}$ dagegen eine
axiale Ladung, so fordert das Goldstonetheorem das Auftreten masseloser
pseudoskalarer Bosonen.

In der Natur sind die acht leichtesten Hadronen $(\pi,K,\eta)$ pseudoskalar.
Dagegen sind die leichtesten skalaren Teilchen schwerer als das Nukleon.
Die chirale Symmetrie ist daher in der Form
\beq
   Q_V^{a}|0> &=& 0 \, , \\
   Q_A^{a}|0> &\neq& 0
\eeq
realisiert. Die Vektorsymmetrie bleibt erhalten, so da\ss\  sich Hadronen
nach irreduziblen Darstellungen von $SU(3)_V$ klassifizieren lassen. Im
$SU(2)$-Sektor der Theorie folgt daraus die Isospinsymmetrie der 
starken Wechselwirkung.

Endliche Quarkmassen brechen die chirale Symmetrie explizit. Die 
Divergenz der Vektor und Axialvektorstr\"ome lautet 
\beq
\label{divv}
\partial^\mu V_\mu^{a} &=& \frac{i}{2} 
               \bar{\psi}\left[M,\lambda^{a}\right]\psi\, , \\ 
\label{diva}
\partial^\mu A_\mu^{a} &=& \frac{i}{2} 
               \bar{\psi}\left\{M,\lambda^{a}\right\}\psi \, ,
\eeq
wobei $M={\rm diag}(m_u,m_d,m_s)$ die Massenmatrix f\"ur die drei 
leichten Flavors bezeichnet. Die rechte Seite von (\ref{divv},\ref{diva})
l\"a\ss t sich besonders \"ubersichtlich darstellen, indem man $M$ nach 
Gell-Mann Matrizen entwickelt, $M=\epsilon_0\lambda^0+\epsilon_3\lambda^3+
\epsilon_8\lambda^8$. Dabei ist
\beq
\epsilon_0 &=& \frac{1}{\sqrt{2}} (m_u+m_d+m_s)\, , \\
\epsilon_3 &=& \frac{1}{2} (m_u-m_d)\, ,  \\
\epsilon_8 &=& \frac{1}{2\sqrt{3}} (m_u+m_d-2m_s)\, .
\eeq
Im Fall entarteter Quarkmassen $m_u=m_d=m_s$ ist $\epsilon_3=\epsilon_8=0$
und die $SU(3)_V$ Flavor-Symmetrie bleibt ungebrochen.
 
Im Hadronenspektrum manifestieren sich die nichtverschwindenden 
Quarkmassen in einer endlichen Masse f\"ur die Goldstonebosonen 
$(\pi,K,\eta)$. Der Zusammenhang dieser Massen mit denen der Quarks
ist durch die Gell-Mann, Oakes, Renner (GOR) Relation gegeben
\cite{GOR68}. Um die bei der Herleitung von Niederenergietheoremen
verwendeten Methoden vorzustellen, wollen wir im folgenden einen
kurzen Beweis der GOR-Relation vorstellen. Zu diesem Zweck definieren
wie die Pionzerfallskonstante durch das Matrixelement
\be
\label{fpi}
 <0|A_\mu^{a}(x)|\pi^{b}(q)> = \delta^{ab} f_\pi q_\mu e^{iq\cdot x}
\ee
Diese \"Ubergangsmatrix kontrolliert den schwachen Zerfall der
geladenen Pionen $\pi^\pm \to \mu^\pm \nu_\mu$. F\"ur das
Matrixelement der Divergenz des Axialstroms findet man
\be
 <0|\partial^\mu A_\mu^{a}(x)|\pi^{b}(q)> = 
         \delta^{ab} f_\pi m_\pi^2 e^{iq\cdot x}\; ,
\ee	     
so da\ss\ sich ein interpolierendes Feld f\"ur das Pion durch
die Beziehung
\be
\label{PCAC}
\partial^\mu A_\mu^{a}(x) = f_\pi m_\pi^2 \phi^{a} (x)
\ee
definieren l\"a\ss t \cite{Col67}. Diese Gleichung wird als PCAC (partially
conserved axial current) Relation bezeichnet. Sie ist 
\"aquivalent zur Divergenzbeziehung (\ref{diva}) und dr\"uckt wie
diese die Tatsache aus, da\ss\  der Axialvektorstrom im chiralen 
Limes $m_\pi \to 0$ erhalten ist.

Wir wollen im folgenden Wardidentit\"aten f\"ur die Zweipunktfunktionen
\beq
\label{axtwop}
\Pi_{5\, \mu\nu}^{ab}(q) &=& i\int d^4x\, e^{iq\cdot x}
          <0|T(A_\mu^{a}(x)A_\nu^{b}(0))|0>    \\
\label{divtwop}	  
\psi_{5}^{ab} (q) &=& i\int d^4x\, e^{iq\cdot x}
          <0|T(\partial^\mu A_\mu^{a}(x)\partial^\nu A_\nu^{b}(0))|0>
\eeq
konstruieren. Zweimaliges Differenzieren des zeitgeordneten Produkts 
(\ref{axtwop}) liefert die Beziehung
\beq
\label{wi}
q^\mu q^\nu  \Pi^{ab}_{5\,\mu\nu} (q) &=& \psi_5^{ab}(q)
   -q^\nu \int d^4x\, e^{iq\cdot x} 
   \delta (x^0) <0|[A_0^{a}(x),A_\nu^{b}(0)]|0> \\
   & & \mbox{} -i\int d^4x\,  e^{iq\cdot x} 
   \delta (x^0) <0|[A_0^{a}(x),\partial^\mu A_\mu^{b}(0)]|0>\, , \nonumber
\eeq
die sich im Limes $q_\mu \to 0$ auf die Gleichung
\be
\label{psi0}
  \psi_5^{ab}(0) = -i \int d^4x \, \delta(x^0) 
        <0|[A_0^{a}(x),\partial^\mu A_\mu^{b}(0)]|0>    	   	       
\ee	
reduziert. Die linke Seite dieser Gleichung folgt aus der PCAC Relation, 
w\"ahrend man die rechte Seite mit Hilfe von (\ref{diva}) und
den kanonischen Vertauschungsregeln f\"ur die Quarkfelder 
bestimmmen kann. Das Resultat ergibt schlie\ss lich die GOR-Relation
\be
 f_\pi^2 m_\pi^2 = - \frac{1}{2}( m_u+m_d)<0|\bar{u}u+\bar{d}d|0>\; .
\ee
Sie verbindet die hadronischen Parameter $m_\pi$ und $f_\pi$ mit
den Quarkmassen und dem Ordnungsparameter f\"ur die spontane 
Brechung der chiralen Symmetrie, dem Quarkkondensat $<\! 0|\bar uu
+\bar dd|0\!>$. Wir haben in der Herleitung von dem Grenz\"ubergang 
$q_\mu \to 0$ Gebrauch gemacht. Die GOR-Relation ergibt daher 
eine Aussage \"uber die Pionzerfallskonstante im chiralen Limes.
F\"ur den physikalischen Wert von $f_\pi$ ergibt die
GOR-Relation daher nur den f\"uhrenden Term in einer Entwicklung
in Potenzen der Quarkmasse.     

Die Wardidentit\"at (\ref{wi}) gilt unabh\"angig von den 
physikalischen Zust\"anden, zwischen denen die Matrixelemente
genommen sind. Ersetzt man die Vakuumzust\"ande durch ein und
auslaufende Nukleonen, so l\"a\ss t sich die Divergenzamplitude 
$\psi_5^{ab}$  mit der Pion-Nukleon-Streumatrix
$T_{\pi N}^{ab}$ identifizieren. Auf diese
Weise k\"onnen wir die Rolle der expliziten Symmetriebrechung
in physikalischen Streuprozessen studieren. Durch zweimaliges
Differenzieren der Zweipunktfunktion $\Pi^{ab}_{5\,\mu\nu}$ ergibt sich 
folgende Wardidentit\"at f\"ur $T_{\pi N}^{ab}$ \cite{BPP71}:
\beq
\label{tpin}
 T_{\pi N}^{ab}(p_1,q_1;p_2,q_2) &=& T_{PV}^{ab}(p_1,q_1;p_2,q_2)
   +\frac{q_1^2+q_2^2-m_\pi^2}{f_\pi^2 m_\pi^2} \Sigma^{ab}(p_1,p_2) \\
   & &\mbox{} +\frac{1}{2f_\pi^2} (q_1+q_2)^\mu V_\mu^{ab}(p_1,p_2)
   +q_1^\mu q_2^\nu R^{ab}_{\mu\nu}(p_1,q_1;p_2,q_2)\, . \nonumber
\eeq
Dabei haben wir $q_1^\mu q_2^\nu\Pi^{ab}_{5\,\mu\nu}$ in die 
Beitr\"age der Pseudovektor-Bornterme $T_{PV}^{ab}$ und die 
Hintergrundamplitude $q_1^\mu q_2^\nu R_{\mu\nu}^{ab}$ zerlegt.
Sie enth\"alt weder Nukleon- noch Pionpole
und verschwindet daher im Niederenergielimes $q_1,q_2 \to 0$.
Des weiteren bezeichnet $V_\mu^{ab}(p_1,p_2)$ den Stromalgebraterm
\be
 V_\mu^{ab}(p_1,p_2) = i\epsilon^{abc}<N(p_2)|V_\mu^c(0)|N(p_1)>\, .
\ee
Unser spezielles Augenmerk liegt auf dem Pion-Nukleon Sigmaterm
$\Sigma^{ab}(p_1,p_2)$, der ein Ma\ss\ ist f\"ur die St\"arke der 
expliziten Symmetriebrechung. Wie bei der Herleitung der 
GOR-Relation ergibt sich
\be
\label{pinsig}
\Sigma^{ab}(p_1,p_2) = \frac{\delta^{ab}}{2}(m_u+m_d)
    <N(p_2)|\bar{u}u+\bar{d}d|N(p_1)>\; .
\ee
Der Sigmaterm liefert an der Schwelle den f\"uhrenden Beitrag zum
isospinsymmetrischen Teil der Streuamplitude. Dar\"uber hinaus kann
man den  Wert von
$\Sigma^{ab}(p_1,p_2)$ am unphysikalischen Punkt $p_1=p_2$
als Beitrag der Strommassen der leichten Quarks zur Nukleonmasse
interpretieren. Diese Gr\"o\ss e l\"a\ss t sich prinzipiell aus dem
beobachteten Baryonspektrum ermitteln. In chiraler St\"orungstheorie
findet man \cite{GL80}
\be
 \sigma =\frac{1}{2}(m_u+m_d)<p|\bar{u}u+\bar{d}d|p>
    = \frac{35\pm 5}{1+y}\; {\rm MeV}\, ,              
\ee
wobei $y=<p|\bar{s}s|p>/<p|\bar{u}u|p>$ das Verh\"altnis des
Kondensats der seltsamen Quarks zu dem  der leichten Quarks im
Nukleon bezeichnet. Um den Wert von $\sigma$ aus
Pion-Nukleon Streuphasen zu extrahieren, ist ein aufwendiges
Extrapolationsverfahren notwendig. Im Gegensatz zu fr\"uheren
Auswertungen liefern neuere Dispersionsanalysen einen 
Wert $\sigma=45 \pm 5$ MeV \cite{GLS91}, der vertr\"aglich
ist mit der Annahme eines verschwindenden Kondensats seltsamer
Quarks im Nukleon.

\section{Ableitung des Niederenergietheorems}
Die Herleitung von Niederenergietheoremen zur Pionphotoproduktion
verl\"auft analog zu der im letzten Abschnitt geschilderten Ableitung 
der GOR-Relation. Ausgangspunkt ist die Darstellung der Streumatrix mit Hilfe
der LSZ-Reduktionsformel
\beq
\label{LSZ}
 S^{a} &=& -(2\pi)^4 \,\delta^4 (p_1+k-p_2+q)\, Z_\gamma^{-1/2}
   Z_\pi^{-1/2} \\
   & & \mbox{}\cdot \int d^4x\, e^{iq\cdot x} (\Box +m_\pi^2)
   <N(p_2)|T\left(\epsilon^\mu V_\mu^{em}(0) \phi^{a}(x)\right)|N(p_1)>
   \nonumber\; .
\eeq
Dabei bezeichnet $\phi^{a}$ das kanonische Pionfeld, $Z_\pi=(2\pi)^3
2\omega_\pi$ dessen kovariante Normierung und $Z_\gamma=(2\pi)^3
2\omega$ die Normierung des elektromagnetischen Feldes. Die
\"Ubergangsmatrix nach der Definition aus dem ersten Kapitel ist
durch
\be
\label{deft}
 S^{a} = i(2\pi)^4\,\delta^4 (p_1+k-p_2+q) Z_\gamma^{-1/2}
  Z_\pi^{-1/2} \epsilon^\mu T_\mu^{a}
\ee
gegeben. Wir betrachten  die Zweipunktfunktion
\be
\label{Pimunu}
\overline{\Pi}^a_{\mu\nu}(q) = \int d^4 x\, e^{iq\cdot x}<N(p_2)| 
T\left( V_\mu^{em} (0) B_\nu^{a}(x) \right) |N(p_1)> \; .
\ee
des elektromagnetischen Stroms $V_\mu^{em}$ und des 'transversalen` Axialstroms
\be
B_\mu^{a}(x) =A_\mu^{a}(x)+\frac{1}{m_\pi^2}\partial_\mu D^{a}(x)
\ee
wobei $D^{a}(x)=\partial^\mu A_\mu^{a}(x)$ die Divergenz des Axialstroms
bezeichnet. Mit Hilfe der PCAC-Relation und der Definition der
Pionquellfunktion findet man
\be
\label{defb}
\partial^\mu B_\mu^a (x) = (\Box +m_\pi^2)\phi^{a}(x) =-j_\pi^{a}(x)\, .
\ee
Einmaliges Differenzieren des zeitgeordneten Produktes in der 
Zweipunktfunktion $\overline{\Pi}_{\mu\nu}$ liefert schlie\ss lich 
die gesuchte Wardidentit\"at f\"ur $T_\mu^{a}$
\be
\label{avward}
T_\mu^a (q) = \frac{1}{f_\pi}\left\{
iq^\nu \overline{\Pi}_{\mu\nu}^a (q) \, - \, C_\mu^a (q)  \, - \,
\frac{i \omega_\pi}{m_\pi^2} \Sigma^a_\mu (q) \right\} \; .
\ee
Dabei haben wir die LSZ-Formel (\ref{LSZ}) verwendet, um die 
Photoproduktionsamplitude $T_\mu^{a}$ zu identifizieren. Die Wirkung 
der Ableitung auf den Zeitordnungsoperator liefert die Kommutatoren
\beq
\label{cmua}
 C_\mu^{a}(q) &=& \int d^4x\, e^{iq\cdot x}\delta (x^0)
   <N(p_2)|[A^{a}_0(x),V_\mu^{em}(x)]|N(p_1)> \\
\label{sig}     
 \Sigma_\mu^{a}(q) &=& \int d^4x\, e^{iq\cdot x}\delta (x^0)
   <N(p_2)|[D^{a}(x),V_\mu^{em}(x)]|N(p_1)>\;
\eeq
wobei wir das Resultat in den Stromalgebraterm $C_\mu^{a}$
und den Kommutator der Divergenz des Axialstroms zerlegt haben.  
In Analogie zum Sigmaterm in der Pion-Nukleon-Streuung bezeichnet
man diesen Beitrag auch als $(\gamma,\pi)$-Sigmaterm. Wie der
$\pi N$-Sigmaterm liefert er eine zus\"atzliche Korrektur, die direkt
proportional zu den Stromquarkmassen in der QCD-Lagrangedichte ist.

Es ist instruktiv, die Wardidentit\"at zu studieren, 
die sich aus der Zweipunktfunktion $\Pi^{a}_{\mu\nu}$ des
\"ublichen Axialstroms ergibt. Analog zu (\ref{avward})
erh\"alt man
\be
\label{avward2}
\frac{m_\pi^2}{q^2-m_\pi^2} T_\mu^a (q) = \frac{1}{f_\pi}\Big\{
iq^\nu \Pi_{\mu\nu}^a (q) \, - \, C_\mu^a \Big\} \; .
\ee
Auf Grund des Pionpropagators vor der Amplitude $T_\mu^{a}$
liefert diese Beziehung die Photoproduktionsamplitude zun\"achst
nur am unphysikalischen ''weichen`` Punkt $q_\mu=0$.  Um die physikalische 
Schwelle $q^2=m_\pi^2$ zu erreichen, ist es notwendig, den Pionpol
explizit abzuseparieren. Diesem Zweck dient der transversale Axialstrom
$B_\mu^{a}$. Tats\"achlich l\"a\ss t sich $B_\mu^{a}$ mit Hilfe der
PCAC Relation als der nichtpionische Anteil des Axialstroms interpretieren
\be
 B_\mu^{a}(x) = A_\mu^{a}(x) +f_\pi\partial_\mu \phi^{a}(x)
    = A_\mu^{a}(x) - A_{\mu}^{a\, (\pi)} (x) \, .
\ee
Wir wollen nun die verschiedenen Beitr\"age zur Photoproduktionsamplitude
$T_\mu^{a}$ im Einzelnen studieren. Beginnen werden wir dabei mit
der Zweipunktfunktion $q^\nu\overline{\Pi}_{\mu\nu}^{a}$. Da dieser
Term proportional zum Impuls $q$ ist, tragen im Grenzfall weicher Pionen
nur die Polterme in $\overline{\Pi}_{\mu\nu}^{a}$ zur Amplitude bei.
Die Summe der Nukleonpolterme im direkten und im Austauschkanal
lautet
\beq
\label{nborn}
f_\pi T_\mu^{a\,(N)} &=& \bar{u}(p_2) \Big( iq^\nu \Gamma_\nu^{B^{a}}
   (p_1-k,-q,p_2) S_F(p_1+k) \Gamma_\mu^\gamma (p_1,k,p_1+k)
      \\[0.2cm]
   & & \hspace{0.5cm} \mbox{}+ \Gamma_\mu^\gamma (p_1-q,k,p_2)
   S_F(p_1-q) iq^\nu \Gamma_\mu^{B^{a}}(p_1,-q,p_1-q) \Big) u(p_1)
   \; .\nonumber
\eeq      
Dabei bezeichnet $S_F(p)$ den Nukleonpropagator und $\Gamma_\mu^\gamma$
bzw. $\Gamma_\nu^{B^{a}}$ die Vertexfunktionen f\"ur die Kopplung
des Nukleons an das elektromagnetische Feld und den Axialstrom $B_\nu^{a}$.
In den Poltermen sind alle Teilchen mit Ausnahme des
intermedi\"aren Nukleons auf der Massenschale. Die allgemeine Gestalt der
Vertexfunktion lautet in diesem Fall
\beq
\label{emvert}
\Gamma_\mu^\gamma (p_1,k,p_1+k)u(p_1) &=& \left( \gamma_\mu F_1 
   +  \frac{M+(p_1+k)\cdot\gamma}{2M} \frac{i\sigma_\mu\nu k^\nu}{2M}F_2^+ 
   \right. \\
 & & \hspace{1.5cm}\left. \mbox{}
   +  \frac{M-(p_1+k)\cdot\gamma}{2M} \frac{i\sigma_\mu\nu k^\nu}{2M}F_2^-
   \right) u(p_1)  \nonumber \\
\label{bavert}
\Gamma_\nu^{B^{a}} (p_1,-q,p_1-q)u(p_1) &=& \left( \gamma_\nu \overline{G}_A 
   +  \frac{M+(p_1-q)\cdot\gamma}{2M} \frac{q_\nu}{2M} \overline{G}_P^{\, +}
   \right. \\
  & & \hspace{1.5cm}\left. \mbox{}  
   +  \frac{M-(p_1-q)\cdot\gamma}{2M} \frac{q_\nu}{2M} \overline{G}_P^{\, -} 
   \right)\gamma_5\frac{\tau^{a}}{2} u(p_1) \; .\nonumber
\eeq     
Analoge Ausdr\"ucke ergeben sich, wenn der Impuls $p_2$ die 
Massenschalenbedingung erf\"ullt. Die Formfaktoren $F_i$ sind Funktionen
des Impuls\"ubertages $k^2$ und des off-shell Parameters $\delta^2=(p_1+k)^2
-M^2$. Ihre Isospinstruktur lautet
\be
 F_i=F_i^{s}+F_i^{v}\tau^3 \; .
\ee
Entsprechend h\"angen die Formfaktoren am $NNB^{a}$-Vertex von den 
Variablen $q^2$ und ${\delta '}^2=(p_1-q)^2-M^2$ ab. Wir haben diese 
Formfaktoren mit einem Querstrich gekennzeichnet, um sie von den 
entsprechenden Funktionen am  Vertex des Axialvektorstroms zu 
unterscheiden.

Die Abh\"angigkeit der Photoproduktionsamplitude vom off-shell
Verhalten der Formfaktoren wurde in einer Arbeit von Naus, Koch
und Friar \cite{NKF90} untersucht. Die Autoren zeigen, da\ss\ 
off-shell Korrekturen erst in derselben Ordnung in $m_\pi$ 
wie andere modellabh\"angige Korrekturen auftreten. Wir verwenden
daher von nun an die on-shell Vertices
\beq
\Gamma_\mu^\gamma &=& \gamma_\mu F_1 + 
          \frac{i\sigma_{\mu\nu}k^\nu}{2M} F_2 \\
\Gamma_\nu^{B^{a}}&=& \left( \gamma_\mu \overline{G}_A
         + \frac{q_\mu}{2M}\overline{G}_P \right) \gamma_5 
	 \frac{\tau^{a}}{2}
\eeq
wobei die auftretenden Formfaktoren nur mehr Funktionen der
Impuls\"ubertr\"age $k^2$ bzw.~$q^2$ sind. F\"ur reelle
Photonen ist
\be
\begin{array}{rclcrcl}
  F_1^{s}&=& 1/2        &\hspace{1cm}& F_1^{v}&=& 1/2     \\[0.2cm]
  F_2^{s}&=&\kappa^s    &            & F_2^{v}&=&\kappa^v  
\end{array}
\ee
mit den anomalen magnetischen Momenten $\kappa^{s,v}=\frac{1}{2}
(\kappa_p\pm \kappa_n)$. Die Vertexfunktionen f\"ur die beiden 
Str\"ome $A_\nu^{a}$ und $B_\nu^{a}$ unterscheiden sich nur 
um die Matrixelemente des pionischen Beitrags 
$A_\nu^{a(\pi)}=-f_\pi \partial_\nu \phi^{a}$. 
Dieser Term erzeugt den Pionpol im induzierten 
pseudoskalaren Formfaktor
\be
  G_P^{\pi -Pol} (q^2)=\frac{4Mf_\pi}{m_\pi^2-q^2} G_{\pi NN}(q^2)
\ee
wobei $G_{\pi NN}$ den Pion-Nukleon Formfaktor
\be
  <N(p_2)|j_\pi^{a}(0)|N(p_1)> = G_{\pi NN}(t) \bar{u}(p_2)i\gamma_5
         \tau^{a}u(p_1)
\ee
bezeichnet. Der Zusammenhang der Formfaktoren an den Vertices ist 
daher durch $\overline{G}_A=G_A$ und $\overline{G}_P=G_P-G_P^{\pi -Pol}$ 
gegeben. Empirische Untersuchungen zeigen, da\ss\ $G_P$ in sehr guter 
N\"aherung durch den Polterm beschrieben wird. Wir setzen daher 
$\overline{G}_P=0$ und erhalten
\beq
\label{nborn2}
f_\pi T_\mu^{a\,(N)} &=& \bar{u}(p_2) \left\{ g_A iq\cdot\gamma \gamma_5 
 \,\frac{\tau^{a}}{2} \frac{i}{(p_1+k)\cdot\gamma -M} \,\Gamma_\mu^\gamma
 \right. \\
 & & \hspace{1cm}\left. \mbox{} + \Gamma_\mu^\gamma 
     \,\frac{i}{(p_1-q)\cdot\gamma -M}\,
  g_A iq\cdot\gamma \gamma_5 \frac{\tau^{a}}{2} \right\} u(p_1), \nonumber 
\eeq
wobei wir daueber hinaus den axialen Formfaktor $G_A(q^2)$ durch den
Wert beim Impuls\"ubertrag $g_A=G_A(q^2=0)$ ersetzt haben. Die axiale
Kopplung l\"a\ss t sich mit Hilfe der Goldberger-Treiman Relation
\be
\label{GT}
\frac{g_A}{2f_\pi} = \frac{f}{m_\pi}
\ee
durch die pseudovektorielle Pion-Nukleon Kopplungskonstante $f$ ausdr\"ucken.
Das Resultat (\ref{nborn2}) entspricht daher der Born-Approximation f\"ur
eine effektive Pion-Nukleon Lagrangedichte mit dem Kopplungsterm
\be
\label{pv}
{\cal L} = \frac{f}{m_\pi} \bar{\psi}\gamma_5\gamma_\mu \tau^{a}\psi
   \partial^\mu \phi^{a}\; .
\ee    
Die geschilderte Herleitung f\"uhrt also in nat\"urlicher Weise
auf eine pseudovektorielle Kopplung des Pions an das Nukleon. Dies 
steht im Gegensatz zu vielen klassischen Arbeiten, in denen 
gew\"ohnlich mit einer pseudoskalaren Kopplung gerechnet wird.
Um Konsistenz mit der Wardidentit\"at (\ref{avward}) zu erzielen,
m\"ussen in diesem Fall zus\"atzliche Korrekturterme zur Bornamplitude
addiert werden.

Der Beitrag des Kommutators $C_\mu^{a}$ l\"a\ss t sich mit Hilfe 
der Stromalgebraregeln berechnen
\be
\label{curcom}
 C_\mu^{a} = -\epsilon^{a3c} <N(p_2)|A_\mu^{c}(0)|N(p_1)>\; .
\ee
Vernachl\"assigt man den Hintergrundbeitrag im induzierten 
pseudoskalaren Formfaktor, so ergibt sich
\be
\label{kr}
\epsilon^\mu C_\mu^{a} = -\epsilon^{a3c} \bar{u}(p_2)
  \left\{ G_A(t)\epsilon\cdot\gamma + G_{\pi NN}(t) f_\pi  
   \frac{\epsilon\cdot (k-q)}{m_\pi^2-t} \right\}
   \gamma_5 \tau^{c}u(p_1) 
\ee   	 	  
als Funktion der Mandelstamvariable $t=(q-k)^2$.
Das Resultat ist antisymmetrisch in den Isospinindices und
tr\"agt daher nur zur Produktion geladener Pionen bei. 
Der erste Term liefert den f\"uhrenden Beitrag zur 
Pho\-to\-pro\-duk\-ti\-ons\-amplitude im Grenzfall weicher Pionen
\be
\label{krtheo}
\lim_{q,k\to 0} T^{a}(q) =\frac{g_A}{f_\pi}\epsilon^{a3c}
   \bar{u}(p_2) \epsilon\cdot\gamma\gamma_5\tau^{a}u(p_1)\; ,
\ee
den sogenannten Kroll-Ruderman-Term \cite{KR54}. 

Der zweite Kommutator enth\"alt die Divergenz des Axialstroms
und l\"a\ss t sich daher mit Ausnahme der Zeitkomponente 
$\Sigma_0^{a}$ nicht modellunabh\"angig bestimmen. Mit Hilfe
des Stromalgebraresultats
\be
 \,[Q_5^{a},V_\mu^{em}(0)]=-i\epsilon^{a3c}A_\mu^{c}(0)
\ee
und der Erhaltung des elektromagnetischen Stroms findet man
\be
\label{sig0}
  \int d^4x \,\delta (x^0)\, [\partial^\mu A_\mu^{a}(x),V_0^{em}
  (0)] = -i\epsilon^{a3c} \partial^\mu A_\mu^{c} (0) \; .
\ee     
Das Matrixelement der Divergenz des Axialstroms zwischen
Nukleonzust\"anden $|N(p)>$ ist die axialen Formfaktoren des
Nukleons bestimmt:
\be
<N(p_2)| D^{a}(x) |N(p_1)> = \bar{u}(p_2) \left[ M G_A (t)
  + \frac{t}{4M} G_P(t) \right] \gamma_5 \tau^{a} u(p_1) \; .
\ee
Ber\"ucksichtigt man wie oben nur den f\"uhrenden Pionbeitrag,
so l\"a\ss t sich die Summe der beiden Kommutatoren 
an der Schwelle in die Form
\be
 \epsilon^\mu T_\mu^{a\,(\pi)}  = \epsilon^{a3c} g_{\pi NN}
   \,\frac{\epsilon\cdot (k-2q)}{m_\pi^2 -t} \,
   \bar{u}(p_2)\gamma_5 \tau^{a} u(p_1)
\ee
bringen. 
Die Raumkomponenten des Sigmaterms $\Sigma_\mu^{a}$ enthalten die 
Information \"uber die explizite Brechung der chiralen Symmetrie 
durch die Quarkmassen in der QCD-Lagrangedichte. Ihre Bestimmung
ist jedoch modellabh\"angig und wird uns in den n\"achsten 
Abschnitten noch besch\"aftigen. Vernachl\"assigt man diese
Korrektur, so ergeben die oben diskutierten Beitr\"age (\ref{nborn2},
\ref{kr},\ref{sig0}) folgende Bestimmung der invarianten Amplituden       
\beq
\label{letamp}
A^{(+0,-)}_1 &=&  \frac{2f}{\mu} \spm
      \left\{ -\frac{1}{\nu+\nu_1} \mp \frac{1}{\nu-\nu_1} 
      + \frac{1\mp 1}{\nu_1} \right\} \\
A^{(+0,-)}_2 &=&  \frac{2f}{\mu} \spm
      \left\{ -\frac{2}{\nu+\nu_1} \pm \frac{2}{\nu-\nu_1} \right\} \\      
A^{(+0,-)}_3 &=& \; \frac{2f}{\mu} \kappa \; (-1 \pm 1)   \\
A^{(+0,-)}_4 &=& \frac{2f}{\mu}\;\kappa
      \left\{ \frac{2}{\nu+\nu_1} \pm \frac{2}{\nu-\nu_1} \right\} \\ 
A^{(+0,-)}_5 &=&  \frac{2f}{\mu}\;\kappa
      \left\{ \frac{4}{\nu+\nu_1} \mp \frac{4}{\nu-\nu_1} \right\} \\       
A^{(+0,-)}_6 &=&  \frac{2f}{\mu}(1+2\kappa)
      \left\{ \frac{1}{\nu+\nu_1} \mp \frac{1}{\nu-\nu_1} \right\} 
      + \frac{2f}{\mu} \kappa\, (1\pm 1) \; .
\eeq
Dabei haben wir der \"Ubersichtlichkeit halber den Isospinindex
der anomalen magnetischen Momente unterdr\"uckt. Es gilt
$\kappa^{(\pm)}=\kappa^v$ und $\kappa^{(0)}=\kappa^s$. 
Mit Hilfe der im ersten Kapitel abgeleiteten Formel f\"ur die
Schwellenamplitude,
\be
 \left. E_{0+}\right|_{thr} = \frac{e}{16\pi M}
 \frac{2+\mu}{(1+\mu)^{3/2}} \, \left. \left(
   A_3 + \frac{\mu}{2} A_6 \right) \right|_{thr}
\ee                
und der in Anhang A diskutierten Kinematik
ergibt sich schlie\ss lich folgende Bestimmung der elektrischen
Dipolamplitude f\"ur die vier physikalischen Kan\"ale 
\beq
\label{LET}
\Epn &=& \frac{e}{4\pi} \frac{\sqrt{2}f}{m_\pi}
    \left\{ 1 - \frac{3}{2}\mu + {\cal O}(\mu^2) \right\}
    \cong 26.6 \su \\[0.1cm]
\Emp &=& \frac{e}{4\pi} \frac{\sqrt{2}f}{m_\pi}
     \left\{ -1 + \frac{1}{2}\mu + {\cal O}(\mu^2) \right\}
    \cong -31.7 \su \\[0.1cm]
\Eop &=& \frac{e}{4\pi} \frac{f}{m_\pi}
     \left\{ -\mu + \frac{\mu^2}{2}(3+\kappa_p ) +
  {\cal O}(\mu^3) \right\}    \cong -2.32
  \su \\[0.1cm]
\Eon &=& \frac{e}{4\pi} \frac{f}{m_\pi}
     \left\{  \frac{\mu^2}{2}\kappa_n  +
  {\cal O}(\mu^3) \right\}  \cong -0.51 \su 
\eeq
wobei wir die Werte $\kappa_p=1.79$ und $\kappa_n=-1.91$ verwendet 
haben. Dieses Resultat liefert den Inhalt des Niederenergietheorems
\cite{Bae70,VZ72}. Die relative Ordnung der nicht bestimmten
Korrekturen wurde mit Hilfe verschiedener Annahmen \"uber
das Verhalten der Untergrundamplitude festgelegt. Wir werden
diese Annahmen und ihre Rechtfertigung im n\"achsten Abschnitt
diskutieren.

\begin{figure}
\label{feyn}
\caption{Diagrammatische Darstellung der f\"uhrenden Beitr\"age zur
Pionphotoproduktionsamplitude}
\vspace{8.5cm}
\end{figure}

Niederenergietheoreme zur Pionphotoproduktion lassen sich auch
direkt aus der Bestimmung der Bornterme in effektiven 
chiralen Meson-Nukleon Theorien ableiten \cite{Pec69}. Die
entsprechende Lagrangedichte unter Einbeziehung der 
elektromagnetischen Wechselwirkung lautet
\beq
\label{leff}
{\cal L} & =& \bar{\psi}(i\gamma\cdot{\cal D}-M)\psi 
  +\frac{1}{2}({\cal D}_\mu\phi^{a})^2 - \frac{1}{2}m_\pi^2
  (\phi^{a})^2  \\
 & & \mbox{} + \frac{f}{m_\pi} \bar{\psi}\gamma_5 \gamma_\mu
 \tau^{a} {\cal D}^\mu \phi^{a}\psi 
  + \frac{e}{4m}\bar{\psi} (\kappa^s +\kappa^v \tau^3)
  \sigma_{\mu\nu}\psi F^{\mu\nu} \nonumber 
\eeq
wobei ${\cal D}_\mu=\partial_\mu+iQ{\cal A}_\mu$ die kovariante
Ableitung, ${\cal A}_\mu$ das elektromagnetische Potential und
$Q$ den Ladungsoperator bezeichnet. Die Eichung der pseudovektoriellen
Pion-Nukleon-Kopplung erzeugt eine $\gamma\pi NN$-Kontaktwechselwirkung,
\be
{\cal L}_{\gamma\pi NN} = \frac{ef}{m_\pi}\epsilon^{3ab}
  \bar{\psi}\gamma_5 \gamma_\mu \tau^{a}\psi {\cal A}^\mu \phi^b
\ee  
die im Rahmen der effektiven Theorie die Kroll-Ruderman-Amplitude
liefert. Die Wirkung der kovarianten Ableitung auf das Pionfeld
bestimmt die Kopplung des elektromagnetischen Feldes an die
geladenen Pionen. Die $\gamma\pi\pi$-Wechselwirkung 
\be  
{\cal L}_{\gamma\pi\pi} = e\epsilon^{3ab}\phi^{a}\partial_\mu
 \phi^{b} {\cal A}^\mu
\ee
bewirkt schlie\ss lich die bereits angesprochene Struktur des
Pionpolterms. 
   
    
\section{Abs