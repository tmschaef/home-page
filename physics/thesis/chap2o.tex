\chapter{Niederenergietheoreme zur Pionphotoproduktion}
Nachdem wir uns im letzten Kapitel mit der experimentellen 
Bestimmung der elektrischen Dipolamplitude an der Schwelle
befasst haben, wollen wir uns nun auf die theoretische 
Bestimmung von $E_{0+}$ mit Hilfe von Niederenergietheoremen
konzentrieren. Die spezielle Bedeutung der Photoproduktion
neutraler Pionen ergibt sich dabei aus der Tatsache, da\ss\
die Schwellenamplitude in einer hypothetischen Welt
mit masselosen Pionen  verschwinden w\"urde.
Dieser Kanal ist daher besonders sensitiv auf die Rolle der
expliziten chiralen Symmetriebrechung, welche sich in  dem
nur approximativen Charakter des Pions als Goldstoneboson 
widerspiegelt. 

Die physikalische Grundidee der Niederenergietheoreme
({\em engl.} Low Energy Theorem, LET) l\"a\ss t sich besonders
\"ubersichtlich an rein elektromagnetischen Reaktionen
diskutieren. Die Anwendung  von Niederenergietheoremen wird 
in diesem Fall durch zwei physikalische Kriterien kontrolliert. 
Die beiden Forderungen lauten, da\ss\ sowohl 
die Wellenl\"ange des Photons gro\ss\ gegen die  
Ausdehnung des Streuzentrums, als auch dessen 
Energie klein gegen die typische Anregungsenergie ist.
Sind diese Voraussetzungen erf\"ullt, so ist das Photon nicht in der
Lage, die innere Struktur des Targets aufzul\"osen. Der
differentielle Wirkungsquerschnitt ist daher ausschlie\ss lich durch 
globale elektromagnetische Eigenschaften des Streuzentrums bestimmt.
Betrachtet man die Comptonstreuung niederenergetischer
Photonen an einem hadronischen Target, so folgt aus dieser
\"Uberlegung, da\ss\ der differentielle Wirkungsquerschnitt 
nur von der Gesamtladung $Ze$ abh\"angt. Insbesondere ergibt 
sich im Grenzfall $k_\mu=(\omega,\vec{k})\to 0$ die klassische
Thomson-Streuung
\be
\label{thomson}
 \lim_{\omega \to 0} \frac{d\sigma}{d\Omega} =
  \frac{Z^2e^2}{4\pi M^2} (\vec{\epsilon}_1 \cdot\vec{\epsilon}_2)
\ee     
wobei $\vec{\epsilon}_1$ und $\vec{\epsilon}_2$ die 
Polarisationsvektoren der ein- und auslaufenden Photonen
bezeichnen. Low, Gell-Mann und Goldberger \cite{Low54,Low58,GMG54}
konnten dar\"uber hinaus zeigen, da\ss\ auf Grund von 
Eich- und Lorentzinvarianz die Forw\"artsstreuamplitude 
f\"ur Comptonstreuung an einem Spin 1/2 Target sogar bis
auf Terme linear in der Laborenergie $\omega$ bestimmt ist
\be
 \lim_{\omega \to 0}  T(\omega) =- \frac{e^2}{M}  
  (\vec{\epsilon}_1 \cdot\vec{\epsilon}_2) -i \frac{e^2}{8 M^2}
  \kappa^2 \omega  (\vec{\epsilon}_1 \times\vec{\epsilon}_2)
  \cdot \vec{\sigma} \; .
\ee
Der erste Term beschreibt  die Thomsonamplitude, w\"ahrend der
zweite Term eine Korrektur liefert, die proportional zum
anomalen magnetischen Moment $\kappa$ des Streuzentrums ist.

Die Anwendung von Niederenergietheoremen auf pionische 
Reaktionen wird durch die endliche Masse des Pions 
erschwert. Die Pionmasse setzt eine untere Grenze f\"ur 
die Energie  $\omega_\pi=(\vec{q}^{\, 2} +m_\pi^2)^{1/2}$
des Pions. W\"ahrend daher die Wellenl\"ange beliebig gro\ss\
gemacht werden kann, gibt es eine prinzipielle Schranke 
f\"ur die Energie. Die Anwendbarkeit von Niederenergietheoremen
setzt daher voraus,
da\ss\ die Masse des Pions klein gegen die charakteristische
Energieskala der Reaktion ist. In hadronischen Prozessen ist
eine solche Skala durch die Masse des Nukeons gegeben. 
Das Verh\"altnis $m_\pi/M\simeq 1/7$ ist daher ein nat\"urlicher
Parameter, der die Abweichung der  Amplitude vom unphysikalischen
Grenzfall $q_\mu \to 0$ kontrolliert.

Formal basieren Niederenergietheoreme f\"ur weiche Pionen auf
der G\"ultigkeit von Stromalgebra, chiraler Symmetrie und
PCAC. Historisch wurden diese Konzepte in den sechziger Jahren 
als Hypothesen \"uber das Transformationsverhalten der in
hadronische Reaktionen eingehenden Str\"ome entwickelt
\cite{AD68,AFF73}. Sie erwiesen sich als au\ss erordentlich
fruchtbar, um die sich ansammelnde Flut  experimenteller 
Informationen \"uber hadronische Reaktionen zu verstehen. 
Die klassischen Anwendungen liegen im Bereich
der $\pi\pi$- und $\pi N$-Streuung, sowie der Photo- und schwachen
Produktion pseudoskalarer Mesonen. Wichtige Vorhersagen ergeben
sich ebenso f\"ur leptonische und semileptonische Zerf\"alle
stark wechsewirkender Teilchen.    
  
Nach der Gr\"underphase trat die Anwendung von Stromalgebra und
chiraler Symmetrie zun\"achst in den Hintergrund gegen\"uber
der Entwicklung von Quantenchromodynamik als fundamentaler
Eichtheorie der starken Wechselwirkung. Inzwischen ist jedoch
klar, da\ss\ die G\"ultigkeit dieser Konzepte eine direkte
Konsequenz von QCD ist. Dar\"uber hinaus liefert die chirale 
Symmetrie  einen wichtigen Zugang zur starken Wechselwirkung
in einem Bereich, in dem die Anwendung von QCD
bislang noch gro\ss en Schwierigkeiten gegen\"uber steht.

\section[Quantenchromodynamik, Stromalgebra \ldots]{Quantenchromodynamik,
 Stromalgebra und chirale Symmetrie}
Quantenchromodynamik ist eine nichtabelsche Eichtheorie, 
beschrieben durch die Lagrangedichte
\be
\label{lqcd}
{\cal L}_{QCD} = -\frac{1}{4} F_{\mu\nu}^{\;\;a} F^{\mu\nu\, a}
 + \sum_{j=1}^{n_f} \bar{\psi}^{\alpha}_{j}( i\gamma^{\mu}
 {\cal D}_\mu^{\alpha\beta} - \delta^{\alpha\beta} m_j )
 \psi_j^\beta \; ,
\ee 
welche eine Summation \"uber $n_f$ verschiedene Quarkarten (flavors)
enth\"alt. W\"ahrend sich die Quarkspinoren $\psi^{\alpha}$ nach
der fundamentalen Darstellung der Eichgruppe $SU(3)$ transformieren,
sind die Gluonen $A_\mu^{a}$ Vektorfelder und tragen einen Index 
in der adjungierten Darstellung von $SU(3)$. Die zugeh\"origen
Elemente der Liealgebra sind
\be
 A_\mu(x) = A_\mu^{a}(x)\frac{\lambda^{a}}{2}\; ,
\ee
wobei $\lambda^{a}$ die Generatoren der Algebra bezeichnet.
Sie erf\"ullen die fundamentalen Vertauschungsrelationen
\be
 [\lambda^{a},\lambda^b] = 2if^{abc}\lambda^c
\ee
und k\"onnen durch
\be
 Tr \,\lambda^{a}\lambda^b = 2\delta^{ab}
\ee
normiert werden. Dabei bezeichnet $f^{abc}$ die Strukturkonstanten
von $SU(3)$. Der Yang-Mills Feldst\"arketensor ist durch
\be
\label{fmunu}
 F_{\mu\nu}^{\;\; a} = \partial_\mu A_\nu^{a} -\partial_\nu A_\mu^{a} 
 + g f^{abc} A_\mu^b A_\nu^c
\ee
gegeben, w\"ahrend die kovariante Ableitung durch
\be
\label{kovd}
 {\cal D}_\mu^{\alpha\beta} = \delta^{\alpha\beta}\partial_\mu
  + i\frac{g}{2} (\lambda^{a})^{\alpha\beta} A_\mu^{a}
\ee
definiert ist. Die Quark-Gluon Wechselwirkung wird durch die
selbe Kopplungskonstante $g$ wie
die Selbstwechselwirkung der gluonischen Felder kontrolliert.
   
Neben der lokalen $SU(3)$ Eichsymmetrie besitzt die $QCD$
Lagrangedichte noch eine Reihe kontinuierlicher globaler Symmetrien.
So ist ${\cal L}_{QCD}$ invariant unter der globalen 
$U(1)_B$ Transformation
\be
\label{uone}
\psi_j(x) \to \exp (-i\theta) \psi_j (x)
\ee
welche die Erhaltung des Baryonenstroms
\be
 j_\mu(x) = \sum_{j=1}^{n_f} \bar{\psi}_j \gamma_\mu \psi_j
\ee
und der baryonischen Ladung 
\be
B=\int d^3x\, j_0(\vec{x},t)
\ee
bewirkt. F\"ur masselose Quarks ist ${\cal L}_{QCD}$ ebenfalls invariant
unter der axialen $U(1)_A$ Transformation
\be
\label{uaone}
\psi_j(x) \to \exp (-i\theta\gamma_5) \psi_j (x) \; .
\ee
Der zugeh\"orige Strom besitzt allerdings eine anomale Divergenz
\be
\label{axanom}
\partial^\mu j_{\mu\, 5}(x) = \frac{g^2}{4\pi}\frac{n_f}{8}
 \epsilon^{\mu\nu\rho\sigma} F_{\mu\nu}^{\;\; a}F_{\rho\sigma}^{\;\; a}
 \; ,
\ee
so da\ss\  die axiale Ladung $Q_5=\int d^3x\, j_{0\,5}(\vec{x},t)$
nur in  Abwesenheit instantonartiger L\"osungen erhalten ist.

Vernachl\"assigt man die Quarkmassen, so ist ${\cal L}_{QCD}$ auch
invariant unter Skalentransformationen. Diese Symmetrie wird in der
quantisierten Theorie durch die Notwendigkeit der Renormierung
gebrochen. Dabei tritt ein dimensionsbehafteter Parameter, der
QCD Skalenparameter $\Lambda_{\mini QCD}$, auf. Sein Wert ist aus der
Skalenbrechung in tief inelastischer Lepton-Nukleon Streung zu
$\Lambda_{\mini QCD}^{\mini\overline{MS}} =230\pm 80$ MeV bestimmt worden
\cite{PDG90}. Der Index ${\kl \overline{MS}}$ bezeichnet eine spezielle
Renormierungsvorschrift, die sogenannte modifizierte minimale 
Subtraktion.

Ebenfalls auf Grund der Renormierung sind  auch die Werte der 
Quarkmassen von der experimentellen Skala abh\"angig.
Im Bereich typischer hadronischer Prozesse lassen sich 
die Stromquarkmassen mit Hilfe von QCD-Summenregeln 
extrahieren \cite{GL82}. Bei $\mu^2=1\,{\rm GeV}^2$ findet man
\cite{Leu89}
\beq
   m_u &=& 5.1 \pm 1.5 \;{\rm MeV} \nonumber  \\
   m_d &=& 8.9 \pm 2.6 \;{\rm MeV} \\
   m_s &=& 175 \pm 55 \;{\rm MeV}  \nonumber
\eeq   
Alle anderen bekannten Flavors haben Massen \"uber einem GeV. Die
up und down Quarks sind ausserordentlich leicht verglichen mit dem
QCD Skalenparameter, w\"ahrend das seltsame Quark eine Zwischenstellung
einnimmt.   

Wir wollen daher im folgenden die drei leichten Flavors als
masselos betrachten. In diesem Fall besitzt ${\cal L}_{QCD}$
eine chirale $SU(3)_L \times SU(3)_R$ Flavor-Symmetrie
\beq
\label{suv}
\psi_i(x) &\to& \exp (-i\theta^{a}\lambda^{a})_{ij} \psi_j (x) \\
\label{sua}
\psi_i(x) &\to& \exp (-i\phi^{a}\lambda^{a}\gamma_5)_{ij} \psi_j (x)
\eeq
wobei die $SU(3)$ Matrizen $\lambda^{a}$ auf den Flavorindex der
Quarks wirken. Die zugeh\"origen N\"otherstr\"ome sind die
Vektor und Axialvektorstr\"ome
\beq
   V_\mu^{a}(x) &=& \bar{\psi}_i \gamma_\mu \frac{\lambda^{a}_{ij}}{2}
     \psi_j  \\
   A_\mu^{a}(x) &=& \bar{\psi}_i \gamma_\mu \gamma_5
   \frac{\lambda^{a}_{ij}}{2}   \psi_j  
\eeq
deren Zeitkomponenten auf die erhaltenen Ladungen 
\beq
 Q^{a}(t)   &=& \int d^3x \, V_0^{a}(\vec{x},t)  \\
 Q^{a}_5(t) &=& \int d^3x \, A_0^{a}(\vec{x},t) 
\eeq
f\"uhren. Die Struktur der zugeh\"origen Liealgebra erkennt man 
am einfachsten, indem man zu den 
Linearkombinationen $Q^{a}_{L,R}=Q^{a}\pm Q^{a}_5$ \"ubergeht.
Sie erf\"ullen die Vertauschungsrelationen
\beq
\label{chalg}
\,[Q_{L}^{a}(t),Q_{L}^{b}(t)] &=& i f^{abc} Q_{L}^{c}(t)  \\
\,[Q_{R}^{a}(t),Q_{R}^{b}(t)] &=& i f^{abc} Q_{R}^{c}(t)  \\
\,[Q_{L}^{a},Q_{R}^{b}]     &=& 0
\eeq
chrakteristisch f\"ur die Liealgebra $SU(3)_L \times SU(3)_R$.  
Das Transformationsverhalten der Str\"ome unter der chiralen
$SU(3)_L\times SU(3)_R$ ergibt sich aus den kanonischen
Vertauschungsregeln f\"ur die Quarkfelder
\beq
\label{curalg}
\,[ Q^{a}(t),V_\mu^b (\vec{x},t)] &=& i f^{abc} V_\mu^{c}(\vec{x},t) \\
\,[ Q^{a}(t),A_\mu^b (\vec{x},t)] &=& i f^{abc} A_\mu^{c}(\vec{x},t) \\  
\,[ Q^{a}_5(t),V_\mu^b (\vec{x},t)] &=& -i f^{abc} A_\mu^{c}(\vec{x},t) \\ 
\,[ Q^{a}_5(t),A_\mu^b (\vec{x},t)] &=& i f^{abc} V_\mu^{c}(\vec{x},t) \; .
\eeq 
Diese Relationen legen  die $SU(3)_L\times SU(3)_R$ Darstellung
fest, nach der sich die Str\"ome transformieren. Sie bilden den Inhalt 
von  Gell-Manns Stromalgebra \cite{AD68} und sind unabh\"angig
von der Frage, ob die Str\"ome erhalten sind. 

Soll die Quantenchromodynamik mit verschwindenden Massen eine sinnvolle 
N\"aherung an die vollst\"andige Theorie darstellen, dann kann der 
QCD Grundzustand nicht $SU(3)_L\times SU(3)_R$ symmetrisch sein. 
W\"are n\"amlich
\be
\label{symvac}
 Q_L^{a}|0> = Q_R^{a}|0> = 0 \; ,
\ee
dann sollten auch die Zweipunktfunktionen der Vektor und Axialvektorstr\"ome
identisch sein
\be
\label{symspec}
<0|T(A_\mu^{a}(x)A_\nu^{b}(0))|0> =<0|T(V_\mu^{a}(x)V_\nu^{b}(0))|0> \; .
\ee  
Zum Beweis zerlegt man die beiden Str\"ome in ihre links und rechtsh\"andigen
Komponenten 
\beq
 V_\mu^{a} &=& J_{\mu\, R}^{a} + J_{\mu\, L}^{a} \\
 A_\mu^{a} &=& J_{\mu\, R}^{a} - J_{\mu\, L}^{a}
\eeq
und folgert aus (\ref{symvac}) und dem Transformationsverhalten der Str\"ome,
da\ss\  die nichtdiagonale Kombination $<0|T(J_{\mu\, L}^{a}J_{\nu\,R}^{a})|0>$
verschwindet.

Die spektrale Dichten zu den beiden Zweipunktfunktionen (\ref{symspec}) sind
experimentell zu\-g\"ang\-lich und zeigen ein sehr verschiedenartige Gestalt.
W\"ahrend der Vektorkanal vor allem durch die $\rho$-Resonanz bei 
$m_\rho=770$ MeV dominiert wird, ist die Axialvektorspektralfunktion in 
diesem Bereich klein und zeigt erst im Bereich der $a_1$-Resonanz bei
$m_{a_1}=1220$ MeV eine ausgepr\"agte Struktur. Wir folgern daraus, da\ss\ 
der QCD Grundzustand tats\"achlich nicht $\chs$ symmetrisch ist. 
Dieses Ph\"anomen, da\ss\  der Grundzustand der Theorie
nicht die volle Symmetrie der Lagrangedichte besitzt, bezeichnet man
als spontane Symmetriebrechung.

Nach dem Goldstone Theorem bewirkt die spontane Brechung einer Symmetrie
das Auftreten masseloser Bosonen. Ist $Q^{a}$ ein  $\chs$ Generator
der das Vakuum nicht invariant l\"a\ss t, dann gibt es einen physikalischen
Zustand $Q^{a}|0>$, der mit dem Vakuum energetisch entartet ist.  
Handelt es sich bei $Q^{a}$ um eine Vektorladung, dann beschreibt
$Q^{a}|0>$ ein masseloses skalares Teilchen. Ist $Q^{a}$ dagegen eine
axiale Ladung, so fordert das Goldstonetheorem das Auftreten masseloser
pseudoskalarer Bosonen.

In der Natur sind die acht leichtesten Hadronen $(\pi,K,\eta)$ pseudoskalar.
Dagegen sind die leichtesten skalaren Teilchen schwerer als das Nukleon.
Wir schlie\ss en daraus, da\ss\  die chirale Symmetrie in der Form
\beq
   Q_V^{a}|0> &=& 0 \\
   Q_A^{a}|0> &\neq& 0
\eeq
realisiert ist. Die Vektorsymmetrie bleibt erhalten, so da\ss\  sich Hadronen
nach irreduziblen Darstellungen von $SU(3)_V$ klassifizieren lassen. Im
$SU(2)$-Sektor der Theorie entspricht das der Isospinsymmetrie der 
starken Wechselwirkung.

Endliche Quarkmassen brechen die chirale Symmetrie explizit. Die 
Divergenz der Vektor und Axialvektorstr\"ome lautet in diesem Fall
\beq
\label{divv}
\partial^\mu V_\mu^{a} &=& \frac{i}{2} 
               \bar{\psi}\left[M,\lambda^{a}\right]\psi \\ 
\label{diva}
\partial^\mu A_\mu^{a} &=& \frac{i}{2} 
               \bar{\psi}\left\{M,\lambda^{a}\right\}\psi
\eeq
wobei $M={\rm diag}(m_u,m_d,m_s)$ die Massenmatrix f\"ur die drei 
leichten Flavors bezeichnet. Die rechte Seite von (\ref{divv},\ref{diva})
l\"a\ss t sich  am einfachsten Auswerten, indem man $M$ nach Gell-Mann
Matrizen entwickelt, $M=\epsilon_0\lambda^0+\epsilon_3\lambda^3+
\epsilon_8\lambda^8$. Dabei ist
\beq
\epsilon_0 &=& \frac{1}{\sqrt{2}} (m_u+m_d+m_s) \\
\epsilon_3 &=& \frac{1}{2} (m_u-m_d)  \\
\epsilon_8 &=& \frac{1}{2\sqrt{3}} (m_u+m_d-2m_s)
\eeq
Im Fall entarteter Quarkmassen $m_u=m_d=m_s$ ist $\epsilon_3=\epsilon_8=0$
und die $SU(3)_V$ Flavor-Symmetrie bleibt ungebrochen.
 
Im Hadronenspektrum manifestieren sich die nicht verschwindenden 
Quarkmassen in einer endlichen Masse f\"ur die Goldstonebosonen 
$(\pi,K,\eta)$. Der Zusammenhang dieser Massen mit denen der Quarks
ist durch die Gell-Mann, Oakes, Renner (GOR) Relation gegeben
\cite{GOR68}. Diese ist ein einfaches Beispiel f\"ur ein
Niederenergietheorem und illustriert sehr sch\"on die
verwendeten Methoden. 

Die Pionzerfallskonstante ist durch das Matrixelement
\be
\label{fpi}
 <0|A_\mu^{a}(x)|\pi^{b}(q)> = \delta^{ab} f_\pi q_\mu e^{iq\cdot x}
\ee
definiert. Dieses Matrixelement kontrolliert den schwachen Zerfall der
geladenen Pionen $\pi^\pm \to \mu^\pm \nu_\mu \nu_e$. F\"ur das
Matrixelement der Divergenz des Axialstroms ergibt sich
\be
 <0|\partial^\mu A_\mu^{a}(x)|\pi^{b}(q)> = 
         \delta^{ab} f_\pi m_\pi^2 e^{iq\cdot x}\; .
\ee	     
Auf Grund dieser Beziehung l\"a\ss t sich ein interpolierendes Feld
f\"ur das Pion durch
\be
\label{PCAC}
\partial^\mu A_\mu^{a}(x) = f_\pi m_\pi^2 \phi^{a} (x)
\ee
definieren \cite{Col67}. Diese Gleichung wird als PCAC (partially
conserved axial current) Relation bezeichnet. Sie ist vollst\"andig
\"aquivalent zur Divergenzbeziehung (\ref{diva}) und dr\"uckt wie
diese die Tatsache aus, da\ss\  der Axialvektorstrom im chiralen 
Limes $m_\pi \to 0$ erhalten ist.

Wir wollen im folgenden Wardidentit\"aten f\"ur die Zweipunktfunktionen
\beq
\label{axtwop}
\Pi_{5\, \mu\nu}^{ab}(q) &=& i\int d^4x\, e^{iq\cdot x}
          <0|T(A_\mu^{a}(x)A_\nu^{b}(0))|0>    \\
\psi_{5}^{ab} (q) &=& i\int d^4x\, e^{iq\cdot x}
          <0|T(\partial^\mu A_\mu^{a}(x)\partial^\nu A_\nu^{b}(0))|0>
\eeq
konstruieren. Kontrahiert man die freien Indices von $\Pi^{ab}_{5\,\mu\nu}$
mit Impulsen $q_\mu$, so findet man
\beq
\label{wi}
q^\mu q^\nu  \Pi^{ab}_{5\,\mu\nu} (q) &=& \psi_5^{ab}(q)
   -q^\nu \int d^4x\, e^{iq\cdot x} 
   \delta (x^0) <0|[A_0^{a}(x),A_\nu^{b}(0)]|0> \\
   & & \mbox{} -i\int d^4x\,  e^{iq\cdot x} 
   \delta (x^0) <0|[A_0^{a}(x),\partial^\mu A_\mu^{b}(0)]|0> \nonumber
\eeq
Im Niederenergielimes $q_\mu \to 0$ reduziert sich diese Beziehung 
auf
\be
\label{psi0}
  \psi_5^{ab}(0) = -i \int d^4x \, \delta(x^0) 
        <0|[A_0^{a}(x),\partial^\mu A_\mu^{b}(0)]|0>    	   	       
\ee	
Die linke Seite von (\ref{psi0}) folgt aus der PCAC Relation, 
w\"ahrend man die rechte Seite mit Hilfe von (\ref{diva}) und
den kanonischen Vertauschungsregeln f\"ur die Quarkfelder 
bestimmmen kann. Damit ergibt sich schlie\ss lich die GOR
Relation
\be
 f_\pi^2 m_\pi^2 = - \frac{1}{2}( m_u+m_d)<0|\bar{u}u+\bar{d}d|0>\; .
\ee
Sie verbindet die hadronischen Parameter $m_\pi$ und $f_\pi$ mit
den Quarkmassen und den Quarkkondensaten $<0|\bar{u}u|0>$ und
$<0|\bar{d}d|0>$, die ein Ordnungsparameter f\"ur die spontane 
Brechung der chiralen Symmetrie sind. Wir haben in der Herleitung
von dem Grenz\"ubergang $q_\mu \to 0$ Gebrauch gemacht. F\"ur die
physikalischen Werte von $m_\pi$ und $f_\pi$ liefert die
GOR Relation daher nur den f\"uhrenden Term in einer Entwicklung
in Potenzen der Quarkmassen.     

Die Wardidentit\"at (\ref{wi}) gilt unabh\"angig von den 
physikalischen Zust\"anden, zwischen denen die Matrixelemente
genommen sind. Ersetzt man die Vakuumzust\"ande durch ein und
auslaufende Nukleonen, so l\"a\ss t sich die Divergenzamplitude 
$\psi_5^{ab}$  mit der Pion-Nukleon Streumatrix
$T_{\pi N}^{ab}$ identifizieren. Auf diese
Weise k\"onnen wir die Rolle der expliziten Symmetriebrechung
in physiaklischen Streuprozessen studieren. Durch zweimaliges
Differenzieren der Zweipunktfunktion $\Pi^{ab}_{5\,\mu\nu}$ ergibt sich 
folgende Wardidentit\"at f\"ur $T_{\pi N}^{ab}$ \cite{BPP71}
\beq
\label{tpin}
 T_{\pi N}^{ab}(p_1,q_1;p_2,q_2) &=& T_{PV}^{ab}(p_1,q_1;p_2,q_2)
   +\frac{q_1^2+q_2^2-m_\pi^2}{f_\pi^2 m_\pi^2} \Sigma^{ab}(p_1,p_2) \\
   & &\mbox{} +\frac{1}{2f_\pi^2} (q_1+q_2)^\mu V_\mu^{ab}(p_1,p_2)
   +q_1^\mu q_2^\nu R^{ab}_{\mu\nu}(p_1,q_1;p_2,q_2) \nonumber
\eeq
Dabei haben wir $q_1^\mu q_2^\nu\Pi^{ab}_{5\,\mu\nu}$ in die 
Beitr\"age der Pseudovektor-Bornterme $T_{PV}^{ab}$ und die 
Untergrundamplitude $q_1\mu q_2^\nu R_{\mu\nu}^{ab}$ zerlegt.
Sie enth\"alt weder Nukleon- noch Pionpole
und verschwindet daher im Niederenergielimes $q_1,q_2 \to 0$.
Des weiteren bezeichnet $V_\mu^{ab}(p_1,p_2)$ den Stromalgebraterm
\be
 V_\mu^{ab}(p_1,p_2) = i\epsilon^{abc}<N(p_2)|V_\mu^c(0)|N(p_1)>\, .
\ee
Unser spezielles Augenmerk liegt auf dem Pion-Nukleon Sigmaterm
$\Sigma^{ab}(p_1,p_2)$, welcher die St\"arke der expliziten 
Symmetriebrechung kontrolliert. Wie bei der Herleitung der 
GOR-Relation ergibt sich
\be
\label{pinsig}
\Sigma^{ab}(p_1,p_2) = \frac{\delta^{ab}}{2}(m_u+m_d)
    <N(p_2)|\bar{u}u+\bar{d}d|N(p_1)>\; .
\ee
Der Sigmaterm liefert an der Schwelle den f\"uhrenden Beitrag zum
isospinsymmetrischen Teil der Streuamplitude. Dar\"uber hinaus kann
man den  Wert von
$\Sigma^{ab}(p_1,p_2)$ am unphysikalischen Punkt $p_1=p_2$
als Beitrag der Strommassen der leichten Quarks zur Nukleonmasse
interpretieren. Diese Gr\"o\ss e l\"a\ss t sich prinzipiell aus dem
beobachteten Baryonspektrum ermitteln. In chiraler St\"orungstheorie
findet man \cite{GL80}
\be
 \sigma =\frac{1}{2}(m_u+m_d)<p|\bar{u}u+\bar{d}d|p>
    = \frac{35\pm 5}{1+y}\; {\rm MeV}\, ,              
\ee
wobei $y=<p|\bar{s}s|p>/<p|\bar{u}u|p>$ das Verh\"altnis des
Kondensats der seltsamen Quarks zu dem  der leichten Quarks im
Nukleon bezeichnet. Um den Wert von $\sigma$ aus
Pion-Nukleon Streuphasen zu extrahieren, ist ein aufwendiges
Extrapolationsverfahren notwendig. Im Gegensatz zu fr\"uheren
Auswertungen liefern neuere Dispersionsanalysen einen 
Wert $\sigma=45 \pm 5$ MeV \cite{GLS91}, der vertr\"aglich
ist mit der Annahme eines verschwindenden Kondensats seltsamer
Quarks im Nukleon.

\section{Ableitung des Niederenergietheorems}
Die Herleitung von Niederenergietheoremen zur Pionphotoproduktion
verl\"auft ganz analog zu der im letzten Abschnitt geschilderten Ableitung 
der GOR Relation. Ausgangspunkt ist die Darstellung der Streumatrix mit Hilfe
der LSZ Reduktionsformel
\beq
\label{LSZ}
 S^{a} &=& -(2\pi)^4 \,\delta^4 (p_1+k-p_2+q)\, Z_\gamma^{-1/2}
   Z_\pi^{-1/2} \\
   & & \mbox{}\cdot \int d^4x\, e^{iq\cdot x} (\Box +m_\pi^2)
   <N(p_2)|T\left(\epsilon^\mu V_\mu^{em}(0) \phi^{a}(x)\right)|N(p_1)>
   \nonumber\; .
\eeq
Dabei bezeichnet $\phi^{a}$ das kanonische Pionfeld, $Z_\pi=(2\pi)^3
2\omega_\pi$ dessen kovariante Normierung und $Z_\gamma=(2\pi)^3
2\omega$ die Normierung des elektromagnetischen Feldes. Die
\"Ubergangsmatrix nach der Definition aus dem ersten Kapitel ist
durch
\be
\label{deft}
 S^{a} = i(2\pi)^4\,\delta^4 (p_1+k-p_2+q) Z_\gamma^{-1/2}
  Z_\pi^{-1/2} \epsilon^\mu T_\mu^{a}
\ee
gegeben. Wir betrachten  die Zweipunktfunktion
\be
\label{Pimunu}
\overline{\Pi}^\alpha_{\mu\nu}(q) = \int d^4 x\, e^{iq\cdot x}<N(p_2)| 
T\left( V_\mu^{em} (0) B_\nu^{a}(x) \right) |N(p_1)> \; .
\ee
des elektromagnetischen Stroms $V_\mu^{em}$ und des 'transversalen` Axialstroms
\be
B_\mu^{a}(x) =A_\mu^{a}(x)+\frac{1}{m_\pi^2}\partial_\mu D^{a}(x)
\ee
wobei $D^{a}(x)=\partial^\mu A_\mu^{a}(x)$ die Divergenz des Axialstroms
bezeichnet. Mit Hilfe der PCAC Relation und der Definition der
Pionquellfunktion findet man
\be
\label{defb}
\partial^\mu B_\mu^a (x) = (\Box +m_\pi^2)\phi^{a}(x) =-j_\pi^{a}(x)\, .
\ee
Einmaliges Differenzieren des zeitgeordneten Produktes in der 
Zweipunktfunktion $\overline{\Pi}_{\mu\nu}$ liefert schlie\ss lich 
die gesuchte Wardidentit\"at f\"ur $T_\mu^{a}$
\be
\label{avward}
T_\mu^a (q) = \frac{1}{f_\pi}\left\{
iq^\nu \overline{\Pi}_{\mu\nu}^a (q) \, - \, C_\mu^a (q)  \, - \,
\frac{i \omega_\pi}{m_\pi^2} \Sigma^a_\mu (q) \right\} \; .
\ee
Dabei haben wir die LSZ-Formel (\ref{LSZ}) verwendet, um die 
Photoproduktionsamplitude $T_\mu^{a}$ zu identifizieren. Die Wirkung 
der Ableitung auf den Zeitordnungsoperator liefert die Kommutatoren
\beq
\label{cmua}
 C_\mu^{a}(q) &=& \int d^4x\, e^{iq\cdot x}\delta (x^0)
   <N(p_2)|[A^{a}_0(x),V_\mu^{em}(x)]|N(p_1)> \\
\label{sig}     
 \Sigma_\mu^{a}(q) &=& \int d^4x\, e^{iq\cdot x}\delta (x^0)
   <N(p_2)|[D^{a}(x),V_\mu^{em}(x)]|N(p_1)>\;
\eeq
wobei wir das Resultat in den Stromalgebraterm $C_\mu^{a}$
und den Kommutator der Divergenz des Axialstroms zerlegt haben.  
In Analogie zum Sigmaterm in der Pion-Nukleon Streuung bezeichnet
man diesen Beitrag auch als $(\gamma,\pi)$ Sigmaterm. Wie der
$\pi N$ Sigmaterm liefert er eine zus\"atzliche Korrektur, welche direkt
proportional zu den Stromquarkmassen in der QCD-Lagrangedichte ist.

Es ist instruktiv, die Wardidentit\"at zu studieren, 
die sich aus der Zweipunktfunktion $\Pi^{a}_{\mu\nu}$ des
\"ublichen Axialstroms ergibt. Analog zu (\ref{avward})
erh\"alt man
\be
\label{avward2}
\frac{m_\pi^2}{q^2-m_\pi^2} T_\mu^a (q) = \frac{1}{f_\pi}\left\{
iq^\nu \Pi_{\mu\nu}^a (q) \, - \, C_\mu^a \right\} \; .
\ee
Auf Grund des Pionpropagators vor der Amplitude $T_\mu^{a}$
liefert diese Beziehung die Photoproduktionsamplitude zun\"achst
nur am unphysikalischen weichen Punkt $q=0$.  Um die physikalische 
Schwelle $q^2=m_\pi^2$ zu erreichen, ist es notwendig, den Pionpol
explizit abzuseparieren. Diesem Zweck dient der transversale Axialstrom
$B_\mu^{a}$. Tats\"achlich l\"a\ss t sich $B_\mu^{a}$ mit Hilfe der
PCAC Relation als der nichtpionische Anteil des Axialstroms interpretieren
\be
 B_\mu^{a}(x) = A_\mu^{a}(x) +f_\pi\partial_\mu \phi^{a}(x)
    = A_\mu^{a}(x) - A_{\mu}^{a\, (\pi)} (x) 
\ee
Wir wollen nun die verschiedenen Beitr\"age zur Photoproduktionsamplitude
$T_\mu^{a}$ im Einzelnen studieren. Beginnen werden wir dabei mit
der Zweipunktfunktion $q^\nu\overline{\Pi}_{\mu\nu}^{a}$. Da dieser
Term proportional zum Impuls $q$ ist, tragen im Grenzfall weicher Pionen
nur die Polterme in $\overline{\Pi}_{\mu\nu}^{a}$ zur Amplitude bei.
Die Summe der Nukleonpolterme im direkten und im Austauschkanal
lautet
\beq
\label{nborn}
f_\pi T_\mu^{a\,(N)} &=& \bar{u}(p_2) \Big( iq^\nu \Gamma_\nu^{B^{a}}
   (p_1-k,-q,p_2) S_F(p_1+k) \Gamma_\mu^\gamma (p_1,k,p_1+k)
      \\[0.2cm]
   & & \hspace{0.5cm} \mbox{}+ \Gamma_\mu^\gamma (p_1-q,k,p_2)
   S_F(p_1-q) iq^\nu \Gamma_\mu^{B^{a}}(p_1,-q,p_1-q) \Big) u(p_1)
   \; .\nonumber
\eeq      
Dabei bezeichnet $S_F(p)$ den Nukleonpropagator und $\Gamma_\mu^\gamma$
bzw. $\Gamma_\nu^{B^{a}}$ die Vertexfunktionen, welche die Kopplung
des Nukleons an das elektromagnetische Feld und den Axialstrom $B_\nu^{a}$
beschreiben. In den Poltermen sind alle Teilchen mit Ausnahme des
intermedi\"aren Nukleons auf der Massenschale. Die allgemeine Gestalt der
Vertexfunktion lautet in diesem Fall
\beq
\label{emvert}
\Gamma_\mu^\gamma (p_1,k,p_1+k)u(p_1) &=& \left( \gamma_\mu F_1 
   +  \frac{M+(p_1+k)\cdot\gamma}{2M} \frac{i\sigma_\mu\nu k^\nu}{2M}F_2^+ 
   \right. \\
 & & \hspace{1.5cm}\left. \mbox{}
   +  \frac{M-(p_1+k)\cdot\gamma}{2M} \frac{i\sigma_\mu\nu k^\nu}{2M}F_2^-
   \right) u(p_1)  \nonumber \\
\label{bavert}
\Gamma_\nu^{B^{a}} (p_1,-q,p_1-q)u(p_1) &=& \left( \gamma_\nu \overline{G}_A 
   +  \frac{M+(p_1-q)\cdot\gamma}{2M} \frac{q_\nu}{2M} \overline{G}_P^{\, +}
   \right. \\
  & & \hspace{1.5cm}\left. \mbox{}  
   +  \frac{M-(p_1-q)\cdot\gamma}{2M} \frac{q_\nu}{2M} \overline{G}_P^{\, -} 
   \right)\gamma_5\frac{\tau^{a}}{2} u(p_1) \; .\nonumber
\eeq     
Analoge Ausdr\"ucke ergeben sich, wenn der Impuls $p_2$ die 
Massenschalenbedingung erf\"ullt. Die Formfaktoren $F_i$ sind Funktionen
des Impuls\"ubertages $k^2$ und des off-shell Parameters $\delta^2=(p_1+k)^2
-M^2$. Ihre Isospinstruktur lautet
\be
 F_i=F_i^{s}+F_i^{v}\tau^3 \; .
\ee
Entsprechend h\"angen die Formfaktoren am $NNB^{a}$-Vertex von den 
Variablen $q^2$ und ${\delta '}^2=(p_1-q)^2-M^2$ ab. Wir haben diese 
Formfaktoren mit einem Querstrich gekennzeichnet, um sie von den 
entsprechenden Funktionen f\"ur den Vertex des Axialvektorstroms zu 
unterscheiden.

Auf die Abh\"angigkeit der Photoproduktionsamplitude vom off-shell
Verhalten der Formfaktoren werden wir sp\"ater noch n\"aher eingehen.
Diese Frage wurde zudem in einer Arbeit von Nauss, Koch
und Friar \cite{NKF90} untersucht. Die Autoren zeigen, da\ss\ 
off-shell Korrekturen erst in derselben Ordnung in $m_\pi$ 
wie andere modellabh\"angige Korrekturen auftreten. Wir verwenden
daher von nun an die on-shell Vertices
\beq
\Gamma_\mu^\gamma &=& \gamma_\mu F_1 + 
          \frac{i\sigma_{\mu\nu}k^\nu}{2M} F_2 \\
\Gamma_\nu^{B^{a}}&=& \left( \gamma_\mu \overline{G}_A
         + \frac{q_\mu}{2M}\overline{G}_P \right) \gamma_5 
	 \frac{\tau^{a}}{2}
\eeq
wobei die auftretenden Formfaktoren nur mehr Funktionen der
Impuls\"ubertr\"age $k^2$ bzw.~$q^2$ sind. F\"ur reelle
Photonen ist
\be
\begin{array}{rclcrcl}
  F_1^{s}&=& 1/2        &\hspace{1cm}& F_1^{v}&=& 1/2     \\[0.2cm]
  F_2^{s}&=&\kappa^s    &            & F_2^{v}&=&\kappa^v  
\end{array}
\ee
mit den anomalen magnetischen Momenten $\kappa^s=-0.06$ und
$\kappa^v=1.85$. Die Vertexfunktionen f\"ur die beiden 
Str\"ome $A_\nu^{a}$ und $B_\nu^{a}$ unterscheiden sich nur 
durch die Matrixelemente des pionischen Beitrags 
$A_\nu^{a(\pi)}=-f_\pi \partial_\nu \phi^{a}$. 
Dieser Term generiert den Pionpol im induzierten 
pseudoskalaren Formfaktor
\be
  G_P^{\pi -Pol} (q^2)=\frac{4Mf_\pi}{m_\pi^2-q^2} G_{\pi NN}(q^2)
\ee
wobei $G_{\pi NN}$ den Pion-Nukleon Formfaktor
\be
  <N(p_2)|j_\pi^{a}(0)|N(p_1)> = G_{\pi NN}(t) \bar{u}(p_2)i\gamma_5
         \tau^{a}u(p_1)
\ee
bezeichnet. Der Zusammenhang der Formfaktoren an den Vertices ist 
daher durch $\overline{G}_A=G_A$ und $\overline{G}_P=G_P-G_P^{\pi -Pol}$ 
gegeben. Empirische Untersuchungen zeigen, da\ss\ $G_P$ in sehr guter 
N\"aherung durch den Polterm beschrieben wird. Wir setzen daher 
$\overline{G}_P=0$ und erhalten
\beq
\label{nborn2}
f_\pi T_\mu^{a\,(N)} &=& \bar{u}(p_2) \left\{ g_A iq\cdot\gamma \gamma_5 
 \,\frac{\tau^{a}}{2} \frac{i}{(p_1+k)\cdot\gamma -M} \,\Gamma_\mu^\gamma
 \right. \\
 & & \hspace{1cm}\left. \mbox{} + \Gamma_\mu^\gamma 
     \,\frac{i}{(p_1-q)\cdot\gamma -M}\,
  g_A iq\cdot\gamma \gamma_5 \frac{\tau^{a}}{2} \right\} \nonumber 	  
\eeq
wobei wir daueber hinaus den axialen Formfaktor $G_A(q^2)$ durch den
Wert beim Impuls\"ubertrag $g_A=G_A(q^2=0)$ ersetzt haben. Die axiale
Kopplung l\"a\ss t sich mit Hilfe der Goldberger-Treiman Relation
\be
\label{GT}
\frac{g_A}{2f_\pi} = \frac{f}{m_\pi}
\ee
durch die pseudovektorielle Pion-Nukleon Kopplungskonstante $f$ ausdr\"ucken.
Das Resultat entspricht dem Ergebnis einer Baum-Niveau Rechnung f\"ur
eine effektive Pion-Nukleon Lagrangedichte mit dem Kopplungsterm
\be
\label{pv}
{\cal L} = \frac{f}{m_\pi} \bar{\psi}\gamma_5\gamma_\mu \tau^{a}\psi
   \partial^\mu \phi^{a}\; .
\ee    
Die geschilderte Herleitung f\"uhrt also in nat\"urlicher Weise
auf eine pseudovektorielle Kopplung des Pions ans Nukleon. Dies 
steht im Gegensatz zu vielen klassischen Arbeiten, in denen 
gew\"ohnlich mit einer pseudoskalaren Kopplung gerechnet wird.
Um Konsistenz mit der Wardidentit\"at (\ref{avward}) zu erzielen,
m\"ussen in diesem Fall zus\"atzliche Korrekturterme zur Bornamplitude
addiert werden.

Der Beitrag des Kommutators $C_\mu^{a}$ l\"a\ss t sich mit Hilfe 
der Stromalgebraregeln berechnen
\be
\label{curcom}
 C_\mu^{a} = -\epsilon^{a3c} <N(p_2)|A_\mu^{c}(0)|N(p_1)>\; .
\ee
Vernachl\"assigt man den Untergrundbeitrag im induzierten 
pseudoskalaren Formfaktor, ergibt sich
\be
\label{kr}
\epsilon^\mu C_\mu^{a} = -\epsilon^{a3c} \bar{u}(p_2)
  \left\{ G_A(t)\epsilon\cdot\gamma + G_{\pi NN}(t) f_\pi  
   \frac{\epsilon\cdot (k-q)}{m_\pi^2-t} \right\}
   \gamma_5 \tau^{c}u(p_1) \; .
\ee   	 	  
Das Resultat ist antisymmetrisch in den Isospinindices und
tr\"agt daher nur zur Produktion geladener Pionen bei. 
Der erste Term liefert den f\"uhrenden Beitrag zur 
Pho\-to\-pro\-duk\-ti\-ons\-amplitude im Grenzfall weicher Pionen
\be
\label{krtheo}
\lim_{q,k\to 0} T^{a}(q) =\frac{g_A}{f_\pi}\epsilon^{a3c}
   \bar{u}(p_2) \epsilon\cdot\gamma\gamma_5\tau^{a}u(p_1)\; ,
\ee
den sogenannten Kroll-Ruderman Term \cite{KR54}. 

\begin{figure}
\label{feyn}
\caption{Diagrammatische Darstellung der f\"uhrenden Beitr\"age zur
Pionphotoproduktionsamplitude}
\vspace{7.5cm}
\end{figure}
Der zweite Kommutator enth\"alt die Divergenz des Axialstroms
und l\"a\ss t sich daher mit Ausnahme der Zeitkomponente 
$\Sigma_0^{a}$ nicht modellunabh\"angig bestimmen. Mit Hilfe
desStromalgebrakommutators
\be
 \,[Q_5^{a},V_\mu^{em}(0)]=-i\epsilon^{a3c}A_\mu^{c}(0)
\ee
und der Erhaltung des elektromagnetischen Stroms findet man
\be
\label{sig0}
  \int d^4x \,\delta (x^0)\, [\partial^\mu A_\mu^{a}(x),V_0^{em}
  (0)] = -i\epsilon^{a3c} \partial^\mu A_\mu^{c} (0) \; .
\ee     
Zwischen physikalischen Nukleonzust\"anden ist der Formfaktor der
Divergenz durch denjenigen des Axialvektorstroms vollst\"andig
bestimmt
\be
<N(p_2)| D^{a}(x) |N(p_1)> = \bar{u}(p_2) \left[ M G_A (t)
  + \frac{t}{4M} G_P(t) \right] \gamma_5 \tau^{a} u(p_1) \; .
\ee
Approximiert auch diesen Formfaktor durch den Polbeitrag, so
l\"a\ss t sich der Pionbeitrag zu den beiden Kommutatoren 
an der Schwelle insgesamt in der Form
\be
 \epsilon^\mu T_\mu^{a\,(\pi)}  = \epsilon^{a3c} g_{\pi NN}
   \,\frac{\epsilon\cdot (k-2q)}{m_\pi^2 -t} \,
   \bar{u}(p_2)\gamma_5 \tau^{a} u(p_1)
\ee
darstellen. Er zeigt die typische $\epsilon\cdot (k-2q)$ Struktur, die
man auch in einer expliziten Berechnung des Feynmandiagramms in
Abbildung 1c) findet.

Die Raumkomponenten des Sigmaterms $\Sigma_\mu^{a}$ enthalten die 
Information \"uber die explizite Brechung der chiralen Symmetrie 
durch die Quarkmassen in der QCD Lagrangedichte. Ihre Bestimmung
ist jedoch modellabh\"angig und wird uns in den n\"achsten 
Abschnitten noch besch\"aftigen. Vernachl\"assigt man diese
Korrektur, so ergeben die oben diskutierten Beitr\"age (\ref{nborn2},
\ref{kr},\ref{sig0}) folgende Bestimmung der invarianten Amplituden       
\beq
\label{letamp}
A^{(+0,-)}_1 &=&  \frac{2f}{\mu} \spm
      \left\{ -\frac{1}{\nu+\nu_1} \mp \frac{1}{\nu-\nu_1} 
      + \frac{1\mp 1}{\nu_1} \right\} \\
A^{(+0,-)}_2 &=&  \frac{2f}{\mu} 2\;
      \left\{ -\frac{1}{\nu+\nu_1} \pm \frac{1}{\nu-\nu_1} \right\} \\      
A^{(+0,-)}_3 &=& \; \frac{2f}{\mu} \kappa \; (-1 \pm 1)   \\
A^{(+0,-)}_4 &=& \frac{2f}{\mu}2\kappa
      \left\{ \frac{1}{\nu+\nu_1} \pm \frac{1}{\nu-\nu_1} \right\} \\ 
A^{(+0,-)}_5 &=&  \frac{2f}{\mu}4\kappa
      \left\{ \frac{1}{\nu+\nu_1} \mp \frac{1}{\nu-\nu_1} \right\} \\       
A^{(+0,-)}_6 &=&  \frac{2f}{\mu}(1+2\kappa)
      \left\{ \frac{1}{\nu+\nu_1} \mp \frac{1}{\nu-\nu_1} \right\} 
      + \frac{2f}{\mu} \kappa\, (1\pm 1) \; .
\eeq
Dabei haben wir der \"Ubersichtlichkeit halber den Isospinindex
der anomalen magnetischen Momente unterdr\"uckt. Es gilt
$\kappa^{(\pm)}=\kappa^v$ und $\kappa^{(0)}=\kappa^s$. 
Mit Hilfe der im ersten Kapitel abgeleiteten Formel f\"ur die
Schwellenamplitude,
\be
 \left. E_{0+}\right|_{thr} = \frac{e}{16\pi M}
 \frac{2+\mu}{(1+\mu)^{3/2}} \, \left. \left(
   A_3 + \frac{\mu}{2} A_6 \right) \right|_{thr}
\ee                
und der in Anhang A diskutierten Kinematik
ergibt sich schlie\ss lich folgende Bestimmung der elektrischen
Dipolamplitude f\"ur die vier physikalischen Kan\"ale 
\beq
\label{LET}
\Epn &=& \frac{e}{4\pi} \frac{\sqrt{2}f}{m_\pi}
    \left\{ 1 - \frac{3}{2}\mu + {\cal O}(\mu^2) \right\}
    \cong 26.6 \su \\[0.1cm]
\Emp &=& \frac{e}{4\pi} \frac{\sqrt{2}f}{m_\pi}
     \left\{ -1 + \frac{1}{2}\mu + {\cal O}(\mu^2) \right\}
    \cong -31.7 \su \\[0.1cm]
\Eop &=& \frac{e}{4\pi} \frac{f}{m_\pi}
     \left\{ -\mu + \frac{\mu^2}{2}(3+\kappa_p ) +
  {\cal O}(\mu^3) \right\}    \cong -2.32
  \su \\[0.1cm]
\Eon &=& \frac{e}{4\pi} \frac{f}{m_\pi}
     \left\{  \frac{\mu^2}{2}\kappa_n  +
  {\cal O}(\mu^3) \right\}  \cong -0.51 \su .
\eeq
Dieses Resultat liefert den Inhalt des Niederenergietheorems
\cite{Bae70,VZ72}. Die relative Ordnung der nicht bestimmten
Korrekturen wurde mit Hilfe verschiedener Annahmen \"uber
das Verhalten der Untergrundamplitude festgelegt. Wir werden
diese Annahmen und ihre Rechtfertigung im n\"achsten Abschnitt
diskutieren.

Niederenergietheoreme zur Pionphotoproduktion lassen sich auch
direkt durch Baum-Niveau Rechnungen auf der Basis effektiver 
chiraler Meson-Nukleon Theorien ableiten \cite{Pec69}. Die
entsprechende Lagrangedichte unter Einbeziehung der 
elektromagnetischen Wechselwirkung lautet
\beq
\label{leff}
{\cal L} & =& \bar{\psi}(i\gamma\cdot{\cal D}-M)\psi 
  +\frac{1}{2}({\cal D}_\mu\phi^{a})^2 - \frac{1}{2}m_\pi^2
  (\phi^{a})^2  \\
 & & \mbox{} + \frac{f}{m_\pi} \bar{\psi}\gamma_5 \gamma_\mu
 \tau^{a} {\cal D}^\mu \phi^{a}\psi 
  + \frac{e}{4m}\bar{\psi} (\kappa^s +\kappa^v \tau^3)
  \sigma_{\mu\nu}\psi F^{\mu\nu} \nonumber 
\eeq
wobei ${\cal D}_\mu=\partial_\mu+iQ{\cal A}_\mu$ die kovariante
Ableitung, ${\cal A}_\mu$ das elektromagnetische Potential und
$Q$ den Ladungsoperator bezeichnet. Die Eichung der pseudovektoriellen
Pion-Nukleon Kopplung erzeugt einen $\gamma\pi NN$-Kontaktterm,
\be
{\cal L}_{\gamma\pi NN} = \frac{ef}{m_\pi}\epsilon^{3ab}
  \bar{\psi}\gamma_5 \gamma_\mu \tau^{a}\psi {\cal A}^\mu \phi^b
\ee  
der im Rahmen der effektiven Theorie die Kroll-Ruderman Amplitude
liefert. Die Wirkung der kovarianten Ableitung auf das Pionfeld
bestimmt die Kopplung des elektromagnetischen Feldes an die
geladenen Pionen. Die $\gamma\pi\pi$-Wechselwirkung 
\be  
{\cal L}_{\gamma\pi\pi} = e\epsilon^{3ab}\phi^{a}\partial_\mu
 \phi^{b} {\cal A}^\mu
\ee
bewirkt schlie\ss lich die bereits angesprochene Struktur des
Pionpolterms. 
   
    
\section{Absch\"atzung der Untergrundamplitude}
Um die Modellabh\"angigkeit des im letzten Abschnitt
vorgestellten Niederenergietheorems zu studieren, ist es
hilfreich die \"Ubergangsmatrix in der Form 
\be
 T_\mu^{a} = T_\mu^{a({\rm LET})} + \delta T_\mu^{a}
\ee
zu zerlegen. Dabei bezeichnet $T_\mu^{a(\rm LET)}$ die 
T-Matrix, die zu den invarianten Amplituden  (\ref{letamp})
geh\"ort, und $\delta T_\mu^{a}$ die vernachl\"assigte 
Untergrundamplitude. Nach dem Kroll-Ruderman Theorem gilt
\be
  \lim_{q,k\to 0} \delta T_\mu^{a} =0 \, ,
\ee
so da\ss\ $\delta T_\mu^{a}$ am weichen Punkt verschwindet.
Um zu untersuchen, in welcher Ordnung in der Pionmasse 
$\delta T_\mu^{a}$ zur Amplitude an der physikalischen Schwelle
beitr\"agt, definieren wir
\be
  \delta T_\mu^{a} = \bar{u}(p_2) \sum_{\lambda} 
   \delta A_\lambda^{a}(\nu,\nu_1) {\cal M}_\lambda u(p_1)
\ee
und nehmen an, da\ss\ sich die die invarianten Amplituden 
$\delta A_\lambda$ in eine Taylorreihe um den Punkt
$\nu=\nu_1=0$ entwickeln lassen
\be
 \delta A_\lambda^{a} (\nu,\nu_1) = a^{a}_{\lambda\, 00}
    + a^{a}_{\lambda\, 10} \nu + a^{a}_{\lambda\, 01}\nu_1
    + \ldots
\ee
Diese Annahme ist sinnvoll, da die einzigen Singularit\"aten bei 
$\nu,\nu_1=0$ von den Nukleon- und Pion-Poltermen stammen,
welche explizit in $T_\mu^{a({\rm LET})}$ ber\"ucksichtigt
sind. Dar\"uber hinaus wollen wir voraussetzen, da\ss\ alle
Koeffizienten $a^{a}_{\lambda\, ij}$ im Limes $m_\pi\to 0$
regul\"ar sind. Das bedeutet, da\ss\ sich diese Koeffizienten
beim Abz\"ahlen von Potenzen in $\mu$ als Gr\"o\ss en der
Ordnung ${\cal O}(1)$ betrachten lassen. Diese Annahme ist
allerdings problematisch, denn im allgemeinen k\"onnen
Pion-Schleifendiagramme Beitr\"age  \cite{LP71,PP71}liefern, 
die nichtanalytisch in $m_\pi$ sind\footnote{In \cite{BKG91}
wird die Gegenwart solcher Terme durch eine explizite 
Rechnung im Rahmen der chiralen St\"orungstheorie best\"atigt.
Dagegen bestreitet \cite{Nau91} diese M\"oglichkeit auf
Grund von \"Uberlegungen allgemeiner Natur.}. 
 
Die Amplitude $T_\mu^{a(\rm LET)}$ enth\"alt im isospinsymmetrischen 
Fall neben den Nukleon- und Pion-Polen nur einen einzigen Kontaktterm. 
Die Gegenwart dieses Terms unterscheidet die Bornamplituden in
pseudovektorieller bzw.~pseudoskalarer Kopplung und ist deshalb
eine Konsequenz der PCAC-Relation. Dieses Resultat l\"a\ss t sich
als eine Bedingung f\"ur die invariante Amplitude $A_6^{(+0)}$
formulieren \cite{AG66}
\be
\label{FFR}
 \lim_{\nu\to 0} \lim_{\nu_1\to 0} A_6^{(+0)} (\nu,\nu_1)
   =  \frac{4f}{\mu} \kappa^{v,s} \; .
\ee
Die Reihenfolge der beiden Grenz\"uberg\"ange in (\ref{FFR})
ist nicht beliebig. Sie ist so gew\"ahlt, da\ss\ der Polterm
keinen Beitrag zum Grenzwert liefert. 
Da der Kontaktterm (\ref{FFR}) bereits in $T_\mu^{a(\rm LET)}$
enthalten ist, verschwindet $a^{(+0)}_{6\,00}$ am weichen Punkt.    
In der klassischen Diskussion \cite{Bae70} wird diese Beobachtung 
zum Anla\ss\ genommen, $a^{(+0)}_{6\,00}$ identisch Null zu setzen. 
          
Nur $\delta A_3$ und $\delta A_6$ tragen zur elektrischen 
Dipolamplitude an der Schwelle bei. Mit Hilfe der 
Eichinvarianzbedingung (\ref{gaugecond}) und der Forderung nach korrektem
Verhalten der Amplituden unter der Austauschtransformation
$(\nu,\nu_1)\to(-\nu,\nu_1)$ l\"a\ss t sich die m\"ogliche
Form der Taylorentwicklungen f\"ur $\delta A_{3,6}$ erheblich
einschr\"anken. F\"ur die isospinsymetrischen Komponenten
findet man
\beq
 \delta A_{3}^{(+0)} &=& a_{3\, 11}^{(+0)} \nu\nu_1 + \ldots \\
 \delta A_{6}^{(+0)} &=& a_{6\, 01}^{(+0)} \nu_1
               + a_{6\, 20}^{(+0)} \nu^2
	       + a_{6\, 02}^{(+0)} \nu_1^2 + \ldots \; .
\eeq
An der Schwelle ist $\nu={\cal O}(\mu)$ und $\nu_1={\cal O}(\mu^2)$,
so da\ss\ die Austauschsymmetrie im wesentlichen das 
Transformationsverhalten der Amplitude unter $m_\pi\to -m_\pi$
spezifiziert. Auf Grund der Beziehung 
\be
\delta E_{0+} \sim \delta A_3 + \frac{\mu}{2} \delta A_6
\ee
folgt, da\ss\ die Untergrundamplitude $\delta E_{0+}^{(+0)}$
an der Schwelle von der Ordnung ${\cal O}(\mu^3)$ ist. Eine analoge
Argumentation l\"a\ss t sich auch f\"ur die isospinungeraden Komponenten 
durchf\"uhren. In diesem Fall findet man $\delta E_{0+}^{(-)}\sim 
{\cal O}(\mu^2)$.

\section{Die Methode von Furlan, Paver und Verzegnassi}
Die Ableitung des Niederenergietheorems im  Abschnitt 2.2
basierte im wesentlichen auf der Reduktionsformel und 
Wardidentit\"aten f\"ur die Zweipunktfunktion $\overline{\Pi}_{\mu\nu}^{ab}$.
In diesem Abschnitt wollen wir auf eine andere Methode eingehen, die
direkt mit Stromalgebrakommutatoren und der Clusterzerlegung arbeitet.
Der Vorteil dieses Verfahrens besteht darin, da\ss\ man direkt mit 
physikalischen Pionen arbeitet und die Rolle expliziter Symmetriebrechung 
daher mit ber\"ucksichtigt ist. So wurde im Rahmen dieser
Methode erstmals darauf hingewiesen, da\ss\ die explizite Brechung
der Symmetrie Korrekturen an das Standard-Niederenergietheorem liefern
kann \cite{FPV74,NS89}. Wir wollen diese Herleitung im
folgenden kurz vorstellen und dabei vor allem auf die Frage
eingehen, wie sie sich in das in dieser Arbeit angewandte Verfahren
einf\"ugen l\"a\ss t.

Nachdem wir im Abschnitt 2.2 Wardidentit\"aten f\"ur den Strom 
$B_\mu^{a}$ konstruiert haben, liegt es Nahe, die Kommutatoren der
zugeh\"origen Ladung
\be
 \bar{Q}^{a}_5(t) = Q^{a}_5(t) +  \frac{i}{m_\pi^2}\,
 \frac{d}{dt} \, \int d^3x\, D^{a} (\vec{x},t)
\ee
zu untersuchen. Der Einfacheit halber verwenden wir  die
leicht abgewandelte Definition
\beq
\label{q5l}
 \qfl (t) &=& Q_5^{a}(t) +\frac{i}{m_\pi}\dot{Q}^{a}_5(t) \\
\label{q5r} 
 \qfr (t) &=& \left( \qfl (t)\right)^\dagger 
                =  Q_5^{a}(t) -\frac{i}{m_\pi}\dot{Q}^{a}_5(t)  \; .
\eeq
Der Operator $\qfl$ hat dieselben Pionmatrixelemente wie die zu dem 
Strom $B_\mu^{a}$ geh\"orende Ladung $\bar{q}^{a}_5$. Die relevanten
Matrixelemente lauten  
\beq
  <0|\qfl |\pi^{b}(q)> &=& \spm 2if_\pi m_\pi \,\delta^{ab}
                           (2\pi)^3  \delta^3 (\vec{q}\,) \\[0.2cm]  
  <\pi^{b}(q)|\qfr |0> &=& -2if_\pi m_\pi \,\delta^{ab}
                            (2\pi)^3 \delta^3 (\vec{q}\,) \\[0.2cm]
 <\pi^{b}(q)|\qfl |0>&=& <0|\qfr |\pi^{b}(q)> = 0 \; .
\eeq
Die axialen Ladungen $Q_{5\,{\mini L,R}}^{a}$ sind nicht hermitesch
und haben die Eigenschaft, zwischen Pionen im Eingangs- und Ausgangskanal 
zu unterscheiden. Diese Tatsache erwei\ss t sich als besonders n\"utzlich 
bei der  Konstruktion von Summenregeln, da sie es erm\"oglicht,
bestimmte Prozesse in der Vollst\"andigkeitssumme zu selektieren. 

Im folgenden betrachten wir konkret Summenregeln f\"ur das
Matrixelement 
\be
 M_\mu^{a} = <N(p_2)| [\qfl ,V_\mu^{em}(0)]|N(p_1)>
\ee
indem sich mit Hilfe der oben angegebenen Matrixelemente die 
Photoproduktionsamplitude identifizieren l\"a\ss t. Das Resultat
besitzt eine sehr \"ubersichtliche Struktur als Summe von Poltermen 
und einem Dispersionsintegral, das die Untergrundamplitude 
repr\"asentiert. 

Da der Operator $\qfl$ nur Pionen in Ruhe produziert, 
verlangt die Berechnung der Schwellenamplitude die Kenntnis
von $M_\mu^{a}$ im Schwerpunktsystem. Um die folgende Rechnung
etwas zu vereinfachen, werden wir $M_\mu^{a}$ statt dessen an
der Breitschwelle berechnen, das hei\ss t f\"ur Pionen, die
im Breitsystem des Nukleons ruhen
\be
\begin{array}{rclcrcl}
  \vec{p}_2 &=&\spm \vec{p} &\hspace{1cm} & \vec{q} &=& 0 \\[0.2cm]
  \vec{p}_1 &=&-\vec{p}     &\hspace{1cm} & \vec{k} &=& 2\vec{p}
\end{array}
\ee
Diese Vereinfachung verursacht nur einen geringen Fehler im Vergleich
zur korekten Schwellenkonfiguration. An der Breitschwelle sind die
Werte der Mandelstamvariablen
\beq
s_{br} &=& (M+m_\pi)^2+ {\cal O}(m_\pi^3) \\
t_{br} &=& -m_\pi^2 
\eeq
gegen\"uber $s_{th}=(M+m_\pi)^2$ und $t_{th}=-m_\pi^2/(1+\mu)$ an
der Schwelle im Schwerpunktsystem.

Die eine Seite der Summenregel f\"ur $M_\mu^{a}$ ergibt sich, indem der
Kommutator 
\be
\label{comqfl}
\, [\qfl,V_\mu^{em}(0)] = [Q_5^{a},V_\mu^{em}(0)]+\frac{i}{m_\pi}
 [\dot{Q}_5^{a},V_\mu^{em}(0)]
 \ee
direkt ausgwertet wird. Dabei findet man den  Kroll-Ruderman Term sowie
den bereits diskutierte Beitrag der expliziten Symmetriebrechung. Die 
andere Seite der Summenregel ergibt sich aus der Vollst\"andigkeitsrelation 
f\"ur den Kommutator (\ref{comqfl}).

Die Clusterzerlegung \cite{AFF73} ist eine systematische Methode, 
um die verschiedenen Beitr\"age zur Vollst\"andigkeitssumme
\beq
\sum_n <N(\vec{p}\,)|\qfl |n><n|V_\mu^{em}|N(-\vec{p}\,)>
\eeq
zu identifizieren. Sie tr\"agt der Tatsache Rechnung, da\ss\
in einer relativistischen Theorie Beitr\"age von Zust\"anden 
mit unterschiedlichen Teilchenzahlen auftreten k\"onnen. Konkret 
zerlegt man $M_\mu^{a}$ in der Form    
\begin{figure}
\label{diag}
\caption{Beitr\"age zur Vollst\"andigkeitssumme f\"ur das
Operatorprodukt $V_\mu^{em}\qfl$.}
\vspace{7.5cm}
\end{figure}
\beq
\label{cluster}
M_\mu^{a\;\;}  &=& M_\mu^{a\,I}+M_\mu^{a\,II}  \\
M_\mu^{a\,I\,} &=& \sum_\alpha <N(\vec{p})|\qfl |\alpha>_c
                             <\alpha|V_\mu^{em}|N(-\vec{p}\,)>_c \\   
 & &        \hspace{0.5cm} -  \sum_\beta <0|\qfl |N(-\vec{p}\,)\beta>
              <N(\vec{p}\,)\beta|V_\mu^{em}|0>\; +\; {\em c.~t.} \nonumber \\
M_\mu^{a\,II} &=& \sum_{\gamma_1} <N(\vec{p}\,)|\qfl |N(-\vec{p}\,)\gamma_1>_c
                             <\gamma_1|V_\mu^{em}|0> \\   
 & &       \hspace{0.5cm} +  \sum_{\gamma_2} <0|\qfl |\gamma_2>
       <\gamma_2 N(\vec{p}\,)|V_\mu^{em}|N(-\vec{p}\,)>\; +\; 
       {\em c.~t.} \nonumber
\eeq
wobei der Index $c$  den zusammenh\"angenden Teil des Matrixelements
und $c.t.$  die Voll\-st\"an\-dig\-keitssumme mit den Operatoren in der
anderen Reihenfolge bezeichnet. Der erste Teil der Clusterzerlegung
beinhaltet solche Zust\"ande, die Baryonenzust\"ande tragen. Die
f\"uhrenden Beitr\"age zu diesem Terme ergeben sich f\"ur Nukleonzust\"ande
$|\alpha>=|N(\vec{p}\,)>$ und Antinukleonzust\"ande $|\beta>=|\bar{N}
(\vec{p}\,)>$. Der zweite Teil von (\ref{cluster}) beschreibt die Produktion
eines Zustands $\gamma_{1,2}$ aus dem Vakuum, gefolgt von der Reaktion
$\gamma_1 +N(p_1) \to \qfl + N(p_2)$ bzw.~ $V_\mu^{em}+N(p_1)
\to \gamma_2 + N(p_2)$. Insbesondere findet man f\"ur Pionzust\"ande
$|\gamma_2>=|\pi^{a}(\vec{q}\,)>$ die Photoproduktionsamplitude
\be
\sum_{\pi^{b}(\vec{q})} <0|\qfl |\pi^{b}(\vec{q}\,)><\pi^{b}(\vec{q}\,)
  N(\vec{p}\,)|V_\mu^{em}|N(-\vec{p}\,)> = f_\pi T_\mu^{a}(\vec{q}=0)
\ee
Auf Grund der speziellen Eigenschaften des Operators  $\qfl$
enth\"alt die Vollst\"andigkeitssumme keine Beitr\"age von der
inversen Reaktion $\pi^{a}(q)+N(p_1)\to V_\mu^{em}+N(p_2)$. Isoliert
man die Photoproduktionsamplitude $T_\mu^{a}$ und separiert 
die Nukleonbeitr"age in $M_{\mu}^{a\, I}$, so ergibt sich 
schlie\ss lich folgender Ausdruck f\"ur $T_\mu^{a}$
\beq
\label{fpv}
f_{\pi} T_{\mu}^{\alpha}(\vec{q}=0) &=&
 i \epsilon^{\alpha 3 \gamma} <N(\vec{p}\,)|A_{\mu}^{\gamma}|N(-\vec{p}\,)> 
                   \\[0.3cm]
   & &\mbox{}-\sum_{N(\vec{p}_n)} <N(\vec{p}\,)|\qfl |N(\vec{p}_n)>
   <N(\vec{p}_n)|V_{\mu}^{em}|N(-\vec{p}\,)> 
           \;+ \; {\em c.~t.} \nonumber \\
   & &\mbox{}  + \frac{i}{m_\pi}<N(\vec{p}\,)|[\dot{Q}_5^{\alpha},V_{\mu}^{em}]
    |N(-\vec{p}\,)> 
    \; + \;  f_\pi \delta T_\mu^{a}  \nonumber ,
\eeq
wobei $\delta T_\mu^{a}$ die vernachl\"assigten Beitr\"age in der
Vollst\"andigkeitssumme bezeichnet. Dabei handelt es sich vor allem um
$\pi N$-Kontinuumszust\"ande und Antinukleonen in $M_\mu^{a\, I}$ 
sowie Vektormesonen in $M_\mu^{a\, II}$.
Die beiden Kommutatoren in (\ref{fpv}) liefern wie in (\ref{avward})
den Kroll-Ruderman Term und die Korrekturen auf Grund expliziter
Symmetriebrechung.

Mit Hilfe der Eigenschaften des Operators $\qfl$ l\"a\ss t sich folgende
Darstellung der Hintergrundamplitude ableiten \cite{AFF73}
\be
\label{ressum}
\delta T_\mu^{a} = -im_\pi \sum_{n\neq\pi,N} (2\pi)^3 \delta^3 
  (\vec{p}-\vec{p}_n) 
  \frac{ <N(\vec{p}\,)|j_\pi^{a}|n><n|V_\mu^{em}|N(-\vec{p}\,)> }{ 
       (E_p-E_n)(E_p+m_\pi-E_n+i\epsilon) }
  \;-\; c.t.
\ee
Die Summation l\"auft \"uber beliebige intermedi\"are Zust\"ande 
mit Ausnahme von Nukleonen und Pionen. $E_p=(\vec{p}^{\,2}+M^2)^{1/2}$
bezeichnet die Energie des auslaufenden Nukleons, $E_n$ die Energie
des Zwischenzustands. Der Energienenner in (\ref{ressum}) verschwindet
nur f\"ur Nukleonzust\"ande. Die obige Darstellung zeigt daher explizit,
da\ss\ alle anderen Beitr\"age im Limes $m_\pi \to 0$ unterdr\"uckt
sind. 

Wir betrachten nun im Einzelnen die verschiedenen Beitr\"age zur
Photoproduktionsamplitude (\ref{fpv}). Den Kroll-Ruderman Term 
und den Sigmakommutator haben wir bereits im Abschnitt 1.2 
diskutiert. Um den Nukleonterm zu berechnen, ben\"otigen wir die
Matrixelemente 
\beq
  <N(\vec{p}\,)|A_0^{a}(0)|N(\vec{p}\,)> &=&
     \frac{g_A}{M} \,\chi^\dagger_f (\vec{\sigma}\cdot\vec{p}\,)
     \frac{\tau^{a}}{2} \chi_i  \\  
 <N(\vec{p}\,)|V_0^{em}(0)|N(-\vec{p}\,)> &=&
     e \,\chi^\dagger_f (G_E^s (t) + \tau^3 G_E^v (t) ) \chi_i \\[0.1cm]
 <N(\vec{p}\,)|\vec{V}^{em}(0)|N(-\vec{p}\,)> &=&
     \frac{e}{M} \,\chi^\dagger_f (G_M^s(t) + \tau^3 G_M^v(t))  
     i(\vec{\sigma}\times\vec{p}) \chi_i 
\eeq     
Im Breitsystem treten die elektrischen und magnetischen Formfaktoren
des Nukleons
\beq
  G_E(t) &=& F_1(t)+\frac{t}{4M^2}F_2(t)  \\[0.1cm]
  G_M(t) &=& F_1(t)+F_2(t)
\eeq
auf. Da wir bereits ein spezielles Bezugssystem gew\"ahlt haben,
ist es sinnvoll die $T$-Matrix in einer nicht kovarianten Form
anzugeben. Der Nukleonbeitrag zur Photoproduktionsamplitude an  
der Breitschwelle $t=m_\pi^2$ ergibt sich schlie\ss lich zu
\beq
   T_0^{a}  &=& -\frac{g_A}{f_\pi}\,\frac{1}{E_p}
        \chi^\dagger_f (G_E^s(t) \tau^{a} + G_E^v \delta^{a3})
	(\vec{\sigma}\cdot\vec{p}\,)\chi_i  \\
\vec{T}^{a} &=& \spm \frac{g_A}{f_\pi} \, \frac{\vec{p}^{\, 2}}{mE_p}
       G_M^v(t) \,\chi^\dagger_f \frac{1}{4} [\tau^{a},\tau^3] 
       \,\vec{\sigma}_{\mini T}\, \chi_i
\eeq
wobei wir die transversalen und longitudinalen Spins 
\beq
   \vec{\sigma}_{\mini T} &=& \vec{\sigma} - \hat{p}(\vec{\sigma}
              \cdot\hat{p}) \\
   \vec{\sigma}_{\mini L} &=&  \hat{p}(\vec{\sigma}\cdot\hat{p})	      
\eeq
eingef\"uhrt haben. Nur der transversale Anteil liefert einen Beitrag
zur Photoproduktion  mit reellen Photonen. Im Falle des Nukleonterms
ist dieser Beitrag von der Ordnung $m_\pi^2$ und proportional zum
magnetischen Moment des Nukleons.   	             
      
Als pseudoskalares Teilchen koppelt das Pion stark an die unteren
Komponenten der Nukleonspinoren. Der f\"uhrende Beitrag zur Produktion
neutraler Pionen kommt daher von Zwischenzust\"anden, die propagierende 
Antinukleonen enthalten ('Z-Graphen`). Mit Hilfe der Darstellung
(\ref{ressum}) findet man
\be
 \vec{T}^{a} = -\frac{g_A}{f_\pi} \frac{m_\pi}{2E_p} \,
     \chi^\dagger_f (G_E^s(t)\tau^{a} + G_E^v(t) \delta^{a3})
     \vec{\sigma} \chi_i
\ee
wobei wir h\"ohere Ordnungen in $m_\pi/M$ vernachl\"assigt haben.
Damit ist die elektrische Dipolamplitude bis zur Ordnung $m_\pi^2$
bestimmt. Vernachl\"assigt man den Beitrag aus der expliziten
chiralen Symmetriebrechung, so ergibt sich
\beq
\label{LET2}
\Epn &=& \frac{e}{4\pi} \frac{g_A}{\sqrt{2}f_\pi}
    \left\{ 1 - \frac{3}{2}\mu + {\cal O}(\mu^2) \right\}
    \cong 24.1  \su \\
\Emp &=& \frac{e}{4\pi} \frac{g_A}{\sqrt{2}f_\pi}
     \left\{ -1 + \frac{1}{2}\mu + {\cal O}(\mu^2) \right\}
    \cong -29.6  \su \\
\Eop &=& \frac{e}{4\pi} \frac{g_A}{2f_\pi}
     \bigg\{ -\mu + {\cal O}(\mu^2) \bigg\}  \cong -3.3 \su
\eeq
Die elektrische Dipolamplitude f\"ur die Produktion neutraler
Pionen an Neutron verschwindet in dieser Ordnung. Die von
(\ref{LET}) abweichenden Werte in den geladenen Kan\"alen 
ergeben sich wie oben diskutiert aus der Verwendung der 
axialen Koopplung $g_A=1.26$ an Stelle der pseudovektoriellen
Pion-Nukleon Kopplung.

\section{Explizite chirale Symmetriebrechung}
Die r\"aumlichen Komponenten des Beitrags aus der expliziten
chiralen Symmetriebrechung 
\be
\label{csbcom}
 \Sigma_\mu^{a}(\vec{q}=0) = 
  \int d^4 x \,\delta (x^0) [\partial^\nu A_\nu^{a}(x),
  V_\mu^{em}(0)]
\ee
sind nicht durch Stromalgebra festgelegt. Aus diesem Grund haben 
wir Ihren Beitrag zur Photoproduktionsamplitude bislang 
vernachl\"assigt. Repr\"asentiert man jedoch die Str\"ome
durch Quarkfelder, so ist auch dieser Kommutator durch die
kanonischen Vertauschungsregeln der Felder bestimmt. 
Der Vollst\"andigkeit halber arbeiten wir in Flavor-$SU(3)$,
so da\ss\
\beq
   \partial^\nu A_\nu^{a} &=& \frac{i}{2} \bar{\psi} \gamma_5
      \left\{ M,\lambda^{a} \right\} \psi  \\
    V_\mu^{em}            &=& \frac{1}{2} \bar{\psi} \gamma_\mu
      ( \lambda^3 + 1/\sqrt{3} \lambda^8 ) \psi
\eeq
mit $M={\rm diag}(m_u,m_d,m_s)$. Es wird sich allerdings zeigen, da\ss\
die Masse der seltsamen Quarks nicht in das Resultat eingeht.
Der Kommutator der beiden Bilinearformen l\"a\ss t sich mit Hilfe 
der Relation              
\beq
\label{bilcom}
 \lefteqn{\delta (x^0-y^0) [\psi^\dagger (y)\frac{\lambda^{a}}{2}
      \Gamma \psi (y),\psi^\dagger (x)\frac{\lambda^{b}}{2}
      \Gamma' \psi (x)] = }  \\
    & & \hspace{1cm}   \frac{1}{2} \delta^4 (x-y) 
      \psi^\dagger (x) \left( if^{abc} \{\Gamma,\Gamma'\} 
      + id^{abc} [\Gamma,\Gamma' ] \right) \frac{\lambda^c}{2}
      \psi (x) \nonumber
\eeq
auswerten. Dabei bezeichnen $\Gamma$ und $\Gamma'$ die Diracoperatoren,
$f^{abc}$ und $d^{abc}$ die antisymmetrschen bzw.~symmetrischen $SU(3)$
Strukturkonstanten. Mit Hilfe von (\ref{bilcom}) ergibt sich f\"ur
$a=1,2,3$
\be
\label{sig0q}
\int d^4x\, \delta (x^0) [\partial^{\nu}A_{\nu}^{a}(x),
   V_{0}^{em}(0)] = i\,\overline{m} \,\epsilon^{3ab} \bar{\psi}
   \gamma_5\lambda^b \psi\hspace{4cm}
\ee
und
\beq   
\label{sigcom}
\int d^4 x\, \delta (x^0) [\partial^{\nu}A_{\nu}^{a}(x),
        V_{i}^{em}(0)] & =&
i\,\overline{m} \,\epsilon_{ijk} \left\{  \delta^{a3}
\frac{1}{\sqrt{3}}\left( \sqrt{2} J_{jk}^{0}+J_{jk}^{8} \right) +
 \frac{1}{3} J_{jk}^{a}\right\}    \\
& &\mbox{} + i\frac{\delta m}{2}\epsilon_{ijk}\delta^{a3} \left\{
 \frac{1}{3\sqrt{3}}\left(\sqrt{2} J_{jk}^{0}+ J_{jk}^{8} \right)
  + J_{jk}^{3}  \right\}  \nonumber
\eeq
wobei wir die Tensorstr\"ome
\be
 J_{\mu\nu}^{c} = \bar{\psi}\sigma_{\mu\nu}\frac{\lambda^c}{2}\psi
\ee
eingef\"uhrt haben. In der von uns verwendeten Normierung ist
$\lambda^0=\sqrt{2/3}\,{\bf 1}$. Man beachte, da\ss\ $1/\sqrt{3}
(\sqrt{2}\lambda^0 +\lambda^8)$ gerade die Einheitsmatrix im 
$SU(2)$-Unterraum ist. Die Str\"ome (\ref{sig0q},\ref{sigcom})
enthalten daher keine Beitr\"age der seltsamen Quarks. Die St\"arke
der chiralen Symmetriebrechung sowie der Isopsinbrechung wird
durch die Parameter
\beq
  \overline{m} &=& \frac{1}{2}(m_u+m_d)  \\
  \delta m     &=& m_u -m_d
\eeq
kontrolliert. Die Zeitkomponente des Sigmakommutators ist im
Fall verschwindender Isopsinbrechung identisch mit der Divergenz des
Axialstroms gegeben.  Das Ergebnis (\ref{sig0q}) reproduziert 
daher das Stromalgebraresultat (\ref{sig0}). Insbesondere
lassen sich Nukleonmatrixelemente der pseudoskalaren Dichte
$\bar{\psi}\gamma_5\lambda^b\psi$ mit Hilfe des Pion-Nukleon
Formfaktors $G_{\pi NN}$ ausdr\"ucken und liefern den bereits
diskutierten Beitrag zum Pionpolterm. 

Die r\"aumlichen Komponenten des Kommutators (\ref{csbcom}) 
liefern dagegen mit den oben definierten Tensorstr\"omen 
einen v\"ollig neuen Beitrag zur Photoproduktionsamplitude.
Dieser Beitrag verschwindet am weichen Punkt $q=0$ und 
bestimmt daher die Extrapolation der Amplitude zur
physikalischen Schwelle.

Die allgemeinste Form des Nukleonmatrixelements der Tensorstr\"ome 
lautet
\beq
  <N(p_2)|\bar{\psi}\sigma_{\mu\nu}\tau^{a}\psi|N(p_1)> &=& 
        \bar{u}(p_2) \left[
     G_T^{a}(t) \sigma_{\mu\nu} + iG_2^{a}(t)
     \frac{\gamma_\mu \Delta_\nu - \Delta_\mu \gamma_\nu}{2M} 
     \right. \\
 & & \mbox{}+ \left. iG_3^{a}(t) 
     \frac{\Delta_\mu P_\nu - P_\mu \Delta_\nu}{M^2}
     + iG_4^{a} \frac{\gamma_\mu P_\nu - P_\mu \gamma_\nu}{M^2}    
     \right] \tau^{a} u(p_1) \nonumber
\eeq
wobei $\Delta_\mu=(p_2-p_1)_\mu$ den Impuls\"ubertrag und $\tau^0 ={\bf 1}$ 
sowie $\tau^{a}\; (a=1,2,3)$ die Paulimatrizen bezeichnet.
Das Matrixelement vereinfacht sich erheblich, wenn man die
r\"aumlichen Komponenten der Str\"ome im Breitsystem des Nukleons
betrachtet
\beq
   <N(\vec{p}\,)|\bar{\psi}\sigma_{jk}\tau^a\psi|N(-\vec{p}\,)>    
  & = & \epsilon_{jkm}\chi^\dagger_f \left[
     \left( G_T(t) +\frac{t}{4M^2} G_2(t)\right)\sigma_{{\mini T}m} 
    \right. \\
 & & \hspace{2.7cm} \mbox{} + \left. G_T(t)\frac{E_p}{M} \sigma_{{\mini L}m}
 \right] \tau^{a} \chi_i  \nonumber
\eeq
Bis auf Korrekturen der Gr\"o\ss enordnung $m_\pi^2$ kann man die
Formfaktoren durch ihren Wert bei $t=0$ ersetzen.
Mit der Definition $g_T=G_T(0)$ ergibt sich schlie\ss lich die 
Korrektur zur Schwellenamplitude f\"ur neutrale Pionen 
\be
\label{delneu}
\Delta E_{0+}(\pi^0 N) = \frac{e}{4\pi f_\pi}\frac{\overline{m}}{m_\pi (1+\mu)}
  \left\{ \left( 1+\frac{\delta m}{6\overline{m}} \right) g_T^0
     \pm \left(\frac{1}{3}+\frac{\delta m}{2\overline{m}}\right) g_T^3
     \right\} \; ,
\ee
wobei das sich das Vorzeichen auf die Produktion am Proton 
bzw.~Neutron bezieht. Verwendet man die oben zitierten  Werte der
Quarkmassen, so ist $\delta m/(2\overline{m}) \simeq -1/3$ und
$\Delta E_{0+}(\pi^0N)$ ist fast vollst\"andig durch die Tensorkopplung
im Singletkanal bestimmt. Man beachte, da\ss\ der Korrekturterm formal
von der Ordnung $m_\pi$ ist, denn nach der GOR Relation gilt $\overline{m}
= m_\pi^2f_\pi^2/|<\bar{u}u+\bar{d}d>|$. 

Die entsprechende Korrektur f\"ur die Produktion geladener Pionen
lautet
\be
\label{delchar}
 \Delta E_{0+}(\pi^-p)=\Delta E_{0+}(\pi^+n) =
  \frac{\sqrt{2}e}{4\pi f_\pi}\frac{\overline{m}}{m_\pi (1+\mu)}
  \,\frac{g_T^3}{3}\; .
\ee  
In diesem Fall tr\"agt der isospinbrechende Term proportional
zu $\delta m$ nicht bei. Der Korrekturterm modifiziert nicht
die Ladungsasymmetrie $|\Epn|-|\Emp|$, liefert aber einen
kleinen Beitrag zum Panofskyverh\"altnis $\Epn/\Emp$.

Die wesentliche Aufgabe bei der Berechnung von $\Delta E_{0+}$
ist nun die Bestimmung der Tensorkopplungskonstanten
$g_T^{a}$ des Nukleons. Diese sind leider experimentell
nicht direkt zug\"anglich, so da\ss\ man in diesem 
Zusammenhang auf Modelle angewiesen bleibt.
Die einfachste M\"oglichkeit ist die Verwendung eines
nichtrelativistischen Konstituentenmodells zur 
Beschreibung der Struktur des Nukleons. In diesem Fall
reduzieren sich die Tensorstr\"ome $\frac{1}{2}\epsilon_{ijk}
\bar{\psi}\sigma_{jk}\psi$ auf Axialstr\"ome $\bar{\psi}
\gamma_i\gamma_5 \psi$. Die Korrektur zur elektrischen 
Dipolamplitude lautet dann
\be
 \DEop = \frac{e}{4\pi f_\pi}\frac{\overline{m}}{m_\pi (1+\mu)}
    (0.90 \cdot g_A^0 + 0.04 \cdot g_A^3) \; .
\ee
In einem nichtrelativistischen Quarkmodell ist $g_A^0=1$ und $g_A^3=5/3$,
so da\ss\ $\DEop = 1.6 \su$. Diese Korrektur ist von derselben 
Gr\"o\ss enordnung wie der f\"uhrende Term des Niederenergietheorems,
$\DEop = -2.3\su$, und besitzt dar\"uber hinaus das umgekehrte Vorzeichen.
        

\section{Eichinvarianz}
Die Forderung nach Eichinvarianz der \"Ubergangsmatrix $T_\mu^{a}$
liefert wichtige Einschr\"ankungen f\"ur die Form der invarianten Amplituden.
Im Falle von Pionen auf der Massenschale ergeben sich diese 
Bedingungen aus der Erhaltung des elektromagnetischen Stroms
im \"Ubergangsmatrixelement
\be
\label{ongi}
k^\mu T_\mu^{a} = ie<\pi^{a}(q)N(p_2)|\partial^\mu V_\mu^{em}(0)|N(p_1)>
=0 \; .
\ee
Die aus dieser Gleichung folgenden Beziehungen (\ref{gaugecond}) haben 
wir bereits im ersten Kapitel angegeben. Man pr\"uft leicht nach, da\ss\ 
die in Abschnitt 2.2 abgeleiteten Amplituden diese Bedingungen erf\"ullen.
Dies gilt jedoch nur f\"ur die Summe von Stromalgebra-, Nukleon- und
Pionpolbeitr\"agen. Keiner dieser Terme ist f\"ur sich genommen 
eichinvariant. 

Die nicht eichinvarianten Terme in den einzelnen Beitr\"agen heben
sich allerdings nur dann gegenseitig weg, wenn keine ph\"anomenologischen 
Formfaktoren an den Vertices verwendet werden. Um die Rolle
der Formfaktoren n\"aher zu untersuchen, wollen wir unsere 
Betrachtungen auf die Elektroproduktion von Pionen erweitern.
In diesem Fall ist das ausgetauschte Photon virtuell und besitzt 
eine nicht verschwindende invariante Masse $k^2$.  Die Kopplung
des Photons wird durch die elektrischen Formfaktoren des 
Nukleons sowie des Pions 
\beq
 \Gamma_\mu^\gamma &=& F_1(k^2) \gamma_\mu + \frac{i\sigma_{\mu\nu}
               k^\nu}{2M} F_2(k^2) \\
 \Gamma_\mu^{\gamma\pi} &=& F_\pi (k^2)(2q-k)_\mu
\eeq
beschrieben. Dar\"uber hinaus liefert der Stromalgebraterm
\be
 C_\mu^{a} = -i\epsilon^{a3c} F_A(t) g_A \bar{u}(p_2)\gamma_\mu
    \gamma_5 \frac{\tau^c}{2} u(p_1)
\ee    
einen Beitrag, welcher den normierten axialen Formfaktor
$F_A(t)=G_A(t)/G_A(0)$ enth\"alt. Wie im Falle reeller 
Photonen lautet die Eichinvarianzbedingung $k^\mu T_\mu^{a}=0$.
Wir zerlegen  die Amplitude in der Form
\be
 T_\mu^{a} = T_\mu^{a(Born)} + \Delta T_\mu^{a} + T_\mu^{a(Res)}
\ee
wobei $T_\mu^{a(Born)}$ die Polterme sowie des Stromalgebrabeitrag
enth\"alt. Der Korrekturterm $\Delta T_\mu^{a}$ ist durch die Bedingung
\be
 k^\mu ( T_\mu^{a(Born)}+\Delta T_\mu^{a}) =0
\ee
definiert, w\"ahrend $T_\mu^{a(Res)}$ eine Untergrundamplitude bezeichnet,
die bis auf die Eichinvarianzforderung $k^\mu T_\mu^{a(Res)}=0$  unbestimmt 
bleibt.

Ber\"ucksichtigt man die Formfaktoren an den Vertices, so ist die
Divergenz des isopsinantisymmetrischen Teils der Bornmaplitude
\beq
\label{ngi}
 k^\mu T_\mu^{(-)(Born)} &=& \frac{ief}{m_\pi} \bar{u}(p_2)\Big(
          2M (2F_1^v(k^2) - F_\pi (k^2) ) \\
   & & \hspace{3cm} \mbox{} - \gamma\cdot k 
	  (2F_1^v(k^2) - F_A(t)) \Big) \gamma_5 u(p_1) \nonumber
\eeq 
Alle anderen Isospinkomponenten erf\"ullen die Eichinvarianzbedingung.
F\"ur die $(-)$-Komponente ist dies nur am Photonpunkt $k^2=0$ der
Fall. Um eine eichinvariante Amplitude zu erhalten, mu\ss\ man einen
Korrekturterm \cite{VZ72,SK91}
\beq
\label{gcor}
\Delta T_\mu^{(-)} &=& -\frac{ief}{m_\pi} \bar{u}(p_2)\left(
          \frac{2Mk_\mu}{k^2} (2F_1^v(k^2) - F_\pi (k^2) ) \right.\\
 & & \hspace{3cm} \mbox{}	  
	  - \left. \frac{k_\mu\gamma\cdot k}{k^2} (2F_1^v(k^2) - F_A(t))
	   \right) \gamma_5 u(p_1) \nonumber
\eeq 
addieren. Dieser Term ist nicht eindeutig bestimmt. Jeder beliebige
Ausdruck, der sich von (\ref{gcor}) nur um einen divergenzfreien
Beitrag unterscheidet, ist ebenfalls ein m\"oglicher Korrekututerm.
Die Summe $T_\mu^{a(Born)}+\Delta T_\mu^{a}$ liefert schlie\ss lich
eine eichinvariante Elektroproduktionsamplitude. 

Es ist instruktiv, die Konsequenzen von Eichinvarianz auch f\"ur 
Pionen abseits der Massenschale zu untersuchen. Dieses Problem
ist vor allem  bei der Bestimmung der Amplitude am weichen Punkt
von Bedeutung. Da sich das Pion nicht in einem asymptotischen
Zustand befindet, kann man zu diesem Zweck allerdings nicht von
Gleichung (\ref{ongi}) Gebrauch machen.  Statt dessen betrachten 
wir die zu (\ref{avward}) analoge Vektorwardidentit\"at
\be
\label{vwi}
ik^\mu \overline{\Pi}_{\nu\mu}^\alpha (q) = - C^\alpha_\nu + 
\frac{i}{m_\pi^2} \big( q_\nu \Sigma_0^\alpha (q) - 
\delta_{\nu 0} k^\rho \Sigma_\rho^\alpha (q) \big) .
\ee
Auch diese Relation beruht auf der Erhaltung des elektromagnetischen 
Stroms. Sie enth\"alt aber keine zus\"atzlichen Annahmen \"uber den
Impuls des Pions. In Verbindung mit der Axialvektorwardidentit\"at
(\ref{avward}) ergibt sich folgender Ausdruck f\"ur die Divergenz
von $T_\mu^{a}$
\be
\label{offgi}
 k^\mu T_\mu^{a} = -i\epsilon^{a3c} \frac{q^2-m_\pi^2}{f_\pi m_\pi^2}
   <N(p_2)|D^c(0)|N(p_1)> \; .
\ee
F\"ur Pionen auf der Massenschale ergibt sich die bekannte Beziehung
$k^\mu T_\mu^{a} =0$. Abseits der Massenschale liefern geladene 
virtuelle Pionen einen zus\"atzlichen Quellterm f\"ur den elektromagnetischen
Strom und bewirken eine nichtverschwindende Divergenz von $T_\mu^{a}$.

Bei der Herleitung der Relation (\ref{offgi}) ben\"otigt man keine 
Annahmen \"uber die modellabh\"angigen Komponenten der symmetriebrechenden
Amplitude $\Sigma_\mu^{a}$. Betrachtet man die einzelnen Beitr\"age 
zur linken Seite von (\ref{offgi}),
\be
\label{divamp}
 k^\mu T_\mu^{a} = \frac{1}{f_\pi} \left\{ ik^\mu q^\nu 
   \overline{\Pi}_{\mu\nu}^{a}
   -k^\mu C_\mu^{a} +\frac{i\omega_\pi}{m_\pi^2} k^\mu \Sigma_\mu 
   \right\}
\ee     
so tragen diese Terme jedoch bei. Die Polterme erf\"ullen in Verbindung
mit dem Stromalgebrabeitrag auch die verallgemeinerte Eichinvarianzbedingung
(\ref{offgi}). Man kann diese Terme daher aus der Gleichung (\ref{divamp})
eliminieren. In der Herleitung des Niederenergietheorems vernachl\"assigt
man Untergrundbeitr\"age zu den Amplituden. Die Gleichung (\ref{divamp})
reduziert sich daher auf eine Beziehung f\"ur die symmetriebrechende
Amplitude: $k^\mu \Sigma_\mu^{(+0)}=0$. Die im Abschnitt 2.5 bestimmten
Beitr\"age erf\"ullen diese Gleichung nicht. Wir definieren daher den 
eichinvarianten Teil von $\Sigma_\mu^{a}$
\be
  \Sigma_\mu^{a(gi)} = \Sigma_\mu^{a} +\Delta\Sigma_\mu^{a}
\ee
durch die Forderung $k^\mu\Sigma_\mu^{(+0)(gi)}=0$. Die isospinungeraden
Komponenten liefern die rechte Seite von (\ref{offgi}). Eine L\"osung
dieser Bedingungen lautet
\beq
 \frac{\omega_\pi}{m_\pi^2}\Sigma_\mu^{(-)(gi)} &=& -
                 \frac{f_\pi}{m_\pi^2-t}\, g_{\pi NN}
                   \, \bar{u}(p_2)i\gamma_5  q_\mu u(p_1) \\
 \frac{\omega_\pi}{m_\pi^2}\Sigma_\mu^{(0)(gi)} &=& \spm
             \frac{4\overline{m}M}{m_\pi^2} \,\frac{g_T^3}{3}
	 \, \bar{u}(p_2)i\gamma_5 \frac{\gamma_\mu \gamma\cdot k}{2M}u(p_1) \\
 \frac{\omega_\pi}{m_\pi^2}\Sigma_\mu^{(+)(gi)} &=& \spm
             \frac{4\overline{m}M}{m_\pi^2} \,
	     \left\{ g_T^0 \left(  1+\frac{\delta m}{6\overline{m}} \right)
	     \pm g_T^3 \frac{\delta m}{2\overline{m}} \right\}
	 \, \bar{u}(p_2)i\gamma_5 \frac{\gamma_\mu \gamma\cdot k}{2M}u(p_1)
\eeq
Auch diese Amplituden sind nicht eindeutig bestimmt. Wir haben sie durch
die Forderung bestimmt, da\ss\ die Schwellenamplitude (\ref{delneu})
unver\"andert bleibt und keine Beitr\"age zum longitudinalen 
Multipol $L_{0+}$ auftreten.	 
   

\section{Resonanzbeitr\"age}
Das Niederenergietheorem zur Pionphotoproduktion beruht auf der
Annahme, da\ss\ sich die Zweipunktfunktion $q^\nu\overline{\Pi}_{\mu\nu}$
in der N\"ahe des weichen Punktes $q^2=0$ durch die Nukleonpolterme
approximieren l\"a\ss t. Dabei vernachl\"assigt man  die Beitr\"age
von Schleifendiagrammen sowie den Austausch von Resonanzen im s- oder
t-Kanal. 

In der Photoproduktion von Pionen bei mittleren Energien 
$\omega^{lab}= 0.3-1.5$ GeV ist die Bedeutung von s-Kanal Resonanzen
in den Multipolamplituden deutlich zu erkennen. 
An der Schwelle sind diese Beitr\"age jedoch durch das
Verh\"altnis $m_\pi/\Delta E_R$ der Pionmasse zur Anregungsenergie
der Resonanz unterdr\"uckt. Der niedrigste Anregungszustand des
Nukleons ist die Deltaresonanz bei $\Delta E_R =294$ MeV. Dieser
Zustand koppelt au\ss erordentlich stark an das Pion-Nukleon System
und dominiert aus diesem Grund die resonante $M_{1+}$-Amplitude
bis in die Schwellenregion. Der niedrigste resonante Beitrag zur
$E_{0+}$ Amplitude stammt vom $N(1535)$ bei einer deutlich h\"oheren
Anregungsenergie $\Delta E_R= 597$ MeV. Im Gegensatz zur Deltaresonanz
zerf\"allt dieser Zustand zu etwa 50\% in $\eta N$ und liefert 
insgesamt nur einen geringen Beitrag zur $E_{0+}$ Amplitude
an der Schwelle.

Um diese Aussagen quantitativ zu belegen, wollen wir die 
Resonanzbeitr\"age mit Hilfe effektiver chiraler Lagrangedichten studieren 
\cite{Pec69,OO75,NB80}. Diese Methode ignoriert die intrinsische 
Struktur der Resonanz, hat aber den wesentlichen Vorteil, mit einem
Minimum an freien Parametern auszukommen. Diese Parameter beschreiben
neben der Masse der Resonanz die Kopplungen $\gamma N\to N^{*}$ 
sowie $N^{*}\to N\pi$ und lassen sich aus den experimentell bestimmten
Helizit\"atsamplituden und Zerfallsbreiten extrahieren.

Zu diesem Zweck betrachten wir resonante Photoproduktion
$\gamma N(\Lambda_i=\frac{1}{2},\frac{3}{2}) \to N^{*} \to \pi N$
mit definierter Helizit\"at $\Lambda_i$ im Eingangskanal. Die
zugeh\"origen Helizit\"atsamplituden $A_{1/2}$ und $A_{3/2}$ sind durch
\beq
\label{helamp}
 A_{l\pm} &=& \mp \alpha C_{N\pi} A_{1/2}  \\
 B_{l\pm} &=& \pm \frac{4\alpha}{\sqrt{(2J-1)(2J+3)}} C_{N\pi} A_{3/2}
\eeq
definiert \cite{PDG90}. Die Helizit\"atskomponenten $(A_{l\pm},B_{l\pm})$ 
sind Linearkombinationen der Multipolamplituden $(E_{l\pm},M_{l\pm})$.
Die entsprecheneden Zusammenh\"ange finden sich im Anhang B. Der
Parameter $\alpha$ lautet
\be
 \alpha = \left[ \frac{1}{\pi} \frac{k}{q} \frac{M\Gamma_\pi}{(2J+1)
    M_R \Gamma^2} \right]^{1/2} \; .
\ee
Dabei bezeichnet $M_R$ die Masse der Resonanz, $J$ ihren Spin
und $\Gamma$ sowie $\Gamma_\pi$ die totalen bzw.~partiellen Zerfallsbreiten.
$C_{N\pi}$ ist der Clebsch Gorden Koeffizient f\"ur den Zerfall der 
Resonanz in den relevanten $N\pi$ Ladungszustand.  Die Definition
(\ref{helamp}) hat den Vorzug, da\ss\ alle Gr\"o\ss en, die mit der
Propagation und dem Zerfall der Resonanz zusammenh\"angen, aus
der  eigentlichen Resonanzamplitude eliminiert werden. Die 
Helizit\"atsamplituden $A_{1/2,3/2}$ liefern daher ein zuverl\"assiges
Ma\ss\ f\"ur die St\"arke des \"Ubergangsmatrixelements $\gamma N\to N^{*}$. 
In Tabelle 1 haben wir die entsprechenden Werte f\"ur die
wichtigsten Resonanzen mit Massen unterhalb 1.6 GeV zusammengefa\ss t. 
    
\begin{table}
\caption{Helizit\"atsmaplituden (in $10^{-3}\,{\rm GeV}^{1/2}$) und
totale und partielle Breiten (in MeV) f\"ur die wichtigsten Nukleonresonanzen
mit Massen unterhalb 1.65 GeV. Alle Angaben nach [PDG90].}
\begin{center}
\begin{tabular}{|l||c|r|r|r|r|} \hline
  Resonanz             & Hel.  &  $A_{1/2,3/2}^p$ & $A_{1/2,3/2}^n$ 
		& $\Gamma_{tot}$ & $\Gamma_\pi$ \\ \hline\hline
 $N(1440)\,P_{11}$ & 1/2   &  $-69\pm 7\;\,$  & $37\pm 19$
                &  200         & 120   \\ 
 $N(1520)\,D_{13}$ & 1/2   &  $-22\pm 10$     & $-65\pm 13$
                &  125         &  70    \\
                       & 3/2   &  $167\pm 10$     & $144\pm 14$
		&              &        \\
 $N(1535)\,S_{11}$ & 1/2   &  $73\pm 14$      & $-76\pm 32$
                &  150         &   65    \\
 $N(1650)\,S_{11}$ & 1/2   &  $48\pm 16$      & $-17\pm 37$ 
                & 150	       &   90    \\
 $\Delta (1232)\,\rm P_{33}$ & 1/2 & $-141\pm 5\;\,$&
                &  115         &  115   \\
		        & 3/2  &  $-258\pm 11$    &          
		&              &        \\ \hline
\end{tabular}
\end{center}
\end{table}

Die dominante Resonanz in der $E_{0+}$ Amplitude ist die $N(1535)S_{11}$
Anregung. Dieser Zustand besitzt wie das Nukleon Spin und Isospin 1/2, 
aber negative Parit\"at. Anregung und Zerfall der Resonanz werden durch
die Kopplungen
\beq
\label{s11coup}
 {\cal L}_{\pi NN^{*}} &=& \frac{f_R}{m_\pi} \bar{\psi}_{N^{*}}
   \gamma_\mu \tau^{a}\psi \partial^\mu \phi^{a} + h.c. \\
 {\cal L}_{\gamma NN^{*}} &=& \frac{e}{4M} \bar{\psi}_{N^{*}} 
   \gamma_5 \sigma_{\mu\nu} (\kappa^s_R +\kappa^v_R \tau^3) \psi
    F^{\mu\nu} + h.c.
\eeq
beschrieben. Die beiden Parameter $f_R$ und $\kappa_R$ werden mit Hilfe
der Beziehungen
\beq
\label{rescoup}
       f_R         &=& \frac{2m_\pi}{M_R-M} 
       \sqrt{\frac{\pi M_R \Gamma_\pi}{ p_1(E_1+M)}} 
       \simeq 0.27  \\
 e\kappa^{p}_R   &=& \frac{(2M)^{3/2}}{\sqrt{(M_R+M)(M_R-M)}} A^{p}_{1/2}
       \simeq 0.51 e
\eeq
festgelegt. Dabei bezeichnen $E_1$ und $p_1$ die Energie sowie den Impuls
des Nukleons im Ruhesystems des angeregten Zustands bei der Resonanzenergie
$\sqrt{s}=M_R$. Unter Verwendung der Vertices (\ref{s11coup}) lassen sich
nun die Borndiagramme zur resonanten Photoproduktion bestimmen. Die 
zugeh\"origen invarianten Amplituden finden sich im Anhang B. Der Beitrag
der s-Kanal Anregung der N(1535) Resonanz zur elektrischen Dipolamplitude
an der Schwelle lautet
\beq
  E_{0+}^{N^{*}}(p\pi^0) &=& \frac{e\kappa_R}{16\pi M}\frac{f_R}{m_\pi}
    \frac{2\mu+\mu^2}{(1+\mu)^{3/2}} 
    \frac{(M_R-M)(M_R+M+m_\pi)}{(M+m_\pi)^2-M_R^2} \\[0.2cm]
    &\simeq& 0.28 \su \, .  \nonumber
\eeq    
Dieses Ergebnis ist formal von der Ordnung $\mu$ und widerspricht daher
der in Abschnitt 2.3 vorgenommenen Absch\"atzung der Untergrundamplitude.
Das liegt darin begr\"undet, da\ss\ die N(1535) Resonanz zus\"atzliche
Kontakterme in den inavrianten Amplitude erzeugt, die in den dort
gemachten Voraussetzungen explizit ausgeschlossen worden sind.

Die leichteste Anregung mit denselben Quantenzahlen wie das Nukleon ist
die Roperesonanz $N(1440)$. Dieser Zustand liefert einen resonanten
Beitrag zur $M_{1-}$ Amplitude, ist in der elektrischen Dipolamplitude 
aber nur als Untergrund pr\"asent. Die effektive Lagrangedichte, welche
die Kopplung des $N(1440)$ an das Nukleon beschreibt, lautet
\beq        
\label{nstarcoup}
 {\cal L}_{\pi NN^{*}} &=& \frac{f_R}{m_\pi} \bar{\psi}_{N^{*}}
   \gamma_\mu \gamma_5\tau^{a}\psi \partial^\mu \phi^{a} + h.c. \\
 {\cal L}_{\gamma NN^{*}} &=& \frac{e}{4M} \bar{\psi}_{N^{*}} 
    \sigma_{\mu\nu} (\kappa^s_R +\kappa^v_R \tau^3) \psi
    F^{\mu\nu} + h.c.
\eeq
Bestimmt man die Kopplungskonstanten aus der Zerfallsbreite und der
Helizit\"atsamplitude bei der Resonanzenergie $\sqrt{s}=m_R$, so
ergibt sich $f_R=0.48$ und $\kappa_R^p=0.58$. Mit diesen Werten 
findet man folgenden Beitrag der s-Kanal Anregung
\beq
 E_{0+}^{N^{*}}(p\pi^0) &=& \frac{e\kappa_R}{16\pi M}\frac{f_R}{m_\pi}
    \frac{2\mu+\mu^2}{(1+\mu)^{3/2}} 
     \frac{m_\pi(M_R-M-m_\pi)}{(M+m_\pi)^2-M_R^2}  
    \\[0.2cm]
    &\simeq& -0.025 \su \, .  \nonumber
\eeq 
Wie erwartet ist die entsprechende Amplitude au\ss erordentlich gering.

Eine gewisse Schwierigkeit stellt die Behandlung der Deltaresonanz
$\Delta (1232)$ dar \cite{DMW91,NS89,NB80}. Dieser Zustand ist
eine $P_{33}$ Anregung und sollte daher nicht zur s-Wellen 
Produktion beitragen. In einer relativistischen Beschreibung
der Deltaresonanz als elementares Spin 3/2 Rarita-Schwinger Feld
enth\"alt der Deltapropagator allerdings abseits der Massenschale auch
Spin 1/2 Komponenten. Die Kopplung dieser Beitr\"age an 
die Zerfallskan\"ale $\gamma N$ und $\pi N$ ist im wesentlichen
unbestimmt.  Je nach Wahl der entsprechenden Parameter findet man
\cite{NS89}
\be
\label{delta}
   E_{0+}^\Delta(\pi^0p) = (-0.10 \ldots 0.34) \su  \; .
\ee
Das angegebene Intervall entspricht der Streuung, die sich aus
verschiedenen Fits der Parameter an die nicht resonanten
Amplituden ergibt.  Das Resultat zeigt deutlich die Grenzen der 
Verwendung effektiver chiraler Lagrangedichten bei der Beschreibung
angeregter Zust\"ande auf. Trotzdem sind auch die Korrekturen
auf Grund der Deltaresonanz letztlich relativ gering. 

Wir haben unsere Untersuchung bislang auf die Rolle von Resonanzen
im s-Kanal beschr\"ankt. Aus dem Studium von Dispersionsrelationen 
ist jedoch bekannt, da\ss\ die Einbeziehung von Vektormesonen als
t-Kanal Resonanzen die Beschreibung der differentiellen Wirkungsquerschnitte 
besonders bei kleinen Energien verbessert \cite{BDW67}. Es scheint daher 
angemessen, die Rolle von Vektormesonen auch direkt an der Schwelle zu 
untersuchen. Dabei beschr\"anken wir uns auf die $\rho$- und $\omega$-Mesonen. 
Das $\phi$-Mesonen sowie die schwereren Vektormesonen liefern nur geringe
Beitr\"age. Die effektive Lagrangedichte lautet
\beq
\label{lvm}
 {\cal L}_{\rho NN} &=& f_{\rho NN} \bar{\psi}
         \left( \gamma_\mu +\frac{\kappa_\rho}{2M}\sigma_{\mu\nu}
	 \partial^\nu \right) \vec{\tau}\cdot\vec{\rho}^{\,\mu} \psi  \\ 
 {\cal L}_{\omega NN} &=& f_{\omega NN} \bar{\psi}
         \left( \gamma_\mu +\frac{\kappa_\omega}{2M}\sigma_{\mu\nu}
	 \partial^\nu \right) \omega^\mu \psi  \\
 {\cal L}_{\rho\pi\gamma} &=& \frac{eg_{\rho\pi\gamma}}{2m_\pi}
         \epsilon_{\alpha\beta\gamma\delta} F^{\alpha\beta}
	 \vec{\phi}\cdot\partial^\gamma\vec{\rho}^{\,\delta} \\
 {\cal L}_{\omega\pi\gamma} &=& \frac{eg_{\omega\pi\gamma}}{2m_\pi}
         \epsilon_{\alpha\beta\gamma\delta} F^{\alpha\beta}
	 \phi_3\cdot\partial^\gamma\omega^\delta \; .
\eeq
Die Kopplung des Photons l\"a\ss t sich aus der gemessen Zerfallsbreite
$\Gamma(\rho,\omega\to\pi\gamma)$ bestimmen. Die Vektormeson-Nukleon
Kopplungskonstante mu\ss\ dagegen indirekt, aus detaillierten Analysen
des Nukleon-Nukleon Potentials gewonnen werden \cite{Dum82}. Die
resultierenden Werte finden sich in Tabelle 2.  
 
\begin{table}
\caption{Parameter f\"ur die wichtigsten t-Kanal Beitr\"age 
zur Pionphotoproduktion.}
\begin{center}
\begin{tabular}{|rcl|rcl|rcl|}\hline
   & $\pi$ &             &  & $\rho$ &             &    &  $\omega$ &  \\ 
                                                                \hline\hline
$m_{\pi^\pm}$&=&139 MeV  & $m_\rho$&=&$770$ MeV    &  $m_\omega$&=&$783$ MeV\\
$f_{\pi NN}$&=&$1.00$    &  $f_{\rho NN}$&=&$2.66$ & $f_{\omega NN}$&=&$7.98$\\
$g_{\pi\pi\gamma}$&=&$1$ &  $g_{\rho\pi\gamma}$&=&$0.125$ 
                                        &   $g_{\omega\pi\gamma}$&=&$0.374$  \\
    & &       &  $\kappa_\rho$&=&$6.6$  &   $\kappa_\omega$&=&$0.$   \\ \hline
\end{tabular}
\end{center}
\end{table}                    

Die invarianten Amplituden, die sich aus der Wechselwirkung (\ref{lvm})
ergeben, haben wir in Anhang B gesammelt. Der Beitrag zur
elektrischen Dipolamplitude an der Schwelle ist
\be
  E_{0+}^V(\pi^0p) = \frac{e}{16\pi} \sum_V\frac{g_V}{m_\pi} 
    f_V (1+\kappa_V)\mu^3
     \frac{2+\mu}{(1+\mu)^{3/2}}\frac{M^2}{m_\pi^2+m_V^2 (1+\mu)}
\ee     	 
wobei $V=\rho,\omega$ zu setzen ist. Dieses Resultat ist explizit
von der Ordnung $\mu^3$ und entspricht daher der Absch\"atzung
aus dem Abschnitt 2.5. Mit den Werten aus Tabelle 2 findet man 
$g_\rho f_\rho (1+\kappa_\rho)=1.50$ und  $g_\omega f_\omega 
(1+\kappa_\omega)=2.87$, so da\ss\ wir schlie\ss lich eine 
Korrektur $E_{0+}=0.024\su$ erhalten.   


 
\section{Abschlie\ss ende Bemerkungen}
blub blub
