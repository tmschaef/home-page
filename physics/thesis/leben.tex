\documentstyle[12pt]{article}
\pagestyle{empty}
\hoffset-1.5cm
\textwidth15cm
\textheight21.5cm
\begin{document}
\baselineskip18pt
\parskip0pt
\vspace{1cm}
\centerline{\huge \rm Lebenslauf}
\vspace{0.8cm}
\centerline{\Large \rm Thomas Sch\"afer}
\vspace{0.8cm}
\begin{center}
\begin{minipage}{8cm}
geboren am 18. Mai 1965 in Hanau am Main \\
Eltern : Dr. Volker Sch\"afer, Physiker \\
Margit Sch\"afer, geb. Kleinschmidt, Lehrerin
\end{minipage}
\end{center}

\vspace{2cm}
\begin{tabular}{lp{11cm}}
Sept. 71 - Mai 84     &  Schulausbildung an der Grundschule Bovenden, dem 
                         Max Planck Gymnasium in G\"ottingen und der 
			 Gesamtschule Somborn. \\
Mai 84                &  Allgemeine Hochschulreife. Gesamtnote
                         "sehr gut (1.0)". \\			  
Okt. 84 - Juni 86     &  Studium der Physik an der Justus Liebig Universit\"at
                         Giessen\\
Dez. 84               &  Aufnahme in der Studienstiftung des deutschen Volkes.\\		 
Juni 86               &  Vordiplomspr\"ufung im Fach Physik. Gesamtnote :
                         "sehr gut". \\ 			 
Sept. 86 - Juli 87    &  Studium als DAAD Stipendiat an der University of
                         Washington in Seattle, USA. \\
Okt. 87 - April 89    &  Physikstudium an der Justus Liebig Universit\"at
                         Giessen. Unter der Anleitung von Prof. U. Mosel am
		         Institut f\"ur theoretische Physik Anfertigung
			 einer Diplomarbeit mit dem Titel "Dirac See Effekte
		         in einem nichttopologischen chiralen  Soliton Modell".\\
April 89              &  Diplompr\"ufung im Fach Physik. Gesamtnote: 
                         "Mit Auszeichnung". \\			 
seit Juli 89          &  Promotionsstudium am Institut f\"ur theoretische 
                         Physik der Universit\"at Regensburg unter der
			 Anleitung von Prof. W. Weise. \\
Juli - Aug. 90        &  Sommeraufenthalt am Institute for Nuclear
                         Theory (INT) in Seattle, USA.			 
\end{tabular}

\newpage
\underline{Ver\"offentlichungen:}
\begin{enumerate}
\item{U. Kalmbach, T. Sch\"afer, T. S. Biro, U. Mosel,  Dirac sea effects in
the chiral quark soliton modell, Nucl. Phys. A513 (1990) 621 }
\item{A. Hosaka, T. Sch\"afer, Spin structure of the nucleon in a
 nontopological chiral soliton model, Z. Phys. A337 (1990) 447}
\item{T. Sch\"afer, W. Weise, Neutral pion photoproduction at threshold and
explicit chiral symmetry breaking, Phys. Lett. B250 (1990) 6 }
\item{T. Sch\"afer, W. Weise, Threshold Pion Photoproduction and chiral
models of the nucleon, Nucl. Phys. A531 (1991) 520}
\item{T. Sch\"afer, Neutral pion photoproduction at threshold, Proceedings
of the workshop on pions and nuclei, Peniscola (Spain) 1991, World
Scientific, in Druck} 
\end{enumerate}
\\

\underline{Vortr\"age:}
\begin{enumerate}
\item{See Effekte im chiralen
Soliton Bag Modell, DPG Fr\"uhjahrstagung Bonn, 1989 }
\item{Neutral pion photoproduction near threshold,
DPG Fr\"uhjahrstagung Strassbourg, 1990 }
\item{Pion photoproduction and chiral models of the 
nucleon, Institute for Nuclear Theory (INT), Seattle, USA 1990} 
\item{Neutral pion photoproduction and explicit chiral
symmetry breaking, Workshop on electromagnetic Interactions, Bosen (Saar) 1990}
\item{Pion photoproduction and chiral models
of the nucleon, DPG Fr\"uhjahrstagung Darmstadt, 1991}
\item{Neutral pion photoproduction at threshold,
International workshop on pions in nuclei, Peniscola (Spanien) 1991}
\item{Current algebra and photoproduction of 
pesudoscalar mesons, IV. International symposium on pion-nucleon Physics
and the structure of the nucleon, Bad Honnef 1991 }
\end{enumerate}     
\end{document}     