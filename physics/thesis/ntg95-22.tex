%%%%%%%%%%%%%%%%%%%%%%%%%%%%%%%%%%%%%%%%%%%%%%%%%%%%%%%%%%%%%%%%%%%%%%%%
%                                                                      %
%   first try     :  03-07-95                                          %
%   revised       :  07-26-95                                          %
%                                                                      %
%%%%%%%%%%%%%%%%%%%%%%%%%%%%%%%%%%%%%%%%%%%%%%%%%%%%%%%%%%%%%%%%%%%%%%%%

\documentstyle[12pt,epsf]{article}
%\documentstyle[12pt]{article}
\def\baselinestrech{1.5}
\setlength{\textwidth}{16cm}
\setlength{\textheight}{23cm}
\large
\hoffset -1.0cm
\voffset -2.2cm
\setlength{\baselineskip}{23pt}
\flushbottom

% GENERAL DEFINITIONS

\newcommand{\be}{\begin{eqnarray}}
\newcommand{\ee}{\end{eqnarray}}
\newcommand\ident{{\bf 1}}
\newcommand\unity{{1\!\! 1}}
\newcommand{\nsz}{\textstyle}

\begin{document}
\setlength{\baselineskip}{23pt}
\setlength{\baselineskip}{27pt}
\pagestyle{empty}
\renewcommand{\thefootnote}{\fnsymbol{footnote}}
\centerline{\bf\LARGE The Interacting Instanton Liquid }
\centerline{\bf\LARGE in QCD at zero and finite Temperature}
\vskip 1cm
\centerline{\bf T.~Sch\"afer, E.V.~Shuryak and J.J.M.~Verbaarschot}
\vskip 1cm
\centerline{\it Department of Physics}
\centerline{\it State University of New York at Stony Brook}
\centerline{\it Stony Brook, New York 11794, USA}
\vskip 1cm

\setlength{\baselineskip}{16pt}
\centerline{\bf Abstract}
   In this paper we study the statitical mechanics of the instanton
liquid in QCD. After introducing the partition function as well as
the gauge field and quark induced interactions between instantons
we describe a numerical method to calculate the free energy of the
instanton system. We use this method to determine the equilibrium
density and the equation of state for the instanton ensemble in
QCD. We find that the system undergoes a chiral phase transition
at $T\simeq 150$ MeV and show that the mechanism for this transition
is a rearrangement of the instanton liquid, going from a disordered,
random, phase at low temperatures to a strongly correlated, molecular,
phase at high temperature.
\vfill
\begin{flushleft}
SUNY-NTG-95-22\\
July 1995
\end{flushleft}
\eject
\newpage
\setlength{\baselineskip}{23pt}
\pagestyle{plain}
\renewcommand{\thefootnote}{\arabic{footnote}}
\setcounter{footnote}{0}
\setcounter{page}{1}

\section{Introduction}

   Understanding the vacuum structure of gauge theories like QCD
is one of the main problems in quantum field theory today. It also
provides the theoretical foundation for hadronic models and hadronic
phenomenology from the underlying field theory of the strong interaction,
Quantum Chromodynamics. There are a number of indications that instantons,
classical tunneling trajectories in imaginary (euclidean) time are an
important ingredient of the QCD vacuum.

   Soon after the discovery of instantons 20 years ago \cite{BPST_75},
it became clear that instantons provide at least a qualitative
understanding of many features of the QCD vacuum. Instantons solve
the $U(1)_A$ problem \cite{tHo_76}, provide a mechanism for chiral
symmetry breaking \cite{CDG_78}, contribute to the gluon condensates
\cite{SVZ_79} and lead to a non-pertubative vacuum energy \cite{Shu_78}.

   The development of a quantitative theory based on these ideas took much
longer. The instanton liquid model was originally suggested by Shuryak
in 1982 \cite{Shu_82}, based mainly on phenomenological considerations.
Later Diakonov and Petrov developed an analytic approach based on the
variational method \cite{DP_84} and numerical simulations of the instanton
liquid were started in \cite{Shu_88}. During the past two years we have
shown \cite{SV_93,SS_94,SS_95} that the ``random instanton liquid model"
(RILM) provides a successful description of a large number of hadronic
correlation functions, including mesons and baryons made of light quarks,
heavy-light systems and glueballs. These correlators give accurate
predictions of the corresponding resonance masses and coupling constants
and compare well with direct calculations of point-to-point correlators
on the lattice \cite{CGHN_93}.

    Following these development, several recent lattice investigations
have focused on the role of instantons in the QCD vacuum, both
at zero and finite temperature. Using a method called 'cooling' one
can relax any given gauge field configuration to the closest classical
component of the QCD vacuum. The resulting configurations were known to be of
multi-instanton type \cite{Ber_81}, but the more recent work by Chu et
al.~\cite{CGHN_94} has provided quantitative measurements of the parameters
of the instanton liquid, as well as detailed studies of the dynamical
effects of instantons. These authors conclude that the instanton density in
the quenched theory (without dynamical fermions) at zero temperature is
$n\simeq (1.3$-$1.6)\,{\rm fm}^{-4}$ while the average size is about
$\rho\simeq 0.35$ fm. These numbers confirm the key parameters $n=1\,
{\rm fm}^{-4}$ and $\rho=0.3$ fm of the instanton liquid model mentioned
above. In addition to that, Chu et al.~studied correlation functions
in the cooled configurations, finding that they hardly change from the
original, fully quantum configurations. This implies that instanton
effects seem to dominate over perturbative and confinement forces
in hadronic structure.

   These finding can and should be made much more accurate, of course.
A first attempt to extend this work to finite temperature was made by
by Chu and Schramm \cite{CS_95}, who find that the instanton density
(more accurately, the topological susceptibility in the quenched theory)
is essentally independent of temperature below the phase transition,
while it is exponentially suppressed above the transition temperature.
This result was also confirmed in \cite{IMM_95}. A lattice measurement
of the instanton size distribution was performed in \cite{MS_95}.
These authors also made an attempt to study correlations between
instantons.

   In this paper we want to report a detailed study of the statistical
mechanics of the interacting instanton liquid (IILM), both at zero and
finite temperature. The purpose of this study is twofold. First, we want
to determine the role of the instanton interactions in the ensemble at
zero temperature. The random model (RILM) mostly used before assumes that
the instanton liquid is completely uncorrelated and characterized
by the density and average size given above. It is very successful
in the description of a large number of hadronic correlation
functions \cite{SV_93} but has a few shortcomings, most notably in
channels that are strongly repulsive (like the $\eta'$ and $\delta$
meson channels). As shown in \cite{SV_95,SS_95}, one has to include
the correlations among instantons caused by their classical and
fermion induced interactions in order to achieve a correct description
of topological charge screening and the $\eta'$ channel.

   The second purpose of this work is to study the nature of the
chiral phase transition in the instanton model. In this area also,
some significant progress was made during the past couple of years.
Originally, it was believed that there is no real phase transition
in the instanton model, but that the instanton density is exponentially
suppressed at finite temperature \cite{PY_80} and that chiral symmetry
is restored at high tempearture due to the disappearance of instantons
\cite{Shu_82,DM_88,NVZ_89}. It was realized by Ilgenfritz and Shuryak
\cite{IS_89} that instantons could be present even above the chiral
phase transition, as strongly correlated instanton-antiinstanton
molecules. Shuryak and Velkovsky later argued that instanton suppression
is essentially a plasma effect and should not be present below the
phase transition \cite{SV_94}. If so, the phase transition is really
driven by the formation of molecules \cite{IS_94,SSV_95} rather than
by the suppression of individual instantons. As mentioned above, there is
support for this scenario from lattice simulations that do indeed find that
the instanton density is roughly constant below $T_c$ \cite{CS_95}.
First evidence for the presence of molecules near $T_c$ was reported
in \cite{IMM_95}. This implies that there can be a substantial number of
instantons present around $T_c$, causing non perturbative effects even
above the phase transition. One such effect, the survival of certain
hadronic modes above the phase transition was recently studied in
\cite{SS_95b}.

   So far, the transition has only been studied using the schematic
'cocktail' model introduced by Ilgenfritz and Shuryak \cite{IS_94}.
In this model the instanton liquid consists of two components, a random
and a molecular one. The free energy is determined separately for the
two components and their concentrations is then determined by minimizing
the total free energy. The chiral phase transition occurs when the
concentration of random instantons is zero. Here we want to improve
on this schematic model, by doing a complete calculation in the
interacting ensemble. In this way, many approximations are relaxed, also
all kind of correlations (and not only the polarized instanton-antiinstanton
pairs) are included.


   The paper is organized as follows. In section 2 we introduce the
partition function of the instanton liquid and specify the gauge
field and fermion induced interactions between instantons. Detailed
parametrizations of these interactions can be found in the appendices.
In section 3 we decsribe the numerical method which is used in order
to calculate the partition function. In section 4 we use this method
to study the instanton ensemble at zero temperature. In the next
two sections we generalize the method to finite temperature and
study the nature of the chiral phase transition in the instanton
liquid. Our first digression into the hadronic spectroscopy is done
in section 7, where we connect spectrum of the Dirac operator with
the integrated correlation functions, or susceptibilities.

\section{The partition function of the instanton liquid}

    The euclidean partition function of QCD is given by
\be
\label{Z_QCD}
  Z = \int DA_\mu\, \exp(-S[A_\mu])
          \prod_f^{N_f}\det(\hat D+m_f)
\ee
where the gauge field action is given by $S[A_\mu]=\frac{1}{4}
\int d^4x\, {\rm Tr}(F_{\mu\nu}F_{\mu\nu})$ and the Dirac operator
is defined by $\hat D\psi=\gamma_\mu(\partial_\mu -iA_\mu)\psi$.


    The main assumption underlying the instanton model is that the
full partition function can be approximated by a partition sum in
which the relevant gauge configurations are superpositions of instantons
and anti-instantons. In this partition function the integration
extends over the collective coordinates associated with $N_+$
instantons and $N_-$ instantons
\be
\label{Z}
Z =  {1 \over N_+ ! N_- !}\int
    \prod_i^{N_+ + N_-} [d\Omega_i\; d(\rho_i) ]
    \exp(-S_{int})\prod_f^{N_f} \det(\hat D+m_f) \, .
\ee
Here $d\Omega_i=dU_i\, d^4z_i\,d\rho_i$ is the measure in the space of
collective coordinates, color orientation, position and size, associated
with a single instantons. For the gauge group $SU(3)$ there is a total
of 12 collective coordinates per instanton. Fluctuations around the
multi-instanton configuration are included in gaussian approximation
for the individual instantons. This gives the semi-classical instanton
amplitude, calculated originally by 't Hooft \cite{tHo_76}. To two
loop accuracy it reads
\be
\label{idens}
d(\rho) &=& C_{N_c} \rho^{-5} \beta_1 (\rho)^{2N_c}
            \exp\left(-\beta_2 (\rho)+(2N_c-\frac{b'}{2b})
            \frac{b}{2b'}\frac{1}{\beta_1(\rho)}
            \log (\beta_1(\rho))
            \right) \, ,\\
  & & C_{N_c} = \frac{ 4.6\exp(-1.86 N_c) }
                 {\pi^2 (N_c-1)! (N_c-2)! }\,
\ee
where $\beta_1(\rho)$ and $\beta_2(\rho)$ are the one and two loop
beta functions
\be
\beta_1(\rho) = -b\log(\rho\Lambda), \hspace{1cm}
\beta_2(\rho) = \beta_1(\rho) + \frac{b'}{2b}
   \log (\frac{2}{b}\beta_1(\rho)) ,
\ee
with the one and two loop coefficients
\be
   b = \frac{11}{3}N_c -\frac{2}{3}N_f \hspace{1cm}
   b'= \frac{34}{3}N_c^2-\frac{13}{3}N_c N_f +\frac{N_f}{N_c}.
\ee
The classical action $S_0 = \frac{8\pi^2}{g^2}$ is included in the
semi-classical amplitude (\ref{idens}). The classical interaction
between instantons is denoted by $S_{int}$. We approximate this
interaction by a pure two-body interaction $S_{int}=\sum_{I\neq J}
S_{int}(\Omega_{IJ})$ which only depends on the relative coordinates
of the two instantons. The importance of genuine three body effects
in the classical interaction between instantons was studied in
\cite{Shu_88}, with the conclusion that this contribution is
negligible as long as the density is not extremely large.

    The two-body interaction $S_{int}(\Omega_{IJ}) = S[A_\mu(\Omega_{IJ})]
-2S_0$ is calculated classically, by inserting an Ansatz $A_\mu(\Omega_{IJ})$
for the two-instanton gauge potential into the action. The resulting
interaction will then depend on the details of the ansatz that was used.
In practice, various ansaetze have been used in the literature: (i) the
sum ansatz \cite{DP_84}, (ii) the ratio ansatz \cite{Shu_88}, (iii) the
Yung ansatz \cite{Yun_88} and (iv) the exact streamline solution
\cite{Ver_91}. The latter is characterized by the fact that the action
is minimized in all directions except along the collective coordinate
describing the separation between the two instantons. In this sense,
the streamline solution is the optimal classical instanton-antiinstanton
configuration.

   In order to discuss the properties of the classical interaction
between instantons, let us introduce the four vector $u_\mu =
\frac{1}{2i}{\rm tr}(U_I U_A^+ \tau_\mu^{-})$ where $U_{I,A}$ are
the orientation matrices of the instanton and anti-instanton and
$\tau_\mu^-$ is the $2\times 2$ matrix $(i,\vec\tau)$. For the gauge
group $SU(2)$, $u_\mu$ is a real unit vector whereas for $SU(3)$
it is a complex vector whith $|u|^2\leq 1$. In any case, we can define
an angle $\theta$ by
\be
   \cos\theta = \frac{|u\cdot\hat R|}{|u|},
\ee
where $R=z_I-z_A$ is the vetor connecting the centers of the two
instantons.  For all the ansaetze mentioned above, the large distance
part of the instanton-anti-instanton interaction is given by
\be
\label{dipole}
  S_{int} = \beta_1(\bar\rho)\frac{4\rho_1^2\rho_2^2}{R^4}|u|^2
                  (1-4\cos^2\theta),
\ee
which is the dipole-dipole interaction originally discussed by Callen,
Dashen and Gross \cite{CDG_78}. The interaction is given in units of
the single instanton interaction $\beta_1(\rho)$. The argument of the
beta function is not exactly determined without preforming a higher
order calculation of fluctuations around the two-instanton configuration.
In practice, we take the argument to be the geometric mean $\bar\rho=
\sqrt{\rho_I\rho_A}$ of the two instanton radii. The interaction is
attractive for the relative orientation $\cos\theta=1$, but vanishes after
averaging over all angles $\theta$. The short distance behavior depends
on the ansatz chosen. In the sum ansatz, there is a substantial
repulsive core at distances $R<\sqrt{6}\rho$ \cite{DP_84}, but the
amount of repulsion at short distances becomes siginficantly weaker
using the more refined ansaetze. We will discuss this question in more
detail in the next section. A parametrization of the interaction in
the ratio and streamline (Yung) ansaetze is given in appendix A. In
fig.1a, we show the ratio and streamline interaction for the most
attractive and repulsive orientations. One clearly observes that the
interactions are similar at large distance, but differ significantly at
short distance. In particular, the streanline interaction has no
repulsion at all for the most attractive orientation. The interaction
smoothly approaches $S_{int}=-2S_0$ at short distance, corresponding
to the annihilation of the instanton-antiinstanton pair. Fig.1a also
shows the effect of a phenomenological core in the streamline interaction.
The reasoning behind this ansatz will be discussed in more detail in
the next section.

   In all the ansaetze except for the simple sum ansatz, the
instanton-instanton interaction is much weaker than the
instanton-anti-instanton one. In fact, in the streamline ansatz, the
instanton-instanton interaction vanishes. This is a reflection
of the fact that there is an exact two-instanton solution (with $S=2S_0$)
for arbitrary values of the relative coordinates.


   The fermionic determinant induces a very nonlocal interaction
among the instantons. Evaluating this determinant exactly in the
instanton ensemble still constitutes a formidable problem. In practice
we factorize the determinant into a low and a high momentum part
\be
\label{det}
  \det(\hat D+m_f) = \left( \prod_{i}^{N_++N_-}\hspace{-0.3cm}
  1.34 \rho_i\right) \;\det(T+m_f),
\ee
where the first factor, the high momentum part, is factorized into the
contributions from individual instantons, whereas the low momentum part
associated with the fermionic zero modes of the individual instantons is
calculated exactly. This is in accordance with the general strategy of
treating the zero modes of the system exactly. It is also of great physical
importance, since the low momentum part of the spectrum of the Dirac
operator determines the structure of chiral symmetry breaking. $T_{IA}$
is the $N_+\times N_-$ matrix of overlap ('hopping') matrix elements
\be
\label{overl}
 T_{IA} &=& \int d^4x\;  \phi_{A\, 0}^\dagger (x-z_A) i\hat D_x
       \phi_{I\, 0}(x-z_I),
\ee
where $\phi_{I,A\, 0}$ are the fermionic zero mode wave functions
of the instanton and anti-instanton. Due to the chirality of the
zero modes, the fermionic overlap matrix elements between instantons
with the same topological charge vanishes. In the following, we will only
consider quadratic matrices with $N_+=N_-$. Ensembles with $N_+\neq
N_-$ correspond to system with finite winding number in the gauge
sector. In the thermodynamic limit, the distribution of winding numbers
is sharply peaked around 0, and imposing an additional condition
corresponding to a finite winding number is not expected to affect
our results.

    The general structure of the overlap matrix elements is given
by $T_{IA}= (u\cdot r) f(r)$.  This means that, like the gauge
field induced interaction, the fermionic overlap is maximal when
the relative instanton-anti-instanton orientation is given by
$\cos\theta=1$. As the gauge field induced interaction
between instantons, the fermionic overlap matrix elements depend on
the ansatz for the two-instanton gauge potential, which enters through
the covariant derivative in (\ref{overl}). In this case, however,
the dependence on the ansatz is much less pronounced. For the sum
ansatz, one can use the eqations of motion and replace the covariant
derivative in (\ref{overl}) by an ordinary one. The result can be
parametrized by
\be
\label{sum_overl}
 T_{IA} = i(u\cdot R)  \frac{1}{\sqrt{\rho_I\rho_A}}
    \frac{4.0}{(2.0+R^2/\rho_I\rho_A)^2},
\ee
which is exact at large distances. The streamline ansatz gives the
same large distance behavior, but somewhat different results at
small and intermediate separations. We give a parametrization of
the streamline matrix elements in appendix B.


\section{The free energy of the instanton ensemble}

   In this section, we describe a method to evaluate the partition
function of the instanton liquid. Using this method, we can calculate
the free energy numerically as a function of the density of instantons,
and determine the equilibrium density from the condition that the free
energy is minimal.

   The problem in determining the free energy is connected with the fact
that the complicated statistical mechanics associated with the partition
function (\ref{Z}) can in general only be dealt with by performing
Monte Carlo simulations \cite{Shu_88,SV_90}. These simulations
are ideally suited for the calculation of various expectation values,
but do not give a direct determination of the partition function,
which provides the overall normalization. Previous Monte Carlo calculations
have therefore been restricted to simulations of the enesemble at a fixed
density of instantons, which was determined from phenomenological
considerations (typically $1\,{\rm fm}^{-4}$). Here, we want to go
beyond this approximation and minimize the free energy. A method
to calculate the partition sum, which is well known in statistical
mechanics (and lattice gauge theory), is 'integration in the coupling
constant'. In this method one splits the effective action
\be
\label{S_eff}
   S_{eff} = -\sum_{i=1}^{N_++N_-}\log (d(\rho_i)) + S_{int}
               +{\rm tr}\log (\hat D+m_f)
\ee
in a free and an interacting piece, $S_{eff}=S_{0}+\alpha S_{1}$.
If the partition function for the free system governed by the action
$S_0$ is known, the full partition function can be determined from
\be
\label{int_coup}
  \log Z(\alpha\! =\! 1) &=& \log Z(\alpha\! =\! 0)
  + \int_0^1 d\alpha'\, \langle 0| -S_1 |0\rangle_{\alpha'},
\ee
where the expectation value is determined from a simulation at the
coupling constant $\alpha'$. The obvious choice for decomposing
the effective action of the instanton liquid would be to identify
the logarithm of the single instanton distribution with the free
action, $S_0=\sum_i\log(d(\rho_i))$. This procedure, however, does
not work since the instanton distribution behaves like $d(\rho)
\sim \rho^{(b-5)}$, so that the $\rho$ integration in the free partition
function would not be convergent. This is the famous infrared problem
which plagues the dilute instanton gas approximation \cite{CDG_78}. As
explained in more detail in the next section, the instanton liquid
is stabilized by the the repulsive core in the gauge field interaction
once the full interaction is taken into account. We therefore consider
the following effetive action
\be
\label{S_var}
   S_{eff} = \sum_{i=1}^{N_++N_-}\left(- \log (d(\rho_i)) +
           (1-\alpha)\nu\frac{\rho_i^2}{\,\overline{\rho^2}\,}\right)
           + \alpha \left( S_{int}
               +{\rm tr}\log (\hat D+m_f) \right),
\ee
where $\nu=(b-4)/2$ and $\overline{\rho^2}$ is the average size squared
of the instantons with the full interaction included. The term
proportional to $(1-\alpha)$ serves to regularize the $\rho$
integration for $\alpha=0$. It disappears for $\alpha=1$, where
the original action is recovered. The specific form of this
term is irrelevant, our choice here is motivated by the fact
that (at least for the one-loop measure $d(\rho)$) $S_{eff}
(\alpha$=$0)$ yields a one-body distribution with the correct
average size $\overline{\rho^2}$. This means that the one-body
distribution generated by $S_{eff}(\alpha$=$0)$ is a variational
ansatz for the full one-body distribution. One may therefore hope
that the integration over the coupling constant $\alpha$ only gives
small corrections to the partition function for $\alpha=0$.

   The partition function corresponding to the variational single
instanton distrbution is given by
\be
\label{Z_free}
   Z_0 = \frac{1}{N_+!\, N_-!} (V\mu_0)^{N_++N_-}, \hspace{1cm}
         \mu_0 = \int_0^\infty  d\rho \, d(\rho)
           \exp(-\nu\frac{\rho^2}{\,\overline{\rho^2}\,} ) ,
\ee
where $\mu_0$ is the normalisation of the one-body distribution.
The $\rho$ integration in $\mu_0$ is regularized by the second term
in (\ref{S_var}). The full partition function obtained from integrating
over the coupling  $\alpha$ is
\be
\label{int_coup2}
  \log Z &=& \log (Z_0)
  + N \int_0^1 d\alpha'\,  \langle 0|
      \nu\frac{\rho^2}{\,\overline{\rho^2}\,} - \frac{S_1}{N}
      |0\rangle_{\alpha'},
\ee
where $N=N_++N_-$. The free energy density is finally given by $F=-1/V\cdot
\log Z$ where $V$ is the four-volume of the system. The pressure and the
energy density are related to $F$ by
\be
 p=-F, \hspace{1cm}
 \epsilon = p \frac{dp}{dT}-p .
\ee
At zero temperature we have $\epsilon=-p=F$ and the free energy determines
the shift of the QCD ground state relative to the perturbative vacuum.
Such a shift is certainly present in our case, since tunneling lowers
the ground state energy.

    The full partition function can be compared to the variational
ansatz introduced in \cite{DP_84} and employed in many works on
the subject \cite{Shu_87,DM_88,IS_89,NVZ_89,Ver_91,IS_94}. For
simplicity we restrict the discussion to pure gauge theory, i.e.
neglect the fermionic determinant. Since the variational ansatz
ignores any correlations between instantons, only the color and
spatial average of the interaction enters
\be
 \int d^4R dU S_{int}(R,U,\rho_1,\rho_2)
    = \kappa^2 \frac{N_c}{N_c^2-1}\rho_1^2 \rho_2^2 ,
\ee
where $S_{int}(R,U,\rho_1,\rho_2)$ is the interaction of two instantons
with radii $\rho_{1,2}$, separation $R$ and relative orientation $U$.
In the sum ansatz, both the $II$ and $IA$ interaction give the same
average repulsion $\kappa^2=\frac{27}{4}\pi^2$ \cite{DP_84}. In the
ratio and streamline ansatz, this repulsion is considerably weaker.
In the streamline ansatz, only the $IA$ interaction is repulsive with
$\kappa^2=4.772$ \cite{Ver_91}.

    If the variational single instanton distribution $d(\rho)\exp
(-\nu\rho^2/\overline{\rho^2})$ is close to the true distribution
for $\alpha=1$ we can calculate the expectation value in
(\ref{int_coup2}) using the variational one. One finds $\langle
S_1\rangle =N\nu/2$ and the resulting estimate for the partition
function is
\be
\label{Z_var}
    Z = \frac{1}{((N/2)!)^2} (V\mu_0)^N \exp(-\frac{N\nu}{2}),
\ee
which agrees with the result derived in \cite{DP_84}. Varying
$F=-1/V\cdot\log Z$ with respect to the density one finds the
expected result $N/V=2\mu_0$. Numerical results for different
interactions were compared in \cite{Shu_85,Ver_91}. For the
sum ansatz, one finds $N/V=0.18 \Lambda^4$ with $\bar\rho=0.45
\Lambda^{-1}$, whereas the streamline ansatz gives $N/V=0.54
\Lambda^4$ and $\bar\rho=0.69\Lambda^{-1}$. The variational
method was extended to light quarks in \cite{DP_85,NVZ_89}.
We will study this problem in detail in the next section.

    If correlations among instantons are important, the variational
method is not expected to provide a useful estimate for the
partition function and other observables. Since the main source
of correlations in the instanton liquid are dynamical quarks, this
issue is particularly important for real QCD with two light
and one intermediate mass flavor. Also, as argued in the introduction,
we expect chiral symmetry to be restored due to the formation of
instanton-antiinstanton molecules. This feature is certainly not
captured by the variational model (at least not in its simplest
form), and we will therefore study the full partition function
numerically using the method introduced in this section.

\section{The instanton ensemble at zero temperature}

    In this section we want to present numerical results obtained
from simulations of the instanton liquid at zero temperature. These
results will serve to fix the parameters of the model and determine
the configurations to be used in order to calculate correlation
functions in the interacting instanton liquid \cite{SSV_95b}. Our main
interest will be to analyzis of the correlations among instantons, in
order to determine how close the configurations are to the random
model used in many previous calculations. The precise amount of
correlations in the system also determines the relative strength
of the molecule induced interaction between quarks studied
in \cite{SSV_95} relative to the more familiar 't Hooft interaction
\cite{tHo_76,SVZ_80}. This is important, since, among other things,
it determines the strength of the vector interaction (not present
in a very dilute random system) relative to the scalar one.

    In the following we will study the system in the streamline (Yung
ansatz). While the streamline ansatz provides the 'best' instanton
interaction, the calculation relies heavily on conformal symmetry and
we do not know how to extend it to finite temperature. In section 6,
we will therefore also present the zero temperature ensemble in the
ratio ansatz.

    A general problem with calculations in the interacting instanton
model is the treatment of very close instanton-antiinstanton pairs.
While long range correlations among instantons are physically
important in order to understand the behavior of the topological
susceptibilty \cite{SV_95,SS_95}, very close instanton-anti-instanton
configurations correspond to perturbative fluctuations
and should not be taken into account in the partition function of
the instanton liquid. Similarly, configurations with two instantons
on top of each other should not be counted as two independent instantons,
but as one object with winding number two.  However, the classical
streamline ansatz provides very little short range repulsion
and leads to a large number of close instanton pairs. In order to
treat these configurations correctly one would have to calculate
the quantum fluctuations around these very close pairs and determine
the space of collective coordinates. In practice we have used a
much simpler solution and introduced a purely phenomenological
short range repulsive core
\be
   S_{\rm core} = \beta_1(\bar\rho) \frac{A}{\lambda^4}
   |u|^2, \hspace{1cm}
 \lambda = \frac{R^2+\rho_I^2+\rho_A^2}{2\rho_I\rho_A}
  + \left( \frac{(R^2+\rho_I^2+\rho_A^2)^2}{4\rho_I^2\rho_A^2}
   - 1\right)^{1/2}
\ee
in the II and IA interactions. Here $\lambda$ is the conformal
parameter that determines the functional form of the streamline
interaction \cite{Ver_91} and $A$ controls the strength of the
core. In addition to suppressing very close instanton pairs, the
strength of the core will also determine the optimum density and
average size of the instantons.

   In practice we will use these two quantities to determine the
parameters of the instanton model. Their values inferred from
phenomenological considerations are $n=1\,{\rm fm}^{-4}$ and
$\rho=0.33$ fm \cite{Shu_82}, close to the lattice result $n=(1.4$
-$1.6)\,{\rm fm}^{-4}$ and $\rho=0.4$ fm \cite{CGHN_94}. These
numbers are used to fix the scale parameter $\Lambda_{QCD}$
and the strength of the core $A$. The scale parameter is known
in principle from applications of perturbative QCD, but the
accuracy of these determinations is not very high. The instanton
density, on the other hand, is proportional to $\Lambda_{QCD}$ to the
fourth power and depends very sensitively on the value of the
scale parameter.

   The results of our simulations are shown in fig.2-5. The partition
function at each instanton density was determined by generating
5000 configurations with 32 instantons at 10 different coupling
constants $\alpha$. The variational ansatz (\ref{Z_free}) for
the partition function was determined from 250 initial sweeps with
the full interaction ($\alpha=1$). The integral (\ref{int_coup2})
was determined by gradually lowering the coupling to $\alpha=0$
and then raising it back to $\alpha=1$. The difference in the
result between the up and down sweeps provides an estimate of
the error in the integral due to incomplete equlibration. The
average between the up and down sweeps usually provides a good
estimate for the correct result, even if equilibration is slow
(as will be the case close to the phase transition).

  Fig.2a shows the free energy versus the instanton density (in units
of $\lambda^4$) for the pure gauge theory (without fermions). At small
density the free energy is roughly proportional to the density, but at
larger densities the repulsive interactions become important, leading to
a well-defined minimum. We also show the average action per instanton
as a function of density. The average action controls the probability
$\exp(-S)$ to find an instantons, but has no minimum in the range of
densities studied. This shows that the minimum in the free energy is a
compromise between maximum entropy and minimum action. If we fix our
units such that $N/V=1\,{\rm fm}^{-4}$ ($\Lambda=270$ MeV), we find
an instanton induced vacuum energy density $\epsilon=-526\,{\rm MeV}/
{\rm fm}^3$. This value is in very good agreement with the prediction
based on the trace anomaly
\be
\label{trace}
\epsilon = -\frac{b}{128\pi^2} <g^2G^2>.
\ee
For a dilute system of instantons, the right hand side is simply
proportional to the instanton density, $\epsilon=-b/4(N/V)$. For
$N/V=1\,{\rm fm}^{-4}$ this gives $\epsilon=-565\,{\rm MeV}/{\rm
fm}^3$, showing that our calculation is consistent with the trace
anomaly. For the variational calculation, this was shown in the
original work \cite{DP_84}.

   The corresponding results for QCD with two light and one intermediate
mass flavor are shown in figure 3. In order to avoid problems with
finite size effects, we take the light quark masses to be $0.1\Lambda$,
while the strange quark mass is near its physical value $0.7\Lambda$.
We find that the free energy as a function of the instanton density looks
very similar to the pure gauge case, but the minimum is shifted to
smaller densities. This does not necessarilly mean that the instanton
density is suppressed with respect to the pure gauge case, it simply
means that the scale parameter changes in the presence of light quarks.
Taking $N/V=1\,{\rm fm}^{-4}$ this gives $\Lambda = 310$ MeV and
$\epsilon=-280\,{\rm MeV}/{\rm fm}^3$. From fig.3 we can also read
off the quark condensate $\langle\bar qq\rangle=-(216 {\rm MeV})^3$,
in agreement with the phenomenological value. This number can be
compared to the quark condensate in 'quenched' QCD (calculated in
the pure gauge configurations, see fig.2c) which is given by
$\langle\bar qq\rangle=-(251 {\rm MeV})^3$, showing that light
quarks suppress the quark condensate.

    The distribution of instanton sizes, the eigenvalue distribution
of the Dirac operator and the distribution of fermionic overlap
matrix elements for the two cases discussed above are shown in
figures 4 and 5. The instanton size distribution shows the perturbative
$\rho^{b-5}$ behavior at small sizes, has a maximum at $\rho=0.5
\lambda^{-1}$ in quenched and $\rho=0.65\Lambda^{-1}$ in full
QCD, and falls off for large sizes. The average sizes are $\rho
=0.43$ fm in quenched and $\rho=0.42$ fm in full QCD. Lattice studies
of the size distribution in pure gauge $SU(2)$ were recently published
in \cite{MS_95}. These data compare well with with a calculation performed
in the interacting instanton model \cite{Shu_95}.

   The distribution of eigenvalues of the Dirac operator, $\hat D
\psi_\lambda=\lambda\psi_\lambda$, contains a lot of useful informtaion
about the spectrum of the theory. We will discuss some of these questions
in more detail below. Here we only mention that the Casher-Banks relation
$\langle\bar qq\rangle=-\rho(\lambda=0)/\pi$ connects the density of
eigenvalues of the Dirac operator near zero with the chiral condensate.
From figs.4b and 5b one clearly observes that the instanton liquid
does lead to a nonvanishing density of eigenvalues near $\lambda=0$.
The results also show that the presence of light quarks suppresses
the number of small eigenvalues. While the spectrum is peaked towards
small eigenvalues in the purge gauge case, it is essentially flat
in full QCD. This is consistent with the prediction from chiral
perurbation theory \cite{SS_93}
\be
\rho(\lambda)= -\pi \langle\bar qq\rangle +
   \frac{\langle\bar qq\rangle^2}{32\pi^2f_\pi^4}
   \frac{N_f^2-4}{N_f} |\lambda|+ \ldots \, .
\ee
The second term is zero for $N_f=2$ and leads to a singular behavior
for $N_f=3$. This result is connected with the fact that for three
flavors there is a Goldstone boson cut appearing in the scalar
isovector ($\delta$ meson) correlator, while there is no $\delta
\to \pi\pi$ decay allowed for two flavors.

   In fig.3c and 4c we also show the distribution of the largest fermionic
overlap matrix elements $T_{IA}$ for each instanton. This means that for
each instanton we select the antiinstanton it has the largest overlap
with and plot the resulting distribution of matrix elements. This
distribution is a measure of the strength of correlations among the
instantons. For a completely random system at the same density as the
simulated quenched and fully interacting ensembles, the distributions
would peak at $T_{IA}=x \Lambda^{-1}$ and $T_{IA}=y\Lambda^{-1}$,
respectively. Instead, the measured distributions peak at $z\Lambda^{-1}$
and $u\Lambda^{-1}$, showing that correlations are not very important
in the quenched ensemble, but play some role in the full ensemble.
A more physical measure of the importance of correlations among instantons
is given by the dependence of hadronic correlation functions on the
different ensembles. We will study this question in some detail in
a forthcoming publication.


%   discussion of IR fixed point

%An alternative proposal to overcome the infrared problem within
%the dilute instanton gas approximation was made in \cite{Shu_95}.
%If the full non-perturbative beta function has an infrared fixed
%point such that the coupling constant freezes at some non-zero vale,
%then the action of large instantons will also remain finite as the
%instanton size goes to infinity. In that case, the semiclassical
%instanton distribution $dn/d^4z\sim S^{2N_c}\exp(-S)/\rho^5$ is
%proportional to $1/\rho^5$ at large $\rho$ and there is no infrared
%problem. This approach stabilizes the instanton liquid but does not
%solve the problem of very close pairs. This is not necessarily a
%disaster for actual simulations, but means that many instantons
%will be 'sterile', i.e. coupled to a very close instanton of the
%opposite charge. It is therfore more economic to solve the infrared
%problem by introducing a hard core.


\section{The phase diagram of the instanton liquid}

    After fixing the the overall scale at zero temperature it is
now straightforward to extend our calculation to finite temperature.
In a euclidean field theory, finite temperature only enters through the
boundary conditions obeyed by the fields. The gauge fields have to be
periodic in the euclidean time direction with the period given by
the inverse temperature while fermions are subject to antiperiodic
boundary conditions. The corresponding periodic instanton and fermion
zero mode profiles can be constructed using 't Hoofts multi-instanton
solution \cite{HS_78}. These profiles are then used to study the
instanton interaction and fermionic overlap matrix elements. For the
ratio ansatz, such a study was performed in \cite{SV_91}. A
parametrization of this interaction is given in appendix C and D.

    The most important qualitative feature of the instanton interaction
at finite temperature is the form of the fermionic overlap matrix
elements. The determinant for one instanton-antiinstanton pair
separated by $\tau$ in euclidean time and $r$ in the spatial direction
is proportional to
\be
\label{detD}
  \det(\hat D) \sim \left| \frac{\sin(\pi T\tau)}{\cosh(\pi T r)}
                    \right|^{2N_f} .
\ee
The form of this interaction is a simple consequence of the periodicity
in the time direction and the fact that fermion propagators are screened
in the spatial direction by the lowest Matsubara frequency $\pi T$.
The interaction clearly favors instanton-antiinstanton pairs (molecules)
that are aligned along the time direction with a separation $\tau=1/(2T)$.
Ilgenfritz and Shuryak \cite{IS_94} proposed that this feature of the
interaction leads to a phase transition in the instanton liquid. In
this phase transition, the system would go from a random liquid to an
ordered phase of instanton-antiinstanton molecules. The transition
was studied in a schematic model in \cite{IS_94,SSV_95}. In the present
work we want to verify that a consistent treatment of the full
partition function of the instanton liquid does indeed lead to the
expected phase transition and perform a quantitative study of bulk
properties of the system near the transition.

    Before we consider the QCD case (with two light and one intermediate
mass flavor) in detail, it is instructive to study the phase diagram of
the instanton liquid for a wider range of theories. In order to cover
many different parameters, we have restricted ourselves to an exploratory
study, in which we do not minimize the free energy of the system, but
consider the instanton ensemble at a fixed density. In our experience,
if the system undergoes a phase transition, it will do so also if the
density is kept fixed (maybe at a different temperature). Nevertheless,
this point will have to be checked in the future.

   The case of only one quark flavor is very interesting. On the one hand,
there are no pions, but only a flavor singlet (the analog of the $\eta'$),
so the only chiral symmetry is the anomalous axial $U_A(1)$. Therefore,
one would not expect a chiral restoration phase transition. On the other
hand, a (first order) transition may exist even without a symmetry, and
our arguments concerning the formation of $\bar I I$ molecules in principle
also apply to the case $N_f=1$. Clearly, fermion induced correlations
become stronger as the number of flavors increases, so whether they are
strong enough for $N_f=1$ to induce a phase transition has to be studied
by performing simulations.

   We have studied the $N_f=1$ system at an instanton density $N/V=1\,
{\rm fm}^{-4}$ for temperatures up $T\simeq 300$ MeV.  We have observed
no phase transtition, the condensate $\langle\bar qq\rangle$ changes
smoothly and is non-zero even at the highest temperature studied. Of
course, even in pure Yang Mills theory (for $N_c=3$) there is a strong
first order deconfinement phase transition at a temperature $T\approx
250$ MeV (where the scale has been set by the quenched $\rho$ meson
mass). The interacting instanton model, however, does not give confinement,
so, naturally, there is no deconfinement transition.

    The case of two light flavors also is quite special. According to
standard universality arguments \cite{PW_84}, the chiral phase transition
is exepected to be of second order, with the same critical exponents as
the $O(4)$ Ising Model. Recently, Kocic and Kogut \cite{KK_95} have
challenged these arguments and suggested that the transition is indeed
second order, but with mean field exponents. Lattice results (see the review
\cite{deT_95}) seem to verify that the transition is second order, but
are not yet sufficiently accurate to determine the critical exponents.

   We have simulated $N_f=2$ ensembles for various temperatures and
values of the quark masses. For sufficiently small quark masses
$m<0.15\Lambda$ we find large fluctuations in the condensate at
$T\simeq 150 MeV$. To illustrate this fact, we show a typical time
history of the quark condensate in fig.6. Every configuration
corresponds to one complete sweep through the instanton ensemble,
with one Metropolis hit performed on every collective variable.
It is clear that the transition is either second order or first
order with a very small barrier between the two phases. In order
to distinguish these two possibilities one would have to perform
a finite size scaling analysis, which goes beyond the scope of
this paper. We will see later that for more flavors (including
the physical QCD) there is a much larger barrier between the two
phases. In fig.6 we also show the Dirac spectrum for $T=150$ MeV
and $m=5$ MeV. The eigenvalue density extrapolates to zero at
$\lambda=0$, but the slope is very steep and there are no signs
of gap in the spectrum.

   We will treat the three flavor case in detail in the next section,
and now proceed to the case of several flavors of light quarks.
Let us first mention several theoretical arguments suggesting
qualitative changes in the vacuum structure as $N_f$ increases.
The first of them, (being a kind of folklore) deals with the
observation that for $N_f>13$ the number of ``pions" $N_{\pi}
=N_f^2-1$ exceeds the number of quarks and antiquarks  $N_{q}
=4 N_c N_f$. This prevents the usual matching of the pion gas
at low temperature to the high-$T$ quark-gluon plasma with a
positive bag constant. It seems plausible that something
drastic should happen before this point is reached. In addition
to that it was noticed in \cite{Shu_88} that a large number
of flavors favors the formation of the $I\bar I$ molecules, and
that for $N_f>?$ the total density of molecules appears to become
ultraviolet divergent (the density diverges at small $\rho$).
%         really ?
%However, this fact do $not$ imply any serious changes in the physics,
%because this divergence does not appear in Green functions of quarks or
%gluons and affect only the vaccum energy.

  Third, lattice simulations have observed a strange ``bulk transition"
for $N_f=8$ \cite{Chr_93}. This either implies a radical change in the
ground state, or a new type of lattice artifacts. Finally, some progress
has been made in understanding $N=1$ supersymmetric generalizations of
QCD \cite{Sei_94}. One result that is of particular interest in this
context is the fact that one can show that there is a critical number
of flavors $N_f^{\rm crit}=N_c+1$ such that chiral symmetry is broken
for $N_f<N_f^{\rm crit}$, but not for $N_f\geq N_f^{\rm crit}$.

   Let us now report on our results concerning the vacuum structure of
the instanton liquid for many flavors. A useful tool to analyze the
structure of chiral symmetry breaking that was suggested by the
Columbia group \cite{Cha_95} is the ``valence quark mass"
dependence of the quark condensate, defined by
\be
\label{qq_val}
<\bar qq(m_v)> = \int d\lambda\,\rho(\lambda,m)
  \frac{m_v}{\lambda^2+m_v^2}\, .
\ee
Here, $\rho(\lambda,m)$ is the eigenvalue density of the Dirac operator,
calculated from configurations generated with the dynamical mass $m$.
The quark condensate in the chiral limit is given by taking $m=m_v\to 0$
in the thermodynamic limit. In a finite system, however, chiral symmetry
is never broken at nonzero quark mass and one has to perform a detailed
scaling analysis. This analysis can of course be done by performing many
simulations at various values of the dynamical mass, which, however, is
a very time consuming procedure. Instead, one can get often get a
good indication of the full behavior by studying the valence mass
dependence of the condensate. At very small $m_v$ the condensate
is proportional to $m_v$ which means that one is sensitive to
induced rather than spontateous chiral symmetry breaking. At very
large mass $m_v$ the condensate is inversely proportional to $m_v$.
Both of these limits are of course unphysical, and spontaneous
symmetry breaking in the continuum limit would be indicated by the
appearance of a plateau in between those two limits.

   In fig.7 we show a number of calculations of the valence mass dependence
of the condensate. The results in fig 7a were obtained for 64 instantons
in a cubic box $V=(2.828\,{\rm fm})^4$ (corresponding to a density $N/V
=1\,{\rm fm}^{-4}$ and a fairly low temperature, $T=71$ MeV) with the
dynamical quark mass $m=20$ MeV and different number of flavors $N_f$=1-8.
Plateaus where the condesate depends only weakly on $m_v$ are clearly
seen for the former $N_f=1,2,3$. These plateaus are clearly absent for
$N_f=5$ or larger: we therefore conclude that the instanton liquid
model has a chirally symmetric ground state in these cases. A more
detailed study of the configurations shows, that all instantons are
bound into molecules. The borderline case appears to be $N_f=4$: if
a condensate is present, it is significantly smaller than in QCD and
due to a relatively small random component of the vacuum.

    The next question we would like to address is the dependence of these
results on the dynamical quark mass $m$. Increasing the dynamical mass, one
decreases the influence of quark determinant, so that a large mass works
against the correlations induced by a large number of flavors. This means
that for a sufficiently large quark mass one can find ''spontaneous"
symmetry breaking even if there is no symmetry breaking in the chiral
limit. This is demonstrated in Fig.7b, where we show a series of calculations
for $N_f=5$ with the dynamical quark mass $m$ ranging from 20 to 100 MeV.
In this case a plateau in the valence mass dependence of the condensate
reappears at rather small critical quark mass of about 35 MeV.

   Repeating these studies for different numbers of flavors and
varying the mass and temperature, one can map out the phase structure
of the instanton liquid in the temperature-mass plane. In fig.8 we
have summarized our results for $N_f=2,3$ and 5. Simulations in which
we have found a signal for chiral symmetry breakdown are marked by
open squares, those where the system is in the restored phase by
solid squares. The (approximate) location of the discontinuity line
between the two phases is marked by the stars connected by dashed
lines. For two flavors we do not find such a line of discontinuities.
We have performed a number of runs in the vicinity of the phase
transition, all showing large fluctuations of the condesate. For
larger $N_f$ we clearly see a transition line with a discontinuity
of the condensate. When the number of flavors increases, one end of
this line moves to the left (the critical temperature $T_c$ decreases
with $N_f$), crossing zero somewhere around $N_f=4$, but before $N_f=5$.
In these cases, the ground state exhibits spontaneous symmetry breaking
only if the quark mass exceed some critical value. We have not tried
to follow the discontinuity lines much beyond $T=200 MeV$, since the
condensate becomes more and more dominated by very few extremely small
eigenvalues, so that numerical accuracy starts to become a problem.

\section{The instanton ensemble at finite temperature}

  In this section we want to discuss in detail the physically
relevant case of two light and one intermediate mass flavor. In
particular, we will perform selfconsistent simulations where the
correct instanton density is determined from minimizing the free
energy. The instanton interaction and fermionic matrix elements
that enter these calculations have already been discussed in the
last section. The semiclassical calculation that leads to the
instanton distribution (\ref{idens}) has also been generalized to
finite temperature \cite{PY_80}, giving
\be
\label{pis}
  d(\rho,T) &=& d(\rho,T=0) \exp(-\frac{1}{3}(2N_c+N_f)
    (\pi\rho T)^2- B(z) ) \\
   B(z) &=& \left(1+\frac{N_c}{6}-\frac{N_f}{6}\right)
    \left(-\log \left( 1+\frac{z^2}{3} \right)
    + \frac{0.15}{(1+0.15 z^{-3/2})^8} \right) \nonumber
\ee
with $z=\pi\rho T$. This correction factor leads to an exponential
suppression of large instantons at hight temperature. Its origin
is mainly the scattering of thermal gluons on the instanton. This
phenomenon is similar to the Debye mass, which in fact has the same
dependence on $N_c$ and $N_f$. It was therefore argued in \cite{SV_94}
that one should not use this perturbative suppression factor at
tempeartures below the phase transition. This suggestion was indeed
verified by lattice simulations for pure gauge theory \cite{CS_95},
which found a very week temperature dependence of the instanton
density below $T_c$, and an exponential suppression of instantons
consistent with (\ref{pis}) above $T_c$. In practice we have determined
the phase transition temperature without the suppresion factor
(\ref{pis}) and then for the final simulations multiplied the
term in the exponent by $(1-\tanh(\frac{T-T_c}{\Delta T})/2$.


   Now, when the partition function is fixed
   at any T, we may proceed to simulations. The free energy is always
taken at its minimum:
 the resulting T-dependence of the instanton density is shown in Fig.??(a).
Although it is not very strong, the corresponding behaviour of the quark
condensate (Fig.??(b)) is very different: in fact, our interacting instanton
liquid has a clear phase transition, with restoration of the chiral symmetry.
It is also encouraging to see that the transition point is
$T_c\approx 150 MeV$, in good agreement with lattice results.


    As it was already mentioned above, in this paper we are specifically
intersted in the correlations between instantons. At T=0 two
 $global$ aspects  of those correlations
were already studied in a
separate paper \cite{SV_94}: (i)
the so called screening of the topological charge by light quarks, and (ii)
fluctuations of the total instanton density, related with the low energy
theorem derived by Novikov et al. (Both were found to agree well
with general theory.) Now we aim at more microscopic
manifestations of those correlations, namely pair-wise correlations between
instantons and antiinstantons, leading eventually to well formed ``molecules".

In Fig.?? we compare some typical
configurations at T=0 and T$\approx T_c$:
one can clearly see (i) that molecules are indeed formed, and (ii) that they
are indeed polarised in time directions, as suggested in \cite{}.

Statistically detailed description is given by the histogramm of the relative
color orientations, shown in Fig.??. As one can see, this distribution is
rather flat at T=0, but develop a large peak at $cos(\alpha)=1$
in the critical region.

\section{Dirac eigenvalues and susceptibilities}

   Starting with classical paper by Casher and Banks \cite{BC_80},
many useful relations have been pointed out
between various observables and the spectrum of the fermionic Dirac operator
$\hat D \psi_\lambda=\lambda \psi_\lambda$. In a way,
in quantum field theories the region of small
$\lambda \rightarrow 0$ is truly the analog of the Fermi surface in the solid
state physics, and understanding of how many states exist close to it
is crucial for understanding of many properties, both at zero and non-zero
temperatures.

  Let us recall  the relations we will need below in relation to our
data.


   At $T\rightarrow T_c$ the quark condensate vanishes: if the $N_f=2$ case
has the second order phase transition, this is determined by the critical
index $\delta$
\be
<\bar q q>|_{T=T_c} \sim m^{1/\delta}
\ee
The value of this and other critical indices remains a matter of controvercy.
Acording to Pisarski-Wilczek suggestion
\cite{PW_84}, the $N_f=2$ case is analogous
to O(4) Ising-type model, which has $1/\delta=0.38$. However, Kocic and Kogut
\cite{KK_95} argued that this analogy is in fact wrong, and in theories with
critical fields composed of fermions one falls into a different universality
class, with the {\it mean field} indices, so $1/\delta=1/3$. Lattice works
have not yet been able to separate these two scenarios unambigously.

  Anyway, comparing it with Casher-Banks formula
\be
<\bar q q(T)>= \int d\lambda \rho(\lambda,m,T) {m \over \lambda^2 +m^2}
\ee
one concludes that at $T=T_c$ it should then develop a  $singularity$,
e.g.
\be
\rho(\lambda,T>T_c) \sim \lambda^{1/\delta-\alpha}m^\alpha
\ee
Further interesting information following from the Dirac spectrum is
related with susceptibilties. Let us start with the ``flavored" ones (e.g.
with operators made of different light quarks
$\bar u \Gamma d$), for which only connected diagram exist. Using the general
form of the propagator in $\lambda$-representation and integrating over
space-time, one gets the following general relations
for the pion and isovector scalar channels\footnote{The former one is
the basis for Smilga-Stern theorem mentioned above.}
\be
\chi_\pi    &=&  2 \int d\lambda \, \rho(\lambda,m,T)
                    \frac{1}{\lambda^2+m^2} \\
\chi_\delta &=&  2 \int d\lambda \, \rho(\lambda,m,T)
                    \frac{\lambda^2-m^2}{(\lambda^2+m^2)^2}
\ee
One of the important aspect of the QCD phase transition is
related to $U(1)_A$ restoration problem \cite{Shu_94}.
The difference between these two susceptibilities is directly related to it
\footnote{This was pointed out to us by N.Christ, whom we thank
for this comment}: if it vanishes
above $T_c$, the symmetry is restored
together with the usual $SU(N_f)$ chiral symmetry.
(For example, if the spectrum has the power shape (??), it means that
$1/\delta > 1$.)

   There is one more useful susceptibility discussed in literature,
corresponding to the scalar isoscalar channel we (old-fashionably) denote by
$\sigma$
\be
\chi_\sigma = \chi_\delta + 2\left( <\bar qq^2>-<\bar qq>^2 \right)
\ee
which however cannot be directly read from the $average$ spectral density
$\rho(\lambda)$ because of the first term.  Restoration of the
usual $SU(N_f)$ chiral symmetry above $T_c$ demand it to coincide with
that of the pion, its chiral partner channel.

   Now we are ready to present our results. In judging them, one should keep
in mind that they do not correspond to the massless limit: all our quark
masses are non-zero (and as explained above, the light ones are even heavier
than in the real world).

   We have calculated the spectrum
of  $\rho(\lambda)$ by diagonalization of the
Dirac operator (written in a matrix form in a subspace of instanton zero
modes): the result for several temperatures is shown in Fig.??.
   First of all, this spectrum does not exist below
very small $\lambda=O(1/V)$ because of
the finite size effects. By now, there exist a
 very detailed theory of those $mesoscopic$ effects worked out in
\cite{LS_92,VZ_93}: it predicts the spectrum in this region to have certain
universal shape, depending on the single combination $m V <\bar q q>$, which
was also sucsessfully compared to lattice and instanton-based calculations.

   Second, at T=0 this spectrum is very flat at small $\lambda$.
As our strange quark mass is rather large compared
to $\lambda$ values in question,
at this point QCD effectively acts as the $N_f=2$ case,
 and thefore such flat $\rho(\lambda)$
agrees with the Smilga-Stern theorem (??).

   Behaviour of the susceptibilities is shown in Fig.??.
The value of the index $\delta$ in the critical region ???.
Above $T_c$, we have clear indications for the restoration of U(1)
chiral symmetry

\section{Conclusions}

   We have studied the structure of the instanton liquid at zero
and finite temperature.

\section{Acknowledgements}
  The reported work was partially supported by the US DOE grant
DE-FG-88ER40388. Most of the calculations presented in this work
were performed at the NERSC at Lawrence Livermore.

\newpage
\appendix

\section{Instanton interaction at zero temperature}

    In this appendix we specify the classical instanton interaction
used in our simulations. From the form of the action for the
two-instanton ansatz, one can easily show that the general form
of the instanton-antiinstanton interaction in the case of the
gauge group $SU(2)$ is given by
\be
\label{su2_int}
 S_{int} = \beta_1 (\bar\rho) \left(
  s_0(R)+s_1(R)(u\cdot\hat R)^2+s_2(R)(u\cdot\hat R)^4  \right),
\ee
where $u_\mu$ is the orientation vector introduced in section 2 and
$\hat R_\mu$ is the unit vector connecting the centers of the
instanton and anti-instanton. The interaction is given in units
of the single instanton action $S_0=\beta_1(\bar\rho)$. The argument
of the beta function is not uniquely determined without a higher
order calculation. In practice we have taken the geometric average
$\bar\rho=\sqrt{\rho_I\rho_A}$ of the instanton radii.

   The interaction (\ref{su2_int}) can be embedded into $SU(3)$ by
making the replacement $(u\cdot\hat R)^2\rightarrow |u\cdot\hat R|^2$
and multiplying the first term by $|u|^2$. For the ratio ansatz, the
instanton-anti-instanton interaction can be parametrized by \cite{SV_91}
\be
\label{S_IA,ratio}
 \frac{S_{IA}}{\beta_1} &=&
   \left[ \frac{4.0}{(r^2+2.0)^2}
   - \frac{1.66}{(1+1.68 r^2)^3}
   + \frac{0.72\log (r^2)}{(1+0.42 r^2)^4}
    \right] |u|^2  \\
   & & \hspace{1cm} +  \left[ -\frac{16.0}{(r^2+2.0)^2}
         + \frac{2.73}{(1+0.33 r^2)^3}\right]
         |u\cdot\hat R|^2 \nonumber
\ee
where $r=R/\sqrt{\rho_I\rho_A}$ is the instanton-anti-instanton separation
in units of the geometric mean of their radii. In the ratio ansatz, the
instanton-instanton interaction is non-zero if the two instantons have
different color orientations. The interaction can be parametrized by
\be
\label{S_II,ratio}
 \frac{S_{II}}{\beta_1} &=&
         \frac{1}{(1+0.43 r^2)^3}
         \left[ 0.63 |\vec u|^2 + 0.071 |\vec u|^4 \right] \\
   & & \hspace{0.6cm} +
         \frac{\log (r^2)}{(1+1.17 r^2)^4}
         \left[ 0.05 |\vec u|^2 + 0.47 |\vec u|^4 \right]
        \nonumber .
\ee
In the Yung ansatz, conformal symmetry dictates that the interaction
depends on the relative separation and the instanton radii only through
the conformal parameter
\be
\label{conf_param}
 \lambda = \frac{R^2+\rho_I^2+\rho_A^2}{2\rho_I\rho_A}
  + \left( \frac{(R^2+\rho_I^2+\rho_A^2)^2}{4\rho_I^2\rho_A^2}
   - 1\right)^{1/2}.
\ee
The instanton-anti-instanton interaction is then given by \cite{Ver_91}
\be
\label{S_IA,Yung}
 \frac{S_{IA}}{\beta_1} &=&
   \frac{4}{(\lambda^2-1)^3}  \Big\{
   \left(\lambda^4-2\lambda^2+1+(2-2\lambda^2)\log(\lambda)\right)|u|^2\\
   & & \hspace{1cm} +
   \left(-4\lambda^4+4\lambda^2+(-4+12\lambda^2)\log(\lambda)\right)
         |u\cdot\hat R|^2
         \nonumber \\
   & & \hspace{1cm} +
   \left(-8\lambda^4+8+(8+8\lambda^2)\log(\lambda)\right)
         |u\cdot\hat R|^4
         \Big\}\nonumber .
\ee
As discussed in section 2, the instanton-instanton interaction vanishes
in the Yung or streamline ansatz.

\section{Fermionic overlap matrix elements at zero temperature}

   A parametrization of the fermionic overlap matrix element
between instantons and anti-instantons in the sum ansatz was
already given in section 2
\be
\label{sum_overl2}
 T_{IA} = i (u\cdot R)  \frac{1}{\sqrt{\rho_I\rho_A}}
    \frac{4.0}{(2.0+R^2/\rho_I\rho_A)^2} .
\ee
The matrix element in the ratio ansatz is very similar and we
employ the same parametrization. In the Yung or streamline ansatz,
the matrix element depends on the separation and the radii again only
through the conformal parameter $\lambda$. The matrix element can
be paramerized by \cite{SV_92}
\be
\label{stream_over}
 T_{IA} = i (u\cdot R)  \frac{1}{\sqrt{\rho_I\rho_A}}
    \frac{c_1 \lambda^{3/2}}
         {(1+1.25(\lambda^2-1)+c_2(\lambda^2-1)^2 )^{3/4}},
\ee
with $c_1=$ and $c_2=$.

\section{Instanton interaction at finite temperature}

    In this appendix we give a parametrization of the classical
instanton interaction at finite temperature. For reasons explained
in section 4, there is no analog of the streamline interaction
at finite tempearture and we only specify the interaction in the
ratio ansatz. Up to small changes
that have been introduced in order to improve the parametrization
at small temperatures, the interaction is identical to the one
given in \cite{SV_91}. Here we also specify how to embed the
parametrization given in that reference for the case of an $SU(2)$
gauge group into $SU(3)$.

    At finite tempearture, the interaction is still at most quartic
in the relative orientation vector $u_\mu$. However, since four
dimensional rotational invariance is broken, the interaction depends
on $|u_4|^2$ in addition to the invariants $|u\cdot\hat R|$ and
$|u|^2$ appearing in the zero temperature interaction. For the
same reason, the interaction can now depend separately on the
spatial and temporal components of the vector $R_\mu=(\vec R,R_4)$.
At temperatures of interest here, the anisotropy in the dependence
on $R_\mu$ turns out to be small. The parametrization
\be
\label{S_IA,finiteT}
 \frac{S_{IA}}{\beta_1} &=&
   \frac{4.0}{(r^2+2.0)^2} \frac{\beta^2}{\beta^2+5.21}|u|^2
   - \left[ \frac{1.66}{(1+1.68 r^2)^3}
   + \frac{0.72\log (r^2)}{(1+0.42 r^2)^4} \right]
     \frac{\beta^2}{\beta^2+0.75} |u|^2  \\
   & & \hspace{0.6cm} +  \left[ -\frac{16.0}{(r^2+2.0)^2}
         + \frac{2.73}{(1+0.33 r^2)^3}\right]
         \frac{\beta^2}{\beta^2+0.24+11.50 r^2/(1+1.14 r^2)}
         |u\cdot\hat R|^2
         \nonumber \\
   & & \hspace{0.6cm} +
         0.36\log \left( 1+\frac{\beta}{r}\right)
         \frac{1}{(1+0.013 r^2)^4}
         \frac{1}{\beta^2+1.73}
         (|u|^2-|u\cdot\hat R|^2-|u_4|^2)
         \nonumber
\ee
therefore depends only on $r=R/\sqrt{\rho_I\rho_A}$ with $R=(\vec R^2
+R_4^2)^{1/2}$. The inverse temperature $\beta=1/(T\sqrt{\rho_I\rho_A})$
is also given
in units of the mean of the instanton radii. One can easily check that
for $T\rightarrow 0$ the interaction (\ref{S_IA,finiteT}) reduces to
the zero temperature ratio ansatz (\ref{S_IA,ratio}). The interaction
between two instantons can be parmetrized by
\be
\label{S_II,finiteT}
 \frac{S_{II}}{\beta_1} &=&
         \frac{1}{(1+0.43 r^2)^3}
         \frac{\beta^2}{\beta^2+5.33}
         \left[ 0.63 |\vec u|^2 + 0.071 |\vec u|^4 \right] \\
   & & \hspace{0.6cm} +
         \frac{\log (r^2)}{(1+1.17 r^2)^4}
         \frac{\beta^2}{\beta^2+1.17}
         \left[ 0.05 |\vec u|^2 + 0.47 |\vec u|^4 \right]
        \nonumber  \\
   & & \hspace{0.6cm} +
       \log\left(1+\frac{\beta}{r}\right)
       \frac{1}{\beta^2+2.08}
       \left[ 0.05 |\vec u|^2 + 0.47 |\vec u|^4 \right]
        \nonumber
\ee

\section{Fermionic overlap matrix elements at finite
temperature}

   At finite temperature, the fermionic overlap matrix element is
still linear in the relative orientation vector $u_\mu$. Due to the
loss of Lorentz invariance at finite tempearture, the matrix element
depends separately on $u_4$ and $\vec u\cdot\vec R$
\be
\label{overl_finiteT}
 T_{IA} = i u_4 f_1 + \frac{(\vec u\cdot R)}{R} f_2 .
\ee
A parametrization of the functions $f_{1,2}$ was given in \cite{SV_91}.
We have changed this parametrization slightly in order to improve
the behavior at small tempeartures. The result for $f_1$ is
%\be
%f_1 &=& \frac{ (\pi/\beta)\sin(\pi\tau/\beta)\cosh(\pi r/\beta)}
%             {(\cosh(\pi r/\beta)-\cos(\pi\tau/\beta)+\kappa_1^2)^2} \\
%    & &\cdot \frac{1}{(\beta^2/\pi^2)
%                 (\exp(-\pi r/2\beta)+\pi r/2\beta)
%             +   (2/\pi)(1-0.69\exp(-1.75r/\beta))}
%             \nonumber \\
%    & & \cdot \left( 1 + \frac{0.76\beta^2}
%                        {(0.82r^2+1)^2(1+0.19\beta^2)^2} \right)
%        \left( 1 + \left(\frac{\pi\tau}{\beta}\right)^2
%                \frac{0.18}{1+0.12r^2} \right),
%\ee
\be
\label{f1}
 f_1 &=& \frac{\left(\frac{\nsz\pi}{\nsz\beta}\right)
             \sin\left(\frac{\nsz\pi\tau}{\nsz\beta}\right)
              \cosh\left(\frac{\nsz\pi r}{\nsz\beta}\right)}
              {\left(\cosh\left(\frac{\nsz\pi r}{\nsz\beta}\right)-
               \cos\left(\frac{\nsz\pi\tau}{\nsz\beta}\right)+
               \kappa_1^2\right)^2} \\
     & & \cdot \frac{1}{\left(\frac{\nsz\beta^2}{\nsz\pi^2}\right)
                  \left(\exp\left(-\frac{\nsz\pi r}{\nsz 2\beta}\right)+
                  \frac{\nsz\pi r}{\nsz 2\beta}\right)
              +   \left(\frac{\nsz 2}{\nsz\pi}\right)\left(1-0.69
                  \exp\left(-\frac{\nsz 1.75r}{\nsz\beta} \right)\right)}
              \nonumber \\
     & & \cdot \left( 1 + \frac{0.76\beta^2}
                         {(0.82r^2+1)^2(1+0.19\beta^2)^2} \right)
         \left( 1 + \left(\frac{\pi\tau}{\beta}\right)^2
                 \frac{0.18}{1+0.12r^2} \right) \nonumber,
\ee
where again $r=|\vec R|/\sqrt{\rho_I\rho_A}$, $\tau=R_4/\sqrt{\rho_I
\rho_A}$ and $\beta=1/(T\sqrt{\rho_I\rho_A})$ are given in units of
the mean instanton radius. The  result for $f_2$ is
\be
\label{f2}
 f_2 &=& \frac{\left(\frac{\nsz\pi}{\nsz\beta}\right)
             \cos\left(\frac{\nsz\pi\tau}{\nsz\beta}\right)
              \sinh\left(\frac{\nsz\pi r}{\nsz\beta}\right)}
              {\left(\cosh\left(\frac{\nsz\pi r}{\nsz\beta}\right)-
               \cos\left(\frac{\nsz\pi\tau}{\nsz\beta}\right)+
               \kappa_2^2\right)^2} \\
     & & \cdot \frac{1}{\left(\frac{\nsz\beta^2}{\nsz\pi^2}\right)
                  \left(\exp\left(-\frac{\nsz 2.06\pi r}{\nsz\beta}\right)+
                  \frac{\nsz\pi r}{\nsz 2\beta}\right)
              +   \left(\frac{\nsz 2}{\nsz\pi}\right)\left(1+0.42
                  \exp\left(-\frac{\nsz 0.34 r}{\nsz\beta} \right)\right)}
              \nonumber ,
\ee
with $\kappa_{1,2}$ given by
\be
\label{kappa12}
   \kappa_1^2 = \frac{1}
   {0.53+\left(\frac{\nsz\beta^2}{\nsz\pi^2}\right)},
   \hspace{0.5cm}
   \kappa_2^2 = \frac{1}
   {0.69+\left(\frac{\nsz\beta^2}{\nsz\pi^2}\right)}.
\ee
One can easily verify that the parametrization (\ref{overl_finiteT}-
\ref{kappa12}) reduces to (\ref{sum_overl2}) in the zero tempearture
limit.



\newpage

\begin{thebibliography}{xx}

\bibitem{BPST_75}
A.~A.~Belavin, A.~M.~Polyakov, A.~S.~Schwartz, Y.~S.~Tyupkin,
Phys.~Lett.~{\bf B59}, 85 (1975)

% the famous paper
\bibitem{tHo_76}
G.~'t~Hooft,
Phys. Rev. {\bf 14D}, 3432 (1976)

\bibitem{CDG_78}
C.~G.~Callan, R.~Dashen and D.~J.~Gross,
Phys.~Rev.~{\bf D17}, 2717 (1978)

% instantons in QCD
\bibitem{Shu_82}
E.~V.~Shuryak,
Nucl. Phys. {\bf B203}, 93, 116 (1982)

% inst and screening
\bibitem{Shu_78}
E.~V.~Shuryak,
Phys.~Lett.~{\bf B79}, 135 (1978)

% variational model
\bibitem{DP_84}
D.~I.~Diakonov, V.~Yu.~Petrov,
Nucl.~Phys.~{\bf B245}, 259 (1984)

% DP on light quarks ...
\bibitem{DP_86}
D.~I.~Diakonov, V.~Yu.~Petrov,
Nucl.~Phys.~{\bf B272}, 457 (1986)

% towards a quantitative theory ...
\bibitem{Shu_88}
E.~Shuryak,
Nucl.~Phys.~{\bf B302}, 559, 574, 599 (1988)

% first correlators
\bibitem{Shu_89}
E.~Shuryak,
Nucl.~Phys.~{\bf B319}, 521, 541 (1989)

% csb and correlations in the inst. liquid
\bibitem{SV_90}
E.~Shuryak, J.J.M.Verbaarschot,
Nucl.~Phys.~{\bf B341}, 1 (1990).

% random correlators
\bibitem{SV_93}
E.~Shuryak, J.~J.~M.~Verbaarschot,
Nucl.~Phys.~{\bf B410}, 55 (1993);
T.~Sch\"afer, E.~V.~Shuryak, J.~J.~M.~Verbaarschot,
Nucl.~Phys.~{\bf B412}, 143 (1994)

% interacting correlators
\bibitem{SSV_95b}
T.~Sch\"afer, E.~V.~Shuryak, J.~J.~M.~Verbaarschot,
Hadronic correlation functions in the interacting instanton model,
preprint, SUNY-NTG-95xx

% wave functions
\bibitem{SS_94}
T.~Sch\"afer, E.~V.~Shuryak,
Phys.~Rev.~{\bf D50}, 478 (1994)

% Shuryak's correlator review
\bibitem{Shu_93}
E.~V.~Shuryak,
Rev.~Mod.~Phys.~{\bf 65}, 1 (1993)

% Negele's correlation functions
\bibitem{CGHN_93}
M.~C.~Chu, J.~M.~Grandy, S.~Huang, J.~W.~Negele,
Phys.~Rev.~Lett.~{\bf 70}, 225 (1993);
Phys.~Rev.~{\bf D 48}, 3340 (1993)

% old papers on cooling
\bibitem{Ber_81}
B.~Berg,
Phys.~Lett.~{\bf B114}, 475 (1981)
M.~Teper,
Nucl.~Phys.~{\bf B20} (Proc.~Suppl.), 159 (1991)

% Negele's cooled correlators
\bibitem{CGHN_94}
M.~C.~Chu, J.~M.~Grandy, S.~Huang , J.~W.~Negele,
Phys.~Rev.~{\bf D49}, 6039 (1994)

% instanton density at finite T, lattice
\bibitem{CS_95}
M.~C.~Chu, S.~Schramm,
Phys.~Rev.~{\bf D}, (1995), in print

% same
\bibitem{IMM_95}
E.-M. Ilgenfritz, M.~M\"uller-Preu{\ss}ker, E.~Meggiolaro,
Nucl.~Phys.~(proc.~Suppl.) {\bf B 42}, 496 (1995)

% instanton size distribution
\bibitem{MS_95}
C.~Michael, P.~S.~Spencer,
Phys.~Rev. {\bf Dxx}, xxxx (1995)

% T_c for N_f=2
\bibitem{BOD*_92}
C.~Bernard, M.~C.~Ogilvie, T.~A.~DeGrand, C.~E.~DeTar, S.~Gottlieb,
A.~Krasnitz, R.~L.~Sugar, D.~Touissant, Phys.~Rev.~{\bf D45}, 3854
(1992)

% the old molecular model
\bibitem{IS_89}
M.-E.~Ilgenfritz, E.~V.~Shuryak,
Nucl.~Phys.~{\bf B319}, 511 (1989)

% the new one
\bibitem{IS_94}
M.-E.Ilgenfritz, E.~V.~Shuryak,
Phys.~Lett.~{\bf B325}, 263 (1994)

% more molecules
\bibitem{SSV_95}
T.~Sch\"afer, E.~V.~Shuryak,
Phys.~Rev.~{\bf Dxx}, xx (1995)

% variational ansatz with ratio interaction
\bibitem{Shu_85}
E.~V.~Shuryak,
Phys.~Lett.~{\bf B153}, 162 (1985)

% finite T instantons
\bibitem{PY_80}
R.~D.~Pisarski, L.~G.~Yaffe,
Phys.~Lett.~{\bf B97}, 110 (1980);
D.~J.~Gross, R.~D.~Pisarski, L.~G.~Yaffe,
Rev.~Mod.~Phys.~{\bf 53}, 43 (1981)

% Yung ansatz
\bibitem{Yun_88}
A.~V.~Yung,
Nucl.~Phys.~{\bf B297}, 47 (1988)

% streamlines
\bibitem{Ver_91}
J.~J.~M.~Verbaarschot,
Nucl.~Phys.~{\bf B362}, 33 (1991)

% streamline matrix elements and fermion number violation
\bibitem{SV_92}
E.~Shuryak, J.~Verbaarschot,
Phys.~Rev.~Lett.~{\bf 68}, 2576 (1992)

% instanton density
\bibitem{SV_94}
E.~V.~Shuryak, M.~Velkovsky,
Phys.~Rev.~{\bf D50}, 3323 (1994)

% instanton sizes
\bibitem{Shu_95}
E.~V.~Shuryak,
preprint, SUNY-NTG 95

% variational model at finite T
\bibitem{DM_88}
D.~I.~Diakonov, A.~D.~Mirlin,
Phys.~Lett.~{\bf B203}, 299 (1988)

% similar with quarks
\bibitem{NVZ_89}
M.A. Nowak, J.J.M. Verbaarschot and I. Zahed,
Nucl. Phys. {\bf B325} (1989) 581.

% instanton solution at finite T
\bibitem{HS_78}
B.~J.~Harrington, H.~K.~Shepard,
Phys.~Rev.~{\bf D17}, 2122 (1978)
B.~Grossman,
Phys.~Lett.~{\bf A61}, 86 (1977)

% instanton interaction at finite T
\bibitem{SV_91}
E.~Shuryak, J.~J.~M.~Verbaarschot,
Nucl.~Phys.~{\bf B364}, 255 (1991)

% spectrum of the Dirac operator, simulations
\bibitem{Ver_94}
J.~J.~M.~Verbaarschot,
Act.~Phys.~Pol.~{\bf 25}, 133 (1994);
Nucl.~Phys.~{\bf Bxxx}, xx (1995)

% instanton vacuum at finite T
\bibitem{KY_91}
V. Khoze and A. Yung, Z. Phys. {\bf C50} (1991) 155.

% gluonic correlators
\bibitem{SS_95}
T.~Sch\"afer, E.~V.~Shuryak,
preprint, SUNY-NTG94-53

% temporal correlators
\bibitem{SS_95b}
T.~Sch\"afer, E.~V.~Shuryak,
Phys.~Lett.~{\bf B}, in print

% topological surface tension
\bibitem{SV_95}
E.~V.~Shuryak, J.~J.~M.~Verbaarschot,
Screening of the topological charge in a correlated
instanton model, preprint, SUNY-NTG94-25

% correlations
\bibitem{Shu_87}
E.V. Shuryak, Phys. Lett. {\bf 193B} (1987) 319.

% review
\bibitem{Pet_93}
B.~Peterson, Nucl. Phys. (Proc. Suppl.) {\bf B30}, 66 (1993)

% thermodynamics
\bibitem{KSW_91}
J.~B.~Kogut, D.~K.~Sinclair, K.~C.~Wang, Phys.~Lett.~{\bf B263},
101 (1991)

% engineering
\bibitem{KB_93}
V.~Koch, G.~E.~Brown,
Nucl.~Phys.~{\bf A560}, 345 (1993)

% SVZ
\bibitem{SVZ_79}
M.~A.~Shifman, A.~I.~Vainshtein, V.~I.~Zakharov,
Nucl.~Phys.~{\bf B 147} 385, (1979)

% fermion determinant
\bibitem{DP_85}
D.~I.~Diakonov, V.~Yu.~Petrov,
JETP 89, 361 (1985);
Phys.~Lett.~{\bf B147}, 357 (1984)

% effective interaction from instantons
\bibitem{SVZ_80}
M.~A.~Shifman, A.~I.~Vainshtein, V.~I.~Zakharov,
Nucl.~Phys.~{\bf B163}, 43 (1980)

% same
\bibitem{NVZ_89}
M.~A.~Nowak, J.~J.~M.~Verbaarschot, I.~Zahed,
Nucl.~Phys.~{\bf B324}, 1 (1989)

% NJL
\bibitem{NJL_61}
Y.~Nambu,G.~Jona-Lasinio, Phys.~Rev.~{\bf 122}, 345 (1961)

% NJL revival
\bibitem{KH_88}
T.~Kunihiro, T.~Hatsuda, Phys.~Lett.~{\bf B206}, 385 (1988);
V.~Bernard, R.~Jaffe, U.-G.~Meissner, Nucl.~Phys.~{\bf B308}, 753 (1988);
S.~Klimt, M.~Lutz, U.~Vogl, W.~Weise, Nucl.~Phys.~{\bf A516}, 429 (1990)

% on \pi T
\bibitem{Eletski}
V.~L.~Eletsky, B.~L.~Ioffe, Sov.~J.~Nucl.~Phys.~{\bf 48}, 384 (1988)

% the Goksch plot
\bibitem{Gocksch}
A.~Gocksch, Phys.~Rev.~Lett.~{\bf 67}, 1701 (1991)

% Ch phase transition  and U(1)_A restoration
\bibitem{PW_84}
R.~D.~Pisarski, F.~Wilczek,
Phys.~Rev.~{\bf D29}, 338 (1984)

% mean field behavior ?
\bibitem{KK_95}
A.~Kocic, J.~Kogut,
Phys.~Rev.~Lett.~{\bf xx}, xxx (1995)

% more on U(1)_A restoration
\bibitem{Shu_94}
E.~V.~Shuryak, Comm.~Nucl.~Part.~Phys. {\bf 21}, 235 (1994)

% susceptibility measurements
\bibitem{Gup_92}
S.~Gupta, Phys.~Lett.~{\bf B288}, 171 (1992)

% quark number susceptibilty
\bibitem{GLT*_88}
S.~Gottlieb, W.~Liu, D.~Toussaint, R.~L.~Renken, L.~Sugar,
Phys.~Rev.~{D38}, 2888 (1988)

% njl and quark number susceptibilty
\bibitem{Kun_91}
T.~Kunihiro, Phys.~Lett.~{\bf B271}, 395 (1991)

% more susceptibilities + temp correlators
\bibitem{BGK*_93}
G.~Boyd, S.~Gupta, F.~Karsch, E.~Laerman,
Z.~Phys.~{\bf C64}, 331 (1994)

% dimensional reduction
\bibitem{AP_81}
T.~Appelquist, R.~D.~Pisarski, Phys.~Rev.~{\bf D23}, 2305 (1981);
S.~Nadkarni, Phys.~Rev.~{\bf D27}, 917 (1983);
Phys.~Rev.~{\bf D38}, 3287 (1988)

% Casher Banks formula
\bibitem{BC_80}
T.~Banks, A.~Casher,
Nucl.~Phys.~{\bf B169}, 103 (1980)

% Sum Rules for the Dirac operator
\bibitem{LS_92}
H.~Leutwyler, A.~Smilga,
Phys.~Rev.~{\bf D46}, 5607 (1992)

% slope of eigenvalue density
\bibitem{SS_93}
A.~Smilga, J.~Stern,
Phys.~Lett.~{\bf B318}, 531 (1993)

% random matrix theory for the Dirac operator
\bibitem{VZ_93}
J.~J.~M.~Verbaarschot, I.~Zahed,
Phys.~Rev.~Lett.~{\bf 70}, 3852 (1993)

% Nf=8 lattice data
\bibitem{Chr_93}
N.~Christ,
Nucl.~Phys.~(Proc.~Suppl.) {\bf B30}, 323 (1993)

% valence mass dependence
\bibitem{Cha_95}
S.~Chandrasekharan,
Nucl.~Phys.~(Proc.~Suppl.) {\bf B42}, 475 (1995)

% lattice thermo review
\bibitem{deT_95}
C.~DeTar,
Quark Gluon Plasma in numerical simulations of lattice QCD,
to appear in ''Quark Gluon Plasma 2", R.~Hwa, ed.,
World Scientific (1995)

% lattice susceptibilities
\bibitem{KL_94}
F.~Karsch, E.~Laermann,
Phys.~Rev.~{\bf D50}, 6954 (1994)

% SUSY N=1 QCD
\bibitem{Sei_94}
N.~Seiberg,
Phys.~Rev.~{\bf D49}, 6857 (1994)

\end{thebibliography}
\newpage\noindent
{\Large\bf figure captions}\\ \\ \\
\underline{figure 1} Classical instanton-antiinstanton interaction
in the streamline (dash-dotted line) and ratio ansatz (short dahed
line). The interaction is given in units of the single instantanton
action $S_0$ for the most attractive ($\cos\theta=1$) and most repulsive
($\cos\theta=0$) orientations. The dash-dotted curves show the original
streamline interaction, while the solid curves show the interaction
including the core introduced in section 3. Fig.1b shows the fermionic
overlap matrix elements in the streamline (solid curve) and ratio ansatz
(dashed curve). The matrix elements are given in units of geometric mean
of the instanton radii.
\\ \\
\underline{figure 2} Free energy, average instanton size and quark
condensate as a function of the instanton density in the pure gauge
theory. All quantities are given in units of the scale parameter
$\Lambda_{QCD}$.
\\ \\
\underline{figure 3} Free energy, average instanton size and quark
condensate as a function of the instanton density in the theory with
two light and one intermediate mass flavor. All quantities are given in units
of the scale parameter $\Lambda_{QCD}$.
\\ \\
\underline{figure 4} Distribution of instanton sizes, eigenavalue
spectrum of the Dirac operator and distribution of fermionic
overlap matrix elements in pure gauge theory. $\rho,\lambda$
and $T_{IA}$ are given in units of the scale parameter.
\\ \\
\underline{figure 5} Distribution of instanton sizes, eigenavalue
spectrum of the Dirac operator and distribution of fermionic
overlap matrix elements in QCD.
\\ \\
\underline{figure 6} Time history of the quark condensate (in units
of ${\rm fm}^{-3}$ for $N_f$=2 in the critical region, $T=150$ MeV.
Also shown is the corresponding spectrum of Dirac eigenvalues $\lambda$.
\\ \\
\underline{figure 7} Double logarithmic plot of the quark condensate
(in units ${\rm fm}^{-3}$) versus the valence quark mass $m_v$ for
different number of light flavors (with dynamical mass $m=20$ MeV).
Fig.b shows the same plot for $N_f=5$ and different quark masses $m$.
\\ \\
\underline{figure 8} Schematic phase diagram of the instanton liquid for
different numbers of quark flavors, $N_f$=2,3 and 5. We show the state
of chiral symmetry in the temperature-quark mass planes. In the figure
for $N_f$=2, open squares indicate points where we found large fluctuations
of the chiral condensate, the cross indicates the approximate location
of the singularity. In two other figures the open squares correspond
to points where we find a plateau in the valence mass dependence of the
chiral condensate, while solid squares correspond to points where such
a plateau is absent. The crosses and the dashed lines connecting them
show the approximate location of the discontinuity line.
\\ \\
\underline{figure 9} Instanton density, free energy and quark condensate
as a function of temperature in the full theory with two light and
one intermediate mass flavor.
\\ \\
\underline{figure 10} Typical instanton ensembles for $1/T=3.00,1.80,1.40
\,\Lambda^{-1}$. The plots show projections of a four dimensional
$3^3\times T^{-1}$ box. Instantons and antiinstanton positions are
indicated by $+$ and $-$ symbols. Dashed, solid and thick solid
lines correspond to fermionic overlap matrix elemnts $T_{IA}>x,y,z$
respectively.
\\ \\
\underline{figure 11} Order parameter histories for $1/T=3.00,1.80,1.40
\,\Lambda^{-1}$.
\\ \\
\underline{figure 12} Spectrum of the Dirac operator for $1/T=3.00,1.80,1.40
\,\Lambda^{-1}$.
\\ \\
\underline{figure 13} Distribution of color orientation angles
$\cos^2\alpha=|u_4|^2/|u|^2$ for $1/T=3.00,1.80,1.40 \,\Lambda^{-1}$.
\\ \\
\underline{figure 15} Mesonic susceptibilities as a function of tempearture.
The scalar $\sigma$, scalar $\delta$ and pseudoscalar $\pi$ susceptibilities
are denoted by ...

\newpage
\begin{figure}
\begin{center}
\leavevmode
\epsffile{molec.ps}
\end{center}
\caption{}
\end{figure}
\vfill

\newpage
\begin{figure}
\begin{center}
\leavevmode
\epsffile{actnf0.ps}
\end{center}
\caption{}
\end{figure}
\vfill

\newpage
\begin{figure}
\begin{center}
\leavevmode
\epsffile{actnf3.ps}
\end{center}
\caption{}
\end{figure}
\vfill

\newpage
\begin{figure}
\begin{center}
\leavevmode
\epsffile{specnf0.ps}
\end{center}
\caption{}
\end{figure}
\vfill

\newpage
\begin{figure}
\begin{center}
\leavevmode
\epsffile{specnf3.ps}
\end{center}
\caption{}
\end{figure}
\vfill

\newpage
\begin{figure}
\begin{center}
\leavevmode
\epsffile{order.ps}
\end{center}
\caption{}
\end{figure}
\vfill

\newpage
\begin{figure}
\begin{center}
\leavevmode
\epsffile{val1.ps}
\end{center}
\caption{}
\end{figure}
\vfill

\newpage
\begin{figure}
\begin{center}
\leavevmode
\epsffile{phase.ps}
\end{center}
\caption{}
\end{figure}
\vfill

\newpage
\begin{figure}
\begin{center}
\leavevmode
\epsffile{dens.ps}
\end{center}
\caption{}
\end{figure}
\vfill

\newpage
\begin{figure}
\begin{center}
\leavevmode
\epsffile{conf300.ps}
\end{center}
\caption{}
\end{figure}
\vfill

\newpage
\begin{figure}
\begin{center}
\leavevmode
\epsffile{conf180.ps}
\end{center}
\caption{}
\end{figure}
\vfill
\begin{figure}
\begin{center}
\leavevmode
\epsffile{conf140.ps}
\end{center}
\caption{}
\end{figure}
\vfill

\newpage
\begin{figure}
\begin{center}
\leavevmode
\epsffile{history.ps}
\end{center}
\caption{}
\end{figure}
\vfill

\newpage
\begin{figure}
\begin{center}
\leavevmode
\epsffile{dirac.ps}
\end{center}
\caption{}
\end{figure}
\vfill

\newpage
\begin{figure}
\begin{center}
\leavevmode
\epsffile{angle.ps}
\end{center}
\caption{}
\end{figure}
\vfill

\newpage
\begin{figure}
\begin{center}
\leavevmode
\epsffile{suscep.ps}
\end{center}
\caption{}
\end{figure}
\vfill



\end{document}



