

\documentstyle[12pt,epsf]{article}
\textheight 26cm
\textwidth 16cm
\hoffset -1cm

\begin{document}
\pagestyle{plain}



\centerline{\Large\bf Phase Transitions in QCD}
\vspace*{0.3cm}
\centerline{\large\bf November 4-7, 1998.}
\vspace*{0.8cm}
\centerline{\large\bf A RIKEN BNL Research Center Workshop}
\vspace*{0.8cm}
\centerline{\large\bf Brookhaven National Laboratory}
\vspace*{0.3cm}
\centerline{\large\bf Upton, NY, USA}
\vspace*{0.3cm}

\begin{figure}[h]
\begin{minipage}{70mm}
\epsfxsize=8cm
\epsffile{rhic.ps}
\end{minipage}
\hspace*{-0.5cm}
\begin{minipage}{45mm}
\epsfxsize=4.2cm
\epsffile{qcd.ps}
\end{minipage}
\begin{minipage}{45mm}
\epsfxsize=4cm
\epsffile{sqcd.ps}
\end{minipage}
\end{figure}



The Workshop `Phase Transitions in QCD' will be held at the RIKEN BNL
Research Center, located at Brookhaven National Lab, from November 4 to
7, 1998. 

The purpose of the workshop is to discuss the phase structure of QCD
and related theories. We will focus on the phase transitions that are 
expected to occur as a function of temperature, chemical potential, the 
number (and masses) of flavors, and other parameters. We would like to 
review what is known from general arguments, analytical approaches, and 
the lattice, and discuss possible avenues for future research. Recent 
advances in unravelling the phase structure of supersymmetric gauge 
theories and the possible impliactions of these results for QCD will 
also be discussed. Finally, we would like to discuss the possible 
dynamical mechanisms common to some (or all) of these transitions. 


The Workshop will consist of invited plenary talks, shorter talks by
participants, and discussion sessions on key topics. \\ \\ \\ 

Organizers: Thomas Schaefer (INT, Seattle) and Edward Shuryak (Stony
Brook) 

\newpage
\vspace*{-2cm}
\centerline{\large\bf Invited Speakers}
\vspace*{0.6cm}

\begin{minipage}{60mm}
\begin{itemize}

\item T. Appelquits, Yale 

\item N. Christ, Columbia$^*$ 

\item D. Diakonov, Nordita 

\item P. de Forcrand, Zurich$^*$ 

\item T. de Grand, Colorado$^*$ 

\item C. de Tar, Utah$^*$ 

\item F. Karsch, Bielefeld 

\end{itemize}
\end{minipage}
\begin{minipage}{60mm}
\begin{itemize}

\item J. Kogut, Illinois$^*$ 

\item M. Mattis, Los Alamos 

\item J. Negele, MIT$^*$ 

\item K. Rajagopal, MIT

\item M. Shifman, Minnesota 

\item M. Strassler, Princeton$^*$

\item F. Wilzcek, Princeton 
\end{itemize}
\end{minipage}
\\
$^*$ to be confirmed\\

\noindent
Proceedings: 
As for previous RIKEN/BNL workshops, the proceedings will be prepared 
quickly and consists of copies of selected transparencies. For this 
purpose, it is necessary for each speaker to supply a few (at most six) 
of your transparencies that outline the main points of the talk.\\

\noindent 
Support: We will provide housing and local expences for all participants
(apply soon!). Travel expenses can covered only in some cases (invited
speakers,..). \\

\noindent
To apply to the workshop, you can go to the online registration form
(as soon as it becomes available) or send the registration form to 
one of the organizers: \\ 

\begin{minipage}{60mm}
\noindent
Thomas Schaefer\\
Institute for Nuclear Theory\\
Box 351550\\
University of Washington\\
Seattle, WA 98195\\
(206) 685 3719 \\
schaefer@phys.washington.edu 
\end{minipage}
\begin{minipage}{60mm}
\noindent
Edward Shuryak\\
Department of Physics\\
SUNY Stony Brook\\
Stony Brook, NY 11794\\
(516) 632 8127 \\
shuryak@dau.physics.sunysb.edu\\
\end{minipage}
\\ \\ 

\noindent
Other information about travel, etc., is also available from the RIKEN
BNL Research Center home page 
{\tt http://penguin.phy.bnl.gov/www/riken.html/workshops.html}.


\end{document}
